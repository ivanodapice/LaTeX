\chapter{La dottoressa}

Né Anaïs né Dorian erano consapevoli della mia vera identità, o almeno
così pensavo, credendomi invece il figlio di un mercante di minore
importanza impegnato nei commerci con Jadd. Ben presto lo scoliasta che
faceva loro da tutore insistette perché parlassi con loro solo in
jaddiano a scopo di farli esercitare. Non ero veramente un loro amico o
l'amico di nessuno. Le mie cose erano state recuperate nei dormitori del
colosseo da alcune guardie del Casato e solo l'Impero sapeva cosa
avessero pensato di questo Switch e gli altri. Intanto io ero confinato
nel vasto palazzo in cima allo ziggurat di cemento e acciaio, mille
piedi al di sopra della città e del livello del mare. Dalla mia stanza
incastonata nel muro esterno potevo vedere tutta Borosevo srotolarsi
come un tappeto sporco, una chiazza sulle verdi acque del mondo.

Lasciato un impegno con i due giovani nobili, stavo attraversando il
pavimento a mosaico di un quadrangolo decorato da fontane dal suono
argentino che avevano al centro statue di rame verde quando incrociai un
paio di schiavi umandh che, con la pelle corallina a scaglie che si
crepava per l'esposizione all'aria, oscillavano sulle tre gambe nel
trasportare nei robusti tentacoli l'enorme statua di una sfinge Mataro.
Un pallet fluttuante avrebbe potuto rendere la cosa più facile, ma usare
gli schiavi xenobiti era una sorta di simbolo del proprio stato sociale.
Il Casato Mataro ne teneva al palazzo parecchie centinaia, e perlopiù
erano addetti a compiti estetici come fare aria con i ventagli a
personaggi importanti quando erano all'aperto o trasportare cose in giro
per il palazzo nel modo più visibile possibile. Emesh poteva non avere
molto sotto forma di ricchezze materiali o di importanza politica, ma
aveva gli xenobiti. Li guardai scomparire oltre un colonnato, con il
loro ronzio sommesso che svaniva con il loro allontanarsi.

«M Gibson! Hadrian!»

Mi girai, riconoscendo quella voce. «Lady Anaïs.» Mi inchinai quasi
prima di essermi voltato. «Chiedo scusa, ma per oggi non avevamo
finito?»

La figlia del conte era più alta di me di tutta la testa, ed era una
mescolanza perfetta dei suoi due padri. Mi sorrise, con le mani posate
sulla morbida curva dei fianchi. «Sì, abbiamo finito. Speravo solo di
riuscire a prenderti, tutto qui.»

«A prendermi?» mi allontanai dagli occhi una ciocca di capelli, già
umida di sudore per la dannata pressione dell'aria.

Lei sorrise come una fiamma scoperta e replicò: «Alla fine della
settimana ci sarà una gara di barche intorno al porto, lo sapevi?» Non
ne ero a conoscenza, e glielo dissi con cortesia mentre mi affiancavo a
lei, adeguandomi come meglio potevo al suo passo più lungo. «Ci saranno
tutti, in città. È l'evento della stagione, se non si considera
l'Efebeia di Dorian, naturalmente.» \emph{Quell'}annuncio era stato
fatto settimane prima, insieme alla rivelazione che un Cielcin sarebbe
stato sacrificato in un trionfo della Cappellania per commemorare
l'occasione. Forse era per questo che la gara di barche era sfuggita
alla mia attenzione. «I figli di lord Melluan sono arrivati da Binah»
continuò. Binah era la luna verde, un posto che si diceva essere coperto
di foreste fitte come quelle famose di Luin. «E l'arconte Veisi in
persona è arrivato da Fonteprofonda. Tutti quelli che contano nel
sistema.»

Annuii cortesemente. «Sembra essere uno spettacolo notevole.»

Lei infilò il braccio sotto il mio con una dolce risata. «Lo sarà, M
Gibson, davvero.» Discendemmo una curva della scalinata, passando
attraverso l'ombra di una torre quadrata che apparteneva alle più alte
sezioni interne del palazzo per dirigerci verso il più basso muro
esterno. In alto, la Torre della Lancia si levava come una colonna di
fumo nel firmamento, alta quanto lo ziggurat su cui posava. Appariva
sottile come una canna, quasi che il vento la potesse rovesciare. Mi
soffermai per un momento a guardare il giardino in ombra che si stendeva
sotto il muro di una terrazza più elevata. Tendoni di tela per vele
decorati con disegni di draghi e manticore si agitavano sopra la nostra
testa.

«Allora si naviga molto su Emesh?» Mi fermai per lasciar passare una
decuria di soldati del Casato, un gruppo di peltasti con le familiari
lance a energia. «Confesso di non avere molta familiarità con la cultura
locale perché a parte il tempo trascorso nel Colosso non ho avuto molte
possibilità di esplorare il vostro bel pianeta.» Non menzionai il
periodo in cui avevo vissuto in strada.

Anaïs mi strinse il braccio e si inclinò un poco contro di me. «Oh, devi
venire con me. Una qualsiasi delle navi sarebbe onorata di avermi a
bordo e se vuoi tu puoi farmi da scorta.»

«Mia signora, tu mi onori.» Inclinai la testa in un leggero inchino.

«Potrebbe essere divertente!» Rise, poi mi lasciò andare. Appena oltre
l'angolo il ronzio di altri schiavi umandh risuonava sommesso mentre
prendevano direttive da un paio di servi che impartivano istruzioni a
voce alta. Senza dubbio, uno di essi doveva avere una di quelle scatole
di comunicazione ronzanti che anni prima avevo visto insieme a Cat in
quel magazzino del pesce.

Poi una voce -- \emph{la} voce -- irruppe per la prima volta nel mio
universo. Ci sono momenti, istanti, che dividono. Intorno a essi il
tempo si frattura, per cui c'è un tempo che viene dopo... e tutto quello
che è stato prima diventa una sorta di sogno. «No, no, dannazione... lo
stai facendo nel modo sbagliato.» Allora non lo sapevo, ma la mia vita
si era appena divisa, si era spaccata in due nel momento in cui avevo
sentito quelle parole. Sbirciai al di là del pilastro di arenaria di un
colonnato e sulla balconata che si stendeva sotto un soffitto a volta
sovrastante il terreno di parata. Fra pochi mesi il trionfo per
l'Efebeia di Dorian Mataro sarebbe cominciato qui, snodandosi per le
strade e lungo i canali di Borosevo fino al colosseo. Dove infine il
Cielcin Makisomn sarebbe stato sacrificato dalla grande priora della
Cappellania, la madre di Gilliam.

Tre Umandh stavano cercando di sostituire le plafoniere del soffitto,
lottando con le grosse ciglia per manipolare attrezzi e strumenti
elettronici fatti per gli esseri con cinque dita. Mentre svoltavo
l'angolo uno di essi lasciò cadere una lunga asta fluorescente,
riducendone in polvere il bulbo. Uno dei douleter, un uomo grasso che
portava un'uniforme di un verde opaco, calò il bastone {elettrico} su
una delle gambe della creatura che crollò su due {ginocchia},
{abbassando} i tentacoli per puntellarsi mentre l'umano urlava: «Fottuta
feccia a forma di albero-calamaro!» Da un lato, il suo compagno stava
armeggiando con i comandi di una di quelle scatole ronzanti che usavano
per comunicare con gli Umandh.

L'uomo trasse indietro il bastone per vibrare un altro colpo e d'un
tratto lei fu lì, con una mano tatuata che gli si chiudeva intorno al
polso. «Avanti, provaci.» La sua voce risuonò limpida, alta e raffinata,
con uno strano accento. Sentirla mi ricordò il dottore tatuato
dell'\emph{Eurynasir}, in quel mio fatale viaggio per lasciare Delos.

Il grasso douleter si tese e sgranò gli occhi mentre quella donna snella
gli tratteneva il braccio, poi le forze lo abbandonarono e se la scrollò
di dosso, lanciandole da sopra la spalla un'occhiata feroce. Adesso mi
piace immaginare che le abbia indirizzato anche un segno protettivo, per
tenere superstiziosamente a bada i suoi demoni. Anaïs sopraggiunse
proprio dietro di me. «Oh, salve, dottoressa Onderra!» disse, sorpresa.
«Stai riparando le luci?»

La donna snella afferrò il terminale per comunicare con gli Umandh,
togliendolo di mano al secondo douleter che era troppo occupato a
inchinarsi ad Anaïs per protestare, poi prese ad armeggiare con
l'apparecchio per almeno cinque secondi prima di rispondere. Quando
finalmente parlò, lo fece in un tono piano, con voce ariosa, musicale e
divertita. «Buon pomeriggio, lady Mataro!» Non si inchinò, né fece una
riverenza o altri gesti di obbedienza o di deferenza. Si limitò a
sorridere, con le labbra piene che si schiudevano mentre intrecciava le
mani dietro la schiena. «Sì, c'è stato un altro sbalzo di corrente in
questo lato del castello ieri sul tardi e ho pensato di offrire la mia
assistenza con gli Umandh, dal momento che \emph{certe persone}...» Si
girò per rivolgere un'occhiataccia ai douleter. «...non capiscono un
accidente nel trattare con loro.»

«Uno sbalzo di corrente?» ripetei, guardando verso Anaïs per avere una
spiegazione.

Lei annuì, asciugandosi il sudore dall'attaccatura dei capelli. «I
generatori del castello sono un po' malridotti da quando è cominciata la
stagione delle tempeste.»

«Le cose si rompono» si affrettò a dire la donna straniera, concentrando
lo sguardo su di me. «Messer...» Anaïs mi strinse un braccio. «Lui è
Hadrian.»

«Hadrian... Gibson» riuscii a dire, porgendo la mano come avevo imparato
a fare nel colosseo.

Lei era chiara di pelle quasi quanto me; un pallore da marinaio, anche
se la sua pelle era tanto bianca e priva di difetti da indurmi a
decidere che doveva essere a prova di radiazioni come la mia. Con
indosso alti stivali e semplici calzoni aveva un'aria trasandata al
confronto di lady Anaïs Mataro con il suo ampio caffettano, ma li
portava con l'orgoglio di una regina. Aveva le braccia nude fino alla
spalla e il sinistro era tatuato con un fitto intaglio di sottili linee
nere, spirali e angoli che formavano una ragnatela dal deltoide fino
all'attaccatura di ciascun dito. Continuando a sorridere venne avanti
protendendo la mano destra... che mancava completamente di tatuaggi...
per stringere la mia. «Sono Valka Onderra Vhad Edda, xenologa.» Non so
come risposi, anche se suppongo di aver detto la cosa giusta, perché
Valka sorrise ancora, aggiungendo: «Piacere di conoscerti.»

Non mi considero un grande artista, anche se lei mi ha fatto desiderare
di esserlo, e in quel primo incontro non sapevo quante volte avrei
fallito nel tentativo di catturarla, nel disegno e nella vita. La
sfacciata dichiarazione di sé stessa, l'orgoglio in quel mento all'insù,
nel naso appuntito e nell'ordinata noncuranza che la poneva al di sopra
delle opinioni degli uomini inferiori... c'è poca traccia della sua
arguzia -- così vicina alla crudeltà -- in tutti i suoi ritratti che ho
fatto, e questa povera prosa non può contenere la sua bellezza di corpo
e di anima. Non ci riescono neppure gli ologrammi che ne sono solo degli
echi, come questo testo.

Qualsiasi esteta imperiale avrebbe detto che lei era troppo di troppe
cose, troppo severa, troppo seria. La sua pelle era troppo pallida, i
suoi occhi troppo grandi. Quegli occhi dorati. Non ne avevo visti di
uguali prima di allora e non ne ho visti in seguito. Conoscevano le
cose, e ridevano di quello che vedevano mentre lo facevano a pezzi. Non
esiste una parola per descrivere il colore dei suoi capelli, di un rosso
tanto intenso da sembrare nero tranne che sotto la luce più intensa. Li
portava corti, e quelli in eccesso erano arrotolati in uno chignon sopra
la testa, con alcune ciocche libere che le ricadevano sulla fronte e
intorno alle piccole orecchie. Aveva un sorriso tagliente generato da un
qualche scherzo che lei sola comprendeva e il portamento di un soldato a
riposo mentre aspettava pazientemente con quel terminale stretto fra le
mani, dietro la schiena.

Dopo quello che temetti essere un momento lungo in maniera imbarazzante
riuscii a chiedere: «Lavori con gli Umandh?» Avrei potuto fondermi e
sprofondare sotto il pavimento in quel preciso istante e luogo. Che
domanda incredibilmente insulsa! Se avessi saputo chi era -- cosa
sarebbe stata -- mi sarei strangolato per la vergogna.

La dottoressa lanciò un'occhiata da sopra la spalla, accigliandosi nel
guardare i tre Umandh, ora impegnati a raccogliere i resti della
lampadina rotta. «Solo incidentalmente. Il mio interesse primario sono
le rovine sul continente meridionale.»

«Non ero consapevole che Emesh \emph{avesse} un continente meridionale.»
\emph{Quali rovine?} Presi mentalmente nota di pilotare in seguito la
conversazione in quella direzione. Non avevo sentito dire niente sul
fatto che Emesh ospitasse strutture aliene, ma del resto non avevo
saputo niente neppure degli Umandh finché Cat non me ne aveva parlato.

«Anshar!» esclamò Anaïs. «Non è grande, ma è dove sorge Tolbaran, la
vecchia capitale che risale a prima che il mio bisnonno conquistasse il
pianeta e costruisse Borosevo.» In seguito appresi che quello era stato
un casino sanguinoso. Prima di diventare un palatinato imperiale, Emesh
era dominata da interessi extrasolari e da un gruppo delle Proprietà
Normanne. Quando era arrivato sul pianeta, oltre un millennio prima,
spalleggiato da tre Legioni imperiali, il Casato Mataro aveva costruito
la parte più antica di Borosevo su un atollo isolato, lasciando la
precedente capitale a marcire nelle mani dei suoi servitori.

La dottoressa Onderra sorrise di nuovo. «Il tuo compagno non è di qui,
vero?» Lo disse così, senza un onorifico, un `Vostra signoria', e si
limitò ad assicurarsi alla cintura il tablet per comunicare con gli
Umandh come un cantore avrebbe potuto fare con il suo libro di
preghiere, appoggiato al fianco. Non si era rivolta direttamente a me,
parlava ad Anaïs come se fosse stata il mio supervisore e io il suo
servitore.

Sua signoria tornò ad afferrarmi il braccio e mi trasse vicino
nonostante i miei tentativi di mantenere le distanze. «No, viene da
Teukros. Era un mirmidone, sai? Ha combattuto nel Colosso per un anno.»

L'accademica tavrosiana inarcò le sopracciglia, adottando quel genere di
espressione vuota e poco colpita di un adulto che abbia a che fare con
un bambino molto piccolo e molto irritante. «Ma davvero?» La sua
iniziale franchezza era scomparsa, il suo calore era stato spento da
quella nuova informazione. Ricordai tardivamente che i clan tavrosiani
non apprezzavano gli sport sanguinosi. Per quegli strani uomini che
vivevano ai confini della galassia la violenza era una cosa adatta ai
primitivi e alle macchine. Potei sentire il calare della temperatura
dell'atteggiamento della dottoressa nei miei confronti. Questo mi
infastidì, e poi fui infastidito dal fatto che la cosa mi seccasse.
Lanciai l'occhiata più tagliente possibile alla dama palatina appesa al
mio braccio, ma lei non la vide.

«Dimmi, M Gibson, ti piaceva uccidere schiavi per i tuoi padroni?»

Mi ci volle un momento per realizzare che la dottoressa stava parlando
con me perché avevo dimenticato il mio nuovo nome. Quando le sue parole
mi investirono lo fecero come un colpo al fegato, ma del resto lei era
una Tavrosiana, e in quella strana e lontana nazione non avevano il
Colosso e neppure schiavi. Avevano giochi simulati, lavori pubblici
obbligatori e una pace mantenuta con terapie e rieducazione, proteste
dove noi avevamo l'ordine, caos dove noi avevamo la pace. Scoraggiavano
la formazione di famiglie, al punto che le lunghe relazioni di coppia
venivano disintegrate dal loro stato bastardo e commettevano il più
grande degli abomini, quello di mescolare la loro carte con quella delle
macchine. Decidendo che lei non aveva capito la distinzione fra
mirmidoni e gladiatori replicai: «Io combattevo con gli schiavi, mia
signora. Contro i gladiatori.»

«Contro di loro?» chiese Valka Onderra con un sogghigno. «Allora
suppongo che sia tutto a posto.» Si passò una ciocca di capelli fra il
rosso e il nero dietro un orecchio. «E non sono una signora, sono una
dottoressa.»

Allora era una studiosa, come me, come aveva già ammesso quando aveva
affermato di essere una xenologa. La cosa però mi era sfuggita per una
frazione di secondo, e questo mi lasciò il cervello confuso e fece sì
che mi sentissi uno stupido. Lei distolse l'attenzione per un momento,
gridando istruzioni ai due douleter e gesticolando con quella mano dagli
intricati tatuaggi. I due uomini risposero girando intorno agli Umandh
con i loro bastoni, e Valka gettò loro il tablet di controllo che aveva
sganciato dalla cintura. «Dilettanti.» Pronunciò quella parola come se
fosse stata un'imprecazione, un effetto accentuato dal secco accento
tavrosiano, mentre si batteva un dito su una tempia e socchiudeva quegli
occhi dorati in un'analisi piena di frustrazione. Poi borbottò un flusso
di imprecazioni in un dialetto tavrosiano. Colsi la parola
\emph{okthireakh} -- `imperiali' -- e un'altra che suonava terribilmente
come `barbari'.

Barbari? Rispolverai le lingue tavrosiane che conoscevo, facendo
scrocchiare le mie dita mentali, poi passai al nordei, che era la lingua
più comune della Demarchia. Ne avevo solo una comprensione minima, ma ci
provai comunque e chiesi: «Come funziona il congegno che usi per
comunicare con loro?»

Le sottili sopracciglia di Valka Onderra si sollevarono per la sorpresa,
ma lei non mi rispose in nordei, bensì in travatsk, un'altra lingua dei
Tavrosiani. Astuta. Non ne conoscevo una sola parola, quindi passai
invece al panthai, l'unica altra lingua di clan che mi permettesse di
mettere insieme una frase, anche se rischiavo di suonare come un bambino
ottuso. «Non ho capito una parola di quello che hai detto.»

Incredibilmente, sul suo volto apparve un sorriso. Intanto Anaïs
spostava lo sguardo da lei a me, con lo sconcerto dipinto sul volto
geneticamente scolpito alla perfezione. «Quello era tavrosiano? Voi due
vi conoscevate già?» Appariva imbronciata sotto la liscia cortina di
capelli fra il blu e il nero. Resistetti all'impulso di dirle che il
tavrosiano di per sé non esisteva.

La straniera mi guardò di nuovo come se mi vedesse per la prima volta,
accarezzandosi il mento appuntito. «No, no, Vostra signoria. Non avevo
mai visto prima quest'uomo in vita mia.» La sua attenzione si spostò
brevemente dalla mia faccia a quella di Anaïs per poi tornare su di me.
«Non molti imperiali imparano una lingua tavrosiana, e tantomeno due.»

«Io non sono `molti imperiali'» ribattei, ergendomi un po' di più sulla
persona e -- speravo -- apparendo più alto ai suoi occhi.

«È evidente.» Inarcò un sopracciglio, poi cambiò approccio. «Gli Umandh
non sono come noi. Loro non pensano.»

«Prego?» dissi, sussultando per quell'improvviso cambio d'argomento.

«Cosa significa?» domandò Anaïs nello stesso momento.

Valka accennò con la testa alle sue spalle, dove gli xenobiti erano di
nuovo impegnati a sostituire le luci sotto l'attento controllo dei loro
supervisori. «Non sono individui, in realtà. Sono più simili a... ecco,
a un merletto neurale.» Non avevo idea di cosa fosse ma rimasi
impassibile. Avevo riscontrato spesso che di fronte all'ignoranza il
silenzio è l'insegnante migliore.

Avrei solo voluto che qualcuno avesse impartito ad Anaïs Mataro quella
stessa lezione. «Che cos'è?»

La tavrosiana inarcò un sopracciglio. «Ogni Umandh è come una cellula e
loro... quel ronzio non è tanto un modo di comunicare, non è un
linguaggio. Loro... sono collegati in rete.»

`Collegati in rete?' Era un termine per sfera-dati, lo sapevo, ma avevo
poca comprensione del funzionamento arcano di una sfera-dati planetaria,
tanta quanta un cane ne ha dei riti di accoppiamento umani. Questa volta
la mia curiosità riguardo a quel campo di studio quasi proibito ebbe la
meglio sulla mia trepidazione. «Vuoi dire che sono un organismo unito?»

La dottoressa si illuminò e fece scorrere lo sguardo da Anaïs a me. Le
sue sopracciglia si aggrottarono e lei inclinò la testa in un lieve
gesto affermativo. «Non del tutto. Sono distinti -- non condividono
tessuti -- ma il ronzio permette loro di armonizzare.»

«Armonizzano alla lettera» commentai, con un sottile sorriso in tralice
che Valka ricambiò. Da qualche parte nel mio petto un'ombra dell'antico
Hadrian -- l'allievo dello scoliasta -- si riscosse come da un sonno
criogenico. Questo era ciò per cui ero vissuto da ragazzo. Erano passati
davvero appena quattro anni? Quattro anni per me, mi corressi, ma in
realtà sono stati quasi quaranta.

Chiaramente contrariata di essere stata esclusa dalla conversazione,
Anaïs si protese in avanti. «Rimarrai a lungo nella capitale,
dottoressa?»

«Solo finché non saranno passate le tempeste. Calagah diventa un po'...
ecco, in questo periodo dell'anno è sott'acqua.» Gli Umandh stavano
finendo il loro lavoro e il tono del loro canto andava salendo, un ritmo
costante che si univa al suono improvvisamente sconnesso del loro coro.
Non riesco tuttora a capire come producessero quel suono, anche se
immaginai che dovessero avere una bocca da qualche parte nella sommità
del corpo, in mezzo ai tentacoli. Mentre una di quelle creature a tre
piedi mi passava davanti notai la sostanza simile a glassa che si era
spalmata addosso in vortici lungo tutto lo stretto tronco e sulle cosce
radiali. Erano una sorta di segni tribali? Avrei voluto chiederlo ma
archiviai la cosa per un altro momento.

Invece seguii il flusso naturale della conversazione, resistendo alla
tentazione di seguire numerose tangenti fino ai loro vicoli ciechi.
«Calagah? Si tratta delle rovine che hai menzionato?»

Rimasi un po' stupito quando a rispondere fu Anaïs e non la dottoressa.
«Davvero non lo sai?»

Lo ignoravo davvero e cominciavo a essere un po' stanco dello schema di
non risposte che stava modellando quella conversazione. Badai comunque a
mantenere la lingua sotto controllo, consapevole che una delle due donne
era una palatina e che io -- nella mia attuale veste di M Hadrian Gibson
-- non lo ero. «No, mia signora, temo di no.»

Valka si passò pensosamente una mano su e giù lungo le linee intricate
tatuate sul braccio prima di appuntare su di me quegli incredibili occhi
dorati. «Sì, Calagah è dove sono le rovine. Mmh...» Adocchiò lady Anaïs,
mordicchiandosi il labbro inferiore. Quando poi Anaïs non si lamentò o
accennò a fermarla, aggiunse: «Sono molto antiche e non sono umane. Quel
sito è antecedente di migliaia di anni all'insediamento qui dei
Normanni.»

Questo mi lasciò sconcertato. «Sono degli Umandh, allora?» Non avevo mai
sentito dire che gli Umandh costruissero niente di tanto permanente. Le
loro case, erette lungo la costa in una zona recintata della città e
nella loro riserva per alieni insulare, erano strutture a crescita
graduale ricavate da detriti. Avevano preso i rottami di barche e
astronavi, frammenti di edifici sgretolati e qualsiasi altra cosa
fossero riusciti a trovarle, legandoli insieme per formare baracche con
il tetto inclinato. Il loro villaggio sull'acqua -- nell'acqua --
sembrava più che altro una sorta di vortice spinto a riva dalle onde.
Senza cure, non sarebbe durato un decennio, tantomeno secoli.

La dottoressa arricciò il naso. «Questa è l'ovvia conclusione.» Adocchiò
la dama palatina che mi gravava sul braccio e parlò in modo più chiaro.
«Non siamo la razza più antica del cosmo, anche se le altre non sembrano
aver mai lasciato il mondo natale. L'estinzione le ha raggiunte prima
che lo facessero.»

«Come gli arcicostruttori di Ozymandias?» domandai, citando la prima
civiltà estinta che mi venne in mente.

Valka sbatté le palpebre. «Proprio così, ma le rovine di Calagah sono
molto più antiche. Gli arcicostruttori sono estinti solo da...»

«Quattromilatrecento anni, sì» finii per lei, ansioso di dimostrare le
mie conoscenze. Valka parve sorpresa, quindi aggiunsi: «Sono una sorta
di appassionato dilettante.»

Lei incrociò le braccia. «Proprio. Ecco, alcuni di noi si guadagnano di
che vivere con questo, M Gibson. Comunque, ho parecchi ologrammi del
sito di Calagah, e se sei davvero interessato potresti passare nelle mie
camere.» Sorrisi, convinto che cominciasse a farsi cordiale verso di me,
o lo fui finché non arrivò la frecciata: «Naturalmente, se non sei
troppo impegnato a uccidere gente.» Poi mi oltrepassò, lasciandomi
piccolo e sudato nella sua scia.

Era già nell'ombra dell'arcata che portava fuori dal colonnato, al
seguito degli Umandh e dei due douleter, prima che trovassi una
risposta. «Conoscerti è stato un vero piacere, dottoressa.»

Non si guardò indietro ma agitò una mano. «Suppongo di sì.»

Non trovai una replica per questo e rimasi fermo là, con Anaïs
dimenticata al mio braccio. Non riuscivo a trovare parole, e per qualche
tempo ci fu solo il rumore distante di Borosevo che arrivava attraverso
la balconata. Le luci installate dagli Umandh si accesero, e la parola
che infine mi apparve nella mente fu un termine dell'inglese classico,
\emph{dumbstruck}: `ammutolito'. Ero rimasto letteralmente senza parole,
come se avessi incassato un colpo. Nel galstani non abbiamo un termine
per esprimere questo concetto, e nessun'altra parola poteva andare bene.


