\chapter{Calagah}

A Borosevo avevo dimenticato come fosse provare qualcosa che non fosse
il caldo, perché la vasta capitale si trovava ad appena dieci gradi a
nord dell'equatore di Emesh, accoccolata sulle isole-scogliere che
dominavano quel vasto mare poco profondo, lontano dal remoto e solitario
continente meridionale del pianeta. Là i resti del vulcanismo da tempo
inattivo apparivano nelle sporgenze di roccia ignea consumata dal tempo,
nera e dura. Era da lì che i colonizzatori normanni avevano mantenuto la
presa sul pianeta, con il loro centro nella città di Tolbaran che era
ora la sede di lord Perun Veisi.

Durante il breve soggiorno nel suo castello di Fonteprofonda appresi una
notevole quantità della storia preimperiale di Emesh, riguardo alla sua
cultura legata alla pesca e alla rete di città insulari che dominava il
pianeta da una rimpicciolita calotta polare all'altra, alla
colonizzazione e ai primi contatti con gli Umandh, alla guerriglia e
alle giunche da pesca dei popolani affondate in mare dai coloni nativi
che le trascinavano sott'acqua. Appresi dell'annessione e del decennio
di lotte fra i liberi proprietari e tre Legioni imperiali sotto il
comando del primo conte Mataro, lord Armand, da cui prendeva nome la
luna più piccola.

E appresi di Calagah.

Valka era una buona insegnante, e dove mancava di informazioni -- il che
era una cosa rara -- sir Elomas Redgrave era pronto a riempire i vuoti
con precisa concisione. Il fatto che avrei sposato Anaïs Mataro quando
fosse diventata maggiorenne era ormai una sorta di segreto conosciuto e
addirittura una sorta di battuta per i miei due compagni, che mi
chiamavano `mio signore' in un modo che riecheggiava la tendenza di Ghen
di riferirsi a me nello stile imperiale.

Quattro mesi locali dopo il giorno del mio duello con Gilliam seguii
Valka e i miei nuovi amici della squadra di ricerca di Fonteprofonda
fuori da sotto l'ala della nostra navetta e sulle pietre coperte di
muschio dell'altura sovrastante la fenditura in cui sorgeva Calagah. Le
alghe pendevano dalle strutture bianche in plastica e acrilico che
sporgevano come ostriche dalla pietra nera, dando l'impressione che
stessimo camminando sul fondo del mare e che le nuvole fra il grigio e
il dorato sopra la nostra testa ne fossero la superficie chiazzata di
luce del sole. A quella latitudine il vento era teso e sferzava l'acqua,
sollevando fini spruzzi che velavano il mondo.

Se non altro, avevo di nuovo freddo nel trovarmi sotto il cielo aperto.
Qui Emesh somigliava un po' di più a Delos, alla mia casa. Dopo così
tanti anni su questo mio nuovo mondo coraggioso avevo cessato di notarne
la maggiore gravità e l'aria pesante come gelatina nei polmoni.

«Bel, Maros! Siete incaricati di occuparvi delle alghe!» gridò sir
Elomas, accartocciando la stagnola della barretta di cioccolato che
aveva in mano mentre agitava un braccio in direzione del sito. «Voglio
vederle bruciare sulla spiaggia entro un'ora.» I due tecnici si misero
in fretta al lavoro senza proteste mentre gli altri procedevano a
scaricare la navetta. Io mi unii a Valka e alla calva scoliasta, tor Ada
-- una dei dipendenti dell'arconte Veisi -- alla base di una colonna
appuntita di basalto, una delle colonne esagonali distorte dal tempo in
modo tale che proiettavano ombre innaturali sulla roccia coperta di
muschio.

Il vento si stava intensificando, e qui sul continente di Anshar era
tagliente, intriso del gelo dell'inverno imminente. Tirai su il colletto
della mia corta giacca e mi inclinai per opporre resistenza alla brezza
marina nell'avvicinarmi al pilastro. L'oceano, che sull'equatore era di
un verde tanto intenso da sembrare quasi putrescente, qui scintillava di
una pallida tonalità acquamarina sotto le nubi grigie che risplendevano
dorate nella luce rifratta del sole. Le due donne e la colonna di cui si
trovavano al riparo dominavano la {fenditura} stessa, un taglio profondo
e quasi diritto nella superficie del mondo. In base agli ologrammi di
Valka, dedussi che ci trovavamo \emph{al di sopra} del sito di Calagah,
e che la sua facciata nera e lucida che avevo visto in quelle immagini,
nella sua camera, mesi e mesi prima, era incastonata nella parete di
pietra dal cui orlo stavamo sbirciando. Quella realizzazione portò con
sé un po' di delusione, perché avevo desiderato quel momento di suprema
rivelazione, il risalire la cresta per vedere i resti di un'altra epoca
che si allargavano ai miei piedi come una mappa.

«Ci vorrà un'altra ora prima di riportare in funzione il campo» affermò
tor Ada, incurvando le spalle e infilando le braccia nelle maniche
voluminose nello sforzo di offrire un bersaglio più piccolo al vento.
«Sarei dovuta rimanere nella navetta. Questo vento è spaventoso.»

Valka sorrise. «A me piace! Mi ricorda casa!» Durante il nostro tempo
insieme avevo appresso che `casa' erano le gole di Edda, sferzate da
venti ancora più violenti di questi che spingevano gli abitanti a vivere
in città erette lungo le pareti di dirupi molto simili a questo.

«Almeno è meglio del dannato caldo della capitale» interloquii,
consapevole dei capelli che mi si agitavano intorno alla faccia. «C'è il
tempo di scendere a vedere il sito, mentre aspettiamo?»

Ada strisciò un po' i piedi. «Credo di sì.»

Scendemmo mediante una sferragliante scala di metallo fissata alla
parete della fenditura. Gli stivali di Valka e i miei risuonavano
stentorei sulla rete di metallo della struttura, mentre le babbucce di
Ada erano quasi silenziose perché il loro suono era soverchiato dal
soffio del vento. Mentre scendevamo le nuvole si aprirono e la luce
solare color rubino cadde in linee gocciolanti sul basalto grigio.
Avvertii un assurdo desiderio della giacca lunga che ero solito
indossare a casa, della sensazione delle sue code che mi sbattevano
intorno alle gambe e dell'ombra che avrebbe proiettato contro la parete
alla nostra sinistra.

Poi arrivammo al sito, più nero delle pietre fra cui era stato
costruito, un tutt'uno con il paesaggio come se fosse stato fuso sul
posto. Ho cercato un migliaio di volte di disegnare quella facciata, di
raffigurare le colonne, gli archi e i contrafforti angolati, e ho
fallito migliaia di volte. Il mio vecchio diario mostra le cicatrici di
quelle pagine strappate, muta testimonianza dell'inadeguatezza delle mie
capacità artistiche.

Il respiro mi abbandonò, prosciugato come la luce solare stessa, perché
le pietre di Calagah erano più nere di qualsiasi cosa avessi mai visto
-- più dei miei capelli, più delle uniformi più nere del nero indossate
dalla Legione. L'ingresso misurava un'ampiezza di circa trecento piedi e
dominava la parete occidentale della fenditura nel suo punto più largo.
La fenditura stessa proseguiva verso il basso fin oltre il livello del
mare, protetta da una sporgenza di pietra che teneva indietro le sue
acque per la maggior parte dell'anno.

Quali strani poteri o persone avevano trovato il modo di erigere una
struttura del genere in un posto tanto isolato? Quella vasta apertura
cominciava come un groviglio di colonne angolose, di supporti e di
contrafforti ad arco -- alcuni larghi quanto un uomo, altri sottili come
un sussurro -- che procedevano verso l'interno per gradi, strato su
strato, assottigliandosi di continuo dai trecento piedi iniziali
all'ampiezza di una porta. Lo sguardo si ribellava di fronte a quella
complessità e vagava fino a perdersi, come in un fitto bosco.

«È...» Non c'era niente da dire. Perché ci stavo anche solo provando?
«È...»

Valka mi toccò un braccio. «Lo so» disse.

`Sbagliata\emph{.'} Quella parola mi affiorò dentro come proveniente
dall'esterno e spinta nella mia testa da uno spazio più elevato. «È
sbagliata.» affermai.

Tor Ada attraversò in fretta la base sabbiosa della fenditura, con i
piedi che sciacquettavano intorno alle pozzanghere che non si erano
ancora asciugate. Il mondo puzzava di pesci morti e in effetti qui e là
si vedevano le loro carcasse spolpate dagli uccelli o lasciate a
marcire. Mi premetti una mano contro la faccia, ricordando Gilliam e il
suo dannato fazzoletto, poi mi affrettai a seguire lo scoliasta,
afferrato dall'innegabile impulso di vedere quelle strane e favolose
rovine.

«Sbagliata?» ripeté Valka, confusa, nel seguirmi. Ci fermammo ai piedi
della scala, mentre tor Ada svaniva all'interno. Dalla base dell'altura
non potevo più sentire i rumori prodotti dalla squadra che lavorava per
preparare il campo in cima all'altura. In realtà pareva che il mondo
intero fosse scomparso, che tutta la galassia e l'universo fossero stati
cancellati tranne che per il contenuto di quella gola.

Negli ologrammi che Valka mi aveva mostrato ogni linea e angolo erano
parsi perfetti, rettilinei, squadrati, ma non era più così. Ogni arco,
ogni curva, ogni colonna di quel posto di pietra grigia erano inclinati,
piegati come la colonna di basalto sull'altura, vicino al nostro campo.
«La geometria è... non c'è niente di parallelo.»

«Lo hai notato?» Valka sgranò gli occhi e si assestò la giacca di pelle
rossa.

Indicai con due dita, scegliendo i dettagli che mi avevano fatto notare
la cosa. «Quel supporto è più corto dei due alla base e sono inclinati
verso l'alto. Vedi? Dovrebbero essere paralleli ma si allargano a
ventaglio verso l'alto. È una cosa sottile, ma divergono...»

«...di 0,374 gradi.»

Fu il mio turno di restare interdetto. «Come hai fatto a memorizzare...»

Si batté un colpetto sulla testa, poi fece una pausa mentre una folata
di vento spingeva la sabbia umida su per la gola. «Volevi essere uno
scoliasta e hai bisogno di chiedermelo? Sono abbastanza vecchia da poter
essere tua madre, ricordi? Ho avuto il tempo di esercitarmi.»

«Mia madre ha tre volte la tua età» scattai. \emph{Una studiosa e una
	scoliasta...} Non che avessero scoliasti nella Demarchia. Quella terra
lontana non aveva mai sofferto sotto i Mericanii, non aveva mai imparato
a temere le loro macchine demoniache, quindi non avevano bisogno di
scoliasti. «Le rovine sono tutte così? Tutte...» Mimai due cose che non
erano parallele.

«Qualche volta mi dovrai spiegare come hai fatto a notarlo.»

«Nessun trucco, l'ho semplicemente visto.» Scrollai le spalle, ma ancora
non salii il primo dei gradini simili a vetro che portavano dentro le
rovine, a causa di un gelo più intenso che mi si diffondeva nel sangue.
Quel posto sembrava come uscito da un sogno, perché di certo solo una
mente in stato di incoscienza poteva dare vita a una pietra tanto nera.
«Di cosa è fatta?» Tutti quei mesi, e non avevo mai pensato di
chiederlo. Ecco, non c'era niente che potesse sostituire l'esperienza
sul campo.

Avvertii una luce che si accendeva negli occhi ardenti di Valka, che mi
si accoccolò accanto sulle scale, accarezzandone la pietra lucida. Il
vento intenso le spingeva i capelli sulla faccia, nascondendo in parte
il suo sorriso, che non era quello crudele e tagliente che sfoggiava di
frequente e neppure quello caloroso che avevo visto per caso la sera del
mio duello, ma uno di gioia sincera, un abbandono infantile che
risplendeva su quel volto duro e austero. «Non ne ho idea.»

Mi accoccolai accanto a lei sul suolo sabbioso della gola. «Dici sul
serio?» domandai.

«Non reagisce ai sensori.» Tracciò una linea sulla superficie, grattando
via la sabbia impastata sul gradino in modo da lasciare una linea di un
nero profondo. «Non riusciamo a staccare un campione da testare e le
unità da campo... niente, nessun risultato. È soltanto nero.»

«Cosa intendi con `è soltanto nero'?» Lei sbatté le palpebre e il suo
sorriso gioioso si spense in un sottile aggrottarsi della fronte.
Rendendomi conto del mio errore, feci un altro tentativo. «È solo nero?»

Una breve luce divertita le passò sul volto. «Non riusciamo a
identificare la sua struttura molecolare. Non sembra averne una.» Passai
una mano sulla superficie di uno dei gradini che portavano nell'oscurità
sottostante quelle colonne inclinate. Anche se la pietra era esposta
alla luce diretta del sole la sua superficie era fredda e liscia come
vino, ma non appariva diversa da pietra normale. Nella sua consistenza
vetrosa non c'era niente che indicasse un qualche profondo mistero.
Appiattii il palmo contro la superficie, come se la mia mano fosse stata
un astuto strumento tramite il quale avrei potuto divinare la verità di
quel posto. La pietra appariva più fredda in alcuni punti e mi accigliai
nello spostare la mano sulla superficie del gradino, come ipnotizzato.
D'un tratto sussultai e lanciai uno strillo come se qualcuno mi avesse
inchiodato la mano alla scala con un punteruolo di ghiaccio. Freddo,
quel genere di freddo che striscia, mi saettò su per il braccio e fino
al cuore, circolando in ogni parte e poro del mio essere. Gridando, mi
serrai la mano al petto. Allarmata, Valka mi afferrò il polso. «Cosa
succede?»

Una folata di vento ritardò la mia risposta mentre serravo la bocca per
proteggerla dalla sabbia e abbassavo la testa. Valka mi stringeva ancora
il polso, e un freddo terribile, tagliente, mi faceva dolere le dita.
Posai l'altra mano su quella di Valka e la rimossi con gentilezza dal
mio braccio. «Non è niente» spiegai, quando il vento si abbatté un poco.
«Era più fredda di quanto mi aspettassi e... mi ha colto di sorpresa.»

Una riga si formò fra le sue sopracciglia e premette una mano sulla
pietra nera mentre il suo cipiglio si accentuava. «A me sembra tiepida»
affermò, nel ritrarre la mano. Mi issò in piedi e si avviò su per le
scale con i fianchi che oscillavano a ogni gradino. Confesso di averla
osservata per cinque secondi abbondanti, distogliendo lo sguardo quando
si girò verso di me. «Vieni?»

Confuso da lei e dalla stranezza di quel posto alieno mi guardai di
nuovo la mano. Doleva per il ricordo del freddo. Non lo avevo
immaginato. Toccai di nuovo il gradino ma sentii solo il tepore
dell'alto sole lontano che irradiava dalla pietra nera. Sollevai lo
sguardo su Valka, che era ferma circa dieci piedi più in alto rispetto a
me sulla scala nera, e nel guardare in alto nell'apertura di quelle
rovine, con le pareti nere che scintillavano tutt'intorno avvertii un
paralizzante senso di vertigini, come se la porta fosse stata la bocca
di un pozzo di incalcolabile profondità e io fossi stato in equilibrio
sul suo orlo. Vidi una figura in verde in piedi sotto l'arcata e per un
fugace momento credetti che si trattasse di Gibson, ma era soltanto Ada.
«Perché ci state mettendo così tanto?»

«Il nostro ragazzo qui presente stava ammirando l'architettura» rispose
Valka, coprendo la mia avanzata su per la scala.

«Può farlo all'interno!» Ada fece un secco gesto di sollecitazione con
il braccio destro, con gli occhi chiari che scintillavano.

Seguii Valka sotto l'ombra delle colonne, oltre file di mucchi di pietre
e supporti rinforzati... era come entrare in una ragnatela. Le pareti
intorno a noi si fecero più vicine mentre salivamo la corta rampa di
gradini e la porta in cima alla scala risultò leggermente più stretta
alla base che non alla sommità, una cosa tanto infinitesimale che solo
il mio occhio di artista poteva notarla ed esserne irritato. «Ci sono
altre camere lungo tutta la fenditura,» disse Valka «perlopiù
abbondantemente al di sopra del livello del terreno. Sono tutte vuote,
ma questo è il complesso principale. Ci sono altri ingressi, più
addentro nelle terre alte. Condotti di ventilazione e cose del genere.»

«Ma questa è l'unica porta?»

«L'unica.»

Ormai eravamo dentro, persi nell'oscurità salmastra. Meno di un mese
prima quelle gallerie buie si erano trovate sott'acqua e il marchio del
mare perdurava ancora. L'aria era fredda anche se non quanto lo era
stata la pietra. Mi ero aspettato una vasta sala come quelle che avremmo
potuto costruire noi, con un alto soffitto e dotata di colonne come
l'esterno, ma lo spazio interno era basso e fioco, illuminato solo da
linee di nastro fosforescente attaccato alle pareti su entrambi i lati.
Mentre camminavamo, con la testa che quasi sfiorava il soffitto, Valka
tirò fuori dalla giacca una sfera luminosa tascabile, la scosse e ne
attivò il circuito di alimentazione prima di liberarla nell'aria.
Quell'arnese doveva essere di fabbricazione tavrosiana e dotato di una
qualche loro tecnologia eretica, perché assunse una posizione vicino
alla spalla di Valka e la seguì mentre camminava, proiettando una luce
fra il rosso e l'oro sulle pareti nere.

La pietra nera -- era pietra? -- restituiva quella luce in scintillii
arcobaleno, e i punti luce che si increspavano su quella superficie
scura facevano apparire il corridoio più luminoso di quanto non fosse.
«Sono tutte gallerie?»

Tor Ada si girò con un fruscio della veste verde. «Ci sono camere e
diramazioni, ma va avanti così per un bel pezzo.»

«Non è... strano?» chiesi, accigliandomi. «A cosa serviva? Di certo non
era una città.» Avevamo percorso circa un migliaio di piedi, e il
corridoio non mostrava segno di diventare più largo.

«Pensiamo che fosse forse una sorta di tempio» rispose la scoliasta. «Un
luogo sacro.»

«Gli archeologi non suppongono sempre che qualcosa abbia un significato
religioso quando non hanno idea di che cosa sia?»

Valka rise. «Proprio così.» La scoliasta non apprezzò la mia battuta
perché continuò a camminare in silenzio. «Comunque la supposizione di
tor Ada è valida quanto qualunque altra. Lo vedrai.»

Arrivammo infine in una camera rotonda simile al mozzo di una ruota, da
cui parecchi passaggi si addentravano sempre di più in Calagah. Là ci
fermammo, decisi a non spingerci oltre per via del poco tempo che
avevamo. Con un gesto Valka ordinò alla sfera luminosa di prendere
posizione vicino al soffitto a volta, dove illuminò una costellazione di
glifi rotondi, che apparivano identici agli schemi di nodi che gli
Umandh usavano per le loro opere d'arte e che coprivano il soffitto,
intrecciandosi e sovrapponendosi uno all'altro finché lo sguardo non si
perdeva in essi, come nella confusione di colonne all'esterno.

Guardai in alto a bocca aperta, comprendendo per la prima volta perché
nessuno avesse ancora decifrato i simboli alieni: la Quiete aveva
parlato, ma le parole erano perse nel tempo. In quei geroglifici -- se
erano geroglifici -- non c'era niente di lineare o di logico. «Non
somiglia a niente che abbia mai visto.» Delle menti li avevano
concepiti, intelligenze strane e incomprensibili. «E tutto questo è qui
da quasi un milione di anni?» Scossi il capo, ricacciando indietro la
meraviglia. «Sembra nuovo.»

Come i grandi e alti templi della Cappellania invitavano chi li guardava
a elevarsi verso il cielo, il peso di quella cupola tanto vicina mi
schiacciava verso il basso. Tor Ada stava recitando un catalogo di
dettagli relativi al sito. «I livelli inferiori sono ancora inondati e
nel corso della prossima settimana dovremo attivare di nuovo le
pompe...» La sentii a stento. In quell'istante da esitante scettico
venni rimodellato in un vero credente.

«È impossibile che gli Umandh abbiano costruito tutto questo» sussurrai.

Quell'affermazione interruppe di netto il monologo di tor Ada. «Già»
disse Valka.

«È incredibile.» Scossi il capo, cercando invano di trovare qualcosa da
fare con le mani che, a corto di opzioni, si agitavano lungo i fianchi,
contagiate dalle implicazioni. «Non mi meraviglia che teniate un profilo
tanto basso riguardo a tutto questo.» Avrebbero dovuto esserci cordoni e
misure di sicurezza intorno al sito, un contingente di sorveglianza
della Cappellania. Qualcosa, qualsiasi cosa. Forse a Emesh era stata
attribuita troppa poca importanza, dato che la circolazione a piedi in
questo remoto angolo delle terre rocciose dell'Anshar era una cosa rara.
Forse la Cappellania sottovalutava l'effetto che quelle rovine avrebbero
potuto avere sull'anima umana, o forse non lo \emph{sapeva}. Cercai di
immaginare la grande priora Vas in piedi fra i pesci morti alla base
della fenditura e avvertii un'eco della sensazione che avevo avuto
seduto nell'appartamento di Valka a Borosevo. Umani, Cielcin... la
Quiete. Non eravamo mai stati la sola forza in ascesa nella galassia.
Per quanto piccolo e strano, il tunnel mi umiliava, mi faceva ricordare
che {nonostante} tutta la mia educazione e tutta la storia della mia
famiglia ero soltanto un singolo uomo. Un uomo solo contro un cosmo
strano, grande e terribile.

«Sì?» Valka stava parlando al terminale da polso, con la testa abbassata
e dandoci le spalle. Mi girai a guardarla con le sopracciglia sollevate
e la scoliasta fece altrettanto. Valka si premette un dito appena sotto
l'orecchio. Ascoltando attraverso la linguetta a conduzione ossea
applicata in quel punto. «Arriviamo subito, signore.» Chiusa la chiamata
si girò verso di noi. «Era Elomas. Il campo è pronto.»


