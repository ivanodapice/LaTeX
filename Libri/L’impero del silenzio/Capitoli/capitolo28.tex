\chapter{Sbagliato}

Già intorpidito per un colpo di storditore che mi aveva sfiorato e
stringendo il braccio sinistro inutilizzabile, aggirai un angolo di
strada e mi arrampicai su una serie di casse, usandole come trampolino
per oltrepassare la tavola a levitazione parcheggiata che bloccava la
strada. Il cemento grigio riluceva per il calore ma la pianta sporca e
sfregiata dei miei piedi non lo percepiva. Non so dire che giorno fosse
o quanti mesi o anni della mia vita si fossero persi come altrettanti
rifiuti nelle acque di quei canali.

Sentii le grida dietro di me e infilai le cinghie della borsa che avevo
rubato nella piega del braccio che ancora funzionava. Ansimando, saettai
su per una rampa di scale e in un vicolo tanto stretto che le mie spalle
larghe per poco non strisciarono contro le pareti. Sopra la mia testa
gli edifici erano molto vicini, con le tubature esposte e fissate con
delle staffe alle pareti di metallo. La camicia mi si impigliò in un
bullone, strappandosi, e senza dubbio mi procurai anche una lacerazione
nel braccio insensibile. Il crepitare elettrico degli storditori e le
grida -- «Fermo! Al ladro!» -- mi seguirono come i segugi in una delle
cacce alla volpe del mio defunto zio. Mi girai e oltrepassai con una
spallata un paio di donne che trasportavano sulla testa il cesto della
spesa, per poi sbucare su una strada che correva parallela alla curva
salmastra di un canale.

Anche se l'istinto mi diceva di correre, e anni di lividi e di ossa
rotte malamente risanate da praticanti clandestini mi urlavano di
continuare a fuggire, sapevo che questo mi avrebbe solo fatto dare
nell'occhio. Molte volte ho visto una folla rivoltarsi contro un ladro e
quella calca di gente vestita a colori vivaci trasformarsi in un solido
muro di carne mentre i prefetti lo abbattevano. Invece, aggirai una
ringhiera di ferro e sedetti al tavolo di vetro vuoto al limitare
dell'area per i clienti di un caffè. Sulla superficie tremolavano gli
ologrammi per le ordinazioni e io finsi di frugare nella \emph{mia}
borsa... impresa non da poco, con un braccio ancora paralizzato e
inutile. Non dubitavo che ci fosse una videocamera da qualche parte
sopra di me, ma non pensavo che l'informazione sarebbe arrivata ai
prefetti prima che mi oltrepassassero. Ignorai la carta universale che
c'era nella borsa e anche i documenti di identificazione, entrambi
appartenenti all'uomo incipriato e vestito con un sarong a cui l'avevo
rubata. Trovai un paio di occhiali leggermente ovoidali -- con la
montatura in argento e le lenti di vetro color rubino -- e me li
sistemai sul naso, poi mi presi un momento per raccogliere i capelli
lunghi e legarli con una fascia elastica, anch'essa recuperata nella
borsa. Soprattutto, trovai una banconota da mezzo kaspum e quasi un
altro kaspum in monete d'acciaio, che mi misi in tasca con un accenno di
sorriso prima di posare la borsa accanto alla mia sedia.

Alle mie spalle sentii un martellare di stivali sulla pavimentazione,
non mi girai ma chinai la testa, fingendo di studiare il menù
incastonato nel tavolo.

«Non ti preoccupare, non dirò niente» commentò una voce rude.

Sollevando lo sguardo vidi un uomo anziano che mi sorrideva dal tavolo
vicino. Lui inarcò le sopracciglia e sorrise da sopra l'orlo del
bicchiere. Su tre tavoli, noi eravamo i soli clienti. Un vecchio libro
chiuso era posato accanto a un pugno dell'uomo, che indossava una giacca
indaco tagliata secondo la moda nipponese, con le maniche squadrate
decorate da un motivo di diamanti oro e neri intorno ai polsini. I
capelli brizzolati erano tirati indietro in una coda unta, sfilacciata e
rigida come il pennello troppo usato di un calligrafo.

Decisi di fare lo gnorri. «Prego?» La mia voce risuonò stanca,
mascherando in qualche misura l'effetto del mio scandito accento di
Delos.

«Hai il braccio paralizzato da uno storditore e respiri come il premiato
cavallo da corsa della duchessa Antonelli. Non ci vuole uno scoliasta
per capire che hai fatto una corsa.» Non era {nipponese}, a giudicare
dalle linee dure del suo volto o dalla voce tenorile. A dire il vero non
riuscivo a decifrare il suo accento. Durantino, forse? O delle Proprietà
Normanne, anche se non aveva i loro colori. Mi alzai di scatto e mi
girai per andarmene. «Non fare così, il Vecchio Corvo non dirà una
parola. Perché non ti siedi con me per un po'?» Diede un calcio a una
sedia con un piede rivestito da un sandalo. Quando non risposi e non mi
mossi sospirò, passandosi dietro l'orecchio una ciocca libera di capelli
grigi. «Da quanto tempo?» chiese.

«Da quanto tempo... cosa?»

«Da quanto tempo sei bloccato qui, ragazzo! Hai addosso il puzzo dello
spazio, e chiunque abbia occhi per guardare può vedere che non sei un
Emeshi.»

Lasciai ricadere il braccio intorpidito e ricacciai indietro l'impulso
di protestare per essere chiamato `ragazzo', ma non accettai la sedia
offerta. «Non lo so» risposi. «Un paio d'anni.»

Il Vecchio Corvo si accigliò. «Un paio d'anni...» Scosse il capo,
facendo oscillare la coda. «È dura.»

Mi arrischiai a lanciare un'occhiata alle mie spalle, dove i prefetti --
erano in quattro -- si stavano aggirando fra la folla. D'un tratto,
nervoso, mi lasciai cadere sulla sedia di fronte a quella dall'altro
uomo.

In base alla sua carnagione, alla stranezza del suo vestiario e al suo
accento era chiaro che era un Emeshi quanto lo ero io, quindi chiesi:
«Da dove vieni, messere?» Mi sarei potuto prendere a calci, perché quel
`messere' mi faceva apparire come una persona istruita e non il comune
ladro che sembravo essere.

Il Corvo sorrise come se avesse capito e agitò una mano in un gesto
evasivo. «Oh, da un mucchio di posti diversi.» Sbirciò oltre la mia
spalla, guardando i prefetti che continuavano a girare per la strada in
direzione di uno dei molti ponti che descrivevano un arco sul canale.
«Però sono \emph{diretto} ad Ascia, nel Commonwealth. E tu?»

Mi morsi un labbro e abbassai lo sguardo sul tavolo. Non volevo
rispondere e rimpiangevo di essermi seduto. Alla fine cominciai: «Vengo
da...»

Lui calò una mano sul tavolo. «Non quello. Dove stai \emph{andando}?»

Sollevai lentamente lo sguardo per incontrare il suo, sentendomi
confuso. «Da nessuna parte. Che ti importa? Non dovrei neppure parlarti.
Qualsiasi cosa tu venda...»

«Non vendo un dannato accidente di niente» replicò il Corvo, grattandosi
dietro un orecchio. «Vedo solo un uomo che ha bisogno di tutto ma non ho
niente da dare.» Gemette, appoggiandosi in avanti contro il tavolo. «Le
ossa non si scongelano come facevano una volta. Dannate capsule
criogeniche.» Nonostante le sue lamentele sembrava essere sulla soglia
della mezza età. Non riuscivo a intuire la sua casta, o anche solo se ne
aveva una. Allargò una mano sul piano del tavolo e solo allora mi resi
conto che era almeno un po' ubriaco. «Un uomo deve andare da qualche
parte. È per questo che il Vecchio Corvo...» Accennò al cielo, poi si
interruppe. «In ogni caso...»

Mi alzai per andarmene, controllando nelle tasche in cerca delle monete
che avevo preso dalla borsa rubata. C'erano ancora, e accartocciai fra
le dita la banconota da mezzo kaspum. I prefetti erano svaniti oltre una
curva e li cercai con lo sguardo sbirciando attraverso le lenti degli
occhiali rossi che avevo rubato. Il marinaio non obiettò al fatto che me
ne andassi, ma disse: «Questo pianeta è un buco di merda, sai? Come fa
un uomo a respirare? Perfino il whisky è sbagliato.» Rivolse una smorfia
al suo bicchiere.

`Sbagliato.'

Era esattamente la stessa parola che avevo pensato un centinaio di volte
dal mio brusco risveglio su questo mondo. Era tutto sbagliato. La luce
del sole, le due lune, perfino l'aria. Dalla mia condizione confortevole
al Riposo del Diavolo non avrei mai potuto immaginare la mia situazione
attuale e non l'avrei augurata neppure al più infimo dei popolani. D'un
tratto, le monete rubate parvero un peso terribile nella tasca, un giogo
sulle mie spalle.

Il marinaio ubriaco stava ancora parlando, a quanto pareva non tanto
rivolto a me quanto al mondo in generale. «Ascolta il Vecchio Corvo. Un
uomo deve avere un fuoco sotto di sé.»

«Come sei diventato un marinaio?» chiesi, senza riflettere. C'era una
vita migliore, una a cui avrei potuto aspirare nella mia infelicità. Il
pensiero di Simeon il Rosso, del viaggiare nella notte eterna, mi
pervase. Anche se non sarei mai stato uno scoliasta o un membro dei
Corpi di Spedizione, non era detto che l'avventura e i misteri che
cercavo, come pure il prestigio dell'esplorazione, mi fossero preclusi.

Il Corvo mi guardò inclinando la testa e grattandosi di nuovo un
orecchio con espressione tale da far supporre che gli avessi rivolto una
domanda difficile. «Ci sono nato. Tanti di noi lo sono.» Mi puntò contro
un dito e contrasse il pollice come il cane di un'antica arma da fuoco,
poi mi strizzò l'occhio e mi guardò in tralice. «Non starai pensando di
arruolarti, vero?»

Il cuore mi balzò in gola. Il pensiero di essere di nuovo su una nave
fredda, pulito e nutrito, crebbe dentro di me. Al di sopra della
ringhiera del caffè lanciai un'occhiata in direzione della strada per
controllare se le uniformi cachi dei prefetti urbani fossero in vista.
Interpretando il mio silenzio come un assenso, il Corvo aggiunse: «È una
vita dura, fratello. Faresti meglio a trovarti un lavoro nelle serre
piuttosto che imbarcarti. Per la Terra nera! Diventa un pescatore. I
pescatori si guadagnano onestamente da vivere senza tutto quel dannato
congelamento.» Si massaggiò la nuca con una smorfia. «Senti, hai pensato
ai combattimenti nell'arena? Un ragazzo vigoroso come te... non hanno
carenza di posti liberi nel Colosso. Qualcuno deve prendere il posto dei
ragazzi morti.»

«Sei un reclutatore?» chiesi.

«Se sono un reclutatore?» ripeté. «Merda, no. Mi piace soltanto un bel
combattimento.» Si protese in avanti e mi puntò un dito al petto. «Tu
però saresti grandioso, hai la corporatura giusta.» Guardò poi con aria
cupa il contenuto del suo bicchiere e aggiunse: «Però ricorda... diventa
un mirmidone, non un gladiatore. Alle ragazze non piacciono gli
assassini professionisti.»

Quella frase mi colpì come una cosa strana. Le ragazze adoravano i
gladiatori soprattutto perché erano uccisori. Eroi. Veri uomini. A
nessuno piaceva un cadavere. Nel ripensarci, suppongo fosse una
profezia, o forse dipese solo dal fatto che quel vecchio bastardo era
uno straniero.

«Ci penserò.» Mossi un piccolo passo per allontanarmi in direzione del
cancello che dava sulla strada, consapevole adesso che il mio essere in
piedi stava attirando l'attenzione di alcuni degli altri clienti.
L'impulso di fuggire lottava con il mio desiderio di rimanere, e dovevo
apparire come uno sciocco, spostando in quel modo il peso del corpo da
un piede all'altro. Se fu così, il marinaio non vi badò. Aveva ripreso a
bere, digitando sul servizio di ordinazione inserito nel tavolo. Mi
parve di vedere una macchia cachi fra la folla e mi ritrassi di scatto.
Quel movimento attirò l'attenzione del marinaio, e per un momento
l'ubriachezza parve scomparire da quel volto rude.

«Non puoi andare avanti così, fratello.» Agitò un dito nella mia
direzione. «Non è modo di vivere, per nessuno. Questa volta sei stato
fortunato che si trattasse di me, ma la prossima volta che ti infilerai
in un caffè con una borsa rubata... qualcuno potrebbe chiamare le
guardie» continuò, abbassando la voce.

Guardai i miei piedi nudi coperti di fango secco. Volevo chiedergli di
portarmi con sé, ma poi pensai a Cat -- al suo sorriso e ai suoi denti
storti -- e mi morsi la lingua.

«Tu non mi conosci» ribattei in tono tagliente.

«Certo che no,» convenne il Corvo «ma prese individualmente le persone
non sono diverse, non importa cosa ci sia all'esterno, e neppure dove
sei stato. Il dolore, il bisogno sono gli stessi. Il bisogno di qualcuno
che ricordi loro che sono umane.» Si grattò pensosamente un orecchio.
«Hai dei mezzi, ragazzo, anche quando non hai niente. Hai te stesso, e
non mi riferisco al sangue.» Giuro che dal modo in cui inarcò le
sopracciglia in quel momento mi riconobbe per quello che ero: un nobile,
un palatino, la progenie di un pari dell'Impero, ora caduto in
disgrazia. Poi però sorrise, con il lampeggiare di un dente d'oro, e
l'impressione che possedesse la saggezza dei sapienti venne infranta e
oscurata da quella rudezza nello stesso modo in cui un antico frammento
di vasellame accenna soltanto all'antico Impero che lo ha modellato.
«Chi sei non importa un accidente, o comunque non conta abbastanza.
Quello che conta è ciò che \emph{fai}, hai capito?»

Diceva la verità, ma non lo stavo effettivamente ascoltando e passò
molto tempo prima che le parole di quel marinaio ubriaco penetrassero in
quel bunker corazzato che io chiamavo cranio. Comunque annuii mentre
allungavo il collo per guardare oltre l'angolo attraverso le finestre
del caffè.

Alle mie spalle sentii un grido provenire dalla strada e nel guardare
scorsi uno dei prefetti indicare dritto verso di me. Il sangue mi
accelerò nelle vene mentre il Corvo sollevava il bicchiere in un saluto.
«Corri,» disse «ma corri \emph{da qualche parte}, eh?»

Corsi.

