\chapter{Familiarità}

«E \emph{iudaritre} significa `tagliare'?» chiese Dorian, abbandonando
lo jaddiano per chiarire la cosa nel galstani imperiale.

Lo accontentai con un sorriso tagliente. «È esatto, Vostra signoria.»
Tecnicamente significava `aver tagliato' -- era l'infinito passato -- ma
la cosa era irrilevante ai fini del nostro esercizio. Tormentai la
catena che portavo al collo, a cui era appeso l'anello che avevo
recuperato dalla squadra di recupero delle astronavi in quel mio primo
giorno su Emesh, un'eternità prima. Guardando fuori dalle finestre
schermate potevo intravedere sulla riva del mare l'astroporto di
Borosevo, una piatta e bianca distesa di cemento punteggiata dai crateri
rotondi delle fosse di lancio. Ricordando l'obbligo che il conte Mataro
mi aveva imposto in cambio di una posizione sicura, tornai a esprimermi
in jaddiano. «Il trucco con i fendenti orizzontali è di attaccare il
braccio, se si è abbastanza veloci» spiegai, continuando il resoconto di
un combattimento contro la gladiatrice Amarei di Mira. Amarei era stata
ospite a corte due settimane prima e il giovane nobile era rimasto molto
colpito da lei. «Soprattutto se stai maneggiando le armi più corte.»
Posai la matita, soddisfatto, e girai il diario in modo da rivelare un
ritratto a carboncino del giovane con indosso l'armatura da gladiatore
in stile antico usata dai mirmidoni e non quella ad alta tecnologia con
misure di sicurezza che portavano Amarei e i suoi compagni.

Il figlio del conte lo guardò con apprezzamento e chiese delle origini
del mio talento. «\emph{Pou imparato iqad... rusimatre}?»

«\emph{Rusimiri}» lo corressi, e scrollai le spalle. «Fin da quando ero
piccolo mi è sempre piaciuto disegnare.» Sollevai di nuovo la matita e
lanciai un'occhiata in tralice ad Anaïs, che sedeva con indosso gli
occhiali da simulazione da cui un qualche fantasy veniva trasmesso
direttamente alle sue retine e nelle sue orecchie. Indicandola, dissi:
«Mio padre ci scoraggiava da attività del genere... diceva che non erano
sacre... quindi ho preso l'abitudine di disegnare. Il mio scoliasta ha
appoggiato la cosa, sostenendo che era un hobby classico, una vocazione
appropriata.» In jaddiano, la parola \emph{munjin, `}vocazione',
significa anche `talento', per cui quella era una sottile vanteria da
parte mia, anche se quella sottigliezza sfuggì a Dorian.

«Sei molto bravo! Dovresti considerare la possibilità di lavorare come
ritrattista reale. Anaïs, vieni a vedere!» La ragazza non reagì
immediatamente, quindi suo fratello prelevò una ciliegia da un piatto e
gliela lanciò contro. Con uno strillo lei si tolse gli occhiali dalla
faccia. «Hadrian mi ha ritratto!» annunciò Dorian.

Sua sorella si alzò con mosse languide da gatto e un'espressione di
petulante frustrazione sul volto grazioso, che si trasformò in una di
deliziata sorpresa. «Oh, ma questo è meraviglioso!» Mostrò in un sorriso
i denti matematicamente perfetti e si chinò sul tavolo in modo tale da
offrire una visuale nella scollatura della sua blusa. Arrossendo,
distolsi lo sguardo mentre Anaïs si sistemava su una sedia accanto al
fratello. «Il prossimo ritratto lo farai a me?»

«Dovremmo parlare in jaddiano, mia signora» la ripresi in tono caustico,
inserendo la matita nell'economico temperamatite di plastica che le
guardie dei due giovani nobili mi avevano dato quando mi avevano
confiscato il coltello per appuntire le matite.

Anaïs mise il broncio e incrociò le braccia sotto il petto. «Oh,
d'accordo.» Inclinò la sedia all'indietro sulle gambe posteriori.
«Pensavo steste parlando del Colosso.»

«Infatti!» esclamò Dorian, battendo sul disegno nel mio diario e finendo
per sbavare le delicate linee di carbone. «È stato per questo che mi ha
ritratto come un mirmidone.» Poi procedette ad aggiornare la sorella
riguardo al mio duello con Amarei di Mira, quello in cui avevo bloccato
le funzioni della sua tuta con parecchi piccoli attacchi, paralizzandola
progressivamente.

Quando ebbe finito, Anaïs batté le mani con apprezzamento. «Tornerai
indietro?» chiese poi.

«\emph{Alla}... Colosso?» domandai. `Al Colosso?' Non conoscevo
l'equivalente jaddiano di `Colosso' e non sapevo neppure se ce ne fosse
uno.

«Sì!» Anaïs si illuminò in volto. «Ci potresti tornare come gladiatore!
Saresti assolutamente al sicuro.»

«Non ricominciamo!» Dovetti trattenermi dallo scattare in piedi serrando
le mani sui braccioli della sedia, mentre la falsa accusa di Valka mi
risuonava nella mente. `Dimmi, M Gibson, ti piaceva uccidere gli schiavi
per i tuoi padroni?' Abbassai e distolsi lo sguardo. «Molti dei
mirmidoni sacrificabili sono miei amici, Vostra signoria.»

Dorian fece una smorfia, ma Anaïs ribatté: «Ecco, di certo questo non
sarà vero per molto tempo...» Non aveva considerato il pieno significato
delle sue parole, e quando se ne rese conto, un momento più tardi, il
colorito scuro del suo volto virò leggermente sul verde. Abbassò lo
sguardo e borbottò: «Mi dispiace.»

Come palatino, come Hadrian Marlowe, avrei potuto offendermi, ma Hadrian
Gibson non godeva di quel lusso. «Vostra signoria è molto gentile nel
comprendere la mia situazione.» Ufficialmente parlando, non potevo
neppure ammettere che mi aveva insultato. Peraltro, il suo rammarico
pareva averla momentaneamente intimidita. «Chiedo perdono a Vostra
signoria, ma di recente l'opinione della dottoressa Onderra ha in certa
misura influenzato la mia.»

«La dottoressa Onderra» ripeté Anaïs. «Dobbiamo proprio parlare di
quella Tavrosiana? Presto se ne andrà.»

Mi irrigidii e girai la pagina del mio diario per nascondere quella
reazione. Se ne sarebbe andata così presto? Per tornare a Calagah,
naturalmente. Presto le maree estreme di Emesh sarebbero cambiate e a
quel punto le sale e le caverne delle sue rovine sarebbero emerse dalle
profondità del mare. Valka era qui solo per lavorare con gli Umandh
della città e apprendere quello che poteva nella sua stagione di pausa.
Non appena il vero punto focale del lavoro della sua vita fosse tornato
disponibile se ne sarebbe andata.

Dorian si intromise nelle mie riflessioni. «La sua opinione sui giochi?
Non combattono, nella Demarchia?» La bocca mi si contrasse e dovetti
reprimere un'espressione accigliata, perché non avevo una risposta
concreta a quella domanda. Ero certo che {dovessero} avere competizioni
di qualche tipo, ma non avrei mai potuto intuire che forma assumessero.
«Credo soltanto che non abbiano un Colosso vero e proprio, Vostra
signoria. Forse questa è una domanda che andrebbe rivolta alla
dottoressa.»

«Sono troppo impegnati a adorare le loro macchine» commentò Anaïs, con
un sogghigno, appoggiandosi al tavolo e puntellando il mento sulle
braccia.

Per un momento studiai il suo volto con la matita appena appuntita in
mano, poi mi misi al lavoro. «Su Tavros non adorano le macchine.»

«Comunque sono eretici» dichiarò la ragazza, dondolando la testa contro
le braccia posate sul tavolo. Cominciai a tracciare i contorni del suo
volto. «Non so perché nostro padre la tolleri.» Contrassi le labbra,
ricordando quanto fosse stata cordiale quando mi aveva presentato la
dottoressa e mi chiesi il perché di quel cambiamento con ingenua
perplessità, senza vedere il ruolo che avevo avuto in esso.

«La sua spedizione è sponsorizzata da sir Elomas Redgrave, sorella, lo
sai. Il sito di Calagah è solo un insieme di vecchie gallerie... perché
non lasciare che quella straniera scavi in giro? Che male può fare?»

«È solo che non mi piace. Gilliam dice che è una strega, che si è donata
alle macchine.» Rabbrividì. «Afferma che in realtà non è più davvero
umana.»

Il figlio del conte inarcò le sopracciglia e si grattò i capelli fra il
blu e il nero. Aveva gli zigomi alti e gli occhi sottili di lord Luthor,
anche se in qualche modo in lui l'effetto era più di sincerità che di
diffidenza. Appariva perennemente sorpreso. «Gilliam è un prete e deve
dire cose del genere. I Demarchici sono strani, te lo concedo, ma nella
dottoressa non c'è niente di inumano. Se devo dire la verità, la trovo
stupenda. E tu, Hadrian?»

Sussultai al punto che lasciai cadere la matita. «Cosa?» Lo guardai
negli occhi. «Oh. Sì.» Non aggiunsi che passavo un paio di sere alla
settimana in sua compagnia, discutendo degli Umandh e dei diversi posti
in cui era stata. «Sai, è una brillante xenologa. Sapevi che è stata su
Giudecca? Che ha visto la tomba di Simeon il Rosso ad Athten Var?»

«Davvero?» Dorian inarcò un sopracciglio ben curato e sgranò gli occhi
scuri. «È incredibile.»

Anaïs sospirò e si sedette, riprendendo gli occhiali. «Non so cosa ci
troviate in lei. Una straniera del genere...»

Sorrisi, ma nelle mie parole ci fu una sfumatura di gelo. «Sono uno
straniero anch'io, se non dispiace a Vostra signoria.»

La ragazza ebbe la buona grazia di mostrarsi ammansita, ma suo fratello
ci interruppe protendendosi in avanti sul tavolo in una posa accasciata
che ricordava quella della sorella. «Senti, posso avere quel ritratto
che mi hai fatto?»

Involontariamente serrai la mano intorno alla matita. L'ultima cosa che
volevo era strappare una pagina dal mio bel diario, che era perfetto, ma
non potevo opporre un rifiuto alla richiesta del giovane nobile.
«Certamente, Vostra signoria.» Con una precisione esagerata staccai lo
spesso foglio bianco e lo feci scivolare sul tavolo verso il giovane
nobile. Strapparlo era stato come spezzarmi un osso.

«Credi che mio padre mi lascerà combattere nel Colosso? Come
gladiatore?»

Sollevai lo sguardo su di lui mentre riprendevo a lavorare di matita al
ritratto di Anaïs. «Potrebbe farlo. Mio padre ha permesso a mio fratello
di farlo.»

Dorian si rincuorò immediatamente. «Dopo il mio Efebeia sento che dovrei
farlo. Non mi ha permesso di uccidere il Cielcin.»

Tracciai i contorni dei capelli di Anaïs, accigliandomi. «A quanto mi è
dato di capire, la decapitazione non è una cosa facile. La Spada Bianca
non è di altamateria, sai, e sono certo che i tuoi padri volessero solo
accertarsi che la cosa venisse fatta in modo corretto.»

«E comunque era una cerimonia della Cappellania.» Anaïs si raddrizzò di
nuovo sulla sedia. «Alla vecchia Ligeia piace accertarsi che tutto sia
fatto nel modo giusto.» Si spinse lontano dal tavolo, ci volse le spalle
poi tornò a girarsi con entusiasmo e con un movimento quasi felino.
«Potresti insegnarci!»

Quel pensiero non era collegato alla conversazione precedente, e questo
mi portò a sollevare lo sguardo su di lei con aria confusa. «Nel sacro
nome della Terra, di cosa stai parlando?» Posai la matita nella piega
rovinata del mio diario.

Anaïs accennò a sé stessa e a suo fratello. «A combattere con la spada,
intendo. Sei un mirmidone, e uno bravo! Ti ho cercato negli ologrammi.»

«Non avete un maestro d'armi?»

«Non uno molto buono, solo il vecchio sir Preston Rau. Quell'uomo è
orribile. Tu sarai molto meglio! Dorian, digli che sarà meglio.» Si girò
con fare implorante verso il fratello e gli assestò un colpo sul
braccio.

Aveva colto Dorian nell'atto di mangiare un'altra ciliegia dalla ciotola
ghiacciata che si trovava sul tavolo. Lui si spinse il frutto contro la
guancia e replicò: «Perché no? Potrebbe essere divertente.»

Aprii la bocca, la richiusi, tornai ad aprirla, ma Anaïs mi batté sul
tempo nel replicare. «In realtà non combatteresti, insegneresti
soltanto.» Si guardò le mani, poi sollevò timidamente lo sguardo da
dietro il velo di riccioli neri e io gemetti interiormente. Sapevo
capire quando ero stato battuto in astuzia.

\begin{figure}
	\centering
	\def\svgwidth{\columnwidth}
	\scalebox{0.2}{\input{divisore.pdf_tex}}
\end{figure}

«Ma perché ci dobbiamo addestrare \emph{fuori}?» gemette Dorian, dopo
che lo ebbi disarmato per la dodicesima volta, e allontanò con la mano
un littore che voleva aiutarlo a rialzarsi... anche questo per la
dodicesima volta.

I miei piedi nudi strisciarono sull'erba tagliata alla perfezione e mi
battei contro la spalla la spada da addestramento di plastica e schiuma.
Avevamo accantonato l'addestramento con lo scudo in favore di canali di
comunicazione più chiari, una mossa che aveva eccitato i miei due
compagni quasi quanto aveva seccato i loro uomini della sicurezza. Dopo
aver ricordato a un sergente dalla mascella marcata che era improbabile
che uccidessi Anaïs o Dorian davanti a una decuria di peltasti del
Casato e sotto la piena luce del sole, l'uomo si era arreso. Battei la
spada contro il terreno davanti ai miei piedi. «I tuoi mirmidoni si
addestrano all'aperto, Vostra signoria.» Lanciai un'occhiata ad Anaïs,
la cui tuta le aderiva al corpo come uno strato di olio bianco. Va a suo
merito che non avesse pronunciato una sola lamentela, lasciando quel
ruolo effemminato al fratello maggiore. «È un bene abituarsi a
combattere in condizioni come queste. Così, quando avrai l'aria
condizionata l'apprezzerai di più.» Sir Roban mi aveva detto qualcosa di
molto simile quando ero bambino, la prima volta che mi aveva fatto
correre in giro per i terreni del Riposo del Diavolo in pieno inverno,
invece che nella palestra del casello.

Asciugandosi il sudore dalla faccia Dorian spostò l'arma dalla mano
destra alla sinistra. Aveva dimostrato una frustrante tendenza a essere
ambidestro, cosa che richiedeva il doppio di esercizio nei movimenti dei
piedi. Anaïs non aveva scherzato riguardo al loro addestramento: non
volevo parlare male di quel loro sir Preston Rau, ma i ragazzi erano i
duellanti più incompetenti che avessi mai visto. Possibile che sir Felix
fosse stato un insegnante così bravo? Pensare a Felix mi ricordò Gibson
-- quello che gli era successo -- e dovetti distogliere lo sguardo dai
miei due allievi per timore che potessero leggere sulla mia faccia il
dolore che provavo.

«Suppongo che abbia senso, ma... per la Terra, fa davvero caldo.»

«La vuoi smettere di lamentarti?» interloquì Anaïs, posizionandosi
davanti a me per il suo turno. Inclinò la spada fra di noi con una mano
in avanti nella tradizionale guardia da sciabola. «Hai detto che sarebbe
stato divertente.»

Dorian fece una smorfia e si sedette all'ombra di un pergolato incolto
addossato a un muro del cortile, appoggiando la spada accanto a sé. «Ho
detto che avrebbe potuto essere divertente, ma questo è soltanto...
sciogliersi.»

«Vostra signoria, mi perdoni.» Mi girai a guardarlo. «Se vuoi combattere
nel Colosso, è meglio che ti ci abitui.»

«Le tute dei gladiatori sono raffreddate ad acqua» obiettò Dorian.

Questa volta fui io a fare una smorfia, espressione che si trasformò in
un ringhio quando Anaïs sferrò un attacco di soppiatto diretto al
fianco. Lo parai senza neppure girarmi a guardare, focalizzandomi in
tempo perché la mia risposta la raggiungesse alla spalla. Il tessuto
memory della sua tuta si fece rosso dove l'avevo colpita e lei si
accigliò. «Come hai fatto?»

«Sfrutti sempre lo stesso fendente laterale da fuori la linea» spiegai,
riferendomi alla linea dipinta di un vero e proprio cerchio di scherma
che ci sarebbe stata se si fosse trattato di un duello formale, poi
mimai il colpo al rallentatore per farle vedere. «Ogni volta punti a un
successo facile, mirando al fianco sinistro. Prova qualcosa...»
Ululando, lei calò la spada come quella di un boia. Non era forte,
quindi parai facilmente l'attacco, mi spostai di lato e ruotai su me
stesso per premerle la lama della mia spada contro lo stomaco.
Consapevole delle guardie che ci osservavano, non portai l'attacco fino
in fondo anche se il relativo dolore le avrebbe insegnato meglio a
ricordare il suo errore. Anaïs indietreggiò mentre la pressione del mio
colpo tracciava sul suo torso una linea rossa simile a un brutto
gonfiore. Mi sorpresi a desiderare che avessimo avuto quel genere di
vestiario quando ero ragazzo. Felix era un tradizionalista, ma almeno
quelle tute avrebbero impedito a Crispin di negare che lo avevo colpito.
«Hai solo bisogno di esercizio.»

Continuammo in quel modo, come avevamo già fatto per tutta l'ora
precedente, con Anaïs e Dorian che facevano a turno ad affrontarmi.
Nessuno dei due mandò a segno un solo colpo, ma c'era da aspettarselo. I
peltasti abbigliati nell'armatura verde e oro dei Mataro si agitarono
nervosamente a ogni colpo, ma intervennero sempre di meno quando risultò
evidente che non avrei ucciso i loro protetti con le spade da
addestramento di schiuma.

Dorian deviò con successo uno dei miei attacchi, ma poi incespicò nel
rispondere con un affondo. Lasciai che cadesse, sporcandosi di fango le
ginocchia nell'erba, poi gli offrii doverosamente una mano e lo aiutai a
rialzarsi proprio mentre la familiare voce strascicata di Gilliam Vas
risuonava in quello spazio ristretto. «Eccovi qui, giovani signori.» Si
raggelò nel vedermi in piedi accanto allo sconfitto Dorian, e le sue
narici disuguali si dilatarono. «Ancora tu!»

«I giovani signori mi hanno chiesto di istruirli nella scherma, Vostra
reverenza.» Spinsi la spada dietro di me e mi inchinai. «Mi ritiro.»

«Hadrian, no!» intervenne Anaïs.

Gilliam Vas si girò a lanciare un'occhiata alla scorta, che se ne era
andata al suo ingresso nel cortile. «Mio signore Dorian, mia signora
Anaïs, i vostri padri mi hanno mandato a prendervi.»

I giovani nobili si mossero in avanti mentre io raccoglievo le loro armi
e mi accingevo ad andarmene, lieto di scivolare in una servile
invisibilità. «Cosa succede, Gil?» domandò Dorian. \emph{Gil?}

«Niente di importante. Lord Balian desidera che lo accompagniate.» Il
prete allargò le mani. «Credo abbia menzionato una gita in orbita.»
Gilliam lisciò i capelli unti, allontanandoli dalla fronte,
apparentemente per nulla disturbato dal caldo e dall'aria densa e
soffocante. Mentre lo oltrepassavo mi afferrò un braccio con dita di una
forza sorprendente. «Un momento, M Gibson.»

«Certo, Vostra reverenza.» Con tre spade sotto il braccio mi trassi di
lato, frugandomi nelle tasche dei calzoni in cerca dei miei occhiali
rossi... qualsiasi cosa pur di mettere uno scudo fra me stesso e
quell'orribile prete. Gilliam spinse i giovani nobili all'interno del
palazzo, affidandoli alla custodia dei peltasti del Casato mentre io
restavo a oziare per qualche tempo nel cortile assolato, con le dita dei
piedi nudi che si contraevano nell'erba morbida.

Al suo ritorno Gilliam mi trovò all'ombra del pergolato, appoggiato a
una delle spade da addestramento e con le altre due appoggiate a una
vicina colonna. Senza preamboli mi afferrò per i bicipiti e si protese
verso di me. «Qual è il tuo gioco, straniero?»

«Prego?»

«Appena poco tempo fa eri al gradino più basso del colosseo e adesso...
adesso fai la lotta con i giovani signori.»

Inarcai le sopracciglia al di sopra del bordo ovale degli occhiali. «La
lotta? Cantore Vas, i giovani signori mi hanno chiesto di mostrare loro
qualche mossa dei tempi in cui combattevo nell'arena. Sarebbe stato
scortese rifiutare.»

«Scortese?» ripeté Gilliam, snudando i denti raddrizzati
artificialmente. «Scortese?» Mi lasciò andare e indietreggiò di un passo
barcollante, come se ripetere due volte quella parola gliene avesse
ricordato il significato. Ergendosi sulla persona quanto più gli era
possibile, levò in alto il mento. «Alcuni fra i cortigiani di Sua
eccellenza ritengono scorretto che un uomo della tua... condizione si
mescoli in modo così evidente con la casta palatina.»

Questo gli fruttò il mio sorriso più aspro, tagliente e deliberato. «La
mia \emph{condizione}? Il conte stesso ha richiesto che stessi con i
suoi figli.»

«Lord Balian ha strane idee in fatto di decoro» ribatté Gilliam, con
tagliente soddisfazione. Era una battuta? Era risaputo che i vecchi
pregiudizi rialzavano la testa di tanto in tanto anche in seno alla
casta palatina. Apparentemente consapevole del suo errore, Gilliam
arrossì, ma quel passo falso servì solo a farlo irritare e contrasse le
sopracciglia su quei suoi occhi spaiati. «Senti, stai mostrando troppa
\emph{familiarità} con i figli del conte, e questo non è... appropriato.
Hai capito?» Questo da parte di un intus bastardo, un mutante, l'avatar
incarnato della scorrettezza. Era davvero quasi troppo per il mio
raffinato senso dell'ironia, e dovetti reprimere un sottile sorriso.

«Appropriato?» ripetei, facendo il finto tonto. «Se pensi che abbia
toccato lady Anaïs, ti assicuro che non ho nessuna intenzione di farlo.»
Cos'era Gilliam per loro, o loro per lui? Era solo puritanesimo di
corte? La protezione del sangue palatino dall'umanità di umile nascita a
cui Gilliam pensava che appartenessi? Lui era un palatino, ma il suo
difetto lo marchiava come qualcosa che sotto molti aspetti era meno di
un omuncolo, e ho constatato spesso che emarginati del genere sono
quelli che maggiormente si attaccano alle etichette loro negate. È per
questo che uomini deboli sono fra i più aggressivi, e i meno abili
quelli che si vantano con voce più alta. Gilliam era un palatino, quindi
il fatto di pensare che io non lo fossi era importante per lui. La sua
era poco più che mera petulanza.

«Toccare lady Ana...» Si interruppe, con la voce che gli si strozzava in
gola mentre ripeteva le mie parole. «Un degenerato come te e i giovani
signori...» Rabbrividì, contraendo la mascella come se stesse cercando
di lacerare una striscia di cuoio bollito, e per un momento pensai che
potesse colpirmi.

Con estrema cautela, attingendo alla mia supposizione e usando la mia
voce oratoria più cortese, dissi: «Assicuro a Vostra reverenza che le
mie intenzioni nei confronti dei giovani signori sono del tutto
innocenti. Sono a corte solo per ordine del conte, e se avessi potuto
scegliere mi sarei imbarcato sulla prima nave in partenza dal sistema.»
Non aggiunsi `ma sto fuggendo dal mio stesso Casato e sono intrappolato
qui per proteggere il tuo signore dall'Inquisizione'. Rabbrividii al
pensiero di quello che un inquisitore della Cappellania avrebbe fatto a
un nobile Casato sorpreso a dare asilo a un fuggitivo come me.

«Allora spiega il tuo spionaggio.»

«Il mio... cosa?» Lo fissai interdetto da dietro gli occhiali rossi. «Ti
riferisci alla mia visita nelle celle del colosseo?»

Gilliam si accigliò. «Sei penetrato nelle prigioni di Sua eccellenza.
Non puoi sostenere in tutta onestà che le tue intenzioni fossero
innocenti.»

«Lo erano!» protestai, forse con troppo fervore. «Ecco, forse non
innocenti, ma innocue! Volevo solo incontrare quella creatura,
parlarle.» Quella, almeno, era una cosa ragionevole. Perfino io dovevo
ammettere che penetrare nelle prigioni per vedere Makisomn {poteva}
apparire una cosa tutt'altro che innocente, il che faceva sembrare la
verità una debole scusa.

«La fraternizzazione è un grave peccato M Gibson, uno dei Dodici» sibilò
il cantore, tracciando inconsciamente il segno protettivo del sole lungo
il fianco. «Cosa potevi imparare da una simile bestia?»

«Non ne ho idea. Volevo solo vederla con occhi non offuscati.»

«Con occhi non offuscati» mi derise Gilliam, con la voce che saliva
verso una nota acuta; anche se dal modo in cui le sue sopracciglia
aggrottate si rilassarono capii che lo avevo sorpreso. Quindi quella non
era la risposta che si era aspettato. «Perché?» chiese con freddezza.

«Innocente curiosità.» Scrollai le spalle, consapevole che la risposta
-- pur essendo quasi vera -- non lo avrebbe soddisfatto. Forse avrei
dovuto parlare di \emph{ossessione}. «Volevo incontrare un membro della
sola specie che abbia mai sfidato la supremazia umana nell'universo.»

«Blasfemia!» ringhiò. «Nessuna specie può sfidare il posto
dell'umanità!» Pensai che mi avrebbe afferrato di nuovo.

Indietreggiai di un passo, con la spada di plastica che mi sussultava
nella mano. «Dillo alla nave da guerra che si sta riprendendo in orbita»
sussurrai. «Dillo alle tue guardie.» Un altro pezzo del puzzle andò al
suo posto. L'avversione di Gilliam nei miei confronti non era dovuta
solo al fatto che mi ritenesse di umile nascita o al suo ritenermi una
spia e un pericolo per il suo signore. Pensava che fossi un eretico, e
credo di esserlo stato, considerato il mio interesse per gli xenobiti.

Le sue labbra si contrassero e potei quasi vedere l'impulso di colpirmi
che passava spasmodico sulla superficie del suo cervello ottenebrato.
Invece cambiò approccio. «Mi è dato di capire che parli la loro ignobile
lingua.»

«Non molto bene.»

«Forse è meglio così.» Accennò a girarsi per andarsene. «Definiscila
innocente curiosità quanto ti pare, ma verrà il momento in cui il conte
perderà interesse per te, ragazzo. Sai qual è la punizione per la
fraternizzazione, vero?»

«Certamente.» Nonostante l'afa della giornata e la mia avversione per
quel prete deforme provai un senso di gelo, perché mi parve di sentire
il rumore dei coltelli di ceramica che venivano affilati e il suono del
ferro mentre veniva temperato che arrivava con il vento. I cathar della
Sacra Cappellania Terrestre non godono senza motivo della loro
reputazione. Eretici tanto scellerati da fraternizzare con gli inumani
venivano scuoiati, crocifissi e lasciati a morire.

Pronunciata la sua minaccia, Gilliam sorrise. «Rifletti su quello che
abbiamo detto e stai alla larga dai giovani signori.»

Era a metà strada dalle porte con i foederati della sua scorta quando
risposi. «Un momento, Vostra reverenza.» Gilliam si girò e si arrestò
pesantemente... aveva una gamba più corta dell'altra? Attese. Per non
indurre le sue guardie notoriamente nervose a puntarmi contro gli
storditori, rimasi sotto il pergolato. Volevo dire qualcosa di
minaccioso, di impressionante, volevo spaventare quel piccolo gargoyle
in vesti clericali, ma tutto ciò che mi venne in mente furono insulti
relativi alle sue condizioni, e quali che fossero i miei sentimenti
personali nei suoi confronti non intendevo abbassarmi a un simile
comportamento. Invece, mossi un passo in vanti e mi tolsi gli occhiali
per trafiggere il prete con il mio sguardo più penetrante. «Non supporre
di sapere tutto.»

