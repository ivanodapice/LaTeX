\chapter{Uomini dal sangue più rozzo}

«Ti sei esteso troppo con l'affondo, Switch!» avvertii, schivando un
colpo di Siran, uno degli altri mirmidoni della squadra con cui ero
stato alloggiato. Il ragazzo dai capelli rossi non mi ascoltò,
gettandosi contro Kiri che parò con l'asta della finta lancia e colpì il
giovane dietro un ginocchio. Switch crollò a terra con un grugnito e con
la spada corta bloccata sotto il corpo. Più in là, l'altro mirmidone
rise.

«Lascialo stare, Had,» mi gridò Ghen «lascia che il ragazzo capisca da
solo. Non è un tuo problema.»

Sollevando una mano per segnalare a Siran di fermarsi, mi sfilai l'elmo
e mi grattai la corta ricrescita sul cuoio capelluto, cercando di non
pensare a quanto dovevo di certo somigliare a Crispin. «Lo sarà se cadrà
alla fine della settimana, Ghen.»

«Ci daranno degli scudi!» esclamò Switch. «L'ho sentito dire ai tecnici.
Ci daranno degli scudi.» Il ragazzo aveva scelto di non portare l'elmo e
le sue grandi orecchie sporgevano dai capelli di un rosso acceso.

«Non ce li daranno» ribatté uno degli altri, dal lato opposto dello
spiazzo. «Gli scudi sono per i gladiatori. Devono proteggere i loro
investimenti!»

«E comunque non serviranno a niente se cadrai su quella tua sciocca
faccia» aggiunse Siran. Non era la più anziana del nostro piccolo
plotone, ma era stata nell'arena più a lungo di tutti, più per il suo
talento nel tenere bassa la testa che per una particolare abilità nel
combattimento. Come me proveniva da fuori dal pianeta ed era di pelle
più chiara della maggior parte dei nativi anche se aveva comunque una
carnagione molto più scura della mia, di una calda tonalità marrone. I
capelli erano tagliati corti sotto l'elmo rivestito da piastre di bronzo
e la faccia era sfregiata dal taglio alla narice destra, come anche
quella di Ghen, entrambi per un crimine così insignificante da non
meritare un tatuaggio sulla fronte; o forse avevano pagato la multa per
evitare quella sorte.

Switch veniva a sua volta da fuori: era stato un catamita dalla
lentigginosa pelle color latte su una di quelle navi commerciali da
spazio profondo e i suoi muscoli erano tutti per figura. Si era
addestrato per danzare, servire il tè e intrattenere gli uomini e le
rare donne da cui lo mandava il suo padrone. Non era decisamente un
combattente. Per contro, Kiri e Ghen erano entrambi nativi di Emesh, con
la pelle più scura di quanto lo fosse stata quella di Cat, ma con le
stesse origini plebee. Ghen aveva le braccia spesse e il collo ancora
più grosso di un bracciante e una robusta mascella squadrata che mi dava
l'impressione che avesse passato la vita a masticare pietre e non cibo.
Kiri era una rarità, una plebea sulla soglia della mezza età. Non era
una criminale come Ghen o Siran e neppure una vagabonda come me o
Switch. Era nell'arena perché voleva esserci. `Hanno cercato di
sottoporre mio figlio agli esami per il servizio militare' aveva detto
in tono allegro, la prima sera che avevo passato negli alloggiamenti
dopo che la dottoressa Chand mi aveva presentato al gruppo. `Dar è così
sveglio!'

«Combattiamo come una squadra» affermò Siran, riferendosi al problema
attuale e passandosi nervosamente la lingua sui denti. «Quel ragazzo è
un peso.»

«Non con una cintura-scudo!» Switch batté ansiosamente la spada contro
un polpaccio coperto dall'armatura. Senza preavviso gli lanciai contro
il mio elmo d'acciaio. Era stata mia intenzione centrarlo sulla corazza,
ma quell'arnese era sbilanciato e mi scivolò di mano, raggiungendolo
invece al ventre. Switch si piegò su sé stesso, sussultando e
farfugliando a corto di parole mentre lasciava cadere la spada. Gli
altri si immobilizzarono, non sapendo bene come reagire. Kiri trattenne
il respiro per la sorpresa. «Quello cosa significava?»

«Infine Sua radiosità è impazzito» commentò Ghen, con la sua profonda
risata. Avrei voluto ringhiargli contro, ma il grosso delinquente non
era il nostro anello debole -- il ragazzo lo era -- quindi ignorai il
soprannome e attraversai lo spazio aperto del quadrangolo sollevando
nuvolette di polvere. Potevo essere stato un misero ladro e un
mendicante ancora più inetto, ma ero stato addestrato per quel genere di
combattimento formale comune nel Colosso. Poteva non essermi mai
piaciuto, ma ci voleva più di un paio d'anni per appannare la memoria
dei muscoli.

«Una cintura-scudo non avrebbe fermato quell'elmo!» gridai. «Non fermerà
le spade o le lance.» Arrestandomi a cinque passi da Switch continuai in
tono più sommesso, imitando in quel momento più Gibson che Felix: «Non
ci daranno gli scudi, Switch, qualsiasi cosa pensi di aver sentito.»

«Non lo \emph{penso}!» ribatté, con le guance lentigginose che gli si
arrossavano. «Ho sentito che...»

«E anche se ce li dessero, questo non ci aiuterebbe.» Mantenni una parte
della mia attenzione sugli altri presenti nello spiazzo, ascoltando il
clangore di spade e lance aggrovigliate nelle geometrie del
combattimento mentre si addestravano. «Gli scudi sono per le armi ad
alta velocità: armi da fuoco, bruciatori e lance al plasma. Non
serviranno quando sarai alla portata dei loro lunghi coltelli.»

«Dovresti dare ascolto all'imperatore, ragazzo. E aiutarlo a spingersi
quel bastone più su per il culo.» Ghen scoppiò a ridere e fece un gesto
osceno con il pollice senza mai distogliere lo sguardo dalla faccia di
Switch. «Non durerai un nanosecondo quando scoppierà il casino.» Si
batté la spada di piatto contro la spalla, con il metallo che colpiva le
piastre sovrapposte.

Sempre materna, Kiri si affrettò a venire avanti per posare una mano
sulla spalla di Switch e offrirgli un silenzioso supporto mentre gli
mormorava qualcosa in tono sommesso. Siran assestò a Ghen un pugno sul
braccio. «Vuoi stare zitto?»

«Cosa?» Il grosso uomo si sfregò il naso rovinato, cercando di
nascondere il proprio imbarazzo. Mentre litigavano io mi presi un
momento per ricompormi, rimpiangendo in piccola parte di aver scagliato
l'elmo contro Switch. Da quando ero arrivato al colosseo, due settimane
prima, ero giunto a rendermi conto di quanto mi avesse indurito il tempo
vissuto sulla strada. Quelle settimane, i mesi trascorsi dopo la morte
di Cat avevano fatto il loro danno. Ricordai la rapina che avevo portato
a termine con la banda di Rells, il modo in cui li avevo traditi e il
coltello che avevo affondato nella spalla della negoziante. Quei bordi
temperati che mi avevano separato dagli altri membri della mia famiglia
erano stati scalpellati via, ero un mosaico sfregiato dall'iconoclasta.
Quegli ultimi giorni trascorsi fra i mirmidoni lo avevano reso evidente,
ero diventato un esercizio di ricostruzione. Trattenni a lungo il
respiro, lieto dell'aria più fresca della notte, piena del fastidio
delle mosche e del sibilo degli ornithon. Mi asciugai il sudore dalla
fronte.

Infine parlai, con la mia voce baritonale che aveva ritrovato in parte
l'eleganza di un tempo... uno dei vantaggi di un cibo adeguato e di
acqua pulita. «Senti, devi rinforzare il tuo lavoro di piedi.»

«Oh, \emph{devi}» ridacchiò Ghen. «Lo hai sentito, ragazzo?
\emph{Devi}.»

«Piantala» intervenne Kiri. «Lascialo in pace.» E mosse un paio di passi
verso il grosso uomo con tanta ferocia che non l'avrei sfidata neppure
se mi avessero detto di farlo. La ferocia di una madre, una cosa che in
passato avevo visto solo una volta.

«Oppure cosa farai?» ribatté Ghen, avanzando verso di lei fino a
trovarsi faccia a faccia, con il bracciante divenuto detenuto che
fissava la donna dall'alto in basso. «Una vecchia cagna come te?» Kiri
non si mosse, non colpì l'uomo più grosso ma neppure indietreggiò. Si
limitò a indurire la mascella e a fissarlo con ira, gli occhi color
ambra duri come pietra.

Siran venne avanti. «Cosa diavolo ti è preso, Ghen?»

Il grosso uomo si allontanò da Kiri e fissò l'altro criminale. «Questo è
il gruppo peggiore che abbiamo avuto da quando siamo arrivati qui,
Siran. Ma guardali!» Abbracciò con un gesto Switch, Kiri, me e gli altri
sparsi per lo spiazzo in gruppetti che si addestravano. In quel gruppo
c'erano alcuni veterani sfregiati come lui e come Siran, perlopiù
criminali. Dopo la peste, Borosevo aveva prodotto una quantità
terrificante di delinquenti e molti si erano ritrovati nell'ipogeo del
colosseo, nei dormitori intrisi di sudore che definivamo come nostra
casa. Sempre meglio che finire impiccati o affrontare tutta la gamma
delle punizioni fisiche dei cathar.

Una parte di me cominciò a mettere in dubbio la saggezza del mio piano.
Quando avevo finalmente convinto la dottoressa Chand a firmare la mia
domanda, elencandomi solo come Had di Teukros, avevo supposto che mi
avrebbe inserito in una squadra piena di duri delinquenti, uomini dal
sangue più rozzo, e non mi ero aspettato i disperati e gli affamati di
Borosevo, una madre preoccupata, un prostituto, o il gondoliere con un
braccio solo che, non avendo una famiglia, era venuto al colosseo in
cerca di una fine gloriosa. Mi chiesi cosa avrebbe potuto dire mio padre
della compagnia che frequentavo adesso, ma suppongo che sarebbe stato
felice di trovarmi nel Colosso. Credo che in questo ci sia una forma di
ironia. In un senso drammatico, era stato il mio disgusto per l'arena
che aveva messo in moto la mia disavventura e adesso ero lì, vestito con
l'armatura quasi medievale di un mirmidone sacrificabile. Neppure un
vero gladiatore, riflettei, immaginando il disprezzo di mio padre.

Ghen stava ancora parlando. «Questo è uno spettacolo di merda, Siran. Lo
sai tu come lo so io e come lo sanno Banks e Pallino.» Accennò con un
dito a un paio dei mirmidoni più anziani che si addestravano con reclute
più recenti. «E puoi scommettere il culo che lo sanno anche i promotor,
che annunceranno la nostra strage nei notiziari.» Avevo visto annunci
del genere sugli enormi schermi che dominavano gli angoli delle strade
in tutta la città. Fece scorrere lo sguardo su Siran, Kiri e alcuni
degli altri mirmidoni che si erano staccati dal gruppo per sentire cosa
fosse tutto quel chiasso.

Per tutto quel tempo tenni la bocca chiusa, attento non tanto a Ghen
quanto alla dozzina di mirmidoni che ci si erano raccolti intorno nelle
loro armature graffiate e ammaccate. Non ce n'erano due che fossero
uguali, anche se gli uomini avevano un'espressione identica. Avevano
tutti la bocca serrata e gli occhi dilatati di un cervo che abbia visto
il cacciatore. Non era necessario essere uno psicologo della Legione per
vedere che Ghen li stava spaventando. Notai che in quel capannello di
spettatori non c'era nessuno degli altri veterani. Erano tutti novellini
come me.

«Forse non sarà così male» commentò Switch, senza convinzione nella voce
acuta. Mi colpì quanto fosse giovane, più di me, abbastanza perché la
sua barba crescesse a chiazze sulla mascella stretta. Io stesso ero
giovane, anche se la mia età era un mistero. Avevo ventun'anni?
Ventidue? Avevo visto sempre e solo calendari locali e non ero certo di
quale fosse la data imperiale. Non sapevo per quanti anni fossi rimasto
congelato sulla nave di Demetri.

Ghen inarcò le sopracciglia rasate con aria incredula. «Forse non sarà
così male?» Pronunciò quelle parole con sarcasmo, poi le ripeté a voce
più alta. «Finirai a culo in aria per il nemico nell'istante in cui ne
avrai la possibilità.» Quel grosso bue oltrepassò a forza Siran con un
braccio proteso e afferrò Switch per la gola. «Io non voglio essere
mietuto, ragazzo. E tu?»

Ecco. Quella era la mia battuta di entrata. Risi, non con spacconeria e
neppure in modo da attirare deliberatamente l'attenzione su di me.
Confesso di aver imparato quel trucco da mio padre. Quel suono sommesso
aleggiò in un breve momento di silenzio, vi si impigliò e indusse Ghen a
girarsi. I novellini più vicini sgombrarono lo spazio intorno a me e io
scossi la testa per disperdere la risata che vi era racchiusa. Sullo
spiazzo era sceso un letale silenzio, e i soli suoni provenivano dagli
altri gruppi che si addestravano alla sua estremità opposta, levando un
clangore di acciaio all'ombra delle colonne.

«Ho detto qualcosa di buffo, Vostra radiosità?»

Risi ancora, questa volta un suono più breve, e allargai le mani. «Qui
sei il solo a parlare di essere mietuto, Ghen. Se non sapessi che non è
così, direi che ti sentivi solo» ribattei. Siran scoppiò in una breve
risata e un paio di novellini ridacchiarono nervosamente.

Il grosso uomo spinse via Switch, mandandolo a rotolare nella polvere e
nell'erba secca, poi si girò di scatto verso di me ed estrasse la spada
con un suono stridente. «Vuoi davvero fare questo, ragazzo?»

«Cosa?» Quell'uomo era talmente prevedibile che sembrava di leggere il
copione di una commedia. «Proprio qui? Davanti a tutti? Senza neppure
cenare prima?» Questa volta le risate furono più forti, abbastanza da
far infuriare Ghen, da renderlo stupido. «Non sei certo per me,
ragazzo...»

Vidi arrivare il colpo da parsec di distanza, un selvaggio fendente che
avrebbe tagliato a metà la mia testa nuda se fosse andato a segno. Lo
schivai con facilità e mi rialzai di scatto, afferrandolo per il polso e
la spalla troppo estesa. La mia spada, riposta nel fodero, mi urtò la
coscia mentre torcevo il braccio di Ghen verso il basso e lontano,
facendogli perdere l'equilibrio. Barcollò, e sfruttai quell'apertura per
estrarre la spada. «Troppo lento, amico.»

Quando si girò di scatto, Ghen mi trovò ad aspettarlo con la spada
posizionata in una guardia alta, inclinata davanti alla faccia nell'aria
calda fra di noi. Il bianco degli occhi gli spiccava sul volto scuro
insieme ai denti snudati. Non parlò, si limitò ad attaccarmi ancora.
Notai che somigliava molto all'affrontare un Crispin infuriato: tutto
muscoli e niente buonsenso. Combatté come un uomo abituato a vincere e a
farlo in fretta. Gli anni vissuti su Emesh mi avevano indurito, ma Ghen
ci era nato, la sua carne era stata modellata dalla sua mano invisibile
ed era strutturato come un antico carro armato, squadrato, tozzo e
compatto.

Quella massa solida mi cadde addosso e per un momento quasi mi ripiegai
sotto il suo impatto, poi scivolai da un lato e calai con forza la
spada, raggiungendolo allo stomaco. Ghen annaspò, ma la sua corazza
d'acciaio intercettò in buona parte la forza del colpo e comunque la
lama era smussata, una spada da addestramento. Mi spostai alle sue
spalle e gli piantai un tallone contro il posteriore, scaraventandolo
nella polvere con un calcio. Tutto troppo facile.

L'istinto mi avrebbe indotto a parlare solo a Ghen, ai miei piedi, ma
ero spinto da necessità più grandi, quindi tenni minacciosamente la lama
vicino alla sua faccia mentre lui giaceva passivo. Invece di rivolgermi
all'uomo sconfitto -- era solo un sintomo -- sollevai la voce facendo
appello a centinaia di ore di addestramento all'oratoria per opera di
Gibson. «Dividereste la nostra forza quando più ne abbiamo bisogno?»
Come mio padre, feci scorrere lo sguardo sulla folla, pienamente
consapevole di quello che stavo facendo e di chi stavo emulando, cosa
che mi rendeva il cuore pesante come piombo. Io però non ero mio padre,
non volevo che loro mi temessero. Il volto delle reclute che mi
guardavano -- uomini e donne nuovi quanto me a tutto questo -- era
improntato alla stessa espressione tesa di paura che avevo osservato
pochi momenti prima. «La gente si aspetta di vederci morire tutti. Tu!»
Indicai una giovane donna con i capelli chiari, con la pelle arrossata e
spelata a causa del sole che quasi oscurava la parola
\foreignlanguage{italian}{ladra} \emph{} tatuata sulla sua fronte. «E
tu! E tu! E io.» Mi colpii il petto con il pugno nell'imitazione del
saluto. «Io ho intenzione di deluderla.»

«Parole coraggiose per un novellino» gridò Banks, un uomo con le
mascelle prominenti e la faccia dura come il cuoio che si teneva in
fondo alla folla. Le sue parole furono accolte da grida di assenso dei
gladiatori più stagionati, con la sola eccezione di Siran, che mi
guardava con espressione indecifrabile. «Non hai la solennità necessaria
per il comando, figliolo.»

«Solennità?» Sorrisi. «Una parola eccessiva.» Però mi ero aspettato
quella risposta e avevo perfino intuito che sarebbe venuta da Banks.
L'avrebbe data Ghen se l'imbarazzo e la rabbia non lo tenessero a
ribollire in silenzio ai miei piedi. «Io non ho parole altisonanti per
tutti voi, solo parole disperate» continuai. «Non voglio morire, e voi?»
Feci una pausa infinitesimale, sperando che qualcuno rispondesse senza
però contarci molto. Nessuno lo fece. «Nessuno di noi sarebbe qui se
avessimo avuto un'altra scelta.»

«Quel tuo naso grazioso dice una cosa diversa» affermò la donna pallida
con la faccia che si spelava, e la sua voce suonò più matura di quanto
mi aspettassi.

Questo mi sorprese e per un momento rimasi fermo con la lama ancora
premuta contro una tempia di Ghen, mordicchiandomi la lingua. «Questo
non vuol dire che sia qui per mia scelta. Abbiamo tutti le nostre
ragioni, e credo che Kiri sia la sola che non avrebbe motivo di
trovarcisi.» Agitai la mano libera in direzione della popolana, il cui
volto scuro era segnato dalla preoccupazione. Quanti anni aveva in
effetti? Quaranta standard? Cinquanta? Era così giovane. Mia madre ne
aveva quasi trecento e sembrava avere quasi meno della metà dell'età di
questa donna. «Ma perfino lei è qui per una ragione. Lo siamo tutti.»

«Chiudi la bocca, ragazzo, o te la chiuderò io!» esclamò Pallino, un
vecchio veterano con la corporatura solida di un soldato di carriera e
una benda nera su un occhio.

Piantando un piede sulla schiena di Ghen sollevai entrambe le braccia
con la spada ancora in mano. «Se ci vuoi provare sei il benvenuto,
signore.» Rimossi il piede da Ghen e avanzai un poco verso la folla.

Quelle parole generarono un'onda di sussurri fra i presenti. Un uomo
minuto con una faccia da topo si protese verso un compagno, mormorando:
«Bastardo.» Ignorai i sussurri e fissai il veterano guercio con occhi
socchiusi.

Lo stridio lontano di un velivolo diretto verso l'ombra massiccia dello
ziggurat pervase l'aria. Pallino non si fece avanti per sfidarmi e
quando il rumore del velivolo fu cessato mi chinai e porsi la mano a
Ghen. Con un sussurro a voce volutamente piuttosto alta diretto all'uomo
che avevo sconfitto, dissi: «Prendi, amico.»

Ghen rotolò su un fianco e vide la mano protesa. Parve ruminare sui suoi
pensieri, letteralmente masticarli come una cartilagine, poi accettò la
mano e mi permise di aiutarlo ad alzarsi. «Sei fottutamente veloce,
Had.»

Avrei dovuto fare il librettista, il commediografo come mia madre: le
sue emozioni si stavano sviluppando \emph{esattamente} come avevo
immaginato e sperato. Avrei dovuto fare l'attore, fare... qualcos'altro,
qualsiasi cosa che non fosse quello che ero o ero diventato. Un soldato?
Un mago? Un esploratore come Simeon il Rosso?

Per un minuto Ghen parve non desiderare niente più che colpirmi, poi
quell'emozione scomparve, precipitando sotto la superficie o dietro una
nuvola mentre un folle sorriso appariva sui suoi rozzi lineamenti. Poi
rise, non come io avevo fatto pochi momenti prima, ma con una risata
lunga, stentorea e limpida. La minaccia di Pallino venne dimenticata,
cancellata da quel suono squillante. «Tornate tutti ai lavoro!» muggì,
assestandomi una pacca sulla spalla. «Il ragazzo ha ragione. Sono quei
bastardi dall'armatura elegante che vogliamo uccidere, non farci fuori
fra di noi!» Dai novellini si levò un applauso poco sentito che però
sopraffece i veterani silenziosi. Ghen si allontanò con passo tranquillo
e Siran andò con lui, avviando una conversazione sommessa.

Gemendo, mi chinai a raccogliere l'elmo prendendolo per l'ampia flangia
che proteggeva il collo. Era un arnese dall'aria vecchia, più che
medievale, fatto di acciaio battuto e modellato nello stesso modo degli
elmi dei legionari imperiali, con la differenza che i loro avevano una
visiera bianca mentre il mio era aperto, con le protezioni per le guance
fissate su cardini alle tempie. Questo non era forgiato, naturalmente,
ma era stato stampato nell'armeria del colosseo e l'acciaio era solcato
da cavi di carbonio -- più sottili del capello più sottile -- per
rinforzarlo. Me lo misi sulla testa e urtai leggermente Switch su una
gamba con la spada. «Vieni, dobbiamo sistemare quel tuo lavoro di
piedi.»

«Mi serve pratica» gemette Switch, senza sollevare lo sguardo dagli
stivali.

«Sì» convenni, lanciando un'occhiata a Ghen e a Siran, che stavano
parlando con tre reclute inesperte. Siran aveva trovato una lunga lancia
-- una finta, senza il contenitore per il bruciatore al plasma -- e vi
appoggiava contro il suo peso. Vista da sinistra, senza la devastazione
sfregiata della narice destra, nelle linee del suo volto, nella forma
degli zigomi alti e negli occhi socchiusi c'era qualcosa che mi faceva
pensare al sangue reale. Non somigliava per niente a Cat. Pensare a Cat
incupì il mio umore e mi mordicchiai distrattamente la lingua. «Abbiamo
tutti bisogno di fare pratica, Switch, è per questo che siamo qui.
Abbiamo una settimana di tempo.»

Lui scosse il capo con i capelli rossi che gli si allargavano a
ventaglio intorno alla faccia. Aveva dimenticato l'elmo negli
alloggiamenti quando ci eravamo equipaggiati per venire a esercitarci
nel quadrato. «Non è un tempo sufficiente, Had. Non lo è.»

Serrando le labbra in una silenziosa espressione conciliante gli
assestai una pacca sulla spalla e mi allontanai per poi girarmi di
scatto in posizione di guardia bassa: ginocchia piegate, schiena eretta.
Per la Terra e l'imperatore, era bello impugnare di nuovo una spada! Non
avevo creduto che mi sarebbe mancata. Il sangue mi martellava ancora
nelle vene, scandito come il battito di un tamburo di una parata
militare. Per quanto fosse stato breve, il mio scontro con Ghen era
stato un vero combattimento, non una rissa in un vicolo. Repressi un
sorriso e mentii. «Certo che è sufficiente. Abbiamo una settimana,
Switch, e puoi fare tantissime cose in una settimana. Coraggio, piede in
avanti e tieni la schiena eretta... è per questo che ti allunghi troppo.
Ti fa inclinare, vedi?» Glielo mostrai, eseguendo un'esagerata
imitazione da commedia del suo errore, sbilanciandomi al punto da cadere
su un ginocchio, decisamente troppo vulnerabile. Fatto questo, ripetei
la mossa nel modo corretto, attento a tenere la schiena diritta. «Ora
tu. Fammi vedere.»


