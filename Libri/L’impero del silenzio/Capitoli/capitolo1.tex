\mainmatter

\chapter{Hadrian}

Luce.

La luce di quel sole assassinato mi brucia ancora. Attraverso le
palpebre la vedo emergere fiammeggiante dalla storia di quel giorno di
sangue, suggerendo fuochi indescrivibili. È come qualcosa di sacro, come
se fosse la luce stessa del paradiso di Dio che ha bruciato il mondo e
miliardi di vite con esso. Porto sempre con me quella luce, impressa a
fuoco in fondo alla mente. Non accampo giustificazioni o dinieghi, o
scuse per quello che ho fatto. So cosa sono.

Gli scienziati possono anche cominciare dal principio, da quando i
nostri remoti antenati sono usciti con le unghie e con i denti dal
sistema della Vecchia Terra sulle loro navi che perdevano aria, da quei
pellegrini che hanno compiuto i loro viaggi fino a mondi nuovi e
viventi, ma no. Fare questo richiederebbe più volumi e inchiostro di
quanto chi mi ospita me ne abbia lasciato a disposizione e perfino io,
che ho più tempo di chiunque altro, non ne avrei a sufficienza per
questo.

Devo allora fare una cronaca della guerra? Cominciare con gli alieni
Cielcin che sono emersi ululando dallo spazio su astronavi simili a
castelli di ghiaccio? Potete trovare storie sulla guerra, leggere il
conteggio dei morti. Le statistiche. Nessun contesto può farvi capire il
suo costo. Città rase al suolo, pianeti bruciati, innumerevoli miliardi
della nostra gente strappati ai loro mondi per servire come carne e come
schiavi a quei mostri, i Pallidi. Famiglie antiche come imperi si sono
estinte nella luce e nel fuoco. Le narrazioni sono {innumerevoli}, e non
sono abbastanza. L'Impero ha la sua versione ufficiale che finisce con
la mia esecuzione capitale, con Hadrian Marlowe impiccato al cospetto di
tutti i mondi.

Non dubito che questo tomo non farà altro che raccogliere polvere
nell'archivio in cui l'ho lasciato, un manoscritto fra i miliardi
raccolti a Colchide, dimenticato, e forse è meglio così. I mondi hanno
avuto tiranni, omicidi e genocidi più che a sufficienza.

Forse però tu continuerai la lettura, tentato dal pensiero di leggere
l'opera di un così grande mostro come quello creato a mia immagine. Non
lascerai che venga dimenticato perché vuoi sapere com'è stato trovarsi a
bordo di quella nave impossibile e strappare il cuore a una stella. Vuoi
provare il calore di due civiltà che bruciano e incontrare il drago, il
demone che porta il nome che mio padre mi ha dato.

Quindi ignoriamo la storia, aggiriamo la politica e il calpestio di
imperi in marcia. Dimentichiamo gli inizi della razza umana fra il fuoco
e la cenere della Vecchia Terra e quindi ignoriamo anche l'apparire
gelido dei Cielcin dall'oscurità. Quelle storie sono documentate altrove
in tutte le lingue dell'umanità e dei suoi sudditi. Passiamo perciò al
solo inizio a cui ho diritto: il mio.

Sono nato come figlio maggiore ed erede di Alistair Marlowe, arconte
della prefettura di Meidua, il Macellaio di Linon e il Signore del
Riposo del Diavolo. Quel palazzo di pietra scura non era luogo adatto a
un bambino, ma era comunque la mia casa, fra i logoteti e i peltasti in
armatura che servivano mio padre. Lui però non aveva mai voluto un
figlio, voleva un erede, qualcuno che ereditasse la sua fetta di Impero
e portasse avanti il retaggio della nostra famiglia. Mi ha chiamato
Hadrian, un nome antico e senza significato tranne che per il ricordo di
quegli uomini che lo hanno portato prima di me. Il nome di un
imperatore, degno di regnare e di essere seguito.

I nomi sono una cosa pericolosa, una sorta di maledizione, perché ci
definiscono in modo che si possa essere alla loro altezza o ci danno
qualcosa da cui fuggire. Io ho vissuto a lungo, più a lungo di quanto
possano garantire le terapie genetiche che le grandi case dei nobili
riescono a trovare, e ho portato molti nomi. Durante la guerra ero
Hadrian il Semimortale e Hadrian l'Immortale. Dopo la guerra sono
diventato il Divoratore di Soli. Per la povera gente di Borosevo ero un
mirmidone di nome Had, per gli Jaddiani ero Al Neroblis. Per i Cielcin
ero Oimn Belu e anche cose peggiori. Sono stato molte cose: soldato e
servitore, capitano e prigioniero, magio e studioso, e poco più che uno
schiavo.

Ma prima di tutto questo sono stato un figlio.

\begin{figure}
	\centering
	\def\svgwidth{\columnwidth}
	\scalebox{0.2}{\input{divisore.pdf_tex}}
\end{figure}

Mia madre ha fatto tardi alla mia nascita, ed entrambi i miei genitori
hanno guardato dalla piattaforma sovrastante la sala operatoria mentre
venivo travasato dalla vasca. Dicono che abbia urlato mentre gli
scoliasti mi facevano nascere, e che avessi già tutti i denti. È così
che la nobiltà è sempre nata, senza oberare la madre e sotto l'occhio
attento dell'Alto Collegio Imperiale, per garantire che le nostre
deviazioni genetiche non si trasformassero in difetti e si cagliassero
nel nostro sangue. Inoltre, un parto tradizionale avrebbe richiesto che
i miei genitori condividessero il letto, cosa che nessuno dei due era
incline a fare. Come molti nobili, si erano sposati per una necessità
politica.

In seguito ho appreso che mia madre preferiva la compagnia delle donne a
quella di mio padre e trascorreva di rado del tempo nella tenuta di
famiglia, assistendo mio padre soltanto durante le funzioni formali. Per
contro, mio padre preferiva il suo lavoro.

Lord Alistair Marlowe non era il genere di uomo che prestasse attenzione
ai suoi vizi. In realtà, non era il genere di uomo che \emph{avesse}
vizi. Era posseduto dalla sua carica e dal buon nome della nostra
Casata.

Quando sono nato, la Crociata infuriava ormai da trecento anni, fin
dalla prima battaglia contro i Cielcin a Cressgard, ma imperversava
lontano, a circa ventimila anni luce di Impero di distanza e nello
spazio aperto, dove il Velo si apriva sul Braccio di Norma. Mentre mio
padre faceva del suo meglio per farmi capire la gravità della
situazione, lo stato di cose a casa era decisamente tranquillo, a parte
i soldati che ogni decennio la Legione imperiale arruolava fra i plebei.
Eravamo a decenni di distanza dal fronte anche per le navi più veloci, e
sebbene i Cielcin fossero la più grande minaccia che la nostra specie
avesse affrontato fin dalla morte della Vecchia Terra, le cose non erano
poi così terribili.

Come ci si può aspettare da genitori come i miei, sono stato messo quasi
immediatamente nelle mani dei servitori di mio padre. Senza dubbio lui è
tornato al suo lavoro entro un'ora dalla mia nascita, avendo sprecato
tutto il tempo che quel giorno si poteva permettere di perdere per una
fonte di distrazione seccante come un figlio. Mia madre invece è tornata
a casa di sua madre, per passare del tempo con i suoi fratelli e i suoi
amanti. Come ho detto, lei non era coinvolta nella cupa attività di
famiglia.

Quell'attività era l'uranio. Le terre di mio padre si stendevano sopra i
più ricchi depositi del settore e la nostra famiglia ne controllava
l'estrazione da generazioni. Il denaro che mio padre incassava tramite
il Consorzio Wong-Hopper e l'Unione dei liberi mercanti lo rendeva
l'uomo più ricco di Delos, ancora più ricco della viceregina, mia nonna.

Avevo quattro anni quando è nato Crispin, e immediatamente il mio
fratellino ha cominciato a dimostrare di essere l'erede ideale, il che
significa che obbediva a mio padre, anche se a nessun altro. Il
castellano di mio padre, sir Felix Martyn, mi aveva insegnato a
combattere con la spada, la cintura-scudo e la pistola. Mi ha insegnato
ad accendere una lancia e ha addestrato il mio corpo per purificarlo
dall'indolenza. Da Helene, il ciambellano del castello, ho appreso le
complessità degli inchini e delle strette di mano, e come rivolgersi
alle persone in modo formale. Da Abiatha, il vecchio cantore che si
prendeva cura della torre campanaria e dell'altare nel luogo sacro della
Cappellania, ho imparato non solo le preghiere ma anche lo scetticismo,
e che perfino i preti hanno dei dubbi. Dai suoi superiori, i priori
della Sacra Cappellania Terrestre, ho imparato a vedere quei dubbi per
quell'eresia che erano. E naturalmente c'era mia madre, che mi
raccontava storie di Simeon il Rosso, del Cid Artù e di Kasia Soulier,
storie di Kharn Sagara. Ridi, ma nelle storie c'è una magia che non può
essere ignorata.

Tuttavia, è stato tor Gibson che ha fatto di me l'uomo che sono, che mi
ha insegnato la mia prima lezione. `Il sapere è la madre degli stolti'
ha detto. `Ricorda che la parte più grande della saggezza consiste nel
riconoscere la tua ignoranza.' Diceva sempre cose del genere. Mi ha
insegnato la retorica, l'aritmetica e la storia, mi ha istruito in
biologia, meccanica, astrofisica e filosofia. È stato lui a insegnarmi
le lingue e l'amore per le parole. A dieci anni parlavo il mandar bene
quanto qualsiasi bambino delle corporazioni dell'interspazio e potevo
leggere la poesia di fuoco di Jadd come un vero accolito della loro
fede. Cosa più importante di tutte, però, è stato lui a insegnarmi chi
fossero i Cielcin, quegli alieni razziatori e omicidi che stavano
rosicchiando i confini della civiltà. Lui mi ha insegnato a essere
affascinato dagli xenobiti e dalla loro cultura.

Spero solo che i libri di storia non lo condannino per questo.

\begin{figure}
	\centering
	\def\svgwidth{\columnwidth}
	\scalebox{0.2}{\input{divisore.pdf_tex}}
\end{figure}

«Sembri a disagio,» osservò tor Gibson, con voce simile a un vento arido
nell'aria immota della sala di addestramento.

Muovendomi lentamente, mi districai dalla complessa estensione in cui mi
ero ripiegato e fluii nella posizione successiva, stirandomi la schiena.
«Sir Felix e Crispin presto saranno qui e voglio essere pronto.»
Attraverso la piccola finestra ad arco incastonata in alto nelle mura di
pietra riuscivo a stento a cogliere i richiami degli uccelli marini, il
cui rumore era soffocato dagli scudi della casa.

Con il volto impassibile come la pietra, il vecchio scoliasta mi aggirò
fino a uscire dal mio campo visivo, con i piedi calzati nelle pantofole
che scivolavano sul pavimento a mosaico. Per quanto incurvato dagli
anni, il vecchio tutore era comunque più alto di me, con il volto
squadrato che adesso sorrideva sotto la massa di capelli bianchi e le
basette che lo facevano somigliare ai leoni che la viceregina teneva nel
suo zoo. «Non vedi l'ora di far finire il piccolo signore con il culo
per terra, eh?»

«Quale culo?» ribattei sorridendo mentre mi chinavo fino a toccarmi le
dita dei piedi e la voce mi si incrinava un poco per lo sforzo. «Quello
fra le sue orecchie?»

Il sottile sorriso di Gibson scomparve. «Faresti bene a non parlare così
di tuo fratello.»

Scrollai le spalle, assestando una delle strisce sottili che mantenevano
aderente alla camicia il mio giustacuore da duello. Lasciando Gibson
dove si trovava attraversai scalzo la sala fino alla rastrelliera dove
le armi da addestramento attendevano in mostra vicino alla piattaforma
di scherma, un disco di legno leggermente sopraelevato del diametro di
circa venti piedi, destinato alle esercitazioni nel duello. «Abbiamo una
lezione questa mattina, Gibson? Credevo che fosse per questo
pomeriggio.»

«Cosa?» Si batté un colpetto sulla testa e si fece un po' più vicino
mentre io dovevo ricordare a me stesso che anche se si muoveva bene
Gibson non era più giovane. Non era più stato giovane già quando
l'ordine lo aveva incaricato di fare da tutore a mio padre, che si
avvicinava lui stesso ai trecento anni standard. Si ripiegò una mano a
coppa intorno all'orecchio. «Cosa hai detto?»

Girandomi parlai più chiaramente, raddrizzando la schiena come mi era
stato insegnato a fare per proiettare meglio. Con il tempo dovevo
diventare l'arconte del castello, e l'arte del parlare era l'arma più
cara a un palatino. «Credevo che la lezione fosse più tardi.»

Non poteva averlo dimenticato. Gibson non dimenticava niente, il che
sarebbe stata una qualità straordinaria se non fosse stato il requisito
di base per diventare ciò che lui era: uno scoliasta. La sua mente era
stata addestrata per sostituire quelle macchine demoniache proibite
dalla legge più sacra della Cappellania, quindi non poteva permettersi
di dimenticare. «È così, Hadrian. Più tardi, sì.» Tossì contro una
manica verde e adocchiò il drone con videocamera che se ne stava
annidato vicino al soffitto a volta. «Speravo di poter scambiare qualche
parola in privato.»

La spada non affilata mi scivolò leggermente dalla mano. «Ora?»

«Prima che arrivino tuo fratello e il castellano, sì.»

Mi girai per rimettere la spada al suo posto fra gli stocchi e le
sciabole, e lanciai io stesso un'occhiata al drone, sapendo benissimo
che le sue lenti erano puntate su di me. Dopotutto, ero il figlio
maggiore dell'arconte e quindi ero soggetto allo stesso livello di
protezione -- e di esame -- di mio padre. Al Riposo del Diavolo c'erano
posti dove era possibile conversare davvero in privato, ma la sala di
addestramento non era uno di essi. «Qui?»

«Nel chiostro.» Momentaneamente distratto, Gibson guardò i miei piedi
nudi. «Niente scarpe?»

I miei non erano i piedi di un nobile viziato, sembravano più quelli di
un qualche servo della gleba, con strati di calli tanto spessi che avevo
rivestito di nastro adesivo le articolazioni delle dita più grosse per
evitare che la pelle si lacerasse. «Sir Felix dice che i piedi nudi sono
la cosa migliore per l'addestramento.»

«Ma davvero?»

«Dice che così è meno probabile slogarsi una caviglia.» Mi interruppi,
fin troppo consapevole dello scorrere del tempo. «La nostra
conversazione... non può aspettare? Dovrebbero arrivare presto.»

«Se proprio deve.» Gibson annuì, lisciando con le mani dalle corte dita
il davanti della veste e la fusciacca color bronzo. Al confronto, mi
sentivo trasandato nella tenuta da duello anche se in realtà il suo
abbigliamento era di semplice cotone, ma ben tinto in quella tonalità
che è più verde della vita stessa.

Il vecchio scoliasta era sul punto di aggiungere altro quando la porta a
due battenti della sala di addestramento si spalancò rumorosamente e
apparve mio fratello, che sfoggiava il suo sorriso da lupo. Crispin era
tutto ciò che io non ero, alto mentre io ero basso, robusto laddove io
ero sottile come una canna, con il volto squadrato mentre il mio era
appuntito. Nonostante tutto questo, la nostra parentela era innegabile
perché avevamo gli stessi capelli neri come l'inchiostro propri dei
Marlowe, lo stesso naso aquilino e sopracciglia arcuate sopra occhi
viola. Eravamo chiaramente il prodotto della stessa costellazione
genetica, con i nostri genomi che erano stati alterati nello stesso modo
per entrare nello stesso stampo. Le Casate palatine -- maggiori e minori
-- arrivavano a misure addirittura bizzarre pur di forgiare un'immagine
tale che gli eruditi fossero in grado di distinguere un Casato sulla
base dei marcatori genetici presenti sulla faccia e sul corpo con la
stessa facilità degli stemmi portati sull'uniforme o dipinti sulle
bandiere.

Il castellano dai lineamenti duri, sir Felix Martyn, procedeva al
seguito di Crispin, vestito in tenuta di cuoio da duello con le maniche
arrotolate fin sopra il gomito. Fu il primo a parlare, sollevando una
mano guantata. «Ohi! Già qui?»

Aggirai Gibson per andare incontro ai due. «Stavo solo facendo un po' di
stretching, signore.»

Il castellano inclinò la testa, grattandosi l'aggrovigliata barba
brizzolata. «Benissimo.» Notò Gibson per la prima volta. «Tor Gibson! È
strano vederti fuori dal chiostro a quest'ora.»

«Stavo cercando Hadrian.»

«Hai bisogno di lui?» Il cavaliere agganciò i pollici nella cintura.
«Adesso abbiamo lezione.»

Gibson scosse rapidamente il capo e rivolse al castellano un accenno di
inchino. «Può aspettare.» E se ne andò, uscendo in silenzio dalla sala.

Le porte sbatterono dietro di lui, scatenando un rimbombo attutito nella
sala a volta. Per un momento, Crispin fece una comica imitazione
dell'andatura barcollante e a schiena curva di Gibson, e quando lo
trapassai con uno sguardo rovente ebbe la buona grazia di apparire
mortificato mentre si sfregava i palmi sul velo nero di ricrescita dei
capelli che gli copriva il cuoio capelluto.

«Gli scudi sono carichi al massimo?» domandò Felix, battendo le mani con
un opaco schiocco di cuoio. «Benissimo.»

Nelle leggende, all'eroe viene quasi sempre insegnato a combattere da
qualche eremita fuori di testa, un mistico che incarica il suo allievo
di inseguire gatti, pulire veicoli e scrivere poesia. Nel Jadd si dice
che i maestri di spade -- i maeskoloi -- fanno tutte queste cose e
potrebbero continuare per anni prima anche solo di toccare una spada.
Non io. Sotto la tutela di Felix, la mia istruzione era un rigore di
incessanti esercitazioni e passavo molte ore al giorno affidato alle sue
cure, imparando a difendermi. Niente misticismo, solo pratica, lunghe
sessioni tediose, finché movimenti come affondi o parate non diventavano
facili da eseguire come l'atto di respirare, perché fra la nobiltà
palatina -- tanto uomini quanto donne -- dell'Impero Solano l'abilità
con le armi è considerata una delle virtù principali, non solo perché
chiunque di noi potrebbe aspirare al cavalierato o a prestare servizio
nelle Legioni, ma anche perché duellare serve come valvola di sfogo per
pressioni e pregiudizi che potrebbero altrimenti trasformarsi in
vendette. Così, ci si aspetta che qualsiasi figlio di qualsiasi Casata
possa prima o poi prendere le armi in difesa del suo onore o di quello
della sua Casata.

«Sono ancora in debito con te per l'ultima volta» disse Crispin, quando
finimmo con le esercitazioni e ci trovammo uno di fronte all'altro sul
terreno di scherma. Le sue spesse labbra si contorsero in un sorriso che
gli diede più che mai l'aspetto di quello strumento smussato che era.

Ricambiai con un sorriso uguale, anche se speravo che sulla mia faccia
l'effetto esprimesse una minore spacconeria. «Prima mi devi colpire»
ribattei, sollevando di scatto la punta della spada in una posizione di
guardia, aspettando il segnale di sir Felix. Fuori, da qualche parte,
sentii il sibilo di un velivolo che sorvolava a bassa quota il castello,
scuotendo rumorosamente l'alluminio trasparente delle finestre e
facendomi rizzare i capelli. Posai la mano sul gancio della cintura che
avrebbe attivato la cortina di energia dello scudo e Crispin fece
altrettanto, appoggiando di piatto la spada sulla spalla.

«Crispin, cosa stai facendo?» La voce del castellano fendette quel
nostro momento di confronto come una frusta.

«Cosa?»

Come ogni buon insegnante, sir Felix attese che Crispin si rendesse
conto del suo errore. Quando questo non successe lo colpì al braccio con
la sua spada da addestramento. Crispin lanciò uno strillo e lo fissò con
occhi roventi. «Se ti appoggiassi l'altamateria sulla spalla, in quel
modo, ti staccheresti un braccio. La lama deve essere \emph{lontana} dal
corpo, ragazzo, quante volte devo ripetertelo?» Imbarazzato, aggiustai
la mia guardia.

«Con l'altamateria non me lo dimenticherei» si giustificò Crispin, ed
era vero. Non era uno stupido, gli mancava solo quella serietà del
carattere che preannuncia la grandezza.

«Adesso ascoltatemi, tutti e due» scattò Felix, troncando qualsiasi
ulteriore argomentazione da parte di Crispin. «Vostro padre mi
consegnerà ai cathar se non faccio di entrambi due combattenti di prima
categoria. Siede dannatamente decenti, ma essere decenti non vi servirà
a niente in un vero scontro. Crispin, devi accorciare le mosse, perché
ogni volta ti lasci esposto a un contrattacco. E tu!» Puntò la spada da
addestramento contro di me. «Le tue mosse sono buone, Hadrian, ma devi
impegnarti. Lasci ai tuoi avversari troppo tempo per riprendersi.»

Accettai la critica senza commenti.

«In guardia!» esclamò Felix, tenendo la sua lama di piatto in mezzo a
noi. «Scudi!» Entrambi prememmo il gancio per attivare lo scudo. Le
cortine di energia non cambiavano niente per quanto riguardava la
rapidità umana nell'usare la spada o nei corpo a corpo, ma era buona
pratica abituarsi a esse e alla lieve distorsione della luce attraverso
la loro membrana permeabile. La barriera del campo Royse avrebbe deviato
con poca difficoltà impatti ad alta velocità. Poteva fermare le
pallottole, arrestare scariche di plasma e dissipare le ondate di
energia dei disintegratori neurali, ma non poteva fare niente contro una
lama. Felix calò la spada da quel boia che a volte era, e la punta
smussata colpì il pavimento. «Cominciate!»

Crispin si mosse con impeto e con la lama tratta all'indietro per
infondere nel colpo tutta la potenza del gomito e della spalla. Vidi il
suo attacco arrivare da anni luce di distanza e lo schivai, lasciando
che mi sibilasse sopra la testa, poi ruotai su me stesso e mi rimisi in
guardia alla destra di Crispin, con un'angolazione perfetta per
colpirgli la schiena e la spalla esposte. Invece gli assestai una
spinta.

«Fermi!» ringhiò Felix. «Hadrian, avevi un'opportunità perfetta!»

Continuammo così per quelle che parvero ore, con sir Felix che ci
inveiva contro a intervalli. Crispin combatteva come un turbine,
colpendo selvaggiamente dall'alto e dai lati, consapevole della sua
portata maggiore, della sua potenza e forza, ma io ero sempre più veloce
e ogni volta intercettavo la sua lama con la mia, barcollando
all'indietro verso il bordo del terreno di scherma. Sono sempre stato
grato che il mio primo compagno di esercitazioni di spada sia stato
Crispin, perché combatteva come un tram per il trasporto merci, come una
di quelle massicce mietitrebbiatrici automatiche le cui braccia mietono
interi campi. La sua maggiore altezza e forza mi hanno preparato a
combattere contro i Cielcin, i più bassi dei quali sono alti quasi due
metri

Crispin cercò di intrappolare la mia lama per abbassarla con la forza e
darsi il tempo di colpirmi alle costole. Io ero già caduto una volta
vittima di quella mossa e potevo sentire il livido che mi si stava
formando sotto il giustacuore, quindi strisciai i piedi sul legno
lasciando che Crispin venisse avanti. Tutto l'impeto che aveva impresso
alla lama lo fece scivolare e lo colpii sopra l'orecchio con una mano
aperta, facendo seguire un colpo di spada quando lui barcollò. Felix
batté le mani, facendoci fermare. «Molto bene. Sei un po' meno
concentrato del solito, Hadrian, ma lo hai centrato.»

«Due volte» precisò Crispin, massaggiandosi l'orecchio mentre si
rimetteva in piedi. «Dannazione, se ha fatto male.» Gli porsi la mano ma
lui la spinse via, gemendo nel rialzarsi.

Felix ci diede un momento di pausa, poi ci fece riprendere posizione.
«Cominciate!» La sua lama colpì il pavimento di legno e il duello
riprese. Girai in cerchio sulla destra mentre Crispin si lanciava alla
carica, finendo per sorpassarmi sulla destra e per impattare contro la
prima parata con cui bloccai il suo attacco mentre mi scivolava accanto.
Serrando la mascella ruotai su me stesso -- troppo tardi -- per colpirlo
alla schiena. Sentii Felix espellere il fiato attraverso i denti.

Crispin ruotò selvaggiamente, facendo descrivere alla spada un ampio
arco per creare spazio fra noi due. Sapeva che l'attacco stava arrivando
e si allontanò con un balzo mentre io eseguivo un affondo tenendo bassa
la spada. Lui la spinse in giù e tentò una risposta alla spalla destra.
Ritrovando l'assetto ruotai il polso e parai il colpo, intercettando la
sua spada con la mia. Lui mantenne la presa ma si torse, esponendo la
schiena.

«Crispin!» Il castellano era violaceo per la frustrazione. «Cosa diavolo
stai facendo?

La violenza della sua voce indusse Crispin a bloccarsi e io lo colpii
allo stomaco, con forza. Mio fratello mi fissò con occhi roventi da
sotto le spesse sopracciglia mentre il castellano-cavaliere saliva sulla
pedana e lo fissava con occhi cupi. «Quale parte di `accorcia le tue
mosse' non capisci?»

«Mi hai distratto!» La voce di Crispin si fece acuta. «Mi stavo
disimpegnando.»

«Avevi una spada!» Sir Felix agitò le mani aperte davanti a sé, a palmo
all'insù. «Avevi l'altra mano! Riprendete!»

Crispin balzò in avanti dal nastro della posizione di partenza con la
spada tenuta alta con entrambe le mani. Io ruotai sulla destra e colpii
con forza sulla sinistra per bloccare il suo bestiale fendente, poi
scattai in avanti e mirai alla sua schiena, ma Crispin si girò e
intercettò la mia risposta con una controparata. Aveva gli occhi
fiammeggianti, i denti snudati. Mi spinse la spada di lato e mi investì
con una spallata, accoccolandosi per scaraventarmi in alto e fuori dalla
rotonda. Finii a terra, senza fiato, e Crispin incombette su di me, sei
piedi di muscoli furenti vestiti tutti di nero.

«Hai avuto fortuna, fratello.» Le sue spesse labbra si contrassero in un
sogghigno mentre mi assestava un calcio nelle costole e io sussultavo,
annaspando per respirare. Lo ignorai mentre continuava a parlare,
dicendo che se avessi combattuto lealmente non lo avrei mai colpito. Se
sir Felix avanzò qualche commento non lo notai perché Crispin era
vicino, torreggiava su di me. Finì di parlare e si girò per andarsene,
ma gli agganciai la caviglia con un piede e tirai. Crollò a terra,
finendo a faccia in avanti contro il bordo della rotonda, e intanto io
balzai in piedi in un secondo, recuperando la spada, poi gli piazzai un
piede contro la schiena e gli assestai un colpetto sulla tempia con il
filo della spada.

«Basta così» intervenne sir Felix. «Ricominciate.»