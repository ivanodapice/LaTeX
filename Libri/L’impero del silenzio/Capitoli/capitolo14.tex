\chapter{La paura è un veleno}

Non lasciai la mia camera per tre giorni, ordinando al personale di
portarmi i pasti e di ritirare i piatti. Non credo di aver pronunciato
una sola parola in tutto quel tempo e non avrei potuto sentirmi
maggiormente prigioniero se fossi stato un ospite in una qualsiasi delle
bastiglie della Cappellania. Una paura gelida e viscida mi stringeva
nelle sue spire, avvelenandomi con la certezza che sarei stato il
prossimo a cadere. Di certo gli agenti di mio padre avevano preso il
libro con la lettera incriminante, agendo sulla base di un sospetto o
delle riprese di una videocamera di sicurezza relative a una delle volte
in cui Gibson non era stato abbastanza \emph{cauto}. Non avevo neppure
letto quella lettera, non ne avevo avuto la possibilità.

`Idee che non coinvolgono lo scommettere sulla carità dei pirati',
questo era ciò che Gibson aveva detto. Lui non avrebbe mai contattato
gli Extrasolari, come avrebbe potuto? La teoria di mio padre, che avesse
elaborato un qualche accordo con un'astronave degli scoliasti perché mi
portassero via, mi pareva più plausibile. Cosa non avrei dato per
leggere quella lettera, apprendendo il vero piano di Gibson. Il suo
piano \emph{compromesso}, dovetti ricordare a me stesso. Quali che
fossero i contatti a cui poteva essersi rivolto o i suggerimenti che
poteva aver racchiuso nella lettera, adesso tutto era nelle mani degli
agenti di mio padre. Quelle porte erano chiuse, e gli scoliasti non mi
avrebbero mai ammesso senza una lettera da parte di uno di loro.

Quindi sarei entrato nella Cappellania. Avrei imparato le loro vuote
preghiere e altrettanto vuoti rituali, mi sarebbero state insegnate le
procedure dell'Inquisizione e i protocolli di interrogatorio. Sarei
diventato un torturatore, un propagandista. Avrei potuto vedere
l'universo, come avevo sognato di fare, ma sarebbe stato solo per
schiacciarlo sotto il mio tallone, e già solo quel pensiero era di per
sé veleno. Ho avuto una vita tragicamente lunga, abbastanza da conoscere
il male che fanno e perché. Ho visto decine di eretici bruciati e
crocifissi, lord abbattuti dall'Inquisizione e vasti imperi che si
inchinavano ai capricci del Sinodo. Loro, che controllano la razza
umana, che le impediscono di peccare e di incorrere nei pericoli
dell'alta tecnologia, comandano tecnologie immonde quanto quelle a cui
danno la caccia.

Era ipocrisia, ed ero disgustato come può esserlo solo un giovane dai
peccati dell'età e della classe dirigente. Nel mio giovanile cinismo
avevo intravisto una di quelle grandi verità: che nonostante tutti i
suoi discorsi sulla fede, la Cappellania non credeva in niente.
Commetteva quell'estremo errore ateo: pensare che ci sia solo il potere
e che la civiltà sorga soltanto dall'abusare degli innocenti da parte
dei potenti. Non c'è pensiero più malvagio, ed esso era l'anima della
loro falsa religione, per cui io non potevo essere un cantore o un
priore.

Però sapevo che lo sarei diventato.

Il giorno della mia partenza sorse orribile come qualsiasi altro avessi
mai visto, con il cielo coperto che minacciava tempesta. Dovevo prendere
una navetta suborbitale che mi avrebbe portato al Sud, al palazzo
d'estate sulle terre di mia nonna, ad Haspida, per vedere mia madre per
quella che sapevo sarebbe stata l'ultima volta. Mio padre non venne a
dirmi addio, e neppure Felix o qualsiasi altro dei consiglieri anziani.
Una pioggia lieve lambiva già il campo di atterraggio oltre le mura
cittadine, nelle terre basse dove finiva la periferia. Crispin sarebbe
venuto con me, ed era impaziente quanto me di avere un'occasione di
lasciare il Riposo del Diavolo per un po' di tempo.

Il castello attendeva all'orizzonte, una chiazza nera al di sopra delle
luci velate di Meidua. Un forte vento soffiava sulla pianura e verso le
dune orientali e le spoglie formazioni di calcare che sporgevano da esse
come gli scafi di navi infrante.

«Pronto a dire addio alla mamma?» chiese Crispin.

«Cosa?»

«Sei pronto a rivedere la mamma?» ripeté, guardandomi da sotto quella
sua fronte squadrata, dove una piega sottile si stava formando appena
sopra il naso. Mi chiesi cosa stesse cercando sul mio volto, e la
paranoia che era andata crescendo nei giorni del mio autoimposto esilio
ordinò ai muscoli di rimanere immobili. Per un momento mi parve di
sentir parlare Gibson, ricordandomi che la paura era un veleno.

Misi quindi a tacere i miei timori. «Pronto?» replicai. «Suppongo di
esserlo. Mi sorprende che tu venga con me.»

Crispin mi batté una pacca sulla spalla. «Scherzi? Adoro il palazzo
d'estate, e comunque...» Si protese verso di me con fare da cospiratore.
«Al momento casa è un po' un casino, sai?» Mi limitai a fissarlo, non
sapendo cosa dire, e lui parve rimpicciolire di fronte a me, mentre il
gladiatore rivestito della corazza che era stato nel Colosso si
trasformava nel mio fratello minore. Lo fissai troppo a lungo, perché
lui inarcò uno spesso sopracciglio. «Cosa c'è?» domandò.

Allontanai dalla faccia una ciocca di capelli neri come l'inchiostro e
distolsi lo sguardo, mentre affondavo le mani nelle tasche della giacca
e le mie dita sfioravano la carta universale riposta nella fodera di una
di esse. Adesso era inutile. Gli scoliasti non avrebbero accettato
nessuno come novizio senza una lettera di un membro del loro ordine, e
la tradizione richiedeva che la lettera in questione fosse scritta a
mano in un codice noto soltanto all'ordine stesso. Deglutii e scossi il
capo, scompigliando i capelli che avevo appena rassettato. «Io non
voglio partire.»

«Cosa? Preferiresti restare qui?» Crispin arricciò il naso, osservando
la decuria di soldati del Casato ferma sull'attenti non molto lontano.
«Non ti perderai molto.» Le parole mi vennero meno e mi ci volle uno
sforzo per serrare la mascella. Volevo colpirlo, infrangere quella sua
compostezza gioviale. Lo potevo leggere nella sua noncurante scrollata
di spalle, lui non voleva governare. «Spero che nostro padre ottenga
quello che vuole. Sta puntando a una baronia nel Velo, lo sapevi? Dice
che il nostro Casato potrebbe trasferirsi lontano dal pianeta quando
sarò... ecco.» Lasciò a mezzo la frase, perché era abbastanza attento da
rendersi conto che quello della successione poteva essere per me un
argomento che mi turbava.

La navetta apparve in mezzo alla pioggia, scendendo come un corvo su una
carcassa con lo stridio dei motori che trapassava il silenzio della
pioggia. Un senso di vuoto mi risuonò alla bocca dello stomaco, un vuoto
che echeggiava come un tempio abbandonato. Ho provato quella desolazione
molte volte, ma solo una in maniera altrettanto intensa, quando
aspettavo di essere giustiziato nelle segrete del nostro amato
imperatore.

Crispin si illuminò in volto. «Ma tu stai andando là fuori! Questo...
questo è un bene, giusto? Forse vedrai i Pallidi.»

Sbuffai. «Tutto quello che vedrò sarà l'interno della cella di
addestramento di un anagnostico.»

«Anagnostico» ripeté in tono riflessivo mio fratello, grattandosi la
fine peluria sulla mascella. «Una strana parola.»

«Sarò infelice» grugnii.

«Certo!» replicò Crispin, assestandosi il giustacuore rosso vino che
portava sopra la camicia nera. «Finché non diventerai un inquisitore e
comincerai a scaricare bombe atomiche sui Pallidi.» Sogghignò.
«Potrebbero perfino mandarti a fare da consigliere alle Legioni, al
fronte.»

Distolsi il volto con una smorfia. «Vorrei che facessimo la pace con gli
alieni.» A quel punto Crispin si fece insolitamente silenzioso, e nel
percepire su di me lo sguardo di quei suoi occhi piatti mi girai,
scoprendo che mi stava fissando intensamente. «Cosa c'è?»

«Pensi davvero che i Pallidi valgano la pena di essere salvati?»

«I Cielcin?» Socchiusi gli occhi, poi girai le spalle a mio fratello per
seguire l'avvicinarsi della navetta. Le nostre guardie si spostarono
leggermente, modificando la loro postura nel prepararsi all'imbarco,
flettendo l'armatura con un sommesso frusciare delle piastre. «Sono
l'unica altra civiltà dello spazio che abbiamo mai incontrato. Non credi
che si meritino un po' di...» Accennai al cielo. «Di tutto questo?»

Crispin sputò sul cemento. «Eresia.»

Lo guardai inarcando le sopracciglia e sospirai profondamente. «Non
voglio essere un prete.»

«Nostro padre me lo ha detto.» Potei avvertire il cipiglio che abbassava
il tono della voce di Crispin.

«Hai parlato con nostro padre? Dopo che Gibson...» Non riuscii a
pronunciare le parole e dovetti chiudere gli occhi per bloccare le
lacrime improvvise. Ci riprovai. «Gli hai parlato di recente?»

Mio fratello scrollò le spalle da bue. «Solo per un momento. Dovresti
sentirti onorato. Ho sentito dire che la Cappellania non accoglie
chiunque.»

«\emph{Ekayu aticielu wo}» ribattei. `Io non sono \emph{chiunque}.' Lui
però riconobbe la lingua, e il suo volto pallido si fece ancora più
pallido. «Vorrei invece andare dagli scoliasti.»

«Ho sentito anche questo» disse, chiaramente sconcertato dal mio piccolo
sfoggio della lingua dei Cielcin. La navetta atterrò appena oltre il
nostro padiglione, e quattro delle nostre guardie si affrettarono a
ormeggiarla. «Non riesco proprio a immaginare perché qualcuno possa
volere una cosa del genere. Soffocare in quel modo i tuoi sentimenti.
Non pensi che sia strano?»

Rimasi in silenzio per un momento, con lo sguardo sulla lontana città
che intravedevo fra la pioggia sempre più fitta. Il Riposo del Diavolo
sembrava quasi essere parte della tempesta, una forma più nera fra tutto
quel grigiore. Per come mi sentivo... oppresso dal timore strisciante
che mio padre non avesse ancora finito con me... desiderai di essere
capace di dominare le emozioni come faceva uno scoliasta, di poter
sprofondare nella quiete della loro apatia e dimenticare me stesso. «È
uno strumento, Crispin.»

«So che è uno strumento, dannazione.» Mosse qualche passo verso il bordo
del padiglione e la navetta in avvicinamento. «Ti ho solo chiesto se non
lo trovi strano.»

«No.» Spinsi di nuovo indietro i miei lunghi capelli neri e socchiusi
gli occhi, scrutando fra la pioggia. Volevo ricordare questo momento, il
modo in cui il castello ombroso sbiadiva sotto la pioggia. Avevo il mio
libro di schizzi e le matite nella sacca di pelle rossa appoggiata
contro il baule. «Là fuori ci sono cose più strane.» Accennai con il
pollice al cielo al di fuori del padiglione.

Questo generò un altro silenzio, e la mia attenzione venne catturata da
due guardie che scesero la rampa della navetta, segnalando alle altre
guardie che era tutto a posto servendosi del codice di gesti che era
unico del nostro Casato. Gli altri si voltarono immediatamente per
prendere il nostro bagaglio, e uno di essi mi porse la sacca rossa con
un sommesso: «Mio signore.»

Mi passai l'ampia cinghia sopra la testa e assestai la sacca al suo
posto. «Voglio dire,» insistette Crispin, cocciuto come sempre «sono
tutti un po' strani, vero? Un po' meccanici. Una volta ho sentito Severn
dire che dovrebbero arrestare tutti gli scoliasti come eretici, quindi
forse è meglio che tu sia dall'altra parte.» Mi rivolse un sorriso
incerto. Adesso capisco che stava cercando di essere conciliante, ma a
quel tempo...

«Eretici? Puoi onestamente startene lì e dirmi che la Terra è morta per
ispirare l'Esodo? Che si è sacrificata perché potessimo spargerci fra le
stelle?» ringhiai, aggredendo verbalmente mio fratello, lieto che la
maggior parte delle nostre guardie non fosse a portata di udito.

Crispin -- che ritengo non essere un vero credente ma che si comportava
con quel tipico atteggiamento difensivo a cui ricorrono sempre i fedeli
se sfidati -- protese in fuori il mento. «Perché non avrebbe dovuto
farlo?» ribatté.

«Perché è un pianeta, Crispin.» Agitai una mano in cerchio. «Un pezzo di
roccia da qualche parte nello spazio. Non tornerà, non risponde alle
preghiere.» Strinsi meglio la cinghia della sacca. «Non verrà a
salvarci.» Alle mie spalle sentii due peltasti trattenere il respiro e
mi girai in tempo per vedere uno di essi tracciare un gesto protettivo
con indice e mignolo. A quel punto però aveva smesso di importarmi.

«Questo non lo puoi sapere, fratello!» mi gridò dietro Crispin.

Lo ignorai e salii la rampa, lasciando la pioggia per l'angusto
compartimento della navetta. Essendo basso, non dovetti chinarmi per non
sbattere contro la murata e ignorai i saluti dell'equipaggio nella sua
scialba uniforme, sedendo su una delle poltrone vicine a una finestra.
Quando Crispin mi raggiunse avevo già preso posto. Avvertii il suo
sguardo rovente provenire dal lato opposto del corridoio mentre riponevo
la sacca sotto il sedile e mi appoggiavo allo schienale per guardare la
nostra scorta salire a bordo.

«Non puoi dire cose del genere, sai?» sibilò Crispin, quando anche
l'ultimo di loro fu passato oltre.

Il bagliore di un lampo preannunciò un tuono che giunse rapido e
improvviso, scuotendo la navetta prima ancora che la luce fosse svanita.
«Altrimenti?» mormorai, non più interessato alla conversazione o a
passare del tempo con mio fratello.

«Ti consegneranno ai cathar, proprio come hanno fatto con il nostro
vecchio tutore.»

Serrai le dita come artigli suoi braccioli della poltrona,
momentaneamente libero dalla paura. «Che lo facciano, allora. Non li
servirò.» Le mie parole suonarono roventi quanto la mia ira. Per come la
vedevo, non avevo niente da perdere. Entro una settimana mi sarei
ritrovato immerso in un sonno criogenico finché non mi avessero
risvegliato nell'orbita di Vesperad. Per allora sarebbero passati anni,
e anni luce, e tutte le stelle sarebbero cambiate.

Anche da quella distanza potevo sentire le campane della città che
rintoccavano e l'enorme carillon del Riposo del Diavolo. La navetta
cominciò a rullare e una voce femminile ci informò che ci stavamo
preparando alla partenza. Scoppiò un altro tuono e il coro discorde
delle campane cessò, lasciando allo scrosciare della pioggia e al sibilo
del motore il compito di scandire l'ora. Uno. Due.

Stavamo partendo in orario. Se pure Crispin intendeva continuare a
discutere con me, la rapida accelerazione glielo fece passare dalla
mente. Le campane del castello suonarono ancora. Tre. Quattro. Cinque.
Per un momento provai l'eccitazione che avevo sempre desiderato, una
breve gioia quando la navetta accelerò bruscamente e si sollevò
nell'aria. Sei. Sette. Fummo pressati gentilmente contro il sedile per
appena un istante, poi il campo di soppressione della navetta si attivò
e contrastò il nostro cambiamento di massa inerziale. Più distanti del
tuono, potevo sentire ancora le enormi campane. Suonarono otto volte,
poi nove, dieci. Anche se la massa inerziale sembrava di nuovo normale,
immaginai che il mio cuore accelerasse e mi si lanciasse fuori dal
petto, nella tempesta, librandosi in alto e lontano... fino a
scomparire, abbandonandomi del tutto. Mentre io sarei ritornato a terra,
esso svanì come una nave in curvatura, scivolando più veloce della luce
verso mondi che io non avrei mai visto. In basso, la possente campana
suonava ancora.

L'orologio stava scandendo l'una di pomeriggio.
