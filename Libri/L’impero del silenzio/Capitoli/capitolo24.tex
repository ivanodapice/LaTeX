\chapter{Quei giorni irrazionali}

Sono passati tre giorni e tre notti dall'ultima volta che ho posato la
penna sulla pergamena. Ho riflettuto a lungo su come procedere, su come
meglio raccontare quei giorni e anni persi sulle strade di Borosevo.
Dicono che quando era ragazzo il principe Cid Arthur venne tenuto dal
fedele arconte di suo padre in un palazzo del piacere, isolato da morte,
malattie e povertà, perché un vate aveva profetizzato che se avesse
visto le brutture del mondo Cid Arthur avrebbe rinunciato al trono
paterno e sarebbe diventato lui stesso un predicatore. Quella era una
cosa che mi aveva sempre lasciato perplesso, perché io stesso ero
cresciuto in un palazzo, ma conoscevo la povertà e le malattie e avevo
avuto i miei incontri ravvicinati con la morte; prima con quella di mia
nonna e in seguito con quella di zio Lucian, che era morto nello
schianto della sua navetta. Di conseguenza non riuscivo a capire come
Arthur avesse potuto essere tanto cieco.

Adesso lo capivo.

Come c'era una differenza fra la notizia della distruzione di un remoto
pianeta e la morte sanguinosa di uno schiavo nel colosseo, così un conto
è sapere che ci sono povertà e malattie, e un'altra è aggirarsi fra i
poveri e i sofferenti. Vedevo spesso mendicanti coperti di ulcere nere e
con la pelle che si staccava implorare la Cappellania di essere liberati
non dal male ma dalla malattia. La necrosi dilagava in città, dono di un
qualche microrganismo alieno molto simile a un batterio. Anneriva la
pelle, consumava la carne e induriva linfonodi e polmoni. I prefetti
cittadini ammucchiavano i corpi nelle piazze e li bruciavano, con il
fumo che portava le lanterne votive verso un cielo che sembrava sordo a
qualsiasi preghiera. Il mio sangue palatino mi difendeva dalla malattia
in sé, ma non c'era difesa contro il suo orrore.

Di notte il volto di mio padre mi tormentava, il suo e quello delle
maschere funebri dei nostri antenati, e degli urlanti prigionieri senza
nome della bastiglia dove aveva desiderato mandarmi. Soffrivo per amor
loro, e di mia madre. Sognavo di vederla trascinata al palo per le
fustigazioni e sferzata da un cathar bendato e indifferente, vestito di
nero. Quando mi svegliavo, sudato e singhiozzante in mezzo a un mucchio
di cartoni fra un negozio di tessuti e un panettiere, lo facevo pensando
a Gibson. Il vecchio aveva sofferto ed era stato esiliato per niente,
perché io ero lì a marcire al limite fra l'Impero e il Velo, in un
sistema che si affacciava sul Golfo Cupo, fra il Braccio di Centaurus e
quello di Norma.

Però mi costringevo ad andare avanti.

Come Cid Arthur nella sua ricerca dell'albero di Merlino, facevo quello
che dovevo per sopravvivere. Mangiavo pesce crudo pescato nei canali,
razziavo i bidoni del compost. Dipendevo dalla gentilezza dei venditori
ambulanti -- una categoria di uomini non molto famosi per la loro carità
-- e avevo imparato a mendicare. Non posso dire di essere stato bravo in
quello, ma alla fine la disperazione aveva distrutto i resti della mia
dignità palatina.

Suppongo che mi sarei potuto arrendere in qualsiasi momento,
sottomettendo il mio sangue e il mio anello all'autorità del conte e
aspettando che mio padre mi recuperasse. Ne fui tentato, terribilmente,
soprattutto in quelle prime settimane. Gli anni scivolavano via nello
squallore, e le giornate erano lunghe. Per quanto possa suonare folle,
nonostante la miseria e le lotte, i prefetti e le bande di strada... ero
felice. Per la prima volta nella mia vita ero davvero libero. Da mio
padre, dalla mia posizione... da tutto.

Non era abbastanza. Potevo anche essere libero, ma esserlo ed essere
meno che un servo è una magra consolazione. Ogni notte le stelle mi
chiamavano attraverso la caligine di Borosevo: non mi erano mai sembrate
tanto lontane. Tutto quello che mi serviva era un modo per lasciare il
pianeta. Desideravo trovare un mercante privo di scrupoli che fosse
disposto ad assumere un uomo senza i necessari documenti, ma non potevo
arrivare neppure a mezzo miglio dal campo di atterraggio perché
sparavano a vista agli intrusi.

Con il tempo qualsiasi sogno ambizioso potessi nutrire svanì fino a
trasformarsi in sogni di cibo. Mi mancava il vino con un'intensità che
trovo difficile descrivere, e così pure la frutta e un pasto caldo... e
l'acqua, più di tutto. Sapete cosa si prova a sentire la mancanza
dell'\emph{acqua}? Ero ridotto a spillarne dai barili per la raccolta
dell'acqua piovana, e nel frattempo rimuginavo sul mio destino,
sull'assoluta distruzione delle mie speranze e dei miei sogni. Nella mia
mente vedevo la lettera di Gibson che bruciava, con il fumo nero che
saliva a spirale dai suoi angoli contorcendosi come un drago che si
divora la coda.
\leavevmode\\
\leavevmode\\
\textit{Non sarei mai andato su Teukros.}

\textit{Non sarei mai andato da nessuna parte.}


