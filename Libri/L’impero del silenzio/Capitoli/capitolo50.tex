\chapter{Senza finzioni}

L'acqua di mare minacciò di riversarsi sui miei stivali mentre saltavo
dal ponte del velivolo, con le gocce che mi schizzavano gli abiti
disturbate dai jet repellenti montati sul ventre dell'apparecchio nero e
argento. Una decuria di soldati nella livrea dei Mataro ci precedette,
sgombrando un percorso lungo la fascia della marea e verso il luogo
sulla spiaggia dove i douleter in uniforme cachi erano in attesa, con i
loro schiavi umandh allineati per un'ispezione. Ricordando l'etichetta,
mi girai per aiutare il giovane lord Dorian -- ormai un uomo secondo
tutti i canoni dell'Impero -- a scendere dall'apparecchio. Lui accettò
gentilmente la mia mano, come fece anche sua sorella, ma la dottoressa
Onderra, che veniva per ultima insieme a una donna più anziana
nell'uniforme cachi della Gilda dei Pescatori, la allontanò con un colpo
e andò a guado fino a riva davanti a me, con il tablet per comunicare
con gli Umandh puntellato contro un fianco rotondo.

Indugiando per un momento vicino al velivolo che si andava quietando, mi
misi sul naso gli occhiali con le lenti rosse e socchiusi gli occhi
nello scrutare il cielo. Quel giorno sarebbe stato un esercizio di
apprendimento: una visita formale alla riserva per alieni degli Umandh,
a Ulakiel. I figli del conte avrebbero imparato qualcosa sugli xenobiti
nativi e mi avevano trascinato con loro nella mia dubbia veste di loro
amico prigioniero. Considerata la presenza di Valka, non avevo
protestato. Come per il continente meridionale presente nelle storie di
Valka, l'isola era una delle rare e antiche cime vulcaniche risalenti a
un tempo antecedente alla morte {tettonica} di Emesh, e la brutta pietra
nera si levava per una cinquantina di piedi al di sopra del livello del
mare, fino a un promontorio consumato dagli elementi su cui il vilicus
umano dell'isola aveva costruito la sua fortezza che dominava gli
sporchi tuguri degli Umandh.

«Vedi la palizzata?» chiese Valka, indicando il mare, dove una
recinzione di maglia metallica si levava libera da ruggine dalle acque
verdi. «Quei poveri bastardi non possono neppure nuotare via.»

«È elettrificata?» domandai.

Valka fece una smorfia. «È riscaldata. In reazione al contatto diventa
tanto calda da cucinare qualsiasi materia organica. Gli Umandh sono
bloccati qui, a meno che non gli vada di bruciarsi qualche tentacolo.»

Fu il mio turno di fare una smorfia. «È orribile.» In quel magazzino di
Borosevo avevo visto gli Umandh che venivano percossi e li avevo visti
mentre venivano brutalizzati nel Colosso. Credevo di avere uno stomaco
robusto, ma in quella recinzione riscaldata c'era qualcosa che mi
lacerava dentro.

«Questo è il \emph{vostro} Impero, M Gibson.» Valka scrollò le spalle e
risalì in fretta la banchina per raggiungere i nostri nobili allievi,
ignorando i peltasti armati che trattavano la sua presenza come qualcosa
di leggermente ostile e si fecero più vicini ai loro giovani padroni.
Qualcosa nel modo in cui aveva pronunciato quel `vostro' mi aveva
inacidito lo stomaco. La seguii, scrollando l'acqua salmastra dagli
abiti trattati e ancora asciutti.

L'uomo che accolse il nostro gruppo sulla spiaggia mostrava in tutto e
per tutto di venire da fuori il pianeta, proprio come Valka e me. Non
aveva la calda tonalità marrone della pelle propria della maggior parte
della gente di Emesh e la sua carnagione era invece cinerea, di un
grigio saturo che non era per nulla naturale. Il mio primo pensiero fu
che si trattasse di un omuncolo, perché mi ricordava soprattutto quella
piccola bestia che Demetri aveva tenuto a bordo della \emph{Eurynasir},
ma nessun omuncolo avrebbe potuto detenere una qualche posizione in una
Gilda sul suolo imperiale. Forse era una qualche mutazione, o una
deliberata scelta estetica acquistata presso un segaossa corporativo.
«Lord e lady Mataro, è così gentile da parte vostra fare visita alla
nostra umile isola. Io sono Niles Engin, il vilicus di Ulakiel.»

Con un cenno, Dorian gli segnalò che poteva smetterla di inchinarsi.
«Grazie, vilicus.»

Il sovrintendente capo dalla faccia grigia si assestò nervosamente i
ricciuti capelli neri. «E la benedizione della Terra sul tuo Efebeia,
mio signore. Ho guardato la trasmissione.»

Anaïs prese il braccio del fratello. «È stato così valoroso, vero,
signore?»

«Sì, Vostra signoria.» Il suo sguardo si posò su Valka e su di me. «E
questi devono essere M Gibson e la Tavrosiana.»

Valka si irrigidì nel sentirsi chiamare soltanto `la Tavrosiana', ma non
si lamentò e porse la mano. «Sono la dottoressa Valka Onderra,
xenologa.»

«Sì, sì.» Il vilicus Engin le strinse il braccio invece della mano --
una differenza sottile -- e senza parere tracciò in modo ancora più
sottile il segno del disco del sole con l'altra mano che teneva lungo il
fianco. Il distacco della Demarchia dalla Cappellania era ben noto, come
pure la sua autonomia, mantenuta soprattutto grazie al suo isolamento.
«Sei l'esperta in fatto di xenobiti, vero? Allora questo piccolo tour
dovrebbe essere estremamente istruttivo. Potrai vedere come vivono nel
loro habitat.»

Valka compresse le labbra in una linea sottile e guardò in direzione
della linea concava della riva, inarcando incredula le sopracciglia
sottili verso i capelli fra il rosso e il nero. Vidi ciò che lei stessa
stava vedendo: il piccolo baraccamento innestato nella base dell'altura
di basalto, dove sporgeva come un fungo dalla riva del mare. Spessi
filamenti di ruggine pendevano dalle lastre di metallo, e in alcuni
punti la ruggine aveva consumato i pannelli, una cosa visibile anche da
quel luogo di accoglienza. Ulakiel poteva anche essere il più grande
rifugio per la specie nativa intelligente del pianeta, ma non era
progettato per fare una gran bella figura davanti a ospiti di qualsiasi
livello. I cinque Umandh che erano venuti con il gruppo di Engin erano
stranamente silenziosi, senza neppure il minimo miagolio o stridio, e
quel freddo silenzio mi inquietava, per cui li tenni d'occhio, là
incombenti con i tentacoli che sembravano assaggiare l'aria e si
muovevano come le mani di un cieco che sperava di trovare la strada al
buio.

\begin{figure}
	\centering
	\def\svgwidth{\columnwidth}
	\scalebox{0.2}{\input{divisore.pdf_tex}}
\end{figure}

«La marea non si alzerà prima di qualche ora» disse uno degli uomini di
Engin, indicando una serie di pali piantati sulla spiaggia per indicare
i livelli a cui arrivava l'acqua del mare di Emesh, che si alzava e
abbassava selvaggiamente a causa dell'effetto delle due lune. D'impulso,
guardai verso il cielo. La luna verde, Binah, era bassa sull'orizzonte
ed era scura dove copriva il bordo del sole gigantesco, mentre la luna
bianca, Armand, non si vedeva perché si trovava sull'altro lato del
mondo. Dorian mi spintonò perché mi affrettassi a raggiungere il vilicus
e Anaïs.

Quel giorno -- come ogni giorno su Emesh -- era rovente e soffocante
come una coperta. Mi spinsi sul naso gli occhiali rossi. La linea
costiera puzzava di pesce e di alghe marce, anche se era un mistero come
l'uno o le altre riuscissero a raggiungere le acque limitate della
riserva, e quell'odore mi costrinse a reprimere l'impulso di arricciare
il naso come avrebbe fatto Gilliam.

«Questo è un dannato spreco» sussurrò Valka, in reazione a una domanda
sulla densità della popolazione. «Una meraviglia neurologica
dell'universo... e questi imbecilli li obbligano a pescare.» Incurvò
leggermente le spalle nel rimuginare.

Io potei soltanto distogliere lo sguardo e ripetere: «È orribile.» Non
lontano, un Umandh emerse dalla risacca trascinando una rete piena a
metà di pesci del colore dell'argento brunito. La creatura cominciò a
ronzare, un suono basso e monofonico che proveniva dalla stessa apertura
sulla sua testa da cui sporgevano i lunghi tentacoli. Sentii lo sguardo
di Valka su di me ma non potevo girarmi per vedere la sua espressione
perché farlo sarebbe equivalso a segnalare l'entità dell'ammissione che
avevo appena fatto. Vedevo ancora la testa di Makisomn, con la lingua
troppo lunga che penzolava mentre essa veniva scossa nel pugno di
Dorian, che procedeva ora non molto lontano da noi insieme a sua sorella
e al vilicus.

Esaminai la baraccopoli che gli Umandh avevano costruito e li guardai
emettere il loro ronzio, attratti dal canto del primo, mentre si
muovevano con apparente assenza di intelligenza, diretti come le
formiche da un impulso comune a riunirsi. Vedemmo quello appena uscito
dal mare distribuire la sua pesca con un turbinio di movimenti
tentacolari, consegnando uno o due per volta quei piccoli pesci ai suoi
fratelli, che portavano quei bocconi oltre l'orlo alla sommità del
tronco, dove supposi dovesse trovarsi la bocca. Non era difficile
immaginare i loro immobili antenati radicati sul fondo del mare mentre
stendevano i tentacoli nella corrente per catturare le prede.

«Abbiamo appena esportato un migliaio di nativi su Tritone. Un buon
campione riproduttivo, anche se la riproduzione non è una sfida
particolare per quelle bestie» stava dicendo il vilicus. Due dei suoi
uomini in uniforme cachi si batterono il bastone elettrico contro la
coscia, sorvegliando attentamente i coloni. Più lontano, la nostra
scorta se ne stava in silenzio sull'attenti con la mano nervosa che non
si allontanava dalle armi nel caso che i figli del conte si fossero
trovati a essere minacciati da quelle creature schiavizzate.

Dorian si accigliò leggermente, ma fu Anaïs a ribattere. «Non ci eravamo
resi conto che stessimo spedendo gli Umandh fuori dal pianeta.»

«Il Casato Coward di Tritone sembra ritenere che quelle creature possano
essere addestrate a tendere cavi sul fondale marino.» Engin si accigliò.
«Non che abbia mai saputo che siano anche solo capaci di avvitare una
lampadina.» Lanciai un'occhiata a Valka. Non era forse questo che i
coloni stavano facendo il giorno in cui l'avevo conosciuta? Contribuendo
a riparare uno dei frequenti guasti elettrici del castello? «Ma quello è
un problema dei Coward, e se riusciranno a trovare un modo per usare
quei bastardi, tanto meglio. Perfino i Normanni non sono riusciti a
farlo, quando dominavano qui.»

A quel punto mi ero fatto un'idea di quell'uomo e di quale fosse il suo
posto nei mondi. «Eri un mercante?» Si dice che gli xenobiti delle altre
razze di coloni vengano ancora venduti lungo le rotte spaziali, insieme
ora al commercio di schiavi cielcin che è la mia sfortunata eredità.

«Sì» confermò Niles Engin, gonfiando il petto. «Gestivo una linea del
commercio dei Cavaraad fra Sadal Suud e i sistemi interni. Ho
contribuito a ravvivare le vecchie corti.» Non avevo mai visto un
Cavaraad, i cosiddetti Giganti, con la loro testa sfondata come se
qualcuno avesse preso un'effigie in argilla e avesse premuto con il
pollice per scavare la faccia. Erano stati popolari nel Colosso alcuni
secoli prima, quando la loro mole enorme veniva contrapposta a mirmidoni
come ero stato io.

«I Cavaraad?» ripeté Dorian, con un'espressione confusa sul volto scuro.
«Eri uno schiavista?»

«Lo è» sbuffò Valka, dalla periferia del gruppo. La vidi allontanarsi a
grandi passi, con le spalle incurvate come quando studiava i dati sul
suo tablet. Stava ascoltando, nel tentativo di comprendere il canto
degli Umandh. Lasciai i due giovani palatini con il vilicus e con le
guardie, andando a fermarmi accanto alla dottoressa, con gli stivali che
scivolavano nella sabbia. Lei mi indicò una serie di pendenti
intrecciati che decoravano le case fatte di rifiuti degli Umandh. «Vedi
questi?»

Annuii, riparandomi gli occhi con una mano. «Sono segnavento?» Anche se
non ci somigliavano minimamente, mi ricordavano infatti le offerte di
preghiera in legno e carta che pendevano nel vestibolo dei templi della
Cappellania o giacevano ai piedi di un'icona... preghiere per chiedere
forza, salute, coraggio. Per ottenere amore o ricchezza.

«Nessuno lo sa» replicò Valka, spostandosi verso la struttura più
vicina. «Sono dappertutto negli alloggiamenti che gli Umandh hanno a
Borosevo. Potrebbe essere una cosa religiosa.» Espirò l'aria fra i denti
con un sibilo. «Detesto non poterglielo chiedere.»

«Non puoi usare il tuo tablet?» Accennai con la mano a quel congegno che
le pendeva di nuovo dal fianco come il cingulum di un legionario.
«Chiederlo a loro?»

Mi scoccò un'occhiata con quegli occhi dorati. «Questo non è un
linguaggio, M Gibson. Gli Umandh armonizzano su frequenze diverse per
eseguire dei compiti. Uno canta che ha fame, per esempio, e gli altri
armonizzano con esso finché non vengono nutriti. Sono più intelligenti
in branco, ma non molto più di uno scimpanzé.»

«Quindi stai dicendo che tutto quello a cui serve quel tablet è...»

«Imitare uno dei loro segnali. Quando sentono il nostro segnale sanno
che è diverso da uno dei loro, ma la tua gente gli ha insegnato a
obbedire.»

Non abboccai all'esca. Non stava parlando di me. «Quante parole...
quanti segnali hanno?»

«Solo qualche dozzina, per quel che ne sappiamo. In tutta onestà, hai
idea di quanto sia improbabile che i Cielcin abbiano qualcosa che noi
possiamo riconoscere come grammatica? Quanto sia insolito?»

«Circa una specie su venticinque» replicai automaticamente.

Gli occhi di Valka si illuminarono. «Non fare il furbo» disse, ma
sorrise.

Incrociai le braccia e mi inclinai verso di lei. «Ho letto una teoria
secondo cui le lingue come le nostre facilitano lo sviluppo della
civiltà e che sarebbe questo il motivo per cui noi e i Cielcin siamo le
sole culture dell'universo che viaggino nello spazio.»

«Dell'universo \emph{conosciuto}» mi corresse. Potei percepire la sua
figura al mio fianco, irrigidita da una qualche preoccupazione nascosta.
«Stai parlando di Filemone di...»

«Filemone di Neruda!» la interruppi, con l'entusiasmo che aveva la
meglio sul mio innato senso dell'etichetta. «\emph{Grammatiche
	innaturali}!» Mi girai per fissare Valka, che appariva sorpresa nello
scoprire che conoscevo il nome di quell'uomo e il suo trattato.

Contrasse le labbra. Era colpita? Si portò una mano al volto. «Tor
Filemone è convincente, ma le dimensioni del campione di cui
disponiamo... sono troppo piccole. E gli Irchtani e i Cavaraad hanno un
linguaggio ma non viaggiano nello spazio. La sua ipotesi mi piace, ma è
soltanto questo, un'ipotesi.»

«Stai dicendo che non possiamo supportare le sue affermazioni finché non
esaminiamo altre...»

«Finché non esaminiamo altre specie. Benissimo!» Adesso sorrideva
apertamente. «Niente male per un \emph{brathandom} come te.» Non
conoscevo quella parola, ma era chiaro che mi stava prendendo in giro...
no, mi stava provocando. Quando aveva cominciato a farlo?

Ci sarebbe stato tempo per pensarci su in seguito, decisi, e accantonai
la cosa. Per quanto interessante, era difficile immaginare una specie
veramente collettivista come gli Umandh che sviluppava le conoscenze
tecniche per progettare delle scarpe, tantomeno un'astronave. «Quindi se
non possiamo \emph{veramente} tradurre, cosa fa quel tablet?» chiesi.

«Hai mai cercato di dire a un gatto cosa fare?»

La guardai e scossi il capo. «No.» La verità era che non avevo mai visto
un gatto perché su Delos non ne avevamo, e se ce n'erano su Emesh
venivano tenuti in casa perché non cadessero preda degli ornithon.

Lei non mi stava guardando, era china proprio sul tablet di cui stavamo
parlando e armeggiava con i comandi. «Questo non è facile, Gibson.»
Inarcò un sopracciglio, con i capelli che le {scivolavano} via da dietro
l'orecchio per arricciarsi in mezzo a noi. «Non è come con il tuo
Cielcin. Gli Umandh non pensano come facciamo noi, la loro sfera
cosciente opera in base a princìpi del tutto diversi.»

Mentre parlava studiai quello che si riusciva a vedere dei tuguri dei
coloni, fatti di rifiuti e pezzi di vecchi edifici ammucchiati insieme,
come un uccello avrebbe potuto fare per costruirsi un nido. Quello che
sembrava il pavimento di una vecchia casa prefabbricata era appoggiato
alla parete dell'altura, accasciato dove il tempo e lo stress ne avevano
piegato i materiali compositi facendo sì che aderisse alla fascia come
un'ostrica. «Quelli cosa sono?»

Indicai una serie di aste di metallo che punteggiavano il panorama
intorno alle basse strutture. Da esse pendevano dei cerchi dalla
superficie intrecciata con frammenti di cordino e di cavo, insieme a una
qualche sostanza simile a spago che poteva essere budella. Anche se
scintillavano di colore, non c'era uno schema apparente, per cui i
colori erano senza senso e lasciavano sulla mente un'impressione neutra,
come fanno i colori quando vengono mescolati da un bambino.

Nel vederli Valka si illuminò e mi guidò più avanti lungo la costa. La
seguii, lanciando solo un'occhiata verso il punto sotto le banchine dove
Engin era accoccolato e circondato da quelli che potei supporre essere
bambini umandh, distinguibili dagli adulti solo per il fatto che erano
più bassi. Engin stava distribuendo pezzi di frutta candita che tirava
fuori da un sacchetto di carta e Anaïs rideva mentre le creature le
prendevano il cibo dalle mani.

Valka aveva tirato giù dal suo posto uno di quei cerchi, e solo allora
vidi i lacci che pendevano da esso. Ce ne dovevano essere una dozzina,
annodati e intrecciati con frammenti di conchiglia e di pietra che
ticchettavano piano nella brezza e con il movimento del cerchio.
Notandolo, feci la domanda più ovvia. «A cosa serve?»

Lei scrollò le spalle e mi porse il cerchio.

Sentendomi come se stessi venendo messo alla prova lo presi e lo rigirai
fra le mani. «Queste pietre non possono fare molto rumore.»

«Per te o per me» replicò Valka, riponendo il tablet nella sua custodia
che aveva al fianco. «Gli Umandh vedono con le orecchie.» Si batté un
colpetto sull'orecchio per dare enfasi alle sue parole. «Per loro è un
suono più forte.» Si fece un po' più vicina, sbirciando da sopra la mia
spalla mentre esaminavo l'oggetto. In quel posto di rifiuti riciclati,
quelle erano le sole cose che sembravano essere state create bene e
davvero. Feci scorrere le dita sull'intreccio che formava una ragnatela
nel cerchio e tirai con delicatezza uno dei tasselli. D'impulso, chiusi
gli occhi e immaginai di vedere l'oggetto come poteva fare un Umandh.

La superficie dell'intreccio all'interno del cerchio sporgeva in una
serie di creste, alcune meno marcate e più dure al tatto rispetto ad
altre. Nel tutto c'era uno schema, una geometria. «È un anaglifo»
affermai, riaprendo gli occhi.

I suoi occhi dorati si illuminarono per la sorpresa. «Hai letto al
riguardo?»

Scossi il capo. «No.» La verità era che ero stato troppo preso a tenere
il broncio per la mia prigionia per studiare.

Lei socchiuse gli occhi e si tormentò di nuovo il labbro inferiore. Dopo
un momento aggiunse: «Le squadre esplorative dei Normanni ci hanno messo
quasi un decennio per capirlo.»

«Forse sono stato solo fortunato.» Non credo che Valka mi abbia creduto,
ma reagì con una scrollata di spalle. «Si tratta di un qualche sistema
di scrittura?» Aveva senso che una specie cieca come gli Umandh
mantenesse una documentazione leggibile al tatto.

Valka scrollò nuovamente le spalle. «Scrittura, arte, mappe. Non
possiamo saperlo per certo.» Anche questo aveva senso, considerato
quanto era difficile comprendere la mente degli Umandh, e tuttavia
percepii che Valka sapeva qualcosa che non mi stava dicendo. «Di certo
significa \emph{qualcosa}, ma...» Pronunciò quella parola, `qualcosa',
con un'enfasi che vietava di fare domande pur invitando la curiosità,
eppure percepii un'ombra nelle sue parole... o mi sembra adesso di
averlo fatto.

Incerto, lasciai che il silenzio si prolungasse fra noi mentre esaminavo
i tasselli che pendevano dal fondo del cerchio grande come un piatto.
Poi mi guardai intorno e vidi gli altri allontanarsi lungo la riva.
«Perché dei cerchi?» chiesi.

«Cosa?»

«Guarda le cornici» spiegai, facendo una pausa abbastanza lunga da poter
allungare entrambe le mani e appendere di nuovo il cerchio sul palo.
«Sono perfette, ma qui niente è perfetto.» Accennai ai tuguri che ci
circondavano. «Perché importa loro di questi oggetti?» Cercai di
immaginare i loro fini tentacoli mentre torcevano frammenti di cavo di
recupero e di giunchi per ottenere quella forma, sbrogliandoli e
lisciandoli con una cura che non avevo riscontrato in nessun altro
lavoro della loro specie. D'un tratto avvertii un senso improvviso
quanto schiacciante di diversità, un muro di comprensione più grande
della mera barriera del linguaggio. Aggirai una piccola cresta della
spiaggia e sbirciai sotto il pezzo di pavimento inclinato che era stato
trasformato in una sorta di tetto. Intanto un'altra domanda prese forma
dentro di me, e dal momento che Valka si era fatta inesplicabilmente
taciturna, la formulai: «Come sono le tue rovine? Non ho potuto trovarne
nessun ologramma nella sfera-dati del palazzo.»

«Calagah?» replicò Valka. «Credevo non stessi leggendo niente al
riguardo.» Era in piedi appena dietro di me e mi seguì quando scivolai
giù per un terrapieno poco profondo e nello spazio angusto e fresco
sotto quel vecchio pavimento. L'interno era più asciutto di quanto
avessi immaginato e le pareti brulicavano di frammenti di rifiuti legati
a cordini che oscillavano leggermente, disturbati dal nostro passaggio.
Mi chiesi che \emph{aspetto} avessero per gli Umandh, quale musica
sommessa aggiungessero a quel posto brutto e buio.

«No, intendevo che non sto leggendo niente riguardo agli anaglifi.»

«Ah. Perché me lo chiedi?»

«Ecco, se hanno costruito una civiltà prima che i Normanni sbarcassero
su Emesh, mi interesserebbe sapere con che cosa l'hanno eretta. Non c'è
terra a sufficienza per far crescere qualcosa di simile a una foresta.
Hanno usato la pietra?» In qualche modo, non riuscivo a immaginare
quegli anfibi simili ad alberi in veste di muratori. Valka era in piedi
sulla soglia e scrutava con occhi socchiusi la riva, dove si trovavano
gli altri. Non potevo vedere quello che lei stava guardando, e quando
non mi rispose la sollecitai. «Dottoressa?»

Lei sussultò e si girò verso di me. «Cosa? Chiedo scusa.»

«Cosa c'è?» domandai, spostandomi verso l'apertura per guardare in
direzione di Engin e degli altri. I bambini umandh si erano sparpagliati
e giravano su loro stessi nell'acqua bassa, emettendo un ronzio acuto e
ululante. Erano risate? «Cosa stai guardando?»

«Vedi le strisce marroni su quello?» Indicò il tronco di uno degli
Umandh più maturo, che aveva rigonfiamenti marrone scuro da cui linee
sottili si allargavano a spirale in un modo casuale che ricordava le
crepe nel guscio rotto di un uovo. «Engin li picchia.»

Sentii un nodo che mi si formava alla bocca dello stomaco e distolsi lo
sguardo. Dopo un momento dissi a me stesso che mi stavo comportando da
sciocco. Avevo visto uomini e donne mutilati sui gradini della
Cappellania. \emph{Homo hominis lupus}. Perché questo mi turbava
maggiormente? Mi costrinsi a guardare. «Li ha fatti frustare?»

Valka stava ancora parlando e il tono luminoso e tagliente delle sue
parole trapassava la nube lanuginosa che avevo nella testa. «...credevo
di averli indotti a smettere. \emph{Meonvari tebon kahnchob ne kar
	akrak}. Non li può frustare! Li ha tenuti rinchiusi e li ha
\emph{incisi}.»

Mi sentii sussultare. «Cosa?»

Lei si passò una mano sugli occhi e volse le spalle alla spiaggia. «Voi
\emph{anaryoch} siete ossessionati dal dolore.» Da sopra la sua spalla
vidi Dorian Mataro sollevare in aria ridendo uno dei bambini umandh. Non
saprei dire cosa stesse provando la creatura.

«Incisi?» ripetei, del tutto dimentico della mia precedente domanda.
Ridendo, il vilicus Engin tolse il bambino alieno dalle mani di Dorian e
lo rimise in acqua con una cura sorprendente.

Valka mimò con il pugno l'atto di piantare un paletto e mi oltrepassò
per addentrarsi nella silenziosa penombra sottostante il pavimento
trasformato in soffitto. «È quella vostra dannata chiesa.»

«La Cappellania?»

«Normalizza la violenza. Guardati.» Agitò una mano nella mia direzione.
«Il gladiatore.»

Mirmidone, pensai, ma non la corressi e non replicai immediatamente,
contemplando le corde intrecciate che pendevano dal tetto sovrastante.
Mi resi lentamente conto che ci trovavamo in un posto libero da
videocamere, dalla costante e attenta sorveglianza della corte e della
Cappellania, e non risposi quindi come il figlio fittizio di un mercante
e neppure come il mirmidone che ero stato in precedenza, ma solo come
Hadrian, senza finzioni. «Quello che fanno è... ignobile.»

Avvertii la pressione dei suoi strani occhi ma mi rifiutai di incontrare
il loro sguardo. Nella mia voce doveva esserci stato qualcosa, una sorta
di solennità. Serrai la mascella, d'un tratto timoroso di dire troppo o
di aver già detto troppo. Avrei voluto aggiungere dell'altro, parlarle
di mio padre, del mio dovere verso la Cappellania, ma quello era un
dovere di Hadrian Marlowe, mentre io ero Hadrian Gibson. Per un momento
fummo come due pezzi di un puzzle il cui proprietario sapesse che
dovevano combaciare ma non riuscisse a capire come metterli insieme. Se
avessimo avuto un giorno o anche un'ora di tempo sarebbe stato diverso,
e tutta l'antipatia residua che provava nei miei confronti sarebbe stata
spazzata via.

Ma non era destino che fosse così.

«Credevo fosse contrario alle vostre leggi dire cose del genere»
osservò.

«È blasfemia \emph{essere sentiti} a dire cose del genere» la corressi,
e mi arrischiai a sollevare lo sguardo. Valka era ferma in mezzo alle
corde pendenti disposte dagli Umandh, con la testa inclinata da un lato,
sudando attraverso la camicia sottile e con i capelli incollati al volto
pallido, la cui espressione ricordava un inaspettato raggio di luce
lunare. «Le puoi dire se non ti sorprendono a farlo» aggiunsi,
squadrando le spalle mentre divampava in me il mio antico senso del
drammatico. «E se le \emph{puoi} dire, allora dovresti farlo.»

Valka annuì, allontanandosi una ciocca di capelli dalla faccia con un
gesto che mi fece comprendere l'impulso di Shakespeare di voler essere
un guanto. La mano le indugiò sulla guancia mentre assumeva
un'espressione perplessa, poi scosse il capo. «Cosa dovrei replicare?»
Ripiegò un braccio per sostenere l'altra mano e si accasciò su di essa.

«Non ha importanza, io...» L'impulso di girarmi a guardare da sopra la
spalla se qualcuno mi stava ascoltando era troppo grande. «Volevo solo
sapessi che non la pensiamo tutti così. Non è facile vivere come
facciamo. Non siamo tutti...» Mostri? Barbari? «Non siamo tutti ciò che
pensi siamo.»

Lei chiuse lo spazio che ci separava e mi posò una mano sul braccio, con
le dita che si serravano mentre sul volto le si formava un piccolo
sorriso dolce e triste. «Questo lo so.» Un grido proveniente
dall'esterno ci riportò al presente e a dove eravamo, e lei chiese: «Di
cosa stavamo parlando? Delle rovine?»

«Cosa?» Potevo ancora sentire freddo nel punto in cui mi aveva toccato
attraverso la manica, e sapevo e temevo che quella sensazione sarebbe
svanita entro pochi momenti. «Oh, Calagah! Ti {avevo} chiesto con che
cosa gli Umandh avessero eretto le rovine. Mi scuserai, ma non sembrano
capaci di costruire granché.»

Lei mi guardò a lungo, ruminando sulla risposta mentre accarezzava quasi
distrattamente i frammenti di arte umandh che ci pendevano intorno. Ero
consapevole di quanto mi fosse vicina, dell'odore del suo sudore e del
mio, del lieve profumo dei suoi capelli al di sotto del sentore di
salsedine e di pesce marcio di quel posto. Dovette giungere a una
decisione perché replicò: «Gli Umandh non...»

«Hadrian!» Anaïs sbirciò oltre il bordo del pavimento inclinato sopra di
noi.

Mi ritrassi di scatto dalla dottoressa e trasformai quel movimento in un
inchino altezzoso, con una mano premuta contro il petto in modo tale da
farvi aderire il cotone appiccicoso. «Vostra signoria!»

La palatina serrò la mascella nel vedere Valka «Oh, dottoressa Onderra,
credevo fossi con gli xenobiti. Non devi vaccinarli o fare qualcosa del
genere?»

Valka reagì con un sorriso simile a una scheggia di vetro rotto ma si
inchinò a sua volta. «Non sono quel genere di dottoressa, Vostra
signoria.»

Mentre si raddrizzava, Dorian apparve accanto alla spalla della sorella
e commentò con quella sua voce baritonale avviata a diventare uguale a
quella da basso di suo padre: «Oh, che posto accogliente, vero?» Non
sembrava sincero. Entrambi apparivano spaventosamente fuori posto con
gli abiti di seta dai colori vivaci e dai ricchi ricami, trattata per
essere impermeabile. Soprattutto la sua pellanda mi colpì perché era
tanto raffinata da sminuire ancora di più il tugurio in cui eravamo.

Un vecchio pensiero mi riaffiorò spontaneo nella mente, e cioè che noi
palatini non fossimo neppure davvero umani. Le parole di Saltus mi
gracchiarono nelle orecchie: `Siamo entrambi figli delle vasche.' Mi
accigliai. `Inumani.' Quel termine si applicava agli omuncoli, agli
Extrasolari il cui corpo era pervertito dalle macchine, non al sangue
palatino dell'Impero. D'un tratto mi sentii nauseato.

«Oh, salve, M Gibson, dottoressa Onderra» continuò Dorian, appoggiando
una mano sulla spalla della sorella e dando l'impressione di accorgersi
di noi per la prima volta. «Mi ero chiesto dove foste andati a finire.»

Il vilicus Engin apparve un momento più tardi. «Nel nome della Terra,
cosa ci fate lì dentro? Fuori, fuori! Non restatevene lì fermi!» Fece un
gesto secco con la mano, poi ricordò a chi si stava rivolgendo e
temporeggiò: «Questi tuguri non sono stabili. Venite, mio signore e
signora, venite via.»

Anaïs stava fissando -- no, stava trafiggendo con lo sguardo -- la
dottoressa come se l'avesse sorpresa a rubare, con la refurtiva a metà
strada dalla sua borsa. In quel momento la cosa mi lasciò perplesso ma
l'accantonai quando la ragazza palatina mi afferrò la mano. «Devi venire
a vedere i primitivi! Il vilicus li farà danzare per noi!»

Rivolsi a Valka uno sguardo assillato ma lei si limitò a guardarmi
inespressiva, con un duro sorriso che mi trafisse come tanti ami da
pesca.


