\chapter{La Quiete}

Ero lì fermo che mi fissavo i piedi, con le dita strette intorno a una
bottiglia di rosso di Kandarene che Dorian mi aveva aiutato a
contrabbandare dalle cantine del castello, acutamente consapevole del
sangue che mi pulsava nelle orecchie e di qualcosa di piccolo e di
fastidioso nello stivale sinistro. A disagio in quella quiete, mi agitai
e spostai la bottiglia in modo da tenerla dietro la schiena. Dopo appena
mezzo minuto, quando pareva che tutte le stelle si fossero consumate e
che l'universo si fosse raffreddato, le serrature della porta scattarono
con uno stridio metallico, la maniglia si mosse rumorosamente e una luce
fra il rosa e l'oro si riversò nel corridoio, disegnando un cuneo
sull'intricato disegno del pavimento di legno.

Un uomo nudo, avvenente e con occhi grigi, mi guardò sbattendo le
palpebre e cercando di coprirsi i genitali con un plaid. Balbettai
qualche parola di scusa. «Ho sbagliato stanza, messere. Chiedo scusa.»
Ero certo che non fosse la stanza sbagliata. Distogliendo cortesemente
lo sguardo, aggiunsi: «Credevo che fosse la camera della diplomatica
tavrosiana. L'hanno trasferita altrove?»

«M Gibson?» La voce giunse dalle profondità della stanza, alle spalle
dell'uomo nudo. «Cosa c'è?»

Il mio cuore... non so dire cosa gli successe. Non sprofondò ma
piuttosto lasciò del tutto il mio corpo per precipitare nel nucleo del
pianeta. Lanciai una breve occhiata al giovane muscoloso sulla porta...
un plebeo, a giudicare dalla lieve asimmetria del volto altrimenti
perfetto, ma al di là di questo non avrei saputo dire se era un
cortigiano o un servitore. Non aveva importanza. Era il suo amante e lei
si era dimenticata del nostro incontro. Entrò nel mio campo visivo,
languida e felina come un gatto, con quegli occhi dorati che spiccavano
luminosi nella scarsa luce della sua camera. Si era infilata
affrettatamente una lunga camicia e i capelli fra il rosso e il nero
erano una massa selvaggia e arruffata. Un velo di rossore le accendeva
le guance pallide, ma in lei non c'era traccia di imbarazzo. Prima che
potessi fermare l'urlo silenzioso della mia anima e articolare una
risposta, lei sussultò. «Avevamo un appuntamento!» Nel ricordarlo si
premette una mano contro la guancia e con l'altra spinse la schiena
dell'uomo sulla soglia.

«Sì.»

«Che ore sono?» chiese, spostando lo sguardo fra me e il suo amante.
Glielo dissi e lei sibilò fra i denti. «Vattene» ordinò. «Fuori.»

Avvilito, mi inchinai, facendo del mio meglio per nascondere
l'incriminante bottiglia di vino. «Certo, mia... dottoressa.» Per poco
non usai di nuovo il titolo sbagliato, e questo mi indusse ad
accigliarmi. «Mi dispiace.»

«No, non tu!» Schioccò le dita in direzione dell'uomo. «Com'è che ti
chiami? Mal?»

«Malo» rispose l'uomo, in tono mieloso, spostandosi per passare un
braccio intorno alle sue spalle esili. Avrei voluto urlare, ma Valka lo
spinse via. Accigliandosi a sua volta, l'uomo mi sgusciò accanto senza
preoccuparsi minimamente della sua nudità. Era forse uno dei servitori
destinati al piacere degli ospiti del palatinato? Mi venne in mente
l'harem di mia madre. Di certo anche il conte aveva un suo harem, anche
solo come simbolo della sua autorità. «Chiamami se avrai ancora bisogno
di me, dottoressa.»

Valka gli rivolse un fragile sorriso e gli assestò una pacca sul
posteriore, inducendolo a sussultare e ad allontanarsi in fretta lungo
il corridoio, nudo. D'altronde non era insolito vedere questi prostituti
di palazzo senza vestiti. Sussultai. Il sorriso di Valka si afflosciò
nel momento in cui Malo scomparve alla vista, mentre io l'osservai
allontanarsi prima di inchinarmi ancora con fredda formalità.
«Dottoressa Onderra...» Mi ritrovai con la bocca d'un tratto arida e le
mie parole suonarono leggermente affannose. «Se devo tornare in un altro
momento...»

«No, no!» Si grattò i capelli. «Per nulla, entra.» Mi tenne la porta
aperta. «È colpa mia, ho perso la cognizione del tempo.» Si morse un
labbro. «Ti va di bere qualcosa?»

Ansioso ed esitante, rivelai la bottiglia di vetro verde che tenevo
dietro la schiena. «A dire il vero ho portato io qualcosa, se ti va.» Da
un lato c'era un angolo cottura con una serie di armadietti rivestiti di
piastrelle e fornelli a gas di freddo acciaio sabbiato. Mi diressi verso
il divano... costellato da un groviglio di coperte e dai resti sparsi
del vestiario di Valka... e di Malo. Non sapevo perché la cosa mi desse
tutto quel fastidio, dato che io stesso avrei potuto avvalermi dei
servitori sessuali del palazzo tanto quanto lei, se lo avessi voluto...
solo che non lo facevo mai. Le due alte finestre erano coperte dalle
tende scure che nascondevano il livido cielo del crepuscolo e lasciavano
penetrare solo una luce scarsa. Anche l'illuminazione artificiale era
stata attenuata e le sfere luminose erano fievoli come luce lunare vista
attraverso una patina di nuvole, appese sensualmente nell'aria come
lanterne di carta.

Valka adocchiò la bottiglia di Kandarene e accennò a una credenza vicino
al cucinotto nell'angolo più lontano del suo piccolo alloggio. «Servilo
tu mentre vado a cercare le mie cose.» Scomparve oltre un portale ad
arco in fondo alla stanza lasciandomi a recuperare un cavatappi e i
bicchieri. Confesso di averla guardata mentre si allontanava -- di aver
studiato la sua andatura oscillante con quello sconcertato candore
proprio di tutti i giovani convinti che le loro attenzioni passino
inosservate.

«Vedo che la vita nella capitale ti si confà» osservai, dal mio posto
vicino al piano piastrellato.

«Cos'hai detto?» trillò la voce di lei dall'altra stanza, sfumata di
allegria anche se mista a una certa tensione dovuta allo sforzo della
ricerca.

«Che la vita nella capitale ti si confà.»

La dottoressa riemerse qualche tempo dopo, portando con sé il terminale
da polso e una serie di cristalli di dati. Posato il tutto su un
tavolino vicino al divano replicò: «Io non direi che sia così.» Aveva
fatto una doccia sonica e si era riordinata i capelli anche se portava
ancora quell'ampia camicia. Sul davanti e sulla manica sinistra erano
stampati spettrali teschi umani avvolti da volute di fumo che
sovrastavano il logo di un qualche gruppo musicale tavrosiano, e aveva
indossato un paio di pantaloni di tessuto elastico, come quelli che
spesso si usano per fare ginnastica. Le si addicevano.

«Mi riferivo al tuo compagno» commentai, con quel modo di fare pressante
che molti uomini giovani e sciocchi pensano passare per sottigliezza.

«Cosa?» Inserì uno dei cristalli in una fessura sul dietro del terminale
e attivò il proiettore. «Oh, Malo!» Sbuffò. «È soltanto... ecco, è a
posto.» Mi sentii un po' meglio mentre le porgevo il vino, che lei
accettò con un inchino aggraziato proprio mentre l'ologramma acquisiva
la sua chiarezza al laser. Con le lunghe gambe distese sul tavolino,
Valka sollevò il bicchiere. «Non volevo tenerti in attesa» si scusò,
protendendosi a posare una mano calda sul mio braccio. «Mi dispiace.»

Annuii, inghiottendo il poco vino che avevo sorseggiato. «È tutto a
posto.» Cercai di non pensare al suo compagno, servitore di palazzo o
meno che fosse. «Noi tutti abbiamo delle dimenticanze.»

«Io no» dichiarò senza vanteria ma come se fosse un semplice dato di
fatto. «Devo aver perso la cognizione del tempo...» Abbassò lo sguardo
sul proprio grembo con un'espressione ironica, quasi a dire che era un
po' tardi per simili considerazioni manierate. Lottai per non arrossire
e mi feci forza con pensieri riguardanti Cat che stroncarono prontamente
i miei desideri e li sostituirono con un devastante senso di colpa. «Tu
come stai?» Continuammo in quel modo per un paio di minuti, mentre Valka
usava la coperta sullo schienale del divano per coprirsi.

«C'è una cosa che mi disturba da un po'» osservai, durante una pausa
della conversazione.

Lei inarcò un sopracciglio con fare provocatorio. «Soltanto una?»

Sbuffai con aria confusa. «Per il momento!» Ci scambiammo un sorriso
privato e nascosi il mio dietro l'orlo della mia coppa di vino. «Hai
detto che Calagah era fatto di pietra, ma quando eravamo a Ulakiel...
gli Umandh, ecco... non sembrano capaci di niente di così complicato
come il costruire in pietra.»

Per un momento Valka sedette del tutto immobile, osservandomi, poi si
alzò in modo repentino, come un ragno indotto a un movimento improvviso
dalle vibrazioni della ragnatela, e lasciò cadere la coperta sul
pavimento. «Non ti ho mostrato gli ologrammi!» Tornò in tutta fretta
nell'altra stanza e di lì a poco {rientrò} con un tablet che depositò
sul tavolino antistante il divano. «Volevo farteli vedere...» regolò un
paio di impostazioni sul piccolo schermo, poi premette un pulsante
hardware sul suo bordo. «Questa è Calagah.»

L'immagine proiettata davanti a noi da un raggio laser concavo mostrava
immagini satellitari di un complesso costruito in una profonda fenditura
fra alture di basalto, vestigia dell'antica e da tempo finita attività
vulcanica di Emesh. Valka descrisse la latitudine e le dimensioni del
sito in questione, passando da un'immagine all'altra con languidi gesti
della mano libera mentre recitava cifre e dati di fatto con una casuale
disinvoltura dietro cui si nascondeva un livello di memorizzazione che
avrebbe destato perfino il rispetto di uno scoliasta.

Infine passò alle immagini al livello del terreno, rivelando una
geometria più precisa rispetto alla forma naturale delle colonne di
basalto... piatte superfici di pietra nera come ossidiana, con colonne e
archi che sembravano un delirio da febbre di un matematico. Sembrava...
come posso spiegarmi? Le rovine non sembravano essere parte della
circostante superficie rocciosa. Piuttosto, era come se un artista
immenso le avesse inserite nel panorama a causa di un qualche difetto di
funzionamento della programmazione del computer, un difetto del
procedimento di esecuzione che aveva intagliato la facciata nella parete
dell'altura.

«È splendida» sussurrai infine. Di certo quella era una delle
Novantanove Meraviglie dell'Universo. Oppure la centesima, non presente
nell'elenco. «Quando hai detto che è stata costruita?»

Gli occhi di Valka scintillarono mentre si appoggiava allo schienale del
divano. «Questa è una domanda interessante. La datazione è difficile
perché il materiale usato dal costruttore condivide alcuni dei princìpi
dell'ossidiana ma la datazione mediante idratazione è fuori discussione
a causa della storia di inondazioni del sito. Inoltre, in ogni caso si
limiterebbe a datare il materiale, e sul sito non ci sono calcina o
materiale organico. Niente dipinti o pigmenti, niente tombe.»

«Hai qualche ipotesi?»

«Ne abbiamo, sulla base dell'età degli strati circostanti.» Sorrise, e
mi chiese in tono incoraggiante: «Quanto pensi che sia antica?»

Da ragazzo ero andato con mio zio Lucian a visitare una vecchia città di
pietra che alcuni dei primi colonizzatori avevano costruito su Delos nei
secoli dell'iniziale colonizzazione del pianeta. Erano stati poveri, e a
parte un paio di strutture prefabbricate per il resto avevano utilizzato
il granito locale. `Sai, la pietra per il Riposo del Diavolo viene dalle
stesse cave' mi aveva detto mio zio. `Si trovano sulle montagne.'
Tuttavia, la pietra di quella città era tagliata rozzamente e unita in
modo anche peggiore, e sebbene quello fosse un sito protetto per ordine
della viceregina, la città dimostrava la sua età. Queste rovine aliene
mi ricordavano quella città in rovina, piena della storia e degli
spettri di tutto quel tempo.

«Cinque o seimila anni?» azzardai, aggiungendo un paio di millenni per
buona misura.

Lei scosse il capo e spinse indietro i capelli prima di rispondere.
«No.»

«Di più?» chiesi. «Di certo non possono essere più di dieci...
dodicimila anni?»

Il suo sorriso si accentuò.

«Ventimila?»

«Hadrian.» Pronunciò il mio nome come se fosse stato quello di un
bambino... come se fossi stato un bambino. «Quelle rovine hanno più di
\emph{settecentomila} anni, forse anche un milione.»

Annaspai come un pesce fuor d'acqua per una decina di secondi. «Cosa?»
Non riuscivo neppure a cominciare a formulare tutte le mie domande. Come
poteva qualcosa durare tanto a lungo? Com'era sopravvissuta la città
alle inondazioni stagionali per tanti innumerevoli eoni? Ero così
assorto che per un momento non badai a cosa mi circondava e parlai senza
riflettere. «Gli Umandh non possono aver costruito tutto questo.
Vorrebbe dire che sono... molto più antichi di noi. Avrebbero dovuto
progredire fino a \emph{superarci}.» La nuda verità mi stava però
fissando da dentro quella luce laser. «Loro non potrebbero aver
costruito niente di simile, non ne hanno la tecnologia.»

Valka mi interruppe. «E così torniamo a Filemone e a \emph{Grammatiche
	innaturali}.»

«Non starai suggerendo seriamente che gli Umandh siano stati una cultura
statica per buona parte di un milione di anni! Noi stessi ne contiamo
appena un quarto di milione, e...» Sollevai le mani a indicare la vasta
espansione della civiltà umana, come se le mie dita allargate avessero
potuto abbracciare tutti quei pianeti e quelle stelle, e il Buio in
mezzo a essi. Valka scrollò le spalle in modo frustrante e distolse lo
sguardo, giocherellando con l'orlo della camicia. Non per la prima volta
sentii che c'era qualcosa che non stava dicendo, e nel ricordare il
nostro incontro con Engin, il mercante di schiavi, dissi: «Prendi i
Cavaraad, per esempio. L'etnografia di Hemachandra... come si chiamava?»
Non riuscivo a ricordare il titolo del volume e ci rinunciai. «Lui
descrive i Cavaraad della prefettura di Mataro, su Sadal Suud, come
vicini all'età del bronzo nel loro sviluppo. È stato per questo che ha
fatto pressioni per far ottenere al pianeta lo stato di protettorato
quando il Casato Rodolfo ha cercato di rivendicarlo per sé.»

«Rivendicarlo...» ripeté Valka, sorseggiando il vino.

«Come specie, i Cavaraad sono però più giovani di noi di appena
cinquantamila anni circa» protestai. «Di certo non vorrai attribuire
questo delta al solo fatto che parlano.»

«In realtà cantano.» Valka trangugiò il resto del vino con una facilità
che mi fece sussultare... quel vintage valeva il prezzo di un velivolo.
«Non hanno labbra e usano il diaframma come un mantice per cambiare
tono.» Fece un gesto con la mano libera, premendo l'aria come se fosse
stata una vescica in un modo che mi fece pensare alle cornamuse. Era un
buon paragone, se mai avete sentito cantare i Cavaraad.

«Li hai visti?» Mi protesi in avanti con estremo interesse perché lei
aveva accennato a quella grande ossessione della mia giovinezza. «I
Cavaraad?»

I suoi denti candidi brillarono nella penombra. «In effetti ho passato
un'estate su Sadal Suud.» Allungò la mano verso la bottiglia per
riempire di nuovo il bicchiere. «A proposito, cos'è questo? È davvero
buono.»

«La migliore annata dell'arciduca Markarian... il `969, credo.» Le tolsi
la bottiglia di mano per guardare l'etichetta. «Sì.»

«Markarian?» Lei sgranò gli occhi. «Deve valere una fortuna!»

«Lord Dorian lo ha prelevato per me dalle cantine di suo padre.» Quando
gli occhi di Valka non si socchiusero accantonai l'enormità indicata
dall'annata. Quella notte avevo sperato di fare una conversazione
diversa, quindi commentai: «Gli ho detto che era per una signora.»
Vedendo che si incupiva in volto aggiunsi: «Non ho specificato che era
per una dottoressa.»

Distolsi lo sguardo e mascherai il mio rinnovato imbarazzo bevendo un
lungo sorso. Scivolammo in un silenzio inquieto, a disagio, nel quale il
solo suono era quello fievole e a stento udibile del proiettore
olografico del terminale di Valka. Alla fine i miei nervi ebbero la
meglio sulla mia pazienza. «Su Sadal Suud hai visto le Torri che
Marciano?» chiesi, impaziente di riportare la conversazione in
carreggiata.

Per tutta risposta si chinò sul terminale e attivò un paio di comandi.
Un momento più tardi l'immagine nell'aria si dissolse per essere
sostituita da quella di una sorridente Valka ferma sull'orlo di
un'altura con le mani chiuse intorno alle cinghie dello zaino che aveva
sulle spalle. Dietro di lei una serie di torri di pietra nera si levava
dalla cresta successiva come le punte sulla schiena di un drago. «Ho
percorso la vecchia strada di pietra con una carovana da Mattar a Porto
Shiell.» Modificò l'immagine in modo da mostrare una serie di carri
trainati da buoi. «Non riesco a credere che voialtri costringiate ancora
gli animali a fare cose del genere.»

«È per questo che sono stati addomesticati sulla Terra» replicai,
accantonando la frecciata. «Deve essere stato un viaggio meraviglioso.
Si può entrare nelle torri?» Non avevo mai visto ologrammi delle torri
aliene prima di allora, e i resoconti che avevo letto da ragazzo erano
stati perlopiù apocrifi.

«Oh, no» replicò Valka. «Sono più obelischi che torri. I Cavaraad
trascinavano quelle pietre massicce dalle terre basse al costone.»
Nell'ologramma successivo vidi uno dei Giganti. Doveva essere alto
trenta piedi, con la carne grigia come argilla bagnata e la faccia che
era una fossa nera priva di lineamenti. Non avevo idea di come facesse a
vederci avendo come faccia solo quel buco aperto, poi l'ologramma si
mosse e l'enorme creatura avanzò con passo pesante nella foresta
fungina, facendo apparire piccoli quei funghi grossi come alberi. A quel
punto il riproduttore prese a passare da un'immagine del viaggio di
Valka alla successiva, proponendo un primo piano di una delle torri. Era
un pezzo di notte, nera come la pietra della mia casa, tanto scura che
non riuscivo a distinguerne nessuna caratteristica. Questo mi disturbò,
ma c'era un altro pensiero che premeva per essere formulato.

«Per quanto mi piaccia tor Filemone» cominciai, prendendo l'argomento
alla lontana. «Non sono certo che la sua ipotesi spieghi tutto. Se
Calagah è davvero così antica come dici, allora c'è qualcosa... qualcosa
di molto sbagliato...» Stavo per dire `negli Umandh', ma avevo appena
registrato un fatto ovvio, e quando ripresi a parlare lo feci con una
voce da cui era stata spremuta fuori ogni emozione: piatta, priva di
vita e piena di paura: «Non sono stati loro a costruire Calagah, vero?»

L'espressione di Valka era del tutto indecifrabile. Mi piacerebbe
adularmi da solo dicendo di averla sorpresa, ma lei è sempre rimasta in
parte un mistero per me. Sembrava molto distante, come se si stesse
concentrando su un rumore in una stanza lontana. Infine si riscosse e
allungò la mano verso il vino mentre scuoteva il capo. «Non lo credo.»

«Non lo credi?» la incalzai.

«No» ribadì, con maggiore fermezza. «Gli Umandh sono usciti dal grembo
dell'evoluzione solo da un mezzo milione di anni circa.»

Il mio problema relativo alle Torri che Marciano mi strattonò
all'improvviso come un mantello che si fosse impigliato in un angolo
appuntito della mia mente e mi spinse a sedere più eretto. «È tutto lo
stesso!»

A quel punto Valka si mostrò sorpresa. «Cosa?»

«Le Torri, Calagah... e scommetto perfino il Tempio di Athten Var su
Giudecca. Anche quello dovrebbe essere di pietra nera.» Temporeggiai,
poi puntai un dito verso di lei. «Tu sei stata ad Athten Var! Non studi
gli Umandh o i Cavaraad, ma...» Mi interruppi, guardando
\emph{attraverso} l'immagine di una delle Torri che Marciano di Sadal
Suud, che si ergeva piena di sfida sullo sfondo di un'alba binaria.
Momentaneamente inconsapevole dell'eresia che stavo pronunciando,
sussurrai: «È tutta una stessa cultura, vero? Una sola specie.» Sedevo
in preda a una sorta di fuga, di trasporto religioso. Si dice che i
nostri antenati guardarono in alto dalla faccia della Terra e scrutarono
il Buio, chiedendosi se eravamo soli nell'universo. Non lo eravamo, ma
questo... Questo.

Non eravamo i primi.

Quel basilare fondamento della fede della Cappellania era... ed è... una
menzogna. In quel momento mi sentii rimpicciolire, sentii il mio mondo
che si restringeva fino a diventare più minuscolo di un atomo,
schiacciato sotto il peso di tutto quello spazio e tempo, con l'umanità
che rimpiccioliva insieme a me, con tutti i suoi orgogliosi re e
imperatori, tutti i suoi guerrieri, i grandi poeti e artisti, i
contadini e i marinai, con le sue grandi imprese e ancor più grandi
atrocità. Tutto svanito nel contesto fornito da quella conclusione, e fu
tutto confermato e reso per me immutabile, una legge naturale, dalla
parola con cui Valka lo consacrò: «Sì.»

Sì.

La Cappellania lo sapeva. Doveva saperlo, altrimenti perché controllare
così tanto le nostre parole, i nostri pensieri, nel nome della nostra
anima? Questo fatto minacciava le fondamenta stesse delle loro credenze,
del loro potere, quindi accentuavano la presa, limitavano i viaggi fuori
dai pianeti e l'accesso alla sfera dei dati. \emph{L'accesso alla sfera
	dei dati.} Il terminale di Valka era scollegato dalla rete del
pianeta... buio e non monitorato, ma quel pensiero mi fece defluire il
sangue dal corpo. Stavamo parlando di eresia.

Immaginai l'Inquisizione che ci tagliava la lingua, vidi fronti tatuate
e marchiate. Immaginai uomini e donne che sedevano ciechi e incurvati
sotto i ponti con la mano protesa a chiedere l'elemosina e la parola
\foreignlanguage{italian}{eretico} che spiccava grande e scura sulla
loro pelle dalle guance devastate dall'agonia dell'essere ciechi e muti.
Quante volte avevo visto persone del genere a Meidua e a Borosevo, con
la loro verità estirpata dai coltelli dei cathar e trasformata per
sempre in dicerie e folklore sussurrati fra la folla? Troppo spesso.
Serrai la bocca, consapevole che ormai il danno era fatto, che Valka e
io saremmo stati interrogati entrambi e che quantomeno io sarei andato
ad aggiungermi a quei gusci umani mutilati seduti sui gradini del tempio
cittadino dove una volta avevo mendicato insieme a Cat... Quasi mi
aspettavo che guardie armate facessero irruzione da un momento
all'altro.

Non successe nulla e Valka parve non essere minimamente preoccupata. Di
certo comprendeva quello che stavamo dicendo! Doveva sapere quello che
sarebbe successo. Lei però si limitò a sorridermi e a bere il suo
costosissimo vino.

Era tutta una follia, ma sentii comunque una voce -- la mia -- che
continuava a parlare. «Come li chiami?» Abbozzai un debole gesto e quasi
rovesciai il mio vino. «I costruttori?»

Valka guardò verso l'alto, come se stesse controllando qualcosa scritto
sotto l'arco del suo cranio. «\emph{Ke kuchya mnousseir}.»

«La... Tomba?»

«La Quiete» mi corresse. «In realtà non dovrei dirti queste cose.» Il
suo tono cambiò, facendosi quasi legnoso, come se avesse voluto togliere
di mezzo quelle parole. «Gli appartamenti diplomatici non sono
monitorati ufficialmente, ma non mi fido di nessuno dei vostri
burocrati.»

Di norma avrei obiettato. Quel palazzo era la sede di un palatinato e
\emph{ogni} stanza era monitorata. Ognuna. Quando sorse il giorno
successivo e quello dopo ancora senza che nessuno di noi due venisse
torturato o tormentato decisi che Valka aveva avuto ragione, ma in quel
momento il mio desiderio di sapere -- quello stesso impulso che mi aveva
sospinto nelle prigioni del colosseo e nelle mani di Balian Mataro -- mi
tenne inchiodato al posto su cui sedevo come una farfalla sulla sua
montatura.

«Perché li chiami così? La Quiete?» Mi morsi un labbro e guardai
nervosamente la porta. «È un po'... drammatico, non trovi?» \emph{E io
	ne so qualcosa, eh?}

«Perché in tutti i loro siti... qui, su Sadal Suud, su Giudecca, su
Rubicon, su Ozymandias, su Malkuth e da tutte le altre parti... non c'è
niente.»

«Prego?»

«Niente attrezzi, navi, corpi, manufatti di qualsiasi tipo.» Per tutto
il tempo in cui parlò lo sguardo di quegli occhi dorati non lasciò mai
il mio volto. In qualsiasi altro momento avrei potuto godere di quel
contatto, ma in quella situazione servì solo a raggelarmi. «Ci sono solo
gli edifici, che sono muti. Quieti.»

Questo fece abbattere su di noi un'altra ondata di silenzio e rimanemmo
lì seduti mentre io elaboravo l'informazione. Qualcosa che lei aveva
detto si incastrò al suo posto, inducendomi a chiedere: «Niente corpi?
Dici sul serio?» Raccolsi di nuovo il bicchiere con il vino. «Com'è
possibile? Dove sono andati?»

Valka eseguì la scrollata di spalle meno eloquente che abbia mai visto,
di per sé un risultato notevole. «Non ne ho idea. Questo rende molto più
facile il compito della Cappellania, non trovi?»

«Sospetti che abbiano portato via tutto? Che abbiano saccheggiato i
siti?»

«Cosa?» I suoi occhi si dilatarono come piatti da portata. «No! Sarebbe
impossibile.» Quegli occhi così luminosi si socchiusero. «La tua
Cappellania è composta da uomini, M Gibson, non da dèi.»

Dovetti frenarmi per non digrignare i denti. «Non è la mia Cappellania.»

Dalle labbra le sfuggì un piccolo suono, il seme di una risata. «Se lo
dici tu, \emph{barbaro}.» Sollevai lo sguardo su di lei per ribattere in
tono mordace ma vidi che i suoi occhi sorridevano, e solo allora mi resi
conto che aveva pronunciato quella parola in tono morbido, quasi
provocatorio.

Questo mi colse di sorpresa al punto che balbettai nel portare avanti il
discorso. «Ma di certo i corpi non possono essere scomparsi. Devono aver
fatto qualcosa con i loro morti. Devono aver lasciato...»

«Non hanno lasciato niente» concluse per me Valka, scrollando ancora le
spalle. «Proprio niente, M Gibson. Solo le costruzioni.»

Mi accigliai e aprii la bocca una, due, parecchie volte, accompagnando
con ogni gesto il passaggio del suo terminale da un'immagine alla
successiva. «Solo le incisioni di cui ti ho parlato, quelle emulate
dagli Umandh con i loro nodi che raccontano storie.» Con un altro gesto
evocò proiezioni degli oggetti in questione, anche se li ricordavo
abbastanza bene e ne avevo perfino disegnato uno sul mio diario.
«Naturalmente, sono del tutto illeggibili.»

Mi assalì l'avvilimento. «Nessuno sa come leggerli? Neppure fuori
dall'Impero?»

«Non che io abbia sentito. Qualche scoliasta ribelle ci ha provato, ma
senza un'idea di come quei simboli si rapportino al linguaggio parlato o
se lo facciano...» Si interruppe.

«È un problema da Stele di Rosetta» affermai, e quando lei inarcò le
sopracciglia con curiosità, le spiegai che sulla Vecchia Terra c'era
stato un tempo un popolo la cui scrittura era illeggibile, almeno fino a
quando non venne trovato un monumento che mostrava quell'antica
scrittura affiancata da altre due lingue note del tempo. Era stata una
chiave che aveva aperto la decifrazione di tutti gli scritti di
quell'Impero perduto. «E la parte più strana,» aggiunsi, farfugliando
per mascherare la massa di emozioni che mi ribolliva nello stomaco «è
stata che quei geroglifici non erano affatto ideogrammi ma un sistema di
logogrammi combinati con elementi alfabetici... cosa c'è?»

Valka mi stava sorridendo, un vero sorriso e non limitato agli occhi.
Durò solo un momento, poi collassò sotto il peso del mio sguardo e lei
scosse il capo. «Niente.»

«Ma perché venire qui?» Abbozzai un gesto vago in direzione degli
ologrammi scintillanti, della facciata di pietra di Calagah, nera e
liscia come vetro. «Ci deve essere una dozzina di siti al di fuori dello
spazio imperiale dove la Cappellania non ha potere.»

Finalmente mi rispose. «Non che non siano controllati dagli
Extrasolari.» Mi puntò contro un dito mentre continuava con voce resa
impastata dal vino di Markarian. «I soli barbari dell'universo che siano
peggiori di voialtri.» Sbuffò, ma non seppi dire se si trattava di
derisione o di una risata.

«Noialtri?» sapevo cosa voleva dire, ma questo mi aiutava a prendere
personalmente le distanze. Fuori dalle strette finestre un ornithon
sibilava in direzione del sole che tramontava.

«L'Impero Solare.»

Ritrassi le labbra sui denti, e una di esse si impigliò per un attimo in
un incisivo scheggiato, memento di un incontro nel Colosso. «Non so
niente degli Extrasolari.» Non sapevo bene come portare avanti quel
filone della conversazione, perciò cambiai argomento. «Quindi questi...
questa Quiete è l'oggetto dei tuoi studi? Non gli Umandh?»

La dottoressa Onderra bevve un sorso più misurato, oramai consapevole
del valore di quel vino, e annuì con insistenza mentre allontanava i
capelli dalla fronte alta con le dita che indugiavano tra i riccioli fra
il rosso e il nero. «È così.»

«Non riesco ancora a credere che questo sia un segreto, che sia vero.»

Valka sbuffò. «I vostri grandi Casati controllano le informazioni.
Controllano l'accesso alla sfera-dati e restringono a pochi individui i
voli extraplanetari, permettendo alla Cappellania di metterli sotto i
piedi di continuo. Non potevi saperlo.»

«Noi... i grandi Casati non \emph{permettono} alla Cappellania di
calpestarli, lo possono semplicemente fare. L'Inquisizione preferirebbe
distruggere un pianeta piuttosto che lasciarlo cedere all'eresia, e ne
ha i mezzi. Pestilenze, armi atomiche, le armi che i Mericanii si sono
lasciati alle spalle. Sono cose che distruggono i pianeti, dottoressa.»

«Eresia...» Valka sbuffò di nuovo, un suono assai poco signorile.

«La verità è tradimento, e tutto il resto.» Agitai una mano, cercando di
ricordare la fonte di quella citazione. Gibson lo avrebbe saputo, ma lui
non c'era più.

Il volto di lei si compose in una maschera solenne. «Peccato.»

«Prego?»

«Voi Solari avete fatto del crimine e del peccato la stessa cosa. Non
potreste vedere la verità neanche se vi danzasse davanti nuda.»

«Io potrei» scattai, pieno di sfida, con una traccia della mia antica
altezzosità di palatino -- il comportamento di Hadrian Marlowe e non di
Hadrian Gibson -- che tendeva ad affiorare. Accennai agli ologrammi che
scintillavano ancora proditoriamente nell'aria al di sopra del tavolino.

Valka si alzò, lasciando il bicchiere vuoto sul tavolino in mezzo a noi.
«In effetti forse potresti.»

Accantonai per un momento il tatto, chiedendo: «Perché non ti hanno
uccisa?»

Lei mi guardò da sopra la spalla. «Prego?»

«Tutto questo!» Agitai una mano in direzione degli ologrammi. «Sai tutto
questo. Non riesco a credere che ti permettano di... di andartene in
giro. Di respirare aria imperiale.»

«Credi che possediate l'aria?» Il suo accento si fece più marcato nel
proferire quella domanda, facendola apparire in qualche modo più strana,
più straniera.

Accantonai quelle parole con la massima energia possibile. «La
Cappellania non può volere che questo venga diffuso. Se sapessero che me
lo hai detto...» Ma cosa stavo dicendo? Era ovvio che lo sapevano.
«Saremmo morti entrambi. Peggio che morti.» Dovetti impedirmi di
sprofondare in una descrizione di cosa significasse esattamente quel
`peggio che morti'.

Valka ripeté quello strano gesto di fissare il soffitto, poi si grattò
la nuca esalando un pesante respiro e infine attraversò lo spazio che ci
divideva, posandomi una mano su un lato del collo. Sussultai, ma lei
rilassò la presa fin troppo in fretta. «Non lo scopriranno, Hadrian.
Tranquillizzati.» Sorrise, ma in quel modo che le persone adulte
riservano ai bambini che non sanno di cosa parlano. «Sono qui da anni, e
ho detto tutto a Elomas proprio in questa stanza.» Il suo sorriso si
trasformò e da legnoso divenne beffardo come luce lunare. «E lui sta
benissimo, giusto?»


