\chapter{Anaïs e Dorian}

Le mie camere mi si adattavano come denti nuovi e gli abiti come una
vecchia pelle di serpente. Camminai ansiosamente da una stanza
all'altra, aggirandomi come un lupo in cerca di una via di fuga. Le
camere erano fra le più belle che avessi mai visto, avevo un bagno
privato, un armadio equipaggiato con vestiario formale fatto a macchina
per me, un mio salotto con divani di pelle e perfino una mia collezione
di vini decenti. Alle pareti erano appesi dipinti a olio di navi, sia
vascelli che astronavi, e tende di velluto scuro come il peccato
coprivano le finestre che si affacciavano sulla cupola della Cappellania
e la sua bastiglia di cemento.

Il letto a baldacchino, con il materasso smart e le lenzuola di lino,
sembrava quasi palpeggiarmi con il suo lusso e dopo due notti mi arresi,
dormendo invece in un groviglio sul pavimento. Ho superato da molto
tempo queste tendenze spartane, ma avendo lasciato da così poco gli
alloggiamenti del colosseo e le strade, trovavo difficile abituarmi. La
cosa peggiore erano i servitori perché mi sembrava strano non dover fare
tutto da solo. Era sbagliato.

Quella era solo un'altra gabbia dorata e qui ero un prigioniero più di
come lo fossi mai stato su Emesh. Avevo commesso un errore. A causa
della mia curiosità e del folle desiderio di incontrare quel Cielcin
prigioniero avevo barattato la mia libertà, il mio sogno di viaggiare
fra le stelle e i miei amici. I miei amici. Cosa dovevano aver pensato
del fatto che fossi scomparso? Avrebbero supposto -- o lo avrebbe fatto
Switch -- che fossi andato a cercare il Cielcin nelle galere
dell'ipogeo, ma sarebbe parso che li avessi {abbandonati}. Se solo
avessi potuto far arrivare un messaggio dall'altro lato della piazza e
al colosseo... ma non mi era permesso di mandarne e rabbrividii al
pensiero di cosa avrebbero fatto gli uomini del conte se ci avessi
provato. Ogni mia mossa veniva osservata, quindi feci quello che mi era
stato insegnato di fare se ero sotto osservazione. Mi comportai come si
aspettavano che facessi.

Ero riuscito a ottenere dal ciambellano del conte nuovo materiale per
disegnare, e stavo finendo una cena solitaria e tracciando dalla
finestra uno schizzo del panorama di Borosevo quando bussarono alla
porta. Un solo colpo, che ignorai nella speranza che si trattasse di
qualche cortigiano in vena di chiacchiere il quale, dopo aver bussato
due o tre volte senza risposta, si sarebbe annoiato. Bussarono ancora, e
dentro di me fui lieto di non aver lasciato la piastra olografica in
funzione sul notiziario serale perché il rumore avrebbe tradito la mia
presenza.

La porta chiusa a chiave si aprì e un paio di peltasti nell'armatura
verde dei Mataro varcarono la soglia, come se le videocamere nascoste
non avessero già rivelato tutti i segreti nascosti delle mie stanze. Uno
prese posizione vicino alla porta, l'altro esaminò la mia persona,
rimuovendo il coltello a serramanico che avevo comprato con il denaro
guadagnato nel Colosso e -- stranamente -- il mio kit di matite. Lo posò
sul tavolo, vicino alla porta e al suo compagno, poi uscì nel corridoio.
Un momento più tardi una giovane donna decisamente avvenente entrò nella
stanza, precedendo un giovane alquanto agitato che poteva essere
soltanto suo fratello.

Prima ancora che aprisse bocca sapevo esattamente chi fossero lei e il
giovane nobile: erano Anaïs e Dorian Mataro, gli eredi del conte.
Entrambi avevano la pelle più chiara di quella di lord Balian, la cui
carnagione nera come l'inchiostro era temperata da quella dorata
dell'altro padre fino a ottenere una tonalità ramata come quella del
legno oleato. Avevano i suoi stessi occhi a mandorla, quelli di lei
verdi e quelli di Dorian neri, gli stessi capelli scuri, la stessa
corporatura robusta e lo stesso vestiario di fine seta.

Già in piedi a causa della perquisizione improvvisata, nel vederli mi
ersi un po' di più sulla persona, poi ricordai che mi sarei dovuto
inchinare perché al momento ero di un rango decisamente inferiore a
quello di un palatino. «Lord e lady, voi mi onorate» dissi nel
raddrizzarmi.

La ragazza mi porse la mano, sul cui mignolo spiccava una fascia di
avorio su cui splendeva un pallido berillio. Presi la mano e baciai
l'anello, com'era corretto fare, e intanto lei commentò: «Allora sei tu
quello di cui ci hanno parlato?» Mi squadrò da testa a piedi con la
testa inclinata da un lato. «Sei più basso di quanto mi aspettassi.»

Il giovane alle sue spalle si lisciò i capelli oliati con fare
decisamente ansioso. «Non essere scortese, sorella!» Poi protese la mano
e strinse la mia con calore. «Sono Dorian Mataro. A quanto mi è dato di
capire dobbiamo diventare amici.» E sorrise in quel modo leggermente
ironico tipico di tutti i bambini quando fanno il verso a qualcosa che i
genitori hanno detto loro. Era strano che pensassi a loro come a dei
bambini anche se non erano molto più giovani di me. Avevo vissuto tanto
a lungo fra i plebei che non trovai strano che il figlio di un palatino
si degnasse di toccarmi la mano.

«Io sono Hadrian» risposi, fingendo un caloroso sorriso che mi lisciò il
volto come olio. Avevo dimenticato quanta parte della vita di corte
fosse così. Maschere invisibili. Alla Corte delle Lune di Jadd, se non
altro i cortigiani hanno la decenza di indossare vere maschere.
Ricordando la mia posizione, aggiunsi: «Hadrian Gibson. È un piacere
conoscere Vostra signoria.»

«Chiamami Dorian, per favore!» Il suo sorriso ironico si accentuò in
qualcosa di più aperto e sincero. «E... e questa è mia sorella...»

«Anaïs» intervenne lei, interrompendolo. Mi girai a guardarla e sorpresi
quegli occhi simili a smeraldi dal taglio a stella che scrutavano le
linee del mio volto. Cosa cercavano? Offrii a quella ragazza -- non
poteva avere più di diciannove anni -- un esitante sorriso. «Benvenuto a
Borosevo, M Hadrian.»

Brevi immagini si formarono rapide dietro il mio sorriso. Cat che
tossiva nel caditoio, un coltello affondato nella spalla di una
commessa, la risata crudele degli altri ragazzi, quella notte. Il
colosseo e gli amici che erano ancora là, quelli che vi avevo
abbandonato. Altro che `benvenuto a Borosevo'. \emph{Oh, Vostra
	signoria, conosco Borosevo in modi che tu non sperimenterai mai}. Pregai
che nessuno di quei pensieri avesse trapassato la mia maschera
sorridente e chinai il capo. «Sei troppo gentile. Io...» Mi interruppi,
guardando verso l'oplite senza volto fermo appena oltre la porta aperta
della stanza. «A cosa devo l'onore di questa visita?»

Lord Dorian si sedette sulla poltrona che avevo da poco lasciato libera,
ma fu sua sorella, ancora assorta nello studio del mio volto, a
rispondere. «Naturalmente volevamo conoscerti. Nostro padre dice che hai
viaggiato ovunque, che parli lo jaddiano, il lothriano e...»

«Eri davvero un gladiatore?» interloquì Dorian, mentre sua sorella si
sistemava su un bracciolo della poltrona. «Sul serio?»

Mi fissai i piedi, in qualche modo incapace di guardare in faccia la
ragazza. Ero rimasto lontano dalla mia razza per tanto tempo che la sua
vista aveva infranto qualcosa dentro di me. Il suo volto -- tutto in lei
-- era un'opera d'arte. Dire una cosa del genere in realtà non è un
complimento, perché non aveva avuto altra scelta che quella di essere
perfetta. Non mi sono mai considerato particolarmente avvenente, con il
mio volto affilato e severo, ma una volta Valka ha detto che avevo il
genere di faccia che si vedeva scolpita nel marmo, e nel ripensare alle
statue nella nostra necropoli non posso dissentire. Io però ero come uno
dei miei schizzi a matita se paragonato con il dipinto a olio che era
Anaïs Mataro.

Anaïs Mataro. Radiosa come una statua di bronzo, cupa come una sera
d'estate, la personificazione dell'icona stessa della Bellezza. Ed era
anche un'opportunista fredda e calcolatrice. A quel tempo non lo sapevo,
quindi sorrisi, sfoggiando il volto che il conte mi aveva chiesto di
mostrare loro. «Ero un mirmidone, Vostra signoria. Uno dei combattenti
sacrificabili.»

«Un combattente dell'arena?» Inarcò un sopracciglio.

«Sai, signore, mi ricordo di te!» la interruppe con irruenza suo
fratello, raggiante. «Sei quello che ha mandato in corto lo scudo
dell'avversario usando la sabbia!» Dorian passò una gamba sull'altro
bracciolo della poltrona, sedendo con la noncurante disinvoltura di...
di qualcuno. Non riuscivo a ricordare chi.

Con le mani intrecciate davanti a me annuii in direzione degli stivali,
sinceramente in imbarazzo. Non mi ero abituato alla mia piccola fama
locale. Avevo trascorso la maggior parte del mio tempo come mirmidone
rintanato nell'ipogeo del colosseo, tranne che per le poche uscite in
cerca di qualcosa da bere, di donne o di un'astronave, in ordine di
successo calante. «Ero io, Vostra signoria.»

«È stato uno spettacolo dannatamente buono, M Hadrian.» Dorian sorrise.
I suoi denti erano molto bianchi.

«Sei gentile a dirlo» risposi, con tanta cortesia da avere la sensazione
che avrebbe potuto farmi marcire i denti, poi mi aggrappai a qualcosa
che il conte aveva detto durante il nostro incontro, e chiesi: «A quanto
mi è dato di capire stai studiando lo jaddiano?»

«\emph{Soli qalil}» rispose Dorian, accompagnando le parole cantilenanti
con un sorriso.

«\emph{Qalilla»} replicai, incurvando la bocca in quel sorriso in
tralice dei Marlowe da tempo assente. «\emph{Qalil} significa `piccolo',
mentre \emph{qalilla} vuol dire `un poco'.» Pronunciare quella seconda
`l' fu difficile dopo tanti anni di disuso, perché era prodotta da un
movimento della lingua alieno per il galstani.

Anche se era rischioso correggere un palatino, il sorriso del giovane
Dorian -- giovane? Aveva appena un paio d'anni meno di me -- si
accentuò. Aveva commesso quell'errore di proposito? Era una prova?
Mantenni sul mio volto il sorriso in tralice. «Sì, hai ragione» affermò,
giocherellando con un filo tirato della cucitura della poltrona, poi
aggiunse: «Sei mai stato su Jadd, signore?»

Sbattei le palpebre, mentre vagliavo mentalmente un migliaio di versioni
della mia risposta. Finta e controfinta. Cos'era che aveva detto il
vecchio filosofo? Proprio come in un duello? No, era meglio la verità.
«No, Vostra signoria. Ho avuto il privilegio di avere come tutore uno
scoliasta quando ero con la compagnia di mio padre.»

«Ed è chiaro che hai avuto anche un maestro d'armi» aggiunse Anaïs,
squadrandomi in un modo che... mi turbò. Come se fossi stato un campione
su un vetrino. No, più come un boccone su un piatto. Per un momento mi
agitai sul posto, comprendendo come doveva essersi sentita Kyra. L'alta
ragazza si appoggiò al lato della mia poltrona, studiando con quei suoi
occhi luminosi i resti della mia cena: il piatto sporco, il bicchiere
d'acqua svuotato a metà con accanto la bottiglia, il vassoio che
sporgeva appena oltre l'orlo del tavolo. Il suo sguardo si posò sul mio
libro di disegno aperto e la vidi illuminarsi in viso. «Disegni?»

Senza attendere che le dessi il permesso, tirò su il foglio di carta
economica che mi aveva dato il ciambellano, facendo cadere rumorosamente
a terra le matite. Serrai i denti per ricacciare indietro una richiesta
inopportuna. Il cortigiano Hadrian Gibson, figlio di un mercante,
impiegò parecchio tempo a soffocare l'ululato offeso del Marlowe. Potevo
sentire i muscoli della faccia che mi si {contraevano} per un misto di
rabbia e di indignazione mentre allo stesso tempo un qualche riflesso
autonomo cancellava quei sentimenti e mi rendeva sottomesso come i
servitori che avevano corredato la mia infanzia. «Sì, Vostra signoria. È
un piccolo hobby.»

Mi soffermai dietro di lei, guardando intorno alla sua spalla -- ma non
da sopra di essa -- il paesaggio incompleto. Avevo ritratto Borosevo con
mano pesante, enfatizzando le ombre proiettate dal sole torrido che in
quel momento era acquattato sull'orizzonte. La città era tutta bassi
edifici, tranne che per le dita di vetro delle fattorie urbane e la
vecchia torre della radio, e le case si allargavano sul panorama con un
aspetto frammentato e irregolare. Non era un disegno lusinghiero.

«È splendido!» dichiarò, girandosi per guardarmi leggermente dall'alto.

«Vostra signoria è molto gentile,» replicai, con un sorriso legnoso «ma
temo che sia una misera immagine, paragonata alla bellezza della vostra
città.»

«Sciocchezze! Hai colto alla perfezione la nostra orgogliosa città.»
Orgogliosa, come no! Chi ha detto che l'orgoglio è cieco? Oppure si
trattava dell'amore? «Che talento! Non trovi che abbia talento, Dorian?»

Suo fratello allungò il collo per guardare. «Sai ritrarre le persone, M
Hadrian?» Ti stupirà sapere con quanta frequenza questa sia la prima
domanda che riceve chiunque abbia anche solo un brandello di talento
artistico, seguita in breve da: «Potresti ritrarmi? Me o... mia
sorella?» La indicò, con un sorriso di aspettativa alquanto blando.
Quando non risposi immediatamente accantonò la cosa con un gesto. «Non
ora, ovviamente. Me lo stavo solo chiedendo.»

«Sarei felice di farlo. Vostra signoria» Il mio sorriso cominciava a
darmi la sensazione di essere un taglio legnoso sulla faccia ed ero
certo che avrei sempre e comunque preferito la rozza derisione di Ghen a
questo nauseante e cortese esigere. «È solo che ho avuto una lunga
giornata, Vostre signorie. Abituarmi di nuovo alla vita di corte dopo
tanto tempo nell'arena...» Lasciai la frase in sospeso, facendo
affidamento sulla loro capacità di cogliere quello che non stavo
dicendo. Non è mai appropriato fare una richiesta a un palatino, e così
tanta parte della conversazione di corte è fatta di {sottintesi}, di due
parti che parlano una intorno all'altra, come due nobili profumati
impegnati in una contraddanza.

Non recepirono il messaggio, oppure non gli importò di farlo.

«Allora attenderò con anticipazione» disse Dorian, lisciandosi gli scuri
capelli oliati. «Magari più avanti nella settimana, il che mi ricorda
perché siamo qui.» Sollevò lo sguardo sulla sorella, esponendo la linea
di barba stupidamente sottile che sfoggiava per accentuare la linea
della mascella. Era chiaro che su Emesh i palatini non si facevano
rimuovere la barba, e la sua sembrava una linea tracciata a matita lungo
la mascella, cosa che mi costrinse a reprimere un piccolo sorriso.
Quella barba appariva ridicola, la parodia della virilità in un ragazzo.
Non riuscii a capire se stesse rimettendo a lei di esporre il motivo
della visita perché si riteneva un suo subordinato o se lo stesse
facendo per il motivo opposto. Le loro facce dai geni perfetti erano
molto più difficili da decifrare dopo la successione di rudi volti
plebei a cui mi ero abituato da quando ero giunto su Emesh, e i
meccanismi interni delle loro dinamiche di potere erano un mistero che
esulava del tutto dalla mia comprensione.

Anaïs agitò i ricciuti capelli neri. «Sì, è vero!» Riportò la sua
attenzione su di me con occhi luminosi che mi catturarono come due fari
e mi inchiodarono come il telemetro del mirino di un cecchino.
«Speravamo di invitarti a uscire, di farti conoscere qualcuno dei nostri
amici, il resto della corte.»

«Sarebbe davvero bello» risposi, chinando il capo, e mi augurai di aver
centrato la giusta nota di meravigliato entusiasmo. Supposi si trattasse
di una gentilezza, o di parte di un piano contorto che non riuscivo
ancora a scorgere. Non ero certo che il conte mi stesse tenendo lì solo
per il mio talento con le lingue, proprio come non ero certo che si
fosse bevuto la mia affermazione che avrei potuto proteggerlo dalla
Cappellania. Ero acutamente consapevole di essere di nuovo su un mondo
di ruote e di spirali, un posto dove niente era dritto. Contraddanza e
controfinte. Un tempo avevo primeggiato in simili acque, adesso non ero
neppure sicuro di avere le branchie.

«Pensavamo di prendere a prestito il palco di nostro padre al colosseo»
suggerì Dorian, allargando le mani. «Potresti raccontarci le tue storie
mentre assistiamo ai combattimenti!» Sorrise di nuovo e fu allora che
realizzai dove avevo visto il modo in cui aveva gettato la gamba sul
bracciolo della poltrona, quel gesto di possesso sicuro e dominante.
Crispin. Il ragazzo si muoveva con lo stesso noncurante abbandono del
mio fratello minore, baldanzoso perché possedeva tutto il suo mondo. Ben
presto appresi però che c'era una piccola differenza fra quel ragazzo e
mio fratello, perché la gioia di Dorian era sincera. Lui amava il
Colosso per il suo aspetto sportivo, non per il sangue.

«Come desidera Vostra signoria» replicai, con un inchino che servì a
mascherare la rigidità delle mie parole. Cercai di immaginare come
sarebbe stato sedere dietro il campo protettivo e in quel palco dorato,
sotto una tenda di seta, mentre gli uomini e le donne con cui avevo
passato quell'ultimo anno lottavano e morivano nell'arena di sabbia e
mattoni. Provai una fitta al cuore, desiderando di essere rimasto
dov'ero con Ghen, Pallino, Siran, Elara e gli altri. E non avevo
aggiustato le cose con Switch. Nulla di tutto questo mi trasparì però
dal volto, e quando mi raddrizzai la mia espressione era una maschera
perfetta. «Ne sarò felice.»

Ero davvero di nuovo fra i palatini.


