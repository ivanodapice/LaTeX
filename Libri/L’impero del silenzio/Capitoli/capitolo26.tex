\chapter{Cat}

La pioggia cadeva in fitte cortine dai tetti inclinati e sui canali di
scolo intasati per riversarsi nei canali ormai talmente pieni da
inghiottire le strade. Percorsi sciacquettando un passaggio
sopraelevato, lieto di quell'acqua fresca nonostante la tempesta. Intere
sezioni di Borosevo -- le più povere -- finivano sott'acqua ogni volta
che arrivava una di quelle tempeste. Sotto i lampi che solcavano il
ventre del cielo livido, carambolando da una nuvola alla successiva, mi
appoggiai alla ringhiera di plastica e mi allontanai dagli occhi i
capelli spinti dalle folate di vento.

Avevo bisogno di un riparo, di cibo, di smetterla di soffrire.

I lampioni si spensero, come pure l'oscillante catena di luci sospesa
sul ponte mediante cavi di alimentazione, e l'oscurità si stese sulla
strada sopraelevata mentre riprendevo a camminare con i piedi nudi che
strisciavano sul cemento logoro. Mi attardai per un momento sul lato
sottovento di un negozio di alimentari chiuso e presi in considerazione
l'idea di rompere la vetrina. Era improbabile che i prefetti si
arrischiassero a intervenire in mezzo a un uragano, neppure per un furto
con scasso, ma... no, meglio di no.

Al di là delle forme scure dei bassi edifici e la distesa della città,
un fulmine si abbatté sul mare, trasformando tutta quella oscurità in
vetro lucente, poi un tuono mi scosse fino al midollo, risuonando sopra
Borosevo come un'astronave in fase di discesa. Un tendone a strisce
schioccava sopra la mia testa, con la pioggia che rimbalzava contro di
esso come il rullare di un migliaio di minuscoli tamburi. Spinto
dall'istinto mi raggomitolai sulla soglia nella speranza di {attendere}
lì che la tempesta si esaurisse. Dalla mia posizione rialzata potevo
distinguere la massa incombente dello ziggurat del palazzo che si ergeva
sopra la città, con la sua forma nera accoccolata sopra Borosevo come un
drago sul suo tesoro. Le luci delle alte torri tremolavano, segno che
perfino il sistema di alimentazione del conte aveva dei problemi sotto
l'impatto della furia di Emesh.

Su Delos avevamo delle tempeste, portate attraverso il mare dalle coste
orientali sulle ali dello scirocco, ma non erano niente, \emph{niente},
al confronto di quelle di Emesh. Nuvole vaste come imperi torreggiavano
sulla città, riempiendo il cielo e seppellendo tutte le stelle.
Nonostante la calura e l'aria intrisa di vapore, stavo tremando. Una
luce si accese alle mie spalle e un uomo rosso in volto calò il pugno
sul vetro, gridando parole ovattate. Cogliendo il messaggio mi alzai
goffamente, barcollando. Quanto era passato dall'ultima volta che avevo
mangiato?

La pioggia sferzava il cemento e rimbalzava contro le vetrine e sui teli
catramati che proteggevano le barche in via di riparazione che
dondolavano sulle onde. Mi avviai a passo spedito, infilandomi in un
vicolo nella speranza di trovare un'area di carico rimasta aperta per
errore. Tuttavia gli abitanti di Borosevo erano tutti diligenti, e
abituati alle tempeste, per cui mi ritrovai a vagare a vuoto. Vecchi
rifiuti mi aderirono alla pianta dei piedi nudi mentre proseguivo con
passo strascicato fino ad appoggiarmi alla parete di latta di una
baracca, a un isolato dalla strada principale, serrando attraverso la
camicia fradicia l'anello appeso alla sua catena come un mago avrebbe
potuto fare con un talismano.

Da bambino avevo desiderato l'avventura, vedere la galassia, sondare le
profondità nascoste dell'universo umano e i preziosi segreti celati
nell'oscurità fra le stelle. Avevo desiderato viaggiare come facevano
tor Simeon e Kharn Sagara nelle antiche storie, vedere le Novantanove
Meraviglie dell'Universo e spezzare il pane con xenobiti e re. Ecco,
avevo avuto la mia avventura, e mi stava uccidendo. Se non altro, gli
edifici del vicolo sporgevano in fuori sulla strada. Non era molto ma
gli aggetti creavano uno spazio di circa un metro che rimaneva asciutto,
o quantomeno più asciutto. Facendo un altro tentativo mi insinuai fra
due bidoni dei rifiuti, riparandomi il più possibile dal vento e
dall'acqua.

Perché era andato tutto storto? Cosa era successo a Demetri? Non era
giusto. Avevo fatto tutto quello che dovevo, seguendo il piano di mia
madre il più accuratamente possibile, e adesso avrei dovuto essere su
Teukros, in un monastero degli scoliasti, ad ascoltare conferenze sulla
matematica dello spazio di curvatura e sui legami diplomatici fra
l'Impero e gli Stati clienti fra i Normanni e la Repubblica Durantina.

«Cosa stai facendo?» All'inizio pensai di aver immaginato quella voce,
tanto era sommessa, un acuto sibilo coperto dal martellare della
tempesta. «Tu!» Sollevai lo sguardo, sempre più su, fino al tetto
dell'edificio che avevo di fronte, dove un piccolo volto scuro mi
sbirciava, coperto dai capelli fradici incollati alla pelle. Pensai di
sgusciare via, di svanire. Non parlavo con nessuno da settimane, da
quella volta in cui un marinaio in licenza di sbarco mi aveva dato mezzo
panino quando gli avevo chiesto un kaspum. Può sembrare strano, ma se
siete mai stati davvero e completamente soli per un qualche periodo di
tempo dovete sapere quanto sia difficile tornare nel mondo della gente.
Quindi, mi limitai a fissare quella faccia.

«Sei stupido o altro?» Quando ancora non mi mossi, lei aggiunse: «Non
hanno protetto quel vicolo con i sacchi di sabbia, \emph{rus}.
Addormentati lì e all'alba dovranno ripescarti nella laguna. Vieni su!»
Accennò con la testa a una grondaia rotta che risaliva un angolo
dell'edificio su cui si trovava.

Per poco non fuggii, e forse lo avrei fatto se fossi stato più in
salute, se non avessi avuto ancora le costole doloranti per il terzo
pestaggio subìto in altrettante settimane. Quando mi alzai, però,
un'ondata di dolore mi trafisse il fianco e mi ripiegai da un lato
contro il bidone di rifiuti più vicino, un grosso contenitore di
plastica che scivolò sul cemento fradicio e cadde da un lato con un
tonfo sordo. Imprecai, senza chiedere scusa a niente e a nessuno. Il
volto della ragazza era svanito dall'orlo dell'edificio. L'avevo
immaginata? Mi mossi verso la grondaia, le cui massicce staffe formavano
quasi una scala... solo che erano distanziate e il metallo liscio era
viscido. Scivolai solo due volte, mordendomi un labbro mentre ricadevo
nei quasi due pollici d'acqua raccolti alla base della costruzione.

Al terzo tentativo una mano piccola e forte mi afferrò per un polso.
«Cosa cazzo hai che non va?» Non era abbastanza forte per tirarmi su, ma
mi diede il tempo di rimettere in posizione i piedi e di riuscire infine
ad afferrare il bordo dell'edificio per poi issarmi su di esso. Un
velivolo ci passò sopra, con le luci di corsa accecanti nell'aria
chiusa. «Sei un qualche tipo di idiota?»

«No!» scattai, snudando i denti e lei si ritrasse, rimuovendosi la
pioggia dalla faccia improvvisamente avvelenata dalla paura. «Mi
dispiace» dissi, perdendo di colpo tutto il mio impeto. «Io...» Chinai
la testa e le scoccai un'occhiata. «Grazie per avermi aiutato a salire.»
Era più giovane di me, aveva forse sedici anni standard, con la pelle
color rame, un volto rotondo e gli occhi che sorridevano in quella sua
rozza faccia plebea nonostante la piega guardinga della mascella. I suoi
abiti più laceri e rappezzati dei miei suggerivano un corpo snello e
denutrito. Era come me, senza una casa e impotente sotto la tempesta.

Inclinò la testa da un lato. «Non sei di qui, vero?»

Scossi il capo e distolsi lo sguardo, esaminando il tetto sferzato dalla
pioggia. Una fila di ammaccati pannelli solari si stendeva lungo il
bordo opposto, dove una corda per il bucato pendeva nuda e agitata dal
vento. Un altro tuono fece tremare il mondo, eco di un qualche fulmine
invisibile. Dalla nostra posizione leggermente soprelevata la bassa
distesa di Borosevo si srotolò come un rifiuto intrappolato in un
vortice.

«Vieni da fuori il pianeta?» mi domandò.

«Sì.» Adocchiai di nuovo i pannelli solari dall'altro lato del tetto e
mi mossi in quella direzione, adagiandomi sotto uno di essi con l'aiuto
della ragazza. Il tetto era comunque bagnato, ma almeno la pioggia non
ci cadeva direttamente sulla testa.

Lei si lasciò cadere accanto a me e mi costrinse a spostarmi per farle
spazio. «Sei ferito?»

«Ho avuto dissapori con...» Stavo per dire `colore locale'. «...alcune
persone.»

Quelle parole suonarono poco convincenti perfino alle mie stesse
orecchie, e la ragazza fece una smorfia. «Hai avuto dei dissapori?»
ripeté. «Significa sì?»

Risposi con un grugnito e mi appoggiai all'indietro, abbassando la testa
dove il pannello si inclinava ad angolo per guardare verso sud. In quel
momento il tetto di cemento dava una sensazione indicibilmente gradevole
e mi sdraiai supino, senza muovermi. «Qui hai proprio un bel posto.» Non
riuscivo a immaginarne uno migliore, durante una tempesta. I pannelli
solari erano una buona copertura dalla pioggia e il tetto era pulito,
oltre a essere rialzato rispetto al livello di qualsiasi inondazione.

Con mia sorpresa lei si illuminò in volto e sorrise, rivelando denti
storti. «Sono stata fortunata... i proprietari non ci sono.» Tacque per
un momento. «Perché stavi cercando di annegare in quel modo, sulla
strada?»

«Stavo solo cercando di dormire.» Avevo gli occhi chiusi e il mio
respiro era volutamente poco profondo in reazione al dolore che avevo ai
fianchi. La ragazza non disse niente per cinque secondi. Dieci. Dopo
mezzo minuto aprii un occhio e scoprii che era lì accovacciata a
guardarmi. «Cosa c'è?»

«Non hai soldi?» chiese, con la confusione che le traspariva chiara
dalla voce. «Chi viene da fuori ne ha sempre. Potesti affittare una
stanza.» C'era speranza nella sua voce?

«Non ho niente» replicai, costringendo la mano a non serrare l'anello
attraverso la camicia. L'ultima cosa di cui avevo bisogno era di dare a
quel piccolo topo di strada una scusa per derubarmi. Dopo un altro
silenzio pervaso di disagio, domandai: «Come ti chiami?»

Questo mi fruttò un'occhiata penetrante. «E tu?»

Aprii entrambi gli occhi ma non mi sollevai a sedere. «Hadrian.»

Fece una smorfia che ancora oggi non sono in grado di descrivere. «Io
sono Cat.»

«È il diminutivo di Catherine?»

Arricciò il naso. «No! Che razza di nome è Catherine?» Strappò una
striscia di nastro isolante che correva sul tetto, accanto a lei,
esponendo un vecchio cavo che proveniva dal sovrastante pannello solare.
«Sono solo Cat.» Dopo un altro momento di silenzio ripeté: «E comunque,
che razza di nome è Hadrian?»

Scrollai le spalle. «Uno antico. Il mio.» La mia sfortunata scontrosità
bloccò di nuovo la conversazione mentre mi premevo una mano contro le
costole e sussultavo per il dolore intenso. La ragazza si mosse, con le
mani sospese sopra di me, non sapendo bene come aiutarmi. Risuonò un
tuono. Sotto di me il tetto era fradicio, ma non importava. Per una
volta ero grato dello sgradevole calore e della pesantezza dell'aria.

«Quanto è stato brutto il pestaggio?»

«Due costole rotte, credo.» Ricordando cosa era successo quella notte a
Meidua, dopo il Colosso, aggiunsi: «Ho subìto di peggio.» Solo che
questa volta non avrei ricevuto tutori medici correttivi. Gemetti. Avevo
creduto di poter derubare un gruppo di adolescenti e farla franca, ma si
erano dimostrati più forti e cattivi di quanto avrei mai creduto, ed
erano stati in dodici.

Si morse un labbro. «È una cosa per cui non posso fare niente.»

«No» convenni. Per un momento ascoltai la pioggia mentre lottavo contro
l'offuscarsi della vista dovuto al sopraggiungere del sonno. «Perché mi
hai aiutato?» Non mi sembrava una cosa furba da parte di una ragazza
sola.

Cat si erse un po' più dritta. «Non lascerei nessuno giù in strada
durante una tempesta, \emph{rus}.» Mi studiò con aria riflessiva.
«Almeno, sai nuotare?»

«Forse, se non avessi le costole rotte.»

«Mia madre diceva che alcuni stranieri vengono da mondi dove non hanno
acqua, e non sapevo...» Lasciò la frase in sospeso, giocherellando con
quel pezzetto di nastro isolante.

Le rivolsi un accenno di sorriso, sperando che servisse almeno in parte
a riparare il danno che avevo fatto guardandola male. «Sono cresciuto
vicino al mare e so nuotare, solo che...» Lasciai vagare lo sguardo sul
tetto. «Qui è tutto diverso. L'aria è sbagliata, la gravità è eccessiva,
le persone sono strane...» Sussultai, consapevole che stavo
sproloquiando.

Lei appallottolò il pezzo di nastro e lo lanciò lontano sotto la
pioggia, guardandomi con un'improvvisa intensità che trovai quasi
spaventosa, tanto che mi ritrassi nonostante il dolore al petto e alle
braccia. «Parli in modo strano, Hadr... Hadrian.» Incespicò nel
pronunciare il mio nome poco familiare. «Da dove vieni?»

«Da Delos» risposi, come se quel nome avesse potuto significare qualcosa
per lei.

`Parli in modo strano.' Quel pensiero mi impedì di dire altro. Era così
ovvio, adesso che qualcuno me lo aveva fatto notare. Parlavo ancora come
un nobile di Delos, come un palatino dell'Impero. Avrei dovuto
notarlo... non c'era da meravigliarsi che gli altri poveri della città
diffidassero di me. Spiccavo in mezzo a loro come un dito rotto.

«Dov'è?»

Solo alcune delle stelle non fisse facevano capolino attraverso la
tempesta, vigili e senza tempo. Intorno a una di quelle luci c'era la
mia casa, ma non avrei saputo dire quale, perché anche se conoscevo i
nomi delle stelle qui la loro posizione era più strana. Poteva essere
una qualsiasi o nessuna di quelle, e la mia casa poteva essere persa fra
le nuvole o nel Buio. La verità però è poco poetica, e mia madre mi
aveva educato meglio di così. Mi morsi un labbro -- in parte per
ricacciare indietro un'altra ondata di dolore generata dal fianco -- e
indicai. «Vedi quella stella, laggiù?» Tossii mentre lei annuiva. «Non
la puoi vedere, ma dietro c'è un'altra stella molto più lontana. È da lì
che provengo.»

«Com'è?»

«È il mio turno di fare una domanda!» interloquii, cercando di mettermi
a sedere. Inutile. Come spesso succede nei casi di lesioni o di
sfinimento, il riposo era per me la cosa migliore e la peggiore. Non
riuscivo a muovermi. «Non hai qualche analgesico, vero? Niente di
pesante. Niente narcotici.»

Lei si ritrasse, incupendosi in volto. «Io non...» Non sapevo bene se mi
stesse rispondendo oppure no. La voce le si incrinò. Era paura? Perché?
Non riuscivo a capire. «Non conosco queste parole.»

«Analgesici» dissi. «Medicine.»

«Si può sapere cosa è successo, esattamente?»

\emph{Questo} mi fece affiorare una smorfia sulla faccia. «Ho cercato di
derubare alcuni bastardi che portavano una fascia bianca al braccio.»
Accennai al bicipite sinistro.

«La cricca di Rells?» Cat impallidì. «Merda, \emph{rus}, tu sei pazzo.»

«In quel momento sembrava una buona idea.» Risi debolmente, cercando di
minimizzare la mia stupidità. Agitai le dita dei piedi. «Volevo...» La
voce mi si spense, spezzata dallo sfinimento, dalla sofferenza e da
quella notte di una lunghezza interminabile.

...\emph{delle scarpe}.

\begin{figure}
	\centering
	\def\svgwidth{\columnwidth}
	\scalebox{0.2}{\input{divisore.pdf_tex}}
\end{figure}

Devo essere svenuto. Quando mi destai, la notte di Emesh avvolgeva fitta
il mondo, le nubi si erano spostate e l'intenso acquazzone si era
ridotto a una pioggia gentile. Ero solo, e non mi ero spostato da sotto
il pannello solare. Cat non si vedeva da nessuna parte e io rimasi
disteso, dolorante, fradicio e infreddolito, su quel tetto duro.
Qualsiasi senso di comodità avessi percepito in precedenza prima di
addormentarmi era scomparso, sostituito da una profonda rigidità e da un
sordo pulsare alla base del cranio. Prima che Cat tornasse rimasi lì
disteso a lungo, a fissare il lato inferiore del pannello solare e il
vorticare delle nuvole lungo l'orizzonte. La tempesta si era spostata
verso nord, e adesso i fulmini erano visibili solo in lontananza, con i
tuoni ridotti a un sordo rullo di tamburo.

«Non sei morto» commentò lei, con la piccola bocca che si sollevava agli
angoli.

«Non ero conciato così male.»

Posò sul tetto accanto a me un sacchetto di plastica per la spesa di
quel genere pesante che marinai e casalinghe insistevano per non buttare
mai via, e lo aprì. «Qualcuna di queste cose ti può aiutare?»

Dovetti aggrapparmi a una delle sbarre di supporto del pannello per
riuscire a issarmi in posizione seduta. Il sacchetto era pieno di
boccette di medicine mezze vuote, con l'etichetta strappata, sbiadita o
scritta in caratteri stranieri. Cominciai a frugare fra di esse. «Dove
hai trovato tutto questo?»

«Nei rifiuti» rispose lei, semplicemente. «Potrebbero andare ancora
bene.»

Non ero nella posizione di fare lo schizzinoso. Misi da un lato cinque
boccette di vitamine e tre con l'etichetta scritta in una forma di
mandari che non conoscevo... inutile rischiare. Finalmente trovai una
boccetta etichettata con i caratteristici caratteri squadrati
dell'alfabeto lothriano. Naproxen. Agitai la boccetta prima di aprirla.
Non restavano molte pastiglie. «Sei certa che le posso prendere?»

Lei fece un gesto come per buttare via qualcosa e uscì sotto la pioggia
che cadeva ancora. Ignorando il dosaggio raccomandato sulla confezione
presi tre di quelle pillole e lasciai il sacchetto sotto il pannello
solare. Mi ci volle un notevole sforzo per costringermi ad alzarmi e uno
anche maggiore per togliermi la camicia fradicia che lasciai appesa a
una sbarra sotto il pannello. Mi soffermai per un momento per sfilarmi
dal collo la corda a cui era appeso l'anello, che infilai in una tasca
dei pantaloni bagnati. Un momento più tardi mi accasciai sul bordo del
tetto che si affacciava sulla città. Nel vicolo sottostante l'acqua era
salita di quasi un piede e mezzo, una vista che mi indusse a distogliere
lo sguardo. Stanco com'ero stato, avrei potuto svegliarmi solo quando
era troppo tardi.

Cat e io restammo lì seduti a lungo, nascosti al mondo pur vedendolo
tutto quanto. «Perché mi hai aiutato?» chiesi infine.

«Te l'ho detto» rispose, mimando di nuovo quel gesto come di buttare via
qualcosa. «Non lascerei annegare nessuno nel sonno in quel modo.»
Sostenni a lungo il suo sguardo e qualcosa sul mio volto dovette
stimolarla ad aggiungere: «E stavi piangendo. So com'è sentirsi soli.»

«Hai detto qualcosa su tua madre» osservai, con la curiosità che aveva
la meglio sul tatto.

Il suo volto grazioso anche se popolano si contorse per la tristezza e
io sentii qualcosa che si rompeva dentro di me per averle causato quello
sconforto mentre lei rispondeva: «È morta. Era malata, aveva la necrosi,
sai...» Gettò qualcosa oltre l'orlo del tetto, forse una scheggia di
cemento. «Hai una famiglia?» Scossi il capo e resistetti alla tentazione
di toccare l'anello che avevo in tasca. «No.»


