\chapter{Una sorta di esilio}

«L'intus aveva ragione» osservò freddamente lord Balian Mataro,
raccogliendo intorno a sé le vesti di seta arancione nel sedere dietro
la scrivania. «Quella donna tavrosiana ti tiene per le palle.» Venendo
dalle labbra di un lord palatino dell'Impero, quella piccola oscenità mi
spaventò più dell'ira nella sua voce. In risposta a un suo gesto brusco
sedetti dall'altro lato della scrivania, facendo scorrere lo sguardo
sulle pareti di vetro colorato del suo ufficio.

Ricordavo l'ufficio di mio padre, a Meidua, e le due stanze non avevano
quasi niente in comune. Quella di mio padre era tutta pietra nera,
tappeti scuri e cupo legno lucidato, piena di cose disposte in modo tale
da indicare una persona dall'immensa disciplina, un uomo che dimorava
più nella sua mente che nel mondo. Questo posto, invece, era tutto linee
pulite e candore moderno, tanto che avrebbe potuto essere il centro di
informazioni di combattimento di un incrociatore da battaglia della
Legione. Non avrei saputo dire che genere di uomo rivelasse se non fosse
stato seduto davanti a me.

«Ho commesso un errore, Vostra signoria, ma conosci la legge.»

«Come la conoscevi anche tu, quando hai costretto quel cantore a un
duello.» Mataro si incupì in volto. «La mia grande priora vuole la tua
testa. No, dimentica quello che ho detto... ti vuole vivo.»

Deglutii a fatica e abbassai lo sguardo sulle mie mani, tormentando
distrattamente le semplici bende applicate all'avambraccio. Non avevo
avuto bisogno di correttivi per quella ferita di poco conto. «Lo so.»

«Il ragazzo era suo figlio.»

«Lo so.»

«E allora, nel sacro nome della Terra, perché hai...» Si interruppe,
mordendosi l'interno della guancia, e scosse il capo. «Ho bisogno di
mandarti via dalla mia città, lontano da... tutto questo.» Accennò un
giro con una mano ingioiellata. «Lontano da lei, finché le cose non si
saranno calmate.» In quel momento entrò un servitore, evidentemente
parte di una qualche routine perché sgranò gli occhi per la sorpresa nel
vedere che il suo padrone aveva compagnia e uscì subito con un inchino,
portando via con sé il più discretamente possibile il servizio da tè con
una singola tazza. «La Cappellania tiene \emph{me} per le palle, lo sai,
e non posso governare il mio pianeta avendola contro di me. Se non darò
a Ligeia quello che vuole ostacolerà i miei accordi commerciali,
assoggetterà le mie navi a perquisizioni e sequestri, tratterrà i miei
ufficiali... qualsiasi cosa che non sia l'arrivare a invocare
l'Inquisizione. Hai ucciso suo figlio, dannazione!» Calò la mano sulla
scrivania per sottolineare quel ritornello.

«Con tutto il rispetto, sire, io miravo a ottenere il primo sangue»
replicai in tono mite, incapace di guardarlo in faccia. Vidi le mani che
mi tremavano in grembo. Non riuscivo a togliermi dalla mente gli occhi
del prete, uno nero e l'altro azzurro, immutabili.

«Non lo hai avuto.» Le nocche del conte sbiancarono contro il bordo
della scrivania, poi si rilassarono di colpo. «Se non avessi bisogno di
te, ti consegnerei a Ligeia in questo preciso momento.» Lanciò
un'occhiata alla pesante porta di metallo, oltre la quale aspettavano i
soldati che mi avevano trascinato via dal sanguinoso terreno dello
scontro. Quanto sarebbe stato facile per loro trascinarmi attraverso il
complesso del castello e fino al tempio della Cappellania per
consegnarmi ai cathar di Ligeia e farla finita.

Il conte prelevò una scatola adorna di gemme da un angolo della
scrivania, smuovendo una pila di foglietti e un'immagine olografica di
lui stesso e lord Luthor durante una spedizione di caccia. «Sì...
ecco...» Si schiarì la gola nel rigirare la scatola fra le mani.
«Intendevo tenere la cosa sotto silenzio per un altro paio d'anni, ma il
tuo piccolo attacco di idiozia mi ha forzato la mano.» Imprecò con una
violenza che mi fece sussultare e per poco non schiacciò la scatoletta
ingemmata in un pugno massiccio. Vedendo {quell'esibizione}, le sue
dimensioni, mi sentii sporco e meschino nonostante i miei geni.
«Dannazione, ragazzo! Credevo dovessi essere intelligente.»

«Furbo» ribattei con arguzia. «Non è come essere intelligente. Come hai
detto...» Avevo lasciato che Valka mi desse alla testa, e lei era ancora
là, accucciata appena oltre i miei occhi, accigliata e furente. Serrai i
pugni in grembo per bloccarne il tremito. Non riuscivo a smettere di
vedere la faccia di Gilliam, gli occhi spaiati che si facevano vitrei,
rilassati, fissi su una qualche luce al di là dei confini della vista
mortale.

Con lentezza forzata, il conte tornò a posare la scatola sulla lucida
superficie di vetro della scrivania, aggrottando le sopracciglia mentre
mi studiava dall'alto della sua considerevole altezza... per la Terra,
era enorme. «Speravo di farti sposare Anaïs dopo il suo Efebeia.»

Impiegai un minuto a richiudere la bocca e un altro a raccogliere i
frammenti del mio cervello quanto bastava per balbettare: «Sposare?
Tua... tua figlia?» Per un momento lo spettro di Gilliam dagli occhi
vitrei e l'immagine accoccolata di una furente Valka si dissolsero come
fumo, schiarendo l'aria fino a rivelare la figura snella e ben tornita
di Anaïs Mataro. Splendida come una scultura di ghiaccio, monotona come
una pozzanghera.

«O mio figlio, se preferivi. Pensavo soltanto...»

«No! No, Vostra signoria.» Sperai che la mia fretta non suonasse come un
rimprovero nei suoi confronti e mi affrettai ad aggiungere, con maggiore
diplomazia: «Naturalmente ne sono onorato, ma... io? Sposato? A una dama
palatina?» Supposi che quello fosse sempre stato uno dei miei possibili
destini, ma erano passati così tanti anni da quando ero stato
\emph{davvero} lord Hadrian Marlowe che tutta quella cosa mi sembrava
come un sogno menzognero. Un incubo. Intrappolato su Emesh, sul mondo
dove la mia vita era andata in pezzi. Mille obiezioni spuntarono come
erbacce, e come erbacce mi soffocarono, permettendo al conte di
continuare.

«Non sarebbe successo prima di tre anni» proseguì, d'un tratto più
imbarazzato che iroso, e quel cambiamento mi allarmò. «Due fino
all'Efebeia, poi un altro come periodo di fidanzamento richiesto dalle
usanze, e avevo sperato di tenere la cosa sotto silenzio, di darti il
tempo di abituarti alla vita qui a Borosevo, di imparare a conoscere la
ragazza, ma il tuo comportamento folle...» Si interruppe, esalando il
respiro fra i denti con un sibilo.

Ancora sommerso dal rumore che mi riverberava nelle orecchie,
farfugliai: «Ma... io? Signore...» Quella non era la forma corretta con
cui rivolgersi a lui, e una sfumatura di disprezzo attraversò il volto
massiccio di Balian Mataro. «Non ho niente a mio nome e come sai ho
lasciato la mia casa in seguito ad alcune difficoltà. Inoltre mio padre
è solo un nobile minore, un arconte. Certo, ha possedimenti e non solo
la carica, ma comunque io...»

«Quanti anni ha tuo padre?»

Quella domanda fuori argomento mi sconvolse e mi fece precipitare a
spirale lungo un nuovo sentiero. «Mio padre! Che il Buio mi prenda!
Eccellenza, se dovesse scoprire dove sono... lui e la Cappellania! Mi
ucciderebbero per essere fuggito.»

Il conte sollevò una mano per farmi tacere. «Quanti anni ha?»

«Non... non lo so» risposi, dopo essermi preso un momento per
ricompormi. «Quasi trecento, credo. Perché?»

«E suo padre?»

«Ne avrebbe compiuti un po' più di quattrocentoventi, ma è stato
assassinato da...»

«Perché?» chiese il conte, poi agitò una mano. «Non importa. Sua madre,
allora.»

Lo fissai socchiudendo gli occhi mentre mi sforzavo di seguire quella
bizzarra deviazione della sua logica. «Seicento... seicentoottantadue.»
Dovetti faticare per ricordarlo. «Che senso ha tutto questo?»

«E quanti anni pensi che io abbia?» Di nuovo sollevò la piccola scatola
ingioiellata, facendola svanire nel suo pugno...

Tirai a indovinare. «Duecento anni?»

«Centotrentatré il prossimo autunno.»

«Cosa?» sbottai, incapace di contenere la mia sorpresa. Era decisamente
troppo giovane. C'era già del grigio nella barba nera e le linee intorno
alla bocca sembravano troppo scavate nella pelle della sua faccia.

Temporaneamente dimentico dei peccati di quella giornata, il conte
allargò le mani massicce in un gesto di innocente sconfitta. «Siamo un
mondo minore, e un Casato minore, e il nostro sangue non è nobile come
il tuo. Ho fatto scannerizzare quel tuo adorabile genoma quando ti
abbiamo portato qui. Il mio Casato dovrebbe pagare con il suo titolo per
ottenere quello che hai lì dentro.» Indicò la mia faccia. «Non ho idea
di come la tua famiglia sia giunta a possedere simili schemi e di come
abbia potuto conservarli. Fai quello che ti chiedo e ti renderò consorte
nella posizione più elevata di questo sistema. Tutto quello che chiedo
sono i tuoi geni.»

«La licenza per l'uranio» dissi semplicemente.

«Cosa?»

«La ricchezza» spiegai, non sapendo che altro dire. «Abbiamo la licenza
per estrarre l'uranio.» Quella era una parte della spiegazione, ma avevo
ereditato anche un gran numero di complessi genetici iperavanzati da mia
madre, figlia di una duchessa e viceregina imperiale, nonché lontana
cugina -- più o meno di dodicesimo grado -- dello stesso Casato
imperiale. Quei complessi, rinforzati da generazioni di riproduzioni con
il Casato Kephalos e il Casato Ormund, che aveva detenuto il ducato di
Delos ai tempi di lord Julian, rendevano la linea di discendenza della
mia famiglia desiderabile quanto quella di molti nobili assai più
potenti di mio padre. Fra il conte e me scese il silenzio e io serrai di
nuovo le mani, questa volta tanto per trattenere la mia rabbia quanto
per soffocare il tremito dell'orrore e del cordoglio collegati alla
morte di Gilliam. Qualcosa di metallico e infranto vibrò dentro di me.
Avrei dovuto saperlo, me ne sarei dovuto rendere conto. Stupido.
Stupido. Stupido. Poi un altro pensiero, cupo quanto il primo, mi salì
dal petto e mi eruppe dalle labbra. «Non sono un dannato stallone!»

«Sì, lo sei!» Il conte sbatté la scatoletta sulla scrivania con un tale
crepitio che mi aspettai di vedere segni di frattura, ma la superficie
rimase intatta. «Sei qualsiasi cosa io dica che sei. Non sei in una
forte posizione contrattuale, lord Marlowe. Hai ucciso un membro anziano
del mio staff!» La sua voce si rinforzò a ogni parola e lui piantò le
mani sui braccioli della sedia.

I pugni mi si serrarono convulsamente. Potevo ancora avvertire il sudore
ormai secco del duello. Fissai gli occhi di Balian, neri come uno di
quelli di Gilliam. «Secondo la legge non è omicidio.»

«La legge!» mi fece eco il conte. «Credi che questo abbia importanza per
la grande priora? Se pensi che una norma di legge possa proteggerti da
quello che hai fatto sei uno stolto. Hai bisogno del mio aiuto.» In tono
d'un tratto più morbido e ragionevole proseguì: «Non ti sto chiedendo
niente di sgradevole. Dovresti esserne lieto. Conosci la ragazza e sai
che ti è affezionata, il che è più di quanto si possa dire di molti
accordi simili.»

Lo sguardo gelido di Ligeia Vas mi venne a tormentare mentre me ne stavo
là seduto con le mani che mi si agitavano in grembo. Non c'era niente
che potessi fare, nessun posto dove andare. Non potevo rifiutare e non
potevo fuggire. Mi accigliai e annuii lentamente. «Ma che dire di Anaïs?
Sa di tutto questo?»

Il conte si accigliò. «Per chi mi prendi? Lo sa da quando sei arrivato
qui.» Alla fine aprì la scatoletta e ne tirò fuori una striscia di
foglia di verrox candita, staccandola da un mucchietto appiccicoso.
\emph{Da quando sono arrivato.} D'un tratto le sue azioni da quando ci
eravamo incontrati si misero meglio a fuoco: il modo in cui mi era
sempre stata intorno, mi aveva sempre invitato a eventi sociali, aveva
continuato a toccarmi, ad appendersi al mio braccio. Era tutto così
ovvio, così... calcolato. Mi sentii volgare, meno di una persona perché
dopotutto non si trattava davvero di me. «Andando avanti, i figli che
avrà da te erediteranno i tuoi complessi genetici e i miei nipoti
saranno ammessi fra i pari dell'Impero.»

Vidi una pecca nel suo piano e la trafissi con un dito mentale. «Ma il
tuo erede è Dorian, giusto?»

«Presunto.» Il conte mi rivolse un sorriso. «I complessi presenti nel
tuo sangue valgono però un piccolo cambiamento di piani. Entrambi i miei
figli sono giovani, c'è ancora tempo per fare un vero lord dell'uno o
dell'altra, non credi?» Si infilò in bocca la foglia di verrox e
masticò. Gli occhi gli si chiusero spontaneamente, poi tornarono a
spalancarsi, vigili, mentre deglutiva.

Quell'espressione drammatica fece scattare qualcosa dentro di me, perché
quasi schizzai in piedi. «Ma, signore, mio padre!» Ero stato sul punto
di sollevare il problema, ma quella deviazione relativa all'età e ai
complessi genetici l'aveva scacciata dal mio cervello intontito dallo
stress. Avevo bisogno di dormire, di focalizzarmi. «Non approverà mai?»
A quel punto ero disperato e mi stavo aggrappando alle pagliuzze, a
qualsiasi mezzo di fuga fossi riuscito a trovare.

Lord Balian Mataro infilò una mano in un cassetto della scrivania e tirò
fuori un foglio di carta di cristallo che posò sulla superficie lucida e
spinse verso di me. Di norma, atti del genere erano scritti su pergamena
e firmati a mano, ma quella era una copia, chiaramente mandata fino a
Emesh attraverso la vasta ed echeggiante quiete dello spazio mediante
telegrafo quantico. Sentii precipitare la temperatura del mio sangue
quando vidi il sigillo stampato sotto i frattali olografici e accanto
alle firme: il diavolo carminio che saltellava con il tridente levato
sopra la testa, su un campo del nero più scuro che esista, sovrastando
le parole: \foreignlanguage{italian}{la spada, nostro oratore.}
Corrispondeva alla perfezione all'incisione sul mio anello e in effetti
era stato impresso da un anello identico a centinaia di anni luce di
distanza. E accanto c'erano i nomi: Alistair Diomedes Friedrich Marlowe
ed Elmira Gwendolyn Kephalos. Mio padre e mia nonna, la viceregina.

Passò un altro terribile minuto prima che realizzassi la natura del
documento che avevano firmato, un tempo tanto lungo che la voce profonda
del conte chiese: «Sai cos'è?»

«Un atto di disconoscimento» risposi, con lo sguardo che si spostava
sull'immagine del raggio di sole imperiale in cima al lungo foglio di
bianca carta di cristallo. Lessi: «Noi decidiamo, sotto il simbolo di
Sua radiosità imperiale, il nostro imperatore William XXIII del Casato
Avent, figlio primogenito della Terra, ecc., di dissolvere ogni legame
legale e familiare con il rinnegato Hadrian del Casato Marlowe, in
passato della prefettura di Meidua, ducato di Delos, provincia di
Auriga. Di concerto con l'esame da parte del Santo Ufficio
dell'Inquisizione, la sua condotta...»

«...la sua condotta è stata trovata carente degli standard e della
grazia che ci si aspettava nella sua posizione. Ha tradito il suo Casato
e il lord suo padre e portato la vergogna sul nome della sua famiglia e
sulla viceregina di Delos» recitò il conte; o forse sapeva leggere un
foglio rovesciato. «Gravi accuse.» Interpretai come un atto di
misericordia il fatto che non continuò a leggere.

Non risposi. C'erano lacrime nei miei occhi e qualcosa di simile alle
lacrime nella gola. Non potevo parlare e il mio anello era un peso di
piombo che mi gravava sul pollice, morto e inutile. Era solo un pezzo di
metallo. Quindi non avevo niente, davvero niente. Prima, quando vivevo
in povertà nelle strade di Borosevo, avevo avuto la mia dignità privata
dovuta al mio rango nascosto e alle tenute che, sia pure in modo tenue,
erano legate al mio nome e al mio rango, ma adesso ero davvero povero,
senza niente se non i geni che avevo nelle ossa. Mataro aveva ragione,
\emph{ero} uno stallone, e nel senso meno lusinghiero del termine, come
un famoso cavallo da corsa che si era azzoppato scivolando. Cercai di
non pensare a Switch, a Pallino, a Elara e alla nave che avevo avuto
intenzione di comprare con l'inganno con i beni fantasma racchiusi nel
mio anello.

Non più. Avrei \emph{quasi} dovuto prostituirmi con Anaïs e il Casato
Mataro per sopravvivere.

«Mi dispiace dirti tutto questo.»

«No, non ti dispiace» replicai, con voce simile alla terra di una tomba.
Avevo appena visto la data su quell'atto e sapevo cosa mi era stato
fatto, e da chi. «Tu lo hai contattato. Hai avvertito mio padre, gli hai
telegrafato subito dopo che sono arrivato.» Il \emph{giorno} in cui ero
arrivato, se avevo letto bene la data. Aveva pianificato tutto questo
fin da quando ci eravamo incontrati, mentre sedevo privo di sensi su
quella poltrona al piano di sopra. Per questo era stato disposto a
tenermi nella sua casa, perché aveva tollerato le mie bravate
nell'ipogeo del colosseo e a cena con la priora, e quel bastardo non
aveva neppure la decenza di negarlo. La sua faccia non ebbe neppure un
minimo sussulto. Sapeva di avermi in pugno. «Per quanto ne so, lo hai
incitato tu a fare questo, hai stretto un accordo per tenermi qui.
Quanto è costato? Trenta pezzi d'argento?»

Il suo volto si fece vacuo. «Cosa?» Non aveva capito il riferimento.
«Dovresti sentirti onorato. Diventerai mio figlio.»

«Diventerò un prostituto.» Per poco non soffocai e mi presi un momento
per ricacciare indietro un singhiozzo prima che mi potesse sfuggire
dalla gola. \emph{Il dolore è vuoto.} Così rimproverato, archiviai il
mio dolore e affrontai questioni pratiche. «Allora, cosa ne farai di
me?» Ci sarebbe stato tempo in seguito per indugiare sugli orrori di
quella giornata. Tempo per annegarvi dentro, se necessario.

«Ti sei attirato addosso tu stesso tutto questo, ragazzo.» Il conte tirò
il documento verso di sé. «Adesso, per venire ai problemi presenti...
una volta che sarai sposato, Ligeia non potrà agire contro di te senza
agire contro il mio Casato, cosa che non farà.» Si alzò con un vorticare
di sete arancione e prese a passeggiare ansiosamente fino all'arco
ricurvo delle finestre; che in realtà erano schermi visori che
proiettavano un'immagine di Borosevo in tutta la sua infima gloria
arrugginita, con i canali intasati di alghe verdi. «Il problema sono
quei tre anni che devono trascorrere. Sai, ti sei fatto una potente
nemica.» Fece una pausa per prelevare un'altra foglia di verrox con la
mano che vibrava visibilmente. «Ho idea di mandarti a Tivan Melluan, su
Binah, per allontanarti dal pericolo.»

Non risposi. Mi stavo fissando di nuovo le mani, che tremavano a loro
volta ma non per tossiemia da verrox. I capelli mi ricaddero sul volto,
nascondendomi al conte e le lacrime si agitarono sulla superficie dei
miei occhi, senza cadere. Troppo, era tutto troppo.

Alla fine sollevai lo sguardo, rassegnato. «Mio signore, se posso avrei
una domanda.»

Valeva la pena di chiederlo. Era tutto quello che potevo fare.


