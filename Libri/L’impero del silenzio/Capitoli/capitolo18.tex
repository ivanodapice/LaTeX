\chapter{L'ira è cecità}

Bussarono per la prima volta quella sera sul tardi, quando il sole
d'argento si era tinto d'oro e stava tramontando sopra le colline
occidentali. Per quella che mi pareva la millesima volta, ero seduto sul
pavimento della mia suite nel palazzo di Haspida, vagliando e riponendo
di nuovo nel bagaglio le cose che intendevo portare su Vesperad... o su
Teukros, se mia madre e io fossimo riusciti nel nostro piano. Raddrizzai
una pila di libri mentre rispondevo da sopra la spalla: «Avanti!»

Crispin entrò con passo tranquillo, aspettando a stento il mio permesso.
Teneva in una mano una mela mangiata a metà e la sua camicia grigia era
abbottonata solo in parte.

Mi alzai in fretta, urtando e scompigliando una pila di camicie piegate
con cura. Imprecando sotto voce mi affrettai a riordinarle. «Cosa vuoi?»
La presenza del comitato di cani da guardia mandato da mio padre mi
aveva reso nervoso, soprattutto Alcuin. Quell'uomo aveva un'attenzione
affilata come un cavo di nanocarbonio ed era altrettanto pericoloso per
qualcuno che si fosse mosso in fretta e con distrazione.

«Partirai domattina» rispose Crispin, allargando le braccia. «Domattina
\emph{presto}. Io... ecco credo che questo sia un addio, almeno per un
po'.»

Mi accovacciai per riporre il vestiario sul fondo del pesante baule di
plastica. «Lo so, ci vogliono undici anni per arrivare su Vesperad. Nel
tempo che impiegheranno a risvegliarmi tu sarai il fratello maggiore.»
Mi rialzai, lisciando il davanti della camicia e sistemando la frangia
di capelli scuri.

Il sorriso tagliente di Crispin gli incurvò un angolo della bocca e lui
ridacchiò sommessamente. «Già, non ci avevo pensato.» Abbassò lo sguardo
sugli oggetti raccolti sul pavimento... i libri, i cristalli di dati, le
scarpe e un paio di lunghi coltelli. «Questo è tutto quello che ti porti
dietro?»

Scrollai le spalle. «La Cappellania non vuole che abbiamo più del
necessario. Si suppone che ci lasciamo alle spalle la nostra vita quanto
più è possibile.» \emph{E gli scoliasti si aspetteranno che rinunci a
	tutto.} Rabbrividii nel ricordare il freddo vuoto negli occhi di Alcuin
e avvertii di nuovo l'ombra di un dubbio.

«Questa parte sembra deprimente. Credevo che Eusebia avesse quei grandi
appartamenti nella Torre Campanaria. Non è tutto suo?»

«Certo» replicai, usando una pila di libri di lingue per comprimere il
vestiario ripiegato. «Ma lei non è una studentessa, giusto? Le regole
sono diverse.»

«Suppongo che abbia senso.» Crispin parlò con la bocca piena di mela,
poi si lasciò cadere sulla mia poltrona. Ero lieto che questa volta,
almeno, non stesse brandendo una lama sguainata. «Non mi ero però reso
conto che sarebbe stata tanto dura per te.»

La mia attenzione si spostò, vagando sul panorama fuori dalla finestra,
sulle piscine dei gigli d'acqua e fino al distante cipresso dalle foglie
che apparivano nere nella luce dorata del crepuscolo. «Sono solo
oggetti, Crispin. Non sono importanti.»

Lui rise, un rozzo suono rauco privo di musicalità. «Se lo dici tu,
fratello.» Posò la mela mangiata a metà sul tavolino dalle gambe sottili
accanto alla poltrona e sollevò una gamba per aggiustare il risvolto di
uno stivale. «Però sei comunque tu quello che sta per andare là fuori,
sai? Potrai vedere l'Impero.»

«Dubito che sia molto entusiasmante» ribattei in tono asciutto,
continuando a non guardare mio fratello. «Come ho detto, per anni vedrò
soltanto l'interno di una cella di addestramento. Questo è tutto.» In
quel momento mi resi conto che stavo imitando l'abitudine di mia madre
di guardare fuori dalle finestre, di andare alla deriva lontano dal
luogo della conversazione nella misura in cui lo permettevano la
cortesia e l'architettura. Volevo disperatamente andarmene, e in quel
momento mi chiesi se in un qualche angolo segreto della tenuta non ci
fosse stata una navetta che veniva {rifornita} di carburante e
controllata per un viaggio notturno fino a Karch, pronta a incontrare il
contatto del Consorzio dei liberi mercanti. Quel piano mi rendeva
nervoso, ma mia madre era ancora disposta ad affidare la mia sicurezza
all'onore di quel capitano, quindi supposi che non ci fosse niente da
fare, a meno di voler diventare un sacro torturatore e inquisitore. Il
volto di Gibson mi apparve davanti agli occhi, come se fosse stato
riflesso nel vetro corazzato della finestra.

Era una cosa che non volevo.

Crispin era rimasto in silenzio per un lungo momento, un fatto che non
notai finché non infranse quel silenzio, richiamando l'attenzione
sull'assenza di parole fra di noi. «Quindi te la fai con la tenente,
eh?»

«Cosa?» Girai la testa di scatto, aggrottando le sopracciglia.

«Quella magra con i riccioli e le tette piccole.» Crispin mimò i seni
con le mani. «La ufficiale pilota.»

Mi feci cinereo in volto. «Kyra.»

«Kyra» ripeté Crispin, con quel suo orribile sorriso. «Allora è questo
il suo nome?» Si stuzzicò i denti con un'unghia che poi si pulì sui
pantaloni. «Allora lei è qualcosa di speciale? Immagino che sia per
questo che tu non sia voluto venire nell'harem con...»

«Basta così, Crispin.» Imitando nostro padre non mi girai neppure a
guardare mio fratello mentre parlavamo e non alzai il tono della mia
voce profonda al di sopra di un sussurro. «Lasciala in pace.»

Mio fratello sollevò le mani in un gesto difensivo, poi se le passò fra
i capelli corti con fare agitato. «Calmati, Hadrian. Ho capito. Bada
bene, suppongo che sia abbastanza graziosa. Forse un po' mascolina, ma
se è così che ti piacciono, ecco...»

«Basta così, ho detto.» Mi girai, sfilandomi di sotto lo sgabello. Avevo
i capelli ritti sul collo e serrai i denti, fissando mio fratello con
occhi roventi.

Crispin fece una pausa e recuperò la mela. Abbassò lo sguardo verso il
proprio grembo e parlò come rivolto alle sue mani. «Senti, mi dispiace.
Solo... vorrei essere io a partire. Nostro padre ha sempre preferito
te.»

Se fossi stato intento a bere, di certo avrei sputato fuori il liquido.
«Cosa?» farfugliai. «Per il sangue dell'imperatore... cos'hai detto?»
Per poco non incespicai in un paio di vecchi stivali rimasti vicino al
baule aperto.

«Ti sta mandando su Vesperad, mentre a me tocca il Riposo del Diavolo.
Cosa me ne faccio del fottuto Riposo del Diavolo?» Staccò un altro morso
dalla mela, guardando a sua volta fuori dalla finestra. «Tu potrai
essere là fuori a combattere, a dare la caccia a nobili traditori e ai
Cielcin...» Si interruppe. Per un momento pensai di averlo frainteso.
Per un momento realizzai che essendo il fratello minore si era sempre
aspettato di non ereditare niente. Proprio come io avevo pensato che il
Riposo dei Diavolo fosse sempre stato mio, Crispin aveva creduto che
fosse destinato a me. Aveva lavorato nella mia ombra e io nella sua,
senza che nessuno dei due sapesse che in realtà quella era l'ombra di
nostro padre, che ci stava facendo annegare entrambi.

«Non farti divorare il cervello.»

«Non è nella mia lista delle cose da fare.» Non riuscii a capire se lo
avesse detto come una battuta o se quella fosse per lui una sincera
preoccupazione. O forse lo aveva detto di proposito per ricordarmi la
cena durante la quale era cominciato il mio fatale perdere il favore di
mio padre. I suoi lineamenti tozzi non tradivano in nessun modo la mente
che scintillava dietro gli occhi opachi, e il sangue mi si raggelò.

Lui continuò a masticare, aprendo e chiudendo le labbra rumorosamente,
come una mucca. «Perché sei qui, Crispin?» azzardai infine.

Lui sbatté le palpebre. «Te l'ho detto! Volevo salutarti!» Si alzò e si
avvicinò abbastanza da assestarmi una pacca su una spalla. «Passerà del
tempo prima che ci rivediamo.» Per un momento rimanemmo fermi spalla a
spalla, guardando fuori dalla finestra, poi Crispin staccò un altro
rumoroso morso dalla mela. «Gli ultimi mesi sono stati... qualcosa di
notevole. Nostro padre dice che potrò combattere ancora nel Colosso.»
Quando mi limitai ad annuire, girandomi per prelevare da un tavolino il
mio diario e l'inestimabile copia di \emph{Il re con diecimila occhi},
aggiunse: «Però è un peccato quello che è successo a Gibson. Non posso
credere che quel vecchio bastardo sia diventato un traditore in quel
modo.»

Mi immobilizzai e il ghiaccio nelle mie vene si trasformò in {granito}.
Con la mascella tanto rigida che avrebbe potuto essere stata chiusa con
un cavo, sibilai: «Non ne voglio parlare.»

Noncurante, stupido, cieco, Crispin addentò di nuovo la mela. «Hai visto
la sua faccia? Disgustosa. Sembrava un proletario.» Calai una mano sul
telaio di mogano della finestra, scuotendo l'alluvetro che risuonò come
un tamburo. Girandomi, vidi Crispin sgranare gli occhi sotto le
sopracciglia squadrate, con una guancia gonfia in modo assurdo per il
cibo al suo interno. «Perché sei così infuriato?»

Non capiva. Davvero non capiva. «Gibson non c'è più.»

«Era soltanto un servitore.» Crispin inghiottì e ruotò appena la mela
per addentarla meglio. Gliela feci cadere di mano con un colpo,
mandandola a finire con un tonfo sulle piastrelle, dove rimbalzò verso
la porta. Crispin mi guardò con le spesse labbra tese in un'espressione
di sorpresa. «Perché lo hai fatto?»

L'ira è cecità. Lo dissi a me stesso, con la voce di Gibson che mi
risuonava nel cervello, borbottando il mantra del vecchio scoliasta. Fu
però un'altra voce, la mia, a rispondere a Crispin. «Era un mio amico.»

Lui mi fissò, incredulo. «Ha cercato di spedirti per diventare un
qualche inutile scoliasta. Ti avrebbe consegnato agli Extra!»

«Non è questo che è successo, imbecille.» Dilatai le narici, sentendo i
muscoli del volto che mi si tendevano pericolosamente, indurendosi in
una ferocia che era ad appena due passi dalla furia. Mentre parlavo
sapevo che non avrei dovuto dirlo, che qualche ufficiale dei servizi
segreti del Casato poteva sentirmi e avvertire la sicurezza, ma ero
ormai al di là del curarmi di queste cose.

Crispin arrossì, con il volto che gli si scuriva di colpo con una
transizione sorprendente dal suo abituale pallore. «Non osare.»

«Di darti dell'imbecille?» Mi avvicinai alla portata delle sue braccia.
Le ossa della mano destra mi dolevano, attirando la mia attenzione su di
esse e sul pericolo a cui mi stavo esponendo arrivando a tiro di un
Marlowe più grosso di me. Però sarei partito l'indomani... quella stessa
notte, se avessi potuto fare a modo mio, e dovevo dirlo. «Sei un
imbecille.»

I poeti parlano della rabbia come di una cosa infuocata che consuma,
distrugge, distorce un'anima spingendola a un'azione sbagliata. Le
canzoni parlano di vendetta, di amanti uccisi nella notte, di passioni
che divampano, di Casati fatti a brandelli. C'è però ben poco calore
nell'ira. Hanno ragione gli scoliasti. L'ira è cecità, un colore rosso
che offusca il mondo. È luce, non fuoco, e quando è finemente regolata,
la luce può tagliare con la stessa precisione dell'acciaio. Vidi le
labbra di Crispin che si incurvavano, preparandosi a una qualche
tagliente risposta che non arrivò mai alle mie orecchie perché non gli
uscì mai di bocca. Gli calai su un lato della faccia i libri pesanti che
avevo nella mano destra, facendolo crollare a terra con le braccia sotto
di sé.

«Lui mi stava aiutando, razza di bastardo.» Lasciai cadere i libri nel
baule e mi ersi su mio fratello. «Gli avevo chiesto io di farlo.»
Crispin era carponi e scrollava la testa come per liberarla da un
fischio nelle orecchie. `Te l'ho detto quando siamo partiti da casa,
Crispin: non voglio essere un priore. Non credo in niente di tutta
quella roba.' Quantomeno, questo era ciò che avrei detto, che volevo
dire. Crispin si sollevò di scatto dalla sua posizione in ginocchio e si
abbatté su di me come un ariete, circondandomi il torace con le braccia
mentre un grido rabbioso gli usciva dalla gola. Sbattemmo contro
l'enorme finestra, e rimasi senza fiato per l'impatto della mia testa
contro l'alluvetro. L'impeto di Crispin tradì il suo equilibrio e lui
barcollò, mancando del vantaggio che costituiva per me la parete di
vetro alle mie spalle.

Lo spinsi e lui carambolò lontano, rotolando per raddrizzarsi e
rialzarsi con i pugni sollevati. «La pagherai per questo!» disse. «Mi
senti?»

L'ira è cecità, ripetei a me stesso, ma non aveva importanza. A quel
punto tutto ribollì in superficie, emergendo scintillante dalle
profondità del mio cranio e lavando via il raziocinio. La fustigazione
di Gibson, l'episodio imbarazzante nelle strade di Meidua, i miei errori
con la delegazione mandari. Tutto emerse vorticando da quella oscurità
interiore, fondendosi con la mia furia per essere stato diseredato,
espropriato di tutto. La mia ira contro mio padre, il disprezzo nei
confronti della Cappellania e la gelosia verso Crispin.

Lui sferrò un colpo ampio che parai con il braccio. Sarebbe stata una
cosa facile, un gioco da bambini, se quel ragazzo non avesse avuto una
simile forza mostruosa. Eravamo entrambi palatini, più alti e forti
degli uomini comuni, ma Crispin era più alto di me di oltre una testa e
pesava almeno quaranta chili in più di muscoli. Lottai per tenerlo a
bada, per deviare il suo rapido sinistro che incassai di striscio su una
spalla al fine di ottenere l'apertura per un calcio a un ginocchio da
lui posizionato malamente. Barcollò con un ringhio. «Vattene da qui,
Crispin» dissi.

«No!» Attaccò ancora e io saltellai da un lato, mandandolo a finire
contro il pesante alluvetro della finestra. Si raddrizzò contro di esso,
lasciandovi una grossa impronta appiccicosa. Mi sentii lieto e perfino
impressionato di constatare che per quanto fossi furioso non ero come
Crispin. Anche con il sangue in fiamme e la mascella tanto serrata che
vibrava, mi mantenevo controllato. Freddo. Forse l'ira \emph{era} calore
per Crispin, forse non esiste un solo tipo di rabbia. Lui mi si lanciò
di nuovo contro con i colpi che cadevano come grandine, come pioggia,
come legionari che scendessero in picchiata dallo spazio su trasporti
corazzati. Incassai un pugno violento alla tempia e rotolai per
assecondarne l'impatto. Abbassandomi, ruotai su me stesso e raggiunsi
Crispin al mento con un tallone. L'impatto lo stordì e barcollò
all'indietro, rimanendo in piedi solo per uno sforzo di concentrazione.

Scosse di nuovo la testa con uno sbuffo ansioso simile a quello di un
toro. «Credi di essere migliore di me!» gridò, calando un dito contro il
pavimento. «Lo hai sempre fatto.»

Grato di avere un momento per riprendermi, mi pulii il naso con un
pollice, ritraendolo sporco di sangue. «Credo che tu sia un idiota,
Crispin.» Scrollai le mani e assunsi una posizione di guardia da pugile.
Mio fratello sferrò un colpo ampio, ma io mi abbassai e risposi con una
sequenza di tre pugni nel ventre. Grugnendo, mi calò un gomito sulla
spalla e io scivolai su un ginocchio, cosa che mi costrinse a rotolare
da un lato... impigliandomi in una pila di bucato che avevo lasciato
pronto per essere rimesso nei bagagli... per poi rialzarmi in tempo e
afferrare Crispin per un polso. Lui abbatté il braccio sul mio e lo
lasciai andare. «Smettila» ingiunsi con il respiro affannoso. «Vattene.»

«Mi hai colpito.» Tornò ad attaccare con un calcio che mi raggiunse al
fianco e mi costrinse a indietreggiare, calpestando i detriti della mia
vita, i vestiti e i documenti, le stupide cose che avevo portato con me.
«Mi hai colpito per primo» ripeté in tono più cupo.

«Non fare il bambino» sogghignai, incapace di trattenermi. «Quello è
stato un colpo scadente. Riprovaci.»

Vidi il bianco dei suoi occhi mentre si tirava indietro per un altro
calcio. Aveva intenzione di farlo per farmi un dispetto, per
sorprendermi, ma conoscevo mio fratello e sapevo che avrebbe abboccato
all'esca. Chiusi le dita intorno alla sua caviglia e gli feci perdere
l'equilibrio. Crollò a terra trascinandomi su di sé e io caddi con un
gomito puntato contro il suo ventre, strappandogli il respiro, poi lo
colpii con forza, senza esitare, mettendo a segno un pugno di striscio
alla faccia. Avendolo momentaneamente stordito riuscii a rimettermi in
piedi. «Fottiti» ansimò, fra un respiro affannoso e il successivo. Era
caduto vicino al mio baule e si sarebbe potuto fracassare la testa
contro un angolo se fosse stato appena un po' meno fortunato. L'afferrò
e si servì del bordo per issarsi a sedere con la testa che gli dondolava
e la schiena addossata al pesante baule.

Mi tenni pronto, con i capelli ritti, preparandomi a sferrargli un
calcio in faccia se avesse tentato qualcosa di stupido. Con il respiro
ancora ansante, la mia voce suonò d'un tratto avvizzita e tesa quando
ingiunsi: «Resta a terra.» Quella non era la voce di un diciannovenne,
era quella di uno spettro, di un vecchio stanco e fragile. «Resta a
terra, Crispin.»

Lui si massaggiò la mascella con una mano e le sue parole suonarono
inspessite e confuse quando ribatté: «Quando te ne sarai andato quella
tua piccola ragazza, la tenente, riceverà quello che ha avuto Gibson. E
dopo...»

Non sentii mai il resto perché la luce dell'ira mi svuotò. Non so se
fosse indignazione infantile o pura e semplice stupidità, ma mi lanciai
contro di lui.

Mi balzò incontro, sollevando il corpo massiccio dagli oggetti sparsi e
rotti che erano sul pavimento come se fosse stato scagliato da una
catapulta. Schivai abbassandomi e afferrandolo intorno alle gambe in
modo da sfruttare tutta quella massa e quel momento per sollevarlo e
spingerlo oltre la mia spalla, facendolo crollare sulle piastrelle a
braccia e gambe allargate. Sentii l'aria che gli sfuggiva dal corpo in
uno spasmo ma non esitai, non mi soffermai a pensare a quello che stavo
facendo e lo colpii alla testa con lo stivale. Crispin si accasciò,
privo di sensi.

Finì tutto così in fretta, ma del resto la violenza finisce sempre in
fretta. Non c'è un decrescendo, come nella musica, si limita a cessare.
A smettere. Come una luce spenta.

Respirando affannosamente cercai di quietare i miei pensieri, di fermare
le acque che precipitavano a cascata dentro di me, scendendo a spirale
nelle cieche grotte del panico. Non so per quanto tempo rimasi fermo là
con il cuore martellante, se per un'ora, un mese, qualche minuto, ma non
può essere stato molto tempo. Ogni atomo, ogni quark dentro di me
vibrava come una corda di violino pizzicata e poi tesa fino a
immobilizzarla. Cercai di ricorrere a uno degli esercizi di respirazione
che sir Felix mi aveva insegnato da bambino, di concentrarmi sulla
struttura a palazzo fatta di memoria e di fatti che Gibson aveva cercato
di insegnarmi a costruire, di cercare sollievo dentro me stesso,
qualsiasi cosa pur di quietare il martellare che mi vibrava nel sangue.
Mi accovacciai e avvicinai una mano alle labbra di Crispin. Se non
altro, respirava ancora, e questo era già qualcosa.

Non lo avevo ucciso. Era vivo.

Di certo le videocamere avevano trasmesso ogni cosa. Guardai una di
esse, una piccola apertura che scintillava come un occhio oscuro in un
angolo, attenta come uno stormo di corvi vicino a un patibolo. Snudai i
denti in un ringhio, imitando senza saperlo l'espressione con cui i
Cielcin esprimono la gioia più profonda, e raccolsi le mie cose sparse,
ficcandole a casaccio nel baule che intendevo portarmi dietro in un
esilio o nell'altro.

«Hadrian!»

La voce era trasformata da un senso di shock e di orrore, e quindi
suonava sconosciuta, ma aveva usato il mio nome.

Mia madre era ferma sulla soglia con aria sconvolta e una mano
dimenticata sulla maniglia. Per caso, o per la grazia di un dio ignoto,
era del tutto sola, senza guardie o un seguito. Sola. «Che cosa hai
fatto?»

«Mi ha attaccato» mentii, senza più preoccuparmi di niente. Un momento
più tardi mi parai le spalle aggiungendo: «Ha detto delle cose, su
Gibson. Sulla tenente.» Mi lanciai un'occhiata alle spalle in direzione
della forma supina di Crispin. «Perché sei qui? È ora?»

Lei guardò Crispin con espressione seria. «Lo è adesso.»

«Madre, mi dispiace, non lo aspettavo. Stavo attendendo la tua gente,
come mi avevi detto, e...»

Lei mi posò con gentilezza le mani sulle spalle e mi indusse a tacere
con un verso fioco. «No, va bene così. È una cosa buona.»

«Buona?» Quasi gridai quella parola. «Buona? Nel nome dell'imperatore,
come può essere buona?»

Da quella narratrice di storie che era, mia madre mi guardò come se
fossi stato uno dei suoi attori olografici, e con voce sommessa, seria e
triste rispose: «Mi hai dato una via di uscita. Dirò che mi hai rubato
la navetta e sei fuggito nella notte. Ne sai pilotare una, giusto?»

Annuii. «Sir Ardian me lo ha insegnato fin da quando avevo sette anni.»

«Bene. Prendi comunque con te la mia gente. Potresti aver bisogno di
aiuto.»

«Questo non ti metterà nei guai?»

«Tuo padre non oserebbe toccarmi. È il mio Casato a governare qui, non
il suo. Devi spicciarti, prendi quello che puoi.»

Mi diede una piccola spinta in direzione del pesante baule che mi ero
portato dietro da casa. Chinandomi, sfilai un paio di pantaloni da sotto
Crispin e li ficcai al suo interno, poi cominciai a gettare alla rinfusa
altre cose in cima al tutto. Fui assalito da un pensiero, pressante
quanto involontario. «Esamineranno le riprese, ci vedranno parlare.»

«Non ti hanno visto parlare con Gibson, il giorno della sua tortura,
giusto?»

Mi immobilizzai con un paio di calzini rossi in mano. «Sei stata tu?»
Per la Terra, non poteva aver saccheggiato lo spazio di archiviazione di
sicurezza del Riposo del Diavolo, oppure sì?

«Mi potrai ringraziare quando sarai al sicuro fuori dal sistema. Ora
spicciati.» Premette un tasto sul suo terminale.

Obbedii, lasciando cadere i calzini al loro posto, poi mi soffermai per
un momento. «Madre?»

«Figlio?» Nel suo tono c'era una nota ironica che non ho mai
dimenticato... un piccolo, udibile sorriso.

Chiusi con forza il coperchio del baule e mi girai. «Perché lo stai
facendo?» Lei si immobilizzò come una statua di marmo. Pensai che fosse
bloccata come lo è la luce sull'orizzonte di una stella collassata,
perché una parte di me sentì che avrebbe potuto non muoversi mai più.

«Madre?» gemette Crispin, da terra.

Un terribile sorriso incrinato apparve su quella pietra che lei definiva
una faccia. Per gli dèi, in un'altra vita sarebbe stata uno scoliasta
migliore di Gibson. «Sei sempre stato il mio preferito» rispose con voce
incrinata, dopo parecchi secondi lunghi un'eternità.

Fui salvato dalla necessità di rispondere dall'arrivo di due legionari
dei Kephalos... e di Kyra, che riservò appena uno sguardo a Crispin.
«Padrone Hadrian, vieni subito con noi.»

«Kyra?» Guardai verso mia madre mentre ogni pezzo si incastrava
rumorosamente al suo posto.

Lei scosse il capo, pratica e seria. «Non c'è tempo.»

«\emph{Tu} sei gli occhi di mia madre?» Lanciai un'altra occhiata a mia
madre, che sorrise.

«Dobbiamo andare!» scattò Kyra.

Lasciai che i legionari prendessero il mio baule. Il loro volto era
coperto da quella visiera bianca che li faceva apparire in qualche modo
irreali, come parte di un sogno. Di una commedia. Incontrai lo sguardo
di mia madre. «Grazie» dissi. Quelle sono state le ultime parole che le
abbia mai rivolto, e come succede sempre con le ultime parole non furono
abbastanza.

