\chapter{La flagellazione alla colonna}

Il mio mondo cambiò con il risuonare delle campane, rintocchi profondi
quanto il creparsi delle pietre sotto il terreno.

In alto, nelle mie camere, lasciai cadere la camicia bianca che stavo
ripiegando per riporla nel baule e rimasi in ascolto. Quel suono
scuoteva le pietre stesse della possente fortezza, limpido, basso e
sonoro. Controllai il mio terminale: mancava circa un'ora a mezzogiorno,
ma non erano le undici in punto e quello fu per me il primo indizio che
qualcosa non andava prima che riconoscessi le campane che convocavano
un'assemblea speciale. In fretta, infilai il libro che Gibson mi aveva
dato, \emph{Il re con diecimila occhi}, nel bauletto che mi avrebbe
accompagnato in esilio, gettai la giacca sopra a tutto quello che
conteneva e chiusi il coperchio. Avevo l'inquietante sensazione che
quelle campane suonassero per me. Questa doveva essere la dichiarazione
formale da parte di mio padre che Crispin sarebbe stato il suo erede?
Sarebbe stato da lui tenere una cerimonia del genere mentre io ero
ancora presente.

Ci andai comunque, scendendo con l'ascensore e sbucando nell'atrio,
all'ombra delle bandiere gemelle, con il diavolo dei Marlowe che
saltellava brandendo un tridente. I tacchi degli stivali ticchettarono
sul mosaico di bronzo che raffigurava il raggio di sole imperiale, ed
ero a metà strada dalle porte massicce quando mi resi conto che la
piazza si sarebbe riempita di tutto il personale del castello: servitori
in uniforme rossa e guardie in armatura nera, le vesti grigie e marroni
dei vari logoteti miste qua e là ai sai a strisce bianche e nere degli
anagnostici della Cappellania che si {affrettavano} ad andare alle loro
preghiere, e gli abiti più vivaci dei membri delle Gilde che erano in
città per affari. Tutti gli occhi sarebbero stati su di me quando avessi
varcato quelle porte e fossi uscito sulla balconata sovrastante la
piazza. Mi guardai intorno, quasi aspettandomi di veder arrivare mio
padre e i suoi littori, già schermati, provenienti dalla stanza verde
tenuta pronta proprio accanto a quella porta.

Non apparve nessuno. Se mio padre fosse stato già fuori lo avrei sentito
parlare, ma l'unico suono che giungeva attraverso quei pesanti battenti
era il mormorio della folla. Indugiai per un momento nell'atrio,
ignorando i cinque opliti vicini alla sala del trono, poi uscii
attraverso una posterla, scesi una stretta scala e sbucai nel cortile.
Consapevole dei pericoli, attivai il mio scudo personale mediante il
congegno di controllo che avevo alla cintura e sentii la cortina di
energia che mi si attivava intorno, poi mi immersi nella folla, vestito
abbastanza semplicemente da non attirare un'enorme attenzione. Alcuni
sguardi mi seguirono, ma tanti fra i presenti erano abituati a vedermi e
non si mossero dal loro posto, rimanendo in silenzio. Anche se allora
non lo sapevo, tre peltasti -- senza scudo e in armatura leggera -- si
erano mossi dal perimetro per tallonarmi.

Per quanto io sia basso, i plebei della folla erano più bassi di me e
potevo vedere al di sopra delle loro teste grigie o avviate alla
calvizie tanto la porta principale quanto la balconata che avevo
ignorato. Un nuovo istinto mi indusse a tenere la mano stretta intorno
all'impugnatura del coltello. L'intera popolazione del castello era
presente, raggruppata intorno alla statua di lord Julian, premuta spalla
contro spalla come legionari che dormissero nelle cuccette di un
trasporto per le truppe. Continuai ad avanzare nel tentativo di avere
una migliore visuale della piattaforma appena al di fuori della porta
della Rocca. Qualcuno aveva portato l'enorme podio d'ebano fuori dalla
stanza verde e lo aveva sistemato vicino alla balaustra, e droni muniti
di videocamere seguiti da quelli del Meidua Broadcast si spostavano
senza meta sopra la calca. Dieci opliti in armatura da battaglia
completa erano schierati sull'attenti lungo la scala doppia che si
avvicinava alla piattaforma, con la lancia a energia pronta e la cortina
dello scudo che scintillava.

Le campane smisero di rintoccare e un araldo -- un ometto dalla pelle
ruvida e rossa -- emerse trasportando la bandiera dei Marlowe su un'asta
d'argento. «Vi presento lord Alistair Diomedes {Friedrich} Marlowe,
arconte della prefettura di Meidua per ordine di Sua grazia lady Elmira
Kephalos, viceregina della provincia di Auriga, duchessa di Delos,
arconte della prefettura di Artemia, e nostro signore del Riposo del
Diavolo per volontà di Sua radiosità imperiale, l'imperatore William
XXIII del Casato Avent.» La sua voce limpida e acuta lo indicava come un
omuncolo, uno dei piccoli androgini non del tutto umani generati e
modellati per ruoli come questo. Il sistema sonoro sulla piattaforma ne
amplificava la voce una dozzina di volte, ed essa si allargò sulla
piazza fino ai muri che circondavano il colonnato.

Dagli altoparlanti uscì il suono registrato di trombe marziali e la loro
fanfara andò a sbattere contro le nostre cupe torri, rimbalzando sui
contrafforti murari e le finestre a sesto acuto fino a soffocare il
mormorio della folla irrequieta. Apparve mio padre, fiancheggiato da sir
Roban e da sir Ardan che indossavano la loro armatura migliore, e con la
dama Uma e gli altri littori disposti alle sue spalle con la lancia
pronta in difesa del loro signore. C'era anche Crispin, accanto a tor
Alcuin, e non mancavano gli onnipresenti Eusebia e Severn, per
l'occasione abbigliati come mio padre con una redingote nera e la toga
rossa che ci si aspettava in apparizioni così formali. Il volto di mio
padre appariva stranamente inespressivo, e nessuno era stato mandato a
chiamarmi. Mi avevano già dimenticato? E dov'era sir Felix? Di certo il
castellano avrebbe dovuto essere con il resto del gruppo.

Le trombe tacquero quando mio padre sollevò una mano. Riuscivo a
cogliere il vago bagliore dello scudo intorno al suo gruppo. Il suo
sguardo esaminò la folla ma mi passò sopra senza vedermi mentre lui
alzava la voce, dicendo: «Mio popolo, vengo davanti a te con notizie
sconvolgenti.» Da quel maestro che era, fece una pausa lunga appena
abbastanza da permettere alle diverse possibilità di cominciare a
formarsi nella mente delle centinaia di convenuti, accumulandosi fino a
colare fuori nella forma di sussurri. Concesse il tempo sufficiente
giusto per lasciar loro credere che si trattasse dei Cielcin e che il
loro mondo stesse per finire. Ci credetti anch'io. Anche se solo per un
secondo, credetti che fosse imminente un'invasione, mentre invece era
solo il mio mondo che stava finendo. «Come molti di voi sanno, il mio
figlio maggiore, Hadrian, dovrà lasciarci entro pochi giorni.» Quasi non
colsi le sue parole successive perché ero pietrificato: aveva usato il
mio nome, ma lord Alistair non lo faceva mai. Una voragine cavernosa mi
si spalancò nello stomaco. «Intraprenderà un viaggio fino a Vesperad per
prendere il suo legittimo posto come pio membro della Sacra Cappellania
della Terra.» Serrai la mano intorno all'impugnatura del coltello, con
le dita che dolevano per il ricordo della sofferenza. Pio, come no! Mio
padre però stava ancora parlando e il suo tono si era fatto più
sommesso, perfettamente amplificato e modulato dal sistema sonoro della
piazza. Quel suo improvviso abbassare la voce spinse la folla a farsi
più vicina mentre lui proseguiva: «Questo nobile piano è stato messo a
rischio da un \emph{traditore}.» A questo punto chiuse le mani a pugno
davanti alla faccia prima di abbassarle verso il podio. «Un traditore
che avrebbe voluto vedere \emph{mio figlio} rapito e venduto agli
Extrasolari.»

«Cosa?» gridai, spingendomi in avanti fra le espressioni di sorpresa dei
logoteti e dei servitori che avevo intorno. Le mie parole però si
persero nell'improvviso tremore della folla, nel rombo crescente creato
dalle persone che conferivano le une con le altre, avanzando
interrogativi con un'unica voce. Cosa stava combinando mio padre? Mi
guardai intorno, quasi che avessi potuto trovare una risposta scritta
sul volto di quelle centinaia di persone radunate, ma non ce n'era
nessuna. Poi lo vidi. Un semplice palo di legno, alto mezza volta in più
rispetto al più alto dei palatini, spiccava leggermente storto fra
quegli ampi gradini, inserito in un buco nel terreno, con la sua ombra
curva che puntava dritta verso di me.

Un palo per la fustigazione.

Il cuore mi si raggelò nel petto e mi girai per fissare con occhi
roventi mio padre sul suo podio, in tempo per sentirgli dire: «Se il suo
piano non fosse stato scoperto, mio figlio sarebbe stato consegnato
nelle mani di quei barbari, forse perfino agli stessi Pallidi.» Scosse
il capo. «Portatelo fuori.»

Spostai lo sguardo sconvolto in tempo per vedere quattro opliti
trascinare in mezzo a loro una patetica figura dai capelli bianchi. Una
figura canuta che portava una veste verde. «Gibson!» Non era vero, non
poteva esserlo. Mi lanciai in diagonale fra la folla, spingendo da parte
i plebei, ma le tre guardie che non avevo visto scattarono in avanti e
mi afferrarono per le braccia. Lo scoliasta guardò verso di me e posso
giurare che sorrise... non quel piccolo accenno di sorriso sardonico che
di tanto in tanto gli affiorava sulle labbra, ma un vero sorriso,
piccolo, triste e sicuro. Lottai contro le guardie, ma il vecchio scosse
il capo. Mi girai, e volevo gridare che non era vero niente, ma poi
ricordai la lettera nascosta fra le pagine di un vecchio romanzo di
avventure e il significato del gesto di Gibson mi risultò più che
chiaro. Ricacciai indietro le mie proteste e mi girai a guardare mio
padre con occhi dilatati, per cui non vidi che le braccia sottili di
Gibson venivano sollevate con le manette che tintinnavano nell'essere
passate nel gancio in cima al palo per le fustigazioni.

Le guardie che mi attorniavano formarono un cuneo che aprì un passaggio
nel mare di uniformi grigie e rosse, costringendomi a salire i gradini
di marmo bianco della piattaforma mentre mio padre osservava il mio
avvicinarsi con occhi velati. «Tor Gibson,» disse «sei stato sorpreso a
complottare con alcuni barbari per rapire mio figlio!» Mentre parlava le
guardie spinsero Gibson a terra, dove giacque sotto la balconata come
una marionetta a cui avessero tagliato i fili.

Una parte clinica e distaccata della mia anima rifletté che questo era
esattamente ciò a cui Gibson aveva fatto riferimento quella mattina --
era successo solo quella mattina? -- quando aveva parlato delle brutture
del mondo.

Mio padre stava ancora parlando. «Ci hai serviti per trecentodiciassette
anni, Gibson, e tuo padre prima di te. Sei stato con noi per
trecentodiciassette anni.» Mio padre avanzò verso la ringhiera e allargò
le mani sulla balaustra. «È vero che intendevi che mio figlio fosse
consegnato nelle mani di questi barbari dello spazio?»

Gibson si sollevò sulle ginocchia e guardò in su verso di noi che
eravamo sulla balconata. Mi colse a fissarlo e scosse il capo con
decisione, in maniera significativa. «È vero, sire.» Scosse ancora la
testa, questa volta più bruscamente, e compresi che quel gesto era per
me. Il contrasto fra parole e azione sfuggì a quanti erano raccolti lì
intorno, perfino ad Alcuin e ad Alma, che avrebbero dovuto notarlo. Era
destinato a me, in modo che sapessi la verità e non interferissi.

Ho avuto a disposizione anni per pensare a questo. A quel momento. Anni
per decifrare il ragionamento di Gibson, per capire perché abbia fatto
quello che ha fatto e si sia addossato la colpa del mio tentativo di
fuga. Forse, lettore, tu lo capisci? Deve aver saputo del denaro o aver
dedotto che avevo fatto qualcosa di quel genere. Se mio padre e i suoi
servizi segreti avessero creduto che il piano era suo avrebbero potuto
non indagare troppo da vicino su di me. La sua confessione mi salvò da
ulteriori indagini e rese possibile tutto quello che seguì. Oh, mio
padre avrebbe dato a me \emph{} la colpa per l'\emph{idea} di fuggire,
ma nella sua mente un piano così grandioso richiedeva una cospirazione,
il genere di cosa che uno scoliasta poteva elaborare.

Non c'era nessuna cospirazione.

C'ero soltanto io.

L'ho già detto prima, Gibson era la cosa più vicina a un padre che abbia
mai conosciuto, e quel momento cementò questo dato di fatto più di
qualsiasi altro. Gibson è morto per me. Non là e non in quel momento, ma
sul lontano pianeta dove era in esilio. Ha dato la sua vita per me, ha
rinunciato al comodo posto che aveva alla corte di mio padre perché io
potessi avere una possibilità di vivere la vita che volevo. Sono lieto
che non abbia visto il mio futuro, perché non è stato un futuro che lui
o io avremmo scelto per me, pieno com'è stato di difficoltà e di
sofferenza.

«Non lo neghi?» La voce di mio padre non tradiva nessuna emozione,
avrebbe potuto parlare a un nemico catturato sul campo e non all'uomo
che aveva istruito i suoi figli per tutta la loro vita e che prima era
stato il suo insegnante.

Gibson riuscì a rimettersi in piedi. «L'ho già confessato, sire.»

«In effetti lo hai fatto» disse una delle guardie sulla piattaforma, e
riconobbi la voce di sir Roban. Sir Roban, che mi aveva salvato dai
banditi fuori dal colosseo, il fedele littore di mio padre. «L'ho
sentito io stesso da lui, mio signore.» Attivò una funzione di playback
sul suo terminale da polso, interfacciando la sua armatura con il
sistema sonoro della piazza.

La voce sottile di Gibson giunse da ogni angolo del cortile, amplificata
artificialmente come quella di un qualche anziano e tremulo gigante.
«Hadrian non avrebbe mai raggiunto Vesperad. Avevo preso accordi...» La
registrazione si fermò bruscamente, senza dubbio interrompendosi appena
in tempo per preservare la finzione scelta da mio padre, che coinvolgeva
i demoniaci Extrasolari.

Un fremito corse fra la folla. Gli scoliasti erano visti con sospetto
dagli ignoranti, com'è destino di ogni uomo di sapere di ogni {epoca}.
Negli scoliasti c'era qualcosa che ricordava le macchine, un residuo
dell'antica storia dell'ordine e della sua fondazione da parte dei pochi
Mericanii sopravvissuti alla Guerra di Fondazione. Inoltre, in ogni
epoca c'è uno stigma annesso all'estrema intelligenza, perché ci sono
vette a cui può essere spinta la mente umana che lasciano stupefatti
quanti rimangono al livello del mare.

Sono sempre stati il primo bersaglio delle epurazioni e
dell'Inquisizione.

«Tor Gibson, per aver cercato di rapire mio figlio io, Alistair del
Casato Marlowe, arconte della prefettura di Meidua e signore del Riposo
del Diavolo, ti rinnego e ti bandisco dalle mie terre e da questo
mondo.»

Lo scoliasta si accasciò, facendo tintinnare le catene del palo a cui si
aggrappava.

«Non è vero.» Tutti quelli che mi avevano sentito, sulla piattaforma e
fra la folla, si girarono verso di me.

«È vero!» Con voce rotta, Gibson avanzò di mezzo passo, tendendo le
braccia bloccate dalle catene. «Un prete, Alistair? Getteresti tuo
figlio fra le mani di quei ciarlatani? Tuo figlio...»

«Eresia!» stridette Eusebia, puntando contro di lui un dito nodoso.
«Vostra signoria lo deve uccidere!»

Lord Alistair ignorò la priora. La familiarità nella voce di Gibson lo
spinse a perdere la compostezza -- nessuno dei suoi servitori lo
chiamava semplicemente `Alistair'. Mai. «Lui è mio figlio, scoliasta,
non il tuo!»

Uno dei soldati sferrò al vecchio un calcio dietro il ginocchio e Gibson
crollò come una torre che si rovescia. Le catene arrestarono la sua
caduta e lui gemette nel rimanere appeso là. Un grido involontario mi
sfuggì dalla gola e mi portai contro la ringhiera. «Lascialo andare.» Mi
girai di scatto verso mio padre, con le lacrime agli occhi. «Per
favore.»

«Lo sto lasciando andare.» Questa volta mio padre non si girò a
guardarmi. Sollevò una mano e schioccò le dita. «Noi siamo
misericordiosi come lo è la Madre Terra» dichiarò, rivolto alla folla.
«Gibson ha servito a lungo il nostro Casato, e in memoria del suo
servizio non ricorreremo alla Spada Bianca.» Ignorò le proteste di
Eusebia, che si arrese tanto in fretta da farmi capire che erano
arrivati in precedenza a un accordo. «Sir Felix!»

Il castellano emerse dall'ombra e io impallidii. Sir Felix, che dopo
Gibson era stato il mio miglior insegnante, non portava l'armatura da
combattimento ma il saio religioso bianco e nero. Accanto a lui c'era un
cathar della Cappellania, con la testa rasata, la pelle simile a
porcellana e una benda di mussola nera che gli copriva gli occhi. Senza
parlare, sir Felix guidò il torturatore giù per i gradini fino a dove
Gibson era legato al palo.

«Fermi!» gridai, incespicando verso le scale. Le mie guardie mi
afferrarono sopra i gomiti e mi tennero indietro con le mani coperte
dall'armatura. Il cathar avanzò verso Gibson estraendo una lama sottile
che risplendeva incandescente come una stella nella luce diurna. Senza
esitare infilò quella lama sottile come un ago nella narice di Gibson e
tirò, con il filo rivestito di plasma che tagliava un cuneo nella narice
del vecchio, su fino all'osso. Gibson emise un grido e un sussulto di
dolore, accasciandosi. Quella ferita -- cauterizzata dal plasma mentre
si praticava il taglio, in modo che la sua faccia fosse priva di sangue
-- lo avrebbe marchiato a vita come un criminale.

Sir Alistair agitò una mano e i due peltasti che trattenevano Gibson gli
strapparono la veste, che si allargò come due ali spezzate sulle sue
esili spalle, con la pelle pallida visibile fra di esse. Sir Felix
accettò una frusta con tre lacci dal cathar, agendo come sostituto di
mio padre per la punizione imminente.

«Pietà, padre.» Mi sforzai di liberarmi dagli uomini che mi trattenevano
mentre mio padre mi ignorava.

I due soldati costrinsero Gibson a addossarsi al palo, uno di essi
premendogli la faccia contro la grana ruvida del legno, poi il
castellano colpì, lacerando la carne dello scoliasta come se fosse stata
stamigna. Il suono emesso da Gibson mi sconvolse fin nel midollo: un
ululato lacerante e miagolante. Da dove mi trovavo con le braccia
bloccate non potevo vedere la sua schiena ma potevo immaginare le
strisce sanguinanti disegnate dalla frusta. Essa tornò ad abbattersi e i
singhiozzi di Gibson si trasformarono in urla. Per un momento vidi il
suo volto, trasfigurato: il vecchio che avevo conosciuto per tutta la
vita, con il suo flusso infinito di citazioni e di quieti ammonimenti,
era scomparso e il dolore lo aveva ridotto a qualcosa di meno dell'ombra
di un uomo.

La frusta tornò a ricadere ancora, per quindici volte, ciascuna
punteggiata dai singhiozzi e da un altro grido del vecchio scoliasta.
Non ho mai dimenticato il suo volto. Quando penso alla forza non mi
vengono in mente gli eserciti che ho avuto modo di vedere, o i
combattenti. Si tratta di Gibson, chino, sanguinante, ma privo di
vergogna.

Mio padre abbassò in silenzio lo sguardo su di lui. «Portatelo via»
disse, quando sir Felix fu tornato al suo fianco con la frusta in mano.
Poi si rivolse ai logoteti e segretari assiepati nel cortile, ordinando:
«Tornate al lavoro! Tutti quanti!»

«Perché lo hai fatto?» Mi liberai dalle guardie che mi trattenevano,
colpendone una alla faccia con il gomito. Davanti al pianerottolo, la
folla si stava riversando di nuovo nel castello, ridotta al silenzio
dallo spettacolo. La vista del sangue, così divertente nel Colosso,
perdeva tutta la sua componente di commedia e di gloria quando si
abusava in quel modo di uno di loro.

«Perché, \emph{Hadrian}, non si poteva trattare di te» ribatté mio
padre, con la voce bassa non più amplificata dal sistema sonoro della
piazza.

Mi paralizzai, stordito, non tanto dalla crudeltà di quella rivelazione
quanto dal fatto che mio padre aveva usato di nuovo il mio nome. Lui
segnalò a Felix di allontanarsi e intrecciò le mani davanti a sé, come
in preghiera. Quel giorno indossava tutti i suoi anelli, tre su ciascuna
mano, con un enorme castone decorato da rubini o granati o corniola,
ciascuna pietra in qualche modo simbolo della sua carica o del suo
potere. Non vidi la fede nuziale. Non la vedevo mai.

«Di me?» riuscii infine a dire, con voce che suonava debole e acuta,
come quella di Crispin.

«So che hai chiesto al vecchio di trovare un modo per farti arrivare in
un ateneo» spiegò lord Alistair, sempre in tono sommesso. Gli ultimi
componenti della folla si erano ritirati oltre la Porta Cornuta e sul
sentiero che scendeva dalla nostra acropoli. «Non so come progettasse di
farlo... senza dubbio qualcosa che aveva a che fare con una nave carica
di scoliasti.»

Questo è quello che credi, pensai, cercando di impedire al veleno
interiore di trasparirmi dal volto. Pensai alla lettera di Gibson e
compresi che adesso una sua parte era inutile. Qualsiasi cosa lo
scoliasta avesse pianificato era stata scoperta.

Il signore del Riposo del Diavolo mi si avvicinò di un paio di passi,
incombendo su di me. Crispin era in disparte da un lato e osservava la
scena con occhi scintillanti e un piccolo sorriso che gli incurvava le
grosse labbra. «Non diventerai uno scoliasta. Lo capisci?»

Non avevo parole. Non avevo niente. Dovevo partire entro cinque giorni,
per non tornare più indietro, e non avevo niente. «Vai all'inferno.»

Mio padre mi colpì di nuovo in piena faccia, questa volta con il pugno
chiuso. Uno dei suoi anelli mi raggiunse alla guancia e lacerò la carne
in una sottile linea irregolare. «Portatelo nelle sue camere. Non vuole
imparare la lezione.» Mentre i due opliti mi afferravano per le braccia,
lui si girò e condusse Crispin su per le scale che portavano alla rocca,
poi si fermò e si girò con la mano sulla spalla di mio fratello. «Fallo
di nuovo e ucciderò lo scoliasta. Hai capito?»

Sputai sulle piastrelle alla base della scala e distolsi lo sguardo.

\begin{figure}
	\centering
	\def\svgwidth{\columnwidth}
	\scalebox{0.2}{\input{divisore.pdf_tex}}
\end{figure}

Di nuovo solo nelle mie camere, non avevo spazio per le lacrime. Mi
lasciai ricadere contro la porta e mi accasciai sul pavimento con il
respiro affannoso. Per lungo tempo rimasi con gli occhi chiusi e con la
mente che si ritirava al di là del sonno e della follia, in un posto di
quiete assoluta dove l'ira si consumò fino ad ardere piano. Rimasi lì
seduto a lungo, con le gambe allargate sulle piastrelle davanti a me,
mentre potevo sentire il sangue che mi circolava dentro e le lacrime che
cominciavano a sgorgare. Vagamente, ricordai il mio sogno, in cui Gibson
era in piedi, eretto, con il naso tagliato, e la cosa mi meravigliò al
punto che aprii gli occhi.

La mia meraviglia svanì all'istante, sepolta sotto una gelida paura. La
mia giacca... quella che avevo gettato sul baule... giaceva sulle
pellicce ai piedi del mio letto, il che era sbagliato. Quasi strisciando
raggiunsi in fretta il baule e aprii il coperchio, gettando di qua e di
là i capi di vestiario e i detriti della mia vita. Poi dovetti calarmi
un pugno sulla gamba per soffocare un suono che era allo stesso tempo un
lamento di dolore e un ululato di furia, unito all'improvvisa
convinzione che anch'io ero stato sconfitto.

Il libro di Kharn Sagara era scomparso.
