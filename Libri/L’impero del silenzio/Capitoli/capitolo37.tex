\chapter{Potrebbe non morire mai}

Festeggiammo in città, portando la borsa con il nostro bonus in un bar
dopo l'altro finché soldi e oscurità non si esaurirono e il sole sorse
in un alone di fuoco. Molti dei miei compagni non ricorderanno nulla di
quella sera vittoriosa, ma per me resta eternamente luminosa. Non
comprai niente, risparmiando ogni monetina e ogni banconota
accartocciata da cinque kaspum della mia parte del bonus. Dovevo pensare
alla nave. Comunque non mi mancò da bere, non mancò a nessuno di noi,
solo che gli altri spesero come se non avessero pensato di vedere
l'indomani, mentre io non pensavo quasi a nient'altro.

Si potrebbe pensare che fosse un'occasione triste, con i bicchieri
levati alla memoria di Keddwen e degli altri che avevamo perduto, ma
anche se ci fu un po' di questa tristezza e l'indomani ci sarebbero
state offerte bruciate davanti alle icone della morte e del Coraggio
della Cappellania, comunque gioimmo, perché eravamo giovani e forti,
vivi, e in quel momento ci sentivamo certi della nostra immortalità.
Levammo un bicchiere ai nostri morti e parecchi a noi stessi, e anche se
molti di noi dissero che l'indomani avremmo desiderato di essere morti,
nessuno diceva sul serio. Che fossero dannate le emicranie, perché erano
una cosa passeggera e noi sentivamo che avremmo potuto non morire mai.

Quella fu la prima di molte vittorie del genere, e con il tempo la
nostra piccola banda di mirmidoni divenne famosa, celebrata dai fan che
ci salutavano. E così camminavamo impettiti nelle strade in cui un tempo
avevo corso o mi ero nascosto per paura dei prefetti e degli altri
criminali. Non parlerò oltre di quel festeggiamento -- o di qualsiasi
altro -- se non per quel momento che vivemmo sulla strada fra il
Distretto Bianco e la mole incombente del colosseo, perché nella fioca
luce rossa del mattino passammo vicino a un piccolo caffè con una
ringhiera di ferro. Il cielo era cupo, appena colorato dal fuoco del
sole, e il vento notturno era ancora fresco e umido come il respiro di
una grotta. La vista di quel caffè accese qualcosa dentro di me,
riattizzò la memoria di una conversazione avuta in precedenza con
Pallino e con Elara, un'altra veterana dell'arena. Quel giorno non aveva
combattuto con noi perché apparteneva a un'altra squadra, ma era noto
che lei e Pallino erano amanti, e lui l'aveva portata con noi.

«Sentite,» dissi, con la voce appena un po' impastata, mentre
camminavamo barcollando «non possiamo continuare così per sempre.» Feci
un gesto vago con la mano. «Switch e io ne abbiamo parlato, e quando il
contratto scadrà prenderemo la nostra paga e cercheremo un prestito per
comprare un'astronave.» Mentre lo dicevo barcollai contro la ringhiera
del caffè, a meno di tre metri da dove il Corvo mi aveva aiutato a
nascondermi, una vita prima.

«Dopo solo un anno?» Pallino si grattò la mascella ispida di barba,
sorretto da Elara che era appesa al suo braccio. «Con quei seimila non
potreste permettervi neppure una bagnarola.»

Sfoggiai un sottile sorriso, barcollando un poco. «È per questo che ne
sto parlando con voi due.» Posai una mano sulla spalla di Pallino. «Eri
un legionario, con una ferma trentennale, giusto?» Ridussi il numero di
anni di proposito. «Devi essere stanco di questa vita.»

«Due ferme di vent'anni, ragazzo, e tu lo sai!» brontolò Pallino,
tirando Elara più vicina. Lei strillò e il vecchio mirmidone proclamò,
con voce alta e da ubriaco: «Ho affilato la spada per la prima volta su
Sulis!»

Elara gli assestò una manata. «Questo lo sanno tutti, caro.»

«Ho ucciso quaranta Pallidi per Sua radiosità!» continuò Pallino, per
chiunque volesse ascoltarlo, poi mi passò un braccio intorno alle
spalle. «Per l'imperatore, chiaro? Non per te, vostra Hadrian-ità.
Ità...»

Sapevo che non era il momento per una conversazione come quella, ma
eravamo tutti ubriachi, ebbri del sapore del sangue e della vittoria.
«Switch e io pensavamo che voi due potreste unirvi a noi. Senti,
compriamo la nave insieme, dividiamo le quote...»

«Ne possiamo parlare» replicò Elara, guardando da sopra la spalla verso
Switch che, perlopiù sobrio, stava aiutando Erdro che si sentiva male.
«Se il prostituto non morderà la polvere alla Festa d'Estate.»

«Non lo farò!» protestò Switch, con la faccia rossa come i suoi capelli
per l'effetto del vino e per il coraggio derivane dalla sopravvivenza.
«Ti farò rimangiare quelle parole!»

«Lo spero, ragazzo» ribatté lei, non senza gentilezza, ignorando Kiri
che le gridava di lasciare in pace Switch. «Però un anno è lungo. È un
miracolo che il vecchio Pal e io siamo ancora vivi dopo cinque e tre
anni.»

Pallino sgranò il singolo occhio. «Un miracolo? Cose del genere non
esistono, donna. Si tratta di abilità.» E da lì il vecchio veterano si
lanciò in un'altra delle sue fin troppo familiari filippiche su come i
gladiatori non fossero veri soldati e non potessero tenere testa a
quelli che lo erano. Si percosse il petto in un'imitazione
approssimativa del saluto dei legionari. «Rimanere vivo mi è costato un
occhio, il che è più di quanto possano sostenere quei pederasti in
armatura verde.» Tossì un poco e si fermò per barcollare, incerto sulle
gambe nonostante la donna che lo sorreggeva.

Scossi la testa, esasperato. Pallino mi piaceva, aveva un rude fascino e
una spavalderia che parlavano a una mia componente atavistica, come se
lui fosse stato -- per usare le parole di Switch -- un vero uomo, ai
tempi in cui questo significava soltanto una cosa. Nonostante questo, in
lui c'era un'onorabilità che scorreva nel profondo, e non perdeva la
testa nel fitto della mischia, come si addiceva a un veterano con
quarant'anni di anzianità. Mi chiesi quale fosse la sua età e ragionai
che doveva avere in sé più di qualche goccia di sangue patrizio, perché
doveva avere più di sessant'anni standard, forse anche sessantacinque, e
tuttavia si muoveva come un uomo di cinquanta, fatto di corno e di cuoio
indurito.

«Senti, Had,» affermò, sembrando di colpo più sobrio «non è una brutta
idea, ma non sai quanti soldi ci vogliono per comprare anche solo una
vecchia chiatta spaziale in grado di decollare e di rimanere in aria.
Anche con quello che io e lei abbiamo da parte non otterrai niente di
nuovo.» Scosse la testa, e dal tono della sua voce compresi che avevo
esercitato anche troppa pressione, quindi chinai la testa e lo seguii.
Non avevo avuto intenzione di comprare niente di nuovo, bastava solo che
riuscisse a volare. «Non comprerai niente con seimila in contanti. Una
nave decente vale il costo di una cittadina, figliolo.» Continuammo a
parlare con quella voce monotona e impastata che passa per un tono
sommesso fra persone veramente ubriache e felici. Dopo alcuni isolati,
Pallino mi afferrò per un braccio. «Non comprare niente con motori
ionici VX-3. Quella merda normanna ti sbatterà direttamente giù dal
cielo.» Compresi di averlo preso all'amo, almeno per il momento. Era un
inizio, il primo passo su una strada che mi avrebbe portato via da Emesh
e nei cieli, lontano da quel lungo e cupo purgatorio dell'anima. Non
dissi altro, ma le parole di Pallino avevano acceso qualcosa dentro di
me. `Il costo di una cittadina.' \emph{} Avevo ancora qualcosa che
valeva così tanto, giusto? Come potevo essermene dimenticato?

Quindi non insistetti e insieme agli altri cantai piano \emph{Fra i
	mondi così lucenti}, e per una volta -- forse per la prima volta --
compresi di essere fra amici e mi sentii appagato.


