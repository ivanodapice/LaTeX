\chapter{Per fare un mirmidone}

Il vomitorium era freddo se paragonato alla calura esterna, e attraverso
il campo statico che teneva all'interno l'aria temperata e più rarefatta
potevo vedere le onde di calore che si levavano dalla strada antistante
l'arena. Durante tutti i miei anni a Borosevo avevo evitato quella parte
della città perché non avevo un aspetto adeguato a essa, ma il denaro
che avevo rubato -- durante il furto che aveva fatto finire cinque
membri della banda di Rells in rieducazione per mano del ministero dei
Servizi Sociali e dei cathar -- era stato più che sufficiente per
rifornirmi di vestiti nuovi, semplici ma funzionali. Avevo pagato in
contanti e avevo perfino preso una camera in un albergo da quattro soldi
vicino all'astroporto che una volta mi aveva buttato fuori. Quella
stanza era poco più di un ripiano abbastanza alto da potercisi sdraiare,
ma stavo dormendo in un vero letto per la prima volta dalla morte di
Cat.

Sull'impulso di un capriccio di cui mi ero pentito quasi all'istante
avevo acquistato un rasoio economico nel dispensario nell'atrio
dell'albergo, felice per l'aria fresca, la sicurezza e il fatto che
nessuno mi guardasse con sospetto. I miei capelli erano un incubo, un
groviglio mostruoso controllato da un elastico che lo teneva tirato
all'indietro sulla nuca. Lo rasai via, tutto quanto, fino a essere calvo
come un uovo, poi gettai i resti in un inceneritore appena fuori da una
delle toilette a pagamento dell'albergo, felice di non avere più
l'aspetto di un completo idiota.

Anzi, mentre mi dirigevo nel vomitorium con passo deciso, passando fra
le bandiere dipinte con la sfinge di giada del Casato Mataro, la mia
figura era scarna e terribile a vedersi. Intravidi il mio riflesso
allungato sui massicci gong di ottone che fiancheggiavano quello spazio
affollato. Ripensandoci, somigliavo un poco a un Cielcin, tutto statura
e snelli muscoli, con la pelle ancora pallida nonostante i recenti anni
di stenti. Mi mancavano solo le corna e gli occhi enormi.

Una fila di donne con un paniere in equilibrio sulla testa si spostò
dalla mia strada e perfino un enorme omuncolo nell'uniforme rossa dei
servizi del Colosso indietreggiò, chinando la testa con qualcosa che si
avvicinava alla deferenza. L'anello pendeva pesante dalla corda che
avevo intorno al collo, ricordandomi della sua presenza e quasi
chiedendo di essere portato di nuovo, ma quella sarebbe stata una cosa
disastrosa. Sapevo che questo mio piano mi avrebbe tolto dalle strade di
Borosevo, ma avrei dovuto essere estremamente cauto. Non potevo farmi
prendere come un gladiatore perché non avevo referenze e non volevo
essere assoggettato a un esame rigoroso. Quello che avevo erano quasi
due decenni standard di addestramento al combattimento, per non parlare
degli anni vissuti in strada a vedermela con Rells e la sua banda.

Fermai quindi il primo assistente che riuscii a trovare, una donna calva
quanto me, posandole una mano sul braccio mentre chiedevo cortesemente:
«Accettate persone per la scorta di sacrificabili?» Mi dimenticai di
sorridere per almeno cinque secondi e quando ci provai fallii il modo
disastroso, almeno a giudicare dall'espressione sul volto di quella
povera donna.

Lei sgranò gli occhi nel notare il mio aspetto... come la camicia
aderisse a muscoli d'acciaio e la linea dura delle mie ossa. Dopo un
momento annuì.

\begin{figure}
	\centering
	\def\svgwidth{\columnwidth}
	\scalebox{0.2}{\input{divisore.pdf_tex}}
\end{figure}

Sedevo sul bordo del lettino medico su uno strato di plastica sanitaria,
nudo e privo di timore. Una luce tremolava in un angolo del basso
soffitto, proiettando ombre sulla fila di apparecchiature mediche
inattive. Se non provieni dal mio Impero, forse non hai familiarità con
i nostri grandi giochi, la loro meccanica e le loro regole, le loro
tradizioni. Ci sono i gladiatori, che sono gli eroi di un milione di
opere, i campioni della stagione sportiva. I bambini conoscono il loro
nome, indossano i loro colori e il loro numero, seguono le loro gesta.
Perfino in tempo di guerra la gente li tratta come eroi, alla pari dei
nostri cavalieri e soldati. Combattono gloriosamente uno contro l'altro
o in piccoli gruppi, e se rimangono feriti vengono scortati fuori
dall'arena, consegnati agli scoliasti e risanati per combattere ancora.

Poi c'è il gruppo dei sacrificabili, i mirmidoni. Arrivano come
criminali e schiavi, oppure spinti dalla fame e dalla promessa di un
pasto unita alla speranza di riuscire a sopravvivere per uno o due
scontri. Arrivano disperati, ubriachi o drogati. I mirmidoni sono uomini
distrutti, pazzi e disperati. Sono uomini rabbiosi e suicidi. Io non ero
nessuna di queste cose. Ero uno stolto della specie più rara. Ero
determinato.

In realtà, mi ero aspettato che mi accettassero come sacrificabile per
il Colosso senza neppure un contratto di una sola pagina di liberatoria
da parte di qualsiasi parente e rinuncia ad azioni legali contro i
proprietari dei giochi -- nella fattispecie il Casato Mataro di Emesh e
il suo conte, lord Balian -- ma firmai un contratto di quel genere e
venni sottoposto a un esame medico.

Il dottore del Colosso, una donna, portava un collare di bronzo da
schiavo e aveva la narice tagliata di una criminale, insieme a un
tatuaggio punitivo sulla fronte che indicava con chiarezza il suo reato:
\foreignlanguage{italian}{disertore}. Aveva altri tatuaggi su un
braccio: un falco araldico sull'interno del polso e un serpente
arrotolato sull'altro per nascondere quelle che sembravano cicatrici da
ustione.

«Un altro per l'arena, eh?» Mi adocchiò da sotto le sopracciglia troppo
lunghe. Un occhio era chiaramente di vetro e puntava nella direzione
sbagliata, l'altro risplendeva scuro nel volto avvizzito. La parola
tatuata sulla sua fronte si corrugò quando mi studiò con l'occhio sano,
le mani piantate sui fianchi. «Qual è la tua motivazione? Fama e
fortuna?» Tirò su con il naso, spingendo indietro le maniche sporche e
infilandosi un paio dei guanti sterili tenuti in un contenitore sul
piano di lavoro.

Mi schiarii la gola. «Cerco solo di tirare avanti.»

«Cerchi di tirare avanti?» La donna sbuffò di nuovo e si fece più
vicina. «Cosa c'è? Il ministero dei Servizi Sociali non ha più bisogno
di sgherri che percuotano gli Umandh per sottometterli?»

Un lampo dell'antica altezzosità aristocratica divampò sotto la
superficie e reagii con irritazione. «Non sono uno sgherro.»

«Oh, scusami tanto» ribatté la donna, azzannando ogni parola come carne
secca. «Credevo che stessi cercando di entrare nella dannata arena. Non
mi dire che non sei uno sgherro.» Mi assestò un colpo sul braccio.
«Spostati. Inutile nascondere il pisello, ragazzo. Qui non importa a
nessuno.» Allontanai lentamente le mani senza guardarla in faccia. «Sei
un ragazzo forte, su questo non ci sono dubbi.» Pungolò una cicatrice
che avevo lungo le costole. «Hai una storia di violenza, vero?» Quando
non risposi subito, mi pungolò ancora.

«Qualche rissa» ammisi.

«È inutile essere così fottutamente conciso.» Mi squadrò con irritazione
mentre l'occhio di vetro guardava di lato qualcosa che io non potevo
vedere. «Hai un nome?»

«Had.»

«Che razza di fottuto nome è Had?» Si interruppe, dirigendosi al bancone
sulla parete opposta per poi tornare indietro con uno stetoscopio e una
sonda di scannerizzazione fra le dita storte. «È il diminutivo di
qualcosa?»

Rimasi in silenzio per un momento, guardandola contare le mie
pulsazioni. «Sta per Hadrian» ammisi infine.

Quel singolo occhio nero mi fissò, con la sospettosità che lo rivestiva
come uno strato di olio sulla sclera. «Hadrian?» Si accigliò. «Un nome
fottutamente strano per un picchiatore.» L'istinto mi spinse a negare
per la seconda volta di essere un picchiatore, ma percepii il pericolo.
Non vedevo videocamere da nessuna parte, ma questo non significava che
non ce ne fossero. Nessuno è mai veramente solo, non nell'Impero del
Sole, non da qualsiasi altra parte. Mi limitai quindi a scrollare le
spalle. «Bene, fai come preferisci, allora» disse. Si sfilò lo
stetoscopio dalle orecchie e lo lasciò pendere dal collo. «Io mi chiamo
Chand, anche se non te lo stavi chiedendo.»

«Chand» ripetei, cercando di piazzare la provenienza di quel nome e le
gutturali marcate del suo accento. «Non hai uno scanner somatico?»
Indicai lo stetoscopio. «Devi proprio usare quello?»

«E sei anche curioso, per essere un picchiatore. Gli scanner si
confondono.» Sollevò l'oggetto in questione, un cilindro di metallo
lungo quanto la mia mano. «È meglio ascoltare, ma comunque ti
sottometteremo a tutti gli esami. Alzati.»

Obbedii e seguii il suo gesto che indicava una bilancia in un angolo,
lasciando che mi pesasse e misurasse, rilevando anche altre misure a
parte la statura. «Vorrai avere un'armatura che ti calzi» disse, a
titolo di spiegazione, poi aggiunse: «Sei davvero in forma, non è così?
Ho visto gladiatori in condizioni peggiori delle tue.»

«Cosa intendevi con `tutti gli esami'?» chiesi, allontanando con un
colpo la sua mano.

«Volevo dire che ti sottoporrò a una visita medica vera e propria,
ragazzo. Può darsi che non sembri, ma sono capomedico qui da prima che
tu fossi una goccia nelle balle di tuo padre, quindi ora piantiamola con
le domande, eh?»

«Incluse le analisi del sangue?» persistetti, imperterrito. Lei sollevò
una mano e mi colpì a un orecchio senza replicare. Reagii con uno
strillo.

«Credevo di averti detto di piantarla con le domande.» Mi fissò con
espressione rovente e la parola tatuata sulla sua fronte si accartocciò
quando aggrottò le sopracciglia. Quando non distolsi lo sguardo dal suo
scoppiò a ridere. «Sei un duro, vero? È una cosa di importanza critica.
Sei un vero e proprio mirmidone. Alla folla piacciono quelli che non se
la fanno addosso la prima volta che vedono le Sfingi piombare loro
addosso in armatura completa. Offrirai un bello spettacolo.» Non seppi
cosa rispondere. Mi auguravo che avesse ragione. Quando mi sorprese a
continuare a fissarla con rabbia, aggiunse: «Sì, incluse le analisi del
sangue.» Mi studiò con espressione seria. «C'è qualche motivo per cui
non dovremmo includerle? Sei dipendente?»

«Dipendente?»

«Dalle droghe, ragazzo.» Mi guidò di nuovo al lettino medico e
procedette a misurare i miei riflessi e a controllare la dilatazione
delle pupille con una torcia a stilo.

Questo mi fece accigliare di nuovo. «Perché al Colosso importa se
inseriscono dei drogati nel gruppo dei sacrificabili?»

«Non importa,» ribatté, con un verso di riprovazione nel notare un'altra
sottile cicatrice su una gamba «ma se \emph{sei} un drogato, vogliono
accertarsi di tagliare quel tuo naso grazioso. Ti renderebbe più
spaventoso.» Fece una smorfia, esponendo denti sovrapposti e ingialliti,
poi lanciò un'occhiata alla mia testa rasata e continuò: «Se non altro,
posso dire che non hai i pidocchi. Che peccato.»

Sfoggiò un sorriso tanto ampio da seppellire il suo volto nelle rughe,
riducendo a mere fessure tanto l'occhio di vetro quanto quello sano.
«Non avevo mai conosciuto prima un palatino con i pidocchi.» Le ultime
parole suonarono piatte e soffocate quando indietreggiò fino a ridosso
della parete perché ero scattato in piedi. A metà del gesto mi resi
conto che quella era esattamente la cosa sbagliata da fare perché
sarebbe servita solo a confermare qualsiasi sospetto la dottoressa
avesse sul mio conto. Incurvai le spalle e mi girai parzialmente
rispetto alla schiava, che scoppiò a ridere. «Allora devo dedurre che ho
ragione? Riconosco uno di voi a un miglio di distanza. Dovrei essere
un'idiota per non farlo.»

Non negai ma neppure le risposi. D'un tratto il mio brillante piano di
entrare nel Colosso come un mirmidone sacrificabile mi parve
incredibilmente sciocco. Afferrati i pantaloni che avevo ripiegato sul
bancone, accennai a rivestirmi.

«Nel santo nome della Terra, dove credi di andare?» chiese Chand, con un
invisibile cipiglio evidente nel tono della voce gutturale. Mi
oltrepassò in fretta per addossarsi alla porta con l'occhio sano puntato
su di me mentre mi infilavo i pantaloni.

Avrei potuto spostarla da un lato, colpirla, gettarla a terra in un
istante, ma aspettai mentre procedevo ad allacciare i miei stivali
nuovi. «Tu non puoi aiutarmi. Questo è stato un errore.»

«Un errore?» Il volto della schiava assunse un'espressione pensosa e il
tatuaggio del suo crimine si distorse quando inarcò un sopracciglio.
«Non ho mai incontrato un palatino che non volesse sentire il suo nome
strombazzato dalla cima dei minareti del tempio.»

«Non sono un palatino» insistetti, cercando di individuare le
videocamere di sorveglianza che ero certo fossero presenti in quella
piccola e squallida sala medica.

«E io non sono una schiava, sono una meretrice dell'harem imperiale con
eunuchi muscolosi e abbronzati che mi ungono il posteriore a giovedì
alterni.» Non si mosse dalla porta. «Rispondi alla mia dannata domanda,
ragazzo. Qui ci siamo soltanto noi due.»

Mi bloccai a metà dell'atto di abbottonarmi la camicia. «Quale domanda?
Non me ne hai fatte.»

«Perché non stai gridando il tuo elegante nome dal minareto del tempio?»
chiese, riformulando l'affermazione di prima. «{Potremmo} fornirti
subito un'armatura adeguata al piano di sopra, mio signore.» Nella voce
della dottoressa schiava affiorò una strana nota di derisione quando
pronunciò quelle ultime due parole, una cosa che mi indusse a
raddrizzarmi di scatto in tutta la mia statura.

«Non sono un lord» ripetei.

Lei sbuffò e posò una mano sulla porta per impedirmi di andarmene, come
se avesse potuto farlo. Si erse sulla persona quanto più poteva, con i
lanuginosi capelli bianchi agitati dall'aria delle bocche di
ventilazioni. «Rispondi alla mia domanda, \emph{momak}.»

Finalmente compresi, individuai quello strano accento che non riuscivo a
mettere a fuoco. Durantino, era di Durannos, o lo era stata.
Un'ausiliaria? Era chiaro come il giorno che i tatuaggi sulle sue
braccia erano quelli delle Legioni imperiali. Avrei voluto ridere,
piangere. Un piano era affiorato dal nulla, spuntando già formato dalla
mia testa come Pallade Atena.

«\emph{Ti si od Resganat}?» domandai, esprimendomi nella lingua pesante
di quella lontana repubblica. `Vieni dalla Repubblica?'

Lei sgranò gli occhi e mi rispose nella stessa lingua. «Parli il
durantino?»

«\emph{Haan}» replicai, inclinando la testa. Dovevo giocare con cura le
mie carte. Ero già avvantaggiato dal fatto che la donna mi stava
ascoltando. Forse era corrotta e ammetteva di continuo mirmidoni nel
gruppo dei sacrificabili anche se non erano qualificati. Infilai una
mano nella tasca posteriore dei pantaloni nuovi e feci scivolare le dita
oltre l'anello con la sua corda aggrovigliata fino a tirare fuori uno
degli hurasam che avevo rubato con la banda di Rells. Glielo porsi
perché lo vedesse, con il profilo aquilino dell'imperatore che
scintillava sotto la luce. «Prendilo.»

La dottoressa diede l'impressione di voler sputare. «Cosa cazzo me ne
faccio del tuo oro, ragazzo?» Infilò un dito sotto il collare e tirò per
indicare la propria schiavitù e quanto poco valesse per lei quella
moneta.

Offrirle del denaro in quel modo era esattamente quello che si sarebbe
aspettata da me e non avevo voluto deluderla. Con quell'offerta fatta e
rifiutata, andai avanti, contando sulla rigida asserzione dei
repubblicani che classi e caste non significavano nulla. «Benissimo.
Senti...» Feci una pausa e trassi un profondo respiro. «Io non voglio
essere un lord.»

Lei mi studiò con l'occhio buono. «Perché?»

Fui più che felice di mentirle. «Nessuno dovrebbe esserlo.» Sbuffò con
evidente incredulità. «La cosa peggiore che può succedere è che resti
ucciso nell'arena, e l'universo avrà un altro palatino in meno. Eri un
soldato, giusto? Un'ausiliaria?» Indicai i tatuaggi. «Ecco la tua
occasione di ordinare a un nobile di andare incontro alla morte invece
del contrario.»

Mi guardò in modo molto strano. «Perché mai vuoi questo?»

«Non ho nient'altro» dissi. Lei parve sul punto di discutere, quindi
aggiunsi: «Ormai dormo per le strade di questa città da tre anni. Non ho
\emph{nient'}altro.» Forse ebbe pietà di me, o forse fu felice di vedere
un nobile in una posizione come la mia, ma percepii che era sul punto di
arrivare a una decisione e insistetti: «Ammettimi nell'arena. Per
favore.»

