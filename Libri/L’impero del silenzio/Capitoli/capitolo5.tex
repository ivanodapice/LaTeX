\chapter{Tigri e agnelli}

Stava emergendo un chiaro schema degli eventi, ma io ero poco più di un
bambino e non ero in grado di vederlo. Forse tu lo scorgi, forse capisci
esattamente cosa mi stavano facendo. Perché non l'ho visto, quando ero
stato addestrato per cose del genere quasi fin da quando avevo imparato
a parlare? Non lo saprò mai. Forse è stata arroganza, la sensazione di
essere migliore di Crispin, più adatto a governare, o forse è stata
avidità, o il fatto che siamo ciechi finché il coltello non ci colpisce,
perché ci crediamo immortali finché non moriamo. Il mondo in cui siamo
nati era una terra selvaggia di tigri che recitano il ruolo di agnelli.
Una volta un uomo saggio mi ha detto che la carne era la risorsa più
economica dell'universo umano e che la vita si spende più facilmente
dell'oro. Quando lo avevo sentito avevo riso e avevo respinto le sue
parole.

Sono stato uno stolto a farlo.

Quanto poco ne sapevo.

L'arco che portava alla rotonda sottostante la Cupola degli Intagli
Radiosi è impresso per sempre nella mia mente, imperituro, come simbolo
del mio fallimento. Svegliato tardi da una cameriera, oltrepassai in
fretta le porte principali per sbucare nella galleria circolare che
cingeva la camera del consiglio, muovendomi con fretta e decisione verso
quel terribile portale. La luce avvelenata del sole cadeva in strani
colori attraverso il sovrastante tetto a mosaico di vetro colorato,
proiettando ombre malate sulle antiche statue -- con la loro antica
vernice crepata e sbiadita -- che decoravano quello strano posto. Da
generazioni era usanza della mia famiglia {commissionare} una scultura
di legno alla nostra gente ogni dieci anni e le migliori fra queste
decoravano la stanza, poste in nicchie e su scaffali, fissate alle
pareti o sospese nell'aria mediante cavi, in modo che la loro ombra
riducesse a strisce la luce colorata del sole. Il resto veniva dato alle
fiamme durante la Festa d'Estate.

Era come se qualcuno avesse rimosso tutto il colore dal nostro cupo
castello e lo avesse compresso in un solo posto come uno spaventoso
segreto. Uccelli e bestie, uomini, navi e demoni scorrazzavano tutti
nello spazio, illuminati soltanto dalla luce filtrata del sole di Delos.
La porta era la cosa peggiore di tutte. Come per la maggior parte delle
porte del nostro castello, il suo arco era a punta e sulla pietra di
volta -- intagliata nell'avorio di una balena di bronzo -- c'era un
volto umano, aquilino e severo. Avrebbe potuto essere il mio, ma era il
gemello del volto della statua antistante la Grande Rocca, quello di
Julian Marlowe, che aveva eretto il castello e portato il nostro nome
alla gloria. Altre facce si raggruppavano intorno alla sua, premute
contro il muro e lungo lo stipite, cosicché erano in trentuno a guardare
da intorno alla porta, tutti di un candore osseo tranne che per gli
occhi viola. Maschere funebri prelevate dalle catacombe dove venivano
sepolte le ceneri dei membri della mia famiglia.

Li conoscevo tutti, li avevo memorizzati come una delle mie prime
lezioni.

I miei antenati, i cui tratti erano bloccati nel tempo e messi dove
tutti li potevano vedere.

Le guardie alle porte non mi opposero resistenza come avevano fatto
fuori dalla sala del trono, e a un mio cenno aprirono i pesanti battenti
di legno su cardini tanto silenziosi che il solo suono fu quello delle
mie scarpe sulle lastre di pietra. Quel rumore si perse immediatamente
sotto il mormorio di una conversazione che si levò a incontrarmi come
una marea, e io mi bloccai di colpo quando una metà delle facce raccolte
intorno al tavolo si girò a guardarmi. Solo tor Gibson sorrise, anche se
fu un sorriso breve e teso, cancellato quasi immediatamente dalla sua
disciplina emotiva. I ministri del Consorzio mi fissarono con fredda
indifferenza e Crispin -- che era là, seduto alla sinistra di mio padre
− sfoggiò quel suo tagliente sogghigno.

Mio padre non si interruppe neppure. «...licenza ci permette la
proprietà univoca di tutto l'uranio estratto nel sistema di Delos e non
solo lungo il Denterosso. Gli stranieri della cintura possono essere
portati a fornire la quantità richiesta con i giusti mezzi di
persuasione.» Da sopra la spalla lanciò un'occhiata a sir Felix, che era
in piedi di guardia alle spalle di Sua signoria. Il castellano indossava
la sua armatura migliore, quella con il nero e carminio dei Marlowe
uniti al bronzo del suo Casato minore. «Se gli operai continueranno a
fare resistenza nei nostri confronti manda sir Ardian. Lui saprà cosa
fare.»

«Gli operai della cintura si stanno ribellando?» chiese Adaeze Feng,
modulando attentamente con la sua ricca voce tanto sorpresa quanto
disprezzo. «Mi era stato dato di capire che la tua presa sulla
situazione era più salda di così, lord Marlowe.»

Il volto di mio padre mostrò ancor meno reazioni di quanto avrebbe
potuto fare quello di uno scoliasta. Si spinse indietro i capelli in un
gesto distratto. «Gli operai della cintura sono sempre in ribellione,
madama direttrice. Presentano le loro puerili lamentele, noi facciamo
qualche concessione, poi la annulliamo quando quella generazione
scompare dalla forza lavoro.»

«L'aspettativa di vita fra gli operai degli asteroidi è di circa
sessanta anni standard» aggiunse tor Alcuin, lo scoliasta con la pelle
del colore della pece che era il principale consigliere di mio padre.
«Possiamo permetterci un ciclo di concessioni per i nuovi operai nel
corso di circa un secolo al fine di ridurre le ribellioni al minimo.
Prendere e dare.» Mentre parlava mi sedetti fra due logoteti della
tesoreria di famiglia, a un buon quarto di quel massiccio tavolo rotondo
da dove c'era il grande seggio di mio padre. Percepii la tensione dei
logoteti, vidi la donna bionda alla mia sinistra girarsi per un attimo
verso di me ma la ignorai, sperando di evitare che il mio ritardo
diventasse oggetto di discussione.

Il ministro con il tatuaggio dorato sul cuoio capelluto si accigliò,
guardando direttamente verso tor Alcuin. «La viceregina lo approva?»

«La viceregina» rispose Crispin, posando il suo tablet a faccia in giù
sul tavolo «è contenta di lasciare che siamo noi a fare il lavoro
sporco. Se smorziamo le ribellioni a casa, questo permette a Sua grazia
di gestire il settore.»

«Il nostro vero punto focale devono essere i Cielcin» dichiarò Eusebia,
la priora della Cappellania di Meidua. «Tutto questo deve servire
l'imperatore scelto dalla Terra.» La vecchia era di un pallore
spaventoso, come la luce della luna, con il volto segnato come carta
spiegazzata e una voce simile all'agitarsi di ragnatele sotto un vento
leggero. Sorpresi Gibson che mi guardava e scuoteva la testa,
grattandosi una guancia. Sapevo cosa fosse la Cappellania, potere
travestito da devozione.

La direttrice agitò una mano adorna di anelli, con uno scintillio di
pietre preziose mentre sfoggiava in un sorriso quei suoi denti
metallici. «Naturalmente, priora, ma bisogna pensare alla situazione
\emph{dopo} la fine di questa guerra. Quando l'avremo vinta...» Allargò
di piatto la mano contro il legno pietrificato della superficie del
tavolo. «Vogliamo che gli accordi con Delos e il Casato Marlowe siano il
più... amichevoli possibile.»

«Quando la guerra sarà vinta?» La voce sommessa di Eusebia salì in tono
e volume, e i suoi occhi velati si sgranarono. «Non dovremmo occuparci
della piccola questione di assicurarci questa vittoria, madama
direttrice?»

Il sorriso di Adaeze Feng non si incrinò. «Di certo questo è un problema
delle Legioni, e del vostro imperatore. Io sono una donna d'affari,
priora, e sono qui per stringere un accordo con l'arconte per la
condivisione delle sue esportazioni, non per infliggere un colpo al
cuore al nostro comune nemico.»

«I Cielcin si fanno sempre più vicini ogni giorno che passa» interloquì
un funzionario minore che portava la veste nera della Cappellania e
sedeva non molto lontano da Eusebia. «Lord Marlowe, ti devo incitare a
prendere in considerazione l'alternativa. Devi fornire armi atomiche
alle Legioni della viceregina.»

Lord Alistair Marlowe non guardò in faccia quel rospo della Cappellania,
ma la sua voce profonda troncò l'improvviso scoppio di commenti che
seguì. Non alzò la voce, non gridò, ma parlò al di sotto degli altri,
indebolendoli. «Lady Elmira preleva ogni trimestre standard il quindici
percento di materia prima da noi prodotta. Non ha bisogno di altre armi
atomiche, Severn, e neppure noi. Il sistema è armato.» Lanciò
un'occhiata a Gibson. «Scoliasta, quante navi Elmira può schierare in
campo nel sistema?»

Il vecchio tossì, sorpreso di essere stato interpellato. «Secondo
l'ultima indagine da parte dell'ufficio imperiale? Un totale di
centodiciassette navi, senza contare quelle più leggere.» Fornì quindi
un elenco di dati demografici, recitando le sottodivisioni di
quell'elenco di navi.

Mio padre gli segnalò di tacere con la mano aperta, concentrando ora la
propria attenzione su Eusebia. «Visto, priora?» commentò, per poi
concentrarsi su Adaeze Feng. «C'è qualcosa che ti inquieta nello stato
degli affari locali, madama direttrice?»

Feng lo fissò con durezza per un istante, riflettendo sulle sue parole
prima di pronunciarle. «Gli operai stranieri...»

«Cederanno alle nostre richieste non appena cominceranno a morire di
fame su quelle rocce senz'aria che definiscono la loro casa» concluse
per lei mio padre, appoggiando il mento sulle mani intrecciate. «Gli
operai sul pianeta costituiscono una maggiore preoccupazione. La Gilda
Mineraria sostiene che c'è una serie di guasti nelle loro
apparecchiature di raffinazione e minerarie. I centri di arricchimento
costituiscono la principale preoccupazione perché non abbiamo modo di
sostituirli e li dobbiamo acquistare nelle vostre fabbriche.»

«E abbiamo qui una capofazione della Gilda che desidera parlare con te,
direttrice Feng» aggiunse Alcuin. «Ha i dettagli della situazione fra i
minatori che si trovano sul pianeta.»

Il giovane ministro Sun si protese in avanti. «Quale porzione di
uranio... er...» Si interruppe per mormorare qualcosa al suo vicino in
mandar, la lingua commerciale del Consorzio. Avendo apparentemente
trovato la parola che gli serviva, riprese: «Quale proporzione del
raccolto di uranio proviene dalle miniere planetarie?»

«Il trentadue percento» risposero all'unisono Gibson e Alcuin, perché i
meccanismi addestrati della loro mente risposero quasi con lo stesso
grado di precisione. Fu però Gibson ad aggiungere: «Mi dispiace dire che
però attualmente non stiamo operando a quella portata. La percentuale di
attrito fra i minatori in assenza di adeguate attrezzature di
perforazione è triplicata nel corso dell'ultimo secolo.»

Lord Alistair percosse con le nocche la superficie del tavolo. «Basta
così, grazie.»

Il direttore arricciò le labbra. «La vena di Delos non è ricca quanto lo
era quella di Cai Shen. Riparare quei macchinari di arricchimento è
un'assoluta necessità. Non vorrai mancare di fornire la \emph{nostra}
percentuale, dopotutto, vero?»

Un sorriso sottile quanto una corda di pianoforte apparve sul volto di
mio padre e il silenzio si fece teso come una garrota. Minacciare il
signore del Riposo del Diavolo era qualcosa che aveva alle spalle una
lunga storia di fallimenti. Una volta, quando mio padre era poco più
grande di me, la viceregina -- mia nonna -- era stata chiamata al
seguito dell'imperatore, a Forum. Era rimasta assente per trentasette
anni, e aveva lasciato al suo posto come esecutore il signore del Riposo
del Diavolo, rimasto recentemente orfano. Il Casato Orin di Linon ci
aveva messo meno di tre anni a cominciare a rifiutare di versare il suo
tributo a mio padre, ed entro l'anno successivo lord Orin aveva messo
insieme un esercito fra i Casati esuli per deporre mio padre e l'assente
viceregina-duchessa. Erano sciamati nel sistema dai pianeti esterni,
cadendo come pioggia dal cielo.

Non c'era stato un secondo anno della ribellione di lord Orin e adesso
il suo castello di Linon era abitato solo da spettri, una rovina
infranta in un cratere illuminato dal crepuscolo su una lontana luna ai
confini del sistema di Delos. Mio padre aveva ordinato la morte di ogni
membro del Casato Orin, la distruzione del suo stock genetico e aveva
razziato le armi atomiche della famiglia. Avrebbe sparso il sale sulla
loro terra, se questo fosse servito a qualcosa su Linon, che era privo
di aria. Così come stavano le cose, si era limitato ad aprire le
finestre della fortezza sigillata e a lasciare che l'aria fuoriuscisse
dal castello.

Credo che la direttrice dovette rendersi conto del suo errore, perché si
passò una mano sul cuoio capelluto ed ebbe la buona grazia di
distogliere lo sguardo. Non dubito che mio padre fosse consapevole di
non aver a che fare con un qualche Casato esule -- quella era la
direttrice della più grande corporazione interstellare per diecimila
sistemi stellari -- ma non cambiò minimamente espressione. «Ti ricordo,
direttrice, che non sono stato io a far fare alla mia astronave una
deviazione di parecchi parsec per poter avere questo incontro. Lo hai
fatto tu. Se credi di poter ottenere l'uranio in quantità paragonabili a
quelle che qui vengono estratte all'interno del sistema -- e di poterlo
fare legalmente -- allora io non ti fermerò. Se d'altro canto la
sfortunata tragedia di Cai Shen significa che devi fare affari con me e
con la mia infrastruttura, allora ti chiedo di smetterla con questi
giochetti e di dirmi di cosa hai bisogno.»

Sedetti in silenzio, rimpiangendo di essere venuto. La riunione si
concluse e Alcuin accompagnò i Mandari a incontrare la capofazione della
Gilda Mineraria, lasciando che i logoteti e il personale della
Cappellania si sparpagliassero più lentamente.

«Tu.» La voce di mio padre fece quella cosa allarmante, scivolando
sommessa sotto gli altri suoni fino ad azzannare come una vipera la mia
attenzione. «Rimani.»

Mi accasciai di nuovo sul mio sedile e distolsi lo sguardo, osservando
la schiena in ritirata di Eusebia e del giovane Severn, con la vecchia
priora che si appoggiava al braccio del suo subordinato. Si muovevano
come un paio di streghe d'ombra, con le vesti più scure dell'armatura
nera dei peltasti del Casato che procedettero a chiudere le porte alle
loro spalle. Vidi la figura curva di Gibson appoggiata al suo bastone,
con un'espressione aggrottata che lui non dissolse. Questo mi turbò più
di qualsiasi altra cosa... il fatto che lui non stesse controllando le
sue emozioni come avrebbe dovuto fare.

Poi rimasi solo con la mia famiglia.

«Nostra madre non è rimasta per la riunione?»

Mio padre sbuffò e si assestò i polsini delle maniche bianche sotto la
giacca scura. «Tua madre è andata ad Haspida.»

«Di nuovo?» Crispin posò il tablet e levò in alto le mani. «Era appena
arrivata.»

Lord Alistair attese per un momento, tamburellando con le lunghe dita
sulla superficie del tavolo. Il suo sguardo era fisso su una maschera di
legno chiodato a forma di cuore che era la decorazione centrale di una
parete. Nell'istante in cui anch'io spostai lo sguardo su quell'orribile
oggetto, lui disse: «Hai promesso loro assistenza.»

Turbato, mi girai con le sopracciglia inarcate. «Prego?»

«La capofazione che si sta incontrando con Feng. Hai promesso nuove
apparecchiature minerarie.»

«Balem?» sedetti più eretto. «Non ho fatto niente del genere.»

Con la sua voce profonda improntata a una calma letale, lord Alistair
troncò ulteriori proteste.

«Hai assicurato a quella dannata donna che avremmo fatto di più per
aiutare i suoi operai.»

«Dovremmo farlo, padre!»

«Hai una \emph{minima} idea di quanto costi uno di quei macchinari di
arricchimento, ragazzo?» Quando non gli risposi immediatamente proseguì:
«Appena meno di quindici milioni di marchi, e questo senza considerare i
costi di importazione e la decima da pagare alla Cappellania.» Si
protese sul tavolo, socchiudendo gli occhi. «E sai di quanti di quei
macchinari è stato denunciato il guasto negli ultimi tre decenni
standard?»

Crispin emise un verso e mi girai a guardarlo prima di rispondere. Mi
stava osservando con gli stessi occhi viola di mio padre. Pensai alle
maschere fuori dalla porta, al volto dei miei antenati, e questo mi
riempì di inquietudine, della sensazione che tutti noi fossimo nati su
ordinazione, ritagliati dalla stessa stoffa con occhi viola. Si dava
però il caso che conoscessi la risposta alla domanda di mio padre,
quindi chiusi gli occhi e dissi: «Nove.»

Crispin emise un fischio. «Nove?»

«È la Cappellania» continuai. «Se avessimo le capacità tecniche per
effettuare riparazioni su vasta scala...» Ma questo era impossibile. A
quei tempi la Cappellania controllava l'uso e il commercio di qualsiasi
macchinario complesso. Era alla ricerca dei demoni, le macchine
intelligenti con cui i Mericanii avevano oppresso il resto della razza
umana molto tempo prima e che a loro volta avevano oppresso anche loro.
Nessun mostro del genere era emerso nello spazio imperiale negli ultimi
duemila anni, ma la Cappellania era sempre di guardia, e quando un
nobile usciva dagli schemi -- costruiva una sfera dati privata, dava
ospitalità a tecnici stranieri, commerciava in tecnologie proibite con
Extrasolari o comprava un macchinario per l'arricchimento dell'uranio di
troppo senza il permesso del priore di quel grande sistema -- c'erano
delle conseguenze. A sentire loro, i demoni erano ovunque. Spettri nella
macchina. Quegli abomini aspettavano soltanto che uno stolto magio li
evocasse dai cristalli di silicio e di itterbio. I nobili che si
dilettavano di quella che era la più nera fra le arti erano assoggettati
all'Inquisizione, alla tortura per mano dei cathar. Nel peggiore dei
casi interi pianeti venivano sterilizzati, assoggettati al fuoco
nucleare o alla pestilenza, a qualsiasi orrore quei neri preti avessero
nel loro arsenale.

Consapevole di quella letale minaccia, mio padre aveva le labbra
pallide. «Vuoi l'Inquisizione, ragazzo?»

«Stavo solo dicendo che...»

«So cosa stavi dicendo.» Lord Alistair si alzò e mi guardò dall'alto in
basso. «E so che sei consapevole di quanto sia pericoloso. Credi che
Eusebia o quel Severn esiterebbero per un momento a sottoporre uno
qualsiasi di noi al coltello? Qui camminiamo su una linea sottile, tutti
noi.»

Crispin si contorse sul sedile per guardare nostro padre. «Non abbiamo
fatto niente di male.»

Appoggiandomi allo schienale del sedile incrociai le braccia. «Sono
consapevole dei nostri obblighi verso il Decreto della Cappellania,
sire. Penso solo che se autorizzare l'acquisto di nuove apparecchiature
è quello che ci vuole per riportare in parità le operazioni minerarie
sul pianeta allora dobbiamo farlo indipendentemente dal costo. Forse
potrei parlare con la direttrice, prima che se ne vada. Lascia che
prenda con me Gibson. Lei ha bisogno quanto noi delle nostre operazioni
minerarie e forse possiamo arrivare a un accordo.»

«Un accordo? Tu?» Lord Alistair mi volse le spalle con la lunga giacca
che gli si gonfiava dietro in un vortice damascato di nero e rosso,
spostando la sua attenzione sull'antico dipinto a olio di una gondola
che si avvicinava a un'isola racchiusa dalle mura bianche di un
sepolcreto.

Con mia sorpresa, Crispin si schiarì la gola. «Perché no, padre? Lui è
bravo in questo.» Aprii la bocca per replicare, poi la richiusi e mi
trovai a fissarlo, pieno di confusione. Aveva appena parlato in mia
difesa?

Rimasi là seduto a guardare quel mio fratello dalla mascella squadrata,
che aveva di nuovo il tablet da gioco fra le grosse dita tozze.

«Perché tuo fratello ha peggiorato di molto questa imbarazzante
situazione con la sua ingerenza.» Mio padre si girò a mezzo busto, con i
piedi ancora saldamente piantati per cui il corpo era contorto mentre mi
fissava da sotto le sopracciglia aggrottate. I suoi colori si
attenuarono, rischiarati soltanto dalla debole luce solare che filtrava
attraverso l'oculo nella cupola sovrastante, affrescata con cupe
immagini di conquista. «Ti avevo assegnato un compito semplice: placare
la capofazione della Gilda, e invece tu l'hai messa in agitazione e hai
troncato i negoziati per tornare indietro in tempo per quella farsa
nella sala del trono.»

Serrai i braccioli della sedia con tanta forza da sentir gemere i miei
tendini. «Non avresti dovuto tagliarmi fuori.»

Mio padre si girò del tutto. «Non presumere di tenermi lezioni di
politica, ragazzo.» Per la prima volta quel giorno lord Alistair alzò la
voce, con quelle pesanti sopracciglia che si contraevano a formare una
piega sottile appena sopra il naso. Non fu proprio un urlo, ma fu
sufficiente. Perfino Crispin sussultò. «So in che modi utilizzarti, per
quanto pochi possano essere.»

Il mio orgoglio ferito ebbe la meglio sulla paura e mi alzai in piedi.
«Pochi? Credevo che mi si stessi addestrando nella diplomazia, padre.
Gibson dice...»

«Gibson è un vecchio stolto che dimentica quale sia il suo posto.» In
quel momento mio padre era un signore a tutti gli effetti, e accantonò
con un gesto i trecento anni di servizio dello scoliasta. «È tempo che
quel vecchio si ritiri. Gli dovremmo trovare un monastero in città, o
magari sulle montagne... gli piacerebbe.»

«Non puoi farlo!»

Mio padre sbatté le palpebre una volta, come un ghiacciaio che si
crepasse, e di colpo la sua voce si fece pericolosamente sommessa. «Mi
pare di averti detto di non darmi lezioni.» Mi girò di nuovo le spalle e
riprese a contemplare il dipinto con la sua isola di morti e la piccola
nave bianca. «Non faremo niente di precipitoso. Come te, anche
quell'uomo ha le sue utilità. A quanto comprendo, il tuo studio delle
lingue sta procedendo bene.»

Percepii una trappola, ma non riuscii a vederne la forma. «Sì» risposi.
«Gibson dice che il mio mandar è eccellente e che ho una buona
conoscenza perfino del cielcin.»

«E il tuo lothriano?»

La trappola era scattata. Come aveva fatto a saperlo? Non c'erano
videocamere nell'alloggio dello scoliasta. Non ce ne potevano essere. I
superstiziosi scoliasti non permettevano a niente di più complesso di un
microfilm di arrivare a un braccio di distanza da loro. Qualcuno aveva
origliato al buco della serratura, o... di colpo ricordai, e sorrisi.
C'era stato un servo che puliva i vetri dal lato del cortile, giusto? Mi
raddrizzai un po' di più sulla persona, imitando un soldato a riposo sul
terreno da parata e sperai di riuscire a nascondere la mia sorpresa. «È
molto buono, ma non tanto da mandarmi a Lothriad... nel Commonwealth.»
Esagerai il mio piccolo sorriso, sperando di mascherare la mia
comprensione con una battuta. «Lo conosco quanto basta per chiedere
dov'è un bagno, ma per altre cose potrei perdermi.»

Crispin rise, e mio padre gli scoccò un'occhiata rovente in tralice
prima di rivolgersi a me. «Pensi che questo sia un gioco?»

«No, sire.»

«Lo scoliasta ti ha detto della visita, vero?»

Negarlo era inutile. «Sì, padre.»

«Invecchia e dimentica quale sia il suo posto.»

«È saggio ed esperto.»

«Quindi lo difendi?»

«Sì.»

«Sai, non è un cattivo insegnante» interloquì Crispin, con una scrollata
di spalle.

«È un grande insegnante» dichiarai, protendendo in fuori la mascella.
«Ha fatto quello che ha fatto solo perché non ha senso tenere nascoste
queste cose a tuo figlio, sire. Se dovrò governare dopo di te, devo
essere coinvolto.»

«Se dovrai governare dopo di me?» Lord Altair sbatté le palpebre e
scosse il capo, sinceramente confuso. «Chi ha mai detto che avresti
governato dopo di me?» Più per un riflesso che per qualsiasi altro
motivo guardai verso mio fratello. No. No, non era possibile. Non aveva
senso. Però mio padre non aveva finito. «Non ho nominato un successore,
e non lo farò per molti anni a venire, per la Terra. Se però continui
così, ragazzo, posso dirti una cosa.» Fece una pausa, continuando a
darmi le spalle, incorniciato dal dipinto di quell'orribile isola. «Non
si tratterà di te.»