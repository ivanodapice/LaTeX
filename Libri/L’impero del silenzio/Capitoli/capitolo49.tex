\chapter{Fratelli d'armi}

«Cosa succede?» chiesi alla guardia, quando arrivai al barbacane del
palazzo. «Il tuo messaggero ha detto che qui c'è qualcuno che mi vuole
vedere.» Avevo ignorato tutte le chiamate arrivate sulla piastra
olografica della mia stanza ed ero uscito sulla balconata finché
qualcuno non aveva mandato un servitore a chiamarmi, quindi la tunica di
seta mi aderiva al corpo e avevo i capelli incollati alla fronte. Mi
ersi comunque in tutta la mia altezza e cercai di rendermi presentabile
e di mascherare la mia trepidazione.

La guardia si limitò a fissarmi, con le parole che chiaramente si
rifiutavano di oltrepassare quelle labbra screpolate.

«Direi proprio di sì!»

Pallino sbucò dal casotto di guardia con una risata che si spense in
parte quando avanzò nel corridoio, un posto echeggiante dai pavimenti di
piastrelle e dalle alte colonne. Un altro paio di guardie lo seguì,
ridacchiando per qualcosa che il vecchio mirmidone aveva detto. Pallino
si arrestò di colpo, parendo masticare qualcosa che non esisteva...
forse una parola... poi si piantò le mani sui fianchi e mi studiò con
quel suo unico occhio azzurro. «Ghen aveva ragione con quel suo
chiamarti `Vostra radiosità'. Ti sei ripulito, ragazzo.» Anche se era
una battuta, il tono di voce era un po' sforzato.

«Tu hai un aspetto orribile» ribattei, strappandogli un accenno di
sorriso. Cercai di ricambiarlo, ma non riuscii a indurmi a farlo perché
nel comportamento del vecchio c'era qualcosa di gelido. Con quanta
rapidità i suoi modi gioviali con le guardie si erano raffreddati. Mi
sentii assalire da un improvviso timore e domandai: «Qualcuno è...?»
\emph{Morto?} Non riuscii a proferire quella parola. Si trattava di
Switch? Di Siran? Non poteva essere Elara perché in quel caso lui non
avrebbe più avuto la voglia di sorridere.

Pallino sussultò, sorpreso. «Cosa? No. Pensavamo che \emph{tu} fossi
morto, razza di bastardo.» Accennò ai miei vestiti eleganti. «È chiaro,
però, che non sei un prigioniero, quindi perché questo segreto?» Il
vecchio soldato dava l'impressione di voler sputare ma si trattenne,
memore del pavimento a mosaico. «Pensi che adesso che hai i tuoi nuovi
amici eleganti ci puoi semplicemente piantare in asso, si tratta di
questo? Switch aveva ragione sul tuo conto.»

«Io...» Mi girai a guardare le tre guardie come se sperassi di trovare
una qualche risposta sul loro volto. Alle spalle del mirmidone, le porte
del castello erano spalancate ad ammettere la luce del sole ma non la
calura, grazie al campo statico che scintillava appena oltre l'alta
arcata, al di là della quale si allargava il terreno di parata, una
grande piazza che si stendeva per quasi tre miglia fino alla bassa
sagoma del colosseo, tutta cemento grigio e metallo lucente. «Cosa?»

Voleva fare \emph{qui} quel discorso? Davanti a un piccolo distaccamento
di peltasti del Casato? Sotto gli occhi di non si sa quanti diversi tipi
di videocamere? Eravamo sui gradini d'ingresso del castello, nel nome
dell'imperatore! Alle mie spalle gli ascensori inclinati salivano per
dozzine di piani fino alla cima dello ziggurat e al mezzanino del
castello, e il posto pullulava di logoteti e di servitori civili, tanto
dei Mataro quanto imperiali. C'erano perfino mercenari foederati nella
loro divisa marrone, una rozza compagnia raccolta in un angolo, in
attesa non so di cosa. E Pallino voleva farlo \emph{qui}?

«Hai preso e sei scomparso, ragazzo!»

Non potevamo tenere lì quella conversazione, o in nessun altro posto
nelle vicinanze del castello. Come potevo rispondere sinceramente senza
compromettere il mio segreto, quel segreto che il conte mi aveva
praticamente ordinato di proteggere? Mi venne un'idea e mi rivolsi al
più vicino dei peltasti. «Guardia, mi piacerebbe parlare in privato con
il mio amico. Possiamo fare una passeggiata in giro per la piazza?»

L'uomo guardò verso il suo superiore, una donna dallo sguardo duro e
dalla mascella squadrata, che scosse il capo. Avevano i loro ordini,
sapevano chi ero, o quantomeno che non mi era permesso di uscire.

«Questo gentiluomo non può lasciare il castello senza una scorta»
affermò il peltasta in tono piatto, evitando studiosamente di guardarmi.

Protesi i polsi verso Pallino come se fossero stati incatenati. «Non
sono \emph{esattamente} un prigioniero, ma come hai visto non sono
neppure libero. Non potevo mandarvi un messaggio.» Quest'ultima cosa non
era del tutto vera, è una di quelle piccole bugie che diciamo per
salvarci l'anima.

Pallino rimase fermo come una marionetta con i fili tagliati mentre
assimilava la nuova informazione. Alla fine riuscì a proferire una sola
parola. «Oh.»

Mi sfuggì una risata e invitai con un gesto l'anziano veterano ad
allontanarsi dalla porta e dalle guardie. In tutto il castello non c'era
un posto dove potevamo avere una certezza di privacy, e non dubitavo che
al servizio del conte ci fosse chi era molto interessato ai miei
movimenti, quindi guidai Pallino nell'ombra di una colonna rossa. «Come
stanno tutti?» domandai, impaziente di avviare di nuovo la
conversazione. «E come sta Elara?» Gli battei una pacca sulla spalla,
cercando di eliminare almeno in parte la tensione del momento. Da sopra
la spalla potevo vedere i tre peltasti che ci osservavano dalla porta
esterna.

«Quello cos'è?» Lo sguardo del vecchio si era spostato sugli affreschi
che decoravano la volta del soffitto e raffiguravano la conquista da
parte dei Mataro, con i Normanni in ritirata schiacciati sotto gli
stivali d'avorio delle Legioni. Sul suo volto apparve un'espressione
quasi malinconica quando si girò di nuovo verso di me. Stava forse
ricordando la sua vita di prima? «Elara sta bene. Uno dei nuovi cadetti
di Amarei le ha ammaccato la mascella nell'ultima mischia, ma è una
vecchia cagna tosta.» Sfoggiò un pigro sorriso. «Adoro questo murale.»
Protese una mano verso l'alto, agitandola per abbracciare l'estensione
dell'immagine.

«È un affresco» lo corressi, incapace di controllarmi. Mentre parlavo,
le luci in alto tremolarono e si spensero per un momento. Le guardai
contrariato, ma simili cali di tensione erano una cosa comune nel
castello. Danni dovuti alle tempeste, almeno a quanto avevano detto i
servitori.

Pallino non parve notarlo o curarsene, perché la luce solare era più che
sufficiente. «Lo so che è un fottuto affresco, ragazzo. Non sono mica
nato in un fienile.» Le luci tornarono ad accendersi.

Abbassai la testa, mormorando qualche parola di scusa, poi mi risollevai
e lo fissai in volto. «Immagino che dovremo annullare l'acquisto di
quella nave, eh?» Ecco, lo avevo detto, e sperai che fosse una cosa
tanto innocua da non dover rispondere in seguito a precise domande da
parte del conte o dei suoi inquisitori.

Pallino esalò un lungo respiro e parve quasi sgonfiarsi. «Non era un
brutto piano.» Privato della rabbia che aveva animato il suo piano
originale e della possibilità di urlarmi contro, non sapeva dove
dirigere la nostra conversazione, il che era un bene. Era
\emph{qualcosa}. Almeno non stava più urlando, non stava facendo una
scenata. «Comprare la nave con un po' della terra di tuo padre era una
cosa... astuta.»

Sconvolto, sussultai come se mi avessero punto. «Te lo ha detto Switch?»

«Però devo dire che non ha senso» continuò, infilando un dito sotto il
laccio della benda sull'occhio per grattarsi la faccia. «Perché un
ragazzo ricco come te dovrebbe combattere nell'arena con cani come noi?
Soprattutto quando saresti potuto venire qui ed essere trattato come una
sorta di persona di sangue reale?»

Come potevo rispondere a questo? Adesso la situazione era peggiore di
quando Switch mi aveva fatto la stessa domanda dopo la nostra visita ai
cantieri, perché allora non ero sotto sorveglianza. Cosa avevo detto a
Switch? Che `non ci sarebbero stati altro che guai finché non avessi
lasciato l'Impero'. \emph{Se} lo avessi lasciato.

«Come per te, era la mia alternativa migliore» risposi. «È stato a
proposito di questo che Switch e io abbiamo litigato.» Pallino incassò
quella risposta meglio di come avesse fatto Switch, limitandosi a
grugnire con le braccia conserte sul petto massiccio e l'attenzione
ancora fissa sugli affreschi. Non replicò, lasciandomi a domandarmi
quante cose Switch gli avesse detto. «Lui non mi ha creduto.»

Il mirmidone grugnì di nuovo, distogliendo lentamente lo sguardo dalle
immagini di conquista sulla volta sovrastante. Accanto a noi una porta
si aprì e un gruppo di logoteti e di rappresentanti corporativi uscì da
una delle sale conferenze, isolandoci per un momento in un mare di abiti
di un grigio scialbo e di colore viola. Alla fine Pallino scosse il
capo. «Voi ragazzi ricchi» disse, mentre quell'occhio azzurro si metteva
a fuoco su di me sotto le sopracciglia aggrottate. A quel punto fui
certo che sapesse, che Switch glielo avesse raccontato e fui grato --
immensamente grato -- che lui avesse il tatto di non dire niente.
Dovetti ricordarmi che un tempo era stato un soldato, congedato come
Centurione di Prima Linea. Aveva prestato servizio attivo su una delle
corazzate imperiali, non era stato uno di quelli tenuti dabbasso nelle
capsule criogeniche, tenuti di riserva per secoli di fila, quindi sapeva
cosa significava essere monitorato, essere sotto l'esame di altri uomini
in ogni istante di veglia della sua vita.

«Non si tratta di questo» replicai, senza sapere bene cosa intendesse
dire ciascuno di noi ma consapevole di essere diverso da tutti gli altri
giovani sotto quell'aspetto. «Io \emph{sono dovuto} andare via da casa,
Pallino.» Diedi una particolare enfasi a quel `sono dovuto',
riversandovi ogni oncia di determinismo di cui ero capace. Dovevo fargli
capire che dicevo sul serio, ma dovevo farlo senza dire niente che
compromettesse la tenue natura della benevolenza del conte. Come fanno
tanto spesso i cortigiani, stavo danzando a piedi scalzi sul filo di un
coltello, fra la verità e una necessaria illusione, fra ammissione e
sicurezza.

Lui dovette cogliere il senso delle mie parole perché lasciò perdere e
cambiò argomento. «Pensi che ti lasceranno andare presto? Riaverti con
noi sarebbe bello.» Gli lanciai un'occhiata per fargli capire che non mi
avrebbero mai più permesso di combattere. Una volta assimilato quel
concetto, lui cambiò approccio. «In ogni caso, sarebbe bello che venissi
a bere con noi, la prossima volta che vinciamo.» Un pensiero gli affiorò
fugace nello sguardo e aggiunse: «Hai saputo di Erdro?»

Deglutii a fatica, annuendo, e questa volta fui io a fissare gli
affreschi del soffitto. «L'ho visto.» Annaspai, non sapendo che altro
dire, poi aggiunsi: «Per un momento ho pensato che avesse Jaffa in
pugno.»

«È stata una mossa stupida» replicò Pallino. «Quelle riproduzioni di
armi antiche sono truccate il più possibile a nostro svantaggio. Lui
avrebbe dovuto saperlo e non avrebbe dovuto caricare quel bastardo.»

«Era un brav'uomo» affermai con semplicità. «Un buon combattente.»

«Sì.» Pallino si sfregò la manica al di sopra del punto in cui il
tatuaggio della Legione gli decorava il bicipite. «Forse è un bene che
tu ne sia fuori. Avrei detestato vederti fare una fine del genere, Had.
Chissà, forse questo è un buon posto per te.»

Stava cercando di essere cortese, ma questo mi faceva male. «Preferirei
lasciare il pianeta.»

«Perché?»

Mi ero aspettato quella domanda, ma non era comunque facile rispondere.
Come spiegare a un plebeo che ricchezza e potere erano accompagnati da
una gabbia? Avrebbe visto solo la seta, non quanto costasse indossarla.
«Questa non è casa, Pallino.» Scrollai le spalle e concentrai la mia
attenzione su di lui. «Non sono io.» Spezzai il contatto visivo,
incapace di reggere lo sguardo di quel singolo occhio. «Potremmo ancora
comprare la nave.»

L'occhio azzurro si socchiuse. «Come, se sei bloccato qui?»

«Potrei usare come anticipo la terra di... di mio padre. Tu, Elara e
Switch potreste lavorare per me, così non dovreste più rischiare la
vita. Sarebbe un lavoro più stabile, più sicuro.» Mi interruppi,
massaggiandomi la mascella. «Immagino che dovremo assumere un pilota,
ma...»

Questo prese all'amo qualcosa nella mente del mirmidone, che parve di
nuovo masticare qualcosa di invisibile mentre contemplava il soffitto
con aria pensosa. Nessuno di noi aveva detto apertamente ciò che era
rimasto taciuto; il mio sottintendere che sarei stato ancora vivo per
incassare la quota del proprietario di qualsiasi entrata. Se stava
prestando attenzione -- e lo stava facendo -- doveva ormai aver capito
che ero quantomeno di estrazione patrizia. «Di certo non sarebbe male.»

«Ci penserai su?» Mi illuminai in volto. Se non potevo sottrarmi alle
circostanze in cui mi trovavo, quantomeno potevo aiutare i miei amici, e
anche se al momento le possibilità erano quasi inesistenti, forse
nutrivo la speranza di poter fuggire, di lasciare Emesh per imboccare un
sentiero scelto da me, per me stesso.

La bocca di Pallino si incurvò in un teso sorriso. «Sì, potrei farlo.»
Si morse un labbro, mentre il sorriso collassava come un'onda. «E
parlerò con Switch, vedrò se mi riesce di farlo ragionare. Non è giusto
che un uomo non parli con i suoi fratelli.» Era questo che eravamo?
Naturalmente, quella parola mi indusse a pensare a Crispin, il solo vero
fratello che avessi mai avuto, anche se la sua perdita non mi aveva
ferito profondamente quanto quella dei miei compagni mirmidoni. Il suono
di quella parola, `fratello', accese nella mia anima una profonda
solitudine di cui non avevo più provato l'uguale da quando Cat era
morta, non perché \emph{fossi} solo ma perché non lo ero e forse
meritavo di esserlo.

Chinai il mento e dovetti chiudere gli occhi per fermare le lacrime. «Lo
apprezzerei» risposi con voce esitante. «Andrei io stesso, ma...»
Accennai in modo vago in direzione del casotto di guardia, poi cercai di
rasserenare l'atmosfera e chiesi: «Cosa hai detto per far ridere quelle
guardie?»

«Storie di guerra, figliolo» rispose, battendomi una pacca sulla spalla.
«Storie di guerra.»

«Dice niente riguardo a me?» Mi andai ad affiancare a Pallino e
incrociai le braccia. Per un momento rimanemmo entrambi in silenzio, con
lo sguardo rivolto all'immagine che tanto affascinava il mio compagno.
Proprio sopra di noi era raffigurato un Umandh che lottava contro due
legionari imperiali armati di lance lucenti. Uno dei due teneva uno
stivale sul tronco del colono e questo mi richiamò alla memoria il modo
in cui quel gladiatore di Meidua si era erto sul corpo del suo nemico
mutilato.

Pallino si trasse indietro e si girò leggermente per guardarmi. «Chi,
Switch? Lui e Ghen ogni tanto sparano qualche cazzata sul tuo conto, ma
credo che senta la tua mancanza.»

«Digli che mi dispiace,» affermai, serrando le mani dietro la schiena «e
che glielo direi io stesso, se potessi.» Accennai con il mento,
assumendo una posa leggermente aristocratica per ricacciare indietro le
mie emozioni. «So che devo a entrambi una spiegazione per...» accennai
ai bei vestiti e all'ambiente opulento «...ma in questo momento non ne
posso parlare. Puoi fidarti di me?»

Il vecchio veterano stava sorridendo quando mi girai a guardarlo. «Sai,
nella Legione impari a fidarti degli uomini della tua decuria, perfino
dei bastardi. A volte soprattutto dei bastardi. Non importa cosa siano,
stanno tirando nella tua stessa direzione, capisci?» Annuii e Pallino mi
puntò un dito in faccia. «Tu non sei il peggior bastardo che abbia
conosciuto, Had, neppure alla {lontana}.» Indicò sé stesso, poi me, poi
ancora sé stesso e sorrise. «Tiriamo nella stessa direzione.»

Per un po' parlammo del più e del meno, dei nostri amici, del colosseo e
di quanto apparissi stupido nei miei abiti eleganti, ma ben presto
Pallino dovette congedarsi, aveva in programma un addestramento con le
nuove reclute sacrificabili. Mi girai per avviarmi per primo e risalire
con l'ascensore fino al palazzo vero e proprio, ma lui mi afferrò per un
polso. «Questo non è un addio, sai?»

«Cosa?» Sbattei le palpebre, sinceramente confuso. «No, certo che no.»

Il vecchio veterano sorrise come un lupo. «Bene, perché conto su quella
faccenda della nave. Adesso che stai con i nobili, per te non dovrebbe
essere troppo difficile farcela.» Avrei voluto correggerlo, ma non ce
n'era il tempo. Mi assestò una pacca sulla schiena e aggiunse ad alta
voce, come se stesse scherzando. «E chi lo sa? Potresti aver bisogno di
noi, se mai volessi abbandonare la città!» E si allontanò nella
tremolante luce diurna, agitando con noncuranza una mano sopra la
spalla.


