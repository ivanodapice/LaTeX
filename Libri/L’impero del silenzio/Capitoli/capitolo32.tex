\chapter{State alla larga}

Gli storditori dei prefetti raggiunsero due seguaci di Rells alla
schiena, facendoli crollare in una profonda pozzanghera in fondo alla
strada tortuosa mentre io osservavo la scena accovacciato nell'ombra di
un gruppo di dischi satellitari e di antenne, sul tetto del negozio
d'angolo che avevamo svaligiato insieme. Stringevo ancora la borsa che
conteneva due hurasam, forse cinquanta kaspum e una manciata di monete
d'acciaio... una piccola fortuna per la creatura che ero diventato. Non
bastava per pagarmi un passaggio e lasciare il pianeta -- non ancora --
ma era tutta mia. Avevo attivato il sistema di allarme del negozio
mentre quei bastardi picchiavano la commessa. Forse era stato un atto
ipocrita, dato che avevo accoltellato la direttrice alla spalla con il
coltello che avevo ancora in mano, ripulito in modo imperfetto dal
sangue che ne copriva la superficie sfregiata. La donna sarebbe
sopravvissuta, o almeno lo speravo, dato che avevo mancato il cuore e
colpito un osso, però doveva averle fatto male.

Sette prefetti che sudavano nell'uniforme cachi e giacca a vento azzurra
si aprirono a ventaglio per mettere con le spalle al muro i due ragazzi
e la ragazza ancora in piedi. «Arrendetevi!» ingiunse il loro capo, un
uomo alto con i capelli scuri quasi quanto i miei e gli occhi nascosti
dietro lenti lucenti. Teneva lo storditore puntato dritto verso Tur, il
più grosso dei tre ancora in piedi, e potevo vedere l'apertura dell'arma
che riluceva di un azzurro gelido, con una striscia verticale di luce
alla sua estremità. «In ginocchio, tutti quanti!»

«Kaller sta annegando, bastardi!» gridò la ragazza, riparandosi
spaventata dietro le spalle massicce di Tur, e indicò uno dei due ladri
storditi che era caduto con la faccia immersa nella pozzanghera. Il
prefetto-ispettore con gli occhiali non si mosse ma la sua partner --
una donna minuta con i capelli scuri tagliati a caschetto -- si spostò
per trascinare Kaller fuori dal fango. Io rimasi immobile, tanto che
avrei potuto essere scolpito nella pietra come uno dei gargoyle che
decoravano le mura e i contrafforti del Riposo del Diavolo.

La donna prefetto controllò le pulsazioni di Keller e il suo respiro. «È
vivo, Gin.»

L'uomo con gli occhiali non annuì neppure e non diede segno che gli
importasse. Alle sue spalle l'unica altra donna del gruppo si spostò per
aiutare la sua controparte a trascinare fuori dal fango l'altro ladro
stordito. La folla attendeva dietro i cordoni olografici proiettati da
droni subintelligenti del tipo ammesso dalle leggi religiose della
Cappellania, e cioè quello che aveva alle spalle un operatore umano
nell'ufficio dei Prefetti all'interno del complesso del palazzo e nel
cuore di Borosevo. Quei droni sembravano poco più di bidoni
aspirapolvere tempestatati di sensori e di un equipaggiamento di
proiezione in equilibrio su singole sfere gommate. Adesso non si
muovevano e rimanevano di guardia intorno alla scena del crimine.

Per un momento persi il filo della conversazione in corso sotto di me e
riuscii a sentire soltanto la ripetitiva voce femminile registrata che
scaturiva da tutti quei droni: «Questa è la divisione di Risposta al
Crimine dell'ufficio dei Prefetti di Borosevo. Per favore, state alla
larga. Ripeto, questa è la divisione di Risposta al Crimine del...» Le
parole scivolarono al di là della soglia uditiva, diventando parte
dell'atmosfera della scena come il rumore dell'acqua e dei velivoli.

«Dovremmo stordirli, Gin» suggerì un altro dei prefetti, un uomo
dinoccolato con spesse basette, magro come un palo e alto quanto un
palatino. «Impacchettarli e portarli dentro per il condizionamento.»

«Un accidente!» ringhiò Tur, allargando le braccia per riparare la
ragazza. «Non voglio le vostre fottute manipolazioni nella mia testa.»
Brandì quel pezzo di tubo incurvato a un'estremità che aveva sempre con
sé. «State lontani, dannati figli di \emph{puttana}!»

L'uomo con le basette gli sparò in pieno petto, stordendolo, e lui cadde
all'indietro, quasi schiacciando sotto di sé quella povera ragazza che
strillò e si ritrasse contro il vetro dipinto della vetrina mentre
l'altro ladro -- non ne ricordo il nome -- le si precipitava accanto.
«State a terra!» ingiunse l'uomo con gli occhiali, puntando lo
storditore contro gli altri. «Non voglio dovervi sparare.» Poi si
rivolse al compagno: «Ko, non fare fuoco.»

«Quel tizio era rabbioso, Gin» replicò l'altro uomo.

«Ti ho detto di non fare fuoco» scattò il prefetto-ispettore,
lanciandogli un'occhiata. «Dov'è la refurtiva?» domandò quindi ai ladri,
guardando le loro mani vuote mentre io serravo le dita intorno alla
borsa rubata.

La ragazza protese il mento in fuori. «Andata, sbirro.» Sorrise. «Voi
cani siete troppo lenti.»

«Questa è la divisione di Risposta al Crimine...»

Il prefetto-ispettore attivò un interruttore sullo storditore e la linea
azzurra nella canna si fece più luminosa. «In ginocchio. Arrendetevi.»

«Perché ci possiate portare dai segaossa e farci mettere in riga il
cervello?» ribatté l'uomo. «In questo sono d'accordo con Tur. No,
grazie.»

Il prefetto-ispettore avanzò di un passo. «Arrendetevi e non dovrete
farlo. Potrete andare al Colosso. Hanno bisogno di altri cadaveri
ambulanti.» La pelle bronzea del ladro si fece bianca e lui non replicò.
Accanto a lui, la ragazza era ancora più pallida. Io continuai a non
muovermi, nascosto com'ero fra le antenne, sperando che i miei ex
complici non mi vedessero. Una paura inutile perché nessuno guardò mai
in alto. «...al Crimine dell'ufficio dei Prefetti di Borosevo...»

La ragazza stava scuotendo il capo. «No. No, preferisco i segaossa.»
Pensai agli schiavi nel nostro Colosso di Meidua, gli uomini e donne
mutilati e vestiti come Cielcin che morivano per mano di gladiatori
professionisti. Non potevo biasimarla se aveva paura. Perfino i popolani
che entravano volontariamente nell'arena di solito non duravano a lungo
contro quei professionisti. Il Colosso era una condanna a morte, per di
più umiliante, perché si andava incontro alla morte mutilati dai cathar,
con il naso tagliato e la fronte marchiata.

Non biasimai Tur e gli altri per aver cambiato idea.

I prefetti non corsero rischi. A un segnale del prefetto-ispettore
l'uomo di nome Ko aprì il fuoco, abbattendo gli altri due ladri.
Ripensai alla malmenata commessa del negozio e alla direttrice che avevo
accoltellato e annuii in segno di approvazione. La banda di Rells era un
gruppo di delinquenti feroci, peggiori di quanto io sia mai stato, e
quello che stava succedendo loro aveva il sapore della giustizia. Quando
finalmente i criminali e i prefetti ebbero sgombrato la strada, dopo che
il cordone olografico e i droni rotolarono via, ciò che mi rimase
impresso nella mente non fu quel messaggio pubblico ripetuto molte
centinaia di volte nel corso dell'incidente. Invece, fu quello che il
prefetto-ispettore aveva detto a Tur. `Potete andare al Colosso.'

`Non hanno carenza di posti liberi nel Colosso.' Quelle parole mi
riaffiorarono dentro con una strana, lucida insistenza. Quel vecchio
marinaio, il Corvo, mi aveva suggerito di combattere nei giochi. Sarebbe
stato un modo per guadagnarmi di che vivere e magari per guadagnare
abbastanza da pagarmi un passaggio lontano dal pianeta. Sarebbe stato
pericoloso, ma che altra scelta avevo? D'un tratto quell'incontro
casuale assunse proporzioni profetiche e mi appoggiai contro il gruppo
di antenne sul tetto.

Perché non lo avevo fatto prima?

