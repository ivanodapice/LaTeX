\chapter{Un monopolio della sofferenza}

«Avresti potuto dire qualcosa!» sibilò Switch, quando lasciammo il
cantiere. Il sole era vicino allo zenit e la sua luce batteva su di me
come una pioggia di pugni. Tirai fuori gli occhiali da sole rubati e me
li spinsi sul naso, incurvando le spalle mentre procedevo in fretta
lungo il canale, impaziente di essere a casa. Avevo molte cose a cui
pensare, ma Switch non mi concesse il tempo di riflettere. Mi afferrò
per una spalla e mi fece girare. «Perché non me lo hai detto?»

«Dirti cosa, Switch?» ribattei. Non stavo facendo il finto tonto, sapevo
cosa intendeva, solo che non sapevo cosa rispondere. «Dirti cosa?»

Le sue guance si arrossarono fino a farsi quasi dello stesso colore dei
suoi capelli, e la mascella appuntita si contrasse come se stesse
cercando si schiacciare un ciottolo fra i molari. «Capisco voler
manipolare il venditore, ma avresti potuto dirmelo!» Torse nel pugno la
stoffa della mia camicia, poi ripeté con voce più sommessa: «Avresti
potuto dirmi che eri uno di \emph{loro}.»

Pronunciò quell'ultima parola con voce che era appena più di un
sussurro. Quasi di riflesso mi ersi in tutta la mia statura, inclinando
il mento in modo da guardarlo dall'alto in basso. Fino a quel momento
non ci avevo badato, ma ero molto più alto di lui. Possibile che mi
fossi abituato a essere circondato da plebei tanto da assuefarmi alla
loro bassa statura? Non sono alto -- non secondo gli standard della
corte -- ma in quel momento mi sentii un colosso e mi aggrappai alla mia
statura di palatino come a quell'emblema che era.

Switch però non si lasciò intimidire perché non era più il ragazzo
spaventato che era stato anni prima, e mi pungolò nelle costole con gli
occhi dilatati. «Si suppone che tu sia mio amico, Had.» Potevo ancora
sentire il rimprovero nella sua voce, vedere come snudava i denti meno
che bianchi. «Dovresti dirmi questo genere di cose.»

Qualcosa si torse dentro di me, una frazione dell'antica ira nobiliare
portata a galla dalla trattativa con la sovrintendente del cantiere.
«Dirti cosa? Di questo?» Tirai fuori l'anello e lo sollevai perché lo
esaminasse, con l'argento tramutato in bronzo dalla luce rossa del sole.
«Cosa vorresti che dicessi?»

Il ragazzo contrasse di nuovo la mascella, lottando per trovare parole
che non volevano affiorare, poi distolse lo sguardo e lo appuntò sul
muro incombente del Distretto Bianco, che si ergeva per cinquanta piedi
sopra la nostra testa. Una serie di cabine a fune ondeggiavano sopra la
nostra testa, trasportando persone dalla parte più ricca della città al
Distretto Inferiore. Sentii che dovevo dire qualcosa, qualsiasi cosa.
Tutto. Su mio padre, su Crispin, su Kyra e Gibson e quello che gli
avevano fatto. Su mia madre e quello che temevo le avessero fatto. Sulla
Cappellania e quello che temevo potessero farmi.

Alla fine non gli dissi niente di tutto questo.

«Non... non sembrava importante» risposi invece. Quelle parole suonarono
sprezzanti alle mie stesse orecchie, ombre proiettate da quelle più
gravi preoccupazioni. Erano piccole cose, e le resi ancora più piccole
affermando: «Non ha importanza.»

«Non ti sembrava importante?» Switch non mi aveva ancora lasciato
andare. Mi scrollò. «Non ti sembrava importante?» La sua voce salì di
tono, attirando gli sguardi di un corriere di passaggio e di una giovane
coppia che portava sarong uguali. «Perché sei nel Colosso? Non hai
bisogno di quel fottuto denaro.»

Serrai la mascella e gli posai una mano sul braccio con fare
conciliante. «Non è così semplice, Switch...»

«Certo che lo è!» sibilò, tirando al punto da farmi piegare all'altezza
della vita. «Sei uno di \emph{loro}. Non mi dire che non
\emph{importa}.» Vidi qualcosa muoversi appena sotto il suo volto,
un'ombra che gli colorava i lineamenti. Il ricordo di quello che era
stato affiorò nella mia sfera cosciente, inducendomi a ricordare le Uri
nell'harem della viceregina, e il modo in cui Crispin e mia madre le
usavano. {Rabbrividii}. Cosa aveva sperimentato Switch per mano dei
palatini? Pensando a come Kyra si era raggelata fra le mie braccia, mi
raggelai a mia volta. «Per te questo è un gioco?» mi chiese.
«Frequentare i bassifondi con il resto di noi?»

«No!» scattai. «Dannazione, no! Non essere assurdo!» `Assurdo.' Non era
una parola plebea e il volto di Switch si contrasse nel sentirla, o
forse fu il mio accento scandito di Delos, riconosciuto ora come un
simbolo di quello che ero.

«Ghen aveva ragione sul tuo conto, Vostra radiosità» sogghignò,
spingendomi indietro.

«Non è così!» Riuscii a stento a trattenermi dall'urlargli contro.
Adesso un paio di persone ci stavano fissando apertamente, quindi
sibilai: «Non avrei mai usato questo se avessi avuto un'altra scelta.
Nel minuto in cui lo farò avrò solo guai finché non avrò lasciato
l'Impero. Lo capisci?»

Switch stava praticamente ringhiando. «Cos'hai fatto? Hai picchiato una
delle concubine di tuo padre quando non sei riuscito a fartelo rizzare?»
Questo toccò un nervo scoperto, facendo involontario riferimento alla
morte di mio nonno per mano di una delle sue concubine.

«Non ne ho mai toccata una e non lo avrei mai fatto. Non so che cosa hai
subìto, Switch, ma non è stato per mano mia. Pensi che siamo tutti
mostri, vero? Io sono lo stesso uomo che ero due giorni fa, lo stesso
che ti ha salvato il culo cento volte nel colosseo. Lo stesso uomo.»

«Non lo sei» ribatté Switch. «Sei uno di loro e hai mentito al
riguardo.»

«Non potevo dire la verità» ringhiai. «Non posso farlo. È troppo
pericoloso.»

«Non sembrava pericoloso mentre agitavi quell'anello.»

Non avevo la pazienza di discutere delle sottigliezze delle transazioni
finanziarie fra Casati planetari. «Avrebbero \emph{dovuto} darmi quella
nave a causa di quello che sono. Avrebbero dovuto \emph{fidarsi} di
darmela.»

«A causa di quello che sei» sogghignò.

«Stavo cercando di rubargliela, di fregare mio padre e questi Mataro»
ringhiai, accennando alla strada sporca, alle case e ai negozi dal tetto
di latta che ci circondavano. «Credi che voglia stare qui? Credi che
volessi questo? Credi davvero che sarei qui se avessi avuto un'altra
scelta?» Era la cosa peggiore che avrei potuto dire.

«Cosa c'è che non va in noi?» ribatté Switch, mantenendo a stento la
voce a un basso ringhio. «È così che è la vita, Vostra radiosità, la
vita vera. Tu non lo sai!»

«Non lo so? Davvero? Io?» controbattei, ma mi strozzai sulle mie
spiegazioni. Potevo sentire il sangue che mi martellava nella testa e
avevo labbra ritratte in quella che era più una smorfia che un sorriso.
«Non sei la sola persona che abbia mai sofferto! Per tre anni ho corso
per questa città, dormendo nei canali di scolo. Sono stato picchiato,
accoltellato, quasi violentato. Sono sopravvissuto alla dannata necrosi.
Ho seppellito la mia...» La mia cosa? Amante? Amica? «Ho perso delle
persone in questa città. Solo perché qualche mercante profumato ti ha
fottuto lungo le rotte spaziali questo non ti dà il monopolio della
sofferenza!» Dal modo in cui Switch sbiancò in volto compresi che
\emph{quella} era davvero la cosa peggiore che avrei potuto dire. Sentii
tutte le spiegazioni, le giustificazioni e l'orgoglio che mi
abbandonavano. Potevo gestire il mio dolore, non avrei dovuto usarlo per
sostenere una tesi, neppure una giusta. Mi ripiegai come uno scrigno che
si chiude, con le spalle che si accasciavano. Quanto vorrei poter dire
che fu la voce di mio padre a parlare dalla mia bocca, che fu quella di
Crispin, di mia madre, dello zio Lucian... ma era la mia.

Non vidi arrivare il colpo finché non mi raggiunse in pieno su un lato
del mento. Mi fece scattare la testa all'indietro e per poco non persi
l'equilibrio, barcollando contro il muro di una panetteria. Qualcuno
sussultò e quando la vista mi si schiarì vidi due giovani con la livrea
argento di qualche nave che armeggiavano per tirare fuori il terminale e
registrare la scena. Sputai. C'era del rosso? O era solo l'effetto della
luce solare? Per un momento fui particolarmente conscio di come i
vestiti mi aderissero addosso nell'aria umida e fumante. All'inizio non
reagii e mi limitai ad assestarmi la camicia marrone. Switch mi fissava
con occhi roventi e il rossore che gli saliva lentamente al volto.
Teneva i pugni lungo i fianchi, ma erano ancora serrati. Gli scoliasti
ci dicono che il flusso del tempo è assoluto, ma stando là, con mezza
strada che mi guardava, sentii i secondi gocciolare come eoni.

«Mi dispiace» dissi infine. Debolmente. Per quanto avessi sofferto -- e
lo avevo fatto -- questo non mi dava il monopolio della {sofferenza}.
Pensai a quella notte in cui la banda di Rells mi aveva trascinato fuori
dalla mia tana, alle molte volte in cui ero stato stordito e percosso
dai prefetti cittadini. Non eravamo così diversi, Switch e io, quali che
fossero le nostre origini, ed eccomi lì ad accusarlo esattamente delle
mie colpe. «Mi dispiace. È solo che non riesco a parlarne.»

«Perché no?» Mi fissò con occhi roventi.

Mi massaggiai la bocca con il dorso della mano. «È stato un buon colpo.»
Mi guardai il braccio. La saliva \emph{era} rossa. «Me lo sono
meritato.»

«Perché non me ne vuoi parlare?» Switch si avvicinò di un passo,
bloccando i marinai che stavano registrando sui loro terminali,
girandoci intorno come i corvi mascherati di una pantomima eudoriana.
«Hai ucciso qualcuno?» chiese, in quello che era praticamente un
sussurro.

Scossi il capo, fissando con ira gli spettatori, quegli stupidi idioti
che nella loro vita non avevano niente da fare se non impicciarsi della
vita degli altri. Trassi un respiro e scossi di nuovo la testa, il primo
gesto un diniego e il secondo un rifiuto di dire altro. Switch sputò...
non proprio ai miei piedi ma abbastanza vicino da non fare differenza.
La mascella mi doleva e mi chiesi con un certo distacco se avessi perso
un dente, perché uno mi sembrava allentato. Sarebbe ricresciuto, perché
ero un palatino. Ricrescono sempre.

Mi schiarii la gola. «Switch, io... non ci riesco. Mi dispiace, io...»

Sollevò una mano. «Risparmiatelo.»

Poi si volse e si allontanò. Lo guardai andarsene mentre mi abbassavo
lentamente fino a sedere con la schiena contro il muro del negozio,
proprio come il mendicante che ero stato.

