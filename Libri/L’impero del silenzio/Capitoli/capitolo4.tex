\chapter{Il diavolo e la dama}

«Non è stata la tua performance più elegante, Hadrian.» La voce di mia
madre attraversò senza difficoltà la porta di pannelli scuri che
separava il mio armadio dalla camera da letto. Con un timbro da
contralto, era ricca degli accenti della nobiltà deliana ed era affinata
da decenni di discorsi, cene formali ed esibizioni. Di professione era
un'autrice di libretti operistici e una regista.

«Crispin non si è voluto spostare.» Quella fu la sola risposta che
riuscii a trovare mentre armeggiavo con i bottoni d'argento della mia
camicia migliore.

«Crispin ha quindici anni e un pessimo carattere.»

«Lo so, madre.» Mi passai le bretelle sulle spalle e le strinsi. «Non
capisco perché nostro padre non... non mi abbia incluso.»

Dal suono opaco della sua voce compresi che mia madre si era allontanata
dalla porta e diretta verso la finestra che si affacciava sul mare. Lo
faceva spesso. Lady Liliana Kephalos-Marlowe aveva la tendenza a
spostarsi verso le finestre, e quella era una cosa che avevamo in
comune... il desiderio di essere altrove, da qualsiasi altra parte.
«Devi proprio chiederlo?»

Non dovevo. Invece di rispondere indossai il panciotto di seta e
velluto, e abbassai il colletto. Vestito a sufficienza, aprii la porta e
uscii nella camera da letto, dove vidi che in effetti mia madre si era
avvicinata alla finestra. La mia stanza era in alto nella Grande Rocca,
inserita nell'angolo nordest della torre squadrata, da dove dominavo la
vista della diga marina e dell'oceano al di là di essa, potendo spaziare
con la vista per miglia fino a dove le Isole del Vento si intravedevano
vaghe all'orizzonte, anche se erano invisibili al livello del mare. Mia
madre si girò a fronteggiarmi. Non vestiva mai di nero, non aveva mai
adottato i colori e lo stemma araldico del Casato di mio padre. Era nata
nel Casato Kephalos e sua madre, la viceregina, era anche la duchessa
dell'intero pianeta, cosa di cui lei era orgogliosa. Per questa
occasione -- il banchetto di accoglienza per la direttrice Adaeze Feng e
il suo gruppo -- indossava un abito elaborato di seta bianca, tanto
aderente da dover essere sintetico, fermato su una spalla da una spilla
dorata nella forma dell'aquila dei Kephalos. I suoi capelli fra il
bronzo e il miele erano raccolti all'indietro per poi ricadere in
riccioli serrati dietro le orecchie. Era splendida, nel modo in cui lo
sono le donne palatine, un'immagine della dimenticata Saffo scolpita nel
marmo vivente, e altrettanto fredda.

«I tuoi capelli sono orribili.»

«Grazie, madre» risposi mentre spingevo la frangia ricciuta dietro le
orecchie.

La bocca dipinta di rosso di lady Liliana si aprì in cerca delle parole
giuste «Non era un complimento»

«No» convenni, indossando la redingote con il diavolo del mio Casato
ricamato sul petto, a sinistra.

«Dovresti proprio tagliarli.» Si allontanò dalla finestra e allungò le
dita bianche per assestarmi i risvolti della giacca.

«Già così mio padre mi confonde abbondantemente con Crispin.» Lasciai
che mi aggiustasse il colletto senza fare commenti a parte il mio
sguardo più tagliente. I suoi occhi erano color ambra, decisamente più
caldi di quelli di mio padre, ma anche così non riuscivo ad avvertire
quel calore. Sapevo che se avesse potuto fare a modo suo sarebbe tornata
ad Artemia presso la sua famiglia e le sue ragazze invece di stare con
noi Marlowe in questo posto ombroso e cinereo. Noi Marlowe con i nostri
occhi freddi e i nostri modi ancora più freddi, con suo marito che era
il più gelido di tutti.

«Lui non fa cose del genere.» Dalla soffocata frettolosità del suo tono
compresi che le era sfuggito il punto del mio commento.

«Allora è sua intenzione che Crispin prenda il mio posto?» Continuai a
trapassarla con lo sguardo mentre mi lisciava le spalle della giacca.

«Non è quello che vuoi?»

La fissai sbattendo le palpebre. Non sapevo cosa rispondere a questo,
non senza infrangere il delicato equilibrio del mio mondo. Cosa potevo
dire? No? Io però non volevo il lavoro di mio padre più di quanto
volessi essere il primo strategos delle Legioni Orionidi. Sì? Ma allora
Crispin avrebbe governato e sarebbe stato una catastrofe. Non volevo
ottenere il trono di mio padre... volevo che Crispin lo perdesse.

Mia madre si allontanò, tornando alla finestra con i tacchi che
ticchettavano sulle piastrelle del pavimento. «Non posso sostenere di
conoscere i piani di tuo padre...»

«Come potresti?» Mi ersi sulla persona in tutta la mia poco
impressionante statura. «Non ci sei mai.»

Lei non si scaldò, non si girò neppure a guardarmi. «Credi che qui
rimarrebbe qualcuno, se potessero scegliere?»

«Sir Felix rimane» ribattei, assestandomi meglio la giacca sulle spalle
strette. «E anche Roban e gli altri.»

«Vedono la possibilità di fare carriera. Terra e un titolo a loro nome.
Una piccola fortezza.»

«Non sono fedeli a mio padre?»

«Nessuno di loro \emph{conosce} tuo padre, tranne forse Felix. Io l'ho
sposato e non posso dire di conoscerlo.»

Lo sapevo, ma sentirlo -- sentire che i miei genitori erano due
sconosciuti -- mi devastava ogni volta. Reagii con un cenno minimo della
testa, poi mi resi conto che mia madre non poteva vedermi con la schiena
girata. «Non inspira familiarità» dissi infine, accigliandomi nonostante
tutto.

«E non dovresti farlo neppure tu, se governerai al suo posto.» Lady
Liliana si volse parzialmente per scrutarmi attraverso i riccioli
bronzei, con gli occhi ambra che apparivano duri e stanchi. Credo fu
allora che notai per la prima volta la sua età; non quella della soglia
dell'età adulta che sfoggiava esteriormente ma i quasi due secoli che
aveva vissuto in realtà. Quell'effetto svanì in un istante mentre
proseguiva: «Dovrai guidare il tuo popolo, non metterti al suo fianco.»

«Se governerò» ripetei.

«Non è una conclusione scontata» affermò. «Potrebbe scegliere Crispin,
oppure ordinare che producano un terzo bambino dalle vasche.» Prevedendo
la mia reazione aggiunse: «Solo perché il Casato Marlowe ha sempre
onorato il figlio maggiore questo non significa che lo si debba fare. La
legge permette a tuo padre di scegliere il suo erede, quindi non
presumere niente.»

«Ottimo» ribattei, un po' punto nel vivo. «Non importa, è...»

Lei mi interruppe. «Hai proprio ragione, non importa. Ora vieni, siamo
quasi in ritardo.»

\begin{figure}
	\centering
	\def\svgwidth{\columnwidth}
	\scalebox{0.2}{\input{divisore.pdf_tex}}
\end{figure}

Pensai che intere stelle erano nate e morte prima del dessert. Rimasi in
silenzio durante tutti i brindisi, mentre servivano l'insalata e nei
cicli e cicli di servitori che posavano e prelevavano i piatti. E
ascoltai, fin troppo consapevole della rabbia che, come la forza di
gravità, piegava lo spazio e il tempo intorno a mio padre. Dentro di me
ero grato che la direttrice e il suo seguito mi avessero rimosso dal mio
posto abituale alla destra di mio padre. Era trascorso un giorno dal mio
arrivo in ritardo nella sala del trono e mio padre non mi aveva ancora
rivolto la parola, e il non essere stato rimproverato mi riempiva di
disagio.

Quindi mangiai e ascoltai, studiando i volti strani e quasi alieni dei
dignitari del Consorzio. Quei plutocrati passavano la vita nello spazio
e l'imitazione centripeta della gravità a bordo delle enormi navi a
rotazione non impediva loro di cambiare. Se non fosse stato per le
terapie genetiche rigorose quasi quanto la mia, non sarebbero stati in
grado di reggersi in piedi su Delos, con la sua gravità standard di uno
e un decimo, e sarebbero stati schiacciati e annaspanti come pesci
spiaggiati sulla sabbia.

«I Cielcin si sono spinti troppo oltre, esercitando pressione al di là
del Velo» disse Xun Gong Sun, uno dei ministri giovani del Consorzio.
«L'imperatore non dovrebbe tollerarlo.»

«L'imperatore non lo tollera, Xun» replicò in tono mite la direttrice
Feng. «È per questo che è in corso una guerra.» Studiai la direttrice.
Come tutti i membri del Consorzio era del tutto senza capelli, con gli
zigomi e la forma delle sopracciglia accentuati per rendere più marcata
l'inclinazione degli occhi. La sua pelle era più scura di quella degli
altri, quasi del colore del caffè. Si girò per parlare a mia madre e a
mio padre, che sedevano a capotavola. «Il principe di Jadd ha impegnato
dodicimila navi nello sforzo bellico, mettendole al comando di quel suo
nipote, Darkmoon. Sono partiti perfino i membri dei clan dei
Tavrosiani.»

Mio padre posò sul tavolo il suo bicchiere di vino di Kandarene e si
soffermò per un secondo prima di rispondere. «Sappiamo tutto questo.
Madama direttrice.»

«Sì, certo.» Lei sorrise, sollevando la sua coppa di vino. Le sue dita
erano come insetti stecco che si agitano. «Volevo solo dire che tutte
quelle navi avranno bisogno di combustibile, mio signore.»

Il signore del Riposo del Diavolo fissò intensamente il direttore con i
denti che scivolavano lungo il labbro inferiore, e ripiegò le mani sul
tavolo. «Non hai bisogno di convincerci. Ci sarà tempo per questo fin
troppo presto.» Quelle parole provocarono una mite risata da parte dei
ministri seduti in fondo alla tavola e di fronte a me. Crispin sorrise.
Lanciai un'occhiata a Gibson e inarcai le sopracciglia. «La fortuna
passa dovunque, e le circostanze attuali ci concedono un momento di
vantaggio nonostante la recente tragedia di Cai Shen.»

Avevo osservato spesso mio padre in quella modalità didattica e
imperiosa. I suoi occhi -- uguali ai miei -- non si posavano mai su un
punto o una faccia ma vagavano su tutto quello che lo circondava. La sua
voce di basso arrivava lontano, risuonando nel petto piuttosto che nelle
orecchie, e lui assumeva un'aria di freddo magnetismo che piegava alla
sua volontà tutti quelli che lo ascoltavano. In un'altra epoca e in un
universo più piccolo avrebbe potuto essere Cesare, ma il nostro Impero
aveva abbondanza di Cesari. Li generavamo. Quindi lui era condannato a
sopportare Cesari ancora più grandi.

«È vero che i Pallidi divorano la gente?»

Crispin. Il diretto Crispin privo di tatto. Sentii i muscoli irrigidirsi
in ogni persona seduta a quel lungo tavolo e chiusi gli occhi, bevendo
un sorso di vino -- un Carcasson blu -- e aspettando che scoppiasse la
tempesta.

«Crispin!» La voce di mia madre risuonò in un sussurro non troppo
sommesso e nel riaprire gli occhi la vidi fissare con espressione
rovente mio fratello, che sedeva con lo sguardo fisso sulla direttrice
del Consorzio. «Non a tavola!»

Ma Adaeze Feng si limitò a rivolgerle un sorriso in tralice. «È tutto a
posto, lady Liliana. Un tempo siamo stati tutti bambini.» Crispin però
non era un bambino, aveva quindici anni ed era un efebo avviato alla
virilità.

«Una volta,» continuò, del tutto inconsapevole del suo passo falso «ho
sentito dire da un marinaio che era vero, che usano le persone come
cibo. È così?» Si protese in avanti con espressione intensa, e non avrei
potuto dire che fosse mai apparso più interessato a qualcosa neppure per
tutto l'oro di Foro.

Un altro dei dirigenti del Consorzio prese la parola con voce più cupa
delle profondità marine. «È molto probabile che sia vero, giovane
signore.» Mi girai a guardare chi aveva parlato, che sedeva accanto a
Gibson e a tor Alcuin, a metà della lunghezza del tavolo, vicino a una
zuppiera piena di fumante brodo di pesce e a un assortimento di vini in
brocche rosse. Era l'uomo con la pelle più scura che avessi mai visto --
anche più scura di quella della direttrice e perfino dei miei capelli --
cosa che faceva apparire i suoi denti candidi come stelle quando
sorrideva. «Ma non sempre. Più spesso portano via la popolazione nativa
di una colonia e la usano come schiavi.»

«Oh.» Crispin parve deluso. «Quindi non sono tutti cannibali?» La sua
espressione si fece avvilita, quasi avesse sperato che gli alieni
fossero tutti mostruosi mangiatori di uomini assassini.

«Non lo sono tutti.» Tutti mi fissarono e mi resi conto di essere stato
io a parlare. Trassi un profondo respiro per ricompormi. Dopotutto,
quella era la mia area di esperienza. «Mangiano noi, non si divorano gli
uni con gli altri.» Quante ore avevo dedicato allo studio dei Cielcin,
con Gibson? Quanti giorni avevo passato a dissezionare il loro
linguaggio, estrapolandolo dai pochi testi e dalle comunicazioni
intercettate durante i trecento anni di guerra? Mi avevano affascinato
fin da quando avevo imparato a leggere e forse perfino prima di allora,
e il mio tutore non mi aveva mai rifiutato le lezioni extra che gli
avevo richiesto.

Lo scoliasta dalla pelle scura annuì. «Il giovane signore ha
perfettamente ragione.» In seguito appresi che non era così e che i
Cielcin si mangiavano a vicenda con la stessa prontezza con cui
divoravano qualsiasi cosa. Solo che a quel tempo non lo sapeva nessuno.

«Terence...» Il giovane ministro Gong Sun posò una mano sulla manica
dell'uomo dalla pelle scura.

Questi, Terence, scosse il capo. «So che è un argomento scabroso di cui
discutere a tavola. Chiedo scusa sir Alistair, lady Liliana, ma i
giovani signori devono comprendere quale sia la posta in gioco. Ormai
siamo in guerra da trecento anni. Alcuni direbbero che dura da troppo
tempo.»

Mi schiarii la gola. «I Cielcin sono nomadi e carnivori quasi fino
all'eccesso. Allevare bestiame nello spazio non è facile anche se si
simula la gravità. È più facile prelevare quello che possono dai
pianeti, e un agglomerato migratorio di Cielcin è composto da una media
di diecimila elementi, per cui di certo non possono aver prelevato tutti
gli abitanti di Cai Shen.»

«I rapporti dicono che era un agglomerato molto grande.» Le sopracciglia
inesistenti di Terence si sollevarono. «Conosci molto bene i Cielcin.»

Da un punto più in giù lungo la tavola giunse la voce acuta di Gibson.
«Il giovane signore Hadrian si interessa ai Cielcin da molti anni,
messere. Gli ho insegnato anche il linguaggio degli alieni, e lo parla
molto bene.»

Abbassai lo sguardo sul piatto per nascondere un sorriso che mi stava
attraversando il volto, timoroso che lord Alistair potesse averlo visto.

La direttrice Feng si girò sulla sedia per guardarmi e percepii un
rinnovato interesse verso di me da parte sua, come se mi stesse vedendo
per la prima volta. «Ti interessano i Pallidi, vero?»

Annuii soltanto perché non mi fidavo di parlare finché non avessi
ricordato le buone maniere. Quella che mi parlava era la direttrice del
Consorzio Wong-Hopper. «Sì, madama direttrice.»

Lei mi sorrise, e per la prima volta mi accorsi che i suoi denti erano
di metallo e riflettevano la luce delle candele sul tavolo.
«Estremamente lodevole. È un interesse raro in un palatino, e in
particolare per uno della casta nobiliare dell'imperatore stesso. Sai,
dovresti considerare una carriera alla Cappellania.»

Invisibili sotto il tavolo, le nocche della mia mano sinistra
sbiancarono contro il ginocchio, e riuscii a stento a costringermi a
sorridere. Niente avrebbe potuto essere più lontano dai miei desideri.
Io volevo essere uno scoliasta, un membro dei Corpi di Spedizione.
Volevo viaggiare sulle astronavi, andare dove nessuno era mai stato
prima e piantare la bandiera imperiale in tutta la galassia, vedere cose
strane e meravigliose. Essere incatenato a una carica, men che meno alla
Cappellania, era l'ultimo dei miei desideri. «Grazie, madama.» Una
rapida occhiata a mio padre fu sufficiente ad avvertirmi che non dovevo
aggiungere altro.

«O magari con noi, se tuo padre potrà fare a meno di te. Qualcuno dovrà
fare affari con quelle bestie, una volta che la guerra sarà finita.»

Mio padre era rimasto marcatamente in silenzio durante tutto quel
dialogo e non potevo fare a meno di sentire che la sua ira era
imminente. Guardai verso di lui, seduto accanto a mia madre con la testa
leggermente inclinata mentre ascoltava un valletto che era venuto a
riferirgli un messaggio. Mormorò un'istruzione, per cui era distratto
quando Crispin commentò: «Potreste vendere loro del cibo!» Sul volto di
mio fratello si dipinse un macabro sorriso a cui la direttrice rispose
con un suo sorriso tagliente come la lama di un bisturi.

«Suppongo che lo faremo, giovane signore. Noi vendiamo qualsiasi cosa a
chiunque. Prendi questo vino, per esempio.» Accennò alla bottiglia da
cui stavo bevendo, un Carcasson St-Deniau Azuré. «Un'annata eccellente,
arconte, devo riconoscerlo.»

«Grazie, madama direttrice» replicò mio padre. Senza neppure guardarlo,
sapevo che il suo sguardo era su di me. «Anche se trovo strano che
abbiate una mente tanto aperta nei confronti dei Cielcin, soprattutto se
si considera la recente tragedia.»

La direttrice accantonò quel commento con un gesto mentre posava
coltello e forchetta sul piatto. «Oh, l'imperatore ne uscirà vittorioso,
che la Terra benedica il suo nome. E la coppa della Misericordia
trabocca, o così dicono i priori.»

Uno dei ministri giovani -- una donna con strisce dorate tatuate sul
pallido cuoio capelluto -- si protese verso la direttrice, dicendo: «Di
certo alla fine della guerra i Pallidi dovranno diventare sudditi del
Trono Solare.»

«Devono proprio?» chiese mia madre, inarcando le eleganti sopracciglia.
«Mi sentirei meglio se scomparissero.»

«Questo non succederà mai» replicai in tono tagliente, consapevole che
avevo fatto un errore. «Hanno un vantaggio su di noi.» Il volto di
entrambi i miei genitori si era fatto duro come pietra, e dalla tensione
nella mascella di mio padre compresi che stava per parlare.

Il giovane ministro del Consorzio però lo prevenne. «Cosa intendi dire,
signore?» chiese con un dolce sorriso.

«Noi viviamo su pianeti, i Cielcin sono Extrasolari» spiegai,
riferendomi ai barbari dello spazio profondo che si spostavano nel Buio
fra le stelle, sempre in movimento, depredando navi commerciali. «Non
hanno una casa, solo agglomerati migratori...»

«I loro \emph{scianda}» interloquì Gibson, usando il termine cielcin.

«Esatto!» Trapassai con la forchetta un boccone di pesce rosa e lo
mangiai, facendo una pausa a effetto. «Non possiamo neppure essere certi
di averli spazzati via. Anche se distruggiamo un intero agglomerato --
un intero \emph{scianda} -- basta che una sola delle loro navi riesca a
fuggire per garantire la loro sopravvivenza. Sono atomici, proteiformi,
e non si può annientare questo con la forza militare, madre, messeri,
signore. Non è possibile. Lo sterminio totale è impossibile.» Mangiai un
altro boccone. «Lo stesso vale per noi, ma la maggior parte della nostra
popolazione è vincolata a un pianeta, quindi subiamo attacchi più duri,
giusto?» Guardai verso la direttrice, facendo affidamento sulla visione
dell'Impero di qualcuno che navigava da tutta una vita perché
confermasse le mie parole.

Lei parve sul punto di farlo, ma proprio allora mio padre ingiunse:
«Hadrian, basta così.»

Adaeze Feng sorrise. «Non ti preoccupare, arconte.»

«Lascia che sia io a preoccuparmi per mio figlio, madama direttrice»
replicò lord Alistair, in tono sommesso, posando il boccale di
cristallo. Una serva si affrettò a riempirlo di nuovo da una caraffa
decorata con ninfe di bosco, ma mio padre la allontanò con un cenno. «In
particolare quando corteggia in questo modo il tradimento.»

\emph{Tradimento}. Riuscii a stento a mascherare la mia sorpresa e
serrai maggiormente la mascella. Di fronte a me Crispin fece una smorfia
e sillabò in silenzio qualcosa che sembrava essere `traditore'. Sentii
il rossore che mi strisciava lungo il collo e l'imbarazzo che scorreva
come argilla umida.

«Non pensavo...»

«No, non pensavi» dichiarò mio padre. «Chiedi scusa alla direttrice.»

Abbassai lo sguardo sul mio piatto, fissando con occhi roventi i resti
del mio salmone al forno e funghi arrostiti... avevo evitato alcuni dei
cibi più esotici preparati per i nostri ospiti extraplanetari. Furente,
rimasi in silenzio, colpito dal fatto che mio padre non mi chiamava mai
per nome, mi parlava solo per impartire ordini oppure non lo faceva
affatto. Ero una sua estensione, l'incarnazione della sua eredità. Non
ero una persona.

«Non c'è niente da perdonare» affermò la direttrice, lanciando una
rapida occhiata ai suoi ministri. «Ma basta con questi discorsi. È stato
un pasto delizioso, lord Alistair, lady Marlowe...» Chinò la testa sul
tavolo. «Dimentichiamo questa conversazione, i ragazzi non avevano
cattive intenzioni -- nessuno dei due -- ma... e se adesso potessimo
tornare agli affari?»
