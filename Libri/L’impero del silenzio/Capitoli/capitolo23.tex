\chapter{Resurrezione nella morte}

Passai il resto della notte nascosto in una zona di carico dietro un
magazzino, raggomitolato in mezzo ad alcuni fusti d'acciaio privi di
contrassegno. Non dormii, ma come potevo farlo? L'aria stessa mi dava la
sensazione di soffocarmi, tanto densa che mi aderiva ai polmoni. Era
forse frutto di qualche antica opera di terraformazione? Oppure era
ecologia naturale? Non sapevo niente di questo mio nuovo mondo, Emesh.
La forza di gravità era senza dubbio più elevata, il che spiegava perché
mi sentissi gli arti di piombo. Una volta avevo sentito la storia di un
uomo, un magio consigliere dell'imperatore, che poteva valutare la forza
di gravità di un pianeta in base alla velocità con cui uno yo-yo
tascabile cadeva e gli tornava in mano, ma era un trucco che non avevo
mai imparato.

Quando finalmente giunse l'alba il sole parve sbagliato. Quello di Delos
era un piccolo punto, grande la metà di un kaspum d'argento e con un
diametro che non superava la sezione trasversale del mio mignolo, mentre
per contro il sole di Emesh era un rabbioso e lacrimoso occhio rosso
grande quanto il mio pugno, che cuoceva le strade e trasformava le
costruzioni di mattoni della mia nuova casa in un forno a cielo aperto,
facendo increspare l'aria con onde di calore visibili a occhio nudo. Gli
abiti rubati mi aderirono ben presto al corpo e sentii l'acqua che
evaporava dentro di me sotto quel cielo chiazzato di arancione, ocra e
rosa, e striato di alte nuvole incostanti.

La città in sé stessa era interessante, una bassa distesa di dimensioni
ignote. I suoi edifici -- la maggior parte dei quali non avevano più di
tre o quattro piani -- si stendevano come una rete su un paesaggio
altrettanto basso. Una o due volte intravidi il mare che mi sbirciava
fra le costruzioni lungo strade tortuose e sopra la testa della folla
sempre più fitta. Solo che anche il mare era sbagliato. Le sue acque
risplendevano di un verde malsano chiazzato qua e là di azzurro e senza
traccia di argento.

Appresi in seguito che Emesh era morto dal punto di vista tettonico, e
che la sua aria e la sua acqua avevano di più in comune con la sorella
della Terra, Marte, di quanto avessero con il nostro Mondo Natale. A
parte un singolo continente le sue rare masse di terra -- poco più di
isole, secondo gli standard di qualsiasi pianeta degno di questo nome --
erano bassi accumuli sedimentari formatisi su scogli e barriere
coralline oppure dragati dal letto dell'oceano poco profondo e
accumulati artificialmente. La città in sé, Borosevo, era costruita su
pilastri d'acciaio conficcati in profondità nel terreno e nel corso di
lunghi anni quell'ottuso fatto architettonico aveva dimostrato quanto
fosse folle tramite le ragnatele di crepe che si erano formate nei muri
e nei piloni. Il palazzo del conte, però, dominava la bassa massa del
colosseo e i nove minareti del sacrario della Cappellania, con la sua
cupola di rame e la cupa bastiglia. Il palazzo posava su uno ziggurat di
cemento grigio come l'oceano del mio pianeta, una grossa piramide senza
punta alta mille piedi che dominava la bassa distesa di tetti di stagno
di Borosevo. Le sue guglie erano di vetro e di arenaria, il tetto dalle
tegole rosse spiccava luminoso sotto l'occhio color sangue del sole.

Usando quell'enorme edificio come punto di riferimento mi spostai lungo
il perimetro della città, ragionando su come l'astroporto dovesse essere
vicino. La gentilezza degli estranei è uno dei più orgogliosi miracoli
dell'umanità, ma ha i sui limiti, ed essi mi dicevano che chiunque mi
avesse trovato non mi avrebbe trasportato molto lontano dal vicolo in
cui ero finito, e che quel vicolo non doveva essere lontano dalle navi.
La mascella mi si serrò per l'ira mentre un nodo dovuto alla fame mi si
formava nello stomaco. Quello che però cominciava a torturarmi era la
sete, perché ero arido come una spugna inutilizzata e mi sentivo sempre
più stanco.

Un paio di ornithon si levarono in volo al mio avvicinarmi, serpenti con
sei ali che salirono nell'aria per raggiungere la scia di condensazione
di una lontana navetta. Guardai quelle creature {allontanarsi} con la
bocca arida spalancata, perché non avevo mai visto niente di simile.
Quello era qualcosa di più tangibile, di più reale della forza di
gravità o dell'aria densa. Strane creature, strani climi.

L'aria fredda del terminal dell'astroporto -- mantenuta all'interno da
un campo statico -- mi investì come l'infrangersi di un'onda, e mi
ritrovai ad annaspare come un pesce nell'atmosfera più secca e
rarefatta, piegato in due con le mani sulle ginocchia appena oltre la
soglia. Dovevo essere un vero spettacolo, con i piedi nudi e infangati,
le gambe dei pantaloni rubati che erano già strappate e chiazzate a
causa della mia fuga notturna, e i lunghi capelli del colore del carbone
che mi aderivano alla faccia fin quasi al mento. Un paio di donne in un
completo viola mi aggirarono tenendosi il più alla larga possibile,
mentre chiudevano ombrelli identici che avevano usato per difendersi dal
sole. Non sapevo come procedere, perché per tutta la mia vita c'era
stato chi mi approntava una navetta e gente che mi sgombrava il passo.
Non avevo mai affrontato una situazione del genere dall'esterno... dal
basso.

«Messere?» chiese una voce cortese, invadendo la mia confusione e
indecisione. «Messere, non puoi stare qui dentro.» Guardandomi intorno
vidi un giovane che indossava una sorta di uniforme costituita da un
caffettano e teneva le mani congiunte davanti a sé, sbirciandomi da una
rispettosa distanza e da sotto la tesa di un cappello piatto. «Stai
turbando la clientela.»

Lo fissai, con la mente d'un tratto del tutto vuota. «La sto turbando?»
ripetei, facendo scorrere lo sguardo sui volti inespressivi delle
persone che mi circondavano, notando come cercassero di non fissarmi. Di
colpo compresi. «Signore, io... perdonami» risposi. «Ero a bordo di una
nave, la \emph{Eurynasir}, e forse c'è stato qualche errore, perché mi
sono svegliato in città, in una clinica...»

Il titolo onorifico sorprese l'uomo con il caffettano, che guardò con
incertezza i due agenti della sicurezza con abiti cachi che lo
affiancavano come i littori avrebbero potuto fare con un grande lord
dell'Impero. Ripeté quella parola fra sé, in tono sommesso. «Signore?»
Poi la ripeté con maggiore energia mentre un sorriso teso affiorava sul
suo volto effemminato dai lineamenti fini. «Su una nave, hai detto?»
Qualcosa nei suoi modi, nella fredda precisione di quel sorriso mi disse
che anche se aveva compreso, anche se mi credeva, questo non aveva
importanza.

Mi ersi in tutta la mia statura, tale da darmi alcuni pollici di
vantaggio su quel piccolo plebeo con il suo cappello piatto e la sua
ampia veste. «Sì» ribadii, piantandomi le mani sui fianchi. «La
\emph{Eurynasir}, proveniente da... da...» Mi frugai nella memoria per
ripescare il nome dei pianeti menzionati da Demetri, ma alla fine mi
arresi. «Da Delos. Controlla i registri di volo. Il capitano era uno
jaddiano, Demetri Arello.»

Qualcosa nel mio volto e nella mia voce... forse perfino nell'uso
dell'onorifico subordinato... indussero l'uomo a una pausa di
riflessione. Spinse indietro la manica, lanciò un'occhiata alla donna
dalla mascella squadrata e dall'uniforme cachi alla sua sinistra e
richiamò un display olografico sul suo terminale da polso. «Non vedo una
nave con quel nome.» Il suo sorrisetto si fece più affilato, al punto da
poter tagliare il vetro. «Sei certo che sia giusto?»

«L'\emph{Eurynasir}» ripetei, scandendolo lettera per lettera con il
respiro un po' affannoso, poi feci una pausa e mi ricomposi. «Deve
essere qui.» Abbassai la voce e mossi mezzo passo verso l'uomo, ma mi
immobilizzai con le mani ben visibili quando le due guardie si tesero e
sollevarono un bastone telescopico. «Senti, signore, mi hanno scaricato
in un vicolo.» Mi guardai intorno, cercando di essere assolutamente
certo che nessuno mi potesse sentire, e per un momento la frescura
dell'aria mi distrasse, spezzando il filo dei miei pensieri per cui le
parole che mi sfuggirono dalle labbra risultarono affannose: «Sto
cercando di capire cosa sia successo.» L'uomo fece un gesto e le guardie
vennero avanti di prepotenza, afferrandomi sopra i gomiti. «Mi devi
ascoltare!» ringhiai, cercando di liberarmi, ma la donna con la mascella
marcata mi colpì al ventre.

«Portatelo fuori di qui» disse l'uomo con il caffettano, agitando una
mano con fare sprezzante e girandosi per andarsene.

Debole com'ero liberai il braccio e mi lanciai in avanti. «Deve essere
qui.»

L'uomo con il caffettano si immobilizzò e abbassò la mano con un gesto
secco. «Forse hanno scaricato questo rottame.» Poi la donna con la
mascella marcata mi colpì di nuovo in pieno ventre. Mi piegai su me
stesso e rimasi in quella posizione mentre l'uomo sfoggiava il suo
sorriso affilato. «Non tornare qui dentro, hai capito?» Quel sorriso
aveva qualcosa di bello, esprimeva una condiscendenza così ampia e
precisa. Non replicai e lasciai che mi trascinassero lungo un corridoio
dalle piastrelle bianche, oltre alte finestre scosse dal lontano decollo
di razzi dalle fosse di partenza che costellavano la massa di cemento al
di là del terminal, arancione sotto la devastante luce solare. La musica
metallica che scaturiva dagli altoparlanti del terminal mi risuonava
nelle orecchie, priva di senso. Mi buttarono fuori dalla porta
posteriore e in una zona di carico simile a quella dove avevo trascorso
la notte precedente, in città.

`Non vedo una nave con quel nome.' Quelle parole mi risuonavano dentro e
mi morsi un labbro mentre riflettevo e mi massaggiavo il livido che ero
certo mi si stesse formando sul ventre. Lo stomaco mi si rigirò,
rosicchiando sé stesso, perché non avevo mangiato niente a parte il
brodo leggero che mi avevano dato alla clinica, e prima di allora non
avevo né mangiato né bevuto dopo il vino che avevo diviso con Demetri a
Karch. `Non vedo una nave con quel nome.' Cosa significava? Che
l'\emph{Eurynasir} non era atterrata nello spazioporto? Mi sedetti su un
muretto di cemento all'ombra di una palma, ascoltando il tuono costante
di un lontano razzo a fusione che saliva verso il cielo. Questa città
aveva un secondo astroporto? Non mi sembrava probabile, considerata la
distanza necessaria per isolare la gente dal fragore devastante di quei
motori a razzo che crepitavano nel cielo.

La mia mente procedette con lentezza meccanica attraverso i passi della
comprensione, con i pensieri distorti dal calore e dalla fame. Però non
avevo ancora abbandonato la speranza. Per quel che ne sapevo, Demetri e
il suo equipaggio potevano essere in attesa, nascosti in qualche fossa
di decollo o al limitare estremo del campo di atterraggio dello
spazioporto. Oppure, disse dentro di me una piccola voce che somigliava
troppo a quella di Crispin, oppure... sono tutti morti. Quel pensiero mi
immobilizzò, raggelandomi nonostante la calura di quel pianeta
infernale. Come Cid Arthur, rimasi seduto a lungo all'ombra di
quell'albero, guardando le barche risalire il canale che avevo di fronte
`Remate verso casa, ragazzi miei. Remate verso casa.'
\leavevmode\\
\leavevmode\\
\textit{Non vedo una nave con quel nome.}

\textit{Quindi non era nell'astroporto. Non era lì.}

\textit{Forse hanno scaricato questo rottame.}

\begin{figure}
	\centering
	\def\svgwidth{\columnwidth}
	\scalebox{0.2}{\input{divisore.pdf_tex}}
\end{figure}

Mi ci volle il resto della giornata e un'impegnativa conversazione con
un prefetto cittadino per trovare la risposta. La donna era stata sul
punto di arrestarmi per vagabondaggio, ma i miei modi l'avevano
trattenuta dal premere il grilletto dello storditore che portava al
fianco. Vedi, nel crescere ho sentito di continuo delle storie di
equipaggi scomparsi nelle profondità dello spazio, con le loro navi
vuote che entravano in porto e in sistemi solari sulla spinta di motori
a curvatura morenti. Gli uomini dicevano che era colpa degli Extrasolari
che depredavano le navi mercantili e rapivano gli equipaggi perché
prestassero servizio sulle loro enormi navi nere, inserendo a forza
delle macchine nella loro carne per schiavizzarli.

Ho visto quelle navi nere, ho percorso i loro corridoi. Ho visto
eserciti immortali, uomini-macchina vuoti e insensibili. C'è qualcosa di
vero in quelle storie. Altri uomini però affermano che erano i Cielcin a
depredare quelle navi, raccogliendone gli equipaggi come si potrebbe
fare con un branco di pesci nel mare, presi in una rete. Sospetto che
entrambe le cose siano vere, e anche altre ancora. Sospetto che alcune
navi siano scomparse negli spasimi della vendetta, nelle guerre fra
Casati, per via delle vendette familiari tipiche del nostro Impero. E
sospetto che sia stata la sfortuna, una cattiva gestione, qualche errore
o forse un incidente che aveva indotto il capitano ad abbandonare la sua
nave.

Cosa sia successo non ha importanza, e adesso sono troppo vecchio perché
mi importi. Ci sono sempre navi vuote e marinai morti. Il mare più nero
e di gran lunga più vasto che riempie il vuoto fra i soli come acqua può
essere crudele quanto il mare dell'antichità. A quel tempo però mi
importava, quindi mi ritrovai -- affamato e sofferente -- ai cancelli di
un'enorme serie di hangar che si stendevano ai margini del mare e della
città. Il disco arancione del sole si increspava nel cielo pomeridiano,
distorto e scintillante nell'aria densa. Quasi mi sembrava di poter
sentire il vapore che si levava dalle acque torbide e spumose, dai
canali sporchi, verdi come foreste, che scorrevano come vene attraverso
la grande città, e da me stesso. Non era stato facile trovare quegli
hangar, quindi piuttosto che ripetere l'episodio dell'astroporto
{ignorai} le due guardie che oziavano nel casotto di guardia vicino alla
strada principale e mi avvicinai all'alta recinzione che cingeva il
deposito di riciclaggio, seguendola fino a dove la maglia metallica
terminava a ridosso di un edificio. Quella recinzione era fatta di un
materiale economico, misero e antiquato, il genere di difesa che non si
sarebbe mai trovata su uno dei vecchi mondi imperiali. Era qualcosa di
semplice, assolutamente privo di artifici, e non era neppure
elettrificata. La scavalcai con facilità, grato per la prima volta da
quando mi ero svegliato su Emesh di essere senza scarpe, perché le dita
dei piedi mi aiutarono ad arrampicarmi sulle fasce intrecciate di
metallo.

Procedetti lungo l'arco convesso della costa, passando dall'ombra di un
hangar a un'altra e sbirciando attraverso le finestre sporche e le porte
aperte mentre facevo del mio meglio per camminare con aria decisa, per
dare l'impressione di appartenere a quel posto, cosa tutt'altro che
facile a causa del mio aspetto lacero e della mia alta statura da
palatino. Nessuno mi fermò e, a parte tre vecchi in piedi nell'ombra di
una tettoia d'acciaio, intenti a bere da bottiglie marroni e a ridere,
pulendosi le mani su sbiadite tute marroni, non vidi nessun altro.

Ogni hangar conteneva una nave, nei più grandi ce n'erano anche due o
perfino tre, tutte navi leggere come la \emph{Eurynasir}, capaci di
passare dallo spazio alla superficie di un pianeta. Il loro guscio nero
e bianco fatto di ceramica e di adamant mostrava in tutte segni di
danni: bruciature da attrito, cicatrici da impatti meteorici, striature
al carbonio prodotte da armi da fuoco. Navi vuote. Ci sono sempre navi
vuote. Mi sussurravano, parlando di battaglie, di pirati, di xenobiti
che emergevano ululanti dal Buio, di come antichi vascelli come questi
fossero sprofondati e si fossero persi in mare, inghiottiti dalle acque
ostili e distrutti dal loro peso. Lo spazio non cercava un equilibrio
del genere e permetteva che i relitti rimanessero intatti, liberi da
vincoli. C'erano intere corporazioni dedicate al recupero di navi
danneggiate come quelle.

Nessuna delle navi nel primo hangar o anche nel quinto, era
l'\emph{Eurynasir.} Con la sera imminente, enormi mosche cominciarono a
emergere dai loro nascondigli diurni e a riempire l'aria del loro
ronzio. Le allontanai con le mani, mentre i miei piedi callosi
strisciavano sull'asfalto rovente.

Cominciando a cedere al panico, con lo stomaco contratto dalla fame e
dal bisogno di una risposta, aggirai un carrello elevatore parcheggiato
appena fuori da una di quelle enormi strutture e per poco non andai a
sbattere contro uno dei vecchi che avevo visto in precedenza. Era basso,
e largo quasi quanto io ero alto, tutto muscoli e peli. La testa era
priva di capelli e del colore del bronzo invecchiato, la faccia si
perdeva in un ammasso di barba come non avevo mai visto. I peli gli
sporgevano dalle guance fin quasi agli occhi e le sue braccia mi
ricordarono quelle dell'omuncolo, Saltus, perché gli arrivavano fin
quasi alle ginocchia. Mi fissò con occhi roventi, con allarme e
indignazione che ribollivano appena sotto la superficie di quella faccia
pelosa ed espressiva. «Nel nome della Terra, chi sei?»

Invece di rispondere mi allontanai dalla faccia i capelli fradici di
sudore. «Ero su una nave, la \emph{Eurynasir}» dissi. «È qui?»

Il grosso uomo sbatté le palpebre e lanciò un'occhiata in tralice alle
mie spalle e verso il cielo aperto che si vedeva dietro di me. «A te che
importa?» Sputò sul cemento ai suoi piedi.

Cercando di non perdere la pazienza, già provata dalla calura, dallo
sfinimento e dalla fame, mi ripetei, questa volta con maggiore lentezza
e precisione. «Ero \emph{sulla} nave, messere.»

All'inizio avevo creduto che l'uomo fosse vecchio per via delle rughe
sul suo volto, per come la pelle sotto la barba arruffata appariva
emaciata e squamata, ma fra i plebei la vecchiaia è una cosa ingannevole
e lui poteva avere anche meno di quarant'anni, senza contare che quegli
Emeshi erano tutti tozzi e muscolosi come tori per il peso della
maggiore forza di gravità. Mi guardò dal basso in alto con gli occhi
socchiusi e i pugni grossi come prosciutti piantati sui fianchi.

«Skag!» chiamò un'altra voce. «Dove sei andato?»

«Sono qui, Bor!» chiamò l'operaio dell'hangar da sopra la spalla. «Quel
pesce che abbiamo buttato sul retro è tornato!» Un altro uomo aggirò
l'angolo dell'hangar, un individuo più pallido del primo, con la pelle
rossa che si spelava come quella della vecchia nella clinica. In essa
sembrava esserci qualcosa che non andava: erano scottature dovute al
sole? Cicatrici? Oppure era il marchio di una malattia passata da tempo?

Sollevai le mani, cercando di mostrarmi pacifico. «Signori, non voglio
guai, voglio soltanto...» Mi interruppi perché intravidi la fascia
d'argento che il secondo uomo portava al dito mignolo. Una fascia
d'argento con incastonata una corniola. La voce mi si tese nella gola,
risuonando acuta e petulante. «Quello è mio!»

I lavoranti si guardarono a vicenda, senza che nessuno dei due sapesse
come rispondere. Ne dedussi che le loro vittime non tornavano spesso.
Forse ero addirittura il primo. Alla fine risposero, ma non con le
parole. Di colpo mi sentii come se fossi di nuovo nelle strade di
Meidua, di fronte a quei delinquenti con le loro motociclette. Quando
giunse, il primo colpo mi colse di sorpresa centrandomi la mascella e
facendomi barcollare. Quando arrivò il secondo ero pronto e rotolai
lontano mentre il piede mi calava addosso, con il risultato che il
tallone dell'uomo ricadde sul cemento invece che sulle ossa. Mi rimisi
in piedi. Non sarei morto e non mi sarei lasciato vittimizzare come
avevo fatto anni e settimane prima a Meidua. Snudai i denti e sputai: la
saliva era chiazzata di rosso.

«C'era una lettera» dissi, tenendo le mani aperte fra me e i due uomini.
«Era scritta a mano. È tutto quello che voglio.» Dopo un momento
aggiunsi: «E le mie scarpe.» Gli occhi mi tradirono, perché mentre
pronunciavo quelle parole il mio sguardo si posò sull'anello con il
sigillo -- il mio anello -- sulla mano dell'altro lavorante.

In una buona giornata avrei potuto sconfiggerli entrambi, se fossi stato
sano, integro, privo di ferite, ben riposato e ben nutrito. Se fossimo
stati a Delos, il semplice sforzo di stare in piedi non avrebbe privato
di forza i miei muscoli. O forse se quei due uomini non fossero stati
due colossi rivestiti di strati di muscoli per una vita di duro lavoro
con quella forza di gravità.

Forse.

Bloccai il colpo successivo, cedendo terreno e barcollando un poco
quando i miei piedi callosi strisciarono sul terreno irregolare. Fui
fortunato perché i due si intralciarono a vicenda e i loro colpi
risultarono selvaggi e privi di coordinamento anche se spaventosamente
forti. Nel mio stato di debolezza cercai di deviare i loro attacchi più
che di pararli, sapendo benissimo che non potevo arrestarli in modo
diretto. Il ricordo di migliaia di sedute di addestramento con Felix o
con Crispin mi riaffiorò nella mente. I due scomparvero altrettanto in
fretta, svanendo come devono farlo le distrazioni in momenti del genere
e ritirandosi finché i soli ricordi furono quelli radicati nei muscoli e
nel sangue. Un gancio spaventoso mi raggiunse nelle costole, facendomi
barcollare. Troppo lento, pensai, più furente che sofferente. Troppo
debole.

L'uomo che portava al dito il mio anello rubato si affrettò a sfruttare
l'apertura che avevo creato incespicando goffamente e mentre già pensavo
di scappare agì come braci su lana asciutta. Trasformai il mio
barcollare in un volteggio e afferrai il polso dell'uomo, torcendolo in
modo che tutto il mio peso si abbatté sul braccio rivolto verso il
basso. Il gomito si ruppe con uno scricchiolio rivoltante e il grido
dell'uomo si trasformò in un urlo. Quel suono fermò l'altro avversario,
il tizio peloso che mi aveva affrontato per primo, per il tempo
sufficiente a permettermi di sfilare l'anello. Avevo il petto che
affannava come un mantice perché non c'era abbastanza aria, o forse ce
n'era troppa. La vista mi si oscurava lungo i contorni, sfumandosi
d'ombra.

Devo essere sembrato un animale, lì in piedi sopra l'uomo con il braccio
rotto. Ero un animale. Un rumore di passi giunse da dietro l'angolo
dell'edificio più lontano e a denti stretti dissi: «Voglio solo riavere
le mie cose.»

«Cosa diavolo sta succedendo?» chiese una rude voce femminile.

«Questo bastardo ha rotto un braccio a Bor» rispose l'uomo barbuto,
senza togliermi lo sguardo di dosso.

Fu allora che apparvero gli altri, sette in tutto, simili ai primi due
quanto è possibile immaginare come se provenissero dallo stesso stampo:
tutti tozzi e larghi di spalle, con un'identica muscolatura dovuta allo
stesso tipo di vita. La donna che aveva parlato era calva, tranne per un
velo di peluria sul cuoio capelluto, e il suo volto sgradevole era
deturpato da una brutta voglia color vino. Esaminò la mia faccia e in
essa qualcosa la indusse a soffermarsi a riflettere, ma un momento più
tardi sorrise. «Torna a casa, ragazzo.» Lanciò un'occhiata all'uomo
ferito che gemeva ancora a terra, con il gomito che cominciava a farsi
violaceo. «Oppure ti succederà di peggio.»

«Mi ha rotto un braccio, Gila, dannazione» ringhiò a denti stretti
l'uomo a terra. «Per l'amore della Terra, chiama i prefetti.»

Mi infilai l'anello sul pollice sano. «Siete stati voi a tirarmi fuori
dalla nave.» Non era una domanda. «Non volete avere qui i prefetti,
vero?» Sollevai l'anello perché lo esaminassero, usandolo per
sottolineare la mia tesi. Non volevo dire niente o richiamare
l'attenzione su ciò che quell'anello significava perché farlo mi avrebbe
esposto a rappresaglie. Se fossero venuti a vedere il mio bluff avrei
dovuto rivelarmi alle autorità, a mio padre. Stavo camminando su una
linea sottile e affilata come un rasoio, con la minaccia di violenza che
incombeva su di me come la spada di Damocle. «Voglio solo riavere le mie
cose.»

La donna, Gila, sputò come aveva fatto l'uomo barbuto. «La nave non c'è
più. Ha risalito il pozzo gravitazionale questa mattina per essere
riparata.»

Quel momento di tregua mi aveva concesso un po' di tempo per riprendermi
e il mio respiro era rallentato. Avevo ancora i capelli incollati alla
faccia che mi coprivano in parte gli occhi e cercai di spingerli di lato
agitando la testa, ma rifiutarono di staccarsi. «Stai mentendo!» L'uomo
ferito si era sollevato in ginocchio e si alzò con l'aiuto del compagno
barbuto e di uno degli altri. «Datemi i miei effetti personali.»

«I tuoi effetti personali?» sogghignò uno dei lavoranti. «Chi diavolo
sei? Il principe di Jadd?»

Non abboccai alla provocazione. «Che ne avete fatto delle cose
recuperate? Dov'è l'equipaggio?»

«Quando è entrata nel sistema la nave era vuota» rispose Gila.
«L'equipaggio ha tagliato la corda, ha preso le navette ed è fuggito
lasciando il tuo miserabile culo nel congelatore.»

Mi umettai le labbra. «Allora lo ammettete?»

«Fottiti, ragazzo.» La donna agitò una mano. «Alza i fottuti tacchi dal
mio posto di lavoro.»

Avanzai di un passo, e ancora oggi non so dire se fu una mossa calcolata
o pura e cieca stupidità aristocratica. «C'era una lettera, scritta a
mano.»

«Un messaggio d'amore della tua ragazza?» commentò l'uomo barbuto.
«Oppure sei tu la ragazza?» Dal resto della folla si levò una risata, un
suono più minaccioso di un ringhio. Frenai la mia avanzata e una parte
di me, quella sensata e razionale, mi sussurrò di fuggire, non verso la
recinzione ma verso la porta aperta che avevo visto, quella con le
guardie che oziavano nel casotto con l'aria condizionata. Nessuno dei
lavoranti aveva chiamato le guardie e potevo farcela, anche se non
sarebbe stato facile perché ero ancora debole e avevo bisogno di acqua.
E dovevo mangiare. Avevo avuto la meglio su quell'idiota che si reggeva
il braccio leso solo grazie al mio addestramento, ed ero stato
fortunato.

Proprio allora Gila mi rispose. «Abbiamo buttato via tutto. Vai a
frugare su una chiatta dei rifiuti.» Uno degli uomini che aveva accanto
accennò ad avanzare ma lei lo afferrò per il davanti della tuta sporca.
Lo sguardo dei suoi piccoli occhi scuri si posò sull'anello, consapevole
di cosa significasse e del pericolo che stava correndo. Era più furba di
quei delinquenti di Meidua, o forse solo meno coraggiosa. «Ora vattene.»

Sapevo che mi sarebbero piombati addosso nel momento in cui mi fossi
girato, per prendere l'anello, se non altro, quindi mi mossi
indietreggiando. Volevo dire qualcosa di arguto, di tagliente, qualcosa
che li facesse tremare negli stivali come avrebbe fatto la vista di mio
padre. Qualcosa che ghiacciasse loro il sangue. Però non avevo nessuna
idea e non dissi niente.

Se non altro fui veloce.

Lo sono sempre stato.


