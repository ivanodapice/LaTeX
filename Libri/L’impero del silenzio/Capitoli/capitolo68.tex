\chapter{Aiuto}

Eravamo tutti raggomitolati in una rientranza poco profonda dell'altura
sul mare quando arrivarono le navette che descrissero un ampio arco
lungo la costa per poi tornare verso Calagah dal mare. Le loro luci
risplendevano azzurre e bianche, i motori erano quasi silenziosi
nell'aria, sostenuti dai repellenti Royse. Arrivarono come le antiche
navi lunghe, sollevando schizzi nella risacca con i motori mentre i jet
di assetto regolavano la loro deriva. Altri velivoli saettarono oltre,
incidendo nel cielo strisce azzurre nel puntare verso il sito a ovest
dove la nave si era schiantata. Nella penombra le navette apparivano
diverse, stranamente simili a pesci, e scintillavano come ottone
nell'allontanarsi veloci.

Quelle che scesero sulla spiaggia -- una dozzina -- erano tozze
losanghe, più alte di quanto fossero larghe. La loro parte anteriore
inclinata si aprì e si trasformò in una rampa da cui scesero un momento
più tardi soldati in armatura da combattimento color ruggine con il
serpente dei Veisi intrecciato sul petto e lungo il braccio sinistro. Il
loro capo -- una centurione, a giudicare dallo stemma di traverso sul
suo casco -- passò la sua lancia a un aiutante e sganciò il sigillo del
casco, che si mise sotto un braccio nel venire avanti verso Elomas e il
giovane Karthik, seduti su una bassa sporgenza di roccia. «Sir Elomas!»
La donna era quasi calva, con la pelle color rame e segnata da uno
strato di cicatrici da ustione sopra un occhio scintillante. Ricordai di
averla vista a Fonteprofonda, una buona ufficiale e una persona
risoluta.

«Vriell!» Karthik si alzò e corse verso la donna, che si fermò per
abbracciarlo. «Sei venuta per portarci via?»

«Sì, piccolo signore!» L'ufficiale arruffò i capelli del quindicenne,
poi si raddrizzò e si rivolse al vecchio cavaliere. «Signore, se non ti
dispiace, ho l'ordine di riportarti direttamente a Fonteprofonda.»

Elomas non si era mosso dal suo posto sulla bassa sporgenza di roccia,
con gli stivali affondati nella sabbia, la giacca gettata a terra
accanto a sé e la camicia aperta fino allo sterno mentre mangiucchiava
una delle sue barrette di cioccolato. Appariva teso, pressato fino a
essere sottile come la carta, e tuttavia sorrise alla centurione
dall'aria severa. «Cosa sta succedendo, Vriell?»

«Non sono autorizzata a parlarne, signore.»

«Neppure con me?»

«Con nessuno» replicò lei, sull'attenti con la mano sull'impugnatura
della corta spada di ceramica. «Non spetta a me farlo, signore. Gli
ordini vengono direttamente dall'alto.»

«Da mio padre?» chiese Karthik, riferendosi all'arconte Veisi, e sbirciò
dal basso verso la centurione. «O dal conte?»

«Dalla tribuno-cavaliere» rispose Vriell, poi si rese conto di aver già
detto troppo e serrò la mascella. «Per favore, signore, fai salire tutti
sulle navette, subito.» Si trasse di lato, accennando con un braccio per
indicare che Elomas doveva assumere il comando. «Da questa parte.»

`Dalla tribuno-cavaliere.' Da dove mi trovavo, accanto all'altura di
basalto, studiai il volto del centurione, poi guardai verso Valka e Ada
che sedevano raggomitolate fra i pochi bagagli che i tecnici avevano
portato via dal sito. \emph{Dalla tribuno-cavaliere Raine Smythe?}
Evocai il ricordo di quella donna, che avevo visto al banchetto e poi in
occasione dell'Efebeia e del trionfo di Dorian. Una donna dal volto
squadrato e insignificante segnato dalle cicatrici tipiche dei patrizi e
dagli anonimi capelli castani tagliati corti. Subito dopo quello che
sapevamo tutti essere un attacco dei Cielcin -- che fosse o meno
confermato ufficialmente -- lei doveva aver preso il potere su Emesh
invocando qualche protocollo di emergenza, e il conte Balian doveva
averle ceduto l'autorità come ufficiale della Legione di grado più
elevato, concedendole il comando temporaneo delle sue forze personali.
Questo spiegava il verde dei Mataro misto al rosso e bianco delle
Legioni di Sua radiosità.

Le luci tremolavano ancora nel cielo, scoppiando e scintillando come
fiammelle di accendini, e altre astronavi si spostavano lungo orbite più
alte o più basse, sezionando il cielo con scie di fiamma. Di tanto in
tanto un punto di luce fioriva silenzioso fino a diventare una rosa di
fiamma fra il rosa e il rosso, che segnava la morte di qualche velivolo
spaziale. I poeti romanticizzano i combattimenti nello spazio, le opere
olografiche li raffigurano come cose fatte di fragore e di furia, ma una
battaglia non è così perfino vissuta dall'interno, e dall'esterno ci
sono solo luce e silenzio, a parte quando il cielo sembra precipitare
fragorosamente. Mi alzai dal mio posto, allungando il collo per guardare
lungo la riva in direzione del fumo dello schianto che offuscava l'aria
in lontananza, rischiarato dal basso dai fuochi al plasma e dal metallo
in fiamme.

«Quanto sono lontani i Cielcin?» domandai.

La centurione mi guardò con una luce di riconoscimento nei suoi occhi
vivaci, poi inarcò le sopracciglia -- o meglio un sopracciglio e il
tratto di cicatrice da ustione -- in un'espressione sorpresa. «Lord
Marlowe.»

«Centurione» risposi. Valka e tor Ada si avviarono con passo fermo verso
i velivoli, facendosi largo fra la folla confusa delle poche dozzine di
tecnici e di archeologi. «Sappiamo che sono i Pallidi.»

La cicatrice da ustione di Vriell sbiancò. «Allora capisci la necessità
di fare in fretta.»

«Chi c'è sugli altri velivoli?» insistetti, indicando il cielo.

«Jaddiani» rispose, incerta su quanto il mio rango la obbligasse a
condividere con me. «E la Legione.»

Lanciai un'occhiata a Valka e poi a sir Elomas. «Signore, devi
permettermi di andare con loro.»

«Cosa?» Elomas sgranò gli occhi, verdi anche nel chiarore che precedeva
l'alba. «Perché?»

«È fuori questione, Vostra signoria» intervenne la centurione, avanzando
di un passo e posandomi una mano sul braccio, al di sopra del gomito.
«Il conte non perdonerebbe mai né me né lord Veisi se ti succedesse
qualcosa di male.»

Spinsi via la mano guantata ma lei rinnovò la presa. Che mi importava se
l'arconte Veisi avesse perso favore agli occhi del conte di Borosevo?
Era solo un governatore regionale. Un barlume del gelo aristocratico di
mio padre si insinuò nella mia voce. «Toglimi la mano di dosso» sibilai.
Vriell si mostrò visibilmente intimidita, ricordandomi per un momento
che ero un palatino, e mi lasciò andare. Il fatto che sarei diventato il
consorte della futura erede del conte era un segreto ed era irrilevante,
ma già da solo il mio lignaggio era sufficiente a generare esitazione.
«Andrai a fare da rinforzo una volta messa al sicuro la squadra
archeologica?» Quando lei non rispose, limitandosi a serrare la mascella
pesante e a protendere il mento in fuori, presi la cosa come una
conferma e battei le mani. «Ottimo. Io verrò con te.»

«Vostra signoria... io...»

«Verrò con te.» La oltrepassai, dirigendomi verso un'altra delle
navette, una di quelle in cui la squadra archeologica \emph{non} si
stava riversando.

La voce di Valka si levò al di sopra del ronzio dei jet repellenti e del
lambire delle onde. «Hadrian!»

Mi girai, venendomi per caso a trovare su una leggera gobba della
spiaggia rocciosa. Le mie parole furono dirette alla centurione, anche
se mi rivolsi a tutto il gruppo come avrebbe potuto fare mio padre. «Se
qualcuno dei Pallidi è sopravvissuto, io sono in grado di parlare con
loro, di indurli ad arrendersi.» Qualsiasi cosa fosse stata sul punto di
dire, Valka rimase in silenzio. «Posso essere di aiuto, centurione, lo
sai anche tu.»

La verità era che non sapevo cosa mi avesse indotto a dire una cosa del
genere, cosa mi motivasse. La curiosità, forse? O l'orgoglio? Ancora
adesso, dopo tanti secoli, tanto sangue versato, dopo la morte di un
sole e quella di una specie, dopo tutto il bene e il male operati da me
e nel mio nome e in quello della razza umana, a suonare vero è
quell'istante, su quella riva rocciosa ai margini del mondo, in una
notte in cui il fuoco regnava e cadeva dal cielo. Trovai uno scopo. Ero
di nuovo \emph{Hadrian Marlowe}, ero un uomo posseduto che sostava a
giocherellare con il gonfiore dell'ustione da freddo che aveva sul
pollice sinistro. Avevo perso il mio anello, gettato via a Borosevo. Il
mattino successivo lo avevo cercato, ma era svanito e supponevo fosse
stato rubato da uno dei servi o raccolto da qualche ottusa macchina per
le pulizie. Non ne avevo denunciato la scomparsa in quel momento perché
l'anello era ormai praticamente inutile ora che i miei titoli e le mie
tenute mi erano stati revocati da mio padre, e comunque avevo altre cose
di cui preoccuparmi.

«È pericoloso.» La centurione scosse il capo e avanzò di nuovo verso di
me. «Devi tornare a Fonteprofonda.»

«In tal caso mi dovrai stordire, centurione, e dopo potrai spiegare ad
Anaïs Mataro perché hai stordito il suo fidanzato.» La cosa colpì nel
segno e il colore defluì dal volto della centurione. «Lasciami andare
dove posso essere di aiuto» ribadii, vedendo la mia opportunità. Vriell
spostò lo sguardo da me a sir Elomas, ma il vecchio si limitò a
scrollare le spalle con il vento che gli spingeva i capelli bianchi
davanti alla faccia, oscurandola. «Se potrò parlare con loro, forse
potrò evitare ulteriori spargimenti di sangue. Di certo non vuoi perdere
nessuno dei tuoi uomini.» La testa di Makisomn mi rotolò davanti agli
occhi e trattenni il respiro, serrando le palpebre per un momento mentre
ritrovavo il controllo e mi costringevo a proseguire: «Trecento anni di
guerra, centurione, e in tutto quel tempo non abbiamo \emph{mai} visto i
Cielcin arrendersi. È una cosa che possiamo cambiare.»

