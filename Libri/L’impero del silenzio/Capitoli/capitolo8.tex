\chapter{Gibson}

Se fossi morto in quel vicolo, oltre mille anni fa, le cose sarebbero
state molto diverse. Ci sarebbe ancora un sole nel cielo di Gododdin, ci
sarebbe ancora un Gododdin, i Cielcin non sarebbero costretti a vivere
come nostri schiavi, nelle nostre riserve per alieni. D'altronde, ci
sarebbe ancora in corso una Crociata. So quello che dicono di me, come
mi chiamate sui vostri libri di storia. Il Divoratore di Soli, Il
Semimortale. Lingua di demone, regicida, genocida. Le ho sentite tutte,
e come ho detto, nessuno di noi è una sola cosa. Come nell'enigma che la
sfinge ha sottoposto al povero, condannato Edipo: noi cambiamo.

Se cercate il mio battesimo, guardate a quel momento in cui giacevo
morente su una strada solitaria di Meidua, con la mano fratturata, il
cranio incrinato e la spina dorsale spezzata. Cercate il momento in cui
sono stato abbattuto mentre Crispin ascendeva alla gloria e
all'adulazione della folla. In seguito avrei guardato gli ologrammi,
mentre ero in convalescenza nelle mie stanze al Riposo del Diavolo, e
avrei visto i fiori e le bandiere gettati a Crispin che piovevano sul
suolo del colosseo, la gente che rideva e applaudiva il figlio
coraggioso del suo arconte con il suo stupido mantello.

\begin{figure}
	\centering
	\def\svgwidth{\columnwidth}
	\scalebox{0.2}{\input{divisore.pdf_tex}}
\end{figure}

Mi svegliai con la vista di costellazioni dipinte. Rotaie d'ebano
incastonate nello stucco color crema, con i nomi delle stelle che
scintillavano intarsiati in ottone. Sadal Suud, Helvetios... la
costellazione di Arma, lo scudo. E appena più oltre, \emph{Astranavis},
l'astronave. Era la mia stanza, con la volta curva del soffitto che era
realizzata in omaggio ai cieli di Delos. La mia stanza. Questo era
sbagliato. Ero morto, quindi come potevo essere nella mia stanza? Cercai
di muovermi ma non ci riuscii. Spostarmi produsse soltanto un sordo
dolore nelle mie ossa. Potevo però muovere la testa, e mi guardai in
giro. Una figura accasciata e vestita di verde sedeva al mio capezzale
con la testa china, come in preghiera, o nel sonno. Dietro di essa
c'erano soltanto i familiari scaffali di libri, la postazione di gioco,
la piastra olografica e il dipinto che raffigurava astronavi infrante,
dipinte in bianco su tela nera, un Rudas originale.

«Gibson» cercai di dire, ma la gola arida mi permise soltanto di gemere.
Sgranai gli occhi nel vedere il congegno fissato alla mia mano. Era come
un guanto o la mano di uno scheletro, un assortimento di piastre di
metallo simile ai petali di un'orchidea... o a un qualche strumento di
tortura medievale.

«Gibson...» Questa volta produssi l'imitazione approssimativa di una
parola, simile a quella di un neonato. Aghi sottili come capelli e
flessibili esercitavano pressione dal guanto fissato alla mia mano
rovinata e un congegno simile, ma più aderente, mi imprigionava il
costato, con entrambi i lati che quasi risplendevano per il calore.
Qualcuno mi aveva legato il braccio sano e le gambe al telaio del letto
per proteggere le parti lese. Immaginai quegli aghi flessibili che si
incurvavano nelle mie ossa, diramandosi come radici per generare il
processo di guarigione accelerato.

Il vecchio si riscosse, muovendosi con la lentezza esagerata di una
persona esausta che riprenda conoscenza e con un gemito simile allo
scricchiolare degli alberi. Il bastone gli scivolò, colpendo le
piastrelle con l'impugnatura d'ottone, ma lui lo ignorò e si protese in
avanti con quella che un non scoliasta avrebbe descritto come
eccitazione. «Sei sveglio.»

Cercai di scrollare le spalle, ma questo generò una trazione contro il
congegno infernale che mi intrappolava il petto e l'avambraccio,
costringendomi a reprimere un sussulto. «Sì.»

«Nel nome della Terra, cosa ci facevi in città da solo?» Non sembrava
irato, non lo faceva mai. Il primo addestramento di uno scoliasta era
nel reprimere le emozioni, nell'elevare la ragione stoica al di sopra
dei venti della mera umanità, e tuttavia... e tuttavia c'era
preoccupazione in quegli occhi grigi e velati, e nel modo in cui le sue
labbra sottili come la carta si incurvavano verso il basso negli angoli
rugosi. Per quanto tempo era rimasto accasciato su quella sedia?

Trassi un rantolante respiro, e invece di rispondere borbottai: «Quanto
tempo?»

Una domanda inarticolata, vaga. Finalmente il vecchio leonino si chinò a
recuperare il bastone caduto. «Ormai sono quasi cinque giorni. Eri quasi
morto quando ti hanno riportato qui. Hai passato il primo giorno in
stato di sospensione mentre tor Alma lavorava per ricostruire il tuo
tessuto cerebrale danneggiato.»

«Danneggiato?» Sentii le sopracciglia che mi si contraevano
involontariamente.

Gibson quasi... \emph{quasi}... sorrise. «In ogni caso nessuno noterebbe
la differenza.»

«È una battuta?»

Il vecchio si limitò a fissarmi. «Alma dice che ti riprenderai
completamente, e lei conosce il suo mestiere.»

Agitai la mano sana, strattonando le cinghie. Mi sentivo la testa tanto
leggera da dare la sensazione che un perverso chirurgo l'avesse riempita
di cotone e di alcol disinfettante, e gli occhi mi pulsavano. «Credo che
preferirei essere morto.» Lasciai ricadere la testa sul cuscino con un
grugnito.

Lo sguardo dello scoliasta si spostò lungo la mia faccia, con un
sopracciglio inarcato. «Non dovresti parlare in quel modo.» Spostò lo
sguardo alle mie spalle e fuori dalle strette finestre, guardando il
mare.

«Sai che non dico sul serio.»

Lui sbuffò e chiuse le lunghe dita nodose intorno all'impugnatura del
bastone. «Lo so.» Cercai di nuovo di muovermi e Gibson si protese a
posarmi una mano sulla spalla. «Non ti muovere, giovane signore.» Non lo
ascoltai e cercai di sollevarmi a sedere. Un dolore lancinante mi
divampò dietro gli occhi e ricaddi all'indietro, privo di sensi.

Quando mi svegliai Gibson mi sedeva ancora accanto con gli occhi chiusi,
cantilenando fra sé in tono sommesso. Qualcosa dovette cambiare nel
ritmo del mio respiro perché il vecchio aprì un occhio come i gufi a cui
somigliava. «Ti avevo detto di non muoverti, giusto?»

«Sono rimasto svenuto a lungo?»

«Solo per un paio d'ore. Tua madre sarà lieta di sapere che sei di nuovo
nel mondo dei viventi.»

Sentendomi più coerente di quanto lo fossi stato al mio precedente
risveglio e anche un po' più coraggioso, ribattei: «Davvero?» Poi mi
guardai intorno per la stanza, spostando solo lo sguardo per paura di
ripetere il mio precedente errore. «C'è dell'acqua?»

Lo scoliasta si alzò con cura chirurgica, lasciando il bastone
appoggiato alla sedia vuota, e attraversò barcollando la stanza fino a
una credenza dove un servizio per bevande in argento dispensò una coppa
di acqua fresca. Gibson trovò una cannuccia e tornò al mio capezzale,
porgendomi la coppa. «Dimmi, Gibson, se le importa tanto, dov'è mia
madre?» Bevvi, e l'acqua risultò avere un sapore più buono di quello del
miglior vino di mio padre. Conoscevo già la risposta, ma insistetti:
«Non la vedo.»

Il volto di Gibson si modellò in una sottile maschera che copriva il
dolore. «Lady Liliana è ancora nel palazzo estivo di Haspida.»

Emisi una sorta di sommesso `oh', più un'esalazione di fiato che una
vera parola e simile alla parola cielcin che significava `sì'. Haspida,
con i suoi frutteti e le sue polle limpide. Pensai alle camere che mia
madre aveva là, con le sue serve e le sue ragazze.

«Anch'io vorrei che fosse qui.» Gibson si sfregò gli occhi. In lui si
percepiva una profonda stanchezza, come se fosse rimasto seduto per
giorni. Cinque giorni, dissi a me stesso. Troppo tempo. «Dovrebbe essere
qui, per il tuo bene.»

Contrassi le sopracciglia. Non stava a lui dire quello che mia madre
\emph{avrebbe dovuto} fare, ma si trattava di Gibson, quindi lasciai
correre. «Quanto è grave la mia situazione?»

«Hai la mano destra completamente fratturata, cinque costole rotte, e
hai riportato danni considerevoli a fegato, pancreas e un rene.» Un
senso di disgusto affiorò per un momento sul volto di Gibson mentre lui
lisciava il davanti della sua veste. «Per non parlare del trauma alla
testa. Prenditela con calma, o romperai di nuovo qualcosa.»

Annuii debolmente e mi lasciai sprofondare contro i cuscini, mentre un
occhio cominciava a chiudersi. «Cosa è successo?»

«Non lo ricordi?» Lo scoliasta si accigliò. «Sei stato aggredito da
qualche farabutto nel distretto dei magazzini. Abbiamo esaminato le
registrazioni della sorveglianza e li abbiamo trovati.» Inclinò la testa
verso un lato del tavolo. «Ardian si è messo a capo dei prefetti e hanno
trovato il tuo anello.» Seguii la direzione del suo sguardo e lo vidi
là... il mio anello con sigillo su cui il diavolo dei Marlowe era inciso
con il laser sul castone, era in mezzo a strane strumentazioni mediche,
con accanto il mio terminale.

«Capisco» mormorai, inclinando la testa per bere altra acqua.
Stranamente, quando non ne hai bisogno l'acqua ha lo stesso sapore
dell'aria, non noti mai quel suo gusto glorioso finché non sei assetato.
«Allora sono morti? Tutti e tre?»

Gibson si limitò ad annuire. «Sir Roban ti ha trovato appena in tempo --
lui e quella tua tenente -- e ti hanno riportato indietro.»

«Kyra?» Contro ogni cautela, cercai di nuovo di sedermi e me ne pentii
quando il dolore tornò a farsi eloquente dentro di me.

Gibson si fece silenzioso, e per un momento pensai che fosse svenuto lì
in piedi come un narcolettico, sopraffatto da quella profonda stanchezza
che lo pervadeva. Mentre il dolore sbiadiva e mi riadagiavo al mio
posto, però, mi accorsi che i suoi occhi erano aperti e che mi stavano
osservando.

«Cosa c'è?» chiesi, sussultando nel girarmi per una improvvisa tensione
allo stomaco, dove tor Alma aveva fissato il suo congegno correttivo.

Il vecchio scoliasta trasse un profondo respiro, armeggiando con il
polsino di una manica voluminosa. «Tuo padre ti vuole vedere non appena
starai abbastanza bene.»

«Digli di venire a trovarmi» ribattei d'istinto. Faceva male che nessuno
dei miei genitori fosse lì e neppure Crispin. C'era soltanto Gibson, il
mio tutore. Il mio insegnante. Il mio amico.

Un piccolo sorriso quasi caloroso prese forma sul volto segnato di
Gibson, che mi batté un colpetto sulla spalla con una mano chiazzata
dall'età. «Sai che tuo padre è un uomo molto impegnato, Hadrian.»

«Qualcuno ha cercato di uccidermi!» Nonostante i lacci accennai
all'imbracatura intorno alle costole. «Sarebbe logico pensare che lui si
prendesse il tempo per venire a vedere come sto. Si è fatto vedere anche
solo una volta? E lo ha fatto mia madre?»

«No, lady Liliana non si è mossa.» Gibson trasse un altro respiro. «Ha
lasciato ordine di avvertirla se le tue condizioni fossero peggiorate.
Quanto a tuo padre... ecco...»

Era tutto quello che avevo bisogno di sentire. «È un uomo molto
impegnato.» Nelle mie parole c'era un che di vuoto e di fragile, come un
pannello di vetro di sicurezza crepato da una pallottola, con i pezzi
tenuti insieme solo perché le schegge erano cadute una \emph{contro}
l'altra, pronte a precipitare alla minima sollecitazione.

«Tuo padre... ti chiede di pensare al modo in cui le tue ferite si
potrebbero riflettere sulla dignità del tuo Casato.» Ricorderò sempre
come Gibson evitò di guardarmi nel riferire quelle parole, quasi con la
stessa intensità con cui ricordo quanto mi ferirono.

Devastato, chiusi gli occhi per ricacciare indietro le lacrime che
sentivo arrivare. Un conto era sapere a livello intellettuale di non
avere l'affetto dei propri genitori, ma tutt'altra cosa era avvertirlo
direttamente. «Ti ha detto di riferirmelo.»

Non ebbi risposta e questo confermò la mia supposizione. Nel guardarlo
di nuovo, tornai a essere colpito da quanto Gibson apparisse stanco.
C'erano cerchi scuri sotto i suoi occhi grigi e fra le folte basette si
vedeva la puntinatura della ricrescita della barba. Ricordai a me stesso
che quell'uomo era rimasto seduto su una sedia per quasi cinque giorni,
l'intero tempo che ci avevo messo a riprendermi. Avevo una sorta di
padre, ma il suo volto non sarebbe mai stato appeso sotto la Cupola
delle Incisioni Radiose.

«Dovresti dormire, Gibson.»

«Adesso che so che stai bene.» Con quelle parole lo scoliasta rimosse la
coppa d'acqua, posandola sul comodino insieme ai miei effetti personali
e alle strumentazioni mediche. «Parla con tuo padre non appena puoi.»

«Gibson...» Protesi la mano sana e gli afferrai la manica per il polsino
con dita gonfie e intorpidite, tendendo le cinghie di contenzione. «Lo
darà a Crispin.»

L'anziano scoliasta mi guardò con occhi che si erano fatti piatti come
pietre coperte di muschio. «Cosa darà a Crispin?»

Indifferente alle videocamere, ai microfoni e a qualsiasi altra cosa si
annidasse nella mia camera, scrollai le spalle e agitai il braccio sano,
sussultando. «Tutto.»

Lui batté il bastone sulle piastrelle con fare pensoso. «Non ha ancora
dichiarato un erede.»

«Ma ha fatto una sorta di annuncio, vero? Dopo il Colosso.» Ero certo di
aver ragione e avrei serrato a pugno le mani se una di esse non fosse
stata imprigionata nel guanto.

«Non ha fatto niente del genere.» Gibson batté di nuovo il bastone per
terra. «Fra l'aggressione nei tuoi confronti e il combattimento di tuo
fratello nel Colosso -- che a quanto mi è dato di capire è stato
qualcosa di spaventoso -- non ha avuto il tempo di dire molto. I plebei
sono incredibilmente entusiasti di tuo fratello. Ho sentito dire che
Crispin è stato decisamente... valoroso.»

«Valoroso?» Per poco non scoppiai a ridere, e sentii l'impulso di
sputare sul pavimento. «Per la Terra nera, Crispin è un pazzo, Gibson.
Liberami la mano sinistra, in modo che possa bere da solo, dannazione!»
Lo fece e mi passò la coppa. Le mani cominciavano a ritrovare la
sensibilità e serrai la sinistra intorno alla pesante plastica
trasparente. Il diavolo dei Marlowe appeso alla parete pareva quasi
ridere di me e accentuai la stretta intorno alla coppa al punto da far
scricchiolare la plastica. «Lui deve saperlo. Ha sentito quello che ha
detto mio padre.»

Gibson inclinò la testa da un lato. «Che cosa ha detto?»

Gli riferii ogni cosa. Il mio fallimento con la capofazione della Gilda,
la sessione del consiglio, ogni cosa. Dopo un momento chiusi gli occhi e
lasciai ricadere la testa all'indietro contro il cuscino di piume per
quella che mi parve la centesima volta prima di formulare la domanda che
temevo più di tutte le altre, decidendo che era meglio tirarla fuori che
lasciarla incancrenire dentro di me. «In tal caso, che ne farà di me?»

Per essere una domanda così seria, Gibson rispose con sorprendente
rapidità e controllo, tanto che dovetti ricordare a me stesso che era
uno scoliasta, addestrato a elevare la logica sopra ogni altra cosa.
«Non è stato ancora annunciato. Tuo padre non ha mai dichiarato la
scelta di un erede, quindi se quello che dici è vero non c'è nessuna
effettiva difficoltà legale. E come ho detto, il popolo è decisamente
entusiasta di Crispin. Per il momento, in ogni caso.»

«Vedremo quanto durerà.»

Gibson mi posò sulla spalla una mano leggera come carta. «Hadrian, tuo
padre ha sempre pensato che tu sia troppo gentile, troppo debole per
governare.»

«Si tratta della capofazione della Gilda...» Mi protesi a posare la
coppa sul davanzale vicino al letto e mi girai sul fianco come meglio
potevo mentre Gibson si rimetteva a sedere.

«Non si tratta solo di lei.» Si appoggiò allo schienale della sedia e il
suo sguardo si spostò dal mio volto al panorama del mare che si godeva
dall'alta finestra. «Tuo padre è un carnivoro, Hadrian, un vero
predatore, e crede che tutti i lord debbano essere così.»

A quel punto mi misi a sedere, sussultando per il dolore al fianco e
premendo una mano contro il sigillo medico. «Ci sono persone che muoiono
in quelle miniere, Gibson. Le radiazioni...»

Il mio tutore si comportò come se non mi avesse sentito e continuò a
parlare, senza mai sollevare la voce al di là di un sussurro sommesso
come il frusciare del vento su pietre consumate dagli elementi. «Ritiene
che il dominio debba essere duro.» La sua voce cambiò improvvisamente,
crepitando dell'intensità di un pedagogo. «Hadrian, elencami le Otto
forme dell'Obbedienza.»

Lo feci.
«\phantomsection\label{fileintero-10.xhtml__idTextAnchor000}{}Obbedienza
per paura del dolore. Obbedienza per paura dell'altro. Obbedienza per
amore verso la persona del gerarca. Obbedienza per fedeltà alla carica
del gerarca. Obbedienza per rispetto delle leggi degli uomini e del
cielo. Obbedienza per fede. Obbedienza per compassione. Obbedienza per
devozione.»

«Qual è la migliore?»

Sbattei le palpebre, perché mi ero aspettato una domanda più difficile
di quella. «L'obbedienza per timore del dolore.» Voleva che lo dicessi
solo per farmi sentire il peso di quelle parole.

Gibson sorrise. «La legge dei pesci. Proprio così. Tuo padre comanda in
questo modo e Crispin farà la stessa cosa. È per questo che crede in tuo
fratello e non in te. Lo capisci? Quello che ti fa è un complimento,
anche se non è questa la sua intenzione.»

Non sapendo cosa rispondere lasciai che la coppa vuota mi cadesse dalla
mano e girai la testa con disgusto. Tutta quella faccenda era amara.
«Non è questo il modo di comandare un popolo.»

«Tutto quello che tuo padre vuole è spremere. Guadagnare dalle miniere
quanto basta per comprarsi una baronia presso l'ufficio imperale ed
elevare il vostro lignaggio fra i Casati dei nobili.»

«Ma perché?» mormorai, sentendo più intensamente il dolore al fianco
anche se era ancora un indolenzimento opaco e caldo. «Più terra e servi
che vi scavino. Una maggiore quantità delle stesse cose...»

La voce di Gibson cambiò, risuonando d'un tratto molto lontana. «Un
tempo i signori degli uomini la pensavano tutti come tuo padre,
considerando tutte le risorse come combustibile per il progresso. Questo
li ha distrutti ed è costato la vita alla Terra. In tuo padre questa
insensibilità è giustificabile solo perché ci sono altri mondi su cui si
potrebbe trasferire dopo aver esaurito questo.»

Mentre parlava il mio campo visivo cominciò a scurire lungo i contorni e
riuscii a stento a replicare. «Questa non è una giustificazione.»

Lo scoliasta mi batté un colpetto sulla spalla. «E questa è la
differenza fra voi due.»

Se avevo una risposta, non la proferii mai perché l'oscurità al confine
del mio campo visivo strisciò in avanti, cadendo come sabbia.

\begin{figure}
	\centering
	\def\svgwidth{\columnwidth}
	\scalebox{0.2}{\input{divisore.pdf_tex}}
\end{figure}

Nei miei sogni ero solo e passavo sotto lo stretto arco che portava
dalla necropoli al mausoleo dove le ceneri della mia famiglia erano
sepolte. Quante volte avevo percorso quella strada nei sogni, quando
nella vita reale lo avevo fatto una volta soltanto? Era stato per il
funerale della madre di mio padre, lady Fuchsia, quando ero ancora un
ragazzo. Non l'avevo conosciuta bene, ma il suo era il primo cadavere
che avessi mai visto, il mio primo incontro con la morte, e il suo puzzo
− insieme al ricordo − non mi ha mai abbandonato. Mi tormentava, e
spesso nel trovarmi di nuovo di fronte alla morte ricordavo l'odore
dolciastro della mirra, il fumo delle candele di incenso e il mormorio
dei cantori e della vecchia Eusebia mentre guidava la marcia funebre giù
per i gradini echeggianti della nostra necropoli. Per la mia mente
giovane non si trattava tanto del decesso di mia nonna quanto di una
visita della morte, per cui a ogni decesso che era seguito al suo avevo
ricordato quella processione, quei gradini, quella marcia nel mondo
sotterraneo.

Nel sogno mio padre procedeva per primo, dietro alla priora,
trasportando le ceneri della nonna, mentre noi -- la sua famiglia -- lo
seguivamo con i vasi canopi. Io avevo i suoi occhi, sospesi in un
liquido ceruleo, mia madre aveva il cuore, mio zio Lucian -- morto ormai
da sette anni nello schianto di un velivolo -- trasportava il cervello.
Un sudario di un qualche colore più scuro del nero nascondeva la statua
di mia nonna eretta sul pavimento grezzo, fra le stalattiti, e potevo
sentire l'acqua che gocciolava dalla volta di pietra per cadere dentro
pozze piatte come specchi. Nel sogno, quando strappai il sudario trovai
sotto di esso la statua di mio padre e non quella della nonna. Ed essa
era viva, mi guardava con occhi simili a stelle morenti mentre lasciavo
cadere il canopo con gli occhi, che si infrangeva sul pavimento della
caverna.

Le sue mani di pietra mi afferrarono, sollevandomi di peso dalla roccia
calcarea, poi la caverna si dissolse intorno a noi, trasformandosi in
fumo e oscurità, finché non rimasero che gli occhi rossi dello spettro
di mio padre. Mi ritrassi da essi, cadendo all'indietro e attraverso una
sorta di portale invisibile circondato dalle maschere funebri di
trentuno lord Marlowe, bianche e sbiadite in quell'ombra senza fine. Mi
sentii come qualcuno che nuotava in profondità schiaccianti, gelato e
soffocato, privato dell'orientamento. Una morsa di terrore mi stringeva
nei suoi artigli e mi parve di svegliarmi per ritrovarmi integro e
guarito. Gibson era ancora là, alto ed eretto come non lo era mai stato
nella mia memoria, con la spina dorsale storta ora diritta, i capelli in
ordine, gli occhi taglienti come bisturi.

Accantonai tutto questo, distratto dalle narici tagliate che ora
marchiavano il mio mentore come un criminale. A volte penso che la
memoria mi sia venuta meno, persa da qualche parte lungo i secoli
trascorsi dalla mia giovinezza, e di tanto in tanto mi chiedo se i miei
ricordi successivi non siano tornati indietro ad annebbiare quell'incubo
dell'infanzia. Eppure, anche se dovessero di nuovo minacciarmi di
decapitazione, credo ancora che giurerei che fosse così, e di aver visto
la lesione di Gibson \emph{prima} che gli venisse inflitta.

Diventai vecchio e tornai giovane prima di capire.
