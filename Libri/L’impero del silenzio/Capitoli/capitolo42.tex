\chapter{Parlare come un bambino}

La folla applaudì quando sopravvivemmo. Avevo ancora il respiro
affannoso, con il sangue della creatura che mi aderiva alla faccia e mi
colava negli occhi, color cobalto, e con l'odore del rame. Stringevo
ancora in mano una delle lance, dimenticata, e le altre tre erano
conficcate nella schiena della creatura morente. Non ho mai saputo il
suo nome; una qualche bestia pelagica tirata fuori dal mare della
distante Pacifica. A dire il vero mi ricordava gli Umandh, tutta denti e
tentacoli, e nella morte si era sgonfiata come un pallone, con le
migliaia di piccoli cuori che mantenevano il sangue pressurizzato e il
corpo rigido che lo pompava nell'aria del mattino.

Ghen stava applaudendo e assestando pacche sulla schiena di Switch, e
perfino Siran appariva soddisfatta nel guardare sorridente il cielo.
Sputai, scagliando sulla sabbia una boccata di catarro mista a sangue
alieno. Aveva il sapore del metallo bruciato, di acido e di fumo, e io
avevo quello stesso puzzo, intriso com'ero di quel sangue perché era
stato mio il colpo che aveva finalmente aperto una lacerazione nella
pelle chitinosa della bestia e leso un importante vaso sanguigno.

Il conte era in piedi nel palco, sopra di noi, e applaudiva come aveva
fatto in ciascuna delle rare occasioni in cui era stato presente. Suo
marito, lord Luthor Shin-Mataro, un plutocrate di un'antica famiglia
mandari, era in piedi al suo fianco, un'esile figura in verde e argento.
«Avete combattuto bene, davvero bene!» Il conte si appoggiò al
parapetto, parlando direttamente a noi mirmidoni. Nell'ombra alle sue
spalle un paio di Umandh agitavano una serie di ventagli di carta per
rinfrescare lui e il suo uomo, come pure i loro due figli e l'insieme
dei consulenti e consiglieri che quel giorno erano stati invitati a
condividere il palco reale. Era un gesto inutile, dato che senza dubbio
il palco aveva l'aria condizionata all'interno delle cortine del suo
scudo. Poi il conte si lanciò in un altro di quei discorsi memorizzati
sulla nostra abilità e il nostro coraggio, questa volta senza le
abituali riserve relative alle perdite di vite e ai nobili sacrifici.

Nessuno di noi era morto.

\emph{Nessuno} era morto.

«Siete tutti esempi della vostra arte.» Gettò dal palco un oggetto, una
borsa di cuoio che colpì i mattoni con un tintinnare d'oro. «Un dono.»
Io ero il più vicino, quindi avanzai per prendere la borsa.

In quel momento, prima che parlassi, il mio sguardo incontrò il suo e
lui chinò appena la testa in quel minimo gesto di deferenza che, venendo
da parte di un così grande signore, era indice di elevato rispetto. A
detta di tutti, il conte Mataro era un uomo dalle passioni classiche.
Gli piaceva cacciare -- anche se su Emesh non c'erano foreste -- e
combattere, sebbene nessun suddito gli offrisse un duello in condizioni
di parità. Tuttavia, ogni volta che i suoi rigidi programmi glielo
permettevano, assisteva a un combattimento nel colosseo o a una corsa
nel circo, e se rimaneva colpito elargiva bonus. Gli altri -- o quelli
liberi di lasciare il complesso del colosseo -- avrebbero portato il
denaro a Borosevo e lo avrebbero speso in prostitute, droghe e ogni tipo
di intrattenimento. Per me, era la terza volta che ricevevo una simile
elargizione e l'oro delle precedenti due era ancora nel mio armadietto,
al sicuro con i miei pochi oggetti personali... nella fattispecie
l'anello di famiglia e il nuovo diario che avevo acquistato. Era una
cosa di lusso, di fine carta bianca e cuoio nero, con il fermaglio in
argento. Vedi, mi mancava il disegno...

Questa volta Pallino non era presente per rispondere, quindi piegai un
ginocchio nella polvere con una fitta dolorosa alla coscia ferita e
dissi. «Ringraziamo Vostra eccellenza per la sua generosità. Non siamo
degni di tanto onore.» In realtà era una miseria, meno di una frazione
di quello che un tempo avevo estorto a Lena Balem -- e perso in modo
epico -- e mi seccava umiliarmi in quel modo, come mi seccava ancora di
più il ricordo di quella perdita. Un ginocchio che, come il mio, non era
abituato a genuflettersi non si piegava facilmente, e quelle erano solo
parole senza significato, che ci si aspettava dalla mia condizione
sociale e dal contesto in cui ci trovavamo.

Ma quando si è a Roma...

\begin{figure}
	\centering
	\def\svgwidth{\columnwidth}
	\scalebox{0.2}{\input{divisore.pdf_tex}}
\end{figure}

Anche se in parte era una prigione, il complesso del colosseo di
Borosevo non era come le segrete della bastiglia di Vesperad o il
pianeta-prigione imperiale su Marte. Tratteneva i suoi segreti fra dita
rilassate, smosse dal tempo e dall'imprudenza. Correva voce che quel
prete, il cantore Gilliam Vas -- che a volte era in compagnia di un
cathar bendato e altre no -- fosse stato visto molte volte nei cunicoli
del colosseo dall'arrivo dell'\emph{Incrollabile} e dall'afflusso nei
nostri ranghi di foederati e legionari che avevano abbandonato il
servizio attivo. Si diceva che tenessero qualcosa in isolamento nelle
prigioni sotterranee, fra i folli e gli assassini che morivano nel
Colosso nei modi più spettacolari. Alcuni dicevano che si trattava di un
Glorificato, di uno di quei demoni che infestano il Buio fra le stelle
come nella storia di Kharn Sagara. Sentii svariate descrizioni, secondo
cui aveva due teste, oppure sei braccia, tutto acciaio articolato e ossa
esposte. Altri ancora dicevano che era un lord traditore, un qualche
maniaco che si era rivoltato contro l'umanità e la luce della Madre
Terra a favore dei Cielcin. Dicevano che aveva spezzato pane e carne con
loro e cenato con le ossa di bambini umani.

Hanno detto lo stesso di me.

Molte volte avevo evidenziato la teoria più ovvia, quella che avevamo
sentito da Kogan, e cioè che il conte avesse comprato un Cielcin
catturato dai foederati che erano arrivati su Emesh a bordo
dell'\emph{Incrollabile}, e quella teoria aveva fatto presa nella mente
tanto dei mirmidoni quanto dei gladiatori.

Sudati per un periodo di esercitazione nel cortile, tornammo tutti nella
frescura ammuffita dell'ipogeo del colosseo. Switch era più avanti e
rideva in compagnia di una delle reclute più giovani. Il suo corpo un
tempo esile si era rinforzato nei mesi del nostro servizio vincolato e
aveva l'aspetto di un vero combattente. Siran mi rivolse di sfuggita
qualche commento mentre lei e Ghen -- insieme ad altri cinque di noi --
venivano riportati nelle loro celle nel blocco delle prigioni. La
salutai con un gesto, ricambiando la sua battuta con un mio commento
ironico.

«Ora che si fa?» chiese la recluta amica di Switch. «Si mangia?»

Sorprendendo tanto il ragazzo quanto la maggior parte di noi, Switch
passò un braccio intorno alla vita della recluta. «Prima ti dobbiamo
ripulire!» La recluta gli assestò una gomitata e Switch si piegò su sé
stesso con un gemito, mentre Pallino e gli altri ridevano a sue spese e
io sorridevo.

«Allora voi due andate a lavarvi» disse poi Pallino, porgendo il suo
elmo a uno degli altri con un'occhiata di intesa. Sembrava avere sempre
a portata di mano un finto attendente, una sorta di scudiero. Si assestò
la benda sull'occhio e continuo: «E noi andiamo a mangiare. Ma non
quella schifezza che servono al piano di sopra. Cucinerò io.» Questo
provocò un piccolo applauso, perché nel suo modo plebeo lui era il cuoco
migliore che avessi mai incontrato. La gente rimaneva sempre sorpresa
nello scoprire questo dettaglio sul conto di quel duro e anziano
veterano. Lo guardai allontanarsi sorridendo agli altri, ma non mi mossi
perché ero stato assalito da un'idea.

Qualcuno mi urtò, riscuotendomi, e mi girai. «Cosa c'è?»

Erdro, che era con noi dall'inizio, ripeté la sua domanda. «Vieni?»

«Io...» Distolsi lo sguardo, rivolgendolo lungo il corridoio e in
direzione di Switch e della recluta che si allontanavano. «No. Vai
avanti senza di me.»

Erdro si accigliò. «Se non mangi perderai massa, amico.»

«Sì, sì.» Lo allontanai con un gesto. Non si sbagliava, ma la perdita di
un singolo pasto non mi avrebbe certo devastato.

«Vuole andare a lavarsi con Switch» commentò una delle ragazze, con un
sorriso.

«Ti piacerebbe!» ribattei con un sorriso in tralice. La mia frecciata
colpì troppo vicino al bersaglio perché lei arrossì. «No, vi
raggiungerò. Potrei prendere qualcosa dal nostro ultimo bonus e andare
in città per comprare qualcosa che non sia cresciuto nelle vasche.» Fui
assalito da un'involontaria fitta di nostalgia per il mercato del pesce
e i bazar di Meidua, per il vecchio nipponese e i suoi involtini di
pesce. Mi mancava il sapore della selvaggina proveniente dalle vallate e
dalle foreste del Denterosso. Vero cibo, una cosa di cui {godevano} solo
i ricchi, che se lo potevano permettere, e i poverissimi, che vi erano
così vicini che nessuno poteva portarglielo via.

«D'accordo, Had.» Pallino abbozzò un saluto militare -- il suo modo
cortese di congedarsi -- e portò gli altri via con sé, gridando da sopra
la spalla: «L'esercitazione è per domattina, alle otto.» Ricambiai il
saluto con precisione imperiale come se lui fosse stato un tribuno o un
legato dei Corpi. Non lo vide, ma non ce n'era bisogno. Poi se ne
andarono.

Non avrei potuto chiedere un momento più perfetto di quello. I mirmidoni
detenuti stavano venendo scortati a fare una breve doccia per poi essere
scortati nel loro blocco per il pasto serale. Sapevo che non venivano
chiusi in cella come nelle segrete di un palatino o in una bastiglia
della Cappellania ma semplicemente confinati in un dormitorio chiuso a
chiave.

Quella chiave era in possesso di una delle due guardie di stanza in
fondo al corridoio in cui settimane prima Gilliam Vas aveva ordinato che
venissi stordito dai suoi foederati. Costantemente annoiati, i due
carcerieri oziavano dietro la loro scrivania con un'espressione stanca e
con indosso l'uniforme cachi con lo stemma della sfinge dei Mataro
ricamato sulle maniche. Li oltrepassai, camminando con decisione su per
una stretta rampa laterale e in uno dei corridoi di servizio rivestiti
di pannelli d'acciaio inossidabile che portavano alle cucine. Se avevo
ragione, i vassoi della cena sarebbero arrivati da un momento all'altro,
con i loro impasti di proteine dai diversi sapori e le verdure acquose,
spinti da parecchi inservienti del colosseo, uomini in abiti civili che
avevano pescato la pagliuzza più corta e dovevano quindi fare il lungo e
noioso giro del blocco della prigione.

Mi arruffai i capelli, ne cambiai la scriminatura e per un impulso
improvviso adottai un'imitazione delle spalle curve del cantore.
protendendo un poco la testa in avanti sul collo per camuffare
leggermente la faccia di Had di Teukros. Saresti sorpreso di scoprire in
quale misura un piccolo cambiamento devia l'attenzione anche di quegli
uomini che si considerano astuti. Una volta ho ingannato un auctor
imperiale con poco più di un accento e un paio di lenti colorate per
nascondere i miei occhi viola.

«Tu!» Un plebeo massiccio che portava la tenuta a strisce bianche di uno
chef sporse la testa da una porta laterale, incorniciato dal vapore che
esalava da una pentola che cuoceva a fuoco lento su un elemento
riscaldante, e fissandomi con un'espressione irosa che evidenziava lo
sporgere pronunciato dei denti inferiori. «Ragazzo, vieni qui!»

«Messere?» Aggrottai la fronte, assumendo un accento durantino mentre mi
venivano in mente la faccia e i toni di Chand.

Nel complesso, quel travestimento non avrebbe ingannato un auctor
imperiale -- e forse nemmeno un bambino scaltro -- ma quell'uomo era
meno brillante della maggior parte dei bambini, quindi imprecò: «Dannati
leccamerda stranieri. Tu! Sì, tu! Sei sordo ragazzo? Vieni qui.» Indicò
con un cucchiaio, gesticolando violentemente. Momentaneamente impegnato
nel mio piano, entrai con passo pesante e lo aiutai a spostare la
pentola fumante da un supporto su uno dei carrelli che ero venuto a
cercare. Alle sue spalle, più addentro nelle cucine, una squadra di
cuochi lavorava agli ordini di un uomo grosso e brutto per finire di
preparare il pasto. Ero in anticipo, quindi passai i minuti successivi
accettando in silenzio gli ordini dell'uomo mentre mantenevo come meglio
potevo la mia posa ingobbita.

Dopo circa dieci minuti una donna dalla faccia rotonda fece capolino dal
corridoio posteriore. «Le proteine sono scongelate, Stromos.»

«Qui non siamo pronti» quasi ringhiò lo chef mostruoso con i denti
sporgenti.

«Sono prigionieri, non stai cucinando per il conte.»

«È un vero peccato» brontolò l'uomo. Di lì a poco, però, il cibo --
pronto o meno che fosse -- venne travasato in padelle riscaldate e
spinto nel corridoio. «Nessuno apprezza più il cibo» borbottò fra sé
Stromos. Provai un impeto di comprensione nei suoi confronti.

«Lo porto io» dissi, afferrando la manica di un assistente che stava per
lasciare la cucina con un carrello che conteneva spaghetti di qualche
tipo in una salsa marrone.

L'uomo mi adocchiò con aria confusa. «Davvero?»

Sorrisi. «Sì. Tu sembri sfinito e io ho una ragazza là dentro. Sai cosa
intendo.»

«Una con i tagli?» Si toccò il naso, indicando la mutilazione. «Perché?»
La sua domanda parve esprimere sincera confusione, come se non riuscisse
a immaginare che qualcuno volesse uno dei detenuti, quale che fosse la
ragione.

«Guardami.» Scrollai le spalle storte, scivolando nel personaggio del
gobbo plebeo, e sorrisi. «E in realtà non sono obbligato a guardarla in
faccia, mi capisci?» Sfoggiai un sorriso che era più che altro un
ringhio e l'uomo rise, assestandomi una pacca su una spalla.

Una parte di me danzò per la soddisfazione quando riuscii a superare la
soglia seguendo la fila di attendenti con il cibo lungo uno stretto
corridoio rischiarato da rosoni in alto sulla sinistra. Pannelli a
specchio nel soffitto nascondevano videocamere e apparecchi di
registrazione, ma che ci fosse o meno qualcuno che stesse guardando era
una questione del tutto diversa. La parte danzante di me si girò,
rallentando l'andatura mentre contemplavo il passo successivo. Avevo
intuito da mesi che questo avrebbe funzionato. Volevo solo vederlo,
sapere se era davvero là. Se la storia di Kogan era vera -- ed ero
pronto a scommettere a favore dell'ex mercenario -- volevo solo dare
un'occhiata.

Al Riposo del Diavolo Gibson mi aveva portato in una delle aule della
vecchia biblioteca e aveva richiamato un ologramma dalla memoria
dell'archivio. Il Cielcin era apparso dal nulla, materializzandosi nel
centro di quello spazio vuoto e scuro, scintillante in vesti di tessuto
setoso nero e zaffiro e bianco, con ogni pollice coperto di glifi
circolari che si sovrapponevano, si intrecciavano, si collegavano. Se
non li hai mai visti, i Cielcin hanno una cresta che si leva come una
corona, con una forte inclinazione sopra e indietro rispetto alla
fronte, e che è dello stesso candore della loro pelle chiara come il
latte. La cresta termina appena oltre il punto in cui ci sono i fori per
le orecchie e la metà posteriore del cranio è coperta da fitti capelli
bianchi.

`Hanno sei dita!' ricordo di aver esclamato, protendendomi a trafiggere
l'ologramma con un dito. `Perché somigliano a noi?'

Gibson si era accigliato e mi aveva elargito una lunga, attenta
occhiata. `Cosa ti induce a dirlo?'

L'evoluzione, o un qualche potere ancora più strano li aveva modellati
in modo simile a noi. Se si guarda al di là delle differenze estetiche,
della corona di spine postoccipitale, delle zanne e dei grandi occhi
freddi, lo si riesce quasi a vedere. Si nota un'affinità con noi nelle
linee delle braccia e delle dita, nella faccia e nella pianificazione
generale del corpo, come pure nei capelli che crescono sulla testa di
entrambe le razze. Per la mia giovane mente erano molto più umani degli
strani e subpensanti Umandh, e in qualche modo quella vicinanza, quella
somiglianza me li rendeva ancora più alieni, più esotici e interessanti
perché c'era già una strada per riuscire a comprenderli. Non capivo gli
Umandh, era come se fossero coralli e alberi, mentre desideravo
comprendere i Cielcin.

E ho pagato per quell'errore.

Deponemmo i vassoi in uno spazio comune dal soffitto basso, ripiegammo i
lati dei carrelli in acciaio inossidabile e bloccammo le ruote,
trasformandoli in tavoli... possibile che questi primitivi di un mondo
di confine non potessero ricorrere a campi di soppressione fluttuanti?
Ultimati i preparativi ci ritirammo su un lato della stanza su ordine
delle guardie, che sostenevano essere presenti per la nostra protezione,
e questa fu una fortuna perché l'ultima cosa di cui avevo bisogno era
che Ghen dicesse qualcosa. Dopo un momento finsi di aver bisogno di
usare il bagno, poi finsi di non capire le obiezioni dei douleter,
gesticolando e gridando in durantino mentre recitavo frasi fatte nella
lingua gutturale di quella repubblica. «Dov'è la biblioteca? Salve, mi
chiamo... sì, sì, dov'è la biblioteca?» E così via. Quei bastardi non
capirono una sola parola, ed è inutile dire che mi lasciarono andare.
Per un momento temetti che la mia guardia mi seguisse, ma quando la
spinsi senza troppa gentilezza contro lo stipite della porta barcollò
all'indietro serrandosi la fronte ammaccata e sussultando.
«\emph{Izvinit}» dissi in durantino. «Tu sei... bene? \emph{Straf}?»

L'uomo imprecò e mi allontanò con una spinta. «Bene? Merda...» Mi mossi
per aiutarlo. «No! Vai!» sibilò fra i denti. «Oltre l'angolo, alla tua
destra.»

Svoltato quell'angolo mi raddrizzai e ruotai le spalle, mentre ritraevo
la mascella nella posizione abituale, abbandonando il travestimento
fatto di postura e accento. Con passo ora fluido percorsi in fretta il
corridoio, affidandomi alla deduzione per quanto riguardava la
planimetria del posto. Superai spedito i bagni singoli destinati
chiaramente alle guardie e scesi una scala a spirale. Era inevitabile
che le videocamere mi riprendessero ed ero certo che ci fossero delle
guardie a sorvegliare il Cielcin o qualsiasi cosa Gilliam Vas e i
foederati del Cavallo Bianco tenessero là sotto, ma pensavo che avrei
sempre potuto sostenere di essermi perso. Cosa potevano fare?
Rinchiudermi nel colosseo? \emph{Ci sono già, ragazzi}. E se poi fosse
successo il peggio e le cose si fossero fatte disperate, mi sarei messo
l'anello e avrei affrontato qualsiasi cosa fosse successa.

Come ho detto, i prigionieri-mirmidoni non erano confinati nelle
segrete, messi alla gogna o tenuti in fredde celle, ma avevano il
permesso di condividere lo spazio comune e i dormitori dietro di esso.
Di primo acchito il non privare i detenuti del contatto umano poteva
essere visto come un atto umano da parte dei carcerieri, ma come tutte
le cose del nostro Impero era in realtà una lama a doppio taglio, perché
questa apertura esponeva i detenuti ad aggressioni, stupri, molestie e a
tutte le crudeltà e le privazioni che una mente umana poteva concepire.
Tuttavia, le prigioni del colosseo di Borosevo contenevano anche una
crudeltà più calcolata, gelida e profonda.

Ricorda che Borosevo era una città costruita su un atollo di corallo, su
banchi di sabbia e sulla laguna nella quale si erano raccolte quelle
misere masse di terra. Avendo vissuto tanto a lungo nella fortezza di
cemento del colosseo era facile dimenticarselo, ma il mare non era mai
molto lontano. Seguendo un'intuizione scesi un'altra rampa di scale,
consapevole che ormai dovevo essere ben al di sotto del livello
dell'acqua, e seguii un corridoio ad arco oltre camere vuote che erano
separate da esso, non da porte sigillate ma dalle classiche sbarre
d'acciaio. Il puzzo di fogna aleggiava intenso nell'aria come in piena
estate, cotto dalla calura dell'ambiente, e si sentiva anche
qualcos'altro... salsedine? Acqua di mare, era acqua di mare. Qualcosa
di umido e di morbido colpì la pietra nella cella alla mia destra con
uno schizzo. Mi fermai ad ascoltare. In alto e in lontananza mi parve si
sentire il trascinarsi di piedi umani: la folla serale che si agitava
nell'aspettare la battaglia navale prevista per quella notte. Io non
avevo niente a che fare con quello sport -- sia lodato l'imperatore --
perché era solo per i gladiatori. L'odore di fogna si intensificò, e
guardai con crescente orrore i canali fangosi scavati nel lato di
ciascuna cella.

E capii.

C'erano dei fori nel soffitto delle celle, condotti che senza dubbio
portavano ai bagni pubblici appena all'interno dei vomitoria del
colosseo. Qui sui prigionieri piovevano letteralmente gli escrementi dei
plebei e dei servi che frequentavano i giochi del Colosso. Mi accigliai,
distratto da un profondo muggire che proveniva da oltre un angolo
lontano ed era seguito dal rauco sussurro di un uomo. Serrai i denti e
proseguii, cercando di non sentire il puzzo di salsedine e di feci che
marcivano.

Serrandomi il naso arrivai in un punto dove il percorso si divideva sui
lati ad angolo retto rispetto a dove ero entrato. Per fortuna l'odore di
salsedine si accentuò, riducendo il puzzo di fogna a livelli gestibili.
Le celle erano tutte vuote... c'era stata qualche grande azione di
sfoltimento nell'arena di recente? Non riuscivo a ricordarlo. Avevo la
sensazione che ci sarebbero dovute essere più guardie, ma senza altri
prigionieri se non quello di cui si parlava, che bisogno c'era?

Il suono opaco di voci umane proveniva dalla mia sinistra, soffocato dal
rumore del mare che sembrava fuori posto. Decisi che quella era una cosa
sbagliata, che non avrebbe dovuto essere udibile a quella profondità nel
sottosuolo.

«Mi sta guardando di nuovo» sibilò una voce brusca. Poi a voce più alta
aggiunse: «Che cosa ho detto riguardo al fissarmi?» Su per il corridoio
risuonò un rumore di metallo contro metallo, punteggiato da sommessi
sciacquii e dallo strano muggito alieno che avevo sentito in precedenza.

Poi una voce più profonda replicò con parole simili a vetro rotto.
«\emph{Yukajji}! \emph{Safigga o-koun ti-halamna}. \emph{Jutsodo de tuka
	susu} \emph{janakayu}!»

Mi immobilizzai a metà di un passo. Pronunciate da un madrelingua,
quelle parole suonavano del tutto diverse da come avrebbero fatto se a
pronunciarle fossimo stati il vecchio Gibson o io. Erano più aspre, più
fluide, più dure, affilate come un rasoio. Entrambe le guardie si
ritrassero dalle sbarre, comparendo oltre la curva del corridoio «Per la
Terra e l'imperatore!» imprecò una delle due, poi tornò a lanciarsi
verso le sbarre con il bastone elettrico. «Quel prete non paga
abbastanza per questo schifo.»

Un lungo braccio, bianco in modo sorprendente, si sporse fra le sbarre e
afferrò la guardia per il polso. Avrebbe potuto essere una mano umana
scolpita da qualche intelligenza che disponesse solo di vaghe
informazioni sul soggetto originale perché era molto più grande, tanto
per cominciare, con sei dita troppo lunghe e con troppe articolazioni.
La guardia lanciò un grido mentre il compagno si lanciava in avanti per
aiutarla. «\emph{Yusu janakayu icehico}.»

«Lascialo andare, demonio!» La voce dell'altra guardia si incrinò mentre
calava il bastone elettrico sul braccio esposto. Il Cielcin ululò,
lasciò andare il carceriere e ritrasse nella cella quel braccio troppo
lungo ringhiando una serie di imprecazioni aliene.

«\emph{Iukatta}!» gridai. `Smettila!' Quello era il tono di comando che
avevo imparato a usare, la voce di lord Alistair Marlowe, risonante,
risoluta, limpida e dura come il ferro. Così il Cielcin avrebbe ottenuto
solo di farsi del male.

Entrambe le guardie si girarono con il bastone sollevato. «Tu chi
diavolo sei?»

Le ignorai perché volevo vedere. Esse non opposero resistenza quando
venni avanti perché qualcosa di imperiale nel mio portamento le ridusse
momentaneamente in silenzio. E lui era là, acquattato nella fanghiglia
in fondo alla stretta cella, stringendosi il braccio intorpidito dalla
scarica con i denti trasparenti simili a vetro messi a nudo in un
ringhio. Avrei voluto ridere, piangere, fuggire. Avevo appena ottenuto
quello che volevo -- vedere un Cielcin -- e più di ogni altra cosa
sentivo il bisogno di andarmene, di essere in qualsiasi altro posto ma
non lì.

Era stranamente più piccolo di quanto mi fossi aspettato, la scultura
stilizzata di un uomo, con braccia e gambe che erano un fagotto di
ramoscelli o di ossa, ma quell'essere minuto era un'illusione data dalla
sua postura, e mentre lo guardavo esso distese arti troppo lunghi per il
torso piccolo da uccello. Qualcuno gli aveva segato le corna dalla
fronte e dall'alta corona, e aveva carteggiato i monconi fino alla
carne. Mi guardava con occhi grandi come mandarini e neri come il
sudario di mia nonna. In essi scorsi qualcosa, ma non era un sentimento
umano. Percepii solo freddo.

Qualsiasi incantesimo di comando avessi operato sulle due guardie, esso
ben presto evaporò e il più vicino dei due uomini mi posò una mano sulla
spalla. «Chi diavolo sei? Nessuno viene qui senza l'espresso permesso
del conte.»

«Dillo a Gilliam Vas» replicai, attingendo quel nome dalla memoria. Per
quanto fosse una menzogna, ebbe l'effetto desiderato e i due si
ritrassero, intimoriti dalla semplice menzione del prete gobbo.
Scrollandomi di dosso le loro mani avanzai fino a essere a portata degli
artigli della creatura, e usando la sua lingua chiesi: «Sei un soldato?
Ti hanno preso in battaglia?»

«Preso?» ripeté la creatura, poi dilatò le quattro narici fessurate che
si trovavano su quel volto segnato laddove ci sarebbe dovuto essere il
naso. «\emph{Nietolo ti-coie luda}.» `Parli come un bambino.' Sorrisi,
un'espressione che per la creatura non aveva significato. Aveva ragione,
ma ero grato di riuscire a trovare le parole, anche se con difficoltà,
lieto che al contrario degli Umandh quel Cielcin parlasse una lingua che
ero in grado di capire.

Accovacciandomi sui talloni repressi un brivido di eccitazione. Stavo
parlando con un Cielcin, un vero Cielcin, non con Gibson o con i
computer subintelligenti della biblioteca del Riposo del Diavolo. «Non
ho mai parlato con un membro del popolo prima d'ora.» Quando la creatura
nella cella non rispose e cambiò solo posizione, strisciando una gamba
pallida sul pavimento di pietra cosparso di sporcizia, insistetti:
«\emph{Tuka namshun ba-okun ne}?\emph{» `}Qual è il tuo nome?'

Rimase lì seduto a guardarmi per quella che mi sembrò metà della vita di
un sole. La sua faccia era così simile a una faccia umana... a un
teschio, con quegli occhi enormi. Sembrava una statua lasciata per
generazioni sotto la pioggia, con il naso e le orecchie erosi... o lo
avrebbe fatto se non fosse stato per quella cresta ossea, che evocava un
ricordo radicato nelle cellule di creature simili a sauri che muggissero
in una qualche giungla fuori dal tempo geologico. Però era stata la
crudeltà a carteggiare quella cresta fino a cancellarla, non il tempo.
«Makisomn.»

«Makisomn» ripetei, faticando a pronunciare il digramma nasale e sapendo
di essere del tutto incapace di rendere il suono vibrante come faceva
lui perché mi mancava il controllo dei condotti nasali da cui la sua
specie dipendeva per produrre quel suono difficile. Mi premetti una mano
sul petto e mi presentai. «\emph{Raka namshum ba-koun Hadrian}.» `Il mio
nome è Hadrian.'

Come io non ero riuscito a pronunciare il suo nome, esso fece lo stesso
con il mio. L'antropologo che non sono mai stato avrebbe sorriso, e in
effetti sulle labbra mi affiorò un accenno di sorriso. «Parli la sua
lingua?» chiese una delle guardie, rovinando quel momento.

Mi girai e sollevai lo sguardo sulla sua faccia piatta e ottusa. In
quegli occhi scuri, opachi, freddi e sconcertati c'era una luce che
compresi essere di \emph{paura}. \emph{Quell'uomo aveva paura di me.} «È
ovvio» risposi a denti stretti. Sapevo che non potevo tirare la cosa per
le lunghe, che presto le guardie che sorvegliavano gli altri inservienti
di cucina sarebbero venute a cercarmi e mi avrebbero trascinato sulla
strada o in una cella. Ed era stato così facile... troppo facile. Bene,
adesso ci ero dentro fino al collo e la curiosità aveva avuto la meglio
su di me. Non sono mai riuscito a resisterle.

«Perché è qui?» chiesi, accennando alla creatura.

«Credevo ti avesse mandato Gilliam.» La seconda guardia mi fissò
socchiudendo gli occhi un po' più intelligenti. «Non lo sai?»

Mentre prima avevo recitato la parte di un servo durantino, adesso
imitai mio padre, ergendomi in tutta la mia statura, pur consapevole che
i miei vestiti sudati e che il mio aspetto non si addicevano davvero a
quella parte. «E sapete cosa vi succederà quando Gilliam Vas e sua madre
scopriranno quanto è stato facile per me arrivare dritto a questa
prigione e a questa cella senza essere fermato neppure una volta?
Rispondete alle mie domande, o lo farete con uno dei cathar.» Ce l'ho
fatta, pensai. Una vera minaccia. Instilla in loro la paura di Dio,
Marlowe.

La seconda guardia -- chiamiamola il Lento -- balbettò una risposta. «È
un dono, messere, per l'Efebeia del figlio del conte. La bestia verrà
sacrificata in trionfo nel Colosso.»

Una sfumatura di disprezzo mi affiorò sulle labbra e volsi loro le
spalle. Se non altro, questo spiegava perché la bestia fosse lì e non
nelle segrete del palazzo. Se non altro, qui era vicina al teatro
dell'azione, al posto dove sarebbe morta. Mi accoccolai, adocchiando la
creatura attraverso le sbarre di ferro storte. In alto alle sue spalle,
la fessura di una conduttura aperta agli elementi faceva colare l'acqua
di mare lungo il muro di fondo della cella, rivestendo i mattoni di
salsedine. «Esso» dissi.

«Cosa?» chiese l'altra guardia -- chiamiamola Ancora più Lento.

«Avete definito il Cielcin un `lui',» spiegai a Lento, senza guardarlo
«ma i Cielcin sono ermafroditi. È un `esso'.» Se pure importava loro di
quella correzione, le guardie non dissero niente mentre io mi rivolgevo
allo xenobita nella sua lingua. «\emph{Ole detu ti-okarin ti-saem gi
	ne}?» `Sai perché sei qui?'

La creatura snudò le zanne simili a vetro in un ringhio, esponendo
gengive fra il blu e il nero. «\emph{Iagamam ji biqari o-koarin.}»

Scossi il capo. «Ucciderti? Non io, no. Ma qualcuno lo farà.»

«\emph{Begu ne}?» chiese. `Come?' Era paura quella che sentivo nella sua
voce?

In tutte le storie che sono state raccontate sul mio conto -- in tutte
quelle che ho sentito e perfino in alcune che io stesso ho messo in giro
-- nessuno ha mai riprodotto questa scena nel modo giusto. Ho sentito
dire che avrei ucciso la bestia nel Colosso, dove tutti gli Emeshi
potessero vedermi, o che non è successo affatto a Borosevo e che il mio
primo incontro con i Cielcin è stato a Calagah, nel Sud, con l'ichakta
Uvanari, fra le rovine. Le opere e gli ologrammi cantano le mie lodi in
battaglia o mi maledicono in quanto stregone, un magio che riversava
veleno nell'orecchio dell'imperatore. Nessuno immagina -- o crede -- che
il nostro primo incontro sia stato fra gli escrementi nelle celle
sotterranee di una segreta del colosseo. La più misera e provinciale
delle circostanze, assolutamente priva di pomposità.

«Come?» chiese di nuovo.

«\emph{Sim ca}» replicai, scegliendo la sincerità al posto di
un'illusione confortante.

`Non bene.'

Non sentii mai la risposta di Makisomn perché qualcuno -- Lento o Ancora
più Lento, non ho mai saputo chi dei due -- mi piantò il bastone
elettrico fra le scapole e il mondo si fece nero.

