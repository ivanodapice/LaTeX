\chapter{Veri uomini}

Non dormii. Non ci riuscivo. Invece sudai, il che non era una rarità su
quel mondo dal caldo soffocante, solo che quel sudore scorreva freddo
come il mio sangue. Lasciai gli altri mirmidoni ai loro sogni agitati e
girovagai per il corridoio dove i portacandele tremolavano nelle opache
pareti di cemento, solide e opprimenti. Non ci badai nel procedere lungo
il corridoio, ma il mio letto non era l'unico a essere vuoto. Non ero
solo nella mia insonnia.

Di notte il mondo era diverso, e l'ipogeo del colosseo lo era ancora di
più. Di giorno ribolliva di attività, delle grida degli uomini e dei
versi di bestie e mostri. Pensai che gli spettri erano solo gli echi
notturni di ciò che ci aspettavamo di trovare di giorno, che venivano a
tormentare la nostra coscienza.

Il colosseo era stato costruito a una certa altezza al di sopra del
mare. Nella maggior parte dei colossei, i dormitori, i canili e le
segrete dell'ipogeo erano letteralmente nel sottosuolo, ma Borosevo
aveva stranezze particolari, costruita com'era su un atollo paludoso.
Comunque, le pareti di pietra gocciolavano e qua e là si vedevano rivoli
di condensa che si formavano sulle tubature di metallo dei sistemi di
controllo climatico sottoposti a uno sforzo eccessivo e sui vetri delle
finestre. La volta del soffitto era tanto bassa che potevo far scorrere
le dita callose sulla pietra liscia, e lo feci. Camminai a lungo, con il
cuore in gola come mai prima di allora. Mi sentivo come si sente un
prigioniero alla vigilia della sua esecuzione, mentre prega che il
priore o il suo signore lo perdonino; una sensazione che conosco fin
troppo bene.

Mi pareva che il corpo di Cat, consumato dalla peste, giacesse sempre ai
miei piedi o appena dietro le mie spalle, e mi ritrovai ad abbassare di
continuo lo sguardo. La morte non sembrava reale, nulla lo sembrava, né
l'ipogeo del colosseo né la città al suo esterno, e neppure gli anni
orribili trascorsi da quando mi ero svegliato nel caos e nella paura. Se
mai vi siete svegliati nel cuore della notte, mettendo in discussione il
mondo fino allo spazio racchiuso fra i suoi atomi, allora sapete come mi
sentivo. Nel mio stato di paura e nell'angoscia del mio cuore, perfino
la carne delle mie mani mi appariva aliena. Mi sorpresi a pensare al
combattimento dell'indomani -- il primo per me -- ma non riuscii a
rimuginarci sopra e ogni volta mi rifugiai in qualche altro ricordo. Le
opere di mia madre, le storie su Simeon il Rosso e Kharn Sagara. Le
lezioni di Gibson, i duelli di addestramento con Crispin. Il sorriso di
Cat e il nostro periodo in quel caseggiato abbandonato. Ricordai il
dolore delle costole rotte la notte in cui gli sgherri di Rells mi
avevano trascinato fuori dal mio riparo di cartone nelle strade di
Borosevo.

Mi fermai davanti all'ingresso delle docce, ascoltando. Si sentiva un
leggero rumore di acqua che scorreva, che ricopriva quello di qualcosa
che lottava e grufolava -- un suono animalesco quasi troppo sommesso per
essere udibile. Mi immobilizzai e inclinai la testa da un lato. La porta
era aperta e il battente si spostò silenziosamente verso l'interno,
riversando un cuneo di aspra luce bianca sul muro opposto. Scalzo
com'ero non produssi quasi alcun suono nell'avanzare furtivo nella
grande camera dei bagni. Le cabine delle docce erano disposte lungo la
parete opposta, ciascuna chiusa da un'unta tenda bianca. L'ultima della
fila era in funzione e vomitava vapore nell'aria silenziosa, senza
mascherare del tutto i suoni animaleschi che avevo sentito dal
corridoio. Non c'erano indumenti sulla singola panca di metallo, né
altri segni che il posto fosse occupato da qualcuno che non fosse uno
dei miei fantasmi.

Una volta dentro, però, la natura di quel rumore risultò più chiara.

Era pianto.

«Salve?» Decisi che era meglio annunciare la mia presenza, perché di
colpo avevo la sensazione di ficcare il naso in qualcosa di molto
privato. Non so dire cosa mi indusse a farlo o perché non mi limitai ad
andarmene. Forse fu la mia curiosità innata, forse ero semplicemente un
ficcanaso o forse... forse mi sentivo solo e molto, molto spaventato.

L'occupante della doccia sussultò e sentii un tonfo sordo seguito da
un'imprecazione e dal tirare su con il naso. «Cosa c'è?» Un altro
momento di versi, poi: «Sei tu, Had?»

Naturalmente era Switch. Chiusi la porta che dava sul corridoio. Ghen
era rinchiuso al sicuro nel blocco delle prigioni ai livelli inferiori
insieme a Siran e agli altri criminali, ma mi spaventava l'idea che
qualcuno come lui potesse interromperci. Non quella notte, non prima di
un combattimento. «Switch?» risposi, con voce compressa come fiori
secchi. «Sì, sono io.»

Il giovane si schiarì la gola. «Io... non riuscivo a dormire.»

Mi sedetti sulla bassa panca d'acciaio posta fra la fila delle docce e
quella degli armadietti e annuii, senza pensare che il ragazzo non mi
poteva vedere. Dopo un momento di silenzio, dissi: «Lo so. Anch'io non
ho mai fatto questo. Intendo che non ho mai combattuto nel Colosso. Ne
ho avuto la possibilità una volta, molto tempo fa, ma...» Le parole mi
si bloccarono dentro e abbassai lo sguardo sulle mani. Sentii Switch
trarre un respiro e compresi di aver commesso un errore. Quel ragazzo
stava appena cominciando a credere in me e adesso stavo minando quella
fiducia.

«Morirò, Had.» Pronunciò quelle parole con un'assenza di emozione che mi
sconvolse. «Perché l'ho fatto? Perché sono qui?» Emise un verso
soffocato, e stavo per replicare, per mostrarmi dispiaciuto, quando lui
continuò: «Forse avrei dovuto rinnovare il mio contratto con mastro Set,
dopotutto. Era meglio della morte. Ghen ha ragione... non sono un
combattente, sono solo un prostituto.»

Con la testa fra le mani, sollevai lo sguardo fissando con ira l'amorfa
plastica bianca della tenda della doccia. «Ghen è un idiota, e questo è
esattamente quello che vuole che tu pensi.»

«È la sola cosa che so fare!» Sembrava quasi pieno di sfida in quel suo
disprezzo di sé.

«Ecco, sei una frana con la spada.» Cercai di sorridere, percependo che
una battuta scadente era meglio della compassione. Quando il giovane
mirmidone non rispose immediatamente mi protesi a calare una manata
sullo stipite della sua cabina. «Non morirà nessuno, amico. E sei
migliorato molto da quando abbiamo cominciato.»

Switch rimase in silenzio per un lungo momento. «Sarei dovuto rimanere.
Mastro Set non si era ancora stancato di me. Avrei potuto fare un altro
viaggio, resistere per avere una paga maggiore. Ho pensato che questo
sarebbe stato meglio, ma...» La sua convinzione si afflosciò. «Ma che
almeno non sarei morto là fuori.»

«Mmm.» Feci una smorfia, lieto che lui non mi potesse vedere. Non poteva
avere più di diciotto anni standard. Per quanto era stato alle
dipendenze di quel Set? Per un anno? Due? Cinque? Era un lavoro onesto e
legale, il che era più di quanto si potesse dire degli ultimi anni della
mia vita, ma il pensiero di quello che era stato mi offendeva. Venduto
con un vincolo contrattuale dai suoi stessi genitori quando era solo un
bambino... nessun bambino avrebbe dovuto vivere in quel modo. Di nuovo,
non gli offrii compassione perché non pensavo che l'avrebbe accettata.
«Allora... come sei finito in questa situazione?»

«Nell'arena?» chiese Switch. Lo sentii muoversi nella cabina della
doccia, appena fuori dalla mia vista. «Pensavo di cambiare, solo che
nessuna delle altre navi mi ha assunto. Non so manovrare o occuparmi di
idroponica o altre cose del genere. So solo...» Immaginai Ghen fare un
gesto osceno per riempire quel silenzio. «Ho pensato che si trattava di
questo o di tornare da mastro Set, e con lui ho chiuso.» Sputò
rumorosamente e nelle sue parole affiorò un po' di fuoco mentre
aggiungeva: «Quello sporco vecchiaccio. Allora questa mi è sembrata
un'idea migliore. Ho pensato che avrei imparato a combattere come...» Si
interruppe, imbarazzato.

«Come cosa?»

«Non lo posso dire.» Dalla cabina giunse una serie di tonfi sommessi e
intuii che stava sbattendo la testa contro la parete. «Ti metteresti a
ridere.»

Abbozzai un invisibile sorriso. «Mettimi alla prova.»

Parve che le parole gli venissero strizzate fuori dal petto. «Volevo
combattere come Kasia Soulier, sai? Hai mai visto quei film? O forse
come il principe Cyrus. Volevo essere un uomo, capisci? Un vero uomo.
Qualcuno capace di farsi valere.»

A quel punto risi, e mi pizzicai l'arco del naso. Potevo sentire il
silenzio imbarazzato che scaturiva ribollente dal ragazzo. «So
esattamente cosa vuoi dire» replicai. «Io volevo essere Simeon il
Rosso.»

«Lui non è un combattente.»

«No» convenni, ripensando a quel giorno lungo il canale, quando avevo
raccontato a Cat la sua storia. «Ma ha dovuto diventarlo quando è giunto
il momento. È questo che sto dicendo. Non importa cosa \emph{sei},
Switch. Devi farti valere quando arriva il momento, e quel momento sta
per arrivare.» Gli parlai un poco di mia madre, del suo narrare storie,
della sua arte, e per un momento fu come se tutto il tormento e la
sofferenza degli ultimi anni fossero finiti dietro una nuvola e io fossi
illuminato dalla luce rosea della mia infanzia. «Non so se esiste un
vero uomo, Switch. Mio padre voleva che diventassi un prete, ma come ti
ho detto... io ho sempre voluto essere come Simeon.» Sorrisi. «Volevo
vedere l'universo.»

Di regola, sarebbe stato il suo turno di ridere di me, ma rimase a lungo
in silenzio. «Suppongo che siamo entrambi nel posto sbagliato» disse
infine, con una debole traccia di umorismo.

Sbuffai. «Lo penso anch'io, ma un uomo si deve guadagnare da vivere, e
qui non si guadagna male, se riesci a incassare.»

«Se si sopravvive» mi corresse. «In realtà veniamo pagati solo alla
fine.»

«Niente discorsi del genere» dissi, forse troppo bruscamente. «Domani a
quest'ora ne rideremo.» Mi interruppi, lanciando un'occhiata
all'orologio sopra la porta che dava nel corridoio. Rimanevano solo due
turni di guardia della notte, cinque misere ore. Così tante e troppo
poche.

«No, non lo faremo.» Dalla cabina giunse un piccolo verso soffocato, in
parte una risata e in parte un singhiozzo. «Non c'è speranza.»

«Non è vero» scattai di rimando, fissando con occhi roventi la tenda
come se avessi potuto aprirvi un buco con il mio sguardo. «Non ti
preoccupare della speranza. La speranza è una nuvola.» Era uno dei molti
aforismi di bilanciamento che Gibson usava per mantenere la sua apatia
di scoliasta. Sembrava strano dire di nuovo cose del genere, strano ma
giusto. Guardando la bassa stanza di cemento avvertii di colpo la fitta
dolorosa causata dalla perdita del vecchio. Cosa non avrei dato per
rivederlo, per parlargli. Ma anche questo non rientrava nell'apatia,
quindi cercai di scacciare quel desiderio che però rifiutò di andarsene.
«Farai quello che devi. Lo faremo tutti. La speranza non c'entra.»

«Ma... e se non ce la facessimo?»

«E se invece ci riuscissimo?» ribattei, assalito da un pensiero.
Ripiegai le gambe sotto di me e sedetti come un saggio in meditazione
sotto un albero. «Se arrivassi in fondo all'anno e ti guadagnassi la
paga? Ci hai pensato oppure sei venuto qui spinto da un desiderio di
morte e dalla speranza di qualche pasto decente?» Non sarebbe stato il
primo. Il suo silenzio lo tradì. Quel ragazzo non aveva un piano,
nessuna ambizione, solo una stupida, vaga speranza e una fantasticheria
infantile, non dissimile da un altro giovane che conoscevo. Ebbene, non
era il primo neppure in quello. Mi sfuggì un pesante sospiro. «Sai cosa
ti dico?» ripresi, aggredendo la paura presente nella sua voce rotta.
«Perché non rimaniamo insieme? Anch'io non ho amici, qui. Me ne
servirebbe uno.»

«Mi piacerebbe» rispose. «Tu sei il solo che non si sia fatto beffe di
quello che ero.»

Stavo pensando a quello che avevo detto a Cat molto tempo prima. `Vorrei
avere una mia nave, vorrei poter viaggiare.' «Non voglio rimanere qui.
Sto cercando di risparmiare per un passaggio su una nave, o almeno
potremmo arruolarci come marinai.»

«Non ne so niente!»

«Dopo un anno saprai combattere» ribattei, secco. «Le navi hanno bisogno
di uomini della sicurezza, di guardie. È solo che non ci hai pensato. Un
anno è un tempo lungo.» Non riuscivo a sopportare la sua disperazione,
avendo superato così di recente la mia.

Switch trasse di lato la tenda e mi fissò con occhi roventi. Era seduto,
raggomitolato sul fondo della cabina del tutto vestito, con la schiena
contro una parete e i capelli rossi incollati ai lati della faccia. Mi
fissò con occhi socchiusi. «A me questa sembra speranza. Credevo avessi
detto di non sperare.»

«Ho detto che la speranza è una nuvola» ribattei. «Questo non significa
che non esista.»


