\chapter{Un regno per un cavallo}

«Questi vecchi modelli anduniani dureranno più del sole» disse Gila,
torcendosi le mani in un modo servile che cozzava con i ricordi che
avevo di lei risalenti ai miei primi giorni a Borosevo, quando i suoi
operai mi avevano gettato fuori dopo aver recuperato la
\emph{Eurynasir}. Grazie alla Terra, non si ricordava di me, anche se in
sua difesa va detto che non ero più scalzo, scongelato da poco,
graffiato e con ustioni criogeniche. Indossavo i miei abiti migliori,
che pur non essendo particolarmente eleganti mi rendevano abbastanza
diverso dal ragazzo di strada che ero stato. Lei era brutta come la
ricordavo, con la testa avviata alla calvizie coperta di croste e di
chiazze irregolari di capelli brizzolati, e con il volto schiacciato
tinto di viola dalla voglia che si stendeva sul naso e su una guancia.
Era una plebea vestita con una tuta sporca e con le mostrine del
servizio della Contea di Mataro che le si staccavano dalle spalle.

Io, dal canto mio, stavo fingendo di appartenere a una qualche vaga
classe superiore. Nonostante quello che avevo detto, quel giorno non ero
lì per comprare ma per farmi un'idea di quale genere di nave si potesse
trovare in un cantiere di recupero. L'\emph{Eurynasir} era andata da
tempo e io ero insieme alla tozza caposquadra sulla rampa di carico di
un mercantile leggero anduniano di classe Pesce Palla, una brutta
astronave con il ventre largo e fatta di adamant butterato. Con il muso
piatto e poco elegante, aveva tutte le qualità di un mattone.

«Sai, l'ho sentito dire sul conto di queste bestie» commentai, {facendo}
scorrere una mano lungo uno degli enormi pistoni che sollevavano e
abbassavano la rampa.

Switch annuì a braccia conserte. «Quanto può trasportare?»

«La capacità di carico?» chiese Gila, massaggiandosi il mento mentre
controllava qualcosa sul suo terminale. «Appena al di sotto delle
trecento tonnellate. È un datato cavallo da tiro.»

Mi morsi un labbro e salii la rampa per portarmi all'ingresso della
stiva echeggiante, tutta linee squadrate e altrettanto brutta e semplice
quanto l'esterno. Anelli di ruggine brillavano come occhi sul pavimento
ripulito come meglio era possibile, e il tutto puzzava di uso e di
vecchio. «Quanto è vecchia?»

Gila imprecò. «Per le ossa della Madre, uomo, questa sì che è una bella
domanda.» Si interruppe, socchiudendo gli occhi nel fissare un paio di
suoi subordinati che passavano in fretta fuori sull'asfalto. «I cantieri
navali di Andun non hanno più prodotto niente da almeno quattro secoli
perché Ilium e Monmara li hanno estromessi dal mercato.»

«Fottuti Normanni» commentò Switch, in tono disgustato.

Doveva essere uno scherzo di qualche tipo -- forse un meme dei marinai
-- perché Gila sputò. «Fottuti Normanni. Costruiscono in fretta e in
modo economico, ma solo l'Impero costruisce in modo duraturo.» Lo disse
come se fosse stato uno slogan di qualche tipo che però io non
conoscevo. Ero lieto di aver portato con me l'altro mirmidone, anche se
mi ci era voluto del tempo per convincerlo.

`A cosa serve?' aveva obiettato. `Ci rimangono ancora mesi di servizio.'
Stavo perfino prendendo in considerazione di rinnovare l'ingaggio nel
Colosso -- dopotutto, altri seimila hurasam non mi avrebbero fatto male
-- ma avevo un terribile bisogno di più informazioni. Dovevo fare
\emph{qualcosa}. Adesso che avevo un piano, non potevo sprecare tempo
oziando nel colosseo. Avevo bisogno di agire, di fare quello che potevo
per avvicinarmi di un passo alla fuga. Anche se quella minaccia era
sbiadita nel tempo, temevo ancora l'Inquisizione, mi svegliavo ancora
sudando per il terrore di qualsiasi punizione potesse essersi abbattuta
su mia madre se si fosse scoperto il suo ruolo nella mia fuga.

Attraverso il davanti della camicia giocherellai con l'anello di
famiglia come se fosse stato un talismano che li potesse tenere a
distanza e non una calamita che li attirasse su di me. Rigirai fra le
dita un pensiero che mi cresceva dentro da mesi, lo stesso che mi aveva
riportato da questi criminali e in quel cantiere dove ristrutturavano
navi rubate, alla periferia di Borosevo. Era la mia arma segreta, se
avessi giocato bene le mie carte.

«Queste cose sono vecchie!» esclamò Switch, scuotendomi dai miei sogni a
occhi aperti. «Ne uscirei con i capelli più grigi di quelli di mio nonno
se viaggiassi in una di queste!» Picchiò sul coperchio in acrilico di
una capsula criogenica. Non avevo idea di cosa il ragazzo stesse
vedendo, ma Switch aveva trascorso tutta la vita a bordo di astronavi e
fra una notte di intrattenimento e l'altra aveva acquisito una buona
quantità di conoscenze pratiche; tante da avere la stoffa di un buon
marinaio spaziale, indipendentemente da quello che pensava di sé stesso.
Lui e Gila discussero per parecchio tempo sulle capsule e io finsi di
gironzolare per la stiva, esaminandola in lungo e in largo prima di
tornare alla rampa di accesso. Lasciai vagare lo sguardo sull'asfalto e
il punto in cui un tempo avevo rotto un braccio a un uomo per poi
appuntarlo sugli hangar di fronte. C'era quello dove l'\emph{Eurynasir}
si era trovata tanti anni prima. Mentre l'osservavo una delle porte si
sollevò, sferragliando nel suo compartimento fra le grida della squadra
di lavoro. E lei era là.

Era una nave, ma era tanto diversa dal mercantile leggero anduniano in
cui mi trovavo quanto mia nonna, la viceregina, era diversa dalla donna
plebea alle mie spalle. Per un momento le dita soffrirono per il
desiderio di avere con me le mie matite per catturare la sua immagine
nella luce arancione del sole, con il suo scafo carminio -- ammaccato e
strinato -- tinto del colore del vino. Non era uno yacht, non era rara e
vistosa, eppure era bella come può esserlo una spada disadorna ma ben
fatta. Aveva una forma vagamente triangolare, come una punta di freccia
o una delle mante che ho visto una volta nei fiumi aerei delle stazioni
mandari dotate di oblò di alluvetro a specchio. Non aveva ali, perché
tutto il suo corpo era una grande ala, sollevata non da bruciatori a
fusione ma da repulsori Royse silenziosi quanto lo spazio stesso.

«Hadrian!»

«Prego?» Mi girai e rimasi quasi sorpreso di scoprire che non ero solo.
«Hai detto qualcosa?»

Gila aprì la bocca, ma Switch la prevenne. «Vuoi vedere il resto della
nave?»

«Cosa?» Lanciai un'occhiata all'astronave dall'altro lato del cortile.
«Io... che ne pensi di quella?»

«L'uhran?» La caposquadra si acciglio. «Non è certo un mercantile.»

La accantonai con un gesto e attraversai in fretta il cortile verso
l'altra nave, ascoltando solo in parte mentre lei mi forniva una lista
di specifiche: capacità di carico, accelerazione, curvatura massima.
«Quanto costa?» Lei mi fornì una cifra e per poco non imprecai perché il
prezzo era quasi il doppio di quello della nave anduniana, che già era
di parecchie volte superiore ai miseri seimila hurasam promessi a me e a
Switch come mirmidoni. «E quanto costa in marchi imperiali?» chiesi in
tono sofferto.

«Quello \emph{era} il prezzo in marchi imperiali» precisò Gila,
scrollando le spalle. «La contea non accetta contanti a meno di pagare
subito tutta la cifra.» Sentii il cuore che mi sprofondava. Tre milioni
e due di marchi imperiali. Neanche con Pallino ed Elara potevamo sperare
di mettere insieme una caparra per una nave del genere, e tantomeno un
pagamento completo. Avrei ceduto alla disperazione se non fosse stato
per una piccola cosa.

Incrociai le braccia e abbassai il mento nel sollevare lo sguardo sulle
linee rosse del mercantile leggero uhran. Avevo sbagliato di grosso nel
valutare il prezzo di un'astronave. Mio padre aveva parlato di acquisti
del genere con un freddo distacco che rasentava il disinteresse. Avevo
però ancora la mia arma segreta. Adocchiai nervosamente Switch. Non
gliene avevo parlato e non sapevo bene come avrebbe reagito.

«Non è che accettereste un prestito dalla Rothsbank?» chiesi, senza
molta speranza. «O un piano di pagamento con un conto mandari?» Non
avevo nessuno dei due e neppure il modo di procurarmelo, ma faceva tutto
parte della danza. Switch mi stava osservando, sempre a braccia
conserte. Lo stavo perdendo? Ne ero certo. Era impossibile che
riuscissimo a trovare tre milioni e due di marchi, né entro la fine
dell'anno né entro quella del decennio. Neppure in cento anni standard.

«Quello è il modo normale di fare affari» affermò Gila, venendo a
fermarsi accanto a me mentre ammiravo la nave uhran. «In tutti gli anni
in cui ho lavorato per la contea c'è stato un solo uomo -- un Lothriano
-- che abbia pagato in contanti. Una dannata rarità.» Stava scuotendo la
testa. Alle sue spalle, Switch mimava l'atto di tagliarsi la gola,
chiedendomi in silenzio di dare un taglio al discorso. Aveva senso, dato
che nelle intenzioni quella era solo una visita esplorativa per
confermare i miei sospetti.

Da ragazzo avevo visto mio padre e il personale del Casato
mercanteggiare con il Consorzio, con le altre case commerciali mandari,
con altri Casati palatini imperiali, e sapevo come si conducevano gli
affari. Soprattutto, ero stato un mendicante e conoscevo le intonazioni
e i modi di fare di chi era disperato con la stessa precisione con cui
conoscevo le lingue del Commonwealth e di Jadd. Avevo commesso un errore
nel dimostrare il mio evidente apprezzamento per l'astronave uhran,
mentre avrei dovuto mantenere la mia maschera di disinteresse. «È ancora
più di quanto il mio socio e io siamo disposti a pagare.» Non essere
disposti non significava non essere in grado di farlo. Il volto di
Switch si fece accuratamente inespressivo e immaginai la conversazione
che avremmo inevitabilmente avuto più tardi.

\emph{Non abbiamo soldi, Had! Non ne abbiamo. Lo hai dimenticato?}

«Che ne dici di un baratto?» domandai, giocherellando con la sottile
catena che avevo al collo.

Gila indietreggiò di un passo e mi squadrò. «Prego?»

Esitai per un momento, guardando in alto verso le travi di supporto
dell'hangar in cerca di telecamere. Non se ne vedevano, ma quel deposito
apparteneva al Casato Mataro e all'Impero, e solo perché non ero in
grado di vederle non significava che non ce ne fossero. D'altronde,
forse non ce n'erano davvero. Dopotutto, questi erano i criminali che mi
avevano lasciato per morto in un vicolo. `Dovresti essere lassù, sai?'
Le parole di Cat mi si agitarono nel petto e serrai le dita intorno
all'anello di famiglia. `Per come parli, il tuo posto è in un castello.'
Ebbene, lo era stato una volta. Guardai Switch per un momento e sentii
un sorriso che mi sfiorava le labbra, sottile ed evanescente come la
speranza, senza però che ne contenesse. «Ho i titoli di proprietà di
ventiseimila ettari di terra su Delos, trattenuti in mio nome sotto il
sigillo della prefettura governativa e della viceregina di settore.
Valgono più di ciascuna di queste navi.» Valevano quanto entrambe e
ancora di più.

Per un momento Gila sgranò gli occhi per la sorpresa e, sperai, per
l'avidità, perché intuivo che la sua commissione sarebbe stata elevata.
Poi quel sogno morì, soffocato dal socchiudersi dei suoi occhi. «Non
dici sul serio.»

Quello era il momento critico. Vedi, lettore, c'era una possibilità che
Gila avesse visto il mio anello, tanti anni prima. Lo tirai fuori e lo
tenni davanti ai suoi occhi socchiusi, che poco dopo si dilatarono,
scuri e fangosi come una pozzanghera prosciugata. In essi c'era paura,
ma nessuna traccia di rabbia, o di riconoscimento. Switch non disse
niente, ma alle spalle di Gila vidi le sue sopracciglia schizzare verso
l'alto, poi i suoi occhi si incupirono fino a diventare qualcosa di
terribile mentre quelli di Gila si sgranavano per la sorpresa. Per la
prima volta da molto tempo il sorriso in tralice mi affiorò sul viso.
Per un momento accantonai la reazione di Switch e mi concentrai sulla
donna che avevo davanti.

«È vero?» chiese lei, protendendosi a toccare l'anello.

Ritrassi la mano di scatto ed evocai un'imitazione dell'ira di Crispin.
«Non lo toccare!» Per poco non le colpii la mano per allontanarla, come
sarebbe stato giusto fare. Mio zio Lucian, morto anzitempo, era solito
portare con sé un antiquato frustino da equitazione apposta per
occasioni del genere. Lo avevo odiato, ma una parte di lui influenzava i
miei gesti. Dopo un istante dissi: «È possibile effettuare il baratto? E
saresti disposta a trattenere la nave finché non avrò concluso i miei
affari qui a Borosevo?» Lei fece per replicare, ma io sollevai un dito.
«Prima di procedere, sii sincera: questa è una transazione che sei
autorizzata a fare? Oppure devo chiamare i logoteti pluripotenti?» A
casa, il ricorso ai pluripotenti prevedeva il pagamento di una parcella
che era a carico non dell'acquirente ma della commissione del venditore.
Era un mezzo per prevenire o quantomeno ridurre pratiche predatorie da
parte del venditore ed era un privilegio riservato solo a clienti
palatini e patrizi.

Gila si inchinò con una grazia sorprendente. «Non mi ero resa conto che
fossi un palatino, sire.» Rimase inchinata. «Chiedo scusa.» Fece una
pausa per un acido momento, poi aggiunse in tono cupo: «La contea
potrebbe barattare tenute su un altro pianeta, ma come dice Vostra
signoria non è una trattativa che io possa gestire.»

Annuii. «È tutto quello che ho bisogno di sapere. Puoi trattenere la
nave per me?»

«Le riparazioni sono ancora in corso» replicò, torcendosi le mani tozze.
«Non è in vendita. Non ancora.»

«Splendido» dichiarai, come se la cosa fosse stata decisa. «È buona
terra... appartiene alla mia famiglia da generazioni.» Il mio sguardo si
posò su Switch, il cui volto era imperscrutabile. «Mio padre me l'ha
lasciata quando mio fratello ha ereditato il titolo.» Nel parlare non
guardai neppure Gila, sapendo che mio padre non lo avrebbe fatto.

Anche se lui aveva congelato i miei beni in seguito alla mia scomparsa,
l'anello avrebbe ancora registrato la mia proprietà, perché lui non
aveva avuto l'opportunità di cambiare quelle impostazioni. L'anello
racchiudeva l'autorità e il peso del nome del mio Casato, e se lo avessi
usato per promettere qualcosa, il Casato Marlowe sarebbe stato vincolato
legalmente a tener fede a quella promessa. Per questo motivo il
venditore non avrebbe trattenuto \emph{me} o rimandato la \emph{mia}
partenza, perché è il privilegio dei palatini di ricevere fiducia in
transazioni del genere, così come è nostro dovere rispettare gli impegni
presi firmando con un anello del genere, impegni controllati da quella
stessa Inquisizione che speravo di evitare.

Mio padre però avrebbe indubbiamente contestato la transazione. Non
dubitavo che avesse già reclamato per sé le mie tenute dopo la mia
scomparsa e speravo che lo avesse fatto, perché questo avrebbe privato
Emesh di qualsiasi guadagno da quella situazione. Soprattutto, avrei
potuto essere su quella nave e pronto per lasciare l'Impero prima che
Gila o i pluripotenti emeshi si rendessero conto di quello che avevo
fatto e che la notizia della vendita arrivasse su Delos e a casa. La
Cappellania avrebbe indagato e scoperto che non solo i lavoranti dei
moli avevano aiutato la mia fuga, ma che avevano scongelato
originariamente lo scomparso lord Marlowe. In un solo colpo avrei
comprato una nave con una terra che non avevo il diritto di vendere e mi
sarei vendicato delle persone che mi avevano gettato in strada e
derubato della lettera di Gibson.

Come vendetta sarebbe stata perfetta se fosse mai andata a buon fine.


