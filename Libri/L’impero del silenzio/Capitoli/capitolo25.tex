\chapter{Povertà e punizione}

Ero accoccolato sulla vecchia borsa frigo che usavo come contenitore per
le mie cose e come sedia nella caditoia che avevo considerato come la
mia casa nelle ultime dieci notti, e stavo mangiando mezza anguilla
affumicata che avevo rubato a un venditore ambulante fuori dal colosseo.
La carne era tenera, spolverata di aglio e di salsa di soia, e
leggermente affumicata. Ormai ero diventato un ladro da quasi una
settimana e avevo deciso che ero bravo, anche se tutto il corpo mi
doleva per la fatica dovuta alla forza di gravità.

Quello era un buon posto, asciutto a meno di considerare il canale sul
fondo della caditoia, e al chiuso se non si contava la bocca aperta
della galleria costruita dentro il fianco del Distretto Bianco, così
chiamato per il colore del bastione rivestito di calce che lo separava
dal Distretto Inferiore. Il Distretto Bianco era raccolto intorno alla
base dello ziggurat del castello, a una cinquantina di piedi al di sopra
del livello del mare, un cimelio dei giorni in cui l'Impero aveva
sottratto Emesh alla Compagnia Normanna Unita che lo aveva colonizzato.
Là vivevano i ricchi del pianeta, i suoi nobili e i plutocrati che
venivano dall'esterno, i capifazione delle Gilde e gli uomini d'affari
importanti, le celebrità locali e i gladiatori. Anche il sacrario della
Cappellania si trovava in quel distretto, una struttura dalla cupola di
rame che sorgeva accanto alla brutale massa di cemento della sua
bastiglia. La presenza di noi mendicanti non era tollerata nel Distretto
Bianco salvo che nel giorno della Somma litania, che cambiava di
settimana in settimana perché il ciclo notte-giorno di Emesh non
combaciava con il calendario standard.

Sedevo sull'orlo del caditoio e guardavo fuori e verso il Distretto
Inferiore, quel labirinto di canali e di bassi edifici con le loro
strutture pensili. Il sole cominciava a calare, livido e rosso sul mare
scuro e io facevo dondolare i piedi nudi e doloranti nel vuoto per dare
loro aria mentre mangiavo quello che restava della mia anguilla con
qualcosa che si avvicinava all'appagamento, desiderando di avere
qualcosa di meglio dell'acqua piovana con cui accompagnarla.

Un volo di piccioni terrestri si levò dall'angolo di una strada e vidi
una lanterna votiva di carta salire oltre la cima dei tetti. Ne salivano
sempre dalla città, per portare alla Madre Terra le preghiere e le
suppliche delle vittime della necrosi. Mi appoggiai alla parete del
caditoio, con la testa che riposava contro il cemento imbiancato a
calce.

«Tu!»

Com'è che si riesce sempre a capire quando una parola è diretta a te?
Ogni muscolo mi si tese come la corda di un arco pronto al tiro e mi
guardai intorno, assalito per un momento dal terrore che qualcuno fosse
sopraggiunto attraverso la porta sbarrata alle mie spalle. Però non era
così. Individuai immediatamente i colpevoli, un uomo e una donna che
portavano l'uniforme cachi dei prefetti urbani che dall'orlo del
marciapiede adiacente il canale si stavano spostando in fretta verso
l'accesso che descriveva un arco sull'acqua, dove una scala a pioli era
fissata alla parete.

«Vieni giù!»

Mi alzai in piedi e, in preda al panico, indietreggiai fino a inciampare
nella borsa frigo alle mie spalle, una cosa dannatamente stupida da
fare. Uno dei miei piedi assaliti dai crampi colpì la borsa su cui ero
seduto e la mandò a cadere oltre l'orlo con un calcio. Sentii un urlo
che mi si bloccava dietro i denti, dovuto alla sorpresa mista allo
sgomento e alla frustrazione, perché in quel piccolo contenitore blu
c'era tutto quello che possedevo al mondo, tranne i vestiti e l'anello
di famiglia. C'erano il mio cibo di scorta, due riviste che avevo
sottratto da un'edicola tre giorni prima, le bottiglie vuote che usavo
per raccogliere l'acqua piovana. E il mio denaro. Ero riuscito a mettere
insieme qualche dozzina di monete d'acciaio, al punto che ne avevo quasi
abbastanza per formare un solo kaspum d'argento. Quei soldi avrebbero
potuto fruttarmi una notte in uno dei molti alberghi da quattro soldi
del Distretto Inferiore, ma li stavo mettendo da parte per comprarmi
delle scarpe.

Un urlo inarticolato mi sfuggì dalle labbra e prima di avere il tempo di
rifletterci sopra, mi scagliai giù dietro alla borsa, tuffandomi a piedi
in avanti nell'acqua verde del canale. I prefetti gridarono, ma il suono
si perse nel liquido spostamento d'aria intorno alle mie orecchie.

Colpii la superficie del canale come un masso, con le gambe ripiegate
sotto di me, e quando riaffiorai mi guardai intorno alla ricerca del
pesante contenitore frigo. Non avevo visto dov'era caduto. Era andato a
fondo? Era riuscito a cadere sulla strada invece che nell'acqua?
Dannazione, avevo agito troppo in fretta, ero stato tre volte stupido.
Poi lo vidi, che dondolava nell'acqua vicino alla parete di cemento da
cui mi ero lanciato e nuotai verso di esso, ora consapevole delle grida
alle mie spalle. «Cosa ci facevi lassù?»

Forse mi sarei dovuto cavare da quella situazione a parole. Nuotando sul
posto con la borsa frigo stretta in una mano mi spinsi lontano dalla
parete del Distretto Bianco e mi diressi verso il marciapiede appena al
di sopra dei canali. I prefetti si affrettarono a lasciare l'arco del
ponte di accesso per intercettarmi, ma io fui più veloce e saltai sul
marciapiede lasciandomi dietro una scia di acqua di mare verde e
puzzolente. Una fila di scolari che procedeva spedita sotto l'occhio
vigile dell'insegnante indicò ridendo quell'uomo grondante con i capelli
arruffati. «Stavo solo godendo del panorama, signora. Da lassù si vede
tutta la città.» Cercai di sorridere e di far passare il mio aspetto
selvaggio per una conseguenza del mio spettacolare tuffo e non per
quella di un mese trascorso senza lavarmi.

Il prefetto mi fissò con sguardo rovente, squadrandomi con gli occhi
neri mentre batteva un colpetto sullo storditore nella fondina.
«L'accesso a quel caditoio è vietato al pubblico. Lo sanno tutti.»

«Sì, signora.» Chinai la testa e indietreggiai di un passo, stringendo
la borsa al petto grondante. «Mi dispiace, io...»

«Documenti» ordinò l'altro prefetto, protendendo una mano. «Vediamo la
tua identificazione.»

Indietreggiai di un passo. «Io...» Cosa potevo dire? «Non li ho con me.»

Il secondo prefetto sospirò. «Allora dovrai venire con noi.»

La forma priva di sensi di Crispin si mosse appena al di fuori del mio
campo visivo, tormentandomi. Cosa dovevo fare? L'anello con il sigillo
parve farsi più pesante intorno al mio collo e mi guardai intorno. Non
molto lontano c'era uno stretto vicolo incastrato fra due negozi, e se
fossi riuscito a imboccarlo...

«Avanti, vieni.» Uno dei due prefetti si mosse per afferrarmi. Io
indietreggiai con decisione di un passo mentre mi afferrava il polso.
Cedendo al panico feci descrivere un ampio arco alla borsa ancora
grondante e lo colpii alla tempia. Lui barcollò all'indietro,
lasciandomi andare con un grido di sorpresa e di dolore, e il coperchio
della borsa si aprì di scatto. Le monete, le riviste e metà di un
vecchio panino piovvero sui due agenti, mentre le banconote e i
tovagliolini umidi aderivano ai lati della scatola. Soffocando un
singhiozzo mi girai e mi misi a correre.

Non arrivai lontano.

Il raggio dello storditore mi sfiorò una gamba, i cui muscoli si
rilassarono come gomma vecchia. Barcollai e persi la presa sulla borsa
ormai vuota, che cadde rumorosamente sulla pavimentazione. Lottai per
rialzarmi, ma prima che riuscissi anche solo a mettermi in ginocchio uno
stivale mi si abbatté sulla spalla, premendomi contro il terreno. La
vista mi si oscurò quando sbattei con la testa contro la pavimentazione
e riuscii a stento a strisciare in avanti. Lo storditore mi aveva
raggiunto solo di striscio, potevo ancora correre se solo fossi riuscito
a ignorare il formicolio che mi vibrava nella gamba, se solo fossi
riuscito a rimettermi in piedi. Qualcuno mi assestò un calcio sotto le
costole e sussultai. Un'immagine di quella notte a Meidua divampò dentro
di me mentre il respiro mi si bloccava e il suono irregolare del mio
sangue mi pulsava nelle orecchie, soffocando le imprecazioni dei
prefetti. Qualcosa mi raggiunse alla testa e la vista mi si offuscò di
nuovo. Rimasi immobile, prono, e serrai i denti per impedirmi di
singhiozzare o di gridare.

Sentii delle mani che mi tastavano e mi rigiravano le tasche. Trovarono
un paio di monete d'acciaio e un buono sconto di una catena di carretti
che vendevano pesce. Per fortuna non trovarono l'anello perché non si
presero la briga di girarmi. «Questo bastardo è povero in canna»
commentò la donna.

«È feccia dei bassifondi, Ren» replicò l'altro prefetto. «Non vale la
pena di arrestarlo.» Mi urtò con la punta del piede e io mi morsi la
lingua, avvertendo il sapore metallico del sangue. «Avresti dovuto
correre più in fretta, \emph{neg}.»

La donna imprecò e avvertii una pressione sul collo, poi qualcosa di
duro mi colpì in pieno sulla nuca ma non persi i sensi, emisi solo un
gemito mentre l'effetto dello storditore mi riscaldava tutto un lato del
corpo dove mi aveva sfiorato, tanto da indurmi a dubitare che adesso
sarei anche solo riuscito a camminare in linea retta, tantomeno a
correre verso il vicolo. Pensai a come avevo pestato quel lavorante un
mese prima e mi sentii sopraffare dalla vergogna. Il prefetto si
sbagliava, non avrei dovuto correre più in fretta, non sarei dovuto
fuggire affatto. Snudai i denti insanguinati e sputai sul cemento
accanto alla mia testa.

In quel momento la donna mi afferrò per i capelli e mi sollevò la testa
dal cemento mentre si abbassava e mi sussurrava all'orecchio: «Non farti
trovare di nuovo da me dove non dovresti essere, \emph{neg}, o te ne
pentirai.»

Sulle labbra mi si formò una risposta piccata -- qualcosa sullo smettere
di assumere ormoni di cavallo -- ma la ricacciai indietro e lottai per
raggiungere uno stato di calma simile all'apatia. Mi afflosciai e sentii
la faccia che mi si ammaccava quando lei mi lasciò ricadere sulla
pavimentazione. Non so per quanto tempo rimasi lì disteso o perché
nessuno degli uomini e donne di passaggio si fermò per aiutarmi.


