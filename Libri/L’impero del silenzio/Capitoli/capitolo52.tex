\chapter{Piccole conversazioni}

Non prestai molta attenzione a Gilliam Vas. Anche se era un prete della
Sacra Cappellania, e addirittura un cantore e non un semplice
anagnostico, la mia fanciullezza su Delos mi aveva abituato a simili
antagonisti dotati di grande potere. Inoltre, le apparizioni dell'intus
erano poche e distanziate fra loro, e comunque avevo altre cose che mi
distraevano. Le mie escursioni con Dorian e Anaïs continuarono come
prima e il loro jaddiano migliorò, com'era bene che fosse dato che su
Emesh aspettavamo la visita di un ambasciatore dei principati, e il
conte e lord Luthor volevano fare colpo sul governatore satrapo jaddiano
in visita con la cultura e la dimestichezza di linguaggio della loro
famiglia.

Non avevo mai incontrato un palatino jaddiano, uno dei tanto decantati
\emph{eali al'aqran}. Un tempo gli Jaddiani erano stati cittadini
imperiali ma, uniti dalla loro identità etnica -- i loro antichi
antenati avevano popolato il Mediterraneo ed erano stati fra gli ultimi
a lasciare la Vecchia Terra --, si erano ribellati contro Forum e il
Trono Solare oltre novemila anni prima. Contro ogni probabilità, quei
nobili un tempo provinciali che vivevano sul confine esterno della
galassia avevano conquistato l'indipendenza e il diritto di controllare
il destino genetico dei loro figli. Per quanto fossero pochi -- appena
un'ottantina di famiglie principesche e i loro fedeli vassalli -- erano
potenti, con eserciti di soldati mamelucchi clonati e una forte
tradizione di servizio militare che faceva sfigurare perfino l'antica
Turchia imperialista. Quei principi ribelli non si sarebbero mai più
inginocchiati di nuovo davanti al nostro sacro imperatore. Liberi dalla
morsa della Cappellania -- anche se molti principati adoravano ancora la
Madre Terra -- quei nobili stranieri si riproducevano al di là di
qualsiasi cosa l'Impero avrebbe mai permesso. Abbracciavano l'eugenetica
in modi che il Sole non avrebbe mai adottato, rassicurati dalla
supremazia del loro programma genetico e del loro modo di vivere. Erano
una nazione di superuomini, di semidèi.

Speravo di poterli incontrare, perché le storie sulle corti di cristallo
dell'Alcaz du Badr, degli harem del Principe Adia e della velocità e
abilità inumane dei maestri di spada maeskoloi erano una leggenda
nell'Impero, esotiche quanto i feroci Cielcin, strane come i sussurri su
Vorgossos e sugli Extrasolari che si nascondevano fra le stelle.

Dopo quel pomeriggio alla riserva, Valka e io avevamo passato più tempo
insieme. Quali che fossero le nostre differenze, avevamo trovato un
terreno comune nel nostro rispetto per gli xenobiti e nel nostro
disprezzo nei confronti della Cappellania. Sapeva che ero stato sincero,
quel giorno sulla spiaggia, quando per un momento mi ero trovato libero
dalla pressione del tallone imperiale. Aveva visto \emph{me}, il ragazzo
che era fuggito da Delos piuttosto che essere spedito su Vesperad, e non
mi aveva disprezzato.

Non mi era però ancora concesso di lasciare il palazzo senza
supervisione, quindi non era scritto che potessi rivelare altre verità a
Valka. Senza dubbio nel castello di Borosevo c'erano posti dove si
poteva parlare davvero in privato, ma solo il conte e il suo personale
anziano sapevano quali fossero, e chiederlo a loro era fuori questione.
Così, mi ritrovai davanti alla porta di Valka, non per la prima volta e
neppure per l'ultima. Non mi lasciava mai entrare, ma passeggiavamo
lungo i colonnati del palazzo e nei corridoi a volta, oppure scendevamo
nei giardini a terrazza che tempestavano la facciata meridionale dello
ziggurat del castello.

In questa occasione lei non aprì quando bussai per la prima volta, e
neppure alla seconda. Rimasi lì a lungo in un silenzio pieno di disagio,
contando le piastrelle a spina di pesce nere e marrone chiaro che
formavano il pavimento. Un logoteta passò in fretta, accompagnando un
dignitario straniero vestito come un Durantino. Nessuno dei due mi
rivolse più di un'occhiata di sfuggita e io finsi di essere assorto
nella contemplazione dell'immagine olografica sulla parete opposta alla
stanza della dottoressa, un vorticante paesaggio impressionistico di
nuvole, con una nebbia fra il rosa e il viola che avviluppava la
geometria di pietra bianca di un monastero dall'architettura mandari.
Una targa in un angolo dell'immagine diceva che quello era il Tempio
Bashang, su Cai Shen.

Cai Shen era scomparso. Devastato dai Cielcin. Mi chiesi se quelle
pietre bianche c'erano ancora o se la guerra le aveva fatte diventare
nere. Quale scherzo crudele degli dèi del fato aveva fatto regolare
quella piastra olografica su Cai Shen? Passai una mano attraverso
l'immagine e premetti le dita contro il metallo della parete
retrostante, poi volsi le spalle, cercando di non pensare che adesso
quel tempio e tutto il suo mondo erano di vetro. L'immagine sfrigolò e
le luci si attenuarono al punto che solo la luce serale del sole venne a
trapassare le alte finestre su entrambi i lati dell'ologramma. Quegli
sbalzi di corrente erano diventati una cosa tanto comune che non battei
neppure ciglio mentre bussavo di nuovo alla porta della dottoressa e
scuotevo il pomolo della porta. «Valka?»

La porta non era chiusa a chiave? Era stata lasciata in quel modo?
Oppure lo sbalzo di corrente aveva fatto scattare la serratura
elettronica? In ogni caso mi ritrovai ad aprire il battente e ad
affacciarmi su una stanza dove non sarei dovuto entrare.
\emph{D'accordo, ragazzo, dentro o fuori?}

Nonostante la mancanza di illuminazione potevo vedere che le stanze di
Valka erano più eleganti e più grandi delle mie, ma nonostante questo
erano un disastro. Mi fermai per un momento quando un capo di biancheria
verde sul pavimento mi ricordò che stavo ficcanasando. Quel disordine la
umanizzava, anche se come tutti i giovani uomini mi aggrappai alla mia
immagine della perfezione di Valka che si andava sgretolando. I capi di
vestiario costellavano il pavimento, pendevano dallo schienale delle
sedie, giacevano su documenti stampati e note di stoccaggio disseminate
su un tavolinetto e sul tavolo da pranzo. Mi costrinsi a rivalutare la
dottoressa e a ricordare a me stesso che all'altra estremità del mio
affetto personale c'era una persona fisica e non una sognata. Nel
rammentarlo a me stesso mi schiarii la gola e chiamai con voce debole:
«Dottoressa Onderra? La porta non era chiusa a chiave. Io...
l'appuntamento di questa sera è ancora valido?» Nessuna risposta. La
sensazione di intrufolarsi si fece ancora più forte, {giustamente}, ma
sentii di essermi spinto troppo oltre per arrendermi a quel punto.
Riluttante ad avanzare ancora nella stanza, mi ripetei. «Dottoressa
Onderra? Valka? Sono Hadrian.»

Finalmente la scorsi, seduta sull'ampio davanzale di una delle finestre,
con la schiena rivolta verso di me e lasciata in ombra dai tendaggi.
Aggirai con passo silenzioso un paio di pantaloni abbandonato mentre
lottavo contro emozioni conflittuali destate dal pensiero che Valka
potesse non essere vestita. Mi spostai con cautela fino a essere nel suo
campo visivo. \emph{Era} vestita, ma aveva gli occhi chiusi. Dormiva?
«Dottoressa?»

Aprì gli occhi e parve acquisire solo per gradi la consapevolezza di
dove si trovasse e di chi io fossi. «Hadrian? Come sei entrato?»

«La porta non era chiusa» spiegai, con il mio inchino più profondo e
caloroso. «Non sapevo se l'avessi lasciata così o se lo sbalzo di
corrente...» Accennai con la mano alla stanza buia. «Dovrebbero proprio
fare qualcosa al riguardo. Questi sono gli appartamenti diplomatici, e
se le porte si aprono durante questi sbalzi...»

Valka reagì con un sorrisetto ironico. «Allora sono fortunata che un
giovane uomo come te abbia fatto irruzione nella mia stanza per
difendermi.» La scelta di parole mi lasciò confuso ed esitai per un
momento prima di ricordare che Valka era molto più matura di me, il
prodotto di una correzione genetica demarchica non dissimile dalla mia.

Sapevo che si stava facendo beffe di me ed ebbi la buona grazia di
arrossire e di distogliere lo sguardo. «Mi dispiace.»

«Lascerò correre» affermò, con un sorriso tagliente che attraversava i
piani bianchi del suo viso. «Per questa volta.»

Le luci scelsero quel momento per riaccendersi, accompagnate da un
sommesso ronzio meccanico di quel genere che non si nota finché non
scompare. Sotto il chiarore delle lampade, il disordine di Valka
appariva ancora peggiore. Imbarazzato, indietreggiai e abbassai lo
sguardo sul tavolino adiacente a un divano coperto da una coperta
aggrovigliata. Sotto i resti di un pasto mangiato a metà, il tavolino
era cosparso di carte, alcune nuove, altre ingiallite e probabilmente
più vecchie di me. In contrasto con il suo modo di tenere la casa, la
sua grafia era notevolmente precisa. Non ero in grado di leggere
l'elaborato scritto tavrosiano ma riconobbi i suoi schizzi degli
anaglifi degli Umandh. Ne aveva disegnati parecchi collegati uno
all'altro in una sorta di bolle che mi ricordavano...

«Hai mai visto un \emph{udaritanu} dei Cielcin?» Mi tastai le tasche,
rammentando tardivamente di non avere con me né una penna né il mio
diario, sul quale ce n'erano alcuni esempi, in mezzo ai ritratti, ai
panorami e alle citazioni miniate in cui mi dilettavo.

Mi guardò sbattendo le palpebre. «Cosa hai detto?»

«La loro scrittura» risposi con un intenso sorriso. «Hai una...»

Valka tirò fuori dal nulla una penna e me la lanciò. L'afferrai senza
riflettere e mi appollaiai sull'orlo del divano. Trovato in mezzo a
quella confusione un foglio di carta non utilizzato, lo sollevai.
«Posso?» Lei assentì sollevando una mano e cominciai a scribacchiare
sulla pagina. L'inchiostro era scadente, ma mi concentrai sul mio
compito. «I Pallidi usano una grafia non lineare per le opere d'arte, la
poesia, i monumenti e cose del genere.» Sollevai verso di lei il foglio
su cui spiccavano ora alcuni rapidi glifi. Mentre lei li studiava, mi
alzai per sbirciare oltre le tende e da sopra la sua spalla. «Vedi,
usano le dimensioni e la posizione relative dei logogrammi per
trasmettere la struttura grammaticale.» Indicai una sequenza di glifi a
spirale che rimpiccioliva di dimensioni. «In questo modo tutta questa
stringa -- questa frase -- è subordinata a questo soggetto.» Sollevò lo
sguardo su di me inarcando un sopracciglio. Improvvisamente imbarazzato,
mi grattai la nuca. «Sono certo di aver sbagliato un logogramma, da
qualche parte, ma puoi vedere il principio di base.»

«Pensi che gli anaglifi degli Umandh siano come questi?»

Mi restituì il foglio e io mi rimisi a sedere. «Non posso dirlo, li ho
solo guardati. Gi... il mio tutore era solito tracciare quadrati intorno
ai diversi elementi di una frase, quando stavo imparando. Sembra un po'
come questo.» Liberai il foglio da lei disegnato e le mostrai i cerchi
degli Umandh che si intrecciavano. «Gli Umandh collegano sempre i loro
simboli in questo modo? Oppure la cosa figura solo in quei segnavento
che mi hai mostrato a Ulakiel?» Lei mi stava sorridendo, un sorriso
decisamente troppo ampio. «Cosa c'è?»

Le luci tremolarono mentre si protendeva a strapparmi di mano i suoi
appunti. «Stavo risparmiando la carta, idiota.» Però non smise di
sorridere, quindi le sue parole non mi ferirono.

«Oh.» Anch'io sorrisi, e arrotolai i segni che avevo tracciato.

«Non lo fare!» protestò Valka, alzandosi dal davanzale. Aveva trascinato
alcuni cuscini dall'altra stanza per crearsi un posto a sedere più
confortevole vicino al vetro. Protese una mano, segnalandomi di
consegnarle i segni che avevo tracciato «Posso tenerli?» Dovetti fare
una smorfia, perché aggiunse: «Come fonte di ispirazione.» Fuori
cominciava a piovere di nuovo e in lontananza, sulle acque verdi, la
massa scura di una tempesta strisciava contro il tetto del mondo,
proiettando lampi di calore simili alle scintille prodotte da una mola.

Percependo una pausa nella conversazione, chiesi: «Perché eri seduta qui
al buio?»

«Cosa?» Si girò verso di me, chiaramente distratta da qualcosa che non
potevo vedere, il che era strano perché non stava guardando fuori dalla
finestra ma verso un angolo della cucina dove non c'era niente. «Oh,
scusami. Stavo solo pensando. Sai che fra non molto la marea si ritirerà
da Calagah?»

Mi appoggiai leggermente all'indietro contro i cuscini del divano,
uguale a quello che avevo nelle mie stanze, elegante e imbottito,
rivestito di pelle marrone. «Allora te ne andrai?»

«Solo per una stagione,» replicò «e manca ancora del tempo. L'anno
locale è lungo, e `presto' su Emesh non ha lo stesso significato che ha
là fuori.» Agitò in cerchio un dito in direzione del soffitto a travi
scoperte per indicare il cielo al di là di esso, poi si spostò nella
zona della cucina e la guardai versarsi un bicchiere d'acqua e buttarlo
giù in un sorso.

Avevo visto altre volte quel gesto, nei miei compagni mirmidoni e in me
stesso, dopo una notte passata a bere in seguito a una vittoria
nell'arena. «Stai bene?»

«Soffro di emicranie» spiegò, portandosi una mano sugli occhi con enfasi
distratta. «Niente di che, davvero.»

«Posso procurarti qualcosa?» domandai, non sapendo che altro dire.

Il suo sorriso riapparve. «Questo è il \emph{mio} appartamento, M
Gibson.» Si versò un altro bicchiere d'acqua e tornò a sedersi sul bordo
del davanzale. Incorniciata là -- linee curve contro linee squadrate,
sullo sfondo della pioggia -- sembrava più grande di quanto non fosse,
una statua simile alle luminose sculture del mio pianeta.

«Mi stai fissando.»

«Scusami.» Mi riscossi e abbassai bruscamente lo sguardo mentre
aggiungevo, rispecchiando inconsciamente le sue parole: «Stavo
pensando.» Non era del tutto vero, non in senso concreto, perché mi ero
perso in un qualche vago panorama di pensiero senza un magnete o una
bussola, intrappolato fra la mia famiglia, la Cappellania, Demetri, Cat,
i miei amici mirmidoni, Valka, il Casato Mataro e gli Umandh. Il mio
mondo si era fatto così grande, e io ero così piccolo. Non potevo dirle
niente di tutto questo, non potevo essere me stesso come lo ero stato a
Ulakiel, non con le videocamere che ci osservavano. «Ti piace qui?»
chiesi invece.

«Mmm?» rispose attraverso il naso, mentre beveva. «Su Emesh?»

I miei capelli, più scuri dei suoi, mi ricaddero sulla faccia quando
scossi il capo. «A Borosevo, nel castello.» Nel parlare battei un
colpetto sul bracciolo del divano per enfatizzarne la presenza
materiale.

Lei bevve un lungo sorso d'acqua con lo sguardo dorato che saettava da
un lato. «Le cose sono molto diverse da casa mia.»

«Anche dalla mia» replicai, non sapendo se era del tutto vero. «Vuoi
tornare indietro?»

Valka sorrise. «Per gli dèi, no. Gli xenobiti sono tutti qui fuori.»

«Non ce ne sono su Tavros?» chiesi.

«Pochi» disse, posando il bicchiere sul davanzale, accanto a sé. «Alcuni
che gli Extrasolari hanno portato là prima che i clan li scacciassero,
ma sono... socializzati. Non alieni quanto è possibile esserlo per degli
alieni, e nella Demarchia non c'è niente come Calagah. Niente di...
antico. È come... come...» Si interruppe, massaggiandosi gli occhi.
«Devi esserci stato per capire. Le rovine sono così antiche, più antiche
di qualsiasi cosa che abbiamo costruito. Ti fanno sentire... piccolo.
Fanno sentire piccoli tutti noi.»

Non risposi, non potevo farlo. Era la contestualizzazione, il modo in
cui lei posizionava l'umanità in mezzo alle creature dello spazio e non
al di sopra di esse che attirava le ire della Cappellania. Lei era una
Tavrosiana -- un incubo diplomatico, dato che nella Demarchia tutte le
persone di età adulta controllavano quote ponderate nel loro elettorato.
Lei era tanto una privata cittadina quanto un dignitario straniero e
accusarla di eresia sarebbe equivalso a dichiarare guerra alla sua
nazione, una guerra che la Contea di {Emesh} non voleva e non si poteva
permettere. Forse era per questo che pensavo a lei come a una persona
sola, una persona trasformata in uno Stato, e quello Stato incarnato
nella sua persona. Era quasi come un palatino, anche se non lo avrebbe
mai ammesso.

Contorcere la mia risposta in qualcosa di accettabile per le videocamere
richiese un certo sforzo, ma riuscii a dire: «So cosa intendi. È un
grande universo e perfino le nostre grandi imprese a volte sembrano
umili. Non umili quanto gli Umandh, naturalmente, ma comunque umili.»

«Gli Umandh?» ripeté. Le sue sopracciglia si aggrottarono, poi
saettarono verso l'alto. «Oh, sì.»

«Strano che non abbiano realizzato di più in tutte le loro migliaia di
anni» osservai, scorgendo un modo di esporre la verità senza dirla. «È
come sostiene Filemone di Neruda, il linguaggio è necessario per lo
sviluppo della civiltà. Se quello che dici è vero, il canto degli
Umandh... è poco meglio di quello degli antichi delfini.»

Lei mi studiò per un lungo momento. Sapevo che avrebbe ricordato la
nostra precedente conversazione non registrata su tor Filemone. Una luce
sottile brillò in quegli occhi dorati, accesi da un pallido dolore. «I
delfini?» Considerò la cosa per un lungo momento, riflettendo su quella
specie estinta da tempo. «È un valido paragone. Sono più intelligenti
dei delfini, ma forse solo perché possono usare attrezzi. Sai...»

Lettore, hai mai visto qualcuno parlare di qualcosa che lo consuma? Che
lo accende fin dalle fondamenta della sua anima? Valka parlava con tanto
fervore che per un attimo mi dimenticai di chi fossi. Qualsiasi
animosità potesse aver provato nei miei confronti nel corso del nostro
primo incontro pareva essere evaporata quasi tutta, trasformandosi in un
esitante rispetto nei miei confronti e in quelli della mia situazione. E
io? Io la temevo, temevo quello che rappresentava e il fatto che mi
importava di quello che pensava di me. Temevo i segreti che mi
costringevano a tenere, il mio nome, il mio sangue. Temevo che mi
avrebbe considerato falso e avrebbe pensato che il mio rispetto nei suoi
confronti fosse finto, mentre era la cosa più vera fra quelle che le
avevo mostrato. È così che veniamo tutti distrutti da quelle cose che
hanno importanza per noi, come lei ne aveva per me, nella mia
solitudine.

Alla fine interruppi la sua dissertazione − troppo bruscamente, posso
ancora sentire il mio tono acuto -- e chiesi: «Dottoressa, hai
mangiato?»

Lei si illuminò. «No, ti andrebbe di farlo?»

\begin{figure}
	\centering
	\def\svgwidth{\columnwidth}
	\scalebox{0.2}{\input{divisore.pdf_tex}}
\end{figure}

Quello fu il primo di molti pasti che condividemmo, a Borosevo e in
seguito. Potevo sentire che l'atteggiamento di Valka nei miei confronti
stava cambiando. Non ero più soltanto il barbaro, il macellaio
dell'arena. Non so dire quando fosse cominciato quel cambiamento -- se
fosse stato a Ulakiel o in seguito -- ma quando tornammo alle sue
stanze, quella sera, mi lasciò con un sorriso e una parola dolce, una
promessa che l'indomani avremmo parlato ancora.


