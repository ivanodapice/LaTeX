\chapter{Il satrapo e il maestro di spada}

«Mia cara satrapo!» esclamò sir Elomas quando gli araldi ebbero finito
di annunciare i nostri augusti ospiti, sfoggiando il suo più profondo ed
elegante inchino. «Benvenuta a Calagah! Mi onori con questa visita.» Si
raddrizzò e si premette contro il petto il cappello di paglia. «Invero è
un grande onore che durante il tuo soggiorno ti sia presa del tempo per
vedere questo nostro piccolo buco.»

Il contingente jaddiano atteso così lungo era finalmente arrivato.

La donna a cui Elomas si era rivolto -- lady Kalima Aliarada Udiri di
Sayyph, satrapo di Ubar e ambasciatrice per conto del principe stesso --
era alta quasi sette piedi, con una pelle oleata color rame e occhi
scuri che ne facevano la personificazione della purezza genetica
jaddiana. Di certo era uno dei \emph{eali al'aqran}, i purosangue di
Jadd, più palatini dei palatini stessi. Lo si poteva percepire nel modo
in cui i suoi piedi disprezzavano il terreno stesso su cui camminava,
nel modo sottile in cui levò il alto il naso mentre Elomas parlava, nel
portamento teso delle sue spalle sotto lo scialle di seta. L'abito era
di sciamito, in terra d'ombra e oro, come d'oro erano le catene che le
decoravano la gola e la fronte, e le gocciolavano dalle orecchie. I
capelli nerissimi ricadevano in una spessa treccia mista a cordoni
dorati e decorata con piccoli phalerae che scintillavano come stelle.

Dietro di lei procedeva una doppia colonna di mamelucchi jaddiani con
l'armatura che brillava lucida sotto il caffettano a strisce blu e
zafferano, e il volto nascosto da un profondo cappuccio. Si muovevano in
silenzio, marciando alla perfezione in ranghi serrati. Subito dopo di
loro intravidi Anaïs Mataro, affiancata da littori che portavano i
colori verde e oro del suo Casato. Mi sorrise, e io le risposi con un
cenno del capo perché dopo l'incidente nel tunnel e la reazione di Valka
non me la sentivo più di sorridere.

La deferenza della satrapo nei confronti di Elomas si limitò
all'inarcarsi di un singolo sopracciglio.

«Spero che sia molto più di un buco, in modo che questo non sia stato
uno spreco di tempo.» A parlare non era stata la satrapo ma lo spettrale
uomo vestito di nero al suo fianco. Il volto del maestro di spada era
impassibile, ma i suoi occhi nerissimi sorridevano. Mentre parlava la
sua mano guantata giocherellava con il fermaglio d'oro della cintura che
tratteneva la sua mandayas cerimoniale, la mezza veste del suo ordine.
L'indumento gli copriva del tutto il lato sinistro del corpo, ricadendo
fino alla caviglia nel pendere morbidamente dalla spalla sinistra e
seppellirgli il braccio nelle sue pieghe di tessuto del colore che ha il
sangue di notte. «Tu sei sir Elomas Redgrave?»

«Sì, sir...?»

«Olorin» si presentò lo spadaccino. «Olorin Milta.»

Quel suo parlare fuori luogo mi sorprese. Poteva anche essere un maestro
di spada, uno dei maeskoloi, ma non avrebbe dovuto parlare prima della
sua padrona e protetta, che però non lo rimproverò. «Perdona il mio
servitore» si limitò a dire. «Olorin non... non è abituato al silenzio.»
Un sorriso attraversò fugace il volto del maestro di spada che inarcò un
sopracciglio come aveva fatto la sua signora. «Ci piacerebbe vedere
queste grotte. Il conte le ha definite una meraviglia di questo pianeta»
continuò lady Kalima, in un tono da cui si capiva quanto poco questo
significasse per lei. Non potevo biasimarla perché intuivo che avesse
visto soltanto Borosevo, che non poteva dare una prima impressione molto
entusiasmante per uno dei famosi sibariti \emph{eali}. Canali fetidi e
la puzza di pesce marcio e di alghe morenti non erano niente a confronto
delle torri e delle terrazze di cristallo di Jadd, dei giardini del
principe Aldia di Otranto e dei bastioni di arenaria della Scuola di
Fuoco.

Elomas agitò una mano. «Sciocchezze, sciocchezze. Non c'è niente da
perdonare. Venite, amici miei, venite. Naturalmente gradirete un
rinfresco e anche se non abbiamo molto facciamo un po' di viticultura
sulle terre di mio nipote, nelle terre occidentali. {Dovete} assaggiare
un bicchiere del suo vino! Ho una bottiglia di vino dorato dolce come un
bacio, vedrete.» Mentre parlava si adeguò con disinvoltura all'andatura
della satrapo, facendola proseguire e inducendo il resto di noi ad
accodarci a lui.

«Signore, ho paura che non ci possiamo trattenere» ribatté la dama,
agitando la spessa treccia con un tintinnare delle catene d'oro che
aveva alla gola e sulla fronte. «Abbiamo solo tre ore da passare qui,
quindi vogliamo andare alle rovine? Mi piacerebbe vedere questa
meraviglia di cui parla lord Balian.»

Dall'abbassarsi della voce di Elomas compresi che la sua espressione si
era fatta avvilita. «Sì, ecco... sì, certamente. Da questa parte.»

Il fato mi portò spalla a spalla con Anaïs, che infilò il braccio sotto
il mio. «È bello rivederti, lord Marlowe» disse, e dopo che ebbi
ricambiato il complimento, continuò. «Mi è dato di capire che presto ci
sposeremo.»

«Ho sentito la stessa cosa.» Oh, dèi, ne stava parlando apertamente.
«Pare però che sarà solo fra alcuni anni. Prima devi superare il tuo
Efebeia.» Non desideravo niente più dello sprofondare attraverso la
terra, sciogliermi e fluire verso il mare. È così che si deve essere
sentita Kyra, pensai, ben sapendo che non potevo fuggire.

Anaïs mi strinse il braccio e si fece un po' più vicina.
«\emph{Tornerai} presto a Borosevo, vero? Ligeia non oserà vendicarsi
per Gilliam, so che non lo farà.» Si protese un po' di più verso di me e
il suo respiro mi investì rovente la guancia quando sussurrò: «La farei
uccidere se ci provasse.» Potevo sentire sulla nuca lo sguardo di Valka,
ma continuai a camminare.

Riluttante a discutere o a farle notare i problemi connessi a
quell'affermazione risposi con un debole sorriso, intrappolato com'ero.
«Tornerò presto, ne sono certo, non appena il lord tuo padre riterrà che
sia sicuro farlo.»

Davanti a noi le forze di sicurezza della satrapo procedevano attraverso
la distesa rocciosa di basalto coperto di muschio, scandagliando il
paesaggio con i sensori e tenendo pronti gli storditori. Riuscivo a
stento a intravedere la cortina di energia dello scudo di lady Kalima,
come pure di quello del maeskolos che era la sua guardia del corpo. Il
mandayas del maestro di spada si apriva sotto il soffio del vento,
dandogli l'aspetto dell'eroe di una qualche opera drammatica. Portava
l'indumento di traverso, come decretava la tradizione del suo ordine, e
un cordone di seta era legato sotto il braccio destro per impedire che
scivolasse. La voluminosa manica squadrata si agitava vuota e il braccio
sinistro gli pendeva davanti come se fosse appeso al collo, con il bordo
dell'indumento di seta rossa che quasi strisciava sulla pietra. Per
contro il braccio destro -- e tutto il resto di lui -- era avvolto in
cuoio nero lucidato fino a brillare in modo cupo e tanto aderente da
calzargli come un fodero fa con uno stocco. «Ti piace il maeskolos?»
chiesi ad Anaïs. «Lo stai fissando.»

«Anche tu» ribatté, schivando la domanda. Perché le avevo chiesto
proprio questo, fra tutte le dannate cose possibili? Ero geloso? Dèi,
era gelosia? Ma se non mi importava di Anaïs, perché avrei dovuto essere
geloso?

«Ecco,» ribattei, cercando una risposta appropriata «è un maeskolos,
giusto?» Era uno dei famosi maestri di spada di Jadd, i più grandi
combattenti dell'universo umano. I candidati alla Scuola del Fuoco
venivano accolti da ragazzi e scelti esclusivamente fra gli \emph{eali}
purosangue, per essere assoggettati a quasi un secolo standard di
addestramento. Non c'è bisogno che elargisca storie della loro abilità
perché le loro lodi vengono cantate spesso e a gran voce dovunque
respirino esseri umani. Si dice che uno dei maeskoloi possa affrontare
cento legionari e sopravvivere, e che si muovono tanto in fretta da
sconvolgere l'aria stessa. Molto tempo dopo avrei visto una di loro
sconfiggere tre delle guardie marziane dell'imperatore senza neppure
estrarre la spada. In loro c'era un certo fascino, un'attrattiva che
trascendeva le caratteristiche personali.

«Lo è» convenne Anaïs. Stavamo scendendo nella fenditura lungo la
sferragliante scala di metallo, camminando in coppie di due con i piedi
che risuonavano sui gradini di metallo. «Vorrei che tornassi indietro
con noi.»

«Sai che non lo posso fare» replicai, guardando in direzione del
pianerottolo dove il maeskolos stava seguendo sir Elomas e la satrapo al
centro di un capannello di guardie jaddiane. Riuscii a stento a non
digrignare i denti. Non volevo tornare indietro. Anche se non riuscivo a
vedere nessuna alternativa, non riuscivo a rassegnarmi al mio destino.
Mi sentivo \emph{usato} dal conte, poco meglio di un animale.
Rabbrividii. «È un bene che la flotta jaddiana sia arrivata qui proprio
ora. A quanto pare le Forze Difensive di Emesh si aspettano un altro
attacco. Si fermeranno qui a lungo?» Anaïs non rispose immediatamente e
le lanciai un'occhiata in tralice mentre percorrevamo la passerella di
plastica che avevano eretto fra la base della scala e le colonne e i
gradini di Calagah. Con mia sorpresa, Anaïs pareva seria, ripiegata in
sé stessa, con la snella figura vestita di seta verde china su sé
stessa. Andando contro il mio buonsenso le posai una mano sulla spalla.
«Stai bene?»

La figlia del conte -- un giorno contessa -- mi sfiorò la mano con la
guancia. «Pensavo che la guerra sarebbe rimasta lontano... molto
lontano, sai? Non sembra reale.»

Nonostante il vuoto che avevo nel petto dove ci sarebbero dovuti essere
i sentimenti non ero tanto freddo da ritrarre la mano. Cosa le potevo
dire? Guardai in direzione della parete di basalto dell'altura, delle
colonne di pietra nera che artigliavano blasfeme il cielo e che mi
ricordarono involontariamente le torri della mia patria, facendomi
avvertire in parte il freddo che Anaïs sembrava sentire.

«\emph{Dolá Deu di Fotí}!» L'imprecazione religiosa jaddiana distolse di
scatto la mia attenzione da Anaïs per spostarla sul resto del gruppo.
Sorpresi uno dei mamelucchi che mi osservava da sotto il profondo
cappuccio blu. La sua visiera era una sorta di specchio cromato
modellato a imitazione di una faccia umana e scorsi un'immagine distorta
di me stesso riflessa nelle sue curve. In quel mamelucco c'era qualcosa
di profondamente inquietante che mi strappò un brivido, e nascosi il
gesto assestandomi la corta giacca. A parlare era stata lady Kalima.
«Che vista incredibile!» Accanto a lei, la guardia del corpo rimase in
silenzio con la mano su una delle tre -- \emph{tre} -- spade ad
altamateria che portava alla cintura.

La dottoressa Onderra cominciò la sua esperta esposizione storica
relativa al sito, editando con cura qualsiasi menzione eretica sulle
similitudini con siti su altri pianeti. La osservai mentre parlava e
indicava piccole caratteristiche dell'architettura aliena. Il suo
sguardo scivolò su di me, mi sorprese a fissarla e un angolo della sua
bocca si contrasse, facendomi arrossire. Dovetti distogliere lo sguardo,
tanto da lei quanto da Anaïs.

«Mi chiedo che sorta di creature abbiano costruito questo posto»
commentò la satrapo, con la sua voce roca che rimbombò contro le lucide
colonne e i pinnacoli di basalto. «I vostri coloni?»

«Gli Umandh?» Tor Ada scosse il capo. «Loro...»

«Non siamo del tutto certi della provenienza del sito» intervenne sir
Elomas mentre saliva i bassi gradini che portavano all'ingresso del
labirinto nero. «Le prove suggeriscono che Emesh abbia forse ospitato
una specie precedente di vita intelligente, \emph{prima} degli Umandh.
Ma venite, mettiamoci al riparo dal vento!» Intercettai lo sguardo
dell'imbarazzata scoliasta e le rivolsi un sorriso comprensivo.

«È possibile?»

Tor Ada si adeguò a quella linea di esposizione con guardinga
riluttanza. «Parecchi mondi hanno nella loro storia eventi che hanno
portato a un'estinzione di massa, Vostra signoria. Si dice che perfino
la Terra fosse un tempo dominata dai draghi, prima dell'avvento della
razza umana.» Non avevo mai sentito prima una cosa del genere, ma doveva
essere vera perché la satrapo annuì con un tintinnare di catene e
precedette il gruppo all'interno, sgomentando due dei suoi mamelucchi
che si affrettarono a oltrepassarla per controllare la foresta di
colonne inclinate e accertarsi che non ci fossero assassini. Li osservai
setacciare l'ambiente, e nel farlo compresi infine cosa mi inquietasse
tanto in loro: avevano gli arti troppo sottili, troppo lunghi, tanto che
mi ricordavano più quelli dei Cielcin che degli uomini mortali. Avevo
sentito dire che gli jaddiani clonavano i loro soldati e avevo sempre
creduto che si trattasse di uomini e non di omuncoli.

Finalmente penetrammo nelle profondità delle rovine, oltrepassando la
sala da cui si diramavano le diverse uscite e scendendo lungo il
graduale pendio della galleria principale fino all'oscurità, alleviata
solo dal nastro fosforescente, di una stanza che Ada ed Elomas avevano
battezzato `il sepolcro', anche se non avevamo nessuna prova che fosse
tale. Laddove le gallerie erano un dedalo stretto e dalla volta bassa,
il sepolcro si allargava arioso nell'oscurità, traforato da aggraziate
colonne che andavano dal pavimento e dalle pareti al soffitto come il
tessuto striato dei polmoni scomparendo nell'assoluta oscurità di punti
che la luce delle sfere luminose non era in grado di raggiungere.

«I nativi sono poco più che gazze ladre, amici miei» stava dicendo
Elomas. «Avete visto i loro tuguri? Mucchi di rifiuti, tutti quanti.»

«Sono lieta che ci siamo presi il tempo per venire a vedere tutto
questo» commentò la satrapo, guardandosi intorno con ammirazione. Non
era un mero complimento di circostanza, anche attraverso il suo marcato
accento cantilenante si capiva che era molto colpita.

Valka avanzò nella luce, con lo sguardo che tipicamente -- e
impudentemente -- non era rivolto alla figura del dignitario straniero
ma alla struttura in fondo alla camera. «È bene ricordare che siamo solo
una piccola parte dell'universo e non il suo centro» affermò, con il suo
sorriso tagliente, aspettando che qualcuno la contraddicesse.

Era chiaro che qualcuno al seguito della satrapo -- forse la stessa
Anaïs -- l'aveva avvertita sul conto della dottoressa, perché lady
Kalima non parve offesa dalle parole di Valka. Anche se su Jadd il
potere della Cappellania variava da principato a principato, la dottrina
della Supremazia dell'Uomo -- del nostro destino manifesto -- non aveva
mai cedimenti. Un tempo gli Jaddiani erano stati Solari e rimanevano
tali nella loro convinzione che le stelle fossero il nostro dominio.
Lungi dall'incarnare un adeguato rispetto per il controllo delle stelle
da parte dell'uomo, Valka diceva spesso che non eravamo più speciali dei
coloni o dei Cielcin, non più del bestiame e dei pesci, degli uccelli,
degli ornithon e dei congrid del mare di Emesh. Eravamo un animale in
mezzo ad altri animali; \emph{animal inter animalia}. Io, che un tempo
ho distrutto una stella, non trovo nessuna verità in questo concetto.
Invero, la disponibilità di Valka a svilire tanto spesso il posto
dell'uomo nell'universo mi sembrava generata più dalla misantropia che
dall'umiltà. Abbiamo un posto in questo universo, anche se ce lo
dobbiamo creare noi stessi.

Invece di replicare e di avviare una discussione con il modo di pensare
chiaramente eretico di Valka, lady Kalima disse: «Mi ricorda l'interno
delle navi dei Cielcin. Non è così, Olorin?» Nel parlare si girò verso
il suo compagno con lo sguardo ancora perso nella confusione geometrica
che ci sovrastava. Poi la sua attenzione si spostò gradualmente
sull'altare e sulla grande massa sospesa su di esso, ora scheggiata dove
la pressione del tempo geologico {aveva} fratturato il materiale vetroso
sotto il peso incombente della roccia ignea che gravava sul sepolcro. La
stanza aveva la forma di una serratura, con la struttura che Ada e Valka
definivano `altare' al suo centro, una lastra di pietra alta fino alla
vita e abbastanza larga perché due uomini vi si potessero sdraiare
sopra, sotto un dito della stessa pietra nera che scendeva dal soffitto
come l'ugola di un gigante addormentato.

Quella domanda strappò la mia attenzione da quell'incombente massa di
pietra. «Sei stata a bordo di una delle loro navi?»

Sir Olorin -- infatti era un cavaliere -- si girò a guardarmi, poi
spostò lo sguardo su Elomas. «Chi è costui?»

Mi inchinai e gli risposi in jaddiano. «Maeskolos, mi onoro di
informarti che il mio nome è Hadrian.»

\emph{«Ay ya}!» Il maestro di spada si batté un colpetto sulla coscia
protetta dall'armatura in segno di riconoscimento o di applauso. «Il
duellante! Mi ero chiesto che ne avessero fatto di te.» Si avvicinò di
un passo come per vedermi meglio, poi rispose alla domanda della sua
signora. «In effetti mi ricorda le navi cielcin.»

Valka emise un udibile verso di disapprovazione, per nulla colpita dai
dignitari stranieri, e io dovetti ricordare a me stesso che lo era anche
lei. «Impossibile. Queste rovine hanno quasi un milione di anni mentre i
Cielcin sono una civiltà che viaggia nello spazio da meno tempo di noi.»

Ignorandola mi accigliai e rivolsi lo sguardo in alto e nell'oscurità di
quella camera simile a una cripta, desiderando che avessero portato più
luci. «Dici che tutto questo somiglia alle navi dei Cielcin?»

«Superficialmente, anche se forse si tratta solo della natura simile a
una grotta di questo posto.» La satrapo continuò a girare in tondo,
ammirando le rovine.

«Non apprezzavano molto la luce, vero?» commentò il maestro di spada con
un verso di riprovazione.

«Forse i costruttori erano ciechi» ipotizzò Valka. «Gli Umandh lo sono.»

«Credevo avessi detto che non erano Umandh» ribatté sir Olorin,
controllando con la destra la posizione dell'impugnatura delle spade
contro il fianco. Nella penombra, il chiarore delle sfere luminose si
increspava sulle cortine di energia della barriera del suo scudo. Pensai
che era strano che solo lui, fra tutti i soldati, ne indossasse uno.
Osservai i mamelucchi nell'aria ombrosa, con le loro vesti blu e
arancioni lisce e lucenti nella scarsa illuminazione e mi parve che non
respirassero. Rabbrividii.

«Non abbiamo nessuna prova fossile che dia un'immagine di com'erano i
costruttori» affermò con semplicità Valka. «Supponiamo soltanto che
potessero essere \emph{come} gli Umandh perché dopotutto...» Fece una
pausa, come per prepararsi a mentire... «Sono derivati dal loro stesso
potenziale genetico, qui su Emesh.» E agitò una mano, come ad
abbracciare il mondo intero con quel gesto.

«\emph{Badonna}, posso farti una domanda?» chiesi alla satrapo, tenendo
lo sguardo adeguatamente basso. «Hai detto che questo posto ti ricorda
le navi dei Cielcin. Quando ne avete catturata una?»

«Alcuni anni fa» rispose lady Kalima, senza guardarmi, cosa che era solo
prevedibile da parte di un membro della nobiltà jaddiana \emph{eali},
più palatina dei palatini stessi. Avrei potuto benissimo essere un
moscerino. «Abbiamo catturato un esploratore avanzato a Obatala. Era...
all'interno era come una grotta. Molto buio. Molto simile a questo.»

Obatala. Quello era stato uno dei pianeti lungo la rotta di Demetri fra
Delos e Teukros, giusto? Questo mi distrasse al punto che quasi persi la
possibilità di replicare. Avevo dimenticato la scomparsa del mercante
jaddiano e del suo equipaggio a causa di tutto quello che era successo e
che lo aveva seppellito sotto anni di povertà e il contratto con l'arena
di Borosevo. Se c'erano stati i Cielcin a Obatala, allora forse ce
n'erano stati anche nel Buio circostante. Comunque ricacciai
quell'antica preoccupazione al suo posto per chiedere: «Quindi adesso
attraverserai il Velo in direzione dello spazio normanno?»

Fu sir Olorin a rispondere. «Siamo diretti al fronte, al di là di
Marinus e al confine estremo del Velo. Questa doveva essere la nostra
ultima fermata prima dell'attraversamento finale.» Ubar. Obatala. Emesh.
Marinus. Cercai di visualizzare la distribuzione di quei mondi sullo
sfondo nero di quella camera, come se fosse il cielo. Gli Jaddiani
avevano descritto un grande arco, partendo dalla loro terra remota al
limitare della galassia e tracciando una spirale verso il suo interno e
attraverso il vecchio Impero in direzione della frontiera normanna,
vicino al nucleo. Olorin continuò a parlare. «A quanto mi risulta, di
recente il vostro mondo è stato tanto fortunato da respingere un
attacco.» Erano passati solo un paio di mesi da quando avevamo avuto
notizia della fallita razzia dei Cielcin ai confini del sistema di
Emesh.

«Di certo non sarete qui per via di quell'attacco, vero?» chiese Anaïs.
«È decisamente troppo presto, giusto?»

«Infatti.» Il maeskolos infilò il braccio sinistro attraverso la mezza
veste che portava sopra la tenuta nera da combattimento e incrociò le
braccia, apparentemente a indicare che quella conversazione era
conclusa. «Perdonami, ma questo posto è... non abbiamo niente di simile
nel nostro angolo della galassia. Siete davvero benedetti, lady Anaïs,
sir Elomas, a vivere su un mondo che ospita simili meraviglie.» Sollevò
lo sguardo sull'enorme, crepata lingua di roccia sospesa al di sopra
dell'altare nel centro della stanza e scosse il capo.

Lady Kalima venne avanti. «Adesso però dobbiamo tornare nella capitale.
Questa piccola avventura è stata... quanto mai gratificante. Grazie.»

«Dovete proprio?» Elomas appariva sgomento, ma la mia attenzione era
rivolta altrove, concentrata sulla prosciugante oscurità che ci
circondava e sovrastava. «Sei certa che non possa indurti ad assaggiare
quel vino?»

\begin{figure}
	\centering
	\def\svgwidth{\columnwidth}
	\scalebox{0.2}{\input{divisore.pdf_tex}}
\end{figure}

Nel buio nessuno si accorse che indugiai nel sepolcro come fa un
penitente molto tempo dopo che nel tempio gli è stata data
l'assoluzione. Gli Jaddiani erano il terzo gruppo di dignitari stranieri
che visitavano Calagah dal nostro arrivo anche se erano di gran lunga i
più importanti. Anche se si è parlato molto del mio interesse per gli
xenobiti, esso non è assolutamente una ossessione rara. Un giovane
direttore del Gruppo Izumo era venuto con la sua famiglia per una visita
di un intero giorno prima dell'incursione dei Cielcin ai confini del
sistema, e dopo di lui c'era stato un gruppo di prospettori normanni che
però erano risultati essere poco meglio che depredatori di tombe. Ero
stanco dei convenevoli e speravo che rimanendo a tenere il broncio nella
grotta avrei potuto risparmiarmi i commiati. Nonostante la mia breve
interazione con il maestro di spada non mi aspettavo che si sentisse la
mia mancanza, e comunque capitava di rado che avessi modo di passare del
tempo da solo nelle rovine. Mi parve quasi di sentire su di me degli
sguardi che mi fissavano dalla pietra nera e nel ricordare le visioni
che avevo avuto rabbrividii. Non ne avrei dovuto parlare con Valka. Da
settimane mi rivolgeva a stento la parola, e quando lo faceva era con
una tesa cortesia e una rigida formalità che parevano derivare in pari
misura dall'imbarazzo e dal disprezzo. `Perché sei un selvaggio
ignorante di un Paese retrogrado che crede ancora nelle favole.'

«Perché sono un idiota che non sa quando tenere la bocca chiusa...»
borbottai.

Sollevai la mano, con le dita protese a serrare un supporto di roccia
largo quanto un braccio che formava un arco fra la parete e il soffitto,
diramandosi in altri due più avanti e andando a unirsi al groviglio di
supporti simili a rami d'albero che sorreggevano la volta. Era del tutto
insignificante. Non era fredda, era solo pietra, come lo era stata da
quando avevo avuto la visione.

Mi accigliai.

«Credevo ti fossi perso.»

Una voce femminile interruppe le mie riflessioni e mi girai con un
sorriso. «No, io...» Non era Valka. «Salve.»

Anaïs Mataro si soffermò sulla soglia, un'esile figura sullo sfondo
dell'oscurità, con i ricciuti capelli setosi che formavano una nuvola
intorno alla testa nel ricaderle giù per le spalle. «Il cavaliere tuo
amico sta consumando le orecchie degli Jaddiani a forza di chiacchiere.
Ho pensato di venirti a cercare.» Le sue scarpe a suola rigida
ticchettarono sonoramente sul pavimento di pietra e le sue vesti
frusciarono mentre superava la distanza che ci divideva. «Sei certo di
non poter tornare indietro oggi, con noi?»

«E far aspettare gli Jaddiani mentre preparo i bagagli?» Era uno scudo
di carta. «Non voglio dipendere dalla loro carità, e tu?»

Sorrise, con i denti bianchi come latte nella luce scarsa. «No, suppongo
di no. Comunque, vorrei che tornassi.»

Ricambiai il suo sorriso con uno più sottile e annuii. Voleva che mi
dicessi d'accordo con lei, che affermassi di desiderare anch'io di
essere a Borosevo, ma non le avrei mentito. «A dire il vero, attualmente
Borosevo potrebbe non essere tanto sicuro. Se i Cielcin stanno
arrivando, intendo.»

«Pensi che sia così?»

«I nostri amici jaddiani sembrano pensarlo.» Le volsi le spalle,
girandomi di nuovo verso la massa nera dell'altare. «E dopo quel primo
attacco puoi sostenere onestamente di non credergli?» Dèi, faceva
davvero freddo in quelle gallerie, in quella camera, nell'oscurità sotto
il mondo, e il peso di tutta quella roccia schiacciava la mia anima in
uno spazio di molte taglie troppo piccolo. I Cielcin sembravano così
vicini, minacciosi quanto le colonne storte di quel labirinto alieno.
Ombre nella mente.

Lei mi si avvicinò e mi circondò con le braccia sottili. Non parlò, ma
mi tenne stretto. Era più alta di me, quindi premette la guancia contro
la mia nuca. Era bello essere abbracciato -- era passato così tanto
tempo dall'ultima volta -- quindi non opposi resistenza, non subito. Poi
il volto di Cat affiorò nel buio, con la pelle scura pallida, consumata
dalla pestilenza. Sussultando, mi tolsi di dosso le braccia di Anaïs.
Quella era la ragazza a cui sarei stato venduto.

«Lasciami andare, Anaïs.»

Lei sussultò ma non si ritrasse. La distanza fra di noi poteva essere
misurata in anni luce, in parsec. «Mio padre mi ha detto del suo
piano...» Serrai le labbra, non sapendo cosa fare. Adesso ero certo che
fosse questo che Kyra aveva provato. «Di come diventerai il mio
consorte.»

«Il tuo uomo da riproduzione, vuoi dire.»

«È tutto quello che ci vedi?» sussurrò, parole che mi alitarono calde
sulla nuca. «Non deve essere così, sai? Potremmo stare... bene. Questo
potrebbe essere un bene per Emesh... per i nostri figli.»

«I nostri figli...?» Quel pensiero non voleva fare presa. Figli. Cosa
potevo dire? Potevano prelevare i loro campioni genetici da me che io lo
volessi o meno. Potevo vivere i miei giorni in una torre, o in un
palazzo estivo, come mia madre, ma non avevo potere, nessuna libertà di
scelta in quella situazione. I muscoli mi si fecero di marmo sotto la
pelle e rimasi immobile come una di quelle colonne deformi che
sostenevano l'oscurità sovrastante. «Anaïs, non riesco a parlarne
adesso. Mio padre mi ha scacciato, tuo padre mi ha fregato per... per le
mie cellule, come se fossi una sorta di cavallo da corsa.»

Lei non rispose, si limitò a stringermi maggiormente. Stava tremando?
Aveva paura? O si trattava solo di me? Senza parlare si protese a
toccarmi la faccia con una mano per farmi girare. Pesante come piombo,
mi voltai a fronteggiarla, a guardarla in faccia nella penombra, ed
esitai sul punto di dire altro, aprendo la bocca per cercare le parole.

Lei mi baciò.

Mi paralizzai.

Nel gelo della caverna lei era calda, disponibile... e io non la volevo.
«Non è così» sussurrò.

La allontanai della lunghezza delle mie braccia. «È proprio così!»

«Ma sarai il signore di tutto Emesh, al mio fianco. Lo immagini?» Potevo
immaginarlo e glielo dissi. Il potere però è come il magnetismo;
funziona in due direzioni, respingendo nella stessa misura in cui
attrae. Mio padre e mia madre erano prigionieri della loro posizione e
del loro sangue, non potevano scegliere, e anch'io ero impotente, e alla
mercé delle attenzioni di Anaïs. Mi riaffiorò nella mente il ricordo di
quella notte senza luna a Borosevo, degli uomini che, ridendo, mi
avevano trascinato fuori dal mio tugurio. Chiusi gli occhi, imponendo al
ricordo di andarsene.

\emph{Non voglio questo}. Era ciò che intendevo dire, ma invece sentii
una voce identica alla mia affermare: «Che gran giorno sarà.» Che altro
potevo dire? Come un soldato davanti al suo legato, un marinaio davanti
al suo capitano e Kyra davanti a me, ero impotente di fronte a quella
ragazza e alle macchinazioni che rappresentava. Sarei stato suo, e
niente che avessi potuto fare avrebbe cambiato le cose. «Però non credo
che la tua corte mi accetterà, non dopo Gilliam...»

Si strinse contro di me, con la faccia annidata nel cavo del mio collo,
e potei sentire il mio corpo che reagiva, tradendomi. Stavo per sentirmi
male. «Non voglio parlare di Gilliam. Li costringeremo ad accettarti.
Sono la \emph{mia} corte, è il \emph{mio} pianeta, il pianeta della mia
\emph{famiglia}. Gliela faremo vedere, tu e io.» Ero pietrificato e per
un momento dimenticai di muovermi quando le sue labbra si premettero di
nuovo sulle mie. Avevano il sapore della cenere.

«Hadrian, io... oh!»

Al suono della voce -- di \emph{quella} voce -- allontanai Anaïs da me,
sentendo il sangue che si era risvegliato salire e scurirmi il volto.
Anaïs trattenne il respiro e si girò con una risatina mentre il mio
cuore si faceva di vetro nella stessa misura in cui i miei muscoli erano
di marmo, infranto quando precipitò al suolo.

Valka era ferma sulla soglia del sepolcro, con la sua sagoma delineata
come lo era stata pochi momenti prima quella di Anaïs. Ancora adesso mi
chiedo se si accigliò o se il suo volto affilato sorrise per lo
sconcerto. «Dottoressa, io... Anaïs era venuta a cercarmi.»

«Questo lo posso vedere» commentò in tono malizioso la voce limpida di
Valka. «Gli Jaddiani la stanno aspettando. Venite.»

Intimidito, con il cuore dolente, deglutii a fatica e annuii. «Non è
quello che... non stavamo...»

«Non mi importa» tagliò corto Valka. Mi piace pensare che lo disse in
tono troppo tagliente, toppo in fretta. Poi però rise. «Forza, venite.»

