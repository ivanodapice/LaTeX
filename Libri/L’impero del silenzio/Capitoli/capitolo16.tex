\chapter{La madre}

Liliana Kephalos-Marlowe era nella sua cabina olografica e mi dava le
spalle mentre si muoveva nell'immagine di un duello spettrale, con una
penna luminosa in mano e un monocolo entottico applicato sull'occhio
sinistro. La cabina era un disco di un diametro di circa venti piedi,
identico a un altro sul soffitto che teneva al suo interno un mondo
tridimensionale. Lo spazio di lavoro di mia madre, racchiuso da una
parete di vetro all'estremità opposta e con una vista delle cupole e
delle esili torri del palazzo d'inverno, sembrava contenere una porzione
di prato erboso con una folla di spettatori in abito d'epoca che
assisteva alla morte di un antico moschettiere. Non poteva essere
tornata da molto ed era già di nuovo al lavoro. Non sapevo se ammirare
quella sua dedizione o odiarla per questo. Come mio padre, aveva così
poco tempo per i suoi figli.

Il servitore si inchinò, battendo i tacchi. «Hadrian Marlowe, Vostra
signoria.»

Mia madre si girò, inarcando un sopracciglio color bronzo finché il
monocolo non le uscì dall'occhio. «Eccoti qui!» Bloccò l'oscillare del
monocolo e lo ripose in un taschino della sua blusa azzurra, poi agitò
la penna luminosa in modo da bandire la nube caliginosa dell'ologramma
con uno scatto sommesso. Il prato e i moschettieri svanirono,
lasciandoci in un grigiore vuoto.

Mantenendomi eretto, strattonai la maglietta da corsa per assestarla.
«Eccomi qui? Madre, sono qui da giorni. Sai che partirò alla fine della
settimana?»

Un vago sorriso aleggiò su quel volto di porcellana. «Sì, lo so.» Si
rivolse al servitore. «Mikal, puoi lasciarci soli.» L'uomo si inchinò e
uscì, sigillando la porta alle sue spalle con un tonfo. Mia madre
sorrise e -- evocando inconsapevolmente Amleto -- disse: «Ora siamo
soli.» Incrociò quindi le braccia, esaminandomi con un'espressione che
non riuscii a decifrare del tutto, con le labbra contratte, le
sopracciglia aggrottate e gli occhi color ambra socchiusi. Se non fosse
stato per quel leggero movimento avrei creduto che fosse un altro
ologramma, un'immagine ricavata dalla luce come una statua lo è dal
bronzo. «Ti dispiacerebbe dirmi cosa, nel nome della Terra, credi di
fare?»

Sbattei le palpebre, sinceramente sorpreso, perché non mi ero aspettato
quell'approccio. Mi guardai intorno. «Cosa vuoi...»

«Non fare il finto tonto con me, Hadrian.» Si girò di scatto con la
veste verde e bronzo che si allargava a ventaglio, e attraversò la
piattaforma olografica in direzione di una credenza cosparsa degli
strumenti del suo mestiere. Notai un paio di pesanti occhiali entottici
come pure una console da computer in vecchio stile e un paio di tablet a
cristalli annidati in una postazione di ricarica accanto ai comandi
delle luci e di polarizzazione per le file di finestre. Liliana
Kephalos-Marlowe afferrò una striscia di nylon e tirò a sé una piccola
valigetta diplomatica del genere usato dai corrieri di massima sicurezza
dell'Impero. Senza cerimonie, con la mascella serrata, me la gettò, e io
l'afferrai agendo di riflesso. «Aprila.»

Lo feci, e per poco non lasciai cadere il pacchetto. Afferrandolo a
stento, sollevai lo sguardo sulla donna che mi aveva dato i suoi geni.
«Sei stata tu?» chiesi. «Come?»

«Ti tengo d'occhio» rispose con freddezza, azionando la console a parete
in modo che le finestre trasparenti si facessero opache, di un grigio
metallico, escludendo il mondo. «Soprattutto dopo l'incidente al
Colosso.»

Con cautela, rimossi l'oggetto dal fondo della valigetta, tirandolo
fuori come se fosse stato una vipera o una mano recisa. «Come ti sei
procurata questo, madre?» Naturalmente era il libro, quel piccolo volume
rilegato in pelle marrone che Gibson mi aveva dato quel giorno sulla
diga marina. \emph{Il re con diecimila occhi}, che si supponeva essere
l'autobiografia dell'antico pirata Kharn Sagara, re di Vorgossos. Lo
aprii e tirai fuori la busta gialla che Gibson aveva messo sotto la
copertina. Il mio nome era scritto sul pacchetto con la sua calligrafia
angolosa. Qualcuno lo aveva aperto e io sbirciai all'interno, riponendo
il libro sotto un braccio.

«Lui aveva elaborato dei piani con una degli scoliasti di lord Alban»
spiegò mia madre, facendosi un po' più vicina a me. «A quanto pare,
quella donna conosceva una nave mercantile che ti avrebbe portato a Nov
Senber, su Teukros.» Si accigliò. «Non era il piano migliore del mondo.
Puoi leggere tutto lì dentro.»

Cento piccole domande presero forma ed esplosero ribollenti dentro di
me, con la più importante in cima alle altre. «Mio padre come lo ha
scoperto?»

«Della lettera?» Mia madre sorrise. «Oh, Al non ne ha idea. La gente di
lord Alban ha avvertito il suo ufficio quando la sua scoliasta ha
inviato trasmissioni non autorizzate al mercantile in orbita alta, e il
piano è andato in pezzi all'altra estremità.» Mentre parlava riposi la
lettera nel romanzo, mentre gli intestini sembravano trasformarsi in
vermi di Gordian dentro di me. «Tuo padre sa che tu hai avuto a che fare
con il piano, ma pensa di aver vinto dopo...» Si interruppe e una strana
espressione annebbiò la severità aristocratica del suo volto. «A
proposito, mi dispiace per Gibson. So che eravate intimi.»

«Sai cosa gli è successo?»

Lei scosse il capo. «Lo hanno caricato su un mercantile diretto solo
l'imperatore sa dove. Tuo padre ha messo il suo nome sul manifesto di
nove navi che dovevano lasciare il sistema, e non le posso contattare
via onda finché non usciranno dalla curvatura... e anche allora dovrei
far autorizzare un'onda telegrafica da tuo padre o da mia madre.» Fece
una smorfia. Le onde telegrafiche erano costose ed erano accuratamente
monitorate dalla Cappellania della Terra perché erano manufatti
tecnologici pericolosi.

«Quindi è perduto.»

«È vivo, se può essere di aiuto» replicò mia madre. Non lo era. Mi
fissai i piedi e le scarpe da corsa autoallaccianti. Le parole mi
abbandonarono, ritirandosi attraverso le finestre opache e sopra le
torri e le cupole di vetro del palazzo per svanire nelle radure della
valle successiva. Poi successe qualcosa che non ho mai dimenticato e che
cambiò il mio mondo come se il passaggio di una cometa avesse alterato
la mia orbita. Mia madre mi circondò con le sue braccia profumate, senza
parlare. Rimasi paralizzato. Nei miei quasi venti anni standard di vita
nessuno dei miei genitori mi aveva mai dimostrato un grammo o un istante
di affetto fisico, ma quel singolo abbraccio compensava quasi tutto
questo. Per un tempo lunghissimo non mi mossi e fu solo con una sorta di
lentezza traumatica che la abbracciai a mia volta. Non piansi, non emisi
nessun suono.

«Ti voglio aiutare» disse mia madre.

La spinsi indietro, guardandola da vicino come non credo avessi mai
fatto. «Cosa intendi dire?» Mi guardai nervosamente intorno nella
stanza, individuando le videocamere in alto fra le lisce pareti di
metallo.

Accorgendosene, mia madre sorrise e si lisciò la blusa azzurra. «Qui
dentro le videocamere sono spente.» Ancora intontito, annuii e deglutii
a fatica, ma prima che potessi parlare lei continuò: «Non hai ancora
risposto alla mia domanda.»

«Quale?» Mi sentivo le ginocchia deboli e mi spostai per lasciarmi
cadere sul divano, accanto alla valigetta che aveva contenuto \emph{Il
	re con diecimila occhi}.

«Quella in cui mi spieghi cosa mai, nel nome della Terra, pensi di
fare.»

Rassicurato dalla sua promessa riguardo alle videocamere le dissi ogni
cosa. La mia paura della Cappellania, l'odio che nutrivo per essa, il
mio desiderio di essere uno scoliasta e di unirmi ai Corpi di
Spedizione. Lei sussultò quando le riferii che mio padre mi aveva
colpito e lo sguardo le si appannò quando le raccontai come Gibson era
stato trattato nella piazza di Julian. Mentre parlavo lei trovò uno
sgabello in un angolo lontano e lo spinse di fronte al divano su cui
sedevo. Quando ebbi finito compresse le labbra e si protese a prendermi
una mano, ripetendo le parole che avevo registrato a stento la prima
volta che le aveva dette. «Ti voglio aiutare.»

La petulanza giovanile fece schioccare la frusta dentro di me. «Come,
madre?» ribattei, secco. «Come? È finita, mio padre ha avuto quello che
voleva. Fra quattro giorni sarò su una nave diretta a Vesperad» In quel
momento mi tornò in mente la risatina di Crispin, che mi irritò.
`Anagnostico. Che strana parola.' Mi chiesi dove fosse Crispin in quel
momento e sperai che fosse molto lontano, fra le braccia di quella sua
ragazza dalla pelle blu e non a chiedersi perché non ero nel palazzo
principale. «Lui ha vinto. Ci vorrebbero giorni per elaborare un piano
di qualche tipo...»

Lei mi strinse la mano. «Ero a Euclide per rintracciare un libero
mercante, qualcuno che ti possa portare via dal pianeta.»

«Un libero mercante? Non sono migliori dei pirati. Non ci si può fidare
di gente del genere.»

Lei lasciò andare la mia mano e sollevò la sua in un gesto inteso a
calmarmi. «La direttrice Feng ha garantito personalmente per lui.»

Questo mi colse alla sprovvista. «La direttrice è ancora su Delos?»

Mia madre sorrise, sfregandosi il labbro inferiore con il pollice.
«Perché credi che fossi a Euclide, con tutti i posti dimenticati da dio
che ci sono nei domini di mia madre?» Me lo chiesi, perplesso anche per
l'espressione remota nello sguardo di mia madre. «No, quest'uomo è
affidabile. Uno Jaddiano. Ada dice che eseguiva i controlli orbitali su
Lothrian per i loro carichi più... sensibili.»

Inarcai le sopracciglia. «Ada?»

«La direttrice Feng» si corresse mia madre, distogliendo lo sguardo, poi
si alzò con disinvoltura e si diresse verso le finestre offuscate.

«Questo lo avevo capito» replicai. «Ma... `Ada'?»

Mia madre reagì con un sorriso riservato... un'espressione che
comprendevo fin troppo bene «Vuoi farlo oppure no?»

Quattro parole, una singola domanda. Ero un uomo in equilibrio su una
fune, pronto a precipitare a destra o a sinistra, ma mai a tornare
indietro. «E tu?» domandai, guardando verso di lei dal mio posto sul
divano. «Non ti preoccupa quello che farà mio padre quando saprà che mi
hai aiutato a sfuggire alla Cappellania?»

Lei si girò, dando le spalle alle finestre opacizzate, e d'un tratto mi
colpì quanto fosse più alta di me. Era dalla sua linea di discendenza
che Crispin aveva ereditato le sue dimensioni mostruose. Torreggiava su
di me come una Venere di alabastro o un'icona della Giustizia di vetro
soffiato bianco posta su un altare della Cappellania. «Mia madre è la
duchessa di tutto Delos \emph{e} una delle viceregine di Sua radiosità»
mi ricordò, gettando indietro la testa ed evocando tutta l'altezzosità
aristocratica di cui era capace. «Tiene in mano le palle di tuo padre.»

«Perché lo stai facendo?»

Lei protese il mento in fuori. «Al non mi ha mai chiesto neppure una
volta se ero d'accordo su questa faccenda di Vesperad, quindi che sia
dannato nel Buio Esterno. Tu sei mio figlio, Hadrian.» Si passò la
lingua sui denti come una leonessa annoiata, concentrando la sua
attenzione su qualcosa che solo lei poteva vedere. «È questo che vuoi?
Una vita da scoliasta? Entrare nei Corpi?»

Mi schiarii la gola, lottando disperatamente per reprimere l'ondata di
emozione che le parole `tu sei mio figlio' \emph{} avevano destato
dentro di me. «Sì.»

