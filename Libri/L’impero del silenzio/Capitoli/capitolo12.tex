\chapter{La bruttura del mondo}

Non avevo idea di come avrei fatto a lasciare il pianeta. Giocherellai
con l'idea di imbarcarmi su una nave mercantile, lavorando per pagarmi
il passaggio fino a Teukros o a Syracuse, o a qualsiasi mondo su cui ci
fosse un ateneo degli scoliasti, ma non ero un marinaio e non possedevo
capacità particolari che potessero invogliare un capitano di nave ad
assumermi. Inoltre, una posizione del genere avrebbe senza dubbio
richiesto un esame del sangue, che avrebbe immediatamente rivelato la
mia nobile nascita e segnalato la cosa all'ufficio di mio padre.
Qualsiasi nave passeggeri di linea avrebbe richiesto quello stesso
esame, per evitare che i servi legati al pianeta potessero fuggire in
violazione dei loro vincoli.

Di conseguenza, la logica richiedeva che mi rivolgessi a una classe di
uomini d'affari meno scrupolosi, anche se la prudenza sconsigliava una
cosa del genere.

E tuttavia, quale altra scelta avevo?

Nella mia giovinezza e ignoranza avevo sperato che Kyra potesse
condividere le mie infantili speranze di una storia o di un idillio fra
di noi. Nella mia lunga vita ho conosciuto troppi palatini, uomini e
donne, che abusavano in quel modo dei loro sottoposti. Ci sono parole
con cui definire le creature che abusano in questo modo del loro potere,
ma nessuna di esse verrà mai applicata a me. Nell'innocenza dei miei
sentimenti pensavo di essere diverso e non avevo pensato che non potevo
esserlo, che nessuna quantità di onestà o di intenti sinceri da parte
mia poteva colmare l'abisso fra Kyra e me, e che se lei mi si fosse
sottomessa non lo avrebbe fatto per desiderio ma per senso del dovere o,
ancora peggio, per paura. Avevo commesso un grave errore ed ero stato --
anche se solo per un momento e contro le mie intenzioni -- proprio quel
peggior genere di uomo che non è affatto un uomo.

Quindi mi nascosi in pari misura da lei e da gran parte della gente del
castello, tranne i miei insegnanti. Tor Alma mi sottopose a una serie di
esami medici prima della mia partenza, elargendo al mio già forte
sistema immunitario una serie di lievi potenziamenti che avrebbero
protetto le mie cellule dai peggiori patogeni esistenti fuori dal
pianeta e dalle radiazioni sottili comuni perfino nelle parti più
tranquille dello spazio. Sir Felix e io concludemmo i nostri
appuntamenti marziali un mese prima della partenza, quando venne
annunciato che nel partire avrei fatto tappa ad Haspida, trascorrendo la
mia ultima settimana sul pianeta con mia madre, al palazzo d'estate. E
poi naturalmente c'era Gibson, con le nostre frequenti passeggiate nei
giardini del castello.

\begin{figure}
	\centering
	\def\svgwidth{\columnwidth}
	\scalebox{0.2}{\input{divisore.pdf_tex}}
\end{figure}

«Quella dei pirati è un'idea orribile» dichiarò Gibson, grugnendo mentre
lo aiutavo a salire le scale che portavano al giardino di rocce
sottostante il monastero. «E lo è ancor di più assoldarne uno dalla
strada. Non sono... ecco, non sono le persone più rispettabili, giusto?»

Con una scrollata di spalle ammisi che aveva ragione, lasciando che mi
si appoggiasse al braccio mentre attraversavamo il cortile, attenti a
non essere sentiti di nuovo. Parlavano in jaddiano, con le sillabe
cantilenanti che si accumulavano come una confusa poesia tanto in fretta
che perfino l'orecchio più attento avrebbe potuto fraintendere sette
parole su dieci. «Non mi viene in mente nient'altro» affermai, girandomi
verso il volto del vecchio per parlare più vicino al suo orecchio
avviato alla sordità.

«È possibile che tu non lo debba fare» replicò Gibson, poi si premette
un dito storto sulle labbra per imporre il silenzio all'avvicinarsi di
tre servitori di passaggio che indossavano l'uniforme rossa degli
addetti alle pulizie della casa. Uno di essi si inchinò al mio passaggio
ma io mi limitai a segnalare loro di proseguire con un cenno e un
sottile sorriso. Rimanemmo in silenzio mentre ci spostavamo nell'ombra
di un portico le cui colonne interne erano coperte da un'edera tanto
scura da sembrare bluastra. Con fare ansioso lanciai un'occhiata alle
piccole videocamere che tempestavano la sovrastante volta poco profonda,
mascherate ma non del tutto nascoste dall'elegante motivo a volute della
modanatura. Da lì passammo nel giardino di arte topiaria, con i cespugli
scolpiti nelle forme fantastiche di uomini e draghi che apparivano quasi
neri nella luce argentea del sole.

«Cosa significa che potrei non doverlo fare?» chiesi, sempre in
jaddiano.

Gibson scosse il capo e sollevò il bordo della pesante veste verde per
non inciamparvi sui sassi mentre salivamo una scala interna fino a un
cammino di ronda che comunicava con uno dei molti viadotti sopraelevati
del castello. «Non ancora. Aspetta.» Poi: «Sei quasi pronto ad andare?»

«Ho più o meno finito di fare i bagagli. Non so cosa prendere. Qualsiasi
cosa succeda, non mi aspetto di possedere molte cose.» Non menzionai i
ventimila marchi sulla mia carta universale. «Non posso conservare
niente, giusto?»

Gibson si fermò per un momento e io rimasi indietro mentre riprendeva
fiato, agitando una mano per tenermi a distanza. «La Cappellania ti
permette un baule di effetti personali, se ben ricordo. Non c'era nelle
istruzioni che ti ha trasmesso tuo padre?»

Battei le palpebre. Non le avevo neppure lette perché avevo pensato che
come scoliasta ci si sarebbe aspettato che rinunciassi a tutto quello
che possedevo, e di conseguenza non avevo fatto nessun serio
preparativo. «Mi dispiace, non le ho lette» ammisi infine.

Gibson mi fissò a lungo, con durezza, e passò dallo jaddiano all'inglese
classico. «Farai meglio a provvedere.» E inarcò un sopracciglio in modo
tale da sottintendere: `Altrimenti la gente comincerà a farsi delle
domande.' Quando lui riprese a camminare oltrepassammo un paio di
peltasti dalle lunghe lance scintillanti al sole, che ci salutarono
quando li superammo per aggirare una scala a chiocciola esterna che si
incurvava intorno a una torre, portando alla diga marina.

Nel vedere il mare, la gente crede che sia l'acqua la prima cosa che
cattura la sua immaginazione, che per prima la affascina e la induce a
sognare di navigare e di vedere terre ignote, mai descritte. Si sbaglia.
Il primo attore del mare non sono le sue acque, ma il vento. E fu il
vento a investirmi per primo e con violenza quando salimmo su
quell'enorme arco semicircolare di pietra nera che formava la propaggine
più orientale del Riposo del Diavolo. Anche se l'èra degli assedi era
morta molto tempo prima della svanita Terra, i miei antenati avevano
eretto quel muro massiccio per difendersi contro eventuali flotte. I
bastioni erano irti di merli triangolari simili ai denti di una sega,
fra i quali perfino un uomo minuto poteva affacciarsi a guardare le
acque grigio acciaio che si abbattevano sulle alture sottostanti.

Mi riempii i polmoni di vento e sospirai, esprimendomi nella lingua
standard per la prima volta in quella giornata. «Vorrei che fosse tutto
finito, Gibson, che tutto fosse stato deciso.»

Più al sicuro là fuori che non in molte parti del castello, Gibson optò
anche lui per la lingua standard, anche se solo per un momento. «Capisco
questo particolare dolore. I periodi di cambiamento possono essere i più
sconvolgenti, ma ho scoperto che presentano le maggiori opportunità di
crescita. Affronterai quello che verrà...»

«O non lo farò» interloquii.

Lui sbuffò, permettendomi di guidarlo lungo il muro verso il dito nodoso
della Torre di Sabine, che si trovava a quasi un miglio di distanza,
lungo l'arco, dal suo monastero e dai giardini inferiori. Stavamo
camminando da un secondo quando Gibson tornò a parlare in inglese antico
e sussurrò: «La paura è veleno, ragazzo mio.»

«Un altro aforisma?» Sfoggiai il mio migliore mezzo sorriso dei Marlowe.

«Ecco... sì,» ammise lui, quasi borbottando «ma è pertinente.»

«Non lo sono sempre?» riflettei, staccandomi dalla sua presa e passando
a parlare in lothriano mentre una pattuglia ci incrociava, anche se
questa parte della conversazione, quantomeno, era del tutto innocente.
Avevamo preso l'abitudine di alternare in quel modo la lingua standard,
il lothriano, lo jaddiano e l'inglese classico, aggiungendo a volte la
lingua bastarda della Demarchia. Occasionalmente, ci esercitavamo
perfino nel cielcin, che già in quei primi giorni parlavo in modo
fluente. Di solito, riservavamo quella lingua a lezioni più private,
dato che qualsiasi cosa avesse a che fare con gli xenobiti Pallidi
attirava i sospetti dei fedeli della Cappellania.

«Teukros è molto più caldo di questo posto» osservò Gibson, adeguandosi
al mio lothriano. «All'interno del sistema non c'era abbastanza massa
cometaria per avviare un perdurante ciclo dell'acqua quando lo hanno
terraformato. I coloni hanno usato plancton della sabbia per regolare
l'aria perché le temperature di superficie d'estate si alzano abbastanza
da cuocere la flora più delicata.» Scivolò quindi nell'inglese classico
e aggiunse: «Dovrai liberarti di quelle tue ridicole giacche.»

Mi strinsi maggiormente intorno al corpo la lunga giacca che indossavo,
con il suo collo alto che mi premeva vicino alla faccia. «Credo che
terrò questa.» La verità era che sapevo che presto mi avrebbero tolto
quell'indumento per darmi la veste bianca e nera della Cappellania o
quella verde degli scoliasti. «Se ci incontreremo di nuovo vestirò in
verde, come te.»

«Non ci incontreremo ancora.» Non lo disse con crudeltà. Venendo da uno
scoliasta, quella poteva essere solo la semplice ammissione di un dato
di fatto, ma mi stordì quanto avevano fatto i colpi di mio padre. Non
replicai, prendendomi del tempo per assimilare quella realizzazione che
portava a riflettere. Avevo già saputo che non avrei mai più rivisto
nessuna di queste persone. L'Impero era vasto e l'universo umano lo era
ancora di più, e io stavo viaggiando attraverso quella quiete, congelato
per anni in una capsula criogenica. Me li sarei lasciati tutti alle
spalle.

Gibson inserì in quel silenzio alcune parole magiche: «Ho scritto la tua
lettera.»

Mi illuminai immediatamente. «Davvero?» Dovetti reprimere la mia gioia,
modellarla sull'apatia di Gibson per impedirle di trasudare.

«E ho qualche idea su come lasciare il pianeta. Idee che non coinvolgono
lo scommettere sulla carità dei pirati.» Il vecchio ripiegò il mento
contro il petto e avanzò per fermarsi nell'ombra di un enorme merlo,
simile in tutto e per tutto a un qualche gufo dalle penne verdi che
svolazzasse nel vento. Aveva le mani nascoste nelle maniche voluminose,
intente a tormentare qualcosa che vi era nascosto. «Ragazzo mio, sai che
viviamo in un mondo veramente meraviglioso?»

Quello non era il genere di domanda che mi sarei aspettato da uno
scoliasta, anche da uno umano come Gibson di Syracuse, quindi rimasi
sconcertato e mi girai a guardarlo. Aveva gli occhi segnati da cerchi
scuri e un grande peso gli gravava sulle spalle curve. Pareva un Atlante
invecchiato che si avvicinasse alla fine della sua eroica lotta per
reggere il peso del mondo. «Sì, immagino di sì» replicai, superando la
sorpresa e il precedente momento di dolore.

Lui sorrise, un gesto sottile come una ragnatela nella luce brunita.
«Non ne sembri convinto.» Contro i miei stessi desideri mi girai a
guardare l'edificio di granito nero e di vetro a specchio che costituiva
la Grande Rocca e il bastione della mia casa. La stessa luce solare che
rivestiva il mare di vetro argenteo non aveva lucentezza per il castello
dei miei avi che -- pur essendo perfettamente esposto al sole -- pareva
sovrastato dall'ombra di una nuvola. Sentii lo scoliasta ridere.
«Suppongo che tu non mi creda, ma non hai visto questo.» Senza neppure
guardare compresi che si riferiva all'oceano.

«L'ho visto.»

«\emph{Kwatz},» ringhiò lui, rimproverandomi. «Guardi soltanto ma non
hai \emph{visto}.»

Guardai.

Come ho detto, l'oceano era una lastra di vetro increspato bordato di
fuoco plumbeo. A quell'ora e da quell'altezza le Isole del Vento erano
invisibili e le poche nuvole proiettavano ombre profonde sul mare,
trasformando l'acqua argentea in un nero che splendeva come le
profondità dello spazio. Gibson aveva ragione... era splendido.

«Nonostante tutto ciò che sta accadendo là fuori, intorno ai soli più
lontani,» scandì Gibson, con le mani che armeggiavano ancora nelle
maniche della veste «e nonostante gli eventi che vi si verificano.
Nonostante tutta quella bruttura, Hadrian, il mondo è splendido.» Tirò
fuori le mani e vidi che in una di esse stringeva un piccolo libro di
pelle marrone. «Tienilo stretto.» Trasse un profondo respiro. «Un'ultima
lezione, prima che tu parta.»

«Signore?» Accettai il libro e ne lessi ad alta voce il titolo.
«\emph{Il re con diecimila occhi}? Di Kharn Sagara?» Aprii la copertina
e andai alla prima pagina. Infilata a ridosso della rilegatura c'era una
piccola busta di un bianco opaco: la lettera che gli avevo chiesto di
scrivere agli scoliasti. Richiusi in fretta il volume, temendo che un
drone con videocamera potesse sorvolarci da un momento all'altro, anche
se l'aria era sgombra, a parte i distanti gabbiani che stridevano e
volavano in cerchio. «Il re pirata? Gibson, questo è un romanzo.»

Lui sollevò una mano per farmi tacere. «È solo il dono di un vecchio,
eh?» Agitò quella mano in un gesto autocritico che {accantonava} la
cosa. «Ora ascolta. Questa è una lezione che nessun tor o primate di
college ti impartirà mai, e neppure un anagnostico della Cappellania --
sempre che possa essere impartita.» Si girò di nuovo a guardare il mare.
«Il mondo è morbido come lo è l'oceano. Chiedi a qualsiasi marinaio cosa
intendo. Ma anche quando è al massimo della sua violenza, Hadrian...
concentrati sulla sua bellezza. Le brutture del mondo ti assaliranno da
ogni lato, è impossibile evitarle. Tutti gli insegnamenti dell'universo
non potranno impedirlo.» Ero così meravigliato di sentire un logico come
lui parlare in quel modo che non mi soffermai a chiedermi cosa avesse
inteso con `le brutture del mondo'. Adesso mi domando se avesse saputo
ciò che quello stesso giorno aveva in serbo per entrambi o con quanta
rapidità lo stivale sarebbe calato, come aveva fatto sulla faccia di
quel povero schiavo nel Colosso. «Però nella maggior parte dei posti
della galassia non sta succedendo niente. La natura delle cose è
pacifica, e questo è un fatto.»

Non sapevo cosa dire, ma mi venne risparmiato di replicare quando Gibson
chiuse l'argomento aggiungendo: «Te la caverai bene, qualsiasi cosa
succeda.»

Infilai il libro nella piega del braccio e, agendo di impulso,
abbracciai di nuovo quel vecchio che era per me meglio di un padre.
«Grazie, Gibson.»

«Non credo che verrai meno a nessuno di noi.» Emisi un verso di protesta
in fondo alla gola, ma prima che potessi far uscire le parole lui
aggiunse: «E neppure ai tuoi genitori. Lo scopriranno dopo che te ne
sarai andato.»

«Non ne sono certo.» Lo lasciai andare. Avevo due settimane per fare i
miei preparativi e i miei addii, ma a parte Gibson non c'era nessuno a
cui valesse la pena di dire addio.

Lui sorrise, mostrando una fila di piccoli denti bianchi. «Nessuno di
noi lo è mai.»

Fu la mia immaginazione, o un qualche scherzo della luce argentea? Mi
parve di vedere un'ombra scendere sul suo volto, come se il sole si
fosse nascosto dietro una nuvola. Quando ripenso alla faccia di Gibson è
così che lo vedo, com'era in quel momento, chino e sferzato dal vento
sui bastioni della diga marina, rimpicciolito e triste. Un vecchio
appoggiato al suo bastone. Ricordarlo in qualsiasi altra luce che non
sia quella di quel giorno splendido sarebbe in qualche modo un
sacrilegio, come se tutti i nostri altri giorni {fossero} brutti.
