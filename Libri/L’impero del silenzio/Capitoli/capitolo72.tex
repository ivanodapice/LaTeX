\chapter{Sangue pallido}

Quel posto riluceva come una sala operatoria. Il che era ciò che
supponevo fosse in effetti. Le camere degli interrogatori sotto la
bastiglia della Cappellania Terrestre -- una struttura brutalista
sorprendentemente senza pretese alla base dello ziggurat su cui era
appollaiato il castello di Borosevo -- erano costruite tutte sullo
stesso modello, come palloni di acciaio inossidabile gonfiati
all'interno di cubi. Le pareti, il pavimento e il soffitto della camera
che occupavamo si fondevano tutti uno nell'altro e pannelli luminosi
erano fissati al soffitto, più freddi dello spazio. In quel posto
orribile non c'erano ombre.

Quando entrai Uvanari mi dava le spalle, guidato da un'inquisitrice
vestita di bianco e da due cathar vestiti di nero, calvi e bendati. Mi
ricordò un'icona del dio pagano Andreas, con le gambe e le braccia
allargate in una X, e anche se era girato di spalle scorsi la macchia
bianca del suo riflesso nel metallo sabbiato del muro della cella, per
cui compresi che era nudo. Obbedendo alle istruzioni dell'inquisitrice
rimasi fuori vista, aspettando in un angolo accanto a un carrello su
ruote carico di strumenti chirurgici e di chiari tubi lucenti. Il mio
respiro si faceva bianco nell'aria e la brina mi scricchiolava sotto i
piedi. Sentivo il peso degli sguardi degli osservatori umani, quegli
stessi piccoli uomini dal cuore di cenere che avevano ordinato questo
interrogatorio. Ligeia sarebbe stata fra loro, e anche Ogir, e Smythe.

Smythe. Mi sarei aspettato di meglio da quella ufficiale che un tempo
era una plebea.

I Cathar si diedero da fare ad applicare sensori in diversi punti del
corpo dello xenobita, poi uno di essi venne a prendere il carrello che
avevo accanto e lo spinse intorno alla croce a cui era appeso Uvanari in
modo che lui lo potesse vedere. Un lamento acuto e sottile sfuggì dalle
labbra del Cielcin, facendo affiorare un sorriso sul volto piatto di
nativa dell'inquisitrice, che lo considerò un segno di dolore e di
paura, le stesse espressioni dipinte sul mio volto.

Uvanari stava ridendo. «\emph{Qisaba}!» imprecò, interrompendo quel
suono penetrante, e le parole della sua lingua crearono un contrasto
gutturale con la stridula acutezza di quella risata inumana. «Perché vi
siete presi il disturbo di guarirmi?» Poi si fece silenzioso e cercò di
allungare il collo per guardarsi intorno senza però riuscire a vedermi.
Doveva sapere che c'era qualcun altro, se non altro per via della
macchia di colore riflessa sulla parete. Le sue parole successive mi
ferirono profondamente. «\emph{Raaka Marlowe saem ne}?» `Dov'è Marlowe?'
«Mi aveva promesso rifugio.»

L'inquisitrice guardò verso di me perché traducessi. Mi resi conto che
non aveva individuato il mio nome nella sfilza di parole aliene e mi
accorsi che stavo trattenendo il respiro. «Ha chiesto perché lo avete
guarito solo per fargli ancora del male.» Non mi presi la briga di
aggiungere la parte relativa alla promessa di rifugio. A che sarebbe
servito?

«Marlowe!» Uvanari girò la testa, cercando di vedermi. «\emph{Bakkute}!
Lo hai detto! Lo hai promesso!» Il suo precedente divertimento per la
situazione era scomparso. Adesso aveva un nemico: me.

«Chiede di me» dissi all'inquisitrice, poi aggiunsi, rivolto al Cielcin:
«\emph{Asvatatayu koarin o-variidu, Uranari-se.}» `Non mi hanno lasciato
scelta.'

«Non parlare con il prigioniero se non per tradurre!» scattò
l'inquisitrice, indietreggiando dalla grata sul pavimento che aspettava,
avida, sotto la croce. Sfilò un nodulo di registrazione dal terminale
che aveva al polso e se lo accostò alle spesse labbra. «Sedici uno
settantadue zero due tredici. L'inquisitrice K.F. Agari presiede
all'interrogatorio. Il soggetto è lo xenobita cielcin chiamato Uvanari
nel rapporto Calagah. Assistono i fratelli Rhom e Udan, e un traduttore
laico.» I suoi occhi neri si socchiusero e abbassò la faccia per
prendere un'annotazione su un'immagine olografica che le scaturiva dal
polso. «Sei il capitano della nave abbattuta su questo mondo?»

L'interrogatorio cominciò in questo modo, freddo e distaccato, quasi
clinico quanto la stanza stessa. Io ero soltanto un'interfaccia, il
sostituto dei congegni di traduzione che si diceva i Tavrosiani, i
Normanni e gli Extrasolari dessero per scontati. In effetti cercai di
essere meno di me stesso, di distaccarmi da quel posto terribile.
Ritirati dentro di te, mi dissi. Non era un aforisma degli scoliasti,
era solo il pensiero confuso di un giovane umano troppo coinvolto e
lontano da casa.

«Sono Itana Uvanari Ayatomn, \emph{ichakta} della nave \emph{Yad Ga
	Higatte.}» Dal tono piatto della voce della creatura compresi che quello
era l'equivalente cielcin del limitarsi a nome, grado e numero di
matricola, e il cuore mi si fece pesante come il piombo nella previsione
del sangue che sarebbe seguito.

L'inquisitrice Agari accettò la traduzione con un cenno del capo.
«Perché siete venuti su Emesh?»

Tradussi, sostituendo Emesh con `questo mondo' perché sapevo che il nome
del pianeta non aveva significato per il Cielcin. Con mio orrore,
Uvanari si limitò a ripetere: «Sono Itana Uvanari Ayatomn,
\emph{ichakta} della nave \emph{Yad Ga Higatte}.»

Qualsiasi altra cosa potesse essere, l'inquisitrice non era tanto
stupida da non cogliere la ripetizione. A un suo cenno uno dei due
cathar si avvicinò, si portò alle spalle del Cielcin e manovrò un
meccanismo sul retro della croce che fece abbassare un braccio in modo
che fosse facilmente raggiungibile per un umano. Con il braccio sempre
bloccato, senza dire una sola parola e senza nessuna esitazione il
cathar rimosse un artiglio simile a vetro dall'estremità del primo dito
di Uvanari. Più spesso di un'unghia umana, esso emise un suono secco e
crepitante nello spezzarsi. Il principio però era lo stesso, e lo
xenobita represse un grido quando il sangue prese a colare, più nero
delle vesti dei cathar, gocciolando sulla grata sottostante la croce.

«Ricordagli che ne ha altri undici.»

«Mi dispiace, ho cercato di impedirlo, ma...» dissi invece. Che altro
potevo dire? Mi interruppi e cercai un altro posto dove guardare ma
trovai soltanto i nostri opachi riflessi nelle pareti di metallo
sabbiato. Immaginai di fare lo stesso ripetutamente, con ciascuno dei
nostri prigionieri, finché ciascuna creatura non fosse sprofondata nella
furia e nella follia, e poi nella morte, dissanguata e {tagliata} finché
non ne rimaneva nulla. Gli antichi erano soliti credere che non ci fosse
nessuna scienza nella tortura, niente da guadagnare, e non voglio
sostenere che si sbagliassero, ma il potere che la Cappellania traeva
dalla tortura non risiedeva mai nel trovare la verità, neppure quando lo
faceva. Piuttosto, serviva a insegnare ai grandi a temerla, perfino
all'imperatore. E adesso la stava insegnando al Cielcin.

\emph{Ritirati dentro di te.} Pregai senza rivolgermi a niente e a
nessuno. Poi però mi immobilizzai per un momento, rendendomi conto che
non avevo tradotto le ultime parole di Agari. Non avevo minacciato
Uvanari, mi ero scusato, e nessuno lo aveva notato. \emph{Nessuno lo
	aveva notato.} Potevo imboccare un mio sentiero -- per quello che poteva
valere -- nel fornire le risposte. Dovevo solo stare attento.

«Perché siete venuti su Emesh?»

«Sono Itana Uvanari Ayatomn, \emph{ichakta} della nave \emph{Yad Ga
	Higatte}.»

\phantomsection\label{fileintero-74.xhtml__idTextAnchor007}{}«Perché
siete venuti su Emesh?»

«Sono Itana Uvanari Ayatomn, \emph{ichakta} della nave \emph{Yad Ga
	Higatte}.»

«Perché siete venuti su Emesh?»

«Perché siete venuti...?»

«Perché siete venuti...?»

Staccarono sette artigli dalle mani di Uvanari prima che rispondesse,
prima che pronunciasse una singola parola. «\emph{Balatiri}!
\emph{Civaqatto balatiri}!» `Siamo venuti per pregare.'

Imprecai, inducendo l'inquisitrice a inarcare un sopracciglio, poi
tradussi: «Dice che sono venuti qui per pregare.» Gli artigli strappati
erano disposti in un contenitore d'acciaio sul carrello e sotto il gelo
della stanza potevo sentire il fetido puzzo metallico del sangue.

La domanda successiva venne sommersa da altre parole
dell'\emph{ichakta}. «Voi non avreste dovuto essere su questo mondo. Ci
siamo avvicinati alla cieca. Non lo sapevamo.»

«Non lo \emph{sapevate}?» ripeté l'inquisitrice quando ebbi finito di
tradurre, e a un suo cenno uno dei cathar pungolò Uvanari con un bastone
stordente. La corrente percorse il corpo della creatura la cui carne si
tese contro le cinghie di cuoio quando si inarcò. «Come potevate non
saperlo?» Fece un brusco gesto orizzontale con la mano quando cominciai
a tradurre automaticamente e tacqui, osservandola mentre ruminava su
quella risposta e con colpi di dito dava una serie di prompt al
terminale da polso. Un diagramma di flusso. Stava basando le sue domande
su una dannata \emph{carta di flusso} terrestre! L'orribile banalità di
quel fatto mi investì come un colpo fisico. Qui non si trattava neppure
di religione, ma di affari.

Si prese un momento, poi mi chiese in galstani: «I Cielcin hanno una
religione?»

«Non ne so molto» risposi, desiderando che ci fosse un dio e che potesse
schiacciare quella piccola stanza a forma di bolla con me ancora dentro.
«La loro parola per indicare `dio' significa... osservatore? Maestro? È
tutto quello che so, mi dispiace.»

Mi segnalò di tacere. Potevo vedere i collegamenti che si stavano
formando nella sua mente di cacciatrice. I Cielcin. Le rovine. La
Quiete... sapeva dell'ipotesi della Quiete? Vedeva degli xenobiti che
venivano su un mondo abitato da altri xenobiti e avrebbe fatto il
collegamento, giusto o sbagliato che fosse. Gli Umandh sarebbero
bruciati per questo, visti senza dubbio dallo zelo della Cappellania e
dai suprematisti umani come in qualche modo complici dei Cielcin nella
guerra contro l'umanità. Un'altra purga. Un'altra marcia dei fedeli.

«Arriveranno altri di voi?»

Il sangue gocciolava da sette delle dodici dita, nero come olio. Uvanari
girò la testa in senso antiorario. «No.»

Incapace di parlare, scossi il capo e l'inquisitrice sollevò una mano.
Invece di staccare l'ottavo artiglio, i cathar rimossero la punta di una
delle dita mutilate, tranciandola all'articolazione. Il secondo cathar
depose il piccolo moncherino amputato nel vassoio di acciaio su una
panca addossata al muro di fondo, accanto a un profondo lavandino. Lo
fecero senza malizia, senza pompa o melodramma, si limitarono a staccare
la punta del dito con un coltello a spinta e un lieve colpetto. Quasi mi
aspettai che le dita simili a vetro della mano si sbriciolassero e che
Uvanari andasse in pezzi come una scultura, ma urlò soltanto. «Perché lo
avete fatto?» stridetti. «Vi aveva risposto!»

«Con troppa facilità» replicò l'inquisitrice, e sollevò lo sguardo su
una delle videocamere come se fosse stata un profeta ed essa il suo Dio,
da cui potessero scaturire le risposte. In quella posa, attese che le
urla di Uvanari si spegnessero.

«Mi dispiace, non lo sapevo» dissi, quando i suoi lamenti furono
cessati.

Mi fissò con ira, i denti di vetro che affondavano in un labbro, le
guance che ansimavano e le quattro narici dilatate, ma si rifiutò di
parlare.

«Non comunicare con il prigioniero, traduttore!» scattò l'inquisitrice.
La trafissi con uno sguardo rovente finché non distolse il suo. Nel mio
stato attuale ero al di là del preoccuparmi che quella era
un'inquisitrice della Cappellania, che avrei dovuto temerla anche se ero
un palatino. La realtà di quel momento mi aveva privato di ogni
preoccupazione. Se non fosse stato per uno scherzo del fato, per mia
madre, avrei potuto essere io con la testa rasata e la veste
inquisitoriale, a porre quelle domande. Confesso che era quel pensiero,
e non la compassione per lo xenobita sanguinante, a consumarmi.
Sperimentai uno strano momento in cui mi vidi sdoppiato, nei panni di
inquisitore e di nuovo nella galleria di Calagah, con uno storditore
premuto contro la testa di quell'altro Cielcin.

L'inquisitrice Agari riformulò la domanda. «Ti credo, ma devi darmi
qualcosa.» Un giorno la registrazione di quell'interrogatorio sarebbe
forse stata vagliata dagli analisti militari della Legione, o dagli
anagnosti della Cappellania, o ancora dai logoteti della corte imperiale
stessa. L'avrebbero tradotta e avrebbero scoperto le mie aggiunte, ma
per il momento potevo farla franca.

Uvanari aveva ancora il respiro affannoso e si inclinò in avanti,
accasciandosi contro le cinghie con le braccia rovinate divaricate.
«\emph{Asvatoyu de ti-okarin, hih siajaev leiudess}.»

«Afferma che non ci può dire quello che non sa» tradussi, poi volsi
parzialmente le spalle all'inquisitrice, a Uvanari e alla croce, a tutto
quanto. Con la coda dell'occhio vidi la donna sussultare leggermente, ma
avanzai verso di lei. «Un momento, per favore. Dagli un momento! Lascia
tentare me.» Attesi, e quando l'inquisitrice annuì mi rivolsi a Uvanari,
parlando con voce rauca. «Quindi non eravate una flotta di invasione.
Hai detto che siete venuti per pregare? Cosa intendi con pregare? Nelle
rovine?» Potevo sentire lo sguardo dell'inquisitrice che mi trapassava
la testa e mi aspettavo che cambiasse idea, che protestasse, ma questa
volta rimase in silenzio.

Alla fine Uvanari riuscì a replicare: «Hai visto le rovine.» Il petto
gli si sollevò e abbassò, il sudore gli imperlava la fronte sotto la
cresta ossea, scorrendo lungo le fini sporgenze di pelle simili a
scaglie dove le corna si facevano bianche e lisce. Annuii, poi realizzai
il mio errore e ruotai invece la testa a imitazione del suo gesto
affermativo. Nel vederlo, la creatura snudò i denti simili a bisturi di
vetro in quel ringhio che, come cominciavo a comprendere, era in realtà
un sorriso.

«Adorate coloro... coloro che hanno creato quel posto?» Non conoscevo il
termine per dire `costruttori', quindi dovetti improvvisare. È buffo
come proprio questo fra tutte le cose, quel piccolo fallimento, mi sia
rimasto impresso nella memoria. L'inquisitrice emise un verso e mi girai
verso di lei, dicendo in galstani: «Inquisitrice, per favore.» Uvanari
inclinò la testa verso destra, un secco gesto cielcin per assentire, poi
sussultò e tornò ad accasciarsi contro le cinghie. Non c'era dove
poggiare la testa, quindi la sua corona pendeva fra le aste che gli
tenevano divaricate le braccia. «\emph{Ichakta}, per favore» insistetti.
«Finora non hanno fatto niente a cui non si possa rimediare. Dimmi,
questo pianeta è minacciato?»

«\emph{Veih}!» ringhiò il capitano. «No, non lo è. Eravamo qui perché
\emph{loro} erano qui, non per voi. Gli dèi. Hanno costruito quelle
grotte, le stesse che ci sono su Se Vattayu.» `Sulla Terra.'

Ci misi un secondo a decifrare quel dettaglio -- la parola
\emph{vattayu} significava `terra', `suolo', `polvere'. Per un momento
immaginai la nostra Terra, il Mondo Natale, dea della Cappellania -- e
dovetti scuotere la testa per liberarmi da quell'idea.

«Sul vostro mondo natale?» \emph{Se Vattayu}. Quella era una nuova
informazione, almeno per me. Anche loro chiamavano `Terra' il loro mondo
di origine. Non lo avevo saputo e ciò che sottintendeva si mise a fuoco
un momento più tardi. «Avevate rovine come queste sul vostro mondo
natale?» Gli studiosi erano tutti concordi nel ritenere che i Cielcin
fossero un popolo sotterraneo, teoria rinforzata dalla natura buia e
cavernosa delle loro navi e dal trucco che avevo utilizzato con le luci
delle tute nelle gallerie di Calagah.

«Cosa sta dicendo?» mi incalzò l'inquisitrice.

Non volevo dirglielo perché ero certo che farlo avrebbe messo in
pericolo Elomas e tutti coloro che lavoravano al sito di Calagah,
avrebbe messo in pericolo Valka. Immaginai quelle gallerie di granito
ridotte a scorie fuse, cariche atomiche che trasformavano gli archi
delicati e quelle parallele non euclidee in cenere. «Non ci sarà un
altro attacco, inquisitrice.»

«Tutto qui?»

«Tutto qui» mentii.

«Non puoi dire sul serio» sogghignò, arricciando il naso. Venne avanti e
strappò il bastone storditore dalle mani di uno dei cathar, piantandolo
contro le costole di Uvanari. Lo tenne lì con gli occhi accesi da una
gioia animalesca che mi nauseò. Cosa le avevano fatto su Komadd, o
dovunque era stata indottrinata? Come avevano creato una donna del
genere? Oppure era sempre stata così? «Cosa state pianificando,
\emph{inmane}? Un'altra invasione?» Trasse indietro la mano e calò il
pesante bastone sulla faccia di Uvanari, che grugnì ma rimase immobile.
«È la gente che volete?» Uno dei cathar si affrettò a venire avanti e a
posare le mani guantate sulla faccia della creatura per controllare che
non ci fossero danni non pianificati. Quell'intervento dei suoi
subordinati riportò in sé Agari, che indietreggiò barcollando. Non avevo
tradotto niente, e lei non lo aveva notato.

Guardai verso il soffitto, desiderando che una qualsiasi voce -- quella
di Olorin o magari della cancelliera -- risuonasse dagli altoparlanti
per porre fine a quell'orribile esperienza. Gli orizzonti della realtà
erano però limitati dalla bolla d'acciaio della cella ed era difficile
che all'esterno ci fosse qualcuno che poteva interferire.

Il cathar controllò che non ci fosse una commozione cerebrale, o ossa
fratturate nella faccia o qualche dente rotto, poi urlò e cadde
all'indietro fra le braccia del suo confratello, stringendosi una mano.
Per un momento non vidi il sangue sullo sfondo più nero del nero delle
loro vesti, ma quello scintillio umido era inconfondibile e il rosso che
colava lungo il mento di Uvatari non era il nero del sangue dei Pallidi.
Per un orribile momento vidi i moncherini di due dita umane che facevano
capolino fra le labbra del capitano, poi svanirono masticate fra i denti
di Uvatari e inghiottite. La voce di Crispin risuonò nella mia memoria:
`Oh, quindi non sono tutti cannibali?' D'un tratto, la distinzione
tecnica fra cosa era e non era cannibalismo non ebbe importanza,
barcollai all'indietro, incapace di controllare l'orrore che mi
sconvolgeva le viscere.

Il secondo cathar impedì all'inquisitrice di infliggere una punizione
sollevando una mano -- per antica usanza avevano il divieto di parlare
durante il rito dell'inchiesta. Per quanto inorridito, non potei fare a
meno di ammirare lo spirito del Cielcin, il suo rifiuto di lasciarsi
intimidire. Mi piacque immaginare che se le posizioni si fossero
invertite avrei avuto quello stesso spirito, anche se avrei sputato le
dita... ma io non ero un Cielcin.

«\emph{Biqathebe ti-okarin qu ti-oyumn}» dissi, mentre l'inquisitrice si
dava da fare per aiutare il cathar ferito. `Ti faranno del male per
questo.'

«Che lo facciano.» Uvatari non poteva pulirsi il mento dal sangue che
gli gocciolò carminio sul petto. La sua lingua fra il blu e il nero
scivolò fuori e spalmò quel sangue sulle labbra. «Voi umani siete tutti
gli stessi. Sempre gli stessi.» A quel tempo non mi colpì quanto fosse
strana quell'affermazione. «Mi dispiace» replicai, incapace di guardarlo
negli occhi. I muscoli sotto la carne cerea si tesero in forme e
sentimenti che mi riuscivano sconosciuti. Per certi versi i Cielcin
erano più strani degli Umandh anche se camminavano e parlavano come
uomini: la mente dietro quegli occhi era la mente di persone
incomprensibili. Quello che interpretavo come coraggio o cocciutaggine
avrebbe potuto non essere affatto tale. Vedendo questo -- vedendoli --
pensai che la Cappellania non si sbagliava completamente.

Forse tutto quello che condividevamo era il dolore.

La creatura sputò ai miei piedi. In quel gesto non c'era malizia perché
fra i Cielcin quello non era un particolare insulto, ma nella saliva
c'era del sangue fra il blu e il nero. Indietreggiai, urtai contro il
carrello e mi immobilizzai.

«Cosa stai facendo?» domandò l'inquisitrice, girandosi di scatto verso
di me mentre la porta d'acciaio si sigillava con un ronzio pneumatico.
«Che cosa ha detto?»

«Parole coraggiose» replicai, inclinando la testa da un lato. «L'ho
avvertito che non avrebbe dovuto farlo.»

L'inquisitrice si raddrizzò, con la veste altrimenti immacolata segnata
da punti e strisce rosso vivo. «Non avrebbe dovuto.»

«Credo però che stia dicendo la verità» continuai, sperando di poter
placare l'ira dell'inquisitrice, mentre muovevo un passo infinitesimale
per mettermi fra lei e la croce. «Non credo che stia arrivando un'altra
flotta. Interroga gli altri.» Poi realizzai la gravità di quello che
stavo dicendo e feci marcia indietro. «\emph{Limitati} a {interrogarli}.
Loro non sono... te lo diranno. Isolali, falli parlare. O ti diranno
cose diverse e staranno mentendo, oppure ti diranno la stessa cosa e
saprai che abbiamo appurato la verità. Questa è la procedura standard,
giusto?»

L'inquisitrice prese il bastone stordente dal carrello e lo soppesò,
pronta a riprendere il suo lavoro, poi fece eco alle parole orribili che
Ligeia Vas mi aveva detto un tempo. «Saresti un buon prete.» Sentii il
sangue che mi si addensava fino a diventare veleno in reazione a quel
commento e volsi le spalle, nascondendo i miei occhi. Uvatari ululò
quando la corrente gli passò nel corpo, un suono che si trasformò in un
acuto lamento nasale quando espulse a forza l'aria attraverso le fessure
che sostituivano il naso. L'inquisitrice non aveva neppure fatto una
domanda. Usò ancora il bastone, e solo dopo aver fatto urlare la
creatura una quarta volta disse: «Chiedigli chi serve, e dov'è adesso la
sua gente.»

