\chapter{Abbandonati}

«Lei ci ha abbandonati!» stridette il vate, protendendo le dita nodose
verso il cielo nella piazza sopraelevata davanti alla cupola massiccia
della Cappellania di Borosevo, nel Distretto Bianco. Il folle sant'uomo
era in piedi su una piattaforma rialzata di dieci piedi rispetto alla
pavimentazione e gridava rivolto a tutti quelli che erano disposti ad
ascoltarlo. Nella maggior parte dei giorni la gente oltrepassava in
fretta soggetti come lui, sbucava dai canali o dai parcheggi chiusi e
riservati a quei pochi cittadini abbastanza ricchi e sciocchi da
acquistare un'auto di terra in un posto come Borosevo, dove c'erano
poche strade. Quel giorno però era venerdì e la grande priora del
sistema -- un'anziana sacerdotessa di nome Ligeia Vas che mi ricordava
la vecchia, avvizzita Eusebia -- stava celebrando la litania. A causa di
questo la piazza era inondata dai fedeli che non potevano essere
contenuti nel sacrario e che guardavano invece la priora sullo schermo
appeso fra le colonne che raffiguravano i Quattro Elementi Cardinali.

I mendicanti, giovani e vecchi, affollavano a loro volta gli accessi
alla piazza ed erano raccolti intorno alle colonne che si levavano
davanti ai due battenti delle porte. Molti erano fasciati, con ulcere
purulente dovute alla necrosi grigia ma ancora di più recavano i segni
della giustizia della Cappellania e mancavano di dita, pollici, occhi,
lingua. Si poteva leggere il loro crimine tatuato sulla fronte in
lettere nere: \textbf{AGGRESSIONE, FURTO, AGGRESSIONE, ERESIA, ERESIA, FURTO, STUPRO, FURTO, AGGRESSIONE}.

Quelli di loro che erano meno vestiti mostravano sulla schiena le cicatrici causate dalla
frusta o brutti gonfiori e cicatrici da ustione vivide come metallo
nuovo. Gli uomini erano un numero spropositato, anche se la necrosi non
mostrava pregiudizi di sesso. Fra i membri in piedi della congregazione
alcuni indossavano una maschera sul volto o portavano i guanti
nonostante il caldo.

Quella folla significava anche buone condizioni per mendicare. Con
un'icona della Carità scolpita sulle porte del sacrario, anche i più
duri di cuore fra i fedeli ci pensavano due volte prima di respingerci
con un piede.

Che ci dessero una misera moneta d'acciaio o un quarto di kaspum era
però tutt'altra cosa, ma io chinavo comunque la testa con placida
gratitudine, inginocchiato come un penitente accanto a Cat vicino
all'angolo di una strada che portava nella piazza. Il vate nudo e
puzzolente urlava ancora dal suo pulpito. «Nostra Madre, la Terra che
Era e che Tornerà a Essere, ha distolto da noi il Suo volto. Questi
demoni Pallidi sono la Sua punizione per il nostro comportamento
presuntuoso! Badate alle mie parole, fratelli e sorelle, figli della
Terra e del Sole! Badate alle mie parole perché la punizione sta
arrivando! Un fuoco purificatore che laverà via tutti i nostri peccati.
Pentitevi! Pentitevi! E siate mondi di nuovo!»

Un uomo lasciò cadere una moneta nella ciotola di Cat. Lei appariva così
piccola e terribilmente triste sotto una delle videocamere di
sorveglianza della piazza. «Dio e l'imperatore ti benedicano, messere»
disse lei, chinando la faccia sulla ciotola. Non potei fare a meno di
notare che aveva un numero di monete quasi triplo rispetto al mio e feci
una smorfia. Se non altro, le mie costole erano guarite.

«Dividerai con me un po' di quel bottino, vero?» chiesi con un sorriso.
Tenni bassa la voce ma non riuscii a mascherare il suo tono scherzoso.

«Per gli dèi, no» ritorse Cat. «Procurati il tuo!» E mi assestò un colpo
con la mano, sorridendo con la luce che scintillava sui suoi denti
storti. Ridacchiai piano. Era bello ridere di nuovo, avere una ragione
per farlo. La sua mano indugiò per un momento sul mio ginocchio, con le
dita che risultarono calde e umide attraverso la stoffa dei pantaloni.
La giornata era rovente, con l'aria spessa e intrisa di vapore. Eravamo
rimasti là per tutta la mattina, come facevamo sempre nel giorno della
Somma litania. Una donna con un completo viola che teneva aperto un
parasole di carta a colori vivaci posò con un sorriso un intero kaspum
d'argento nella ciotola di Cat, che per poco non pianse e si alzò per
inchinarsi in segno di gratitudine.

Guardai la mia ciotola, dove c'era una misera raccolta di monete
d'acciaio e qualche misera banconota spiegazzata da un dodicesimo di
kaspum. Non ho mai dimenticato il sorriso sul volto di quella donna,
anche se non ha mai detto una parola, e quando penso alla gentilezza
rivedo la forma di quella bocca, con il suo dozzinale rossetto rosso.

«Abbiamo rifiutato la natura!» gridò ancora il vate. «Pieghiamo il
ginocchio e il collo davanti ai lord resi meno che umani!» A questo
punto il profeta nudo afferrò il proprio membro con una mano nodosa
mentre il vento gli agitava la barba. «La Madre lo sa! Sa che i nobili
l'hanno dimenticata, l'hanno eliminata dal loro sangue!» La parte di me
che era stata il figlio di un arconte sussultò, quasi aspettandosi di
vedere i prefetti -- o perfino i soldati nella livrea verde e bianca del
conte -- venire a portare via quel vecchio predicatore pazzo. Invece non
venne nessuno, perché si dice che i folli siano più vicini alla Terra.

