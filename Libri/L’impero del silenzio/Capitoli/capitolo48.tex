\chapter{Il trionfo}

La musica che accompagnava il marciare della parata fuori dal vomitorium
e sotto il palco del lord era assordante, un suono marziale di ottoni e
tamburi amplificato dai droni che volavano al di sopra della folla. Io
me ne stavo in un angolo remoto del palco, a sorseggiare un bicchiere di
vino rosso di Kandarene all'ombra di un tendone a strisce mentre
guardavo il conte e suo marito salutare dalla parte anteriore del palco,
con Anaïs in mezzo a loro. Il giovane lord Dorian, di cui si celebrava
il compleanno, era in piedi su un catafalco alla testa della parata, in
armatura da combattimento completa smaltata in verde e oro, con un manto
bianco fissato sulle spalle e una spada di altamateria che gli splendeva
in mano.

Lo vidi a stento.

Invece, vedevo Crispin, mio fratello, che in nero e carminio veniva
portato in parata intorno al colosseo di Meidua mentre mio padre e mia
madre -- mia madre era mai stata presente? -- guardavano dal posto
equivalente a quello dove si trovavano lord Balian e lord Luthor. Invece
di tor Vladimir e della cancelliera Ogir vedevo sir Felix e tor Alcuin.
Al posto della muscolosa dama Camilla per me c'era Roban, che una volta
mi aveva salvato la vita. Solo la presenza della Cappellania Terrestre
era immutata: due figure spettrali in nero, più scure dello spazio
stesso, con la casula bordata di un bianco iridescente. In effetti, se
non fosse stato per le deformità del cantore Gilliam, lui e il suo
superiore, l'alta grande priora di Emesh dal naso adunco, avrebbero
potuto essere la copia di Severn e della vecchia Eusebia.

La donna era Ligeia Vas, la madre naturale di Gilliam, resa innaturale
dal tempo. Cercai di guardare al di là del suo volto avvizzito, della
lunga treccia di capelli di un bianco argenteo avvolta intorno alle
spalle come una sciarpa, delle dita nodose appoggiate al bastone, e
cercai di vedere la donna palatina che aveva volontariamente portato a
termine un bambino dentro di sé. Non riuscii a scorgerla, ma del resto
chi può vedere una vita del genere in un volto tanto segnato? Di certo
nessun uomo giovane, perché nessun giovane ha mai visto niente negli
anziani che non fossero i danni arrecati dal tempo.

I fuochi d'artificio salivano crepitando dalla colonna della parata,
accesi dagli opliti in armatura formale, e riempivano il crepuscolo di
colori: verde acceso, una morbida tonalità dorata, fiamme scarlatte come
stelle cadenti. Ogni impatto di colore era accompagnato da un colpo
tanto profondo da far vibrare i timpani. Gli scoppi venivano più
avvertiti che sentiti perché il loro rumore si perdeva negli applausi
della folla e nello squillare della musica. Le Sfingi di Borosevo -- i
gladiatori a cui ero sopravvissuto per due anni -- erano tutte sul
catafalco, un gradino più in basso rispetto a Dorian, armate per il
combattimento che sarebbe seguito. Dietro di loro, su catafalchi simili,
c'erano i Casati che avevano giurato di servire il conte Mataro:
Melluan, Kvar e Veisi, come pure una tribuno-cavaliere di nome Smythe e
i suoi ufficiali, in rappresentanza delle Legioni imperiali. I
catafalchi erano seguiti da un plotone di opliti che portava i colori
dei Mataro, dietro i quali procedeva un'intera centuria di legionari
imperiali con indosso la loro armatura d'avorio senza faccia,
fiancheggiati da una doppia fila di musici della banda e dai fuochi
d'artificio.

Finii il vino e lasciai il bicchiere su una ringhiera di pietra,
facendomi largo fra la folla di dignitari che inondava il palco e
aggirando con un inchino lady Veisi in cerca di un punto di osservazione
migliore. Correva voce che ci sarebbe stata una mischia, seguita da una
serie di duelli fra Dorian e i gladiatori; duelli che Dorian avrebbe
vinto, naturalmente, ma solo di stretta misura, e in modo tale che
nessuno mettesse in dubbio la sua validità. Prima di tutto questo però
sarebbero apparsi i cathar che avrebbero trascinato il Cielcin fuori
dalla sua prigione e incontro alla sua morte sacrificale.

Non mi sarebbe potuto importare di meno del giovane nobile. Era un tipo
abbastanza decente, ma insulso quanto sua sorella, anche se senza la sua
astuzia. La mia preoccupazione era la mischia. Appena un mese prima
sarei stato un sacrificabile, e non riuscivo a decidere se ero pronto a
rivedere i miei compagni di un tempo: Pallino e Ghen, Switch e Siran e
gli altri. Mi augurai che non combattessero tutti, quel giorno.

«Devi sentirti decisamente a tuo agio.»

L'accento straniero nella voce la tradì prima che la vedessi.
«Dottoressa Onderra!» dissi, sovrastando il mormorio presente nel palco
e il ruggito della folla e del trionfo. «No.» La verità era che non
riuscivo a immaginare che quel giorno al Colosso potesse esserci
qualcuno che si sentiva più fuori posto di me. Avrei dovuto essere
intento ad armarmi invece di sprecare tempo fra la nobiltà. \emph{La
	nobiltà}. Quando avevo cominciato a considerarmi qualcosa di diverso?

Quando glielo dissi si accigliò. «Allora ti manca?»

«Neppure per idea, ma preferirei essere laggiù piuttosto che quassù.
Vino?» Afferrai due bicchieri da un vassoio di passaggio e gliene misi
uno in mano, mordicchiandomi l'interno di una guancia. Dopo un momento
mi protesi in avanti per gridare: «Mi dispiace di non essere venuto a
vedere quegli ologrammi che hai menzionato, ma il mio signore il conte
mi ha chiesto di insegnare ai suoi figli le finezze dell'etichetta
jaddiana perché aspetta l'arrivo di un emissario entro l'anno, a quanto
mi è stato detto.» Quella era solo una spiegazione parziale, perché
soprattutto avevo paura di fare ancora di più la figura dello sciocco in
sua presenza.

«Sei...» Si interruppe. «Eri una sorta di diplomatico prima...» Accennò
alla parata che girava lungo il perimetro dell'arena del colosseo. Un
musicista locale danzava intono al catafalco, serrando il microfono fra
le mani mentre volteggiava quasi nuda e cantava. A un certo punto l'inno
marziale della banda si era affievolito fino a quando la banda si era
unita agli artisti che accompagnavano la cantante nel produrre una
musica rauca dominata dalla chitarra sintetica.

I ricordi mi fecero affiorare sulle labbra un sorriso dolente e mi
costrinsi a bere un po' di vino per nascondere quell'espressione mentre
distoglievo il volto. «Avrei potuto esserlo.» Quale che fosse
l'attrazione che la dottoressa esercitava su di me non avevo intenzione
di cominciare a narrarle la storia della mia vita. Quel compito era già
una sfida per gli uomini anziani, e quello non era né il tempo né il
luogo. «I piani... sono cambiati.» Di certo avrebbe capito se le avessi
detto tutto. Lei veniva dalla Demarchia e avrebbe condiviso i miei
sentimenti riguardo alla Cappellania. Lanciai un'occhiata alla forma
ingobbita di Gilliam Vas. Su Delos c'erano dei posti -- la corte stessa
della viceregina, tanto per cominciare -- dove simili abomini venivano
dati alle fiamme alla nascita. Nel guardare Gilliam pensai di capire
quell'usanza per quanto era grande il mio disagio. Controllai quel
sentimento, incerto se la mia avversione fosse per la sua deformità o
per l'uomo in sé, o se fosse dovuta a un misto delle due cose. Quanto
pensavo fosse semplice il mondo, a quei tempi, immaginando che tutti i
nemici potessero essere contorti fuori quanto lo erano interiormente.

Da allora ho imparato che non è così.

«I piani lo fanno sempre» commentò la dottoressa, bevendo un sorso di
vino prima di annuire con approvazione. Si protese verso di me, più per
proteggersi dall'essere sentita che per una forma di intimità, e disse
in un forte sussurro: «Tutto questo mi sembra un po' troppo per una
festa di compleanno, non credi?»

Pensavo che forse era un po' in piccolo per un palatino e il potenziale
sovrano del pianeta, ma del resto Emesh non era Delos, o Ares, o
Rinascimento, o uno dei mondi antichi. Avrei potuto dissentire ma agitai
una mano in segno di riconoscimento. «Ai palatini piacciono le loro
feste.»

«A casa noi ci ubriachiamo spaventosamente.»

«Lo facciamo anche qui.» Soffocai una breve risata e sollevai la mia
coppa di vino. «Credo lo facciano ovunque.»

Valka abbassò lo sguardo in direzione della parata. «Ho sentito dire che
gli Eudoriani non bevono alcolici.»

«Ricordami di non viaggiare con gli Eudoriani» replicai, impassibile,
strappandole una risatina. Accompagnai quella battuta un po' debole con
un sorso di vino e sorrisi nella breve pausa di silenzio.

Uno degli ospiti, un uomo alto che era chiaramente un palatino e
indossava un completo nei toni della terra con guarnizioni bianche, si
fece largo fra la folla con una mano sollevata. «Valka!» Nonostante
l'età avanzata si muoveva con grazia e nobile dignità. Una spada ad
altamateria ripiegata gli rimbalzava contro un fianco ossuto,
contrassegnandolo come un cavaliere. Abbassò la mano per appiattirsi la
massa di ribelli capelli bianchi il cui taglio sarebbe stato più adatto
su un ragazzo di quattordici anni, arruffato e selvaggio, anche se
poteva aver raggiunto i quattrocento anni, a giudicare dalle linee
profonde sulle guance e sulla fronte. «Valka, mia cara, è bello
vederti!»

«Sir Elomas!» La dottoressa sorrise e permise all'uomo anziano di
prenderle la mano e di baciarla, sorridendo per tutto il tempo. «Sono
felice di vederti.» Rivolsi al cavaliere un sorriso politico mentre lei
continuava: «Devi conoscere M Gibson. Hadrian, questo è sir Elomas
Redgrave, il mio sponsor. Signore, lui è Hadrian Gibson, ed è...»
S'interruppe, con un'espressione perplessa che le affiorava sul volto.
«Cosa sei, esattamente?»

Mi inchinai al cavaliere, attento a non versare il mio vino. «Un tutore.
Ho l'onore di insegnare le lingue ai figli del nostro signore.»

«Che onore... ho conosciuto quegli stronzetti.» Elomas sorrise, attento
a tenere bassa la voce. «Non ho familiarità con il Casato Gibson. Da
dove vieni?»

Memore della falsa identità che utilizzavo, sollevai una mano con fare
difensivo. «Da Teukros.» \emph{Se solo fosse stato vero}. «Noi non siamo
palatini, signore. Mio padre gestisce una compagnia di spedizioni.
Commercia in stoffe naturali con i principati\emph{.»}

«Ah.» Sir Elomas stava sfoggiando un sorriso tutto denti. «Questo
spiegherebbe tutto.» Com'era intenzione che facesse.

La dottoressa Onderra si protese in avanti, tenendo fra le mani la coppa
di vino mentre la musica saliva assordante dal terreno di parata, quasi
cancellata dagli applausi della folla. «Sir Elomas è lo zio dell'arconte
Veisi.»

«Zio acquisito per matrimonio» la corresse il cavaliere.

«Ed è lo sponsor dei miei lavori sul sito degli scavi.»

«A Calagah?» chiesi, spostando la mia attenzione da Valka a sir Elomas.

Il cavaliere si illuminò in volto. «Hai familiarità con il sito?»

«Ne ho sentito parlare solo di recente. La dottoressa Onderra ha
promesso di mostrarmi alcuni ologrammi sull'argomento.» Sorrisi. «Allora
sei un archeologo?»

Il vecchio si assestò sul fianco l'impugnatura della spada e si sistemò
la cintura con lo scudo sopra la giacca del completo. «Solo un
appassionato dilettante.»

«È di certo il genere migliore.» Lanciai un'occhiata a Valka, poi mi
fissai i piedi, ricordando che lei e io avevamo avuto una sorta di
dibattito proprio su quell'argomento durante il nostro primo incontro.

«Ah, sir Cavaliere! Bentornato.» La voce autoinvitata aveva un familiare
tono aristocratico e strascicato. «Lieto di vederti tirare fuori la
testa da quel buco nel terreno abbastanza a lungo da venire a...»
Gilliam Vas ammutolì quando lo sguardo dei suoi occhi spaiati si posò su
di me. «Tu!»

Al mio fianco sentii Valka indietreggiare di un passo per la sorpresa,
ma mi limitai a spostare sull'intus il mio sorriso cortese. «Vostra
reverenza, è passato un po' di tempo dal colosseo.»

L'intus arricciò il naso come se avesse pestato qualcosa di immondo e il
suo polso ebbe un piccolo sussulto -- senza dubbio reprimendo a forza
l'impulso a tirare fuori il fazzoletto. «Avevo sentito che il conte
aveva ammesso un plebeo nel castello di Borosevo,» disse poi «ma non mi
aspettavo che fosse sceso tanto in basso da ammettere qui te.» Valka e
sir Elomas mi guardarono entrambi con un'espressione che somigliava alla
curiosità. Al di sotto del mio falso sorriso serrai la mascella,
lasciando così alla creatura il tempo che le serviva per continuare:
«Quest'uomo mi ha aggredito nelle gallerie del colosseo mentre lavoravo
per mettere al sicuro la preda di Sua signoria.»

«La preda?» Valka guardò con aria confusa il prete gobbo, contraendo le
sopracciglia.

Gilliam gonfiò il suo torace da piccione. «Per il trionfo.»

«Si riferisce al sacrificio» spiegai. Gli eventi del giorno erano ormai
noti a tutti. «Al Cielcin.»

«Taci, barbaro.» Il brutto prete serrò le mani davanti a sé; non stava
bevendo vino. Per un assurdo momento sembrò una parodia eudoriana di un
prete, perché la lieve asimmetria del cranio faceva apparire la faccia
cerea più come una maschera che come carne. Lo fissai socchiudendo gli
occhi e cercando di memorizzare quei lineamenti: disegnarlo sarebbe
stata una sfida. Decisi che appariva come avrebbero dovuto fare tutti i
preti della Cappellania: come se fosse stato il volto segreto di
quell'istituzione debosciata. «E sei anche quello che ha fatto irruzione
nella sua cella.» Non era una domanda, e potevo vedere gli ingranaggi in
funzione nel suo cervello. «Mi sorprende che tu stia ancora
respirando... oppure il conte sta avviando un harem?»

«La gelosia non ti si addice, Gil» interloquì sir Elomas, battendo un
colpetto sul braccio dell'intus con fare amichevole. «Però sono certo
che da qualche parte in città c'è una prostituta disposta ad
\emph{accettare} te.» Pronunciò la parola `accettare' con una
vendicativa alacrità che accese una fiamma nei suoi occhi color muschio
e mi costrinse a reprimere un sorriso, tanto per la frecciata quanto per
la segreta consapevolezza che io ero quello che Gilliam desiderava
essere. Se lo avesse saputo, non avrebbe osato parlarmi come stava
facendo, indipendentemente dalla sua carica ecclesiastica. Le mie dita
si posarono per un momento sull'anello del Casato, appeso a un cordino
sotto la camicia

Gilliam Vas rivolse il naso verso il cielo. «Stai attento, signore.
Ricorda che è mia madre ad approvare la tua piccola spedizione. Una
parola da parte mia, e...»

Elomas si mostrò del tutto imperturbato. «Perché sei venuto qui,
reverenza?» Aveva abbandonato la familiarità di poco prima. Questo mi
indusse a farmi delle domande e archiviai la cosa per rifletterci sopra
più tardi.

«Volevo farti le mie congratulazioni per la nascita di tua nipote.»
Gilliam tirò \emph{infine} fuori il fazzoletto e se lo premette sul
naso. Dal tessuto veniva un vago profumo di caffè e di cannella. Da
sopra quel panno appallottolato mi fissò con occhi roventi. «Ma vedendo
con chi ti accompagni... barbari e prostitute eretiche...» Nel parlare
ritrasse in un sogghigno il labbro superiore con aria di disprezzo e
lanciò un'occhiata a Valka... «credo che quella cortesia fosse fuori
posto.»

Per un momento dimenticai la mia condizione immaginaria e Hadrian
Marlowe parlò dalle mie labbra al posto di Hadrian Gibson o di Had il
mirmidone. «Fino a questo momento, prete, non credevo che una cortesia
potesse essere fuori luogo.» Mi accorsi che mi stavo accigliando e mi
costrinsi a rilassare la fronte, lieto di avere una risposta alla mia
precedente domanda: quello che odiavo era l'uomo, il suo aspetto era
secondario. Quali che fossero i miei pregiudizi.

Gilliam snudò i denti mentre cercava una risposta. Ne trovò una e aprì
la bocca per replicare, ma le sue parole svanirono, trascinate via da un
sibilo terrificante che si levò come vapore dalla folla, inducendo
quelli che non vedevano bene a spostarsi sull'orlo della sedia e a
rivolgere lo sguardo in alto, verso gli enormi schermi che proiettavano
le riprese di quanto accadeva nell'arena. Quante volte il mio volto
protetto dall'elmo si era trovato in mezzo ai miei compagni mirmidoni?
Mi pareva quasi di non aver mai visto prima quegli schermi -- quel
colosseo -- in considerazione di quello che vi stavo vedendo adesso: lo
xenobita, pallido come la morte. Gli avevano dato una tunica, uno di
quel dishdasha bianchi che i chierici indossano sotto le vesti. Le corna
avevano cominciato a ricrescergli, trasformando la sua cresta in uno
scempio di bozzi irregolari e i capelli bianchi che sporgevano dalla
parte posteriore della testa erano vaporosi nell'aria che li agitava.
Serrò gli occhi enormi, segno che la sua biologia aliena non era immune
agli effetti del sole di Emesh, grande quanto un pugno. Nella violenta
luce diurna, Makisomn appariva meno simile a un uomo e sembrava
piuttosto una qualche creatura precambriana estratta da un camino
vulcanico. Le narici fessurate si dilatarono e snudò quei denti bianchi
come il latte in un ringhio. Urlando, la folla gli scagliò contro
bicchieri e pezzi di cibo: l'arcinemico dell'umanità era arrivato.
Intanto la musica formale ricominciò, con un tempismo perfetto deciso da
qualche astuto pianificatore, dominata da un rullo di tamburi grossi
come veicoli da terra.

L'esibizione raggiunse il suo culmine quando i due cathar emersero
dall'ombra del vomitorium opposto a quello da cui era uscita la parata.
Gli altri catafalchi si erano distribuiti a ventaglio, con i legionari e
gli opliti dei Mataro disposti in unità compatte alle estremità più
lunghe del piano ellittico. Il giovane lord Dorian scese dal suo
catafalco accompagnato dalle Sfingi di Borosevo armate di lancia come da
una guardia d'onore, unendosi sul campo ai due cathar mentre dal nostro
palco la grande priora Ligeia levava la voce -- un gemito da strega -- e
cominciava l'invocazione. «La Terra ci ha abbandonati, è svanita nel
Buio.» Lasciò che quelle parole rimanessero sospese nell'aria per un
momento, attendendo l'abituale risposta.

«Ci ha abbandonati.» Dopo quelle parole la folla scivolò nel silenzio.

«Siamo benedetti, noi figli dell'assente Terra» dichiarò la grande
priora.

«Lei ritornerà.»

C'è qualcosa di estremamente umano in silenzi come quello,
nell'immobilità soprannaturale di cinquantamila anime. Tutti loro erano
intimoriti dal peso di quel silenzio quanto lo ero io stesso, animati da
quella comunione spirituale che l'umanità chiama Dio. Accanto a me, la
dottoressa Onderra osservava impassibile con una strana espressione
sulle labbra. Cercai di non fissarla e ripresi a studiare da dietro la
priora mentre continuava il suo rito accompagnata dai rintocchi delle
campane che si levavano dalla città, al di là del colosseo, e dal
castello di Borosevo al di sopra di essa.

«Questo è un giorno di gioia,» riprese la priora, la cui voce veniva
amplificata e diffusa nell'arena «un giorno da celebrare, perché oggi il
figlio del nostro signore, Dorian del Casato Mataro, diventa
maggiorenne. Da oggi è un uomo!» Se fosse stato il conte a portare
avanti il rito, la gente forse avrebbe applaudito, avrebbe sentito più
lo spirito della celebrazione che della cerimonia, ma le parole della
grande priora furono accolte solo dal silenzio e dal chinarsi delle
teste con pia reverenza.

Unità mediante la giustizia. Giustizia mediante la devozione. Devozione
mediante la preghiera.

Non sono mai andato su Vesperad, ma so come ragionano.

Lei stava ancora parlando. «Contemplate questa bestia, questo demone! Un
figlio del Buio, catturato in battaglia dai nostri coraggiosi guerrieri
e ora divenuto un sacrificio! Un modo per ricordare che il Buio non
resisterà, che le sue creature non saranno vittoriose. Le stelle sono
nostre, così ha detto la Madre Terra! Sono là per noi!» In basso, uno
dei cathar estrasse una spada enorme da un fodero avvolto in quella che
sapevo essere pelle umana. La lama era tanto lunga che il cathar si
dovette inginocchiare in modo che l'altro la potesse estrarre, poi
entrambi si avvicinarono, piazzandosi ai lati del giovane nobile come
due ombre.

«Lo ucciderà lui stesso?» sussurrò Valka, che mi aveva seguito sul
davanti del grande palco, e mi toccò un braccio, d'un tratto molto
vicina.

Il mio sguardo rifiutava di distogliersi dal Cielcin, ma scossi il capo.
«Improbabile. Farebbe solo un pasticcio.»

«Non sono... ciechi, vero?» sibilò. «I vostri preti, intendo.»

«No, riescono a vedere attraverso quella benda che è meramente
simbolica, perché l'icona della Giustizia della Cappellania è cieca.»
Spostai lo sguardo dallo schermo alla scena effettiva che si stava
svolgendo nell'arena del colosseo. Il Cielcin venne costretto ad alzarsi
e due legionari lo trascinarono giù dal catafalco. Da come procedeva
accasciato in mezzo a loro, con le gambe molli che si intralciavano a
vicenda, compresi che lo avevano drogato. «Lo hanno sedato» dissi,
serrando la mascella. «Vigliacchi.» Ma era teatro, tutto teatro, come le
commedie in maschera degli Eudoriani o le opere olografiche di mia
madre.

Non sentii il resto delle chiacchiere della priora, o meglio mi rifiutai
di ascoltarle perché il mio sguardo era fisso sulla Spada Bianca, la cui
lama di ceramica era lunga quasi cinque piedi e riva di punta. Era un
oggetto ridicolo, decisamente troppo ingombrante per usarlo in
battaglia, e risplendeva di un tono fra il bianco e l'argento, spettrale
nella luce diurna arancione di quel cielo di un tono fra il rosso sangue
e il crema. Serrai la mano libera intorno alla ringhiera, con le vene
che spiccavano sotto la pelle. A un ordine di Ligeia Vas, i legionari
costrinsero il Cielcin a inginocchiarsi. «Possono razziare le nostre
città, bruciare i nostri mondi, ma non ci spezzeranno mai!» La sua voce
era simile al secco fruscio di un ramo di legno contro la pietra.
«Guardalo, o popolo! Contempla il demone, il nostro grande nemico!»
Grugnii per l'uso improprio dei pronomi, ma la cosa parve importare solo
a me. «Ricacceremo lui e la sua razza oltre le stelle e nell'oscurità
finale da cui non c'è ritorno.»

Quando giunse, il colpo non scese dall'alto, perché quel cathar ci
teneva troppo a fare spettacolo. I legionari che trattenevano il Cielcin
gli torsero le braccia all'indietro per il tempo necessario perché il
cathar potesse far roteare la spada in un arco orizzontale che staccò la
testa di Makisomn dalle spalle. Un fiotto di sangue non umano, nero come
olio, si riversò lungo il davanti del corpo, seguito da un ansito e da
un getto d'aria. Sperimentai un momento di delusione estremamente
profonda, creata ad arte per sminuire la creatura morta.

\emph{Guardate con quanta facilità può morire}.

Gli applausi arrivarono un momento più tardi, riversandosi sull'arena
come una tempesta, cinquantamila persone che urlavano, abbandonando il
precedente decoro religioso. Distolsi lo sguardo, appuntandolo sulle mie
scarpe, mentre la folla lanciava stelle filanti di seta sintetica
bianca, verde e oro che fluttuarono pigre sulla testa di chi occupava i
posti vicini e si depositarono sul campo di battaglia dell'arena.
Sollevai lo sguardo giusto in tempo per vederle posarsi come una parodia
di neve, irradiando come le striature di un occhio umano dalle figure
del giovane nobile e dei cathar che lo accompagnavano.

Il secondo cathar ripulì la lama con un panno bianco che poi ripiegò con
precisione a metà, pressandolo prima di riaprirlo perché tutti lo
vedessero. La pressione aveva creato una chiazza simmetrica sull'altro
lato, un sigillo come quello che gli antichi mistici usavano mostrare
agli uomini per scrutare nella loro anima. Ultimata quella
presentazione, il torturatore ripiegò il panno, se lo appese al braccio
e aiutò il suo fratello assassino a riporre la Spada Bianca nel fodero.
Fatto questo afferrò la testa dell'alieno, affondando le dita nella
frangia epoccipitale che cresceva dietro la corona di corno, si
inginocchiò e offrì la testa a Dorian, che per poco non la fece cadere
nel sollevarla per i capelli in modo da mostrarla alla folla.

Continuando a fissarmi i piedi sussurrai una singola parola in cielcin:
«\emph{Udatssa}.»

`Addio.'


