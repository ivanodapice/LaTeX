\chapter{Lingua di demone}

Con i Cielcin alle nostre spalle storditi o morti e i legionari di
Bassander che si affrettavano a venirci in aiuto, riprendemmo ad
avanzare seguendo le vaghe istruzioni del Cielcin con la consapevolezza
che probabilmente era una trappola. Tenevo ancora in mano il
disintegratore a fase jaddiano che pendeva dalle mie dita inerti con il
calcio d'ottone che brillava sotto la luce delle nostre tute. Lo fissai
mentre procedevamo sempre più in profondità fino ai tunnel che correvano
diritti e alle strisce di nastro fosforescente che contrassegnavano
passaggi a me familiari.

Eravamo nelle vicinanze del sepolcro, quella stanza a forma di serratura
dove Anaïs mi aveva baciato. I legionari ci raggiunsero a quel punto e
dopo una breve spiegazione di quello che era successo Bassander mandò
quattro dei suoi soldati indietro lungo il tunnel perché raggiungessero
la tenente Azhar e i mamelucchi che Olorin aveva lasciato a sorvegliare
i tre prigionieri e il cadavere del Cielcin che il maeskolos aveva
ucciso.

«Stai bene, signore?» mi chiese Bassander, avendo appreso della mia
ordalia nella grotta superiore.

Non avrei saputo dire se erano stati preoccupazione o interesse
personale a motivare la domanda, ma annuii. «Hadrian» mormorai. «E, sì,
sto bene.»

Il tenente della Legione, il cui volto era nascosto dietro l'arco
convesso di ceramica bianca, annuì a sua volta. «Allora fai strada. Voi
due!» Indicò due legionari armati di fucile al plasma.
«All'avanguardia... coprite Marlowe e il maestro di spada, che non sono
equipaggiati per questo.» Poi, come parlando a sé stesso, aggiunse:
«Stupido, dannato rischio.» Per fortuna Olorin non lo sentì. «A che
livello è la carica del tuo scudo?» chiese quindi, tornando a rivolgersi
a me.

Controllai. «Ottantuno percento.»

«Basterà.»

La via di accesso al sepolcro era un singolo corridoio pianeggiante, con
le pareti inclinate leggermente verso l'interno che creavano una sezione
trasversale e uno spazio abbastanza ampio perché tre uomini potessero
camminare affiancati.

«Sensori attivati» disse uno dei legionari, fermandosi per controllare
il terminale inserito nella tuta. «Nessun segno di vita.»

«Hanno nascosto un'intera navetta» ribatté Bassander, segnalando di
tenersi pronti e ordinando ad altri soldati di passare all'avanguardia
con un gesto secco della mano sinistra. I mamelucchi fluirono in avanti
in risposta a un processo autonomo del loro cervello di omuncoli che
diceva loro di andare a colmare i vuoti che si erano aperti nella
formazione di Bassander.

«Dovrebbero essere soltanto in sette» dissi, sempre con lo sguardo
basso. «Quello con cui ho parlato ha detto che erano in undici.»

«A meno che non intendesse che ce ne sono undici davanti a noi.» Olorin
si girò a guardarci. «Non mi piace affatto questo passaggio, ci limita
nei movimenti.» Mi guardai alle spalle, dove gli uomini di Bassander --
tre davanti in ginocchio e due alle loro spalle -- avevano bloccato il
corridoio. Da qualche parte l'acqua gocciolava nella camera davanti a
noi, condensa dovuta all'umidità dell'aria di mare che colava lungo le
antiche colonne vetrose intorno alla struttura crepata dell'altare. Mi
sentii trasportare altrove. La nostra non era una colonna di soldati ma
la processione funebre di mia nonna che ancora una volta scendeva nel
mondo sotterraneo. È strano come ricordi del genere dominino la nostra
vita, echeggiando attraverso il tempo in posti dove non hanno ragione di
essere.

\emph{Drip-drip-drip}.

Se non fosse stato per quel lieve rumore e per il frusciare di passi
silenziosi, una quiete spaventosa si sarebbe stesa su quel posto buio.
Era davvero appropriato che dovesse succedere proprio lì, fra tutte le
camere di Calagah. La Quiete aveva costruito le gallerie con un solo
punto focale, una serie di diramazioni, di cappi e di spirali ascendenti
e discendenti che portavano tutte a quel singolo vicolo cieco, come se
le loro riflessioni aliene fossero sempre tornate a quell'unica tesi, a
quella singola idea. Le parole del Cielcin mi risuonarono nella testa.

`Non è per voi.'

Per noi? Si era chiaramente riferito all'umanità. Ebbi l'improvvisa
sensazione di trovarmi nel cuore di un vortice, nell'occhio di un
uragano che non potevo né vedere né comprendere.

«Marlowe.» Olorin mi assestò una spinta. «Continua.»

In quel momento avrei potuto trovarmi al sicuro su una navetta di
ritorno a Fonteprofonda insieme a Valka e a sir Elomas, avrei potuto
essere in attesa di un altro giorno di combattimenti nel colosseo o di
furti lungo i canali. Avrei potuto essere curvo su un indice nello
scriptorium degli scoliasti -- come lo sono mentre stilo questo
resoconto -- o chino su un prigioniero nella bastiglia di Vesperad.

\emph{Drip-drip-drip.}

Invece ero lì, quasi accoccolato in una galleria sotto metri di basalto,
sulla riva di un mare incostante, a caccia di xenobiti fra rovine ancora
più aliene. Il corso della mia vita non è mai fluito in modo liscio, i
suoi momenti sconcatenati, che sono collassati inesorabilmente verso
questo, sono stati causati da tutto quello che li ha preceduti e dalla
mia singola, diretta dichiarazione rivolta alla centurione sulla
spiaggia. `Posso essere di aiuto.' Le mani mi tremavano ancora. Aiuto,
come no. Però trovai la forza e le parole che mi servivano.
«\emph{Kavaa...}» `Salve.' Era una piccola parola, schiacciata dalla
tensione nervosa e dalla bile che mi salivano dentro al pensiero di quel
piccolo episodio nella grotta. Ritrovai l'uso dei polmoni e tentai
ancora. «\emph{Kavaa, Cielcin-saba}! \emph{Bayareto okarin'ta
	ti-kousun'ta}!» `Salve, Cielcin! Siete tutti circondati!' Avanzai,
muovendomi in modo tale che solo un sottile cordone di soldati rimase
fra me e l'ingresso del sepolcro.

Non era una traduzione letterale, e avevo dovuto avanzare delle
supposizioni. Formulando l'affermazione in quel modo, al passivo
intendevo proiettare sul gruppo dei Cielcin il genere ricettivo
femminile che speravo si annidasse nella camera, davanti a noi. Dato che
si trattava di soldati, sapevo che era d'uso ricorrere al maschile e che
mi stavo mostrando scortese. Sempre in Cielcin, aggiunsi: «\emph{Nasca
	nietiri}!» `Voglio parlare!'

Da bambino mi era stato insegnato non solo a parlare ma anche a tenere
discorsi, e nel crescere la mia voce si era irrobustita. Ero stato
preparato per sedere sul seggio nero di Meidua, sotto la Cupola delle
Incisioni Radiose e per governare un continente. Avevo una buona voce, e
quella notte risuonò nello spazio silenzioso, riecheggiando contro le
pareti. Quando penso a me stesso, spesso mi vedo così: in piedi nel
buio, rischiarato alle spalle dalla luce delle tute dei legionari e dei
mamelucchi, ma percepisco anche un'ombra che si abbatte su quella scena,
proiettata non da me o dalle forme dei due soldati al mio fianco ma dal
sole di Gododdin, che avrei distrutto. A volte ho la sensazione che
essere là all'imboccatura di quella camera sia stato come trovarsi sul
ponte del visitatore mentre guardavo esplodere il sole. Nella mia
memoria quella che pervade la scena non è la luce bianca delle tute ma
quella del sole assassinato, proiettata all'indietro attraverso il
tempo.

\emph{Drip-drip-drip}.

Ripetei la mia dichiarazione un paio di volte, con la voce che
rimbalzava sulle dure pareti dell'alta camera. Dopo la terza, battei il
disintegratore contro la spalla, ve lo lasciai appoggiato e gridai in
cielcin: «C'è nessuno là dentro?»

«Siamo qui.» Dal buio giunse una voce simile alla fine del mondo. «Siete
in pochi.» La voce di chi aveva parlato era più alta di quella del
Cielcin che avevo interrogato. «E siete piccoli. Alcuni di noi
potrebbero fuggire.»

«Oltrepassando tutti i miei soldati?» Dovevo sembrare loro un primitivo,
un bambino, come aveva detto Makisomn. «Non mi piacciono le vostre
probabilità di farcela.»

Dall'oscurità si levò un suono acuto e freddo come il vento che sfiora i
merli del Riposo del Diavolo nel cuore di un inverno delosiano.
Involontariamente, mi sentii rabbrividire. Poi la voce parlò ancora,
cominciando con un lungo sibilo, come di gas che sfugge da un dirigibile
grande quanto una piccola luna. «\emph{Canasam ji okun ti-koarin'ta
	ne}?» `Ci stai minacciando?'

«\emph{Canasa ji ne}?» ripetei, sinceramente incredulo e non sapendo se
quell'emozione si traducesse in modo adeguato. `Minacciarvi?' «È ovvio
che vi sto minacciando» ribattei, lanciando un'occhiata da sopra la
spalla a Bassander e a Olorin. «Deponete le armi. Arrendetevi.»

«Arrenderci?» Ci fu di nuovo quel suono acuto e freddo. Indignazione?
Una risata? Non avrei saputo dirlo. «Perché ci dovremmo arrendere?»

«\emph{Siajenu iagari o-peryuete, akatha.}» Aprii le mani in una sorta
di scrollata di spalle, lasciando che la mano si allentasse intorno
all'arma. `Perché non avete dove andare.'

«Il Popolo non si arrende agli animali!» gridò un'altra voce aliena, più
profonda della prima.

«Taci!» ingiunse una terza voce, e aggiunse qualcosa che non riuscii a
capire.

«Ascoltatemi!» gridai in cielcin. «\emph{Ubbaa}!» Le voci aliene
tacquero. Ebbi un'intuizione, una crescente realizzazione, la sensazione
di quello che potevano provare. «Qualsiasi cosa abbiate sentito sul mio
popolo, qualsiasi storia... io non vi farò del male.» Cercai di non
pensare al Cielcin che avevo torturato con il disintegratore a fase ai
piani superiori.

«Menti!» ribatté la seconda voce.

«Se combattete, morirete sicuramente» risposi senza esitazione.
Oltrepassai le due guardie in posizione all'estremità del corridoio e
avanzai di un passo esitante nella penombra rischiarata dalle tute.
«Siete arrivati fin qui, soldati, non gettate via tutto in un'ultima
spinta finale. Gettate le armi e provvederò personalmente a che ciascuno
di voi torni a casa vivo.»

\emph{Drip-drip-drip}.

Alla fine vidi chiaramente uno di essi quando emerse dall'oscurità,
vomitato dall'ombra come un figlio della notte, tutto metallo nero
flangiato e gomma, con la faccia fortunatamente nascosta da una
maschera. Era troppo alto, troppo esile per essere reale. «E chi sei tu
per promettere qualcosa?»

«Ho una linea di tiro pulita» disse uno dei soldati accanto a Bassander
Lin.

«No!» sibilai in galstani. Cosa potevo rispondere al Cielcin? Cosa
potevo dire che avesse un qualche significato? Pensai alle storie che
avevo sentito da bambino, di viaggiatori che avevano fornito il loro
nome ai Cielcin solo per essere vincolati in schiavitù, ingannati come
Faust era stato indotto con l'inganno dall'astuto diavolo a consegnargli
la sua anima per sempre. Solo che i Cielcin non erano i diavoli, io lo
ero. Anche così, qualcosa mi trattenne dal fornire subito il mio nome.
«Sono qualcuno che sta combattendo una guerra che non è stato lui a
cominciare. Questa non è la mia guerra più di quanto sia la tua,
soldato. L'abbiamo ereditata dai nostri genitori, voi come noi.
Arrendetevi e potremo farla finire.»

Quella voce alta e fredda riempì l'aria fra di noi. «Sei un soldato o un
prete, creatura?»

\emph{Drip-drip-drip}.

«Sono soltanto un uomo!» replicai, senza soffermarmi a riflettere sulla
mia risposta. «Qui però sono il solo che sappia parlare la vostra
lingua.» In cielcin la parola `uomo', con il genere in forma attiva,
indicava qualcuno che agisce, un creatore di azioni, e non soltanto il
sesso di appartenenza.

«Un uomo?» ripeté una voce nuova. «Cos'è questo? Uno scherzo?»

«Nessuno scherzo» dichiarai, gettando il disintegratore in direzione dei
soldati e della galleria alle mie spalle. «Solo la verità. Tratterete
con me?»

Più avanti si sentì di nuovo un borbottio, poi risuonò la quarta voce.
«Tratterò io con te. Parleremo.» In quell'istante notai un cambiamento,
non nel tono alieno della voce, ma nella scelta delle sue parole. Finché
non aveva parlato, la conversazione era stata male assortita, con
ciascuna parte che usava il genere maschile attivo per descriversi,
definendo così la controparte come femminile e ricettiva, ma in
quell'ultima affermazione la quarta voce -- che intuii essere quella del
comandante -- aveva usato il femminile per descrivere sé stesso e i suoi
soldati. Mi resi conto di quanto fosse indicatore quel cambiamento.
Avevano perso il vantaggio. Fino a quel momento non mi ero reso conto di
come la forma del linguaggio potesse mostrare assenso, con tutti che
parlavano nella stessa modalità. L'aspirante scoliasta che era dentro di
me ne fu silenziosamente affascinato, ma venne messo a tacere da quella
parte del mio io che pensava con la voce di mio padre. Ci sarebbe stato
tempo per questo in seguito.

Parlando in galstani, mi girai e mi rivolsi a Bassander e a Olorin, al
di là dei soldati. «Vado a parlare con il loro comandante, vicino
all'altare.»

Se rimasero tutti sorpresi nel sentire quella notizia non lo diedero a
vedere. Dopo un secondo, Bassander Lin oltrepassò i suoi soldati e
replicò: «Vengo con te.»

Annuii. «Sì, certamente.» Mi guardai le mani: avevano smesso di tremare.
«Credo che funzionerà.»

\begin{figure}
	\centering
	\def\svgwidth{\columnwidth}
	\scalebox{0.2}{\input{divisore.pdf_tex}}
\end{figure}

L'eternità è silenzio, è la quiete del mondo, nell'oscurità e nella
solitudine del cuore. Queste sono le cose che trasformano in eternità i
nostri istanti, che fanno del tempo necessario a percorrere quaranta
piedi di pavimento spoglio interi eoni. Mentre avanzavo precedendo
leggermente Bassander Lin mi parve di poter sentire su di noi -- su di
me − il peso di tutti quegli sguardi alieni. Se pure trovarono da
obiettare alla presenza del tenente, i Cielcin non lo diedero a vedere e
rimasero in silenzio, vigili come le stelle onnipresenti che si
annidavano da qualche parte al di là della pietra che formava il tetto
del nostro mondo. Bassander e io ci fermammo, soli in quell'Impero di
silenzio.

Poi si sentì un passo pesante. Un altro. Un terzo.

Il capitano dei Cielcin emerse dal buio, appoggiandosi pesantemente
contro uno dei pilastri inclinati intorno all'altare, favorendo una
gamba. Portava il casco nel cavo del braccio lungo quanto un uomo è alto
e nell'apparire ritrasse le labbra dai denti affilati. Non mi venne in
mente nessuna saggia parola scoliastica, nulla che arginasse la marea di
rabbia o di paura, ma non mi servivano le parole perché non c'era
rabbia, e neppure paura. Mi sentivo pulito. Lucido. Pronto. Mentre mi
immergevo nel buio, lontano dalla luce e dall'ordine di quel mondo
ordinato in cui avevo faticato per tanti anni, le preoccupazioni che gli
erano proprie si dissolsero.

«Dimentico quanto siete piccoli voialtri.» La voce del Cielcin era arida
come paglia, come un osso, prosciugata di ogni vitalità. Quella creatura
era più bassa di quella su cui avevo usato il disintegratore nella
caverna sovrastante ma misurava comunque almeno otto piedi. I lineamenti
del suo strano volto erano più appuntiti, gli occhi più inclinati che
rotondi, i capelli raccolti in una treccia che ricadeva sulla spalla
sinistra.

Mi fermai prima di parlare, notando che la sua mano era premuta contro
il fianco, appena al di sotto del punto in cui in un essere umano ci
sarebbero stati i reni. «\emph{Tuka okarin ikuchem.}» `Sei ferito.' Usai
il femminile-ricettivo, con costruzione passiva. L'alieno non obiettò.

«\emph{Eka}» confermò, ruotando la testa in senso orario. `Sì.' Poi
aggiunse: «Non è niente di serio, la tuta ha assorbito il peggio. Posso
parlare per i miei uomini.»

«Ti credo.» Accennai al tenente che era al mio fianco. «Cielcin, lui è
Bassander Lin, un...» Cercai una parola che potesse esprimere cosa
significava `tenente'. «Un piccolo capitano.» Quando Lin notò che avevo
usato il suo nome, ripetei per lui in galstani quello che avevo detto,
poi tornai a rivolgermi al Cielcin. «Il mio nome è Hadrian Marlowe.»

«Hadrian...» Il Cielcin cercò di pronunciare il mio nome con quella
bocca tutta denti. «Marlowe.» Ritrasse di nuovo le labbra e nella nuda
luce bianca della tuta di Bassander vidi che le sue gengive erano nere.
«Io sono Itana Uvanari Ayatomn, un tempo \emph{ichakta} della nave che
avete abbattuto.» \emph{Ichakta} significava `capitano'. Puntò un dito
verso la volta che ci sovrastava, gemette, sibilò come un gatto ferito e
si accasciò contro la colonna.

«Vi arrenderete?» domandai, sempre in cielcin.

«Puoi garantire la sicurezza dei miei uomini?» Mi squadrò da testa a
piedi e compresi cosa stava vedendo: un piccolo uomo che non era vestito
per la guerra, teso, con i capelli arruffati che gli ricadevano sulla
faccia. I dettagli del mio vestiario dovettero portare a una maggiore
comprensione, superando l'abisso fra le specie: non ero un soldato e non
lo sarei mai stato.

«No, non posso.» Sapevo che poco prima avevo detto il contrario, quindi
moderai quella rivelazione aggiungendo: «Ma ci posso provare e lo farò.»
Lanciai un'occhiata al tenente Lin e gli espressi la preoccupazione del
Cielcin.

Lui scosse la testa protetta dal casco. «Non so cosa ne farà di loro la
tribuno-cavaliere, Marlowe.»

«Non glielo posso dire» ribattei, passando al galstani per rispondergli
meglio. Lui si limitò a scrollare le spalle, il volto invisibile dietro
il casco e le mani pronte sul calcio del fucile al plasma.

«Che alternative ha?» Potei quasi sentire le sopracciglia di Bassander
che si inarcavano dietro il casco.

Ripetei la domanda a beneficio dell'\emph{ichakta} dei Cielcin, che
snudò i denti simili a vetro in un ringhio velenoso e ruotò la testa in
senso antiorario. «C'è sempre una scelta.» Inclinò la testa verso l'alto
e la confusione di colonne storte e di nere arcate che sosteneva la
lontana volta del soffitto. «Il Popolo non dimentica mai come morire,
\emph{yukajji-do}.» Era strano sentire il termine Popolo --
\emph{Cielcin} -- pronunciato in quel contesto. Lo dicevano in modo
diverso da noi, con consonanti dure e aspre. Forti.

«Vuoi morire?»

Uvanari mi guardò dall'alto della sua statura, con le labbra leggermente
distese come quelle di un cane sul punto di ringhiare. Le fessure delle
narici si dilatarono, poi distolse di nuovo lo sguardo. «\emph{U
	ti-wetidiu ba-wemuri mnu, wemeto ji}.» Sembrava poesia, come una
citazione di una scrittura. Tradussi lentamente:
\phantomsection\label{fileintero-72.xhtml__idTextAnchor006}{}`Nel tempo
della morte, noi moriremo.' Soffocai un gemito. Se non era poesia era di
certo una citazione. Per la Terra e l'imperatore, era come parlare con
me stesso. Lottai per incorporare quella nuova filologia nella mia
comprensione del capitano alieno, che sfruttò il mio silenzio come
un'opportunità per aggiungere: «Ci sono posti peggiori dove morire.»

Quello fu un errore perché mi rese facile rispondere. Consapevole degli
altri Cielcin annidati nel buio ribattei: «Forse, ma non oggi! Nessuno
di voi deve morire \emph{oggi}. Arrendetevi.» Infusi nel mio tono quanta
più urgenza possibile, pregando non so chi che quell'emozione riuscisse
a valicare il vuoto fra le specie. \emph{Fai che Platone avesse
	ragione.} «Arrendetevi, gettate le armi e non vi faremo del male.
Potrete uscire da questo posto. Nessuno di voi deve morire qui.» Volevo
spiegare che ero un nobile, che potevo offrire protezione, per quel che
valeva, ma se i Cielcin avevano concetti equivalenti nella loro cultura
io non li conoscevo, e questo mi rendeva muto.

Sentii movimenti nell'ombra, una serie di sussurri. Uvanari si guardò
indietro da sopra la spalla e gridò qualcosa di rozzo e incomprensibile
per farli tacere. Il Cielcin con la voce profonda ribatté troppo in
fretta perché potessi capirlo.

«Cosa succede?» domandò Olorin dalla soglia, cogliendo l'opportunità per
imitare il comportamento del Cielcin.

«Non ora» gridò di rimando Bassander.

Ingiunsi a entrambi di tacere con un gesto e avanzai di un passo.
«Questa non è la nostra guerra, capitano. L'abbiamo ereditata, tu e io e
tutti noi, e andrà avanti soltanto finché il tuo popolo e il mio saranno
disposti a morire per essa.»

«Nessuno del Popolo si è mai arreso alla tua specie, non in tutte le
nostre generazioni.»

«Allora è tempo di cominciare.» Pronunciai quelle parole senza
esitazione, senza pensare. Con convinzione. «Il tuo sacrificio e quello
dei tuoi uomini non cambierà nulla.» Poi feci quella che forse fu la
cosa più stupida che abbia mai fatto in tutta la mia vita... camminai
intorno al capitano e mi fermai fra lui e i suoi uomini nascosti, senza
sapere se fossero sei o dieci. Bassander emise un verso soffocato ma non
si mosse per seguirmi e mi augurai che i fucili al plasma dei soldati
fossero più efficaci degli storditori contro i Cielcin, nel caso avessi
avuto bisogno che lo fossero. «Deponi le armi, capitano. Per favore.»
Uvanari emise quel suono acuto e freddo che avevo sentito in precedenza,
poi sussultò e si serrò il fianco.

Mi spostai in avanti per sorreggerlo prima che crollasse in ginocchio.
Dietro di me gli altri Cielcin sibilarono e uno di loro si affrettò ad
avanzare anche se Bassander stava portando il fucile alla spalla. «State
indietro!» L'altro Cielcin comprese il senso delle sue parole e sollevò
le lunghe mani con sei dita per mostrare che non era armato.
«\emph{Lenna udeo, Tanaran-kih}!» ordinò Uvanari, facendo eco all'ordine
del tenente. L'altro Cielcin, che intuii chiamarsi Tanaran, si
immobilizzò dove si trovava. Era diverso dagli altri che avevo visto,
non portava una goffa tuta corazzata come il capitano ma indossava un
indumento aderente a portafoglio con strette maniche che mi ricordava le
tute da combattimento che avevo visto una volta indosso a una
gladiatrice nipponese in visita a Meidua, da ragazzo. I capelli bianchi
erano arruffati, tagliati alla meglio sulle spalle, e la bocca era
aperta con aria incerta. Qualcosa nella geometria del volto, nella
tensione della pelle alla base della corona occipitale mi indicò che
quello era forse un Cielcin molto giovane. Quando non si mosse, Uvanari
ripeté il suo ordine: «Stai indietro.»

«\emph{Tuka udata ne}?» chiesi, adagiando il capitano con la schiena
contro la colonna che aveva retto il suo peso.

«Non è niente, una costola rotta.» Sangue nero, ancora caldo e denso,
aderiva agli strati di tessuto isolante visibili in mezzo alle piastre
di carburo dell'armatura. La tecnologia di quella tuta era davvero
antica, indietro di secoli rispetto a qualsiasi cosa noi avessimo, e
questo mi sembrò sbagliato. Di certo una specie che come i Cielcin
dipendeva tanto dai viaggi stellari avrebbe dovuto lavorare più
duramente per sviluppare la tecnologia delle tute, giusto? La creatura
poteva anche avere una costola rotta, ma valutai che quella era la
minore delle sue preoccupazioni. Non ero un dottore, ma mi sembrava che
qualcosa avesse trapassato la tuta, la pelle e la carne, penetrando nel
torso e fracassando una di quelle ossa trasparenti.

Sollevai lo sguardo sul Cielcin di nome Tanaran. Volevo chiedere un kit
medico, ma non conoscevo la parola. «\emph{Panathidu}!» dissi in tono
secco, protendendo una mano verso l'altro xenobita che se ne stava fermo
lì, confuso dal mio stupido farfugliare. «Medicina! Medicina!» Mi girai
verso Bassander, che era ancora incurvato in avanti, e mi rivolsi ai
soldati raccolti all'imboccatura del tunnel. «Portate un kit medico! Il
loro capitano è ferito.»

Tanaran guardò verso Uvanari, socchiudendo gli enormi occhi neri fino a
ridurli a fessure inclinate, e ruotò la testa in senso orario. «Cosa
vuole?» chiese.

«Lui... sta cercando di aiutarmi» rispose Uvanari, usando un esplicito
pronome di genere attivo nel riferirsi a me, poi riportò la sua
attenzione su di me. «Non abbiamo niente, \emph{yukajji-do}. Lascia
perdere.» Grugnendo, si costrinse a sedere un po' più eretto. «Forse
morirò qui, pace o non pace.» Il suo volto alieno assunse un'espressione
quasi umana nella sua solennità. «Ed eravamo così vicini...»

«Vicini?» ripetei. «Vicini a cosa?»

Uvanari serrò gli occhi prima di rispondere con parole che si fecero
fievoli come spettri e sospiri. «Loro non sono qui... non qui.» Appoggiò
la testa contro la colonna, inducendo l'altro Cielcin, Tanaran, ad
accorrere al suo fianco. «Puoi avere la tua pace, piccolo umano.»

Tanaran trattenne il respiro. «\emph{Veih}, no. Capitano, non puoi
farlo.»

«\emph{Eka de}» ribatté Uvanari. `Posso.' Aprì gli occhi nerissimi.
«\emph{Uje ekau.» `}E lo farò.' Incrociò sul petto i pugni chiusi. Un
saluto? Una resa? Un gesto di fedeltà? Non avrei saputo dirlo. «Ci
arrendiamo, umano.»

