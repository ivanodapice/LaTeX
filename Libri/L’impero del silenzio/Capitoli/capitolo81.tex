\chapter{Dizionario}

\emph{Una nota sulla traduzione}

C'è stata una certa difficoltà nel trasporre questo testo in inglese
classico dall'anglo-hindi dello standard galattico, in quanto nel
galstani esistono molte parole che sono state coniate dopo la morte
dell'inglese come lingua viva. Ho scelto di rendere parecchie di esse,
come \emph{sakradaas} (`anagnosta') e \emph{hauviros} (`centurione') con
termini presi a prestito dal greco antico e dal latino in quanto per una
persona di lingua inglese esse suonano pregne di una tradizione
paragonabile a quella che questi termini -- essi stessi intrisi di
diciassettemila anni di tradizione imperiale -- hanno per quanti leggono
la supposta autobiografia di lord Marlowe nella sua forma originale. In
altri casi, come per \emph{groundcar} (`auto da terra') e
\emph{cryoburn} (`ustione da gelo'), ho coniato in inglese classico
parole composite che ritengo si avvicinino alla sensazione che danno
questi termini plebei più contemporanei. Ho sostituito i nomi di certe
creature viventi -- molte delle quali non erano state scoperte o create
nel periodo dell'inglese classico -- con neologismi mitologici o con
termini derivati dalla nomenclatura scientifica (per esempio
\emph{azhdarch} e \emph{congrid}).

Ho mantenuto i nomi propri di persone o pianeti esattamente come li ha
riportati lord Marlowe, inclusi quelli degli xenobiti Cielcin. Devo però
sottolineare che questi nomi sono stati translitterati mediante il
sistema usato da Marlowe e non con i sistemi codificati dall'Ordine
Petersoniano o dall'ufficio dell'Intelligence della Legione. Di
conseguenza la loro grafia potrebbe differire da quella che si può
trovare nei testi degli scoliasti o nei documenti ufficiali.

I riferimenti alla letteratura antica, classica e contemporanea sono
stati scritti esattamente come nell'originale e non sono stati
modificati per meglio comunicare i collegamenti simbolici a un pubblico
moderno. Lord Marlowe ha una particolare passione per gli scritti della
tarda età aurea del II e III millennio (vecchio calendario), cosa che
dobbiamo attribuire al nostro fratello tor Gibson di Nov Acor, che era
indubbiamente un classicista.

A seguire troverete un elenco dei termini speciali rilevanti per questo
primo volume del resoconto di lord Marlowe. Sono profondamente grato a
quei miei fratelli e sorelle che mi hanno aiutato nell'esplorazione di
questo testo e prego che con il tempo si possa essere sicuri della sua
autenticità.

\emph{Tor Paulos di Nov Belgaer}

\textbf{Adamant}: uno dei diversi materiali a concatenazione di carbonio
usati per lo scafo e la corazzatura delle astronavi.

\textbf{Adoratore}: membro di un antico culto religioso mantenuto
dall'Impero e tollerato dalla Cappellania.

\textbf{Aeta}: un principe-condottiero cielcin con diritto di proprietà
sui suoi sudditi e i loro beni.

\textbf{Alchimista}: un congegno pesantemente regolamentato usato per
trasmettere elementi chimici mediante processi nucleari. In alternativa,
un tecnico addestrato a usare un congegno del genere.

\textbf{Alienage}: nell'impero Solano un ghetto o una riserva destinata
alla segregazione di una popolazione xenobita.

\textbf{Alluvetro}: una forma ceramica di alluminio trasparente più
resistente del vetro usata comunemente per le finestre, in particolare
sulle astronavi.

\textbf{Altamateria}: una forma di materia esotica prodotta dagli
alchimisti e usata per fabbricare le spade dei cavalieri imperiali. Può
tagliare quasi qualsiasi cosa.

\textbf{Alto Cancelliere}: il funzionario nominato a capo del servizio
civile di un lord palatino. Un primo ministro.

\textbf{Alto Collegio}: ufficio politico imperiale incaricato di
esaminare la richiesta da parte dei palatini di avere figli e di
sovrintendere a quelle gravidanze. Impedisce le mutazioni.

\textbf{Anagnosta}: un iniziato del clero della Cappellania.

\textbf{Androgino}: un omuncolo che non ha caratteristiche sessuali
oppure ha tanto quelle maschili quanto quelle femminili.

\textbf{Anello con sigillo}: congegno portato al dito da un palatino,
contenente le sue informazioni genetiche, conti finanziari e atti di
proprietà delle sue terre.

\textbf{Apatia}: lo stato di assenza di emozioni perseguito dagli
scoliasti per facilitare le loro capacità computazionali. Ha radici
nello stoicismo classico.

\textbf{Arcicostruttori}: specie estinta di xenobiti nativi del pianeta
Ozymandias, così chiamati per la forma delle enormi strutture che hanno
eretto sulle pianure del loro pianeta.

\textbf{Arciduca/arciduchessa}: il rango più alto della nobiltà
palatina. Governa su un dominio planetario. Il titolo può essere
ereditato.

\textbf{Arconte}: rango più basso della gerarchia nobiliare imperiale.
Governa su una prefettura planetaria. Posizione attribuita con nomina o
ereditaria.

\textbf{Arma}: Una costellazione visibile nel cielo notturno di Delos,
modellata come uno scudo medievale.

\textbf{Assunzione della Terra}: nella religione della Cappellania,
indica la partenza della Madre Terra in seguito alla sua distruzione. Si
dice che tornerà quando gli uomini saranno degni di lei.

\textbf{Astranavis}: una costellazione visibile nel cielo notturno di
Delos, modellata come un'astronave.

\textbf{Ateneo}: complesso di ricerca/monastero degli ordini
scoliastici.

\textbf{Atomiche}: armi atomiche. Il loro possesso è legale per i Casati
nobiliari in base ai Grandi Atti Costitutivi, ma è illegale usarle.

\textbf{Atto di disconoscimento}: nella giurisprudenza imperiale si
tratta di un documento formale con cui un nobile disconosce uno o più
membri della sua famiglia.

\textbf{Auctor}: nominato dall'imperatore per servire come suo sostituto
quando non può essere presente.

\textbf{Autotrad}: una macchina che traduce automaticamente le lingue in
tempo reale. Illegale nell'Impero.

\textbf{Azhdarch}: un predatore xenobita comune nel Colosso. Simile a
una lucertola, con un lungo collo aperto da cima a fondo a formare una
bocca piena di zanne.

\textbf{Badonna}: onorifico jaddiano che si applica alle donne
\emph{eali}.

\textbf{Baetan}: nella cultura cielcin è una sorta di prete-storico
della \emph{scianda}.

\textbf{Balena d'ottone}: grossa creatura marina che respira aria,
importata da altrove negli oceani di Delos.

\textbf{Barone/baronessa}: rango più basso della nobiltà palatina
imperiale che possa governare un intero pianeta. Rango superiore ad
arconte ma inferiore a visconte. Governa un domino planetario. Il titolo
può essere ereditato.

\textbf{Bastiglia}: centro giudiziario e penale della Cappellania, di
solito annesso al sacrario di un tempio.

\textbf{Bastone storditore}: arma non letale usata soprattutto nel
controllo delle folle. Poco più di un'asta elettrificata.

\textbf{Battaglia di Cressgard}: prima battaglia contro i Cielcin,
combattuta nell'\foreignlanguage{italian}{isd} 15792.

\textbf{Battaglia di Linon}: battaglia impegnata
nell'\foreignlanguage{italian}{isd} 15863 dal Casato Marlowe contro gli
esuli ribelli del sistema di Delos.

\textbf{Battaglia di Wodan}: battaglia combattuta contro i Cielcin
nell'\foreignlanguage{italian}{isd} 16129 al di sopra del pianeta Wodan.

\textbf{Bit}: moneta d'acciaio usata dalle classi plebee imperiali.
Centoquarantaquattro bit corrispondono a un kaspum. Ne esistono diversi
tagli.

\textbf{Boedromion}: l'ottavo mese del calendario locale deliano, che
segna l'inizio dell'autunno nell'emisfero settentrionale.

\textbf{Bouillir}: classe di vini rossi speziati vinificati su parecchi
mondi. Il più famoso è quello dei vigneti del Casato Markarian di
Kandar.

\textbf{Braccio del Sagittario}: una delle cinque braccia della galassia
colonizzate dalla razza umana. Si trova fra Orione e Centaurus e
comprende il grosso delle colonie imperiali e il Lothrian.

\textbf{Braccio di Centaurus}: una delle cinque braccia della galassia
colonizzate dalla razza umana. Si trova fra il Sagittario e Norma e
comprende la maggior pare delle colonie imperiali.

\textbf{Braccio di Orione}: una delle cinque braccia della galassia
colonizzate dalla razza umana. Vedere Sperone di Orione.

\textbf{Braccio di Perseo}: una delle cinque braccia della galassia
colonizzate dalla razza umana. Si trova al di là di Orione, sul confine
esterno e contiene i principati, la Repubblica durantina e colonie
imperiali.

\textbf{Bruciatore al plasma}: arma da fuoco che usa un forte cappio di
energia magnetica per proiettare un arco di plasma surriscaldato
attraverso brevi e medie distanza.

\textbf{Buio/Buio Esterno}: lo spazio, che nella religione della
Cappellania è un luogo di desolazione e di tormento.

\textbf{Campagne Mathuran}: una serie di battaglie fra clan tavrosiani,
con assistenza imperiale. Concluse con un armistizio poco dopo
l'apparizione dei Cielcin.

\textbf{Campo Royse}: qualsiasi campo di forze che usi l'effetto Royse
per impedire a oggetti ad alta velocità di penetrare una cortina di
energia.

\textbf{Campo statico}: una variante altamente permeabile del campo
Royse, usata per mantenere l'aria condizionata negli edifici.

\textbf{Canti}: il libro sacro della Cappellania, un insieme di canti,
leggi e parabole.

\textbf{Cantore}: comune prete o sacerdotessa della religione della
Cappellania.

\textbf{Cappellania}: vedere Sacra Cappellania Terrestre.

\textbf{Capitano}: grado di ufficiale di marina della Legione imperiale,
superiore a comandante ma inferiore a tribuno. Può comandare
un'astronave.

\textbf{Capofazione}: principale funzionario di una Gilda commerciale.

\textbf{Capsula}: capsula di stoccaggio criogenico usata per trasportare
le persone durante lunghi viaggi interstellari.

\textbf{Castellano}: principale ufficiale militare di una tenuta
nobiliare, incaricato della difesa del castello e delle sue tenute. Di
solito è un cavaliere.

\textbf{Cathar}: chirurgo-torturatore della Sacra Cappellania Terrestre.

\textbf{Cavaliere}: onore militare solano conferito dalla nobiltà per
servizi resi. Di solito include un piccolo feudo. Un cavaliere può usare
armi ad altamateria.

\textbf{Cavaraad}: specie di giganti xenobiti nativi di Sadal Suud. Sono
umanoidi alti in media fra i venti e i trenta piedi e hanno un livello
di sviluppo approssimativamente da paleolitico.

\textbf{Centuria}: unità delle Legioni imperiali formata da dieci
decurie e agli ordini di un centurione e di un optio.

\textbf{Centurione}: grado delle Legioni imperiali, comanda una
centuria.

\textbf{Chiamare via onda}: contattare qualcuno mediante radio o
telegrafo. In alternativa, `onda' indica il congegno di comunicazione
stesso. Termine gergale.

\textbf{Chimera}: qualsiasi animale alterato geneticamente o creato
artificialmente, di solito ottenuto mescolando i codici genetici di due
o più animali.

\textbf{Chiostro}: alloggio di uno scoliasta, mantenuto sgombro da
qualsiasi congegno tecnologico.

\textbf{Cielcin}: specie aliena che viaggia nello spazio, umanoide e
carnivora. Principale nemico dell'umanità durante la Crociata.

\textbf{Cingulum}: tipo di cintura indossato dai legionari e da altri
soldati dell'Impero Solano.

\textbf{Cintura-scudo}: un congegno di autodifesa che si porta intorno
alla vita e utilizza un campo Royse per fermare proiettili, plasma e
altre armi ad alta energia.

\textbf{Città Bassa}: il distretto povero della città di Meidua, su
Delos.

\textbf{Clan}: nella cultura tavrosiana, tutti i membri di un gruppo
affine attingono i loro geni da una stessa banca dei geni. È un analogo
delle costellazioni imperiali.

\textbf{Coloni}: razze di xenobiti intelligenti a un livello di sviluppo
preindustriale nativi di mondi occupati dagli umani, in particolare
nell'Impero Solano.

\textbf{Colosso}: una serie di eventi sportivi che si tengono in un
colosseo e coinvolgono gladiatori professionisti, schiavi, mirmidoni,
animali, gare e altre cose ancora.

\textbf{Coltello-missile}: un tipo di drone che è poco più di un
coltello volante controllato da remoto. Arma preferita fra gli
assassini.

\textbf{Comandante}: grado di ufficiale di marina della Legione,
inferiore a capitano ma superiore a tenente. Assiste nel comando e nella
gestione di una nave.

\textbf{Commonwealth lothriano}: la seconda più grande entità politica
della galassia, uno Stato collettivista totalitario da lungo tempo
antagonista dell'Impero.

\textbf{Compagnia Normanna Unita}: l'antico governo democratico normanno
del pianeta Emesh prima della sua annessione all'Impero.

\textbf{Congrid}: creatura simile a un'anguilla nativa di Emesh. È
considerata una prelibatezza.

\textbf{Consorzio Wong-Hopper}: la più grande delle corporazioni
commerciali mandari. Detiene parecchi contratti governativi, soprattutto
nell'industria della terraformazione.

\textbf{Conte/contessa}: titolo della nobiltà palatina imperiale.
Governa un dominio planetario. Il titolo può essere ereditato.

\textbf{Corpi di Spedizione}: branca delle Legioni imperiali incaricata
di esplorare la galassia e di svolgere il lavoro di preparazione alla
colonizzazione.

\textbf{Correttivo}: congegno medico usato per trattare traumi fisici
riportati per incidenti o combattimenti.

\textbf{Costellazione}: fra i palatini, indica un supergruppo di
famiglie imparentate, di solito caratterizzate da certe caratteristiche
e tratti significativi.

\textbf{Crociata}: la guerra imperiale contro i Cielcin.

\textbf{Culto della Terra}: vedere Sacra Cappellania Terrestre.

\textbf{Custode}: qualsiasi membro degli eserciti privati della
Cappellania usati per difendere i siti sacri da incursioni e saccheggi.

\textbf{Data stellare imperiale}: il calendario standard imperiale. Mesi
e settimane corrispondono al calendario gregoriano della Vecchia Terra e
l'anno è calcolato a partire dall'incoronazione del primo imperatore.

\textbf{Decreto}: codice legale e morale della Cappellania, fatto
rispettare dall'Inquisizione e dall'Indice.

\textbf{Decuria}: nelle Legioni imperiali è un'unità di dieci soldati
formata da tre gruppi di tre soldati e da un decurione.

\textbf{Decurione}: grado delle Legioni imperiali. Comanda una decuria.

\textbf{Dèi}: vedere Icona.

\textbf{Demarchia di Tavros}: piccolo sistema di governo interstellare
che si trova nel Fuoco Fatuo ed è radicalmente aperto alla tecnologia.
Le persone votano tutte le misure adottate usando impianti di merletti
neurali.

\textbf{Demone}: un'intelligenza artificiale. Termine a volte applicato
erroneamente a sistemi computerizzati non intelligenti.

\textbf{Demoniaco}: una persona che abbia incorporato macchine nel suo
corpo, in particolare con l'intento di alterare i processi cognitivi.

\textbf{Disintegratore a fase}: un'arma da fuoco che attacca il sistema
nervoso. Ai settaggi più bassi può stordire.

\textbf{Divisione di Risposta al Crimine}: sottodivisione delle forze
dell'ordine di ciascuna prefettura locale incaricata di intervenire e
indagare su un crimine.

\textbf{Diyugatsayu}: nella cultura dei Cielcin esprime il concetto di
libertà, del non essere posseduti da un altro Cielcin. Condizione
pesantemente stigmatizzata.

\textbf{Dodici Abomini}: i dodici peccati più gravi secondo l'Indice
della Cappellania. In questi casi non si applicano i privilegi legali.

\textbf{Dom/domi}\emph{:} onorifico jaddiano, applicato di solito a
uomini \emph{eali}.

\textbf{Dominio}: un territorio imperiale governato da un palatino. Può
essere trasmesso agli eredi a discrezione di chi governa.

\textbf{Douleter}: supervisore o mercante di schiavi.

\textbf{Drone}: qualsiasi macchina subintelligente operata da una
semplice programmazione o da un essere umano.

\textbf{Duca/duchessa}: titolo della nobiltà palatina imperiale. Governa
un dominio planetario e il titolo può essere ereditato.

\textbf{Eali}\emph{:} la casta regnante jaddiana, prodotto di un intenso
sviluppo eugenetico. Praticamente sovrumana.

\textbf{Efebeia}: celebrazione che si tiene per il ventunesimo
compleanno di un ragazzo o di una ragazza per festeggiare il suo
passaggio all'età adulta.

\textbf{Efebo/a}: giovane uomo o donna fra gli undici e i ventuno anni
standard.

\textbf{Effetto Royse}: un metodo scoperto da Caelan Royse per
manipolare le forze elettrodeboli e che permette l'esistenza di campi di
forze e di repulsori.

\textbf{Elegia per la Terra}: la chiamata della Cappellania alla
preghiera che si tiene ogni giorno al tramonto. Inno funebre per la
Madre Terra.

\textbf{Entottico}: congegno per la realtà aumentata che proietta
immagini direttamente sulla retina.

\textbf{Eolderman}: capo eletto di una comunità plebea. Lo si incontra
generalmente nelle regioni più rurali dei pianeti imperiali.

\textbf{Esecutore}: un funzionario nominato da un nobile palatino e
incaricato della gestione delle sue tenute in sua assenza.

\textbf{Esodo}: il periodo espansionistico seguito al collasso
ambientale della Terra. I Vagabondaggi dal sistema della Vecchia Terra
prima della Guerra di Fondazione.

\textbf{Esule}: qualsiasi lord palatino che non abbia una sua base su un
mondo abitabile. Il termine si può riferire anche a tutta la sua
famiglia.

\textbf{Età aurea della Terra}: l'epoca mitica che ha portato all'Esodo
ed è culminata con l'Assunzione della Terra e la colonizzazione del
sistema solare.

\textbf{Eudoriani}: una qualsiasi delle bande che viaggiano nello spazio
e sostengono di discendere dalla colonia fallita su Europa, nel sistema
della Vecchia Terra. È un gruppo etnico noto per i suoi vagabondaggi
interstellari.

\textbf{Extrasolare}: qualsiasi barbaro che viva al di fuori del
controllo imperiale e che spesso indulge in pratiche illegali.

\textbf{Festa d'Estate}: festività di mezza estate celebrata in tutto
l'Impero. La sua data varia da mondo a mondo, a seconda del calendario
locale.

\textbf{Feudo}: territorio imperiale concesso a un palatino o a un
patrizio tramite nomina. Non può essere trasmesso in eredità.

\textbf{Filatterio}: ampolla in cui sono conservati campioni genetici,
soprattutto per l'uso nella riproduzione.

\textbf{Foederatus}: un mercenario.

\textbf{Forza di Difesa Orbitale}: flotta mantenuta in essere da
qualsiasi lord palatino per la difesa del suo pianeta o del suo sistema.

\textbf{Fuoco Fatuo, il}: la sottile stringa di stelle che ospita la
Demarchia di Tavros, situata al di sopra dell'ellittica galattica. È
molto lontano dall'Impero.

\textbf{Galstani}: vedere standard galattico.

\textbf{Giganti}: vedere Cavaraad.

\textbf{Gilda}: qualsiasi organizzazione che svolga la sua attività
tramite approvazione legale. Un sindacato soggetto a un signore
planetario.

\textbf{Gladiatore}: atleta combattente professionista nel Colosso.

\textbf{Glorificati}: una fazione di Extrasolari noti per gli estremi
accrescimenti cibernetici.

\textbf{Golfo Cupo}: l'enorme distesa di spazio vuoto fra il Braccio di
Norma e quello di Centaurus della galassia.

\textbf{Grandi Atti Costitutivi}: antica raccolta di codici legali
imposti all'Impero da una coalizione di Casati palatini. Mantiene
l'equilibro fra i Casati e l'Impero.

\textbf{Gravimetro}: congegno per misurare la densità e la struttura
della materia esaminando le distorsioni della forza di gravità.

\textbf{Gruppo Izumo}: corporazione interstellare nipponese
specializzata nel commercio di metalli pesanti.

\textbf{Guerra di Fondazione}: guerra fra il primo Impero e i Mericanii,
nella quale i Mericanii vennero distrutti e fu fondato l'Impero Solano.

\textbf{Haqiph}\emph{:} termine jaddiano che significa `intoccabile' o
`spregevole'. Riferito agli omuncoli e ad altri soggetti considerati
subumani.

\textbf{Hudr}\emph{:} termine jaddiano per indicare uno scoliasta.
Letteralmente significa `verde'.

\textbf{Hurasam}: monete dorate usate fra le classi plebee dell'Impero.
Valgono il loro peso in oro. Esistono banconote di tagli diversi.

\textbf{Ichakta}\emph{:} titolo cielcin che si riferisce al capitano di
una nave.

\textbf{Icona}: nella religione della Cappellania è uno spirito o un dio
che incarna un ideale, una virtù o una legge naturale, come Coraggio,
Evoluzione o Tempo.

\textbf{Imperatore}: il supremo sovrano dell'Impero Solano, considerato
un dio e la reincarnazione del suo predecessore. Detiene un potere
assoluto.

\textbf{Impero}: vedere Impero Solano.

\textbf{Impero Solano}: la più grande e antica entità politica nello
spazio controllato dagli umani, è formato da circa mezzo miliardo di
pianeti abitabili.

\textbf{Indice}: catalogo delle punizioni -- monetarie, corporali e
capitale -- stabilite dalla Cappellania e applicate dall'Inquisizione.

\textbf{Inglese classico}: l'antica lingua parlata tanto dai Mericanii
quanto dai primi coloni imperiali insediatisi su Avalon, e usata ancora
dagli scoliasti.

\textbf{Inmane}\emph{:} insulto offensivo che indica qualcuno meno che
umano. Letteralmente significa `impuro'.

\textbf{Inquisitore}: un funzionario della Cappellania incaricato di
condurre le indagini giudiziarie e di sovrintendere alla tortura dei
criminali.

\textbf{Inquisizione}: braccio giudiziario della Cappellania Imperiale
che si occupa prevalentemente dell'uso di tecnologie illegali.

\textbf{Intus}: un palatino nato al di fuori della supervisione
dell'Alto Collegio e che di solito possiede parecchi difetti fisici o
psicologici. Un bastardo.

\textbf{Irchtani}: una specie di xenobiti nativi del pianeta Giudecca.
Simili a uccelli e dotati di ali enormi. Considerati un esempio di
assimilazione dei coloni.

\textbf{Ipogeo}: il complesso di manutenzione sotterraneo sottostante un
colosseo o, più genericamente, qualsiasi complesso sotterraneo.

\textbf{Jaddiano}: la lingua ufficiale dei principati di Jadd, un misto
di antica lingua romanza e di linguaggi semitici con qualche influenza
greca.

\textbf{Jubala}: un potente e popolare narcotico che viene da altri
pianeti. Può essere inalato o ingerito come una sorta di tisana.

\textbf{Kaspum}: monete placcate in argento, usate dalla classe popolana
dell'Impero. Dodici kaspum compongono un hurasam d'oro, ed esistono
anche banconote stampate di pari valore.

\textbf{Lancia}: vedere lancia a energia.

\textbf{Lancia a energia}: una lancia dotata di lama con un laser ad
alta energia inserito nell'asta. Usata come arma formale dalle guardie,
soprattutto nell'Impero.

\textbf{Legge dei pesci}: un precetto filosofico secondo cui il mondo è
una landa selvaggia e la sopravvivenza è la virtù più grande. La legge
della giungla. La sopravvivenza del più forte.

\textbf{Legionario}: qualsiasi soldato delle Legioni imperiali,
soprattutto comuni soldati di fanteria.

\textbf{Legioni}: branca militare dell'Impero Solano, fedele
all'imperatore e al Casato imperiale. Comprende forze navali e di terra.

\textbf{Libero proprietario}: un cittadino di una qualsiasi delle
Proprietà Normanne o di qualsiasi governo planetario o multiplanetario
non associato con uno dei grandi poteri interstellari.

\textbf{Libro della Mente}: antologia di parecchi testi raccolti o
composti dallo scoliasta Imore. Forma la base della loro filosofia.

\textbf{Littore}: guardia del corpo di un nobile o di un altro
dignitario. Di solito è un cavaliere.

\textbf{Logoteta}: un ministro di una qualsiasi agenzia governativa di
un Casato palatino. Termine usato colloquialmente per indicare qualsiasi
funzionario civile.

\textbf{Logoteta pluripotente}: un logoteta il cui ufficio sovrintende
al trasferimento di terra e di altre tenute connesse a titoli fra
palatini e patrizi.

\textbf{Lord planetario}: qualsiasi nobile che abbia acquisito o possa
trasmettere in eredità il suo dominio planetario all'interno della
famiglia. Nobile con possedimenti terrieri.

\textbf{Lothrian}: vedere Commonwealth lothriano.

\textbf{Lothriano}: linguaggio parlato dai Lothriani.

\textbf{Madre Terra}: il deificato mondo di origine dell'umanità e
principale divinità della religione della Cappellania.

\textbf{Maeskolos}: leggendario maestro di spada di Jadd, tratto
esclusivamente dalla casta degli \emph{eali}. Si attribuiscono a lui
velocità e abilità sovrumane.

\textbf{Maestro di spada}: vedere maeskolos.

\textbf{Magister}: un giudice laico che giudica i casi relativi ai
plebei.

\textbf{Magio}: un intellettuale, soprattutto uno scienziato o un
filosofo naturale.

\textbf{Mamelucco}: un soldato-schiavo omuncolo dei principati jaddiani.

\textbf{Mandar}: il linguaggio delle corporazioni commerciali mandari.

\textbf{Mandari}: un gruppo etnico semi-distaccato dalla società
imperiale, che costituisce in genere il personale delle enormi
corporazioni commerciali interstellari.

\textbf{Mandayas}: indumento tradizionale dei maeskoloi. Una mezza veste
fermata in vita con un'ampia manica indossata sulla spalla sinistra.

\textbf{Marco imperiale}: valuta digitale dell'Impero Solano e delle
corporazioni mandari. Fortemente competitivo al contrario degli hurasam
e delle altre valute fisiche.

\textbf{Meidua}: città portuale di Delos, sede della prefettura di
Meidua e dominio ancestrale del Casato Marlowe.

\textbf{Membro di una Gilda}: qualsiasi suo appartenente.

\textbf{Mercante}: venditore o uomo d'affari, di solito un plebeo.

\textbf{Meretrice}: la madama di un bordello o, più spesso, dell'harem
di un lord palatino.

\textbf{Mericanii}: gli antichi primi coloni interstellari. Una civiltà
tecnologica iperavanzata gestita dalle intelligenze artificiali.
Distrutta dall'Impero.

\textbf{Merletto neurale}: un computer semiorganico impiantato nel
cervello di un ospite. Illegale nell'Impero.

\textbf{Messere/madama}: forme di cortesia dell'Impero, usate con
chiunque non abbia un titolo formale.

\textbf{Mirmidone}: qualsiasi combattente nel Colosso sotto contratto o
schiavo che non sia un gladiatore professionale addestrato.

\textbf{Mute}: termine gergale, abbreviazione di `mutante'. Si riferisce
a un omuncolo o a un intus.

\textbf{Nanocarbonio}: tessuto fatto di nanotubi di carbonio, simile
all'adamant.

\textbf{Nave a rotazione}: qualsiasi astronave che usi la forza
centripeta per generare un'illusione di forza di gravità.

\textbf{Nave leggera}: qualsiasi astronave abbastanza piccola da poter
atterrare su un pianeta.

\textbf{Nave-mondo}: una qualsiasi delle enormi navi dei Cielcin --
alcune grandi come lune -- che formano il nucleo delle loro flotte.

\textbf{Ndaktu}\emph{:} nella filosofia cielcin, indica il peso di una
responsabilità morale che ricade su un individuo le cui azioni hanno
causato sofferenza a un altro in modo diretto o indiretto.

\textbf{Neg}\emph{:} una persona senza valore. Gergo emeshi.

\textbf{Necrosi grigia}: una pestilenza proveniente da fuori, portata su
Emesh nel XVII millennio \foreignlanguage{italian}{isd}. Ha spazzato via
il diciotto percento della popolazione.

\textbf{Nipponesi}: i discendenti dei coloni giapponesi fuggiti dal
sistema della Vecchia Terra nel Terzo Vagabondaggio.

\textbf{Nobile}: termine generico per indicare qualsiasi membro della
casta palatina o patrizia dell'Impero Solano.

\textbf{Nordei}: il linguaggio principale della Demarchia, un misto di
nordico e di thai con qualche influenza slava.

\textbf{Novantanove meraviglie dell'Universo}: novantanove fra le
strutture più grandiose -- umane e aliene -- dell'universo conosciuto.

\textbf{Ologramma}: immagine di luce tridimensionale proiettata da laser
scansionanti. Viene usato per intrattenimento, per le pubblicità, per le
comunicazioni, ecc.

\textbf{Omuncolo}: qualsiasi umano o quasi-umano artificiale, creato per
un lavoro o per scopi estetici.

\textbf{Opera}: qualsiasi narrativa d'intrattenimento che abbia un
copione, che sia musicale, drammatica o seriale, interattiva o meno.

\textbf{Oplita}: fante dotato di scudo della fanteria pesante.

\textbf{Oplon}: scudo rotondo in stile antico usato nel Colosso.

\textbf{Optio}: comandante in seconda di un centurione delle Legioni
imperiali.

\textbf{Ornithon}: serpente piumato volante nativo di Emesh. Non
velenoso, mangia primariamente pesci.

\textbf{Palatinato}: qualsiasi dominio o feudo che abbracci un intero
pianeta.

\textbf{Palatini}: l'aristocrazia imperiale, discendente di quei liberi
umani che si sono opposti ai Mericanii. Potenziati geneticamente,
possono vivere per parecchi secoli.

\textbf{Pallidi}: i Cielcin. Termine gergale, considerato offensivo
dagli xenofili.

\textbf{Panegirista}: prete della Cappellania incaricato del richiamo
alla preghiera del tramonto.

\textbf{Panthai}: linguaggio tavrosiano sviluppato dai popoli di lingua
thai, lao e khmer che si sono insediati nel Fuoco Fatuo insieme ai
nordei.

\textbf{Pari}: termine per indicare la costellazione palatina che
include la famiglia imperiale e i suoi parenti. I suoi membri sono tutti
nella linea di discendenza che può ereditare il trono.

\textbf{Patrizio}: qualsiasi plebeo o plutocrate a cui venga concesso il
potenziamento genetico per volere della casta palatina e come ricompensa
per i servigi resi.

\textbf{Peltasta}: soldato di fanteria senza scudo. Fanteria leggera.

\textbf{Perseo Esterno}: la regione di espansione lungo la fine del
Braccio di Perseo. Frontiera coloniale.

\textbf{Phasma vigrandi}: creature fluttuanti luminescenti native delle
foreste di Luin, chiamate a volte fate.

\textbf{Plebeo}: i popolani dell'Impero, discesi da stock umano puro
presente sulle più antiche navi coloniali. Non ha il permesso di usare
la tecnologia.

\textbf{Plutocrate}: qualsiasi plebeo che abbia guadagnato abbastanza da
comprare costosi accrescimenti genetici. Di fatto, è un patrizio.

\textbf{Poine}: una guerra strutturata su piccola scala fra Casati
palatini. Soggetta a esame dell'Inquisizione.

\textbf{Presenza Imperiale}: il personaggio formale dell'imperatore
solano e l'area intorno alla sua persona.

\textbf{Praxis}: alta tecnologia, di solito del genere proibito dalla
Legge della Cappellania.

\textbf{Prefetto}: funzionario delle forze dell'ordine.

\textbf{Prefettura}: nell'Impero, qualsiasi distretto amministrativo
governato da un arconte.

\textbf{Pretoriano}: qualsiasi membro della guardia pretoriana
dell'imperatore solano, scelto fra gli elementi migliori della Legione
imperiale.

\textbf{Primarca}: il principale viceré imperiale in ciascun braccio
della galassia: Orione, Sagittario, Perseo e Centaurus. In pratica, sono
coimperatori.

\textbf{Primarcato}: regione dell'Impero governata da un primarca e
formata da parecchie provincie.

\textbf{Primate}: la più alta carica amministrativa di un ateneo
scoliastico, equivalente di un rettore universitario.

\textbf{Principati di Jadd}: nazione costituita da ottanta ex provincie
imperiali di Perseo che si sono rivoltate a causa dei diritti
riproduttivi dei palatini. Fortemente militaristici e governati da
caste.

\textbf{Priore/a}: nel clero della Cappellania è il chierico a capo di
una prefettura.

\textbf{qet/telegrafo quantico}: un congegno che usa particelle
quantiche correlate per comunicare all'istante attraverso vaste
distanze.

\textbf{Quattro Elementi Cardinali}: nella religione della Cappellania,
sono le quattro icone più importanti: Giustizia, Coraggio, Prudenza e
Temperanza.

\textbf{Quiete}: l'ipotetica prima civiltà della galassia che si suppone
sia responsabile della creazione di parecchi siti antichi, inclusi
quelli su Emesh, Giudecca, Sadal Suud e Ozymandias.

\textbf{Registro standard}: un indice tenuto dall'Alto Collegio
Imperiale che comprende tutti i Casati palatini, insieme a campioni di
sangue di tutti i loro componenti.

\textbf{Repulsore}: un congegno che usa l'effetto Royse per permettere
agli oggetti di fluttuare senza disturbare l'aria o l'ambiente.

\textbf{Repubblica durantina}: repubblica interstellare di circa tremila
mondi. Paga un tributo all'Impero.

\textbf{Rothsbank}: antico istituto bancario privato le cui radici
risalgono all'età aurea della Terra.

\textbf{Rus}: giovane uomo. Slang emeshi.

\textbf{Sacra Cappellania Terrestre}: religione di Stato dell'Impero.
Funge da braccio giudiziario dello Stato, soprattutto quando è coinvolto
l'uso di tecnologia proibita.

\textbf{Sala medica}: ospedale, in genere a bordo di un'astronave.

\textbf{Satrapo}: governatore planetario dei principati di Jadd,
subordinato a uno dei principi regionali.

\textbf{Scianda}\emph{:} il termine cielcin per indicare uno dei loro
gruppi di navi migratorie. Una flotta.

\textbf{Scoliasta}: qualsiasi membro dell'ordine monastico di
ricercatori, accademici e teorici che fanno risalire le loro origini
agli scienziati mericanii catturati alla fine della Guerra di
Fondazione.

\textbf{Scudo di protezione preventiva}: una forma di campo Royse usata
per la sicurezza, soprattutto nel colosseo e negli hangar delle
astronavi. Intrappola l'aria e gli oggetti in rapido movimento.

\textbf{Scuola di Fuoco}: famoso monastero e accademia di Jadd dove
vengono addestrati i maeskoloi.

\textbf{Segaossa}: chirurgo genetico non approvato dall'Alto Collegio e
che opera nel mercato nero.

\textbf{Segno del disco del sole}: gesto di benedizione fatto formando
un cerchio con pollice e indice e toccandosi la fronte e le labbra prima
di levare la mano verso il cielo.

\textbf{Servitore}: operaio non qualificato.

\textbf{Servo}: qualsiasi plebeo imperiale che per nascita abbia la
proibizione di allontanarsi dal pianeta di nascita a meno di prestare
servizio militare nella Legione.

\textbf{Sfera-dati}: qualsiasi rete di dati planetaria. Nell'Impero,
l'accesso è rigidamente ristretto alle caste dei patrizi e dei palatini.

\textbf{Sfera luminosa}: un'intensa luce sferica che fluttua su
repellenti Royse. Alimentata chimicamente o con batterie.

\textbf{Signore}: onorifico usato per rivolgersi a soggetti socialmente
inferiori, in genere maschi.

\textbf{Sinarca}: la più alta carica clericale della Cappellania
Imperiale. La sua funzione più importante è l'incoronazione dei nuovi
imperatori.

\textbf{Sinodo}: autorità normativa della Sacra Cappellania Terrestre:
un collegio di arcipriori presieduto dal sinarca.

\textbf{Sire}: onorifico usato per riferirsi ai propri superiori dal
punto di vista sociale, di solito maschi e in genere patrizi e palatini.

\textbf{Somma litania}: nella religione della Cappellania è un rito
settimanale tenuto per commemorare la distruzione della Terra e per
pregare per un migliore futuro per l'umanità.

\textbf{Sonno criogenico}: lo stato di sospensione criogenica indotto
per garantire che gli umani e altre creature viventi sopravvivano a
lunghi viaggi fra i soli.

\textbf{Spada Bianca}: una grande spada di ceramica usata dai cathar
della Cappellania per le esecuzioni capitali formali, soprattutto dei
nobili.

\textbf{Spazio profondo}: un territorio all'interno dello spazio
imperiale che non è stato colonizzato formalmente dall'Impero ed è
spesso un rifugio degli Extrasolari.

\textbf{Sperone di Orione}: il più antico dei quattro primarcati
dell'Impero, comprende le parti più antiche dell'Impero stesso e il
Vecchio Sistema Solare.

\textbf{Standard galattico}: lingua comune dell'Impero Solano, discesa
dall'inglese classico con pesanti influenze hindi e franco-tedesche.

\textbf{Stock genetico}: nella terraformazione, è il materiale di base
genetico, terrestre e non, usato per creare ecologie civilizzate.

\textbf{Stock di seme}: Vedere stock genetico.

\textbf{Storditore}: un disintegratore a fase a bassa potenza usato per
causare una paralisi momentanea o una perdita di conoscenza. Arma
preferita del personale delle forze dell'ordine.

\textbf{Strategos}: ammiraglio delle Legioni imperiali, responsabile del
comando di un'intera flotta formata da parecchie Legioni.

\textbf{Tavrosiani}: una qualsiasi delle lingue della Demarchia di
Tavros. In genere si riferisce al nordei.

\textbf{Tenente}: ufficiale di marina delle Legioni, inferiore a
comandante ma superiore a marinaio.

\textbf{Terminale}: congegno di telecomunicazione che accede alla
sfera-dati di un pianeta. Di solito lo si porta al polso.

\textbf{Terranico}: nella terraformazione e nell'ecologia si riferisce a
qualsiasi organismo originario della Vecchia Terra. Non extraterrestre.

\textbf{Tossiemia da verrox}: condizione medica cronica causata
dall'abuso di verrox. Include tremore ed eventuale atrofia dei muscoli.

\textbf{Travatsk}: una lingua tavrosiana che prende il nome dal gruppo
etnico dei Travatskr e si riconosce per la mancanza di neutralizzazione
delle vocali.

\textbf{Tribuno}: un ufficiale della Legione che ha il comando di una
coorte, quattro delle quali compongono una Legione. Comanda gli
ufficiali tanto delle forze di terra quanto della marina.

\textbf{Trionfo}: una parata che si tiene per celebrare una vittoria in
guerra. Di solito gli sconfitti vengono fatti marciare in catene per
essere giustiziati.

\textbf{Troglodita}: un essere umano privo delle funzioni mentali più
elevate, sia volontariamente per motivi personali o religiosi, sia in
conseguenza di un incidente.

\textbf{Trono Solare}: il trono imperiale. Intagliato in un singolo
blocco di quarzo citrino. Termine usato a volte come sinonimo di
Presenza o ufficio imperiale.

\textbf{Udaritanu}: un complesso sistema di scrittura non lineare usato
dai Cielcin.

\textbf{Ufficio dell'Intelligence della Legione}: l'agenzia imperiale
dell'intelligence militare, dello spionaggio e dell'intervento estero.

\textbf{Ufficio imperiale}: l'amministrazione dell'imperatore, i
ministeri e servizi civili -- incluso il personale di palazzo -- che
formano il governo centrale imperiale.

\textbf{Umandh}: una specie di coloni nativa del pianeta Emesh. Anfibi e
con tre gambe, hanno un'intelligenza paragonabile a quella dei delfini.

\textbf{Unione dei liberi mercanti}: una coalizione di piccole compagnie
commerciali e navi mercantili indipendenti che fa pressioni per avere
privilegi di spedizione e spazio di attracco sui pianeti.

\textbf{Ustione da gelo}: ustione causata da un congelamento criogenico
improprio.

\textbf{Vagabondaggio}: una qualsiasi delle storiche evacuazioni dal
sistema della Terra in direzione delle colonie extrasolari.

\textbf{Vate}: qualsiasi predicatore o sant'uomo che non faccia
formalmente parte del clero della Cappellania.

\textbf{Veicolo di terra}: un'automobile, di solito alimentata a energia
solare o a combustione interna.

\textbf{Velivolo}: veicolo volante più o meno delle dimensioni di
un'auto di terra, usato per i voli all'interno dell'atmosfera e per
rapidi spostamenti.

\textbf{Velo di Marinus}: territorio conteso fra l'Impero e le Proprietà
Normanne. Comprende la maggior parte del fronte nella Crociata contro i
Cielcin.

\textbf{Verrox}: una potente pseudoanfetamina che si ricava dalle foglie
della pianta di verroca. Si assume ingerendo le foglie, che sono di
solito candite.

\textbf{Viceré/regina}: il governante di una provincia imperiale,
nominato dall'imperatore. In genere il titolo è ereditabile, ma non
sempre.

\textbf{Vilicus}: capo di una squadra di douleter; supervisore capo.

\textbf{Vincolato al pianeta}: per la legge imperiale, a nessun plebeo è
permesso di viaggiare fuori dal suo pianeta. Un servo.

\textbf{Vostra eccellenza}: forma di cortesia usata per rivolgersi a
nobili governanti di rango superiore a quello di arconte.

\textbf{Vostra grazia}: forma di cortesia usata per rivolgersi a viceré,
viceregine e primarchi dell'Impero Solano.

\textbf{Vostra radiosità}: forma di cortesia usata per rivolgersi
esclusivamente all'imperatore.

\textbf{Vostra reverenza}: forma di cortesia usata per rivolgersi ai
membri del clero della Cappellania, in particolare ai priori.

\textbf{Xenobita}: una forma di vita di origine non terranica o umana,
soprattutto quelle considerate intelligenti. Un alieno.

\textbf{Yamato Interstellar}: compagnia manifatturiera interstellare di
proprietà del Casato Yamato e con base su Nichibotsu.

\textbf{Yukajjimn}: termine cielcin per indicare l'umanità. Ha
collegamenti etimologici con la parola che usano per `animali nocivi'.

\textbf{Zvanya}: un distillato alcolico all'aroma di cannella molto
diffuso a Jadd.

\newpage\blankpage