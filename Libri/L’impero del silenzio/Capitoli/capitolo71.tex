\chapter{L'Inquisizione}

`Ed eravamo così vicini.' Quelle parole mi si agitavano nella testa,
ripetendosi come avevano fatto un mezzo milione di volte da quando le
avevo sentite pronunciare dal capitano cielcin. \emph{Uje ekurimi su
	keta}. Così vicini a cosa? `Loro non sono qui...' Mi sarei potuto
rompere le mani sul piano del tavolo per la frustrazione e avrei dato il
braccio sinistro per avere la possibilità di parlare di nuovo con
Uvanari e tirargli fuori una risposta, se potevo. No, non \emph{tirargli
	fuori}. Avevo tirato fuori abbastanza. Lanciai un'occhiata attraverso il
piano del tavolo di legno pietrificato in direzione del punto in cui sir
Olorin Milta sedeva insieme alla sua satrapo e non guardava nella mia
direzione. Non aveva parlato del mio interrogatorio del Cielcin nelle
gallerie di Calagah e io non avevo intenzione di menzionarlo.

\leavevmode\\
\emph{Loro non sono qui.}

\emph{Loro.}

\emph{Uvanari si era riferito alla Quiete? Era possibile?}
\leavevmode\\

«...dovrebbe essere pronto in una settimana circa» stava dicendo la
tribuno-cavaliere Smythe, con i lineamenti tozzi improntati a un'intensa
concentrazione. L'argomento era appena passato dai frequenti sbalzi di
corrente nella rete energetica del castello e dalla sorveglianza ai
Cielcin. Con mia sorpresa, l'affermazione del centurione Vriell che il
duro ufficiale della Legione stava gestendo le cose a Borosevo pareva
essere vera, anche se non avrei saputo dire se questo dipendesse da un
ordine imperiale o se il conte si era semplicemente fatto da parte.
Balian Mataro occupava l'alto seggio sulla piattaforma che dominava il
tavolo del consiglio e sedeva con il mento appoggiato a una mano come un
annoiato Zeus scolpito nel marmo nero. «I miei medici mi dicono che le
ferite della creatura stanno guarendo bene.»

«E gli altri?» domandò l'Alta Cancelliera Ogir, congiungendo le mani
davanti a sé. «Abbiamo cominciato con loro?»

I sottintesi impliciti in quelle parole trascinarono il mio sguardo
sull'unica persona presente al tavolo che più mi ero sforzato di
evitare. Ligeia Vas indossava le abituali vesti nere. Aveva il volto
incipriato, cosa che attirava ancor più l'attenzione sul suo pallore di
Extraplanetaria, e i capelli bianchi erano raccolti nell'abituale doppia
spirale intorno alle spalle sottili. La cosa peggiore fu che i suoi
occhi gelidi e taglienti come coltelli da lancio incontrarono i miei...
era stata intenta a fissarmi... e non si distolsero dalla mia faccia
mentre rispondeva. «Non lo abbiamo fatto dietro richiesta del nostro
emissario jaddiano.» Finalmente si girò per lanciare una rapida occhiata
a lady Kalima di Sayyiph. «Ci ha chiesto di sospendere le operazioni in
previsione di questa riunione, una sospensione che abbiamo concesso per
rispetto verso i nostri visitatori e per la loro assistenza nella
cattura degli xenobiti.» Da come pronunciò quella parola, `rispetto',
dedussi che esso si estendeva solo fino a un certo punto.

Dal mio posto solitario in fondo al tavolo, dov'ero un'isolata macchia
di colore in mezzo al grigio dei logoteti, studiai la quasi celestiale
satrapo jaddiana. C'erano rubini che le scintillavano intorno alla gola
e alle orecchie, e aveva così tanti gioielli d'oro intorno al collo e
nei capelli da lasciarmi stupito che non si piegasse sotto il loro peso.
Sir Olorin Milta era in piedi alle sue spalle, con la mano che
giocherellava con l'impugnatura delle tre spade ad altamateria
assicurate alla coscia e lo sguardo fisso in lontananza, su un punto del
mare visibile attraverso l'ampio arco di alluvetro che formava la parete
opposta della camera del consiglio. Ebbi l'improvvisa sensazione che in
realtà gli Jaddiani fossero dalla mia parte, disposti a cercare di
parlare con i prigionieri e di fare la pace.

«In ogni caso,» affermò la tribuno-cavaliere, tamburellando sul tavolo
con le nocche squadrate e facendo increspare l'acqua nel bicchiere
accanto al suo gomito «grazie al tenente Lin ci troviamo in possesso di
dieci prigionieri Cielcin.»

«Nemici catturati» non riuscii a trattenermi dal precisare. «Si sono
arresi a noi. Se l'\emph{ichakta} fosse un umano staremmo cercando di
ottenere per lui un riscatto dal suo signore.»

La grande priora calò una mano sul tavolo, richiedendo attenzione.
«Quella bestia \emph{non} è umana, eretico.»

«Quella bestia è un ufficiale nemico» dissi, rivolgendomi a Raine
Smythe, che compresse le labbra ma per il momento parve disposta ad
ascoltarmi. «Non c'è nessuna procedura esistente nel trattare con
ufficiali non umani, giusto? Non dovremmo trattarli in modo onorevole?»
Non aggiunsi il mio sospetto personale che Tanaran potesse essere
qualcosa di diverso da un ufficiale. Qualsiasi altra cosa potesse
essere, il giovane Cielcin non era un soldato e di certo non era vestito
come tale. Peraltro non condivisi neppure il collegamento sottinteso da
Uvanari fra il suo popolo e la Quiete, un frammento di informazione che
avrei condiviso con Valka quando fosse tornata a Borosevo entro quella
settimana, in quanto ogni attività nel sito era stata sospesa in
conseguenza dell'attacco, con le squadre di recupero che lavoravano sodo
per recuperare il relitto della nave di Uvanari.

La cancelliera parve aver appena inghiottito un cucchiaio di estratto di
limone. Si leccò i denti e il volto color cenere si incupì e si tese
mentre ringhiava: «\emph{Inmane}! Ti ricordo, lord Marlowe, che la tua
presenza qui è a stento tollerata.»

«Cancelliera, lui è qui perché è il solo che abbia parlato con i
prigionieri e che possa offrire delle opinioni» interloquì sir Olorin.
Guardai verso di lui e chinai il capo in un silenzioso ringraziamento.
Lui rispose a sua volta con un cenno, smuovendo un groviglio di capelli
scuri.

Le cicatrici chirurgiche indicanti che la cancelliera Ogir era una
patrizia sbiancarono mentre lei serrava le labbra. «Quando vorrò il tuo
parere, littore, ti inviterò a fornirlo.»

«Ora basta, Liada» intervenne lord Balian. «Basta così. Gli Jaddiani
sono nostri ospiti.» Alquanto ammansita, la piccola ed esile cancelliera
fece marcia indietro e parve scoprire qualcosa di assolutamente
affascinante nella rete di vene in rilievo sul dorso delle sue mani
simili a cuoio. A quel tempo mi sorprese che la satrapo non fosse
balzata in difesa del suo servitore, e mi sorprende ancora di più
adesso.

L'intervento di Olorin aveva avuto anche una seconda funzione, come
venne rivelato un momento più tardi quando Raine Smythe osservò: «Lord
Marlowe ha reso un servigio all'Impero, questo è innegabile, e la sua è
una valida considerazione. Se vogliamo cercare di negoziare usando i
Pallidi catturati come ostaggi dobbiamo tenere in considerazione come
verranno trattati.» Sentendo di aver segnato un punto sorrisi alla
grande priora, ma quella strega-sacerdotessa non si degnò di guardare
verso di me. La tribuno-cavaliere riprese a tamburellare sul tavolo con
le nocche. «Qui però abbiamo l'opportunità di ottenere vere informazioni
tattiche. Perché i Cielcin sono venuti qui su Emesh? E perché ora?»

Seguì un momento di pregnante silenzio, scandito dal tamburellare di
quelle nocche e dai versi nervosi dei logoteti seduti a ciascun estremo
dell'arco di quel tavolo punteggiato di rosa e di verde. Tutti noi
sapevamo di cosa si stava parlando in effetti, e forse era per questo
che il conte taceva, preferendo lasciare che fossero i militari e il
clero a prendere le redini. Guardai in basso, verso le mani strettamente
intrecciate in grembo, ricordando come avevano tremato nella galleria, e
la paura che aveva raggiunto la soglia del panico.

«I prigionieri devono essere interrogati» dichiarò Ligeia, in quel
pregnante silenzio, intrecciando le mani sul piano del tavolo con
un'immobilità che faceva da contrappunto ai movimenti nervosi della
tribuno-cavaliere.

«Sì, bisogna indurre i prigionieri -- i nemici catturati -- a darci
qualcosa.» Raine Smythe inclinò la testa per ascoltare una parola
sussurrata da sir William Crossflane, il primo ufficiale dai capelli
bianchi che aveva accanto, poi chiuse gli occhi per un momento.

«La posizione della loro flotta?» suggerì la satrapo jaddiana, con lo
sguardo ancora fisso sulla città sottostante.

Per nulla intimidita da quella benevola interruzione, Raine Smythe
proseguì nella sua rude voce di contralto: «Necessariamente, quello che
rimane da decidere è che genere di informazioni riteniamo di poter
ottenere.»

«Senza mettere a rischio il valore di ostaggi delle creature» interloquì
uno dei ministri del conte, attirandosi un'occhiataccia della
cancelliera.

«Di conseguenza,» aggiunse lo scoliasta tor Vladimir, parlando dal suo
posto accanto al conte e al centro del tavolo semicircolare «dobbiamo
soppesare il valore dei nostri prigionieri come strumenti diplomatici
contrapposto al loro valore strategico.» Le sue parole sommesse e del
tutto prive di inflessione pervasero la stanza come una sorta di gas
soporifero.

Stentavo ancora a credere che stessimo facendo discorsi del genere ed
esplosi: «State parlando di torturarli?»

«Al nostro posto loro non farebbero niente di diverso, figliolo» replicò
l'attempato primo ufficiale seduto accanto alla tribuno-cavaliere.
«Questa è guerra. Noi...» Dama Raine gli posò una mano sul braccio,
inducendolo a calmarsi. Lui si mostrò furioso ancora per un momento, con
le labbra che si contraevano sotto enormi basette cespugliose,
ricordando quelle di un pesce fuor d'acqua. «Negli ultimi mesi hanno
lanciato due attacchi contro di noi. Chi può dire che non ce ne sarà un
terzo?»

«Avevo l'impressione che i primi assalitori fossero solo... com'è che si
dice? Un'avanguardia? Esploratori che precedevano il secondo attacco?
C'era una sola flotta da battaglia» obiettò sir Olorin, sempre
antagonistico.

«Davvero?» domandò la grande priora, torcendosi sulla sua sedia ad alto
schienale per guardare in faccia il maestro di spada jaddiano. «Per
essere un umano, maeskolos, sembri avere un'estrema familiarità con i
piani e le intenzioni del nemico. Forse l'eresia di lord Marlowe è
contagiosa.»

«La dedizione di lord Marlowe alla fede non è oggetto di discussione per
il momento, Vostra reverenza» intervenne la tribuno-cavaliere,
lanciandole un'occhiata in tralice senza però girare la testa. «Per
favore, potremmo accantonare la religione quanto basta per arrivare a
una decisione?» Si nascose la faccia fra le mani e si massaggiò gli
occhi con dita corte e dure. Tutto in lei era tozzo e brusco − i
lineamenti, i modi, i movimenti -- ma era una donna abituata al potere,
e non a quello relativamente piccolo di un nobile possidente terriero,
bensì al pugno delle Legioni imperiali. La sua autorità era quella
dell'Impero, della Presenza e del Trono Solare stesso, e lei non
prendeva con le molle le priore delle Cappellanie di provincia. Trasse
un profondo respiro ed esalò il fiato. «Se da un lato vedo l'utilità di
preservare i prigionieri per il riscatto, dall'altra credo che essi, e
in particolare il loro capitano, siano di interesse molto maggiore per
l'Impero grazie alle informazioni che possiedono in merito ai movimenti
della flotta cielcin.»

La disgustosa incandescenza del sorriso di Ligeia fece cagliare ogni mio
fluido corporeo e serrai i denti al punto che mi parve si crepassero.
\emph{No. No, no}. Dovevo però rispettare quel gioco per quello che era.
Raine Smythe aveva portato la grande priora a fare il suo gioco
nell'arco di pochi momenti, prima rimproverandola e poi dandole quello
che voleva per metterla a tacere, garantendo così che l'ultima parola
sull'argomento la sottomettesse alla sua volontà. Il politico che avrei
potuto essere applaudì dentro di me anche se il mio spirito soffocava al
suono delle urla.

Compresi che ci sarebbe stato sangue, qualsiasi cosa fosse successa.
Sempre sangue. Il sangue non è il fondamento della civiltà -- nostra o
di altri -- ma la pervade come calcina a ogni livello, cola dalle sue
pareti. Nonostante i vetri e la luce ariosa della stanza mi sentivo
incastrato come se fossi stato raggomitolato per il terrore in qualche
catacomba della mente, umida, ammuffita e sperduta. Quando pensiamo alla
guerra e alle sue atrocità, immaginiamo che le cose imperdonabili siano
commesse sul campo di battaglia, nel calore e nel fuoco, ma non è così.
L'atrocità è decretata da uomini pacati nelle camere del consiglio,
davanti a bicchieri di cristallo pieni di acqua fresca. Strani piccoli
uomini, con la cenere nel cuore, senza passione, senza speranza... senza
niente tranne la paura. Paura per loro stessi, per la loro vita, per
qualche futuro immaginario. E in nome della sicurezza, della devozione,
lavorano per fondare il paradiso futuro sull'orrore del presente. Il
loro regno paradisiaco è però nella loro mente, in quel futuro che non
sarà mai, mentre gli orrori del presente sono reali.

«Non puoi dire sul serio, tribuno-cavaliere» protestò lady Kalima,
spostando per un momento lo sguardo sul volto della tribuno. «Di certo i
prigionieri ci sono molto più utili se rimangono... ah... indisturbati.»

Raine Smythe guardò di sfuggita il conte sul suo alto seggio prima di
rivolgersi alla nobildonna jaddiana. «Se hai un suggerimento
alternativo, satrapo, mi piacerebbe molto sentirlo, ma questo pianeta è
sotto minaccia. So che non è uno dei \emph{vostri} pianeti, ma è
nell'interesse dell'Impero che Emesh rimanga... ah... indisturbato.» Nel
pronunciare l'ultima parola imitò la cadenza della satrapo. La mano di
Olorin si serrò intorno all'impugnatura rosso sangue di una delle sue
tre spade, preparandosi a sganciarla dalla cintura, e per un momento
pensai che a Borosevo stessimo per avere un altro duello. Poi però
l'alto maestro di spada lasciò andare l'arma senza commenti e nell'arco
di un momento il suo volto apparve composto quanto quello di uno
scoliasta.

«Non erano una forza di invasione.» Tutti gli sguardi si concentrarono
su di me, perfino quello della satrapo, e per un lungo istante non
riuscii a capirne il motivo, poi qualcosa scattò nella mia mente e aprii
la mia stupida bocca per parlare ancora. Ormai costretto a spiegare,
dissi: «Stavano cercando qualcosa. Sir Olorin, signore, tu eri presente.
Sei un soldato, tribuno-cavaliere, disponi dei rapporti: il design della
nave abbattuta sopra Anshar è coerente anche in minima parte con quello
di una nave militare?» Quando nessuno rispose mi guardai intorno,
allargando le mani. «No, dico sul serio. Lo è? Non sono esperto di
progettazione di navi. Qualcuno lo sa?»

Un logoteta minore, un plebeo tarchiato con i capelli che ingrigivano e
la pelle del volto pendente, si schiarì la voce e batté con lo stilo sul
legno pietrificato della superficie del tavolo. «Non sono stati trovati
armamenti per combattimenti da nave a nave nei rottami del velivolo
xenobita. Sembrerebbe che...»

Inarcai le sopracciglia. «Niente armamenti per combattimenti da nave a
nave, eh?» Imitai il tamburellare con le nocche della tribuno-cavaliere
e scrutai i nobili di due nazioni, la priora della Cappellania che mi
voleva morto, gli alti ufficiali della Legione imperiale e la folla di
logoteti prima di continuare. «Forse non c'è una terza ondata. Forse i
nostri amici Cielcin sapevano che non c'era speranza di essere soccorsi.
Forse la loro ritirata dentro Calagah rappresentava un'ultima, disperata
resistenza? L'\emph{ichakta} -- il loro capitano -- si è arreso solo
quando ho promesso assistenza medica.» Non era strettamente vero, come
hai visto, lettore, ma le sole persone in grado di convalidare la
storia, a parte me e Uvatari, erano Bassander Lin e il Cielcin Tanaran,
nessun dei quali era presente o parlava la lingua dell'altro.

«Vieni al punto, per favore» disse la cancelliera Liada Ogir.

«I nostri \emph{amici} Cielcin?» ripeté la grande priora, con il rossore
che le incupiva le guance incipriate.

«Una figura retorica» mormorò tor Vladimir, venendo in mio aiuto con la
sua voce sonnolenta.

Lasciai che la digressione della priora si spegnesse, tornando a imitare
il tamburellare di nocche della tribuno-cavaliere. «Sentite, sono pronto
a scommettere che le navi che avete distrutto in orbita erano una scorta
mandata a coprire l'approccio della nave che si è schiantata. Non erano
equipaggiate per un'invasione.»

«Allora cosa cercavano?» gracchiò il primo ufficiale Crossflane, dal suo
posto accanto a Raine Smythe, con un cipiglio che contrasse le sue
basette. «Sono spie?»

Lo fissai a bocca aperta. Naturalmente avevo un sospetto perché le
parole dell'\emph{ichakta} mi echeggiavano nella testa: `Loro non sono
qui.' Avevo bisogno di Valka, di parlare con lei. Valka avrebbe capito,
avrebbe potuto aiutarmi a dare un senso a tutto, avrebbe \emph{saputo}.
Il fatto che la grande priora fosse seduta proprio \emph{lì}, un
avvoltoio in vesti nere con la malignità che esalava dalla sua persona
come profumo, non contribuì di certo ad alimentare il mio nascente
coraggio. Balian Mataro mi osservava con la testa non più appoggiata sul
pugno, gli occhi neri che scintillavano come scarabei o come la pietra
nera di Calagah, e le labbra serrate. Il mio patrono. Il mio sponsor. Il
mio carceriere. Un folle sorriso minacciò di apparirmi sul volto e lo
repressi. \emph{La gioia è un vento.} A ogni parola mi esponevo a un
pericolo sempre maggiore con la Cappellania, ma non era la Cappellania a
motivare il mio gioco. Pensando ad Anaïs, al patto matrimoniale che
pendeva informale fra di noi, mi dissi: Vediamo se continuerai a tenere
i tuoi artigli piantati addosso a me, bastardo.

«Spie?» ripetei. «Non vedo come questo sarebbe possibile, signore.» In
base allo stemma diviso in quarti che spiccava sul petto della sua
uniforme nera sapevo che era un cavaliere, anche se il suo nome mi era
sconosciuto. Mi protesi in avanti, rivolgendomi unicamente alla
tribuno-cavaliere Smythe. «Se però mi permetteste di passare del tempo
con i prigionieri, e in particolare con il loro capitano, sono certo che
potrei ottenere di più da loro.» Avrei potuto aggiungere altro,
menzionare il collegamento fra i Cielcin e la Quiete, ma questo avrebbe
significato qualcosa solo per Ligeia Vas, e per quanto ne sapevo avrebbe
potuto far torturare \emph{me} per il disturbo che mi stavo prendendo.

«Qualcosa \emph{di più}?» sogghignò il primo ufficiale, girandosi con
rabbia incredula verso la sua superiore più giovane. «Raine, questo
ragazzo non può dire sul serio...»

«Lasciatemi tentare. Difendete lo spazio locale per... per una
settimana. Bloccate il pianeta, se vi aiuterà a rilassarvi, ma datemi
una possibilità. Il loro capitano parlerà con me, ne sono sicuro. Sono
certo di poter...»

«Basta così, Marlowe.» Il conte non gridò, non alzò neppure la voce. Era
proprio come mio padre, esattamente come lui. Si limitò a... a parlare e
scosse la testa sul suo seggio ad alto schienale sopraelevato rispetto
ai sui ospiti e ai consiglieri. Cambiò posizione e mise in linea le
spalle da toro. «Sono d'accordo con la tribuno-cavaliere e con la grande
priora Vas. Il nemico sarà interrogato e non voglio sentire altro al
riguardo.»

Era proprio come mio padre. Aprii la bocca per ribattere, con lo sguardo
fisso sulla tribuno e sull'ufficiale, entrambi in una divisa nera come
un sudario. Dovevo convincerli, dimostrare che potevo essere utile. Se
avessi potuto persuaderli ad accettarmi, avrebbero potuto reclutarmi e
sottrarmi a Mataro sotto il suo naso. Lo fissai con occhi roventi.
«Vostra eccellenza...» Mi alzai e mi inchinai profondamente sulla
superficie verde e rosa del tavolo. «Perdonami, ho insistito troppo e
chiedo scusa.» La punta del mio lungo naso sfiorò la superficie del
tavolo e sollevai di scatto il mento a guardare verso la piattaforma.
Per un momento pensai di trasformare la cosa in una farsa, gettandomi a
terra, percuotendomi il petto e implorando perdono. Non sarebbe stato
d'aiuto, ma quella presa in giro mi avrebbe fatto sentire meglio.

`Tutto quello che dici deve proprio sembrare tratto da un melodramma
eudoriano?'

Sì, Gibson, pensai.

«Rimettiti a sedere, lord Marlowe. Non abbiamo ancora finito con te.»

Mi sedetti con lo sguardo basso. Qualcosa nel modo in cui il conte aveva
pronunciato quelle parole era come un coltello che mi si rigirasse nel
ventre, ma nel mio stato sconvolto non vi badai più di tanto.
`Obbedienza per fedeltà verso la persona del gerarca.' Era vero, la mia
obbedienza non derivava certo da amore nei suoi confronti, ma non lo
odiavo neppure perché era fondamentalmente un uomo perbene. Piuttosto
ero risentito di ciò che rappresentavo per lui. Mi sentivo come
immaginavo che potesse sentirsi una principessa particolarmente capace
di destreggiarsi in una delle storie fantastiche della Vecchia Terra
scritte da mia madre: non solo ridotto al livello di animale da
riproduzione, ma anche disprezzato come persona, come intelletto.

La tribuno-cavaliere Smythe riprese il controllo della conversazione
come se non ci fossero state interruzioni. «La mia proposta è questa: la
massa dei prigionieri verrà tenuta nella bastiglia e trattata con
gentilezza. Nel frattempo isoleremo il capitano e lo consegneremo alla
Cappellania per l'interrogatorio. Siamo d'accordo?» Intorno al tavolo si
levò un mormorio di assenso, poi lei proseguì: «Dunque siamo d'accordo.
Questo...» Guardò verso di me. «Questo \emph{ichakta} verrà consegnato
alla Cappellania per l'interrogatorio. Gli Jaddiani \emph{}
parteciperanno come osservatori in quanto sono già coinvolti nella cosa,
e ogni informazione verrà condivisa fra i nostri gruppi.» Con un gesto
della mano abbracciò lord Balian, lady Kalima e sé stessa.

Dietro i miei occhi, ogni degradazione del corpo e dello spirito
inflitta dai cathar scorreva come un video ripetuto centinaia di volte a
velocità naturale. I tagli e le ustioni, le ossa rotte e la pelle
rimossa, le fronti marchiate e le narici tagliate, gli sventramenti, le
decapitazioni e gli stupri. Le urla che immaginavo si levassero da
Vesperad, dalle sue celle dalle pareti d'acciaio, fiorirono, avvizzirono
e tornarono a fiorire come i boccioli fanno una stagione dopo l'altra. E
quegli uomini e donne se ne stavano seduti al sole e al caldo, senza
sorridere ma comunque soddisfatti, mentre Ligeia delineava la fase
successiva dell'operazione.

E faceva di me un bugiardo. Avevo promesso al Cielcin che non gli
sarebbe stato fatto del male, gli avevo dato la mia parola come
palatino. Per il Grande Atto Costitutivo, la mia parola era una sorta di
legge, e adesso mi stavano chiedendo di infrangerla. Soprattutto, era un
colpo personale, un affronto al mio senso dell'io, a chi ero su questo
mio nuovo mondo: di nuovo Marlowe, ma non di Meidua.

«...naturalmente dovrà essere presente. Ci serve un traduttore.»

Traduttore. Quella parola -- le sue particolari associazioni e la sua
affinità con me stesso -- emerse dal pantano di fallimento che rimaneva
di quella riunione. Traduttore. Poi il suo significato mi trafisse,
affondando come una freccia, come una lama. «No!» Quasi mi alzai di
nuovo in piedi. «No, non lo farò!»

Ligeia Vas sorrise. Per lei era una vittoria morale, anche se non una
che terminasse con la mia morte. «Non hai scelta. Come hai detto tu
stesso, pare che nessuno sia più adatto a questo compito.»

«No!» A quel punto mi alzai in piedi, sorprendendo i due logoteti che
occupavano i posti accanto al mio, e spostai lo sguardo sconvolto su
Raine. «Vuoi dirmi che non avete traduttori su quella tua nave?»
L'\emph{Incrollabile} era un super trasporto per truppe che conteneva
dozzine di fregate più piccole, migliaia di membri dell'equipaggio.
«Nessuno?»

«Non ci sono molti scoliasti a bordo delle navi della Legione, ragazzo»
replicò sir William Crossflane.

Disperato, mi rivolsi a lord Balian. «Vostra eccellenza, per favore.
Devi proibirlo.»

«Volevi parlare con i demoni, ragazzo» commentò la priora, rispondendo
al posto del nobile di cui era nominalmente al servizio. La sua faccia
bianca risplendeva della stessa tonalità delle maschere funebri della
mia famiglia, e quegli occhi azzurri avrebbero potuto essere viola per
un qualche gioco di luce. Scintillarono, poi tornarono a essere solo
dell'azzurro di quello spento di Gilliam, fisso e cieco a ogni cosa.
«Parla con loro.»

