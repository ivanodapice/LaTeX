\chapter{La Spada, Nostro Oratore}

Mentre stringevo il giustacuore marrone da duello decisi che c'erano
posti peggiori dove morire che non la grotta artificiale che sovrastava
il bianco campo erboso ed era intagliata con figure della Giustizia
cieca, del Coraggio dagli occhi spalancati e della morte stessa,
riparata dalla pioggia da una facciata naturale di arenaria grezza, la
stessa pietra con cui era stato costruito il castello di Borosevo.
Quell'altare, che evocava i santuari delle antiche divinità pagane della
Terra, si trovava all'estremità più stretta del giardino ellittico
bordato di piante di tasso terrestre dalle foglie di un verde cupo che
apparivano nere nell'ardente color ruggine del sole di Emesh. Quelle
piante creavano un adorabile contrasto con l'erbaosso bianca come il
latte che ricopriva quello strano posto.

E poi c'erano i fiori.

Quei boccioli nativi del pianeta, e grossi come la testa di un uomo
quando erano chiusi, erano sparsi nelle alte pareti delle siepi, umidi e
pesanti sui loro viticci dorati, e pervadevano l'aria di un profumo
intossicante e così gradevole da rendere respirabile l'aria pesante. E
si muovevano, aprendosi e chiudendosi come un cuore che batte o come lo
sbattere di ciglia di centinaia di occhi sonnambuli. Mi sentii
trasportato, come se l'arco d'ingresso alla grotta fosse stato il
portale di un regno fatato e quel piccolo giardino una fetta delle
foreste di Luin di cui parlava Cat.

Le ferite che avevo riportato nello scontro con gli Umandh prudevano
ancora. Dopo quel pomeriggio nel magazzino, il conte Mataro aveva
provveduto a farmi ricevere le cure migliori, e a dire il vero ero
stupefatto per la rapidità con cui ero guarito. La carne si era richiusa
bene, ma la pelle nuova bruciava ancora dove erano stati applicati i
correttivi. Grattandomi, mi guardai di sfuggita alle spalle in direzione
della mia piccola folla di sostenitori per offrire un sorriso di
incoraggiamento. Valka era assente, cosa di cui ero in pari misura lieto
e addolorato, ma Switch era presente come mio secondo -- aveva già
trattato con il secondo di Gilliam per la scelta di quel posto -- e con
mia sorpresa anche Anaïs e Dorian erano venuti. Ancor più sorprendente
era la figura di sir Elomas Redgrave che, seduto su una panca, beveva
del tè dal cappuccio di un thermos mentre parlava in tono sommesso con
Switch. Valka gli aveva chiesto di venire? Rappresentava i suoi occhi?

«Hai paura, Marlowe?» Gilliam mi fissò con occhi roventi dalla sua
postazione dall'altro lato della distesa di erba bianca, assistito da un
paio di anagnosti della Cappellania in vesti nere. Il gargoyle portava
alti stivali e calzoni neri, e perfino il suo giustacuore di cuoio era
nero. Senza le pesanti vesti che ne nascondevano la figura, la miriade
di imperfezioni della sua forma urlava per chiedere attenzione: la gobba
pronunciata, le gambe di lunghezza impari, il modo in cui si muoveva,
come se stesse barcollando sull'orlo di un qualche abisso.

Chinai il capo. «Per niente.» Ero stato un mirmidone per anni, e prima
ancora un allievo di sir Felix Martyn per più di un decennio. Ero alto,
sano, il mio sangue non era contaminato e i miei arti erano dritti e ben
definiti. Che minaccia poteva mai costituire quell'intus? Lo guardai
spostarsi sull'erba come un granchio.

«I combattenti non devono parlare!» intervenne l'arbitro plebeo, un uomo
dal volto squadrato con capelli castani che si andavano diradando ed
erano pettinati all'indietro al di sopra di una fronte alta e
aggrottata. Questo impedì all'intus di ribattere.

Switch si affrettò a venire avanti, aggirando un paio di assistenti
all'arbitro che portavano la divisa dei prefetti cittadini. «Siamo quasi
pronti a cominciare. Hai le armi?»

L'arbitro dal volto squadrato estrasse un paio di spade a filo singolo.
Con cura, ne controllò il filo con il pollice e annotò qualcosa sul
terminale da polso. «Lo ferirò per primo e la chiuderò lì.» Scossi il
capo. «La dottoressa aveva ragione, non avrei dovuto farlo.» Gilliam
allontanò da sé i suoi assistenti e si infilò un guanto nero da scherma
che appariva fatto su misura per il suo braccio dalla forma strana. Mi
accigliai, perché quel guanto pareva logoro per l'uso.

«Ma non mi dire, Had» commentò istintivamente Switch, poi con l'angolo
dell'occhio lo vidi irrigidirsi. «Chiedo scusa, signore, mio signore.»

«Vuoi smetterla?» Mi sedetti su una pietra piatta e abbassai la cerniera
degli stivali. «Sono esattamente la stessa persona che sono sempre
stato.»

Switch si agitò, a disagio, e distolse lo sguardo. «Non... non dà questa
sensazione.»

A quanto pareva, avevo un vero talento per alienarmi le persone.
Sollevando lo sguardo mi sfilai gli stivali e i calzini neri fino a
esporre lo spesso strato calloso sulla pianta dei piedi. Switch aveva
visto quella routine un migliaio di volte, durante gli addestramenti,
quindi non chiese cosa stessi facendo. «Grazie per essere qui, Switch.
Significa molto per me. Davvero.»

Non rispose mai perché in quel momento l'arbitro dalla calvizie
incipiente levò la sua limpida voce nasale. «I combattenti si
avvicinino.»

«È la mia battuta per entrare in scena» commentai rivolto a Switch,
cercando di apparire gioviale, ma non sono certo di esserci riuscito. A
piedi scalzi attraversai lo strato di erbaosso fino a dove i funzionari
civili erano raccolti intorno alle apparecchiature richieste per legge
in un \emph{duello palatino}. Occhi muniti di videocamera orbitavano
intorno al gruppo di uomini dell'ufficio del prefetto, pronti a
registrare il duello, e in quel momento uno di essi stava già
registrando una prova visiva dei testimoni legali, fra cui i principali
erano Anaïs, Dorian ed Elomas. Avevo formalizzato le mie accuse giorni
prima, la sera del fallito attentato degli Umandh contro la vita del
conte. `Per pericolose calunnie contro una conoscente personale e a
sostegno di passate accuse di aggressione e di insulti contro la mia
esaltata persona.' Il gergo formale era irritante quanto il colletto di
un'uniforme. Forse avevo perso un po' della mia finezza imperiale.

Piegando brevemente a terra un ginocchio per riceverla, accettai la
spada dall'arbitro dopo che Gilliam ebbe preso la sua. «Combatterete
finché uno dei due non sanguinerà. A quel punto, al combattente non
ferito verrà concessa l'opportunità di porre fine al duello, com'è
usanza in base all'Indice e ai Grandi Statuti dell'Impero fin
dall'Assunzione della Terra.» La semplice menzione del Mondo Natale
indusse Gilliam e i suoi colleghi della Cappellania a tracciare con
discrezione il segno del sole mentre io, da quello sporco apostata che
ero, rimasi impassibile. Il prete intus se ne accorse e sogghignò, ma
non fece commenti. L'ometto non aveva finito di parlare. «Se non si
approfitterà di quella possibilità di porvi fine, il duello continuerà
fino a quando una delle due parti non potrà più combattere. È tutto
chiaro?»

Due `sì' risuonarono nell'aria immobile, due uomini presero posizione
uno di fronte all'altro nell'ombra dei tre prefetti che sovrintendevano
al duello. Tre icone della Cappellania ci osservavano dalla parete
artificiale del tempietto nella grotta, facce di pietra cieche come
maschere funebri. Una nebbia sottile si levava dall'erbaosso, riempiendo
la densa aria di Emesh di umidità. Era come camminare in un sogno, dove
tutto era annebbiato, silenzioso. Non sentii veramente l'arbitro mentre
recitava le accuse formali e invece osservai Gilliam, con i capelli
biondi unti per tenerli lontani dalla fronte alta e deforme, intento a
osservarmi con quei suoi occhi socchiusi e spaiati.

Spinsi in avanti la spada con la lama sollevata e leggermente inclinata
verso l'alto, e mi premetti il pugno sinistro sul petto nel saluto di un
cavaliere, un gesto a cui avevo ben poco diritto ma che nessuno mi
avrebbe contestato. Poi tenni la lama in avanti, pronta.

«Ti sei confessato?» chiese Gilliam, ora che non eravamo più obbligati
al silenzio. «Sono disposto ad ascoltare i tuoi peccati, prima che tu
muoia.»

Non replicai e non mi mossi se non per flettere le dita dei piedi
nell'erba umida. Sarebbe stata una giornata rovente, con il grosso sole
che sorgeva tinto del colore del sangue. Avrei potuto essere una statua,
immobilizzato in quel momento, con ogni precedente secondo che mi aveva
portato a quella grotta e a quella umida mattina nebbiosa. Le mie
decisioni avevano garantito che non potessi fare altro che percorrere il
sentiero su cui ero avviato. Avevo scelto.

Gilliam eseguì un affondo e io parai di scatto, indietreggiando di un
passo. Il clangore del nudo acciaio -- e non dell'altamateria delle
spade vere -- suonò come una musica nell'aria immota mentre inarcavo di
scatto le sopracciglia. L'intus era veloce, molto più di quanto
lasciassero supporre le sue gambe ineguali, e la sua forma era buona,
sorprendentemente diritta nonostante il suo corpo distorto. Quello non
sarebbe stato un duello facile, o almeno non facile quanto avevo
creduto. Lasciai dissolvere la mia iniziale arroganza. \emph{L'orgoglio
	viene prima della distruzione.} Gibson non era mai lontano, con le sue
infinite citazioni che mi risuonavano all'orecchio. Il mio orgoglio si
dissolse con la mia fortuna, ritirandosi di fronte all'avanzata del
prete. Con i denti snudati, Gilliam mirò basso e io mi girai per parare,
distraendolo mentre usavo la mano sinistra per colpirlo alla faccia.
Barcollò all'indietro con un sussulto e un livido che cominciava a
formarglisi su una guancia.

Ringhiando, si lanciò in una nuova offensiva e io lo aggirai sulla
sinistra, riluttante a farmi sospingere verso la parete della grotta,
dove si trovavano i nostri testimoni. Sentii una donna (Anaïs?)
sussultare mentre spingevo da un lato la spada di Gilliam con un
risonante clangore. Tutta quella situazione non era reale, non poteva
esserlo. La lama di Gilliam scattò verso uno dei miei reni ma la deviai
con un colpo di gomito, spingendo la punta da un lato prima di portarmi
a tiro della sua spada e di levare la mia arma in alto e intorno alla
testa in un fendente roteante di quelli che si eseguono con la sciabola,
che avrebbe dovuto di certo abbattersi sulla congiunzione fra il suo
collo e la spalla.

Gilliam però svanì, gettandosi a terra e rotolando in senso antiorario
di quasi novanta gradi. La cosa mi lasciò tanto stupito che quasi mancai
di saltare da un lato e sentii la punta della sua arma che mi strisciava
sul giustacuore. Mi ero abituato a tal punto a combattere usando il
pesante scudo rotondo dei mirmidoni che mi ero dimenticato come
impegnare un vero duello? Sentii Switch emettere un sibilo di
partecipazione che passò inosservato.

Forse Valka aveva ragione, forse eravamo davvero dei barbari. Che
bisogno avevo di questo? Se non fossi stato sovraeccitato sulla scia
dell'attacco degli Umandh contro il conte, se non avessi permesso al mio
cuore e ai miei inutili sentimenti nei confronti di Valka di governare
la mia testa, nulla di tutto questo sarebbe successo, ma tutti noi
commettiamo degli errori e dobbiamo accettarli. Non so dire se la
competenza di Gilliam, la sua abilità, resero il duello una farsa in
misura maggiore o minore. Spinse di lato la mia arma e accennò un
affondo, ma io lo misi a distanza, in grado di ritirarmi sulle gambe
lunghe e diritte più di quanto lui potesse avanzare con le sue, che
erano deformi.

Lo sbaglio dei poeti, dei librettisti come mia madre, è quello di
credere che un combattimento sia sempre e comunque una cosa soltanto.
Credono che il lavoro del soldato, il mestiere del gladiatore, l'arte
del duellista siano tutti la stessa cosa e che tutte siano uguali al
caos di un combattimento per le strade e nei villaggi di campagna di un
migliaio di mondi. Però ci sono modi diversi di combattere. Ricordai
quella volta che avevo gettato a terra Ghen nel cortile di addestramento
per indurre tutti a spalleggiarmi; in realtà, per salvare Switch.
Dov'era adesso quell'Hadrian?

Sferrai un calcio dietro il ginocchio a Gilliam, facendolo barcollare
prima di sferrare un colpo dall'alto. Attaccai una volta, due, una
terza, questa volta lateralmente, ma quel piccolo goblin parò ogni colpo
con una forza che mi stupì. Adesso stava indietreggiando, e per un
momento vidi l'espressione cerea sul volto dei due figli di Mataro. Cosa
avrebbero pensato se il loro nuovo amico avesse ucciso il loro prete su
quel campo sanguinoso, in quella cupa mattina? Avevo bisogno di versare
il primo sangue, di chiudere quella farsa che avevo scritto io stesso,
avevo la necessità di scusarmi con le azioni oltre che con le parole, di
redimermi agli occhi di Valka.

Valka.

Non vidi arrivare il colpo che mi ferì, ne sentii solo il dolore acuto
che mi percorreva il braccio, poi il sangue colò rosso sul color crema
della manica lacerata, tingendosi leggermente di marrone a contatto con
il tessuto. Come mi era sfuggito? Barcollai all'indietro con un sibilo,
imprecando sotto voce in mandar, con tutte le lingue che ci sono, e
gemendo mentre le mie promesse scorrevano via con il mio sangue. Dal
punto di vista clinico si trattava di una ferita superficiale, un taglio
pulito sull'avambraccio, ma in un modo che trascendeva i fatti della
nostra disputa quella era una ferita mortale.

«Fermi!» Adesso la voce nasale dell'arbitro si era fatta profonda: stava
imitando il tono di comando di un sergente. «Primo sangue a favore del
convenuto!» Concentrò il volto squadrato su Gilliam, con gli occhi verdi
dilatati sotto le sopracciglia aggrottate. «Il gentiluomo è
soddisfatto?» chiese, con mortale serietà.

Per quanto si era mosso in fretta, non c'era traccia di sangue sulla
lama di Gilliam. Resistetti all'impulso di serrarmi il braccio e
indietreggiai tenendo pronta la spada in una pallida eco del mio
precedente saluto. Solo allora mi resi conto che avevo il respiro
affannoso e che ero stanco. Per la Terra e l'imperatore, dopo essere
arrivato a palazzo avrei dovuto mantenere il mio regime di
addestramento, invece gli ultimi mesi mi avevano rammollito. Serrai la
mascella. La ferita mi doleva, ma quella non era la cosa peggiore.

Non avevo potere in quel momento, mentre attendevo la risposta di
Gilliam. Il mio mondo, la mia promessa a Valka, dipendevano da quella
risposta. Avevo fallito. Ero stato così sicuro che avrei versato il
primo sangue e che avrei potuto porre fine a tutto in quel momento
critico, uscendone con un po' di grazia e di dignità. Ne ero stato così
sicuro, e adesso tutto questo mi veniva strappato via e lasciato nelle
mani di quella creatura asimmetrica che impugnava una spada identica
alla mia. Trattenni il respiro e sentii il mondo rimanere sospeso su di
esso.

«No!» Gilliam sogghignò e si lanciò in un affondo.

Dannazione al mio talento nel crearmi dei nemici!

Sventai il primo nuovo attacco, spingendo da un lato la lama diretta al
mio cuore avvilito mentre il tempo si restringeva di fronte a me e il
futuro che avevo davanti si assottigliava fino a passare dalla vaga
chiazza quantica di un potenziale a un singolo paio di porte. Al di là
di una mi ergevo sul corpo muto di Gilliam con una spada insanguinata in
mano, al di là dell'altra le nostre posizioni erano invertite. Sollevai
con forza la spada, muovendomi in avanti e verso destra per bloccare una
carica e sfilarmi dalla direzione del rinnovato attacco di Gilliam.

«Sorpreso, ragazzo?» chiese, con i denti serrati. «Ti aspettavi che
sarebbe stato facile? Coraggioso da parte tua.» Parò un affondo con un
movimento fluido e preciso della spada e avanzò con una risposta bassa
che avrebbe dovuto trapassarmi una coscia. Saltellai all'indietro,
coprendo la ritirata con un fendente alla spalla dell'intus, mentre
riflettevo che era strano che sua madre non fosse presente. «Coraggioso
da parte tua sfidare uno storpio.»

Lo raggiunsi con un fendente sul davanti della coscia sinistra. Il cuoio
nero si aprì leggermente e Gilliam sussultò. «Avresti dovuto accettare
il primo sangue. Non commetterò di nuovo lo stesso errore» ribattei,
rinnovando i miei attacchi e costringendolo a {indietreggiare} sull'erba
bianca verso il muro di fiori pulsanti, con il clangore dell'acciaio che
risuonava nell'aria spessa. Il braccio sinistro mi bruciava e sanguinava
mentre avanzavo, ma serrai i denti e accentuai la pressione, mirando
alla testa e alle spalle del prete. La verità di quello che avevo
cercato di dire a Valka mi risuonava nelle orecchie: non ero un
uccisore, non lo ero mai stato.

Gilliam mi si scagliò contro sputando, con la spada che puntava dritta
ai miei occhi, e fui salvato dalla cecità o dalla morte soltanto da un
blocco istintivo che mi lasciò esposto a un controfendente, e fui
fortunato che la rapidità della mia difesa avesse sorpreso Gilliam al
punto da immobilizzarlo. Rimanemmo così per un istante studiandoci a
vicenda. Se mai ci fu un momento in cui parlare e arrivare a una
comprensione reciproca, si trattò di quello, ma non lo facemmo.

Ringhiando, tornò a lanciarsi contro di me e io parai, agganciando la
sua lama e colpendo verso il basso e di traverso in direzione del suo
fianco destro. La punta della mia spada incontrò l'osso e lui represse
un grido. Per un momento ebbi una nitida via di attacco alla sua gola ma
non ne approfittai e invece indietreggiai come avevo fatto migliaia di
volte con Crispin, aspettando. Uno degli arbitri mormorò qualcosa al suo
compagno dal volto squadrato, e anche se non riuscii a cogliere le
parole il tono fu di ansiosa disapprovazione. Lanciai un'occhiata ai
presenti. Il volto rugoso di Elomas si era incupito mentre mi osservava
nel sorseggiare il suo tè.

«Avrebbe potuto concludere...»

Girai intorno a Gilliam con la lama posizionata in una guardia bassa,
ruotando in modo da mantenere il fianco destro orientato verso di lui.
«\emph{En garde}!» dissi. \emph{Vieni avanti e combatti, bastardo.}
«\emph{En garde}!» ripetei, agitando la spada, la cui punta descriveva
piccoli cerchi ansiosi. Volevo pungolarlo, indurlo a commettere un
errore. Crispin aveva abboccato quasi ogni volta, con la chimica del suo
cervello annullata dai suoi androgeni, che gli mettevano i paraocchi
come a un cavallo in una parata. Non funzionò. Il prete zoppicante
rimase dov'era con la mascella serrata e le spalle quanto più in linea
gli era possibile.

Non potevo più aspettare e venni avanti, con la spada che abbatteva un
colpo dopo l'altro sulla sua guardia. Adesso stava {combattendo} con
cautela, senza più i movimenti spastici e il rapido lavoro di piedi che
avevo imparato ad aspettarmi. Con uno scatto del polso spinsi di lato la
sua spada e lasciai di nuovo aperta una chiara linea di attacco, questa
volta al suo torace da piccione.

Non ne approfittai. Non potevo farlo. Non vidi neppure
quell'opportunità, accecato com'ero dai sentimenti. Non avevo mai
ucciso, quindi non potevo farlo. Cedetti terreno, ritirandomi nella
sicurezza offerta dalla posizione di guardia. Percepivo l'inquietudine
degli osservatori anche se non la comprendevo, confondendola con il
disagio che chiunque avrebbe provato nel sapere di essere là per
assistere a un'uccisione. «Non ti voglio uccidere» dissi infine,
incurvandomi in posizione di guardia.

Gilliam descrisse un arco alla mia destra e io seguii il suo movimento,
tenendo il piede anteriore puntato nella sua direzione. «Hadrian, stai
giocando con lui!» La voce era quella di Anaïs, acuta e pervasa di
tensione nervosa. Seguì un momento di assoluta immobilità, con
l'ologramma della nostra vita messo in pausa, sospeso. Solo i fiori si
muovevano, solo loro respiravano.

Qualcosa di simile a un'ombra passò sul volto irregolare del cantore,
scivolando attraverso i suoi occhi fino alla sua anima. Come per la
forza di gravità, non poteva essere visto che tramite i suoi effetti. Le
labbra distorte si contorsero, l'occhio scuro si incupì e quello azzurro
si paralizzò fino a sembrare creparsi. In lui ogni tendine era teso come
la corda di un arco. «Ti ucciderò, eretico» ringhiò. «Non ti permetterò
di distorcere questo posto. Questa gente. La mia gente.»

Sono rimasto a lungo seduto qui nella mia cella, a Colchide, senza
scrivere una sola parola. L'inchiostro vermiglio che il mio ospite mi ha
fornito si è asciugato e le candele si sono spente. Ho mandato a
prendere una nuova boccetta di inchiostro e altre candele; qui la notte
è interminabile. Forse in tutto questo c'è un significato.

Gilliam era mosso dall'ira, che lo accecava e che per poco non accecò
anche me, tanto furono rapidi i movimenti di quella spada. La sua fretta
e la sua furia però lo resero impreciso, tanto che avrei potuto
ucciderlo altre tre volte, con un affondo all'addome, con un ampio
fendente alla gola e di nuovo con un colpo che gli avrebbe sfondato quel
brutto cranio e tinto di rosso i capelli biondi. Eppure non potei farlo.
Potrai trovare strano che io, che ho cenato consumando più sangue di
quanto abbia fatto la maggior parte degli imperi, non riuscissi a
uccidere un singolo uomo. Lo ripeto: una singola morte è una tragedia.

Invece lo trafissi di nuovo al fianco, con l'acciaio che strideva contro
l'osso, e la punta della mia spada si tinse di un rosso
sorprendentemente vivido nell'aria del mattino. Gilliam digrignò i denti
al mio indirizzo e quasi mi aspettai di vedere sangue sulle sue gengive.
«Demone!» sputò invece. «Abominio!» Di cosa parlava? Barcollai
all'indietro, tenendo la spada fra di noi e cercando di non pensare al
sangue sulla sua punta. «Minaccia...» stava dicendo. «Spia...» Era
ancora convinto che fossi parte di una qualche cospirazione contro il
Paese, contro la fede, e tutto a causa del mio interesse per il suo
Cielcin.

A volte non c'è un momento culminante, una cosa succede ed è tutto
finito. Gilliam eseguì un altro affondo, io lo parai e mi estesi in una
risposta. Con un semplice movimento la mia lama passò davanti al mio
petto con la punta ancora diretta in avanti per spingere da un lato il
suo selvaggio affondo. Intanto venni avanti, abbassando la spalla destra
per allineare la punta con le costole di Gilliam con uno stridere di
metallo contro metallo, poi sul cuoio. Su un osso. Poi il sangue fiorì
rosso, tingendosi di scuro sul nero del suo giustacuore, e il respiro
gli sfuggì dalle labbra in un gemito inarticolato. Rosso e nero, pensai.
I miei colori.

L'impeto in avanti di Gilliam lo portò dritto contro la mia spada e lui
si accasciò là, trasformato in un peso morto. Rantolò, un risucchiante
suono umido nel profondo del petto, segno che dovevo aver trapassato un
polmone. Non c'era niente da fare. Lo spinsi indietro e dovetti
piantargli un piede sul petto per liberare la spada dalle sue costole.
Colpì l'erba con un gemito che si trasformò in un gorgoglio, mentre io
lottavo contro l'impulso di buttare da un lato la mia arma. Ero in
mostra, sotto lo sguardo vigile del mio pubblico silenzioso. Le
ginocchia mi cedettero e caddi, puntellandomi su quella lama traditrice
che stringevo nella mano ancora viva. \emph{Valka, perdonami.}

La spada dell'intus era scivolata via dalle dita rilassate, ed ebbi
abbastanza presenza di spirito da gettarla da un lato. La tradizione
vietava interventi medici: entravamo sul campo sapendo cosa {stavamo}
per fare. Potevo già avvertire il disprezzo di Valka. Le mani mi
tremavano mentre ogni battito del cuore di Gilliam riversava altro
sangue sul terreno. Faceva caldo, troppo caldo. Il cantore sollevò una
mano che era salda, al contrario delle mie, e la protese lentamente.
Pensai che stesse per tracciare il segno del sole in un'ultima
benedizione, ma si protese invece verso i presenti e i figli del conte.
«Mia signora... lord Dorian. Non... vi fidate...»

Sollevai di scatto lo sguardo, appuntandolo con espressione rovente
dall'altro lato del campo dove Anaïs e Dorian erano fiancheggiati da
Elomas e dai prefetti. Il volto scuro di Anaïs appariva in certa misura
sbiancato. Lei scosse furiosamente il capo, poi saettò verso l'arcata
dell'uscita, mentre suo fratello la chiamava e un paio di peltasti in
armatura si affrettava rumorosamente a seguirla. In ginocchio, a bocca
aperta, la guardai andare via.

Il prete impiegò molto tempo a morire, con il petto che si alzava e
abbassava con movimenti sempre meno marcati, ridotti dal suo declinare.
Respiri piccoli, più piccoli.

Immobilità.

Ero ancora in ginocchio accanto al suo cadavere quando i soldati vennero
a prendermi. Il loro capo, una donna alta che non conoscevo, con le
spalline che la indicavano come un centurione della guardia personale
del conte, disse: «Lord Marlowe, devi venire con noi.»

Non risposi, mi limitai a chiudere gli occhi e con uno sforzo tremendo
mi rialzai in piedi.

