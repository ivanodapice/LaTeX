\chapter{Meidua}

Fuori l'aria era più fredda e i rumori del tumulto nel colosseo erano
soffocati e lontani. Il pomeriggio cominciava a cedere il passo alla
sera e l'enorme sole pallido era rosso e accoccolato sopra le basse
torri di Meidua. In lontananza la nostra acropoli e il castello nero che
era la mia casa incombevano come una nube temporalesca e io ero solo.
Gli uomini e le donne che camminavano lungo la strada fuori dall'arena e
dai giardini del Circus mi sembravano tanto distanti che avrebbero
potuto essere i membri di un'altra specie.

Forse mio padre aveva ragione a dubitare di me. Se non riuscivo a
sopportare la violenza del Colosso, come ci si poteva aspettare che
governassi una prefettura come faceva lui? Come ci si poteva aspettare
che facessi le scelte dure e sanguinose che sono l'anima del governare?
Mentre mi allontanavo in fretta dal colosseo, oltrepassando l'ippodromo
e il grande bazar, la mia mente si rivolse al Casato Orin, alle sale
infrante su Linon, a mezzo sistema di distanza, e dissi a me stesso che
io non avrei potuto fare una cosa del genere, che non ero tanto forte o
tanto crudele. Pensai agli schiavi morti, allo stivale del gladiatore
che percuoteva la carne sbiancata fino a schiacciare la testa sotto la
sua suola. Accelerai il passo, desiderando di poter camminare abbastanza
in fretta da lasciare tutto il mondo.

Immediatamente al di là dei terreni del Circus la città raggiungeva
altezze più elevate, con i lampioni già accesi, mentre le torri
scolpivano la loro ombra sulle strette strade. Una navetta passò in alto
e in lontananza riuscii a distinguere a stento la scia di condensazione
a fusione di un razzo che saettava verso il cielo. Desiderai di essere a
bordo, diretto da qualche parte, da qualsiasi altra parte. Sapevo che
sarei dovuto tornare indietro, almeno alla mia navetta se non al palco e
ai giochi, ma il pensiero di tornare al colosseo, di vedere Crispin
impegnato nel suo gioco sanguinoso mi riempiva di inquietudine. Mi
fermai per un momento nell'ombra di un arco trionfale, osservando le
macchine che mi oltrepassavano nello zigzagare lentamente in mezzo ai
pedoni, che erano numerosi così vicino al colosseo.

Un vento teso soffiava da dietro una curva della strada, portando con sé
l'odore della salsedine e del mare insieme ai versi di uccelli lontani.
Lo svanire del giorno era tinto di un vago senso di freddo che
annunciava la fine dell'estate, e mi strinsi maggiormente nella giacca.
Decisi che sarei tornato a casa a piedi, e al diavolo mio padre e quello
che avrebbe avuto da dire al riguardo. Il castello non era troppo
lontano, appena poche miglia verso nordovest su per le strade tortuose
che aggiravano una curva nelle alture di calcare e si dirigevano verso
le scale della Porta Cornuta.

Mi avviai quindi sul lungofiume, procedendo parallelo alla Punta
Denterosso e salendo verso le cascate, con le loro grandi chiuse.
Accanto a me il fiume ronzava al passaggio di piccole giunche da pesca e
delle pesanti chiatte che trasportavano merci provenienti da monte. Al
di sopra dell'acqua giungeva un suono di rauche e rozze voci maschili.
Indugiai là per un momento, osservando il passare di un'antiquata galea
manovrata da servi, che lottava contro la lenta corrente del grande
fiume per tornare verso le distanti montagne. Vago, mi giunse
all'orecchio il grido del capo vogatore insieme al ritmo del tamburo.
«Remate verso casa, ragazzi miei» gridava, scandendo le parole al ritmo
del tamburo. «Remate verso casa.»

Mi fermai per un momento a osservare quella nave in vecchio stile finché
un mercantile con il diavolo dei Marlowe dipinto sulla fiancata non la
nascose alla mia vista. I servi non avevano altre possibilità. Avevano
la proibizione di ricorrere perfino alle tecnologie concesse agli operai
della nostra Gilda, quindi si arrabattavano con il sudore della fronte e
la forza delle braccia.

Avevo una mezza idea di tornare indietro per dirigermi ai moli e al
mercato del pesce in cui mi ero aggirato spesso da ragazzo. In un
negozio d'angolo c'era un nipponese che arrotolava il pesce con il riso,
e nella Città Bassa c'erano intrattenitori che facevano combattere gli
animali uno contro l'altro. Ero però consapevole del pericolo che
correvo, un nobile di palazzo che camminava apertamente per strada
abbigliato con l'eleganza che si addiceva alla sua posizione. Rigirai
nervosamente intorno al pollice l'anello con il sigillo e armeggiai con
il sottile bracciale del mio terminale. L'istinto mi diceva di chiamare
per ricevere rinforzi o quantomeno per avvertire Kyra che non l'avrei
incontrata alla navetta.

Come tutti i giovani che si trovino di fronte a un periodo difficile
della loro vita ci tenevo a proteggere la mia privacy. Mi allontanai dal
lungofiume, attesi che le macchine smettessero di passare e attraversai
la strada, seguendo una via che risaliva tortuosa il pendio,
oltrepassando vetrine illuminate, bancarelle che vendevano prodotti
agricoli, icone della Cappellania in plastica stampata e in finto marmo.
Declinai cortesemente l'offerta di una donna di intrecciarmi i capelli,
poi ignorai le sue grida irose in cui diceva che qualcuno con i miei
capelli lunghi doveva essere un catamita. A quei tempi era di moda che
gli uomini portassero i capelli corti, come facevano Crispin o Roban, ma
io -- e forse questo era un emblema delle mie manchevolezze come erede
-- preferivo ignorare la popolazione. Avrei voluto far notare alla donna
che anche il suo archon portava i capelli lunghi, ma mi trattenni e la
lasciai a gridare sull'angolo.

Come ho detto, mi sentivo quasi parte di un'altra specie, e a causa
della lunga storia di modifiche genetiche della mia famiglia in effetti
-- nonostante la mia bassa statura -- ero più alto della maggior parte
dei plebei che oltrepassavo, i miei capelli erano più scuri, la pelle
più chiara. Anche se ero ancora un bambino di appena venti anni
standard, mi sentivo vecchio al confronto di quei mercanti e operai
prematuramente invecchiati, non perché lo fossi davvero ma perché sapevo
che quei volti rugosi e quelle mani simili a cuoio erano a stento più
vecchi dei miei. Il loro corpo li aveva già traditi. Forse era a causa
della barbarie del Colosso, o forse i miei sentimenti sono stati
annebbiati da ciò che ha fatto seguito a quella passeggiata, ma
ripensando a quel momento ricordo i volti dei Meidusani come poco più
che caricature. Ciascuno di loro sembrava il bozzetto di un bambino o
una scultura primitiva eseguita da qualcuno che aveva solo una vaga idea
di come dovesse essere un umano. Le sopracciglia erano massicce e i pori
larghi, la pelle unta era chiazzata e indurita dal sole come la mia non
lo sarebbe mai stata. Non mi soffermai a chiedermi chi di noi fosse
davvero umano. Ero io, con i miei geni modificati su misura dal Collegio
e il mio portamento regale? Oppure loro erano più umani di quanto io
avrei mai potuto esserlo perché la loro condizione era uno stato di
natura? A quel tempo credevo si trattasse di me, ma come nella parabola
del capitano che ripara la sua nave un'asse dopo l'altra finché non
torna a essere nuova, adesso non posso fare a meno di chiedermi quante
linee si possono riscrivere nel sangue prima che un uomo cessi di essere
tale, o addirittura di essere umano.

La strada successiva che trovai si snodava verso l'alto e sulla destra,
con le facciate di vetro e calcare e che si incurvavano ed erano
artisticamente ricoperte di viti, anche se era il periodo sbagliato
dell'anno perché dessero i loro frutti. Oltrepassai gli uffici di
parecchi importatori di merci e un posto dove la gente comune poteva
pagare per farsi sostituire i denti marci. I loro non ricrescevano a
tempo indefinito, e mi era stato detto che crescevano in modo
imperfetto. Lo avevo sempre trovato strano, ma poi ripensai alla madama
direttrice Feng e ai suoi impianti di acciaio inossidabile. Perché aveva
optato per una cosa del genere quando erano disponibili denti bianchi?
Quel pensiero mi assorbì in modo così completo che non interpretai il
rumore del motore della moto per quello che era e non sospettai il colpo
in arrivo finché non mi raggiunse alla schiena.

L'impatto mi strappò il respiro dal corpo e caddi sulla pavimentazione
con un grugnito, bloccando sotto di me il lungo coltello. La schiena mi
doleva e riuscii a stento a puntellare le braccia sotto di me e a
sollevarmi in ginocchio. I miei capelli troppo neri mi ricaddero sulla
faccia e di colpo compresi l'utilità del taglio più corto adottato da
Crispin. Questo mi fece quasi ridere ma una parte troppo grande della
mia mente era assorbita da quello che stava succedendo. Dove erano
andati tutti? E dov'era la strada adorna di viti che avevo tanto
ammirato pochi momenti prima? Dovevo aver girovagato, imboccando un
ampio vicolo che risaliva con una notevole pendenza la faccia delle
alture e la nostra acropoli. In alto, le torri del Riposo del Diavolo
apparivano ancora come il palazzo elevato di un qualche dio turbolento.

In distanza sentii le campane della Cappellania della città che
cominciavano a scandire il tramonto e le voci dei cantori, amplificate
da enormi altoparlanti posti nei templi e su pali in tutta Meidua, che
intonavano il richiamo alla preghiera.

«Beccalo, Jem!» gridò una voce, e qualcosa di pesante e di metallico
colpì la pietra. Mi girai giusto in tempo per vedere un grosso uomo su
un motociclo dare gas e il primitivo motore a petrolio eruttare veleno
nell'aria. Non ho mai saputo dove si fosse procurato quell'arnese -- se
nel parco macchine di qualche Gilda o in qualche trattativa sotto banco.
In una mano stringeva un pezzo di tubo, e dalla sua postura compresi che
lo aveva usato per colpire il terreno. Quel tubo doveva essere anche ciò
che mi aveva gettato al suolo. Ciò che però catturò la mia attenzione fu
la faccia dell'uomo. La narice sinistra era stata tagliata fino all'osso
per cui rimaneva aperta in modo orribile alla luce dei lampioni, e il
testo sulla sua fronte proclamava il suo crimine in rabbiose lettere
nere: \foreignlanguage{italian}{aggressione}.

«Colpiscilo!» gridò l'altro ragazzo.

Spostai la mia attenzione di lato, dove altri due uomini su una moto
aspettavano all'estremità del viottolo, più avanti, e incitavano l'uomo
con il tubo. Sollevai una mano e mi rialzai barcollando. «Mi arrendo!»
dissi, ricordando le lezioni che mi erano state impartite riguardo a
situazioni come queste, poi premetti senza parere il pulsante di allarme
sul mio terminale da polso. «Mi arrendo.» In modo vago, rammentai
istruzioni che risalivano alla mia infanzia che suggerivano di
arrendersi quando si era circondati e disarmati o inferiori
numericamente, e di sperare in una richiesta di riscatto. Qualsiasi
Casato nobiliare avrebbe onorato una richiesta del genere, come parte
delle regole della \emph{poine}.

Quelli però non erano nobili.

«Fottiti!» disse uno degli uomini alle mie spalle. «Al diavolo la resa.»

«Sei un nobile, vero?» commentò l'altro, passandosi la lingua sui denti
storti. «Con quei bei vestiti e tutto il resto. Scommetto che hai
addosso un sacco di denaro.»

Nessuna risposta che avessi potuto dare avrebbe soddisfatto
quell'avidità. Ebbi solo un secondo per assimilare ciò che mi
{circondava}, poi l'uomo sulla moto diede gas, facendo girare sul posto
la ruota posteriore e sollevando la polvere e le schegge di pietra
smosse della strada, prima di scagliarsi verso di me tirando indietro il
braccio per vibrare un altro colpo. In quel momento quasi tutti i miei
anni di addestramento al combattimento mi vennero meno e barcollai da un
lato sul marciapiede, sperando che il piccolo rialzo del cordolo
rallentasse il mio assalitore. Mentre mi spostavo premetti
l'interruttore di attivazione della cintura-scudo e sentii il secco
ronzio della cortina di energia prendere forma intorno a me. I suoni
circostanti si fecero tutti ovattati, ma intanto mi resi conto che anche
sulla moto il mio aggressore sarebbe stato troppo lento perché lo scudo
servisse a qualcosa. La soglia di velocità per un campo Royse era
superiore a qualsiasi accelerazione un essere umano potesse raggiungere.
Il campo mi avrebbe protetto se uno di quei bastardi aveva un'arma da
fuoco, ma contro quel pezzo di tubo? No.

Il colpo dell'uomo mancò il bersaglio e lui ruotò la moto con una
derapata, ridendo nel girarsi per fronteggiarmi di nuovo. Sarei dovuto
fuggire, ma invece mantenni la posizione ed estrassi la mia mancina. La
lama era lunga soltanto quanto il mio avambraccio, fatta di ceramica che
spiccava di un candore latteo nella luce rossa del tramonto. «Chi vi ha
mandati?» domandai, adottando una posizione difensiva. Assurdamente,
ripensai al combattimento contro l'azhdarch a cui avevo assistito in
precedenza nel pomeriggio, ma mi resi subito conto di quanto poco quella
situazione avesse in comune con la mia. Lo xenobita volante aveva messo
facilmente in stato di inferiorità gli schiavi gladiatori, mentre questo
ricordava di più gli antichi combattimenti contro i tori, che si
praticavano ancora quando al colosseo mandavano mostri più pittoreschi.

E io ero un misero matador, privo perfino di una spada adeguata.

«Chi vi ha mandati?» ripetei, ora più in tono di sfida che
interrogativo. Gli altri due uomini mi si lanciarono contro sui loro
veicoli, uno brandendo un manganello da prefetto e l'altro una mazza di
alluminio come quelle con cui i bambini giocavano a palla. Mi lanciai in
avanti, facendo affidamento sulla strategia di uno spadaccino di
accorciare le distanze per salvarmi dall'attacco, ma funzionò solo in
piccola parte e ben presto mi ritrovai steso per terra supino. `Steso',
questo era il termine. Circondato, rotolai in ginocchio e riguadagnai la
mia posizione mentre premevo di nuovo il pulsante antipanico del mio
terminale. Ormai Roban doveva aver ricevuto l'allarme, come pure Kyra e
le altre guardie. Cercai di immaginare i peltasti che si ammucchiavano
sulla navetta di Kyra, le lance a energia che si aprivano come scrigni
di gemme nella modalità di attacco.

«Prendi i suoi anelli, Zeb!» disse l'uomo con il tubo. «Li vedi?»

\emph{No.} Con un sussulto realizzai che non era un uomo, ma un ragazzo.
I miei assalitori erano tutti bambini, efebi non più vecchi di Crispin,
con il volto chiazzato da piccoli e miseri peli arricciati e butterato
dall'acne. Comuni ratti di strada. Una gang. Ma dove si erano procurati
quelle moto? Di certo cose del genere non potevano costare poco, anche
se non erano controllate dalla Cappellania.

La Cappellania... Le campane infernali del luogo sacro adesso
risuonavano in tutta la città e in alto, al Riposo del Diavolo, Eusebia
si stava preparando per l'elegia serale. Le persone stavano pregando,
oppure erano in adorazione dello spargimento di sangue nel colosseo.
Immaginai Crispin in piedi nel cerchio polveroso del colosseo mentre
petali di rosa piovevano su di lui e sui suoi nemici sconfitti. Da
qualche parte sir Roban si stava dirigendo verso la mia posizione, ma
intorno al mio piccolo nodo di caos il mondo continuava a scorrere
immutato.

Protesi il coltello. «Affrontatemi lealmente!» gridai, da sciocco
ingenuo. Quello non era un duello, non c'era un arbitro, non era uno
scontro uomo contro uomo con armi alla pari e con l'andamento del
combattimento che dipendeva solo dall'abilità.

«Credevo che ti fossi arreso!» commentò uno degli altri ragazzi... forse
Zeb, non ho mai appreso chi fosse chi. I due alle mie spalle si fecero
più vicini, girando in cerchio con la moto in folle. «Probabilmente
questo stronzo vive in uno di quei palazzi con le guglie della Città
Alta, e parla di lealtà.» Il ragazzo sputò. «Sbattilo di nuovo a terra,
Jem. Questo non è il suo territorio.»

«Questa è la mia...» Stavo per dire `città' quando il grosso ragazzo con
il tubo puntò dritto verso di me. Scartai di lato e cercai di ruotare il
braccio con il coltello, ma fui troppo lento. Il tubo mi raggiunse al
braccio, appena sopra il polso, e lasciai cadere l'arma. Ululando,
crollai in ginocchio, consapevole che il polso era {fratturato}. Con
grida di trionfo, gli altri due ragazzi saltarono giù dalla moto.
Stingendomi al petto il braccio rotto, annaspai per recuperare il
coltello che mi era caduto.

«No, nobilotto, questo non si fa!» Qualcuno mi afferrò per il dietro
della giacca e io mi contorsi, colpendo il popolano al mento con la mano
sana. Lo sentii lanciare un grido e snudai i denti con soddisfazione,
con le narici dilatate e il respiro affannoso. Ringhiando, l'altro
ragazzo mi si lanciò contro. Pensando a Crispin -- a come lui avrebbe
combattuto quasi nello stesso modo -- reagii con un calcio secco che lo
raggiunse fra le gambe. Sussultando, incespicò all'indietro e questo mi
diede il tempo sufficiente per arrivare al coltello, che recuperai
proprio mentre il ragazzo con il tubo si univa alla mischia.

Sapevo di non poter vincere. Forse con entrambe le mani avrei potuto
avere la meglio su quei tre criminali delle strade di Meidua. Forse. Se
fossi stato adeguatamente armato con una spada? Di certo. Ma nelle
condizioni in cui ero, con una frattura e vittima di un'imboscata,
munito solo del coltello? Tutto quello che potevo fare era guadagnare
tempo. Fui fortunato perché i tre ragazzi si ostacolavano a vicenda più
di quanto avrebbero fatto, per esempio, tre legionari delle forze
imperiali. Quindi mi raddrizzai, abbandonando il mio lignaggio e la mia
educazione come i trogloditi che abbandonano la civiltà per vivere come
bestie.

Quello con il tubo -- Jem, credo -- attaccò per primo e io sgusciai
all'indietro proprio mentre uno dei suoi amici si portava alle mie
spalle. Cercai di ferirlo ma fui troppo goffo e lento con la mano
sinistra. Ero molte cose, ma mancino non era una di esse. Qualcosa mi
raggiunse alla schiena -- il manganello o la mazza di alluminio -- e mi
strappò il respiro. Barcollai fino a cadere e uno stivale mi colpì con
violenza nelle costole. Persi la poca aria che mi era rimasta e
annaspai, cercando di rialzarmi. Un altro piede mi calò sul polso sano,
non con tanta forza da spezzarlo ma con un impatto sufficiente a farmi
sfuggire il coltello. Persi conoscenza per un momento. Dovevano avermi
sferrato un calcio alla testa. Qualcosa mi calò di nuovo sulla schiena,
ma solo con l'immediatezza di un lontano sparo di cannone. Credo che il
mio spirito cercasse di risollevarsi mentre il mio corpo crollava al
suolo e nell'oscurità. In modo vago recuperai la consapevolezza quanto
bastava per {sentire} una voce che sibilava: «Questo dovrebbe insegnarti
a non venire quaggiù come se fossimo una tua proprietà.»

«Prendigli gli anelli, Zeb!»

«Ha anche un terminale! Prendilo!»

Alcune mani mi sfilarono dal pollice sinistro l'anello con il sigillo e
cominciarono ad armeggiare con il fermaglio magnetico del mio terminale.
Poi lo sentii. «Ragazzi, siamo fottuti. Guardate.» Sorrisi, anche se
avevo la faccia contro la pavimentazione. Sapevo che stava mostrando
l'anello agli altri. «È un fottuto Marlowe. Siamo fottuti.»

Avrei voluto sorridere ma le mie labbra non reagirono. L'anello di un
nobile è tutto. Racchiude la sua identità: la sua storia genetica, tanto
della sua famiglia quanto della sua costellazione; i suoi titoli; e i
titoli delle sue tenute personali. Se lo avessero preso e avessero
cercato di usarlo da qualsiasi parte su Delos gli uomini di mio padre o
di mia nonna li avrebbero trovati.

Stolti.

Non ricordo nient'altro, tranne l'oscurità e l'assoluta certezza di
essere morto. Crispin avrebbe governato, adesso non c'erano dubbi al
riguardo. Avrebbe avuto il suo trono, il suo posto nell'Impero. Che mio
padre finisse per rimpiangere la sua scelta. Non mi importava.

Certi scoliasti insegnano che ciascuna esperienza è solo la somma delle
sue parti, che la nostra vita può essere ridotta a una serie di
equazioni che possono essere calcolate, soppesate, bilanciate e
comprese. Credono che l'universo sia un oggetto e che noi siamo solo
oggetti fra gli oggetti, che perfino le nostre emozioni non siano altro
che processi elettrochimici eseguiti nel nostro cervello, accessori alle
pressioni dell'Evoluzione dalle Mani Insanguinate. È per questo che
lottano per arrivare all'apatia, alla libertà dalle emozioni. Questo è
il loro grande errore. Gli esseri umani non abitano un mondo fatto di
oggetti, la nostra consapevolezza non si evolve per vivere in un posto
del genere.

Noi viviamo nelle storie e in esse siamo soggetti a fenomeni che vanno
al di là dei meccanismi dello spazio e del tempo. Paura e amore, morte e
ira e saggezza... queste cose sono parte del nostro universo quanto lo
sono la luce e la forza di gravità. Gli antichi le definivano dèi,
perché siamo le loro creature, modellate dai loro venti. Filtrate le
sabbie di ogni mondo e vagliate la polvere dello spazio fra di essi, e
non troverete un atomo di paura, un grammo di amore o una dramma di
odio, eppure essi sono là, invisibili e incerti come i più piccoli
quanti e altrettanto reali. E come i più piccoli quanti sono governati
da principi che esulano dal nostro controllo.

Ma qual è la risposta a questo caos?

Costruiamo un Impero più grande di qualsiasi altro nell'universo
conosciuto. Mettiamo ordine in quell'universo, modellandone la natura
esteriore sulla base della legge interiore. Diciamo che il nostro
imperatore è un dio affinché ci possa tenere al sicuro e controllare il
caos della natura. La civiltà è una sorta di preghiera, in cui speriamo
che con l'azione giusta possiamo realizzare la pace e la quiete che sono
l'ardente desiderio di ogni cuore decente. La natura però oppone
resistenza perché perfino nel cuore di una città grande come Meidua, su
un mondo civilizzato come Delos, un giovane può semplicemente svoltare
nella strada sbagliata ed essere assalito dai briganti. Nessuna
preghiera è perfetta, come non lo è nessuna città.

Di colpo fece molto, molto freddo.

