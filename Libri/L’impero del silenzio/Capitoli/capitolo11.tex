\chapter{A quale prezzo}

Adesso penso che il vecchio furfante voleva che lo facessi, che perfino
durante la nostra piccola conversazione sulla spiaggia avesse continuato
a spingermi e a pungolarmi affinché arrivassi in modo indipendente alle
mie convinzioni. In modo del tutto privato mi concentrai sulla fuga,
avendo solo una vaga idea di come realizzare una cosa del genere. Io,
che non ero mai stato neppure una volta fuori dal sistema, sognai modi
in cui noleggiare o rubare un'astronave e pilotarla da qualche parte che
non fosse il college Lorica di Vesperad. Presi in considerazione la
possibilità di corrompere l'equipaggio della nave che mio padre aveva
noleggiato per il viaggio o di sgusciare via durante una qualche fermata
intermedia del lungo tragitto per poi dirigermi verso i territori.
Sapevo che era stato fatto, ma la logistica della cosa esulava dalla mia
limitata esperienza.

Per come la vedevo, avevo due problemi essenziali: capire come lasciare
il pianeta e come pagare per poterlo fare. Perversamente, il secondo
problema era più facile da risolvere del primo. Dopotutto, ero il figlio
del lord del palazzo e avevo accesso a certe fonti di ricchezza che i
popolani non potevano neppure immaginare. Forse starai pensando a
scrigni di pietre preziose e di diademi d'oro, ma se pure l'oro mantiene
abbastanza valore a causa della sua scarsità come pure per la miriade
delle sue funzioni pratiche, d'altronde è una cosa abbastanza comune e
le monete coniate dall'Impero -- hurasam d'oro, kaspum d'argento e il
resto -- circolano prevalentemente fra gli strati inferiori della nostra
civiltà. Le pietre preziose, che nella maggior parte dei casi sono poco
più che carbonio, non hanno valore presso i circoli dell'élite da prima
dell'ascesa dell'Impero. Diamanti, zaffiri, rubini e tutto il resto
potrebbero essere ottenuti per poco da chiunque potesse entrare in
contatto con un alchimista.

Invece, le fortune della casta palatina sono supportate dalla
complessiva ricchezza chimica dell'Impero. L'oro è uno di questi
materiali degni di essere smerciati. L'uranio è un altro, molto più
prezioso soprattutto perché per poterlo estrarre legalmente è necessaria
una licenza da parte dell'ufficio imperiale. Di conseguenza, mentre un
singolo hurasam è disponibile per chiunque, il marco imperiale -- che
nominalmente è la valuta standard dell'Impero -- è qualcosa di
accessibile soltanto per coloro la cui occupazione li eleva al di sopra
della sporcizia e del grasso di motore alla base della nostra società.

I marchi valgono molto di più, in un cambio uno a uno, e sono assai più
facili da spostare di un carico d'oro, in quanto sono soltanto dati su
un acconto. Il trucco consisteva nello spostarli senza essere notato. I
logoteti e i segretari di mio padre presso i diversi ministeri -- per
non parlare della tesoreria del Casato -- avevano già fin troppe cose da
tenere sotto controllo, ma c'era sempre la possibilità che un impiegato
troppo zelante esaminasse con eccessiva attenzione la mia diaria e i
vari conti bancari d'emergenza a mio nome. Ed esisteva anche la
possibilità -- per quanto scarsa -- che mio padre mi tenesse d'occhio in
modo particolare.

Tre mesi.

Quanto è breve in realtà quel lasso di tempo anche se i mesi di Delos
sono più lunghi di quelli standard, come lo sono anche i nostri giorni.
Perfino per un palatino -- forse soprattutto per un palatino -- i giorni
passano in fretta. Io dovevo agire altrettanto in fretta e optare per la
sola linea di azione a cui sapevo che il lord mio padre -- nonostante
tutta la sua vantata freddezza -- non avrebbe mai potuto trovare da
obiettare.

La beneficenza.

\begin{figure}
	\centering
	\def\svgwidth{\columnwidth}
	\scalebox{0.2}{\input{divisore.pdf_tex}}
\end{figure}

«Cosa vuoi fare?» La capofazione della Gilda dava l'impressione che
l'avessi appena schiaffeggiata, con gli infossati occhi color fango
sgranati nel volto che cominciava prematuramente a invecchiare.

Con calma, ripetei la mia offerta, in piedi dall'altro lato della
scrivania ingombra, cercando di non pensare a Kyra e alle altre due
guardie in piedi appena fuori dalla porta dell'ufficio, come se pensarci
avesse potuto attirare la loro attenzione su di me e su quello che
facevo. «Voglio fare una donazione alla Gilda. Dai miei conti
personali.»

Il volto comune di Lena Balem si contrasse in un'espressione sospettosa.
«Perché?»

Non riuscendo a incontrare il suo sguardo, fissai l'ologramma a parete
alle sue spalle che mostrava una vista dall'alto della regione della
Valle del Denterosso in forma tridimensionale. I siti minerari erano
contrassegnati con glifi gialli indicanti radioattività e le regioni
avevano tonalità diverse a seconda dei livelli di rischio. Ero stato
laggiù molte volte, e nonostante gli sforzi dei biologi soltanto le
piante più resistenti riuscivano a gettare radici nella zona collinare
sovrastante il fiume. Il parere comune era che una qualche enorme
collisione nelle profondità del passato geologico avesse seppellito i
depositi di uranio della regione, ora esposti nuovamente in virtù dei
sottili cataclismi dovuti alla nostra remota opera di terraformazione.

«Hai sentito che sto per lasciare Delos?» chiesi infine.

Presa alla sprovvista, la capofazione della Gilda si protese in avanti
con i gomiti sul bordo della sua scrivania economica. «Allora è vero? Ne
parlavano sul notiziario diurno, ma credevo...»

Scossi il capo. «È vero. Devo partire a bordo della \emph{Lavoratore
	Lontano} il 33 di Boedromion, ma alla luce di tutto quello che è
successo nelle ultime settimane, io... uh...» A questo punto riuscii a
guardarla in faccia di nuovo, consapevole che quello era l'opposto di
ciò che avrebbe fatto mio padre. «Non mi piaceva come avevo lasciato le
cose, qui. Mi è dato di capire che quando erano qui, i membri del
Consorzio sono stati in grado di sopperire ad alcune delle vostre
necessità, giusto?»

Lei sbuffò. «Una macchina cingolata per la raffinazione e un paio di
trivelle. Bilancerà alcune delle nostre perdite, ma abbiamo ancora in
quelle gallerie persone che usano equipaggiamento manuale.» Distratta, o
forse cercando di focalizzarsi, si protese a riordinare un assortimento
di documenti sulla sua scrivania. «Devo chiedertelo, lord Marlowe.
Perché questo improvviso interesse nelle nostre operazioni?»

Allargai le mani con aria di assoluta innocenza. «Voglio solo correggere
un errore che ho fatto.» Attesi un paio di secondi prima di aggiungere,
come per un ripensamento: «E comunque dove andrò non avrò bisogno di
denaro. Mio padre mi ha venduto alla Cappellania.» Prima che lei avesse
il tempo di pensare ai sottintesi di quelle parole, continuai: «Di
conseguenza voglio fare una donazione di centoventimila marchi.»

I suoi occhi si dilatarono fino a sembrare piatti di portata. «Dici sul
serio?» Se la sua mandibola si fosse staccata come quella del povero
Yorik e fosse caduta sulla scrivania la cosa non mi avrebbe sorpreso.
Perfetto... era esattamente la reazione che volevo.

«Con quei soldi potreste equipaggiare una dozzina di squadre di lavoro
con tute di sicurezza, vero? Nuove tute, con la schermatura a elettroni
e tutto il resto.» Spinsi indietro la manica della giacca per
controllare l'ora sul terminale.

Lena Balem infilò una mano sotto la scrivania e tirò fuori un pacchetto
di sigarette senza tabacco. Esitò un momento, come per chiedere il mio
permesso, poi si portò alle labbra una sigaretta la cui estremità si
fece di un rosso ciliegia quando l'accese e soffiò un po' di fumo fra
noi due. «Potremmo, ma questo non risponde ancora alla mia domanda.»

«Quale domanda, capofazione?»

«Perché stai facendo questo?»

«Te l'ho detto» ribattei con finta esasperazione, accelerando in
direzione della verità. «Non voglio avere sulla coscienza la morte di
quegli uomini, e se mio padre non vuole pagare per l'equipaggiamento,
allora lo farò io.» Inclinai la testa verso la scrivania, come a
indicare la formalità di un documento ufficiale. «Stila un contratto, se
non vuoi accettare la mia parola. Anzi, insisto perché tu metta la cosa
per iscritto.» Altro fumo annebbiò l'aria e cercai di ripulirla --
tossendo -- agitando la mano. Sapevo a che gioco stava giocando,
cercando di darmi fastidio, ma sorrisi ed espirai con decisione. Il
tabacco geneticamente modificato non avrebbe lasciato depositi nei
polmoni ma aveva un odore orribile. Avrei dovuto dirle di non accendere.
Forse ero troppo debole.

Lei frugò sulla scrivania fino a localizzare un faldone rilegato in
finta pelle, lo aprì e rivelò un tablet a cristalli e uno stilo dalla
punta di gomma. «Ecco» disse, tenendo la sigaretta fra i denti gialli.
Per un momento il silenzio incombette su di noi, tranne per il rumore
del traffico a terra nella strada antistante le vetrine degli uffici
della Gilda.

Ora veniva la parte delicata.

Presi il tablet e compilai il semplice modulo con facilità, colpendo lo
schermo con lo stilo mentre esso convertiva la mia scrittura in ordinati
caratteri galstani, poi ripetei il processo in una nuova finestra. Avevo
quasi finito quando feci una pausa, consapevole che era arrivato il
momento, e posai il tablet sul tavolo. «Sai, M Balem, mi è venuto in
mente che ci potremmo aiutare a vicenda.» Le rivolsi il mio miglior
sorriso non da Marlowe.

Il suo volto plebeo dal mento sfuggente si incupì. «Come sarebbe?»

Mantenni il mio sorriso più cortese. «Convieni con me che centoventimila
marchi è una somma... adeguata, sì?» Lei annuì una volta, lentamente e
senza parlare, con l'aria di qualcuno che stesse aspettando di vedere un
mago eudoriano completare il suo gioco di magia. «Che ne diresti di
centotrenta?»

Assurdamente, il mio cuore ribelle prese a battere più in fretta contro
le costole ancora doloranti. Avevo parlato a bassa voce ed ero certo che
le guardie non mi avessero sentito dal corridoio, non attraverso la
semplice porta a scorrimento in acciaio. Perché dovevo avere paura? Qui
ero io ad avere tutto il potere. Avevo i soldi, il nome, e la
capofazione della Gilda aveva... cosa? I mezzi per denunciarmi? Ma
questo sarebbe servito soltanto a coinvolgerla, se avesse accettato. E
lo avrebbe fatto, lo sapevo, e con quella consapevolezza le feci la mia
offerta. «Firmerò questo contratto per una cifra di
\emph{centocinquantamila} marchi se...» A quel punto passai due volte la
mano sul tablet in modo da trasferire i due documenti sull'olografia a
parete. «Se tu firmerai questo contratto parallelo per centotrentamila
marchi che terrò sulla mia persona. Non registrati.» Vidi la confusione
nei suoi occhi e la incalzai. «Voglio che tu mi dia la differenza su una
carta universale oppure -- meglio ancora -- in hurasam, se li hai.»

«Sai di quanti hurasam si tratta?» Belem suonava incredula. «Hai una
paletta di sollevamento?»

«Su una carta, allora.»

«Mi stai chiedendo di riciclare del denaro.»

«No» insistetti, sperando di riuscire a continuare ad attenermi al piano
che avevo nella mente. «Ti sto chiedendo di... di sentirti in colpa per
l'enorme somma di denaro che ti sto regalando e di restituirmene sotto
banco una piccola parte per placare la tua coscienza.» Sorrisi, solo che
questa volta si trattava del sogghigno in tralice dei Marlowe. Con cura,
mi rimossi dal pollice sinistro l'anello con il sigillo, tenendolo
pronto per sigillare i due contratti e trasmettere così i terabyte della
chiave crittografica formale. Pensai a tutto ciò che quell'anello
significava: il mio nome, il mio sangue, la mia storia genetica, la mia
personale proprietà di ventiseimila ettari di terra sui monti del
Denterosso.

Balem spostò lo sguardo dal mio volto ai contratti sull'olografia a
parete, poi guardò verso la porta. Potevo vedere la cupidigia che le
accendeva gli occhi color fango. La sigaretta le si stava consumando fra
le dita, momentaneamente dimenticata. «E se rifiutassi?»

Dovevo spiegarle proprio tutto? «È a questo che serve l'altro contratto.
Lo presento alla tesoreria e dico che qui qualcuno deve aver violato i
file dei contratti e modificato la somma. A chi pensi che crederebbero?
Mio padre ce l'ha già con te per quel pasticcio con il Consorzio.» Vidi
la sua carnagione rozza farsi più pallida di qualche tono.
«Naturalmente, sei libera di rifiutare la mia offerta.»

Lei snudò i denti con gli occhi che ardevano di disprezzo. «Qui non si è
mai trattato di beneficenza.»

Reagii con un triste sorriso, di nuovo sincero. «Vi voglio aiutare, M
Balem. Che tu ci creda o meno non ha importanza, ma anche tu devi
aiutare me. Queste sono le mie condizioni.» Sollevai l'anello, pronto ad
applicarlo ai due documenti. «Vogliamo procedere?»

\begin{figure}
	\centering
	\def\svgwidth{\columnwidth}
	\scalebox{0.2}{\input{divisore.pdf_tex}}
\end{figure}

Con ventimila marchi su una carta universale numerata riposta nella
tasca interna della giacca e i dati tanto del contratto pubblico quanto
di quello che consideravo la mia polizza di assicurazione immagazzinati
nella matrice all'interno del mio anello, sedevo sul sedile posteriore
del velivolo che ci stava riportando al Riposo del Diavolo. Quel giorno
la vecchia fortezza sembrava una nube temporalesca al di sopra della
città, che era a sua volta sovrastata da un cielo coperto che minacciava
tempeste estive.

«È stato gentile da parte tua fare quella donazione ai minatori»
commentò Kyra, da sopra la spalla.

Venendo proprio da lei, quell'affermazione mi riempì di vergogna.
Dopotutto non lo avevo fatto per i minatori, giusto? Mi sentii di colpo
la lingua spessa e distolsi il volto. «Grazie.» Dovevo dirle qualcosa
prima di andarmene? Dirle che era bella? Forte? Serrai le mani a pugno
in grembo, con la destra che mi faceva un male terribile a causa delle
ossa doloranti. Mi inflissi comunque quel dolore, sentendo che in
qualche modo lo meritavo. Una volta avevo letto che i preti di questa o
quella religione si fustigavano con corde cosparse di nodi affinché il
dolore potesse redimere i loro peccati. Non ho riscontrato che sia così,
ma solo che il dolore molto spesso dà la sensazione di essere giustizia.

«Tenente» dissi in tono sommesso.

«Sire?»

«Potresti cambiare rotta, per favore? Portaci all'attico in città.»

Una delle mie due guardie protestò. «Sire, pensi davvero che dovresti
andare in città, dopo l'ultima volta?»

Era a metà della frase quando mi girai a guardarla, lieto forse per la
prima volta nella mia vita di avere gli stessi occhi di mio padre, e
sovrastai la sua voce. «Soldato, ti ricordo che sono il figlio del tuo
arconte.» Dentro di me c'era un improvviso veleno derivante dal mio
nuovo senso di vergogna. «Apprezzo la tua preoccupazione, ma
consideriamo chiusi i danni prodotti da quella faccenda, d'accordo?»
Riportai la mia attenzione in avanti. «All'attico, Kyra, per favore.»
Non volevo tornare al castello, non quel giorno.

\begin{figure}
	\centering
	\def\svgwidth{\columnwidth}
	\scalebox{0.2}{\input{divisore.pdf_tex}}
\end{figure}

Adesso dovevo considerare l'altro problema, che era di gran lunga il più
complicato. In un certo senso avevo meno diritto di viaggiare del più
insignificante dei plebei. Qualsiasi comune lavorante dei moli o tecnico
di fattoria urbana non vincolato al pianeta dal suo sangue poteva
guadagnare di che pagarsi un passaggio per lasciare Delos, oppure si
poteva arruolare nelle Legioni... dopotutto, c'era una guerra in
corso... ma io... io venivo esaminato, sorvegliato, protetto. Almeno
quando non mi facevo pestare quasi a morte da una gang di motociclisti
nelle strade di Meidua. Tuttavia, quel particolare episodio mi infondeva
una certa dose di sicurezza. Mi ero già sottratto una volta alle mie
attente sentinelle, giusto?

Potevo farlo di nuovo.

Il sole stava tramontando, con il giallo che si tingeva d'oro sopra le
montagne occidentali, e sotto e intorno a me le luci cominciavano ad
accendersi nella città di Meidua, le code serpentine di traffico di
veicoli terrestri accendevano lentamente le luci di marcia settandole in
modalità notturna. Un pannello olografico più grande di una casa
cominciò a risplendere sulla torre di fronte a me, dapprima
pubblicizzando i Diavoli di Meidua -- i gladiatori -- e poi mostrando
una pubblicità di reclutamento in cui figurava una donna dalla mascella
marcata nell'armatura bianca delle Legioni imperiali. Mi appoggiai
pesantemente contro una balaustra di pietra intagliata, accasciandomi
contro di essa. Il mio senso di colpa per aver ricattato la capofazione
stava già sbiadendo, e una parte di me era ebbra del mio successo. Avevo
in mio possesso ventimila marchi di cui mio padre ignorava totalmente
l'esistenza, e quando o se i logoteti e i banchieri del Casato avessero
controllato i miei conti avrebbero visto solo la mia generosa donazione
al capitolo di Meidua della Gilda Mineraria di Delos.

Chi poteva trovare da ridire su una simile compassione civica? Dopotutto
sarei diventato un prete.

Risi debolmente nella piega del braccio, sperando che Kyra e le mie
guardie non mi vedessero. Le mie azioni di quel giorno mi avevano
lasciato addosso l'intenso bisogno di rimanere indisturbato e
inosservato, solo con i miei pensieri. Temevo di abbandonare la mia casa
ancestrale nella stessa misura in cui desideravo andarmene, e le stelle
che fiorivano nel buio sempre più intenso premevano sulla mia mente,
incombenti come il Riposo del Diavolo non era mai stato.

Gli antichi avevano un detto, una maledizione, che recitava: `Possa tu
vivere in tempi interessanti.' Supponevo di farlo. Con il sole che
tramontava e il nero dello spazio che in alto diventava visibile, mi
parve in qualche modo che i Cielcin si facessero più vicini ed ebbi
quasi la sensazione di poter vedere le loro navi che scendevano come
castelli nell'aria notturna, anche se non ne avevo mai vista una nella
vita reale. Nella mia mente, le loro torri si stendevano come dita di
mani gelide, delicate strutture rivestite di ghiaccio che scintillavano
come il palazzo di cupi esseri fatati. Era una visione o un sogno a
occhi aperti? Oppure era il futuro che esercitava pressione sul mio
presente? Forse era soltanto un pugno serrato intorno alla mia anima, il
dolente terrore dovuto al fatto che stavo per lasciare la mia casa.

Fu solo allora che la realizzazione mi investì. La dichiarazione di mio
padre che avrei dovuto entrare nel clero non mi era mai parsa
\emph{reale} perché l'avevo così convenientemente respinta, ma con la
plastica della carta universale nella tasca della giacca e il sapore del
ricatto ancora amaro in bocca, il pensiero che stavo per lasciare quella
miserabile città -- la sola casa che avessi mai conosciuto -- mi investì
violento e improvviso come quei delinquenti plebei sulle loro moto.

«Mio signore?»

«Tenente?» Sussultai e mi ritrassi dalla ringhiera. Kyra era ferma
appena fuori dalla porta dell'attico, come si conveniva, con le mani
congiunte davanti a sé e lo sguardo basso. «C'è qualcosa che non va?» Le
sorrisi, e scoprii che il suo era un volto che potevo continuare a
fissare senza dover distogliere lo sguardo come mi succedeva con mio
padre. I suoi capelli del colore del bronzo battuto si arricciavano come
una visione petrarchiana intorno al volto a forma di cuore e le curve
del corpo erano improntate a snellezza.

«Sono solo venuta ad avvertirti che l'appartamento è chiuso a chiave per
la notte. Nessuno può entrare o uscire senza attivare il sistema di
allarme.»

«Cosa?» Le sue parole impiegarono un momento a penetrare le nuvole
generate nel mio cervello dalla sua bellezza e dalla mia vergogna. «Oh,
certo... molto bene. Kyra.» Sorrisi di nuovo, più debolmente, questa
volta. «Di' agli opliti che si possono ritirare, o possono montare la
guardia a turno, se preferiscono.»

«Signore?»

«La cosa dei turni... di' loro di farla» aggiunsi, agitando un dito.

La tenente si premette un pugno sul petto in un saluto. «Certamente,
sire.» Si girò per andarsene.

«Kyra.» Si fermò con le spalle tese, anche se il significato della cosa
mi sfuggì.

«Mio signore?» Il suo tono era sommesso. Poi trovò un momento di
audacia. «Perché lo hai fatto?»

«Fatto cosa?» Sbattei le palpebre, sinceramente sconcertato.

«Mi hai chiamata per nome» spiegò, sempre di spalle. «Non è
appropriato.»

La voce di mio padre echeggiò dentro di me. `Non è appropriato, mio
signore.' La misi a tacere, non sapendo come rispondere. La cautela ebbe
però la meglio sul desiderio e replicai in modo goffo: «Volevo...
sentirmi vicino a qualcuno, tutto qui.» A quel punto lei si girò e nei
suoi occhi verdi scorsi la comprensione; comprensione e... paura? Di
certo non era possibile. «Mi dispiace se ti ho offesa.»

«Sono la tua serva, sire, non ti devi scusare.» Lei scosse con forza il
capo, poi serrò gli occhi e chiese: «Ma... perché io?»

«Prego?» Un velivolo passò sopra di noi a bassa quota, facendo
scintillare gli scudi dell'attico in modo tale che l'aria tremolò per le
linee di forza disturbate. Per un momento mi girai per seguirlo con lo
sguardo, con le sue luci di posizione che ammiccavano rosse e verdi nel
crepuscolo.

La tenente si erse un po' più dritta e protese in fuori il mento
appuntito. «Da settimane mi ordini di portarti in giro per le tue
commissioni, già da prima dell'incidente.»

L'incidente, pensai con amarezza, ordinando al mio volto di rimanere
immobile mentre ricordavo l'aggressione.

Kyra però non aveva finito, e ripeté: «Perché io?» Il volto era ancora
abbassato, gli occhi erano chiusi. Attraversai lo spazio infinito che ci
separava e mi allungai a stringere la sua piccola mano callosa. Lei si
tese come una molla pronta a scattare e credo smise di respirare. Più
coraggioso di quanto lo fossi mai stato, in mancanza delle parole per
spiegarmi la baciai.

Lei si raggelò.

Le strinsi la mano in quello che speravo essere un gesto di conforto.
Ero poco più di un bambino e in realtà non sapevo cosa fare. Dicono che
in momenti come questi il tempo si fermi, ma in realtà è solo il tuo
respiro a farlo.

E Kyra, che era immobile come la pietra.

Mi ritrassi, imbarazzato e timoroso. «Mi dispiace, non avrei dovuto
farlo. Io...» Lei si premette la mano libera contro il petto, con
l'altra ancora stretta nella mia. La lasciai andare e mi trassi
indietro. Il suo volto abbronzato aveva perso tutto il suo colore. Di
nuovo formale, distolsi lo sguardo, continuando a farfugliare: «Tenente,
io...»

Con una voce arida e spenta che confermò i suoi peggiori timori sul mio
conto, lei mi interruppe: «Se il mio signore lo desidera io... io
potrei... dividere il suo...»

Non la sentii mai dire `letto' perché gridai: «No!» Non così,
dannazione, non in questo modo. Poi anch'io mi immobilizzai, rendendomi
conto che non ci sarebbe potuto essere nessun altro modo. Io ero un
palatino, il figlio di un arconte, destinato a diventare un priore della
Cappellania della Terra. Come poteva lei, una tenente della guardia di
palazzo, oppormi un rifiuto? Mi sentii nauseato, volgare. Un vigliacco.
La oltrepassai ed entrai nell'appartamento senza aggiungere altro perché
non mi fidavo di parlare.

