\chapter{Il Consorzio}

Quando giunse il giorno dell'arrivo del Consorzio il castello non poté
più nascondere i segni dei preparativi. Wong-Hopper, la Yamato
Interstellar, la Rotshbanc e l'Unione dei liberi mercanti... quelle
istituzioni trascendevano i confini dell'Impero e tenevano insieme
l'universo umano. Perfino nel lontano Jadd, satrapi e principi si
piegavano alle esigenze dell'industria e nonostante la sua grandezza mio
padre era solo un piccolo nobile. Ogni pietra e mattonella del suo
castello nero che chiamavo casa venne preparata, ogni uniforme di ogni
servitore e peltasta delle guardie del Casato si presentò immacolata.
Erano stati fatti tutti i preparativi possibili: i giardini erano stati
potati, gli arazzi sbattuti per rimuovere la polvere, i pavimenti
incerati, i soldati si erano esercitati, le suite per gli ospiti erano
state predisposte. Il segnale più rivelatore era che io ero stato
bandito dalla scena.

«Semplicemente non abbiamo le apparecchiature, Vostra signoria» affermò
la rappresentante della Gilda Mineraria. Lena Balem appiattì le mani
contro il piano della scrivania, con le unghie color vino che
scintillavano nella rossiccia luce sovrastante. «La raffineria di Punta
Denterosso ha un notevole bisogno di riparazioni, e senza una maggiore
attenzione al contenimento è probabile che le morti fra i lavoranti
superino il cinque percento entro la fine del periodo standard.» Avendo
visto il suo schedario sapevo che aveva più o meno il doppio della mia
età, un po' più di quaranta anni standard. Appariva così vecchia! Il suo
sangue plebeo -- non manipolato dall'Alto Collegio -- la tradiva con il
grigio nei capelli dorati, le pieghe agli angoli della bocca e degli
occhi e i cedimenti della pelle lungo la mandibola. Il tempo stava
davvero esigendo da lei un duro pedaggio, anche se era soltanto una
bambina se paragonata ai secoli che io mi aspettavo di vivere. Dovetti
fissarla, o forse rimasi troppo a lungo in silenzio, perché di colpo si
interruppe, dicendo: «Mi dispiace, ma credevo che avrei esposto la
questione al tuo signor padre.»

Scossi il capo, lanciando una rapida occhiata nello specchio sopra e
dietro la sua scrivania, osservando i peltasti in armatura nera che mi
aspettavano vicino alla porta di metallo grigio, appoggiati all'asta
delle lance a energia che erano più alte di loro. Quella loro silenziosa
presenza mi indusse a riflettere, e fu per me uno sforzo anche solo
mantenere sul volto il mio sorriso in tralice. «Mio padre è stato
trattenuto da impegni improrogabili, M Balem, ma sarò lieto di ascoltare
le tue preoccupazioni. Se invece preferisci aspettare, posso riferire a
lui direttamente i tuoi problemi.»

La rappresentante della Gilda socchiuse gli occhi castani. «Questo non
basta.»

«Prego?»

«Ci serve del denaro per rimpiazzare alcune di queste macchine!» Lena
Balem calò un pugno sul tavolo, sparpagliando un groviglio di note di
debito di stoccaggio, una delle quali cadde dalla scrivania e finì ai
miei piedi. Senza che me lo chiedesse, mi chinai a raccoglierla per lei
e quello fu un errore, perché non era cosa che qualcuno del mio rango
dovesse fare. Immaginai il tono di pallore che il volto di mio padre
avrebbe assunto nel vedere suo figlio aiutare una plebea. Senza
commentare il mio gesto, Lena Balem si protese sulla scrivania per
fronteggiarmi. «Alcune delle tute antiradiazioni per i nostri minatori
hanno venti o venticinque anni, non sono adeguate a proteggerli, M
Marlowe.»

Senza bisogno di incitamento una delle guardie avanzò di mezzo passo
nella stanza, alle mie spalle. «Si rivolgerà al figlio dell'arconte
chiamandolo `sire', o `mio signore'.» La sua voce era appiattita dalla
visiera dell'elmo dotato di corna e suonava vaga e impersonale nel
proferire la sua minaccia.

Il volto prematuramente invecchiato di Balem impallidì quando lei si
rese conto del suo errore e io avvertii il forte impulso di zittire il
soldato con un cenno. Nel profondo sapevo però che aveva {ragione} e che
mio padre avrebbe fatto percuotere la rappresentante mineraria per
quell'offesa, ma io non ero mio padre. «Capisco le tue preoccupazioni, M
Balem,» replicai, soppesando le parole e concentrando lo sguardo su un
punto al di sopra delle spalle accasciate della donna «ma la tua
organizzazione ha i suoi mandati e noi richiediamo risultati.» Mio padre
era stato preciso nel descrivere quello che mi era permesso dire in
quell'incontro, cosa era accettabile ordinare a quella donna
aspettandosi obbedienza. Avevo già detto tutto.

«Sire, il vostro Casato ha mantenuto le quote allo stesso livello per
gli ultimi duecento anni senza fare niente per coprire le perdite subite
dal nostro equipaggiamento. Combattiamo una battaglia persa, e quanto
più uranio estraiamo dalle terre alte, più in profondità è inevitabile
che si debba scendere. Abbiamo perso un'intera trivella per i crolli
lungo il fiume.»

«E quanti lavoranti?»

«Prego?»

Posai con estrema precisione sull'orlo della sua scrivania di finto
legno la nota di debito che avevo recuperato, con l'intestazione rivolta
verso l'alto. «Quanti lavoranti avete perso nel crollo?»

«Diciassette.»

«Ti faccio le mie più sentite condoglianze.» Un bagliore di sorpresa
apparve negli occhi della popolana, come se l'ultima cosa che si fosse
aspettata da me fosse stata una qualsiasi parvenza di gentilezza umana,
per quanto vacua e priva di significato. Sentivo comunque che dovevo
provarci. Quella era una tragedia, non una statistica, e la donna che
avevo davanti aveva perso delle persone. La sorpresa la lasciò a bocca
aperta per un momento.

Poi lo stupore scomparve. «A cosa servono le tue condoglianze alle
famiglie di quelle persone? Dovete fare qualcosa in proposito!» Alle mie
spalle sentii il peltasta che aveva parlato prima spostarsi in avanti e
lo bloccai con un gesto che passò inosservato a Belem mentre continuava:
«Non si tratta soltanto di incidenti, mio signore. Queste macchine sono
vecchie... alcune di esse quanto mio nonno, che la Terra lo accolga. E
non si tratta neppure solo delle trivelle cingolate, ma anche delle
raffinerie, come ho detto, e delle chiatte che usiamo per trasferire a
valle lungo il fiume il materiale giallo. Ogni parte di questa
operazione è sul punto di guastarsi e di andare in pezzi.»

«Mio padre ama il suo margine di profitto.» Il patos e l'amarezza che mi
trasparivano dalla voce mi sorpresero. «Devi però capire che attualmente
io non ho il potere di offrire un risarcimento.»

«Allora si deve trovare il denaro per sostituire almeno una parte di
questi macchinari, mio signore.» Balem allungò una mano sulla scrivania
e trascinò un piccolo blocco sopra un mucchio di fogli. «Così come
stanno le cose, abbiamo uomini e donne che lavorano in quelle gallerie
con picconi e pale, in turni di tredici ore.» La sua voce salì di tono.
«Hai idea di quante persone ci vogliono per eguagliare la produzione di
quelle macchine?»

Sentii il mio sorriso che si incrinava quando Balem si rese conto di
aver appena alzato la voce con un nobile. Immaginai Crispin che ordinava
alle sue guardie di percuoterla e invece serrai la mascella. Io non ero
Crispin, o mio padre. «M Balem, quelle macchine vengono prodotte fuori
dal pianeta.» Non sapevo bene dove. «Con i Cielcin che tormentano le
colonie del Velo, il commercio interstellare si è fatto più costoso ed è
molto difficile...»

«Ci deve essere \emph{qualcosa} da fare» mi interruppe, rigirando il
cubo fra le mani. Nel fissarlo, mi resi conto che era soltanto un
fermacarte. Per un momento avevo pensato che fosse un cristallo di
stoccaggio, del genere usato per contenere giochi di simulazione e
ambienti virtuali, ma alle classi inferiori non era permesso possedere
cose del genere. Erano proibite loro perfino le conoscenze tecniche per
rimpiazzare i loro malconci macchinari minerari, e i mezzi di produzione
erano lasciati interamente nelle mani dei Casati nobiliari e della
manciata di artigiani manifatturieri che lavoravano per loro. L'alta
tecnologia, perfino i congegni di intrattenimento come i giochi di
simulazione, erano appannaggio dell'élite. Quello era un fermacarte, e
niente di più.

«Molto probabilmente c'è.» Usando un tono di voce morbido, distolsi lo
sguardo da quello ferreo di lei.

Prima che potessi continuare, Lena Balem mi prevenne. «E le attuali
miniere dureranno ancora solo per qualche tempo, mio signore. Senza
quelle trivelle non abbiamo modo di aprire nuove gallerie, a meno che
tuo padre non voglia che usiamo le nostre mani.»

Potrebbe volerlo, pensai, deglutendo a vuoto. «Lo capisco, M Balem.»
Trassi un altro respiro.

«Allora perché non si fa niente per risolvere il problema?» La sua voce
tornò a salire di volume. Stavo perdendo il controllo della
conversazione, se non lo avevo già perso. Una delle sue mani si chiuse
intorno al cubo d'acciaio, con le unghie rosse simili ad artigli
insanguinati che serrano un cuore.

«La rappresentante della Gilda dovrebbe ricordare che sta parlando al
figlio di lord Alistair Marlowe.» Questa volta fu l'altro peltasta a
parlare. Entrambi erano cani da guardia di mio padre.

Il colore abbandonò le guance di Lena che si abbandonò sul suo sedile.
Il nome di mio padre aveva quell'effetto sulle sue terre e sul resto di
Delos. Anche se il nostro era solo uno dei 126 Casati minori del sistema
che avevano giurato fedeltà alla viceregina-duchessa, il nostro era di
gran lunga il più ricco, il più nobile e nel consiglio era il più vicino
a lady Elmira. Negli ultimi anni mio padre aveva passato periodi di
tempo sempre più lunghi ad Artemia, nel castello della viceregina, e
molti anni prima aveva perfino rivestito il ruolo di suo esecutore
quando lei era lontana dal pianeta. Non era impossibile che presto gli
venisse chiesto di lasciare Meidua e il Riposo del Diavolo per occupare
un feudo e un titolo su un nuovo mondo tutto nostro.

«Chiedo scusa, mio signore.» Lena Balem posò il fermacarte come se
l'avesse ustionata. «Perdonami.»

Accantonai le sue scuse con un gesto e ritrovai il mio più cortese
sorriso. «Non c'è niente da perdonare, M Balem.» Mi morsi un labbro,
pensando ai soldati alle mie spalle che avevano ritenuto che \emph{ci
	fosse} qualcosa da perdonare. «Naturalmente presenterò le tue lamentele
a mio padre. Se hai delle proiezioni in merito al rapporto costi e
benefici di queste macchine sostitutive, credo che tanto lord Alistair
quanto i suoi consiglieri vorranno vederle.» Controllai l'ora sul
terminale da polso, impaziente di andarmene perché avevo ancora la
possibilità di essere presente all'arrivo dei visitatori Mandari. «M
Balem, ti suggerisco anche di dare una priorità alle tue necessità prima
di parlare con mio padre e con il suo consiglio, ma adesso ti prego di
scusarmi.» Guardai di nuovo il terminale con un gesto più che esplicito.
«Ho un appuntamento a cui andare.» La mia sedia strisciò sulle
piastrelle del pavimento quando mi alzai.

«Questo non basta, mio signore.» Lena Balem si alzò a sua volta,
guardandomi dall'alto del suo naso troppo grosso. «Le persone muoiono
\emph{regolarmente} in quelle miniere. Hanno almeno bisogno di adeguate
tute ambientali. La mia gente muore per via del gas radon, delle
radiazioni... ho delle fotografie.» Frugò fra i fasci di stampe sulla
sua scrivania, lucide immagini di torsi lesionati e di pelle ruvida.

«Lo so.» Le volsi le spalle, con le guardie che venivano avanti per
posizionarsi ai miei fianchi, e sentii la punta della daga che mi urtava
la gamba. In quel momento realizzai che quella donna avrebbe potuto
aggredirmi. Non si sarebbe mai comportata così con mio padre, ero stato
troppo morbido, mentre mio padre l'avrebbe fatta frustare e mettere ai
ceppi, nuda, lungo la strada principale di Meidua. Crispin l'avrebbe
decapitata lui stesso.

Io mi limitai ad andarmene.

\begin{figure}
	\centering
	\def\svgwidth{\columnwidth}
	\scalebox{0.2}{\input{divisore.pdf_tex}}
\end{figure}

«Hai avuto successo, mio signore?» chiese la giovane tenente, dopo che
il nostro velivolo decollò dal complesso della Gilda, nel distretto
inferiore della città sottostante le alture calcaree. Prendemmo
lentamente quota al di sopra dei tetti di tegole e oltre le guglie della
Città Bassa per salire a unirci al rado traffico aereo. Sotto di noi la
città di Meidua si stendeva come un bozzetto anatomico lungo la riva del
mare, sotto la possente acropoli sulla quale i miei antenati avevano
eretto la loro antica fortezza, la mia casa.

Mi arrischiai a lanciare un'occhiata alla tenente e scossi il capo.
«Temo di no, Kyra.» La navetta passò attraverso un pennacchio di vapore
bianco che si levava da un impianto nucleare sulla riva del mare mentre
descrivevamo un'ampia virata sull'acqua per avvicinarci al Riposo del
Diavolo da est. In cima a un'acropoli di pietra bianca, il granito nero
del muro di cinta e le guglie gotiche al suo interno assorbivano la luce
del sole, apparendo fuori posto sull'altura di calcare su cui si
ergevano come se un qualche potere inumano avesse strappato quelle
pietre ancora fumanti dal cuore del pianeta, come in effetti era stato.

«Mi dispiace sentirlo, sire.» Kyra infilò un ricciolo biondo sotto
l'orlo del cappello da volo mentre io lanciavo un'occhiata in tralice ai
due peltasti seduti in fondo alla navetta, sentendomi addosso il loro
sguardo.

«Ormai sei con noi da qualche tempo, vero, tenente?» domandai,
protendendomi in avanti contro le cinture di sicurezza.

«Sì, sire» rispose la donna da sopra una spalla, lanciando una rapida
occhiata nella mia direzione. «Da quattro anni.»

Il sole del pomeriggio che si riversava attraverso la calotta anteriore
incorniciava il suo volto di un fuoco candido, e provai una fitta al
cuore per lei. In Kyra c'era qualcosa che in qualche modo mi appariva
più reale delle dame di palazzo a cui ero stato presentato. Era più
viva... più umana.

«Quattro anni...» ripetei, sorridendo al suo profilo che potevo vedere
dal mio posto, sui sedili posteriori. «E hai sempre voluto essere un
soldato?»

Lei si irrigidì, segno che nella mia voce qualcosa l'aveva messa in
tensione. Forse era l'accento. Più di una volta mi era stato detto che
parlavo come il cattivo di qualche opera eudoriana. «Volevo volare,
sire.»

«Allora sono contento per te.» La mia attenzione non poteva più rimanere
sul suo volto. Arrossendo, spostai lo sguardo fuori dal finestrino e
verso la città... la mia città... osservandone il groviglio, il modo in
cui le strade si incrociavano come una ragnatela sulle alture
sottostanti il Riposo del Diavolo e al di sopra del mare. Potevo vedere
la cupola coperta di verderame della Cappellania, con i suoi nove
minareti protesi come lance verso il cielo, e all'estremità opposta
della strada principale la grande ellisse del Circus, che quel giorno
era aperto agli elementi. «È bello, qui.» Sapevo che stavo farfugliando,
ma mi distraeva dal pensiero di ciò verso cui stavo volando: mio padre e
gli ospiti Mandari che aveva voluto tenermi nascosti. Pensai a Crispin,
e al suo sorriso tagliente. «Niente di cui preoccuparsi.»

«Solo degli altri velivoli, Vostra signoria.» Vidi gli angoli della sua
bocca che si sollevavano e il fugace bagliore latteo dei suoi denti.
Stava sorridendo.

Sorrisi anch'io. «Sì, naturalmente.»

«Tu voli, sire?» domandò lei, prima di aggiungere contegnosamente: «Se
la domanda non secca Vostra signoria.»

Girandomi sul sedile guardai in modo esplicito i due peltasti che
sedevano vicino alla rampa di uscita sul retro del velivolo, serrando
con la mano guantata i cappi di supporto che pendevano dal soffitto
rivestito di pannelli grigi. «Non mi secca. Sì, so volare, ma non bene
quanto te. Qualche volta chiedi a sir Ardian al riguardo.»

Lei rise. «Lo farò.»

Incapace di liberarmi della nuvola che mi stava calando addosso cambiai
argomento, fissando ora la moquette che copriva il fondo della cabina.
«La delegazione è già arrivata al castello?»

«Sì, Vostra signoria» rispose la tenente, facendo descrivere al nostro
velivolo una brusca discesa che ci portò al di sotto della corona delle
alture, là dove finiva la roccia viva e cominciava il granito nero. Non
so perché, ma nel guardare la vecchia costruzione da sotto, in quel
modo, immaginavo sempre di sentire lo schianto di un tuono. «Alcune ore
fa.»

Era come avevo temuto e mi ero aspettato. Avrei perso la cerimonia.
«Cosa fa tuo padre, Kyra?» Non era stata mia intenzione farmi sfuggire
quelle parole, e tuttavia lo avevo fatto... cose piccole, e pericolose.

«Sire?»

«Tuo padre» ripetei. «Che cosa fa?»

«Lavora alla griglia d'illuminazione della città, sire.»

Le mie labbra si contorsero, formando una battuta scadente. «Vuoi fare a
cambio?»

\begin{figure}
	\centering
	\def\svgwidth{\columnwidth}
	\scalebox{0.2}{\input{divisore.pdf_tex}}
\end{figure}

Il castello del Riposo del Diavolo, prodotto di un'èra più grande della
nostra, era di per sé esteso quanto una città, anche se vi dimorava meno
di un decimo delle anime rispetto a quelle che si trovavano intorno e
sotto le sue mura. Quando esse erano state innalzate, l'Impero Solano
dominava fra le stelle, senza opposizioni alla sua potenza e maestà, il
solo potere umano nel cosmo. Anche se quei giorni felici di sangue e di
tuono erano passati da tempo, il castello resisteva ancora, una
confusione di contrafforti, di guglie e di muratura che si levava come
ossa segnate dagli elementi al di sopra delle colline sovrastanti
Meidua. Per quanto grandiosa, la vecchia fortezza era piccola in base
agli standard moderni. La Grande Rocca, un massiccio bastione squadrato
d'acciaio rivestito in pietra nera, si ergeva di appena cinquanta
livelli al di sopra della piazza su cui si stagliava, e tuttavia
dominava tutte le altre strutture del castello, perfino i minareti della
nostra Cappellania privata. La torre di dodici piani del monastero degli
scoliasti appariva miserevole nell'estremo angolo vicino ai giardini e
alle mura esterne. Mi diressi con passo deciso verso la Rocca, passando
attraverso le ombre di un colonnato con i tacchi degli stivali che
risuonavano sul mosaico.

Avevo perso le mie due guardie nell'hangar di atterraggio e {lasciato}
Kyra a finire di spegnere il motore del velivolo ma non ero solo:
peltasti e opliti in armatura leggera dotati di scudo e piastra in
ceramica completa erano disposti a intervalli lungo il colonnato e lo
scalone che portava al viadotto da cui si accedeva alla piazza alla base
della Rocca. Là mi trovai spalla a spalla con una folla di logoteti in
uniforme del personale domestico che amministrava la nostra piccola
fetta di Impero, ma anche se fossi stato l'unico presente sul sentiero
non sarei stato da solo, nessuno lo era mai, perché le videocamere
vegliavano sempre.

Oltrepassai la statua di Julian Marlowe -- morto da tempo e raffigurato
in sella al suo cavallo con la spada protesa verso il cielo in un atto
di sfida -- e salii gli ampi gradini di marmo, proseguendo attraverso le
porte principali dove mi soffermai a salutare dama Uma Sylvia, la
cavaliere-littore di guardia all'ingresso.

«Mio padre?» chiesi, con la voce che risuonava nitida nell'aria
pomeridiana.

«È ancora nella sala del trono, giovane padrone» replicò Sylvia, senza
infrangere la perfetta posizione di attenti.

Attraversai il pavimento di piastrelle bianche e nere, passando di netto
sopra il raggio di sole ramato del sigillo imperiale e dirigendomi verso
la scala interna. Bandiere nere pendevano pesanti dalle alte mura e un
rumore di piedi e di trombe echeggiava lungo il condotto cavo al centro
di quello spazio per trenta dei cinquanta livelli della Rocca. Quella
nobile bandiera, stemma dei miei padri fin dalla notte dei tempi, era
adesso contaminata dalla mia mano. Forse l'hai vista. Più nera dello
spazio più buio, con il suo diavolo rosso che saltella con il tridente
in mano sulle nostre parole: \foreignlanguage{italian}{la spada, il
	nostro oratore.} Due diavoli come quello si fronteggiavano accanto ai
battenti di ferro battuto della sala di mio padre, facendo apparire
minuscolo l'arco a punta dell'ingresso e gli uomini che lo
sorvegliavano.

Erano strane, quelle porte... cose pesanti di ferro colato, grezzo e
trattato con una qualche resina opaca per proteggerlo dalla ruggine.
Ogni battente era alto il triplo di un uomo e spesso parecchi pollici,
per cui la confusione di forme umane realizzata in rilievo su ciascuna
superficie spiccava in modo netto. Ogni battente doveva pesare parecchie
tonnellate, ma si spostavano con delicatezza, bilanciati da lenti
contrappesi, per cui anche un bambino avrebbe potuto muoverli.

«Padrone Hadrian» mi salutò sir Roban Milosh, un uomo furtivo dalla
pelle scura e dai capelli ricciuti. «Dove sei stato?»

Socchiusi gli occhi e mi presi un momento per ritrovare il controllo,
sussurrando sottovoce un aforisma degli scoliasti: «L'ira è cecità.»
`L'ira è cecità.' «Sono stato trattenuto presso la Gilda Mineraria»
risposi a Roban. «Ordini di mio padre. Loro sono dentro?»

«Da circa trenta minuti.»

A disagio, perché ero consapevole del mio aspetto arruffato, dei capelli
troppo lunghi e aggrovigliati e dell'imperfezione spiegazzata della mia
giacca formale, battei un colpetto sul braccio del cavaliere. «Vuol dire
che hanno concluso la parte noiosa. Lasciami passare.» Accennai a
oltrepassarlo, premendo di piatto il palmo sulla porta, ma la
controparte di Roban si fece avanti e mi afferrò per un braccio.
Indignato, mi girai di scatto e trafissi l'oplita con uno sguardo
rovente. Il suo elmo, come quello della maggior parte delle tute da
combattimento, non aveva la visiera, solo un carapace a coste fatto di
solida ceramica che gli nascondeva la faccia. Le videocamere
trasmettevano le immagini a uno schermo all'interno della maschera,
dando l'impressione che lui fosse una statua e non un uomo.

«Lord Alistair dice che non può entrare nessuno mentre riceve la
direttrice.» Mi lasciò andare con fare esplicito e deciso. «Mi dispiace,
giovane padrone.»

Sforzandomi di controllare l'improvvisa ondata di indignazione ripetei
mentalmente l'aforisma degli scoliasti, cercando di impedirmi di
focalizzarmi in modo eccessivo sul timore persistente che cresceva
dentro di me. Mi sarei dovuto girare e andarmene. Sarebbe stato più
facile.

Invece mi schiarii la gola. «Soldato, fatti da parte.»

«Hadrian.» Roban mi posò una mano sulla spalla. «Abbiamo degli ordini.»

Mi girai, e confesso che la frustrazione fu la mia rovina. «Toglimi le
mani di dosso, Roban» ingiunsi, e spalancai le porte prima che il
cavaliere-littore o il suo luogotenente potessero fermarmi. I battenti
non fecero rumore nell'aprirsi. Fatto il danno, mi girai e trafissi con
uno sguardo rovente l'oplita proteso ad afferrarmi. Avevo gli stessi
occhi di mio padre, e sapevo come usarli. L'uomo si perse d'animo.

Nessuna fanfara accompagnò il mio ingresso, a meno di {considerare} i
cenni di saluto dei peltasti appena oltre la soglia. C'è un limite alla
vastità dello spazio che un occhio e una mente umana possono apprezzare
a fondo, al di là del quale l'impatto della grandiosità si fa
travolgente. La sala del trono superava quel limite in quanto era allo
stesso tempo troppo ampia, troppo alta e troppo lunga. File e file di
colonne scure si allontanavano a destra e a sinistra, supportando volte
affrescate che raffiguravano la morte della Vecchia Terra e la
successiva colonizzazione di Delos. Anche se i sensi umani non erano in
grado di registrarla, la distanza fra pavimento e soffitto si riduceva
in modo sottile fra le porte e la piattaforma all'estremità opposta, in
modo che un supplice venisse ingannato e indotto a percepire l'arconte
come più grande di quanto qualsiasi umano avrebbe dovuto essere. Dicono
che il Trono Solare, a Forum, utilizzi quello stesso trucco in modo che
l'imperatore possa apparire più grande del più nobile duca della sua
costellazione.

Il trono stesso era ammantato nell'ombra e le due corna ricurve dietro
di esso -- che in realtà erano le costole di un grande balena d'ottone
-- torreggiavano fino a metà della distanza dal lontano soffitto,
bloccando la luce del rosone incastonato dietro il trono in modo tale
che la figura seduta su di esso apparisse velata.

La delegazione del Consorzio era riunita davanti al trono, in piedi ed
eretta alla base della piattaforma, con la sagoma di ciascuno allungata
in modo assurdo dalla microgravità presente sulle navi su cui vivevano.
Erano in sette, tutti con abiti uguali, accompagnati da due dozzine di
soldati in un'uniforme grigio opaco e armati ciascuno con un corto
fucile al posto delle lance a energia preferite dagli uomini di mio
padre.

«Perdona il mio ritardo, padre.» Usai la mia voce da oratore, sfruttando
appieno la potenza dell'addestramento retorico ricevuto da Gibson. «La
rappresentante della Gilda Mineraria mi ha portato via un po' più tempo
del previsto.»

«Perché \emph{tu} sei qui?» Il suono di quella voce in quel luogo mi
fece cagliare il sangue e sentii un vento gelido soffiare attraverso la
mia anima. Non solo Crispin aveva saputo dei visitatori appartenenti al
Consorzio Wong-Hopper, ma era stato invitato a presenziare.

Ignorai la sua domanda petulante e mi avvicinai fino a dieci passi dalla
fila di ospiti del Consorzio raccolti ai piedi del trono di mio padre.
Non ero ancora nell'ombra di quella grande seduta, e mio padre era
soltanto una forma più scura nell'oscurità di quel seggio fatto d'ebano
e di ferro battuto. Piegando a terra un ginocchio davanti al trono
chinai il capo di fronte ai visitatori Mandari. «Onorati ospiti,» dissi,
apprezzando l'esperta profondità della mia voce dopo il piagnucolio di
Crispin «perdonate il mio ritardo. Sono stato trattenuto da questioni
locali.»

Uno degli alti visitatori mosse un paio di passi verso di me. «Alzati,
prego.» Lo feci, e il rappresentante del Consorzio si girò verso mio
padre. «Qual è il significato di questo, lord Alistair?»

Mio padre si mosse sul suo trono. «Questo è il mio figlio maggiore,
direttrice Feng.» La sua voce, che mi sarebbe dovuta risultare familiare
quanto la mia, era per me come quella di uno sconosciuto.

La donna che si era rivolta a me annuì, lasciando ricadere lungo i
fianchi le mani simili a zampe di ragno con un fruscio delle maniche
grigie. «Capisco.» Gli altri membri del Consorzio strisciarono un poco i
piedi calzati di babbucce.

«Siediti.» La figura di mio padre si mise lentamente a fuoco mentre i
miei occhi si abituavano alle ombre profonde intorno al trono. Il suo
aspetto era più simile al mio di quello di Crispin, perché il telaio
genetico lo aveva strutturato alto, snello e duro, con un volto aquilino
tutto piani affilati e angoli duri. Come me, lui ignorava la moda locale
e i suoi capelli erano lunghi, pettinati all'indietro in modo che gli si
arricciassero appena sotto le orecchie. Il volto freddo era rasato, con
le labbra spesse e con occhi viola che osservavano indifferenti tutto
quello che era sotto di lui.

Deglutii a fatica e oltrepassai la direttrice Feng e i suoi colleghi,
fissando la mia attenzione su una fila di tre seggi posta al di sotto e
sulla destra del trono. Crispin sedeva là da solo, sul seggio più vicino
a nostro padre. Mi fermai a fissarlo come di certo nostro padre stava
fissando me. «Spostati, Crispin.»

Lui si limitò a inarcare un sopracciglio, scommettendo -- giustamente --
sul fatto che non avrei insistito davanti ai nostri ospiti. Non lo feci.
Ero troppo un gentiluomo per cose del genere, ma ero abbastanza
infantile da prendere il più piccolo seggio di legno padouk vicino a lui
e da spostarlo due gradini più in su. E mi sedetti, ignorando la cupa
indignazione che sentivo provenire da mio padre, sul suo trono scuro.