\chapter{Conversioni sul letto di morte}

Sempre avanti, sempre giù, mai a destra o a sinistra. Mi serviva un modo
per liberare Uvanari dai tormenti che gli avevo procurato, dovevo
placare gli altri Cielcin seguendo un copione culturale che capivo a
stento e proteggere me stesso dalla mia gente quando le mie traduzioni
sarebbero state inevitabilmente confrontate con le registrazioni. E
dovevo proteggere Valka, che adesso era mia complice, una strega delle
macchine che si era macchiata di uno dei Dodici Abomini della
Cappellania. Soprattutto, dovevo andarmene da Emesh, dal conte, dalla
grande priora e da tutte quelle persone che insistevano nel considerarmi
una mosca intrappolata nella loro ragnatela.

Penso che Gibson avesse ragione sul mio conto; melodrammatico fino alla
fine. A parte i Cielcin, non dissi a nessuno del mio piano. A nessuno
tranne che a Valka.

Entrai per l'ultima volta nella cella degli interrogatori con acido al
posto del sangue e i nervi in fiamme in ogni fibra e tessuto del mio
corpo. Dovevo stare attento, molto attento, perché stavamo volando alla
cieca. Se Valka non fosse riuscita a manomettere il sistema di
sorveglianza della bastiglia una seconda volta, se Tanaran e gli altri
avessero agito prima che lei avesse operato la sua magia tecnologica, se
Uvanari fosse stato troppo danneggiato per recitare la sua parte...
ecco, c'è un minotauro in ogni labirinto, a volte anche più di uno.

«Siamo consapevoli che quello chiamato Tanaran è il capo della vostra
spedizione» cominciò l'inquisitrice.

Quando tradussi questo, Uvanari snudò i denti e si tese un poco contro i
blocchi elettromagnetici che gli trattenevano i polsi e le caviglie.
Guardò verso di me. «Glielo hai detto?»

«\emph{Ekaan}» risposi, aggiungendo quella sorta di sospiro che
significa `sì' nella lingua dei Cielcin. «Non ho avuto scelta. Tanaran è
imparentato con il tuo padrone, l'\emph{aeta} Aranata? È... suo
\emph{figlio}?» Mentre parlavo studiai le bende intorno al braccio
scuoiato e la flebo che immetteva anestetico nel dorso della mano
rovinata e nel circolo sanguigno. Chiusi la mano intorno al coltello che
portavo addosso, una mancina come quelle che avevo usato a casa,
acquistata durante una delle mie rare visite in città.

«Perché Tanaran dovrebbe essere suo \emph{figlio}?»

Ripetei la domanda ad Agari, aggiungendo: «Comunque gestiscano la
successione, non è ereditaria. Sempre che si tratti di questo.» Per
quanto interessante, l'informazione era accademica, e l'inquisitrice non
era una scoliasta.

Sbuffò, e invece di permettermi di rispondere a quella domanda ne pose
una nuova. «Il tuo padrone tratterebbe per Tanaran?»

«Reverenza, dici sul serio?» chiesi, inarcando e contraendo le
sopracciglia. «Non mi è stato detto che stavamo considerando...»

Agari dilatò le narici e lanciò una rapida occhiata al cathar Rhom prima
di rispondere. «Limitati a porre la domanda.» Il cuore mi si alleggerì
un poco e mi girai per obbedire.

Uvanari contrasse la mascella un paio di volte e sollevò il mento di
scatto. «È possibile, ma l'\emph{aeta} ha altri eredi.» Usò la stessa
parola che aveva usato in precedenza: \emph{baetayan, `}radici'. Qui
però il contesto era più chiaro e vidi come noi potevamo usare quella
parola. «\emph{Masvii iagami caicane wo ne}?»

«Chiede se lasceremo liberi gli altri» dissi, desiderando di sentire io
stesso la risposta.

Gli occhi di pietra di Agari si ridussero a due fessure, mentre gli
ingranaggi della mente dietro di essa prendevano a funzionare in modi
che preferivo non comprendere. I muscoli della sua mascella si
contrassero, con i denti che si muovevano come se cercassero di
masticare una cartilagine. Era come se il suo cervello fosse alimentato
dal loro attrito. «Rispondi all'\emph{inmane} che stiamo considerando
tutte le possibili alternative.»

Lo feci, omettendo l'insulto imperiale. Sul volto inumano di {Uvanari}
apparve quel ringhio che passava per un sorriso. «Vedo che anche nella
vostra specie ci sono dei politici.»

Sorrisi, poi mi resi conto di quello che stavo facendo e mostrai invece
i denti, imitando come potevo il sorriso dei Cielcin. Uvanari parve
capirlo perché ricambiò il gesto mentre dicevo: «Sì.» Quando Agari mi
chiese un chiarimento risposi: «Afferma che quelle parole non hanno
senso.»

L'inquisitrice sbuffò per indicare che aveva capito e armeggiò con i
pannelli olografici, alzando di un livello il tenore delle domande.
Mentre esaminava quel cambio di direzione io trassi un lungo respiro per
calmarmi e l'ascoltai prendere annotazioni al terminale: «...modo per
salvare i prigionieri. Quello catalogato come prigioniero A009 deve
essere detenuto privatamente fino a decisioni ulteriori. Raccomandazione
di Agari, K.F...» Snocciolò la data. Per un momento sospettai che
l'orrore fosse finito e lanciai un'occhiata ai pannelli olografici sulla
parete, ciascuno dei quali forniva un identico feed dal vivo degli altri
Cielcin in un recinto di raccolta. Avvertii l'assurdo desiderio di
salutarli con la mano anche se non ci potevano vedere. Forse tutta la
mia pianificazione era stata inutile.

«Capitano,» riprese Agari, con voce d'un tratto così gelida che nel
raddrizzarmi sentii il mio fluido spinale che si cristallizzava «se
vogliamo tentare una negoziazione del genere per la tua gente ci serve
sapere la loro posizione.»

Mi si formò un nodo in gola. Sapevo che fra tutte le linee di
interrogatorio questa non avrebbe dato frutti, non importava la quasi
cortesia che l'inquisitrice aveva appena esibito. A quel punto ogni
dettaglio della stanza tornò di colpo a mettersi a fuoco: la croce con
le sue assi regolabili e i suoi blocchi magnetici, la grata sul
pavimento, le pareti sterili, gli strumenti disposti sul carrello, tanto
per curare quanto per torturare. Poi c'erano i cathar, identici nella
loro calvizie, nelle vesti nere e nei grembiuli più clinici che
liturgici, più scuri di quanto qualsiasi tessuto avesse il diritto di
essere. Questo non era un posto fatto per gli scherzi, o anche per gli
atteggiamenti di sfida. Chiusi la mano intorno all'impugnatura del
coltello fino a farmi dolere le ossa mentre aspettavo la risposta di
Uvanari.

Il Cielcin rifiutò. «Questo non ve lo posso dire.»

A un cenno dell'inquisitrice, il cathar prelevò dal retro del carrello
un congegno lungo quanto il mio avambraccio che aveva a un'estremità un
pomolo che lo faceva somigliare a una mazza. In silenzio, protese
l'oggetto perché l'\emph{ichakta} lo esaminasse, poi azionò un
meccanismo nell'impugnatura e il materiale composito che formava la
testa dell'oggetto si riscaldò rapidamente fino a risplendere come un
carbone appena tolto dal fuoco. La paura si accese negli occhi di
Uvanari, che cercò di ritrarsi sputando imprecazioni.

Il secondo cathar si chinò e rimosse il panno che copriva i genitali
dello xenobita, poi armeggiò con le cinghie di cuoio che lo
trattenevano, lasciando solo i blocchi elettromagnetici. In una commedia
eudoriana, il personaggio del Torturatore espone sempre nei dettagli le
sofferenze che intende infliggere alla sua vittima, di solito l'Eroina.
Si attorciglia i baffi e strizza l'occhio al pubblico, sfregandosi le
mani e ridacchiando. I cathar non parlarono, non fornirono spiegazioni.
Si limitarono ad agire.

«Cosa stanno facendo?» chiese Uvanari, la cui voce profonda si era fatta
di colpo molto più alta. «Hadrian, cosa stanno facendo?» Una volta
rimossa l'ultima cinghia infiammabile, il secondo cathar si trasse
indietro dalla vittima e rimase fermo con le mani intrecciate.

Non ebbi il tempo di rispondere perché il primo cathar calò la mazza,
riversando sul Cielcin gocce rosso ciliegia piccole come lacrime.
Uvanari urlò e l'aria gelida della cella si riempì del fumoso puzzo
metallico della carne che bruciava. Il suo petto esile ansimò,
risucchiando aria per alimentare altre urla, un suono che collassò
attraverso la sofferenza fin nel vestibolo della follia. Vesciche
irritate cominciarono a formarsi sulla sua pelle bianca, grigie e nere
con un contorno di un giallo itterico. Sangue simile a inchiostro prese
a filtrare dalle ferite insieme a qualcos'altro, qualcosa di argenteo.

Piombo.

Il bastardo aveva usato il \emph{piombo.}

Non riuscii a trattenermi. «Inquisitrice! No!»

Uvanari ansimò qualcosa mentre sottili lingue di fumo salivano verso
l'alto e verso un condotto di aspirazione nel soffitto. Agari mi ignorò.
Il bianco dei suoi occhi era troppo visibile nella cella nuda. Mi
aspettavo che mi rimproverasse, che mi urlasse contro, che chiedesse
come osavo dire una sola parola contro di lei, invece chiese soltanto:
«Che cosa ha detto?»

Ascoltai ancora e scossi il capo. «Non riesco a capirlo, razza di
maniaca. Ti sei spinta troppo oltre.» Fu solo con uno sforzo tremendo
che mi trattenni dall'estrarre la daga in quel preciso momento.

L'inquisitrice scrollò le spalle, poi ripeté la domanda, l'ordine.
«Dicci la posizione della vostra flotta.»

«Che l'aria ti faccia marcire» riuscì a ribattere Uvanari. Era un
qualche tipo di imprecazione? Non c'era modo di esserne sicuro. «Tu
tradiresti la tua gente?»

L'inquisitrice agitò una mano e il cathar tornò a colpire il Cielcin con
la mazza, lasciando un medaglione di carne bruciata grande quanto il mio
pugno. «Dacci la posizione della vostra flotta!» ripeté l'inquisitrice e
io le feci eco con la mia voce più flebile, con lo sguardo fisso sul
cathar e non sul prigioniero bloccato sulla sua croce. Speravo solo che
Valka non ci mettesse più molto.

Il mio sguardo incontrò quello di Uvanari. I denti di vetro brillavano
nelle gengive nere, le labbra sottili erano ritratte. Per un orecchio
non addestrato l'ansimare del suo respiro avrebbe potuto essere una
ripetizione della parola cielcin `sì'. Trassi un profondo respiro, ben
sapendo che non si trattava di questo. Da un momento all'altro Valka
avrebbe tolto energia ai livelli più bassi e Tanaran e gli altri si
sarebbero scagliati gli uni contro gli altri.

«Dacci la posizione della vostra flotta.»

Urla.

«Dacci la posizione della vostra flotta.»

Fumo.

«Dacci...»

Urla.

Uvanari pendeva accasciato dalla croce con la faccia abbandonata contro
il petto in maniera tale da mostrare solo le corna della sua area
epoccipitale. Senza le cinghie di cuoio, legato solo ai polsi, alla vita
e alle caviglie, sembrava a stento attaccato alla croce. Mentre
guardavamo, tutti silenziosi come i cathar, Uvatari vomitò un fluido blu
che schizzò la grata, poi vomitò ancora e sputò per pulirsi la bocca.
«Voi umani. Siete... tutti uguali.» Mi accigliai, colto sul punto di
riferire quell'affermazione all'inquisitrice. «Lo hai già detto in
passato!» esclamai in cielcin, costringendo l'inquisitrice a uscire dal
suo loop. «Lo hai già detto prima! All'inizio.» Mi era sfuggito. Come
aveva fatto a sfuggirmi?

«Cosa sta dicendo?» domandò Agari, alzando una mano per fermare i
cathar. «Marlowe.»

«\emph{Sssh}.» Ero tanto distratto dalle implicazioni di quelle parole
che non mi girai neppure a guardarla. Dimenticai la Cappellania, il
\emph{ndaktu}, il piano. «Hai parlato in passato con degli umani?» Di
certo quella non era solo una figura retorica. «Dove?» Oltrepassai i
cathar, ripetendo quella domanda, quella parola. «\emph{Saem ne}?
\emph{} Dove? \emph{Saem ne}? \emph{Saem ne}?»

Agari non era tanto ottusa da non notarlo e mi afferrò appena al di
sopra del gomito. «Cosa sta succedendo?»

Le spiegai quello che il Cielcin aveva detto e aggiunsi: «Se hanno già
incontrato altri umani prima d'ora, forse quelle persone sanno dove
trovare la loro flotta. Non abbiamo bisogno di fare questo.»

A un cenno di Agari, il cathar che brandiva la mazza con lo spruzzatore
ne sbatté la testa contro lo stomaco di Uvanari, appena al di sopra di
dove ci sarebbe stato l'ombelico in un essere umano. La vescica lasciata
dall'impatto tinse di grigio la pelle bianca, lasciando una bolla
liscia, e Uvanari serrò i denti, sibilando: «Digli di fermarsi.»

Si riferiva ad Agari. Scossi il capo. «Lei non lo farà.»

«Lei?» Uvanari sanguinava da un centinaio di minuscole ferite e nel
guardarlo mi sentivo intontito quanto lui era in agonia. Valka ci stava
mettendo troppo. Aveva cambiato idea? Aveva abbandonato sia me sia
l'intero piano? Di certo non lo avrebbe fatto. Con uno sforzo tremendo,
come se stesse sollevando una pusterla e non la testa, Uvanari mi guardò
dritto in faccia, socchiudendo gli occhi. «La vostra razza... mostri.
Prendete quello che volete.»

«Non siamo mostri. Non mangiamo i vostri simili» ritorsi, mentre mi
tornavano alla mente tutte le storie della mia infanzia, tutti i
racconti di uomini messi allo spiedo e arrostiti su fuochi di plasma,
dei Pallidi che divoravano cervelli e bambini nella notte. Avevo sempre
creduto che fosse propaganda della Cappellania.

«Lo facciamo se dobbiamo» sibilò Uvanari.

«Tutto questo è inutile» dichiarò Agari, spegnendo i prompt olografici
del suo terminale con un gesto della mano. «Inutile!» Si girò verso i
cathar, esitando sull'orlo di una decisione inespressa, incerta su come
procedere. Anche lei era nel labirinto, persa quanto me.

\emph{Vwaa}! \emph{Vwaa}!

Luci di allarme pulsarono scarlatte agli angoli della cella mentre le
sirene levavano il loro lamento inarticolato. Gli occhi di Uvanari --
due pozzi di assoluta oscurità -- si spostarono per guardarmi in faccia.
Agari si immobilizzò, sollevando lo sguardo verso il soffitto, e perfino
io, che pure mi ero aspettato quell'interruzione, sobbalzai mentre mi
premuravo di essere il primo a parlare. «Cosa succede?»

L'inquisitrice controllò il suo terminale e imprecò sottovoce. «Un altro
calo di tensione. Non mi importa quanto i generatori siano rimasti
danneggiati durante i tumulti dovuti alla peste... giuro che questo è un
atto deliberato.» Lottai per mantenere un'espressione attenta mentre
pensavo a Valka e ai suoi impianti tavrosiani eretici. Nonostante la sua
condizione di diplomatica, le avevo chiesto di varcare un limite.

\emph{Vwaa}! \emph{Vwaa}!

Anche il terminale di Agari cominciò a emettere un allarme e lei
premette un pulsante su di esso, accostandoselo alle labbra. «Cosa c'è?»
Non potei sentire la risposta, che le giunse attraverso l'auricolare
nascosto dietro un orecchio, ma il colore defluì dal suo volto scuro.
«Allora fermateli!» disse. «No! No, dannazione! Usate gli storditori.
\emph{Gli storditori}.» Il suo sguardo saettò dai due cathar a me. «No!
Sto arrivando!»

\emph{Vwaa}! \emph{Vwaa}!

«Cosa c'è?» chiesi, venendo avanti in fretta, anche se pensavo di
conoscere già la risposta. «Cosa è successo?»

«Hanno perso l'energia nell'ala occidentale. Le videocamere sono
inattive.»

In quel momento adorai Valka.

«Un'altra interruzione della corrente elettrica?» dissi. Poi: «Aspetta,
l'ala occidentale? Ma non è dove gli altri...» Mi interruppi di colpo e
spostai lo sguardo su Uvanari che pendeva dalla sua croce. Era difficile
dire se ci stesse ascoltando o meno, perché aveva di nuovo la testa
accasciata e il suo respiro era più difficoltoso.

L'inquisitrice si accigliò -- c'era un'accusa nei suoi occhi neri? -- e
si diresse verso la porta che si aprì con un sibilo dei suoi meccanismi
idraulici. «Vieni, Marlowe.» Non mi mossi e lanciai un'occhiata a
Uvanari da sopra la spalla. «Vieni. C'è bisogno di te. Gli altri
Cielcin...»

«Lasciami rimanere.»

«No.»

«Lascia che gli parli.» Guardai verso i cathar. «Da solo.»

«Si stanno \emph{aggredendo} gli uni con gli altri. Verrai con me,
adesso.»

Digrignai i denti con la massima forza di cui ero capace e puntai un
dito verso il pavimento. «L'ultima volta che mi hai lasciato solo ho
fatto più progressi di quanti ne abbia ottenuti tu in due settimane
tagliando e bruciando. Dammi cinque minuti. Dieci.» Gli allarmi levavano
ancora il loro urlo stridente, luci rosse e bianche descrivevano spirali
sulle pareti e sul soffitto di metallo. Quasi per un ripensamento,
aggiunsi: «Solo finché non avrai messo in sicurezza gli altri.»

\emph{Vwaa}! \emph{Vwaa}!

Parve che lei impiegasse un secolo al rispondere. Non era il genere di
persona abituata a essere sotto pressione per la mancanza di tempo. Era
abituata ad avere il controllo assoluto, a essere al comando di ogni
dettaglio e parola di una situazione, al potere. Il caos la irritava,
infrangeva la sua disciplina, quale che fosse. La paura è un veleno e la
stava avvelenando. Mi morsi una guancia per bloccare il sorriso in
tralice dei Marlowe che minacciava di dipingersi sul mio volto e di
rubarmi i miei piani. La parola che le uscì di bocca era tutta aria,
priva di suono e informe quanto il `sì' dei Cielcin.

Indicò. «Rhom, rimani con lui. Tu, con me.» E uscì in un vortice di
vesti nere e un sibilare di meccanismi idraulici, portando con sé
l'altro cathar e tutte le mie speranze.

I cathar. Mi ero dimenticato di quei tre volte dannati cathar. Non
parlavano mai, quindi erano scivolati del tutto al di sotto del livello
della mia attenzione. Una cosa stupida, veramente stupida. Il cathar
Rhom era fermo in un angolo, in silenzio. Non osai discutere. Quando
tutto questo fosse finito, quando avessi fatto quello che avevo promesso
di fare, mi avrebbero incolpato di una quantità di cose anche senza che
il cathar testimoniasse in merito alla mia richiesta che venisse mandato
via, ma con lui presente ero finito.

Mi avvicinai ulteriormente a Uvanari nella speranza che il mio sussurro
gli arrivasse nonostante il gemito degli allarmi. «\emph{Eka yitaya}» mi
scusai, con la mano stretta intorno all'impugnatura della daga. «Non
posso fare quello che mi hai chiesto.»

Lo xenobita sollevò la faccia, e perfino assicurato com'era a quel
telaio per le torture mi guardò dall'alto in basso, con i denti di vetro
che sembravano del colore del sangue umano sotto le luci d'allarme
rosse. Anche se era difficile dirlo con certezza mi parve che i suoi
occhi neri guardassero in direzione del cathar fermo nel suo angolo per
poi spostarsi sulle luci d'allarme che pulsavano nel soffitto. «Hai
fatto tu questo?»

Misi a nudo i denti, sperando che la sottigliezza di quel gesto
sfuggisse a chi sarebbe venuto dopo di me. Volevo dirgli quello che
stava per succedere -- in realtà avevo bisogno di tirarmi indietro -- ma
indipendentemente dalla melodrammaticità eudoriana avevo anche la mia
parte da recitare. «Allora avevi già incontrato degli umani prima
d'ora?» domandai.

Lui emise quell'orribile lamento fischiante che nella sua specie passava
per una risata, anche se suonò più flebile di come l'avevo sentita in
precedenza, poi protese in fuori il mento in uno di quei gesti che per
coincidenza avevamo in comune, e ribatté: «Adesso sei tu a fare le
domande?»

«Per favore» insistetti, guardando nervosamente verso il cathar da sopra
la spalla. Era ancora là, immobile, un terribile spettro del fallimento
fin troppo visibile. Indietreggiai di un passo.

Il Cielcin esalò un respiro attraverso le fessure nasali. «Sì.»

«\emph{Saem ne}?» chiesi. `Dove?' \emph{} Non rispose e si ritirò in un
atteggiamento cupo, certo che i suoi tormenti non fossero finiti. Quello
era solo un momento di calma, o lo sarebbe stato senza le sirene.

Poi lo fu davvero.

Le sirene cessarono di risuonare, tutte le luci si spensero tranne
quelle di servizio ancora accese sul carrello chirurgico dei cathar,
minuscoli indicatori simili a stelle che quasi non emanavano luce in
quel posto buio. Non c'era più energia, niente videocamere, niente di
niente. Se non fosse stato per il cathar saremmo rimasti soli abbastanza
a lungo perché potessi soddisfare il desiderio del Cielcin e la mia
tacita promessa di liberarlo da quel posto.

Davanti a me qualcosa cadde sul pavimento come un tonfo, pesante e duro
come un osso.

Incespicai all'indietro con una mano ancora sul coltello mentre davanti
a me si muoveva qualcosa di più oscuro del buio. Un paio di momenti più
tardi realizzai cosa era successo.

I blocchi.

I blocchi elettromagnetici.

Per la Terra, i cathar avevano rimosso le cinghie di cuoio per impedire
che prendessero fuoco durante la tortura con il piombo e non le avevano
rimesse, lasciando così libero Uvanari quando Valka aveva disattivato la
corrente. Mi aspettai che giacesse in un mucchio informe alla base della
croce, che si stringesse le braccia intorno al corpo, che piangesse,
perché era quello che avrei fatto io, o qualsiasi altro essere umano che
avesse subìto quello che aveva sofferto Uvanari.

Il Cielcin non era umano.

Si alzò, ringhiando attraverso le fessure sul volto piatto.

Fratello Rhom realizzò troppo tardi tutto questo. Con una velocità che
non avrei mai potuto immaginare, il Cielcin mi oltrepassò a passo di
carica con le mani rovinate protese, di un grigio pallido nella
penombra. Un sordo crepitio risuonò nella cella, riecheggiando nel
silenzio: la testa di Rhom sbattuta contro la parete. Uvanari gemette
nel passare il braccio scuoiato e fasciato intorno alla gola del cathar
per poi chiudere l'altra mano intorno alla testa più piccola dell'umano.
Avrei dovuto fermarlo, trafiggerlo alle spalle: sarebbe stato perfetto,
avrei avuto un'eccellente giustificazione. Adesso me ne rendo conto, ma
allora non lo feci e rimasi immobile, paralizzato, con la mascella
rilassata, mentre Uvanari torceva il collo al cathar, rimuovendo la
benda sugli occhi. Per un momento vidi i suoi occhi -- bianchi e puri,
senza iride o pupilla, risplendere nel buio -- prima che il Cielcin gli
affondasse i denti che gli rimanevano nella carne della gola,
lacerandola. Quegli occhi si spensero come scintille. L'oscurità mi
risparmiò il colore del sangue, solo la sua forma vaga come un tuono
lontano si impresse nella mia mente pietrificata. Uvanari si accoccolò
sul corpo di Rhom e giuro che bevve il suo sangue. Ancora sconvolto,
dopo un momento estrassi il coltello. I Cielcin hanno un udito più acuto
del nostro anche se non so come questo sia possibile quando le loro
orecchie sono due fori incassati nella testa, come quelli delle
lucertole. Comunque il sottile stridio della ceramica contro il fodero
di cuoio attirò l'attenzione del capitano Cielcin, che si girò, con il
sangue rosso che gli copriva la faccia e gli colava dal mento per cadere
sul torace appuntito. «Allora hai intenzione di farlo?» La lingua
bluastra, che appariva nera nell'oscurità, saettò verso il basso,
visibile per la sua lucentezza, e assaggiò l'aria.

«È quello che volevi, il motivo per cui sono qui.» Sollevai la mancina,
la cui opaca lama bianca risplendeva fra il blu e il grigio nella
penombra, quasi dello stesso colore della spada di altamateria del
maeskolos. Desiderai con fervore di avere in pugno quell'arma invece di
quel pezzo di comune zircone.

Il Cielcin si alzò, osceno e satanico nella sua nudità. Sentii le narici
che si dilatavano. «Sì. Non so bene come hai fatto, ma eccoci qui.»

«\emph{Ndaktu}» replicai, rimanendo in guardia. «Hai detto che era colpa
mia. Mi volevi qui!» Sapevo che stavo cercando di ragionare con la
morte, e che non si poteva mai ragionare con essa. Avevo lavorato così
\emph{duramente} per realizzare questo momento, per disfare quello che
avevo fatto e risparmiargli le sue sofferenze.

«E se morirai qui, chi interrogherà la mia gente?» chiese con semplicità
Uvanari. «Hanno bisogno di te per parlare con loro.»

«Troveranno un altro traduttore!» La disperazione mi si insinuò nella
voce. «La tua gente è morta, senza di me.»

«Sono morti comunque, ma non li vedrò soffrire» dichiarò Uvanari.

Scattò di nuovo, con passi irregolari e scomposti a causa della
sofferenza e di tutto il tempo passato su quella croce. Mi gettai da un
lato in modo da mettere il congegno della croce fra me e lui,
costringendolo a rallentare. Anche in una buona giornata, armato com'ero
e senza armatura, lo xenobita mi avrebbe fatto a pezzi perché era più
grosso, più alto, con un allungo maggiore, era dotato di artigli, di
corna e di zanne abbastanza affilati da uccidere e aveva muscoli
rinforzati da procedimenti alieni per resistere al deterioramento
causato dallo spazio. Come potevo sperare di resistere contro tutto
questo con il mio misero coltello? Uvanari cercò di colpirmi con la mano
sinistra, quella che aveva ancora quattro artigli, e io sollevai il
coltello in modo che il suo avambraccio si ferisse da solo sulla mia
lama mentre giravo da un lato e mi abbassavo per passare sotto un
braccio della croce e tenerla fra me e il mio assalitore.

«Dico sul serio» insistetti, a denti stretti. «Li voglio salvare. Voglio
parlare...» Afferrai il braccio della croce e lo torsi verso l'alto,
usandolo per colpirlo in piena faccia prima di saltellare all'indietro.
«Voglio contattare il vostro \emph{aeta}. Tutto quello che ho detto era
vero.»

«\emph{Paiweyu}.» `Non mi importa.'

\emph{Gilliam.} Ricordo chiaramente di aver pensato questo, vedendo
nitido come la luce del giorno l'intus gobbo che mi fronteggiava in
quella cella. \emph{È di nuovo Gilliam.} La mia stupidità, la mia
arroganza mi avevano spinto avanti in un posto e tempo dove non
desideravo essere, e mi avevano derubato della libertà di scelta. Adesso
ero vittima non del fato ma di una sorta di prova logica.

Sibilando come un nido di serpenti Uvanari mi venne contro da sotto il
braccio della croce, con l'arto scuoiato e fasciato premuto contro il
petto e l'altro proteso ad afferrarmi. Cercai di colpirlo, ma la
creatura ritrasse l'arto e il mio coltello sibilò attraverso l'aria,
risuonando contro l'asse posteriore metallica della croce con un impatto
che mi vibrò in tutto il braccio. Lo xenobita mi oltrepassò con passo
pesante con le dita che incontravano rumorosamente la superficie del
carrello degli strumenti per poi ritrarsi prelevando dal suo vassoio di
raffreddamento la mazza con il suo spruzzatore di piombo che riluceva
ancora di un vago chiarore arancione. Imprecando, schivai un colpo
selvaggio, riluttante a rischiare una parata.

Quel selvaggio fendente sgombrò un po' di spazio fra noi e per un
momento rimanemmo fermi a un braccio di distanza, io con il coltello
sollevato in una guardia bassa mentre la creatura armeggiava con la
mazza, spostandone l'impugnatura. «Devi smetterla! La corrente non
rimarrà inattiva a lungo» Eseguii un affondo ma Uvanari mi spinse di
lato il braccio con il suo, che era più lungo. «Loro \emph{torneranno}!»
Il mio avversario calò la mazza verso il basso e per poco non mi
raggiunse a una tempia. La mazza si abbatté su una delle braccia della
croce, piegando l'intelaiatura di metallo all'indietro con uno
scricchiolio. Uvanari non mi rispose. Aveva finalmente svelato il
mistero del piccolo anello di bronzo alla base dello strumento di
tortura che serviva per attivare l'elemento riscaldante nella testa
dell'arma.

Un bagliore arancione simile a quello di un fuoco morente fiorì
nell'oscurità, intagliando rossi punti di luce sul volto contorto
dell'inumano nudo. Solo gli occhi erano scuri, fosse tanto profonde che
attraversavano il pavimento della bastiglia e fendevano l'aria fino a
raggiungere la notte impenetrabile dello spazio. Sorrise, con i denti di
cristallo che scintillavano come ossidiana nella luce {infernale} dello
strumento di tortura e sollevò il braccio per calare la sua arma verso
il basso.

In un duello con la spada si para con la lama, confrontandosi il più
delle volte spada contro spada. Chi combatte con un coltello attacca il
braccio. Avanzai al di sotto della sua traiettoria, afferrandogli il
polso con la mano aperta e torcendolo verso il basso mentre passavo la
lama del coltello lungo l'interno del braccio, come avrebbe fatto un
carpentiere per piallare un pezzo di legno duro. A rigor di logica il
sangue avrebbe dovuto colare nero e abbondante dalla ferita, ma si
limitò a gocciolare.

Uvanari comunque imprecò e indietreggiò verso l'angolo e il carrello
degli strumenti chirurgici. Sfruttai il vantaggio acquisito, ma esitai
per un istante prima di un altro affondo. Intanto lui tornò a calare la
mazza: la mia esitazione mi aveva salvato la vita perché la testa di
piombo mi si abbatté come una pioggia rovente sulla spalla e sul fianco
sinistro, dove avevo sollevato il braccio appena in tempo per
intercettare l'attacco. Il calore mi si diffuse nel fianco e nella
schiena, con il dolore che mi spuntava come erba novella nella carne. Mi
sentii urlare, un suono in qualche modo remoto, come se stesse
echeggiando in un pozzo profondo. Qualcosa mi colpì alla schiena, ma
quel dolore passò in secondo piano rispetto al bruciore, poi mi sentii
scaraventare in un angolo, dove mi abbattei sul carrello. Gli strumenti
chirurgici e per la tortura rotolarono rumorosamente a terra quando
cercai di rialzarmi. L'aria puzzava di fumo e di carne bruciata, e mi
parve che la stoffa della mia camicia ardesse ancora senza fiamma. Il
cadavere lacerato di fratello Rhom giaceva poco lontano, con lo squarcio
aperto nel collo fin troppo visibile in tutta la sua nauseante umidità.
Il suo assassino si erse nudo su di me, con la sua ombra che tremolava
sulle pareti d'acciaio, prodotta dalla luce della mazza incandescente.
Risi, scuotendo la testa mentre mi spostavo carponi con la mano libera
che cercava nel buio qualcosa -- qualsiasi cosa -- che mi proteggesse.

«\emph{Dein}?» Uvanari fece una pausa, cedendo a quell'errore di
curiosità che era il mio più grande peccato. «Cosa stai facendo?» Non
aveva mai sentito un umano ridere, e non sapeva come interpretare quel
suono. Del resto non lo sapevo neppure io. Credo stessi singhiozzando.

Non feci nessuna mossa improvvisa. «Una volta ho visto giustiziare uno
della tua specie» dissi invece. «Mi sono appena reso conto che questa è
la stessa cosa.» La voce mi si incrinò, lacerata dal calore che ancora
mi cresceva nella carne mentre il piombo si raffreddava, indurendosi
come cera contro la mia pelle. La parte della mia mente che era Gibson
sussurrò nel tentativo di ripetermi più e più volte che la paura era un
veleno, ma non avevo bisogno di sentirlo. La paura era scomparsa, si era
consumata. La schiena e il fianco mi pulsavano di dolore dove il metallo
fuso scorreva e si raffreddava sulla mia carne, e quel dolore aveva
cancellato tutto, lasciandomi con la mente limpida e incapace di pensare
ad altro. Arrischiai un'occhiata sopra la spalla e vidi il Cielcin
sollevare l'arma, con l'indice rovinato infilato nell'anello che apriva
la testa per far fuoriuscire le scaglie di piombo che ribollivano
all'interno. Finalmente la mano mi si chiuse intorno a qualcosa di
pesante e sentii le nocche crepitare.

Uvanari grugnì e calò la mazza verso il basso con le fessure che si
aprivano per riversare fuori il loro orrore. Mi contorsi, raccogliendo
quello che avevo afferrato e sbattei il pesante carrello e gli attrezzi
ancora su di esso sul percorso discendente della mazza, spingendola di
lato. Frammenti di piombo sibilarono sul pavimento d'acciaio come soli
caduti, il carrello risuonò come una campana nella mia mano e il tempo
si allungò mentre lo scagliavo a terra e mi rialzavo, insinuandomi al di
sotto dell'allungo del Cielcin. Chiusi la mano sinistra intorno al suo
polso destro e gli conficcai la punta del coltello nell'addome, colpendo
una, due, tre volte. Uvanari emise a stento un suono a ogni colpo, più
sussulti che parole. Con la terza coltellata rigirai la lama e il
Cielcin imprecò, lasciando cadere rumorosamente la mazza riscaldata che
mi sbatté contro una gamba. Avanzai con il coltello ancora piantato nel
suo ventre e lo sbattei contro la parete. L'aria gli sfuggì dalla bocca
e scivolò lungo il muro lubrificato dal suo stesso sangue nero.

«Resta giù» sibilai a denti stretti, resistendo al dolore. Avevo detto
la stessa cosa a Crispin, molto tempo prima. «Stai giù oppure io...»

«Oppure tu cosa?» Potevo sentire il respiro di Uvanari sulla mia faccia
e solo il persistente rigirarsi della lama del coltello gli impediva di
attaccare con quelle orribili zanne. «Mi ucciderai?»

«No!» ringhiai. «Ti lascerò vivere. Fra non molto l'energia si
riattiverà e loro torneranno qui dentro... e cosa credi che faranno
quando vedranno quello che hai fatto a fratello Rhom?»

«Non oseresti!» Gli occhi rotondi di Uvanari si dilatarono maggiormente,
continuando a essere fosse lucenti il quel volto simile a un teschio.
«Non lo faresti!»

«Sì, invece!» Quasi gridai quelle parole. «Mi accerterò che tu viva, lo
giuro sulla Terra.» Adesso mi chiedo se quelle ultime parole abbiano
avuto un qualche significato nella lingua dello xenobita. \emph{Eyudo Se
	ti-Vattaya gin}: `Lo giuro sulla Terra.' Mi domando se per il vecchio
capitano avesse un significato paragonabile a quello che ha per noi
umani. «Farò tutto il possibile per tenerti in vita, \emph{ichakta}.
Però se mi dai qualcosa...» Quella parola rimase sospesa fra di noi,
piena di sottintesi. Sentii i tratti del mio volto che si facevano di
pietra e la mia voce che diventava tanto gelida che fu lord Alistair a
parlare, e non Hadrian. «Dove hai incontrato gli umani? Dove li hai
incontrati prima d'ora? Dimmelo.» Invertii la presa sul coltello e lo
mantenni saldamente piantato contro le sue costole. «\emph{Marerra
	ti-koarin}!»

«\emph{Fusumnu}!» rispose Uvatari, fra un respiro affannoso e l'altro.
`Un mondo'? No, perché il termine per mondo era \emph{fusu'un}.
Aggrottai la fronte.

«Un mondo oscuro?» chiesi, accentuando leggermente la pressione del
coltello.

Lo xenobita grugnì e l'aria gli sfuggì in ansiti sussultanti. «S... sì.
In mezzo!» La parola \emph{vohosum, `}in mezzo', significava
letteralmente `fra le stelle'. Mi trassi leggermente indietro e la mia
presa si allentò un poco quando compresi.

«Gli Extrasolari» dissi in galstani. Forme a stento umane sussultarono
nella mia mente come ragni, uomini che si erano dati completamente alle
macchine demoniache. Tornai ad accentuare la stretta e usai il coltello
per sbattere di nuovo Uvanari contro la parete. «Dove?» Quando non
rispose insistetti: «Dimmelo, oppure farò in modo che comincino a
interrogare Tanaran!» Non dicevo sul serio, non potevo farlo e mi
vergognavo, lieto che la creatura potesse decifrare ben poco della mia
espressione facciale. Uvanari non rispose e cercò di liberarsi
artigliandomi gli occhi. Torsi il coltello verso l'alto, spingendo in
profondità la lama nel suo corpo e facendola strisciare lungo la curva
di una costola. La sua presa si allentò per il dolore mentre gli artigli
mi graffiavano la faccia. Gli sbattei la mano destra contro il pavimento
e sentii le ossa che si spezzavano. Lui urlò, e io gli gridai di
rimando: «Dove?» Calai con violenza la mano aperta contro la parete,
accanto alla sua testa. «Dammi le coordinate, dannazione a te!»

«Non lo so! Allora non ero \emph{ichakta}, ero un bambino!» Qualcuno
picchiava contro la porta, gridando parole strane e soffocate. Da quanto
erano lì? Cosa volevano?

«Un nome, allora? Come si chiamava il pianeta?» Il Cielcin scosse il
capo in quel movimento antiorario che era diventato tanto familiare.
«Come si chiamava, Uvanari.»

Forse fu il suono del suo nome che liberò quella risposta. Nella
creatura qualcosa si ruppe, gli tolse le energie e la fece rattrappire
fino a che sotto di me ci fu un guscio vuoto. Ruotò di nuovo la testa in
un gesto spasmodico, ma non capii se fosse un sì o un no, oppure il
gesto senza senso di qualcosa che stava morendo. I colpi contro la porta
si intensificarono, incarnando la pressione dell'immediato futuro e
della necessità su quell'interminabile presente. «Di' agli altri che la
mia ferita mi ha ucciso» sussurrò Uvanari. «Di' loro qualsiasi cosa, ma
non che è stato così.» Potevo sentire la sconfitta nel suo tono, la
resa. Quasi mi sentii soffocare, perché avevo già rivelato la verità a
Tanaran e agli altri. Risposi con un rigido cenno di assenso, un gesto
che per lui non aveva nessun significato. Poi Uvanari parlò ancora. «Il
mondo? Vorgossos.»

Mi immobilizzai, e la sensazione di forme che mi strisciavano come ragni
nella mente si intensificò. \emph{Vorgossos}... «Vorgossos è un mito!»
Ma... un mito di cui i Cielcin avevano sentito parlare? Di certo un mito
del genere aveva le sue radici nella verità, nel mondo degli atomi e
dell'oscurità. Al di là di qualsiasi minaccia, l'\emph{ichakta} stava
morendo perché quell'ultima torsione del coltello aveva reciso
un'arteria principale o colpito qualche organo vitale. Il sangue mi
colava rovente sulla mano in schizzi veloci, più scuro dell'inchiostro.

Poi le luci si riaccesero proprio mentre Uvanari, ormai debole,
mormorava: «Vorgossos.»

«Non è reale» affermai, incapace di dire altro. «Vorgossos non esiste.»

Alle mie spalle la porta si aprì con un sibilo e cedetti al panico,
liberando bruscamente il coltello e segnando profondamente una costola
mentre altro sangue si riversava sulla carne bianca come l'oscurità fra
le stelle. Mentre i legionari si riversavano nella stanza barcollai
all'indietro e mi accasciai sul pavimento ai piedi della croce.

Il mio coltello aveva fatto il suo lavoro. Uvanari morì prima che anche
un solo soldato potesse raggiungerlo.

