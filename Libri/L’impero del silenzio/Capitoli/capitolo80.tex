\chapter{Indice dei mondi}

\emph{Una nota di astrografia}

Nel leggere il resoconto di lord Marlowe appare chiaro che aveva
un'estrema familiarità con il panorama astropolitico del XVII millennio.
Fa riferimento a parecchie dozzine di mondi colonizzati mentre la
maggior parte della popolazione comune del tempo non avrebbe conosciuto
più di una manciata di quei nomi. Questa però non è una sorpresa,
considerato il suo addestramento nel campo della diplomazia
interstellare.

A tutt'oggi gli sforzi coloniali umani hanno inseminato più di mezzo
miliardo di questi mondi abitabili distribuiti su cinque braccia della
nostra galassia. Quasi la metà di questi deve fedeltà all'Impero Solano,
mentre il rimanente è distribuito fra sistemi di governo più piccoli
come il Commonwealth lothriano, i principati di Jadd, la Repubblica
Durantina, la Demarchia di Tavros, nonché in microstati come le
proprietà comuni tanto nel Velo Normanno di Marinus quanto nel Perseo
Esterno. Fin dai tempi dell'Esodo e dei primi Vagabondaggi dal sistema
della Vecchia Terra prima della caduta dei Mericanii, gli sforzi
coloniali umani hanno seguito due direzioni generali, la prima verso il
nucleo della galassia, attraverso il Braccio del Sagittario e quello di
Centaurus, e attraverso il Golfo Cupo fino a Norma; la seconda verso il
confine della galassia, dallo Sperone di Orione fin dentro il Braccio di
Perseo, dove una rapida espansione ha portato alla separazione degli
Stati jaddiani.

In senso ampio, l'Impero Solano è diviso in quattro primarcati: uno in
Orione, uno in Perseo, uno nel Sagittario e più di recente uno in
Centaurus. Gli sforzi per stabilire un quinto primarcato in Norma sono
stati intralciati dall'invasione dei Cielcin, e in seguito alla
battaglia di Gododdin e dalla morte dell'imperatore William XXIII Avent,
le colonie normanne sono rimaste non incorporate e sotto la supervisione
del primarca centaurino di Nessus. I principali primarchi vengono
nominati dall'imperatore, che è lui stesso primarca di Orione. Ciascuno
dei primarcati imperiali è diviso in parecchie provincie. Nominalmente,
il titolo di viceré viene assegnato, ma nella pratica la vicereggenza di
molte provincie passa di genitore in figlio, come fanno i titoli
palatini di arciduca, granduca, duca, marchese, conte, visconte e
barone. Ciascuno di questi nobili governa un palatinato che comprende il
suo dominio planetario e i relativi sistemi solari. La differenza fra un
barone e un duca, per esempio, è determinata dall'importanza relativa
del palatinato in questione. Gli arconti servono questi nobili palatini
amministrando i distretti dei suddetti domini e possono trasmettere le
loro tenute ai figli, anche se molti arconti vengono nominati e sono
poco più che governatori. Un arconte tale per nomina, come è il caso di
un console coloniale, governa un feudo e non un dominio, in quanto i
suoi diritti al suddetto territorio si estinguono nel caso della sua
morte e non vengono trasmessi ai figli.

Per la comodità del lettore, ho anche aggiunto questo indice dei pianeti
menzionati dal magio nel suo resoconto, fornendo una breve descrizione
di ciascuno, quanto basta per comprendere il significato dei riferimenti
fatti da lord Marlowe.

\emph{Tor Paulos di Nov Belgaer}

\textbf{Andun}: pianeta imperiale nel Braccio del Sagittario, noto per
le sue strutture e per la costruzione di astronavi.

\textbf{Ares}: uno dei più antichi mondi imperiali dello Sperone di
Orione, sede della Scuola di Comando Ares, dove vengono formati i
migliori ufficiali della Legione.

\textbf{Armand}: una delle due lune di Emesh, piccola e bianca. Ha
alcuni insediamenti sotterranei ed è governata dal Casato Kvar. Prende
il nome da lord Armand Mataro, che tolse Emesh ai Normanni.

\textbf{Ascia}: un pianeta nel Lothrian.

\textbf{Asherah}: mondo arido e attivo tettonicamente, noto per le sue
montagne e per una specie di grandi creature rettiliane.

\textbf{Avalon}: una delle colonie umane originali e sede di una
massiccia colonizzazione europea da parte di un'arca generazionale.
Mondo natale dell'Impero Solano.

\textbf{Bellos}: colonia imperiale nella Distesa Normanna, sito della
battaglia di Bellos.

\textbf{Binah}: una delle due lune di Emesh, governata dal Casato
Melluan. Un caso di precoce terraformazione coronata dal successo. La
sua superficie è verde per una massiccia crescita di alberi.

\textbf{Cai Shen}: mondo dell'Impero di etnia mandari, famoso per il suo
Tempio di Bashang. Era governato dal Casato Min Chen fino
all'\foreignlanguage{italian}{isd} 16130, quando è stato distrutto dai
Cielcin.

\textbf{Colchide}: la prima colonia imperiale nel Braccio di Centaurus.
È una luna del gigante gassoso Atlante. Non è mai stata una colonia
importante ma è comunque nota per l'enorme ateneo scoliasta a Nov
Belgaer.

\textbf{Cressgard}: colonia imperiale perduta nel Velo di Marinus. Sede
del primo contatto con i Cielcin nella battaglia di Cressdard
nell'\foreignlanguage{italian}{isd} 15792.

\textbf{Delos}: luogo di nascita di Hadrian Marlowe e sede del ducato
del Casato Kephalos nello Sperone di Orione. È un mondo temperato con
una luce solare debole, noto per i suoi depositi di uranio che lo hanno
reso estremamente ricco.

\textbf{Durannos}: capitale della Serena Repubblica dei Durannos. Morta
dal punto di vista tettonico e paludoso, molta della sua superficie è
coperta da un'enorme città.

\textbf{Edda}: un mondo arido e ventoso della Demarchia, noto per le sue
gole, le doline e gli oceani sotterranei. Il suo popolo è composto
primariamente da Nordei e da Travatskr.

\textbf{Emesh}: mondo acquatico nel Velo di Marinus. Sede del Casato
Mataro e mondo dei coloni umandh, su di esso si trovano le rovine
sotterranee di Calagah. In origine era una colonia normanna.

\textbf{Forum}: capitale dell'Impero Solano. È un gigante gassoso con
un'atmosfera respirabile nella cui cintura di nuvole ci sono parecchie
città di palazzi volanti che servono come centri amministrativi
dell'Impero.

\textbf{Gododdin}: un sistema fra il Braccio di Centaurus e quello del
Sagittario, notoriamente distrutto da Hadrian Marlowe nella battaglia
finale della Crociata.

\textbf{Helvetios}: antico nome della stella 51 Pegasi, il cui pianeta
principale è Bellerofonte.

\textbf{Ilio}: una Proprietà Normanna famosa per la fabbricazione di
astronavi. Attualmente sono considerati i migliori in questo campo.

\textbf{Jadd}: il pianeta del fuoco e sacra capitale dei principati
jaddiani, il cui suolo nessuno può calpestare senza l'esplicito permesso
dell'Alto Principe.

\textbf{Giudecca}: un mondo gelido e montuoso nel Braccio del
Sagittario, famoso per essere la sede del Tempio di Athten Var e luogo
di nascita della specie degli Irchtani. Sito dove si svolse la famosa
lotta di Simeon il Rosso contro gli ammutinati.

\textbf{Kandar}: mondo agricolo estremamente ricco nella parte interna
dell'Impero, considerata di frequente la fonte definitiva dei prodotti
di lusso, soprattutto di vino e bestiame genuino. Sede del Casato
Markarian.

\textbf{Komadd}: colonia imperiale nel Velo di Marinus e sede di un
seminario della Cappellania.

\textbf{Linon}: luna di un gigante gassoso nel sistema di Delos. In
precedenza dominio dell'esule Casato Orin e sito della battaglia di
Linon dell'\foreignlanguage{italian}{isd} 15863 nel quale Alistair
Marlowe uccise l'intero Casato.

\textbf{Malkuth}: pianeta imperiale che porta i segni di una civiltà
xenobita estinta.

\textbf{Marinus}: prima Proprietà Normanna conquistata dall'Impero e una
delle prime colonie nella Distesa.

\textbf{Marte}: uno degli antichi pianeti fratelli della Vecchia Terra e
sede dei primi importanti sforzi coloniali durante l'Esodo ma prima dei
Vagabondaggi.

\textbf{Mira}: pianeta in orbita intorno alla stella omonima dello
sperone di Orione. Naturalmente privo di aria, le sue città sono coperte
da una cupola o sigillate ermeticamente. La sua superficie contiene
quantità enormi di metano ghiacciato che viene estratto.

\textbf{Monmara}: mondo acquatico e Proprietà Normanna, noto per
l'economica produzione in serie di astronavi.

\textbf{Neruda}: vecchio mondo imperiale nel Braccio del Sagittario,
noto per la sua flora simile a funghi. È la sede di un importante ateneo
scoliasta.

\textbf{Nessus}: sede del Primarcato centaurino.

\textbf{Nichibotsu}: più importante fra i mondi nipponesi e sede
dell'arciduca del Casato Yamato.

\textbf{Obatala}: colonia imperiale nel Braccio del Sagittario che si
trova lungo le rotte commerciali fra Delos e Teukros.

\textbf{Ozymandias}: vecchia colonia imperiale nel Braccio del
Sagittario, è un mondo arido, noto per essere stato la dimora degli
estinti arcicostruttori e per gli enormi archi di pietra che si suppone
siano stati lasciti da loro.

\textbf{Perfugium}: mondo imperiale nel Braccio di Centaurus usato come
base militare e campo profughi nelle parti conclusive della Crociata.

\textbf{Rinascimento}: uno dei mondi più popolosi dell'Impero, situato
nello Sperone di Orione. Centro culturale coperto quasi interamente da
sviluppo urbano.

\textbf{Sadal Suud}: mondo selvaggio nello Sperone di Orione che è
perlopiù libero da vincoli. Pianeta natale dei Giganti Cavaraad, una
specie di enormi xenobiti e sede delle Torri che Marciano, una delle
Novantanove Meraviglie dell'Universo. Governato dal Casato Rodolfo.

\textbf{Se Vattayu}: mitico mondo natale dei Cielcin. Pare che la sua
superficie sia crivellata da un labirinto di gallerie simili a quelle
che la Quiete ha scavato a Calagah, su Emesh.

\textbf{Siena}: colonia imperiale nel Braccio del Sagittario che si
trova lungo le rotte commerciali fra Delos e Teukros.

\textbf{Sulis}: una Proprietà Normanna che è in effetti uno Stato
cliente dell'Impero Solano. Luogo dove è stata respinta un'invasione dei
Cielcin.

\textbf{Syracuse}: mondo temperato dell'Impero, famoso per essere la
sede dell'ateneo scoliasta di Nov Acor.

\textbf{Teukros}: mondo desertico dell'Impero, noto come sede
dell'ateneo scoliasta di Nov Senber.

\textbf{Thessaloniki}: sede degli Ottanta principati di Jadd.

\textbf{Triton}: mondo acquatico dell'Impero, sede del Casato Coward.

\textbf{Ubar}: un'arida satrapia jaddiana fedele al Casato di Otranto,
che è stata per generazioni quella degli alti principi di Jadd. Fedele
al principe di Thessaloniki.

\textbf{Uhra}: Proprietà Normanna che di recente è passata dall'essere
un regno al diventare una repubblica. Famosa per la fabbricazione di
astronavi.

\textbf{Vecchia Terra}: luogo di nascita della specie umana. Rovinata da
guerre nucleari e vittima di un collasso ambientale, è protetta dai
guardiani della Cappellania e nessuno vi può andare.

\textbf{Vesperad}: una luna che orbita intorno al gigante gassoso Ius. È
il più antico pianeta controllato dalla Cappellania e vanta il più
grande dei suoi seminari.

\textbf{Vorgossos}: mitico mondo extrasolare che si dice essere dominio
di demoni e di pirati.

\textbf{Wodan}: colonia imperiale nel Velo di Marinus, tolta ai
Normanni. Sito della battaglia di Wodan
dell'\textit{ISD} 16129.

\newpage\blankpage