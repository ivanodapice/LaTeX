\chapter{Insegnare loro come combattere}

Sferragliando, l'ascensore portò su venti di noi per il primo evento del
Colosso di quel giorno, con ciascun combattente che sudava e puzzava in
quello spazio ristretto. Switch mi era accanto e borbottava fra sé una
preghiera, un mantra con cui invocava l'icona del Coraggio della
Cappellania. «Benedicimi con la spada del tuo ardimento, o Coraggio»
sussurrava con voce appena udibile. «Concedimi la forza in questo
momento di bisogno. Benedicimi con la spada...» Chiusi gli occhi. Il
coraggio è la prima virtù degli stolti, il patrono di quanti hanno
troppa paura per fuggire.

Il mirmidone che si trovava dall'altro lato di Switch gli diede una
gomitata. «Piantala, d'accordo?»

Switch lo guardò e borbottò qualche parola di scusa. Io feci una
smorfia, assestando lo scudo rotondo in vecchio stile che avevano
fornito a me e a tutti gli altri, un oplon di fibra di carbonio lungo
tre piedi. Nonostante tutti i miei incoraggiamenti, Switch aveva ragione
riguardo a sé stesso -- ci sarebbe voluto più di una settimana per farne
un combattente. E Ghen non si sbagliava, qualsiasi cosa dicessero Siran
e Kiri: il ragazzo non sarebbe durato un nanosecondo. Serrai i denti,
ricacciando indietro un rimprovero mentre il vecchio e sibilante sistema
di altoparlanti dell'ascensore emetteva un acuto stridio e la voce rude
di Pallino risuonava sopra la nostra testa «State uniti, come ci siamo
esercitati a fare. Gruppi di cinque. Non permettete al nemico di
circondarvi.»

«Sappiamo come sono armati, capo?» chiese Keddwen, un ragazzo locale che
aveva superato alcuni combattimenti e che si {distingueva} per i capelli
sbiancati e filacciosi. Per farsi sentire con la sua voce roca dovette
gridare.

«Hanno le stesse cose che abbiamo noi» gridò di rimando Pallino. «Spade,
lance e scudi rotondi. Si tratta però della squadra di Jaffa, quindi mi
aspetto che quegli stronzi scaglino dei giavellotti.»

«Allora glieli lanceremo indietro!» esclamò Siran, chiamando a raccolta
abbastanza energia da indurre gli uomini alla sua estremità della cabina
ad applaudire, sollevando le armi nella cupa luce arancione.

Percependo il nervosismo di Switch, mi protesi verso di lui e battei un
colpo sul suo scudo rotondo. «Almeno abbiamo uno scudo, giusto?»

Lui fece una smorfia e si assestò l'elmo sulla sua selvaggia criniera
rossa. «Non sei divertente, Had.» Lo capivo perfettamente, dato che
avrei dato il braccio sinistro per un vero e proprio scudo Royse.

Prima di ognuno dei miei combattimenti nel Colosso, c'era un momento,
appena prima che avanzassimo sui mattoni coperti di sabbia dell'arena,
in cui tutto quello che volevo era essere altrove. Da qualsiasi altra
parte. Allora lo provai per la prima volta. Le viscere mi si contorsero,
annodandosi, il sangue prese a martellare contro l'incudine del mio
cranio e fissai le travi d'acciaio che supportavano il soffitto ad arco
al di sopra del condotto dell'ascensore, contando gli enormi bulloni che
le tenevano al loro posto. Perché lo stavo facendo? Ci dovevano essere
altri modi per guadagnarsi un passaggio per lasciare il pianeta. È per
la tua nave, dissi a me stesso, liberando l'immaginazione. Allora
\emph{avrei potuto} lasciare l'Impero, andarmene per non tornare mai
più, perdere la Cappellania, mio padre e tutto il resto. Sarei potuto
diventare un mercante nel Perimetro Esterno. Un pirata. Un mercenario.

Però i pochi hurasam che avevo rubato in quel negozio d'angolo non
avrebbero potuto permettermi di comprare un ripostiglio per le scope a
bordo di un incrociatore stellare, e comunque ero sotto contratto al
colosseo per un anno standard, sessantacinque combattimenti, ciascuno
dei quali poteva essere letale, e qualsiasi tentativo da parte mia di
infrangere il contratto e di fuggire sarebbe finito con i miei piedi
tranciati, il naso tagliato e il mio corpo trascinato nel Colosso alla
mercé delle grandi bestie, cibo nel senso più letterale della parola.
Gli sponsor -- non ultimo il conte Balian Mataro -- avrebbero ricavato
da me quello che costavo in un modo o nell'altro.

Là in piedi cercai di controllarmi mentre la vescica della mia vicina si
svuotava lungo la sua coscia, e i miei pensieri si concentrarono sul
nemico. I nostri avversari avrebbero avuto scudo e armatura; un vero
scudo e non quei pezzi di antiquariato che avevamo noi. Eravamo vestiti
come se quella che intendevamo assalire fosse stata la Troia di Omero e
non cinque gladiatori che disponevano di un'armatura avanzata dotata di
sensori. L'intera cosa era una farsa. Le nostre armi smussate potevano
solo ammaccare la loro armatura, che avrebbe interpretato i colpi in
maniera matematica e inibito chi la portava in una simulazione di danno
senza causare ferite effettive. I veri gladiatori morivano di rado,
perché simili atleti professionisti costituivano un investimento troppo
grande. Noi potevamo solo immobilizzarli.

«Shock e timore, Hadrian» mormorai a me stesso, massaggiandomi gli occhi
con le nocche e passandomi le dita in mezzo all'ispida ricrescita dei
miei capelli. «Sangue e tuono.» Un gemito mi sfuggì fra i denti. Sentivo
la mancanza di Gibson, di Roban, perfino di Felix. Il castellano avrebbe
saputo cosa fare, avrebbe pensato ad almeno una dozzina di vie di uscita
-- un centinaio -- qualsiasi cosa ci fossimo trovati di fronte. Di colpo
ebbi la sensazione che avrei dovuto prestare maggiore attenzione alla
tattica, che non avevo appreso tutto quello che dovevo e adesso non lo
avrei mai fatto. Felix era a centinaia di anni luce di distanza e
Gibson... ecco, solo mio padre e la Madre Terra sapevano dove fosse. Mi
presi un momento per assestarmi l'elmo, stringendo il sottogola.

L'ascensore si fermò con un sussulto e quasi immediatamente le enormi e
pesanti porte si aprirono con uno stridio di metallo contro la pietra.
Giuro che il rumore ci investì prima della luce del sole, una grande
tempesta risonante profonda e schiacciante come il mare. Era il suono
animalesco di ottantamila gole che urlavano e gridavano con piacere da
ubriachi e una furia gioiosa. Quel suono influenzava ciascun uomo in
maniera diversa, atterrandolo o esaltandolo.

La paura è veleno, dissi a me stesso, ripetendo le antiche parole come
Switch aveva fatto con la sua preghiera: `La paura è veleno.' Sentivo
quel veleno che mi scorreva come ghiaccio nelle vene. Seguii Ghen e i
veterani sotto la malsana luce arancione del sole, soppesando lo scudo
in fibra di carbonio mentre estraevo la corta spada in previsione del
combattimento. Il pavimento dell'arena era lungo quasi un centinaio di
metri ed era largo forse la metà. Noi eravamo sbucati in mezzo a una
foresta di colonne di pietra distribuite a casaccio in tutta l'arena per
offrire copertura ai combattenti. Non ce n'erano due uguali, l'altezza e
il diametro erano variabili, ma sporgevano tutte dal pavimento di
mattoni, intasando quello che altrimenti sarebbe stato un piatto terreno
aperto.

Tutt'intorno alla grande distesa dell'arena sorgeva un muro di arenaria
alto venti piedi, trapassato a intervalli dalle porte d'acciaio di
ascensori come quello da cui eravamo sbucati. Uno scudo a energia di
classe elevata scintillava sopra di noi, smorzando una porzione degli
applausi, delle grida e dei fischi. La sua presenza serviva a proteggere
la folla dal fuoco delle armi, come in quel giorno in cui avevo visto
trenta schiavi che venivano massacrati dai Diavoli di Meidua. Un raggio
al plasma vagante non avrebbe potuto tagliare il muro e colpire qualche
logoteta o funzionario di corporazione.

Niente di tutto questo mi si registrò nella mente. Lo spartano campo di
battaglia non era tanto uno spazio morto quanto una distesa in primo
piano rispetto alla fitta giungla di colore e di movimento costituita
dalla folla, distorta e indistinta a causa del campo Royse. E loro erano
là, schierati vicino a un cancello all'estremità del campo: cinque
gladiatori che portavano l'armatura dei legionari imperiali, con le
piastre di ceramica bianca dipinte di verde e oro. Ciascuno impugnava
una lancia più lunga di quanto io fossi alto e ci guardava attraverso
una visiera di un verde compatto, priva di giunti e senza dettagli o
fessure per gli occhi. Erano pazienti come pietre coperte di muschio.

«Lo scudo è alzato» dissi, afferrando Switch per il tricipite mentre
avanzavamo tutti in fretta sul campo.

In lontananza, come se la sentissi dal fondo di un pozzo profondo,
potevo cogliere le parole indistinte del maestro di cerimonie del
Colosso, anche se ciò che disse si perse negli applausi della folla e
nell'effetto offuscante del campo. Comunque noi non avevamo bisogno di
sentire. Sapevamo perché eravamo lì.

Switch si protese verso di me. «Che cosa hai detto?»

La mia attenzione era divisa, metà sui gladiatori che avevamo di fronte,
cinque delle Sfingi di Borosevo in equipaggiamento completo, e l'altra
metà sul palco del lord nel centro di un muro e sull'uomo che lo
occupava, dietro un numero ancora maggiore di strati di schermatura.
Balian Mataro, conte di Emesh. Lo avevo visto nei notiziari trasmessi
dagli schermi cittadini, ma adesso era là in carne e ossa, seduto fra i
suoi littori e i suoi schiavi umandh. Mentre entravamo, un paio di
quelle creature aliene stava aiutando i suoi douleter umani a trascinare
la carcassa di un qualche predatore terrestre cefalopode oltre una porta
laterale e su un altro ascensore, lasciando sui mattoni una scia di
sangue verde. Dopotutto, non eravamo il primo evento della giornata.
Indussi Switch a tenersi indietro e attesi che Kiri e Pallino ci
raggiungessero con una recluta che non conoscevo.

Accennando verso l'alto con un pollice, ripetei: «Hanno usato la
precauzione di attivare lo scudo.»

Pallino lo scrutò con il suo unico occhio. «Oh, che io sia fottuto»
imprecò, proprio quando il primo colpo raggiungeva i mattoni sopra la
nostra testa e il plasma violetto strinava di nero il muro. Da un lato,
Banks stava formando il suo gruppo di cinque, tre lancieri e due uomini
con la spada che si tenevano tutti accoccolati per sfruttare al massimo
i loro scudi di fibre di carbonio.

«Abbassatevi!» gridai, allungando un braccio su Pallino e Switch per
trascinarli a terra con me mentre una raffica arroventava l'aria sopra
la nostra testa. «Questi arresteranno le lance al plasma?» domandai,
puntellando lo scudo in verticale davanti a me per darmi un po' di
copertura mentre rimanevo disteso nella polvere.

Pallino grugnì. «Sì, ma non ci fare affidamento a lungo.»

«Allora non mettiamoci molto, signore» ribattei. Kiri e la recluta
corsero in avanti, dopo aver in qualche modo evitato la scarica che per
poco non ci aveva uccisi, e ci aiutarono a rialzarci tenendo alto lo
scudo. «Da questa parte!» Indicai una delle molte colonne che spuntavano
dal pavimento dell'arena, sperando di mettere un ostacolo fra noi e i
gladiatori con le lance a energia.

«È un abbattimento selettivo!» Pallino sputò nella polvere mentre ci
riparavamo dietro la colonna. Quando mi limitai a guardarlo senza capire
accennò con il mento in direzione degli altri mirmidoni. «Sfoltiscono la
mandria.» Switch era sprofondato in un silenzio mortale. Pallino mi
afferrò con la mano libera. «Avrei dovuto ucciderti quando ne avevo la
possibilità.»

La mente mi si svuotò. «Me?» farfugliai. «Nel nome della Terra, questo
cosa c'entra con me?»

«Hanno sentito quel tuo discorsetto nello spiazzo, devono averlo fatto.»
Mi lasciò andare ed estrasse di nuovo la spada. «Ci stanno
ridimensionando.»

Quella era una minimizzazione. Stavamo per essere massacrati. Cercai di
rallentare gli eventi, di ragionare. Di respirare. «La paura è un
veleno» mormorai, nel tentativo di calmare il battito veloce del mio
cuore ribelle. Numericamente, eravamo superiori ai gladiatori nella
misura di quattro contro uno, ma le loro armi erano di millenni
superiori alle nostre e sapevo già che il piano originale sarebbe
fallito. Se ci fossimo stretti in piccole unità di combattimento saremmo
di certo morti tutti. «Ci dobbiamo dividere, cercare di isolarli.»
Sbirciai da dietro la colonna e vidi Siran e Ghen bloccati dietro un
altro di quei massicci pilastri. C'era una possibilità. Senza dubbio
quelle colonne non erano una parte standard della costruzione del
colosseo, perché sarebbero state d'intralcio nelle corse di cavalli e di
cani comuni in posti del genere e in alcune forme di combattimento.
«Switch, resta con me, voialtri andate a sinistra, tenendo le colonne
fra voi e loro.»

«Da quando sei tu a comandare?» chiese la recluta, quell'uomo che non
conoscevo.

Il fato scelse quel momento per interferire e una scarica al plasma lo
raggiunse in pieno petto, strinandomi i vestiti al suo passaggio,
accompagnato dal puzzo chimico del metallo fuso e della carne umana
cotta. L'uomo non ebbe neppure la possibilità di gridare. Io avrei
voluto gridare, piangere, fare qualcosa, qualsiasi cosa, ma tutto quello
a cui riuscivo a pensare era: A cosa serve tutta questa armatura,
allora? Spinsi Switch con lo scudo, in avanti e lontano da chi mi aveva
risparmiato il disturbo di rispondere alla domanda della recluta. Al di
sopra dell'arena, la voce acuta del maestro di cerimonie del Colosso
narrava gli eventi parlando a raffica, ma le sue parole si perdevano
attraverso lo scudo protettivo.

Cinque. Avevano mandato cinque uomini a ucciderci. Cinque soldati
giocattolo professionisti nella loro elegante armatura, {ciascuno} senza
dubbio pronto a uccidere, con il sangue carico di supplementi di
testosterone. Cinque Crispin. Quel pensiero per poco non mi fece
affiorare un sorriso sulle labbra mentre in parte spingevo e in parte
trascinavo Switch dietro a un'altra colonna. Dovevamo mettere dello
spazio fra noi e l'uomo che aveva ucciso la recluta. Pallino e Kiri non
mi avevano seguito, dirigendosi da un'altra parte. Non ho mai saputo se
avessero accolto il mio suggerimento o fossero stati trascinati via dal
caos. Nella mia memoria quel primo giorno è sospeso come in una nuvola,
con alcuni momenti evidenziati da una luce argentea e radiosa, e altri
nell'ombra, bruciati, come se quelle colonne grigie fossero state fumo e
il clamore dell'arena un distante tuono primitivo.

«Dividetevi!» gridai, vedendo un capannello dei nostri mirmidoni
accalcato dietro le colonne. «Gruppi di due o tre! Allargatevi a
ventaglio!» Non rimasi a vedere se mi davano ascolto. Qualcuno urlò e
pensai: Adesso siamo in diciotto. Mi augurai di avere ragione, che nella
fretta non mi fosse sfuggita un'altra morte. Al mio fianco, Switch si
era fatto quasi catatonico, paralizzato, con gli occhi dilatati per la
paura. Lo scrollai. «Riprenditi!» Lo spinsi con lo scudo contro la
colonna. «Mi servi qui!» I suoi occhi si misero lentamente a fuoco e io
assestai un pugno al muro, accanto alla sua testa, con la mano che
stringeva la spada. «Se vogliamo uscire da questa situazione ho bisogno
di te.»

«Uscirne? Come?» ribatté, guardandosi intorno.

«Ho un piano.» Era una menzogna solo per metà perché avevo un'idea. Una
sensazione.

La voce di Banks sovrastò il fragore. «Voi due! Con noi!» Il veterano
dalla faccia simile al cuoio era insieme a un gruppo di sei dietro la
colonna più larga del campo.

Scossi il capo. «Dovete dividervi, Banks! Dobbiamo prenderli sui
fianchi.»

«Cosa?» urlò di rimando, aggrottando le sopracciglia sotto l'elmo. «Sei
impazzito?»

«\emph{Loro} stanno combattendo insieme?» urlai, calcando sul pronome
per maggiore enfasi. «Volete allinearvi perché vi facciano fuori o
volete contrattaccare?» In quel momento avvistai una dei gladiatori, una
donna alta nella caratteristica armatura verde e oro, mentre passava fra
due colonne. In quel caos non ci aveva visti e teneva la lancia spianata
nell'inseguire un altro bersaglio. Senza aspettare, trascinai con me
Switch fino alla colonna successiva, seguendo i suoi passi.

Banks la vide e ringhiò, imitandomi con due dei suoi mirmidoni che si
appostarono dietro una colonna opposta alla nostra, sporgendosi per fare
il punto della situazione. La gladiatrice si fermò per un momento con la
baionetta all'estremità della lancia abbassata mentre armeggiava con
qualche comando sull'impugnatura ed espelleva un dispersore di energia
con un rumore metallico. «Spero che tu sappia quello che stai facendo»
ringhiò Banks.

«Io attirerò il suo fuoco. Tu colpiscila alle spalle.»

«Alle spalle, eh?» Banks snudò i denti in un sorriso ferino.

Sentii la mia faccia perdere ogni espressione. «Cresci.» Mi rivolsi a
Switch. «Resta dietro di me, capito?» Il ragazzo annuì, poi guardò la
spada che aveva in mano come se fosse stata un tumore letale e non
rispose, limitandosi a continuare a fissare l'arma. Imprecai dentro di
me. Ghen aveva avuto ragione su di lui. Per la seconda volta lo spinsi
contro il pilastro. «Non devi combattere. Bada solo a restare con me.»
Non lo aspettai e spiccai la corsa attraverso l'arena di mattoni, con il
rumore prodotto dall'armatura che veniva soffocato dal fragore della
folla. Potevo solo sperare che la donna non mi sentisse.

La gladiatrice aveva la lancia sollevata e puntata contro un qualche
bersaglio alla mia sinistra, con l'impugnatura infilata nel cavo del
braccio e l'arma al livello degli occhi nel prendere la mira. Con il
sangue che mi martellava nelle orecchie più stentoreo della folla, calai
di piatto la spada sulla lunghezza della lancia mentre lei faceva fuoco.
Il plasma violetto colpì i mattoni ai nostri piedi, trasformando i
silicati presenti nella sabbia in vetro fuso e riempiendo l'aria di fumo
chimico. La donna emise un verso di sorpresa e si girò di scatto,
cercando di raggiungermi alla faccia con l'impugnatura della lancia.
Reagii calando con forza lo scudo e l'impatto produsse un risonante
rumore sgradevole mentre l'impugnatura mi scuoteva la mano. Feci una
smorfia ma continuai a incalzarla, costringendola a indietreggiare verso
Banks e i suoi. Switch era svanito da qualche parte in quella follia.
Serrai con forza i denti e colpii con la spada.

L'arma raggiunse la gladiatrice a una gamba e sentii uno stridio
artificiale. Lei barcollò nel muovere il successivo passo all'indietro e
cercò di puntarmi contro l'arma, che però era troppo vicina e una
pessima alternativa in uno scontro a distanza ravvicinata. E adesso
l'armatura l'aveva tradita, registrando il mio colpo alla gamba come
danno da battaglia pur proteggendo la sua preziosa carne con una spessa
imbottitura. Emise un ringhio rabbioso attraverso gli altoparlanti nel
collo della sottostante tuta aderente e lasciò cadere la lancia,
portando la mano al lungo coltello che aveva al fianco. Era una mossa
coraggiosa e sensata, e avrebbe potuto funzionare contro Switch, e
perfino contro Kiri. Forse avrebbe potuto funzionare anche contro di me,
considerato quanto erano tesi i miei nervi.

Banks però le si abbatté addosso con uno dei suoi mirmidoni e una lama
trapassò il suo scudo di energia -- che non venne attivato a causa della
lentezza umana dell'attacco -- risuonando contro il suo elmo. Lei
imprecò orribilmente quando l'armatura color giada si bloccò,
eliminandola dallo scontro. Pensai alla mostruosità cefalopode che gli
schiavi umandh avevano trascinato fuori dal campo, e immaginai quegli
xenobiti a tre gambe mentre trascinavano quella donna fino a un
ascensore laterale per far sbloccare la sua armatura. In alto la folla
emise un ruggito, al di sopra del quale risuonarono le parole indistinte
del presentatore che narrava tutto quanto.

«D'accordo, useremo il tuo piano» dichiarò Banks, sorridendo come un
idiota.

«Ne rimangono ancora quattro!» disse il suo compagno più vicino, la
donna con la faccia che si spelava. «Possiamo usare quella?» Indicò la
lancia che la gladiatrice stringeva ancora fra le mani paralizzate.

Il veterano scosse il capo. «Sono fatte per interfacciarsi con i guanti
dell'armatura. Se ci proviamo non faranno fuoco.»

«Ma abbiamo la superiorità numerica.» Allungai il collo. «Dov'è Switch?»

«Il prostituto?» La donna scrollò le spalle. «Non ne ho la minima
fottuta idea.»

La oltrepassai e tornai in fretta da dove ero venuto, chiamando il
ragazzo. Ben presto però tacqui, rendendomi conto che anche i nostri
cacciatori avrebbero avuto modo di sentirci. Oltrepassai il corpo di due
dei nostri che fumavano sui mattoni... uno era quello che avevo visto
morire prima? O erano entrambi nuovi? Allora eravamo in sedici? O in
diciassette?

«Ehi, Vostra radiosità!» La voce profonda di Ghen sovrastò il vociare
della folla. «Da questa parte!» E si accoccolò per superare lo spazio
fra due colonne, nascondendo il più possibile la sua mole dietro allo
scudo al carbonio.

Con la schiena addossata a un pilastro attesi che si avvicinasse,
seguito da Siran. I due detenuti si appiattirono accanto a me proprio
mentre qualcuno urlava in lontananza. La parte di me che pensava con la
voce di Gibson eliminò un altro elemento, impassibile. \emph{Quindici?}
«Banks e io ne abbiamo preso uno,» dissi, stando in guardia da un nuovo
attacco «ma ho perso Switch.»

«L'ho visto con Keddwen ed Erdro, appena un secondo fa» replicò Siran.
«Pare che stia per farsela addosso.»

«Ho sentito dire che qualcuno lo ha già fatto» ribatté Ghen,
sogghignando.

Scossi il capo. «Non è stato Switch. È solo spaventato. Qual è il
conteggio?»

Siran scrollò le spalle. «Credo abbiano beccato quattro dei nostri.» Si
massaggiò il naso rovinato con il dorso della mano in cui stringeva la
spada, poi abbassò la testa per guardare oltre Ghen.

«Io ne ho contati almeno cinque» replicai, cupo. «Non possiamo perdere
cinque persone per ciascuno dei loro. Banks e io ne abbiamo
neutralizzato uno da soli. Incudine e martello.»

Ghen annuì. «Mi sorprenderebbe se Pallino non ne avesse già beccato uno.
Ormai sono cinque anni che quel bastardo se la vede con le Sfingi.»

«Avanti, muoviamoci» dissi.

Risultò che il grosso mirmidone aveva avuto ragione perché oltrepassammo
il corpo paralizzato di un secondo gladiatore che sedeva a ridosso di
una colonna non molto lontana. Il cuore mi si fece più leggero, ma poi
sprofondò quando trovammo i corpi. Uno era decisamente quello di
Keddwen, il ragazzo locale con i capelli filacciosi che chiamava Pallino
`capo'. L'altro non era quello di Switch ma di una ragazza emeshi che
somigliava troppo a Cat per i miei gusti. \emph{Ne restano solo tre.}

Ci imbattemmo nel corpo di altri tre dei nostri prima di trovare gli
altri gladiatori -- tutti uomini -- disposti schiena contro schiena in
una compatta triade da legionari. Avevo sentito parlare di una cosa del
genere e l'avrei vista innumerevoli volte su molti campi di battaglia:
gli orgogliosi soldati al servizio dell'imperatore con la lucente
ceramica bianca, la sopravveste carminia che si agitava loro intorno
alle ginocchia e la visiera senza volto che fronteggiavano impassibili
una sfida incredibile con i Cielcin, ancora più bianchi, che premevano
su di loro da ogni lato. Là fummo noi a esercitare quella pressione,
spostandoci fra le colonne per evitare le scariche di plasma. Uno dei
tre uomini aveva perso la lancia e impugnava una spada in ciascuna mano.

Mi resi conto che in alto la folla si era fatta stranamente silenziosa,
trattenendo collettivamente il respiro. Nel corso della lotta avevo
perso il conto, ma con un'occhiata mi parve che fossimo rimasti in
tredici. C'era Switch, raggomitolato con Kiri nell'ombra di una colonna,
e questo mi fece esalare un respiro di sollievo. «Cosa cazzo fate fermi
lì?» ringhiò Ghen. «Se non ci muoviamo abbasseranno le colonne.»

Non ebbi il tempo di rispondere prima che Pallino ribattesse: «Dovranno
venire loro da noi.»

«Non succederà» ribatté Siran, e io fui d'accordo con lei.

Una serie di scatti risuonò in profondità nel pavimento, sotto di noi, e
con un ruggito stridente le colonne di cemento cominciarono ad
abbassarsi. Sarebbero scese fino a trovarsi al livello del terreno,
lasciandoci esposti. Fissai con occhi roventi Pallino, e Ghen,
guardandomi intorno. «Dobbiamo agire ora, altrimenti ci potranno
abbattere con tutto comodo.»

«Con tutto comodo?» ripeté Pallino, in tono sprezzante. «È così che
funziona, ragazzo. È così che funziona tutto.»

«Benissimo» ringhiai, poi mi rivolti a Ghen. «Lo farò io. Sei con me?»

Lui mi guardò con le sopracciglia aggrottate sotto l'orlo dell'elmo, poi
annuì. «Il nostro tempo si sta comunque esaurendo.»

«Ci sto anch'io» disse Siran. «Qual è il piano?»

Sollevai lo scudo. «Lasciate per ultimo quello con le spade.»

«Come piano non è granché» commentò un mirmidone senza nome che ci aveva
raggiunto.

«No» convenni, e rinfoderai la spada. Di norma si attaccava la persona
del nemico, ma adesso avevo bisogno di colpire la sua arma, e questo mi
diede un'idea, anche se folle.

Non gridammo. Gridare nel lanciarsi alla carica serve solo ad attirare
l'attenzione del nemico, e io volevo farlo il meno possibile. Dieci
passi di spazio scoperto separavano il capannello di gladiatori dalle
nostre coperture sempre più ridotte, e quegli assassini addestrati
avrebbero avuto tempo in abbondanza per puntare le due lance contro di
noi. Vidi la bocca delle lance farsi di un incandescente azzurro
biancastro, sentii il loro sibilo mentre succhiavano aria per riscaldare
il plasma. Era uno dei modelli ad azione lenta, alimentato ad aria, con
scorte di munizioni. Bene, avrebbero avuto a disposizione un solo tiro.

Le armi eruttarono fuoco.

Quando combatti -- non importa per quale causa -- fai una scelta. Scegli
di accantonare per un momento qualsiasi altra cosa, di incanalare tutto
quello che sei e che sei stato e di spingerlo attraverso la cruna di un
ago. Rischi tutto. La scarica di plasma investì il mio scudo,
arroventando un'area di fibre di carbonio fino a farle risplendere. Il
secondo gladiatore cedette al panico e il suo tiro andò a vuoto. Quello
con le due spade si girò, attonito, giusto in tempo perché Kiri ed Erdro
lo prendessero sul fianco destro, sbucando nel momento stesso in cui le
colonne scomparvero. Qualcuno dietro di noi, una donna, urlò quando un
colpo al plasma del secondo gladiatore la centrò in pieno, ma non mi
girai a guardare.

Chiusi la distanza fra me e il gladiatore più vicino, spostando lo scudo
dalla mano sinistra alla destra e serrando quel disco incandescente
prendendolo per il bordo. Poi -- disintegrando il buonsenso di un
migliaio di generazioni -- mi fermai di colpo e contai i secondi fin
quando mi parve che l'uomo potesse far fuoco di nuovo. La lancia sibilò
nel succhiare l'aria, e Ghen e Siran mi oltrepassarono correndo. Vidi i
pensieri del gladiatore che ci scrutavano da dietro la visiera senza
volto mentre lui cercava di scegliere un bersaglio. La barriera del
campo Royse sul suo scudo scintillava nel calore della giornata,
crepitava per la statica creata dalla terra e dalla polvere. Scagliai lo
scudo e il leggero tondo di carbonio fendette l'aria come un disco nel
pentatlon della Festa d'Estate. L'uomo lo ignorò, perché forse una parte
ben addestrata della sua mente si aspettava che il campo di energia
deviasse il colpo senza problemi. Non lo fece. Per quanto veloce, il mio
scudo rotondo era troppo lento per quella cortina di energia e lo colpì
in pieno petto, sollevandogli di scatto il braccio mentre faceva di
nuovo fuoco, spruzzando il plasma contro lo scudo di protezione
preventiva. Barcollò.

Poi Ghen gli fu addosso, e anche Siran. Estratta la spada, li
oltrepassai e mi posizionai in modo che la mole di Ghen mi schermasse
dal secondo gladiatore ancora armato. Sentii lo stridio dei
servomeccanismi nel petto dell'uomo mentre l'armatura si bloccava sotto
l'assalto di Ghen e Siran allontanava la lancia con un calcio. Con la
spada, colpii al braccio il secondo gladiatore armato di lancia con
tanta forza che la tuta lo costrinse a lasciar andare l'arma. Il braccio
gli oscillò lungo il fianco mentre i polimeri della tuta aderente si
indurivano fino a diventare qualcosa di simile alla pietra. Rifiutando
di arrendersi, estrasse il coltello dalla cintura e si girò di scatto
per trapassarmi. Era una cosa che nessun vero avversario avrebbe fatto.
Il mio colpo avrebbe potuto recidere il braccio di un uomo equipaggiato
come lo eravamo noi, ma il gladiatore non era davvero ferito. Il suo
attacco mi colse alla sprovvista e la sua arma mi scivolò sulla corazza,
lasciandovi un solco profondo.

«Allora non sei mancino?» commentai, e lo trafissi in una coscia. La
tuta stridette e lo azzoppò, ma non cadde come avrebbe dovuto fare e
dovetti eseguire un preciso secondo colpo, con il gomito che mi
crepitava mentre calavo la spada per disarmarlo prima di metterlo fuori
gioco con un risonante colpo alla testa.

Poi fu finita. Con l'ultimo gladiatore eliminato da Siran e da uno degli
altri. Come con Crispin, mi aspettavo che ci fosse un momento glorioso
che segnasse la fine della battaglia. Ma non successe nulla. Non succede
mai. Lo scontro finisce, il filo passa nell'ago e tutto ciò che eri
torna ad abbattersi su di te. Per un momento, tutto quello che riuscii a
sentire fu il sangue che mi pulsava ancora nelle orecchie, la sola cosa
che provai fu il peso dell'armatura che mi affondava nella pelle con i
suoi lacci di cuoio. La sola cosa di cui ero consapevole era il
sollevarsi e abbassarsi del mio petto al ritmo del respiro affaticato,
in quell'aria densa e umida.

Poi lo scudo protettivo si disattivò e il trionfo estatico della folla
ci trascinò, dandomi quel mio momento di esaltazione. Era uno
stravolgimento titanico delle aspettative. Mi chiesi quante di quelle
persone si fossero aspettate che morissimo tutti, mentre solo otto di
noi avevano perso la vita e altri dodici erano lì in piedi. In mezzo a
quel clamore guardai verso il palco del conte. Balian Mataro era in
piedi sotto il baldacchino a strisce oro e giada, un grosso uomo
massiccio come un toro con un altro più snello al fianco. Sollevò una
mano, e la sua immagine venne proiettata su alcuni schermi accanto al
palco... schermi di cui non mi ero neppure accorto fino a quel momento.
Il tumulto della folla si placò fino a quando la voce amplificata del
nobile poté sovrastarlo, ricca e superlativa.

«Avete combattuto bene, miei mirmidoni, davvero bene!» Stava battendo le
mani, e anche da dove mi trovavo, trenta piedi più in basso, vidi l'oro
che gli scintillava sulle dita, sulla fronte, sulla gola. Il metallo
spiccava sullo sfondo della pelle nera come il carbone. Era vistoso
quanto mio padre era sobrio, un vero esteta, e la sua voce era come vino
forte. «Posso dire sinceramente che nessuno di noi qui riuniti era mai
stato testimone di una sorpresa grande come questa.» Si appoggiò al
legno chiaro della ringhiera del suo palco.

Sollevai lo sguardo, notando per la prima volta lo sciame di droni
dotati di videocamera che orbitava sopra il campo. Switch era in piedi e
ci scambiammo qualche parola, quanto bastava perché mi accertassi che
stava bene. Guardandomi intorno colsi l'espressione sul volto di Ghen.
Sorrideva, ma nei suoi occhi c'era qualcosa di diverso dalla gioia. Mi
sorprese a guardarlo, annuì e continuò a sorridere. Era rispetto? Non ne
ero sicuro, ma ritenni di non dovere più temere nulla da quel grosso
uomo. Kiri venne avanti per abbracciarmi, mormorando sommesse parole di
congratulazione. «Quella cosa con lo scudo è stata dannatamente astuta»
disse, mentre mi abbracciava.

«Sono felice che ce l'abbiamo fatta» risposi, districandomi dal suo
abbraccio. Il guercio Pallino mi sorrise e vidi che nello scontro si era
scheggiato un dente. L'ex legionario si premette il pugno sul petto in
un saluto e chinò la testa. Ricambiai il gesto con un accenno di
inchino.

Il conte stava ancora parlando, rivolto più alla folla che ai vincitori.
«Un combattimento di cui non vedevamo l'uguale da molte stagioni! Molte
stagioni. Siamo molto compiaciuti, per cui assegniamo a ciascuno di voi
una somma di cinquanta hurasam per il vostro coraggio!» L'applauso che
seguì era studiato ad arte, un palliativo per lavare via da ogni cuore e
da ogni bocca il sapore della pestilenza.

Toccai il punto della mia corazza sotto cui l'anello del mio Casato
pendeva ancora dalla sua corda.

\emph{Pane e giochi circensi.}

