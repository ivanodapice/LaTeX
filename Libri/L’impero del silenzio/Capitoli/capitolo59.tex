\chapter{Alla vigilia dell'esecuzione}

Sei mai stato costretto a contemplare la tua morte? Magari rinchiuso in
una cella di una torre o in qualche bastiglia della Cappellania, ad
aspettare la fine portata dal filo della Spada Bianca? Sei mai rimasto
seduto per tutta una notte insonne a contare i secondi che ti rimangono
come fossero granelli di sabbia? Prego che tu non lo abbia fatto. Un
conto è morire, e un conto del tutto diverso è soffrire della paura
della morte e sopravvivere. A te che le hai patite entrambe, non auguro
nessuna delle due. Sei come una candela solitaria in una cappella, la
cui luce tremoli contro il Buio, un'oscurità fatta non di spazio ma di
tempo, delle fauci spalancate di un qualche futuro vuoto ed echeggiante
che ti è precluso per sempre.

È confortante sapere che il sole sorgerà sempre... almeno finché non lo
farà più e si dissolverà in cenere fredda con il deteriorarsi
dell'universo, o sarai tu a farlo. Il fuoco si spegne. E anche la vita.
O viene spenta. Per la candela della cappella questa non è una tragedia,
non sa quando viene spenta, è soltanto un simbolo, un avatar del sole
indomito acceso per vegliare attraverso la notte nel tempio della
Cappellania, ma la fiamma umana \emph{lo sa}, e trema non per il vento
ma per la paura. Per un malessere del cuore. Così tremavo io nel mio
letto soffocante e sul pavimento accanto a esso quando non riuscii più a
stare disteso fra le coltri. Anche se avevo appena ventitré anni --
niente se paragonato ai secoli che ho contato da allora -- sentivo la
mia età e lo spettro della mia fugace mortalità. Sentivo dolore in ogni
osso che mi ero rotto, avvertivo la cicatrice di ogni ferita guarita
quando vivevo in strada o nel colosseo.

Ho commesso molti errori e ho fatto di proposito molte cose terribili.
Per millecinquecento anni ho infestato la nostra galassia e accumulato
rimpianti. Forse Gilliam Vas meritava di morire, non era un brav'uomo
perché era maligno, meschino e crudele quanto la natura -- e
indubbiamente anche altri -- lo era stata con lui. Ho smesso di credere
che spetti a un qualsiasi uomo decidere cosa meritino gli altri uomini.
Ho conosciuto santi puniti per le loro virtù e mostri elogiati per la
loro mostruosità, e sono stato entrambe le cose.

Non molto tempo fa ho detto che mi sono sempre considerato un agnostico,
ma credo che gli uomini debbano avere un'anima. Non è sempre stato così,
in passato pensavo che fossimo solo animali e questa idea giustificasse
il modo come trattavamo da animali alcuni di noi. Pensarlo mi aiutava a
ignorare le brutture del mondo. Adesso ho imparato la lezione. Qualsiasi
cosa fosse Gilliam -- e in lui c'era ben poco di buono -- era un uomo
come qualsiasi altro. Io però avevo fatto una scelta, come ognuno di noi
deve fare sempre. Ero giovane e infuriato, imbarazzato e spaventato. Non
volevo morire e non volevo uccidere, e non volevo che Valka mi odiasse.

Dovevo scegliere.

La vita non riguarda mai quello che meritiamo. Non so se esiste un Dio,
che sia la Madre Terra o la sua icona o uno degli antichi dèi increati,
ma se esiste, soltanto Dio può dispensare una giustizia perfetta che non
è possibile avere nel nostro mondo decaduto. Noi possiamo solo sforzarci
di ottenerla. Forse Gilliam meritava di morire, o lo meritavo io. Forse
ci saremmo dovuti uccidere a vicenda o avremmo dovuto fare la pace senza
ricorrere alle armi. Non importa, non sta a me dirlo. Avevo fatto una
scelta, come ne ho fatte parecchie altre da allora, e questo è tutto ciò
che possiamo fare, per poi vivere in base a quelle scelte e alle loro
conseguenze. Giudicami tu, se vuoi. E che qualsiasi Dio esista, possa
perdonarmi e avere pietà della sua anima.


