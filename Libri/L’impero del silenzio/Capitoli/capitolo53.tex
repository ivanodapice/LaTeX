\chapter{Il gioco del serpente e della mangusta}

La tavola del conte grondava praticamente cibo. Perfino a casa di mio
padre avevo visto di rado simili ricchezze. Emesh non aveva foreste
reali in cui allevare selvaggina per la tavola alta, la carne veniva
generata nelle vasche e messa il più lontano possibile dalla famiglia
palatina per essere divisa fra gli artigiani e i funzionari minori della
corte seduti alla sua estremità più bassa. Il cibo naturale proveniva
tutto dal mare ed era in parte nativo e in parte di origine terrestre.
Vassoi di salmone grigliato e di capesante saltate in padella erano
accompagnati da barcate di salsa bianca e piatti di patate arrostite e
di peperoni ripieni. La portata principale, disposta in fette ordinate
come strati geologici, era un congrid nativo lungo dieci metri, una
creatura aliena simile a un'anguilla arrostita e che nuotava in una
salsa di vino azzurro.

Campi statici sulle arcate che portavano all'esterno davano alla sala
l'apparenza di uno spazio aperto e una musica sommessa proveniva non
dagli altoparlanti ma da un quartetto di archi e arpa posizionato
nell'angolo più lontano.

«Che ricetta deliziosa» commentò una dei capifazione della Gilda,
rivolgendosi a sua moglie. «Non sei d'accordo?» La donna annuì con fare
contegnoso, continuando a sorridere. La grande sala era tutta occupata
da un unico lungo tavolo, com'era usanza in occasione di simili
banchetti formali di Stato. Ci sono occasioni del genere alla corte di
ogni lord dell'Impero, come se fosse il cibo che in realtà tiene insieme
la nostra società. Come apparente pupillo del conte, io mi ero
guadagnato una posizione alquanto elevata a circa una dozzina di posti
da capotavola, il meno importante degli ospiti personali di lord Mataro,
incastrato fra un mercante profumato che proveniva dalla luna verde e un
tenente della Legione di nome Bassander Lin.

Valka era lì vicino, seduta accanto al suo mecenate, sir Elomas
Redgrave, mentre più in su alla mia destra, al di là di una manciata di
logoteti e di scoliasti, erano disposti i grandi del sistema di Emesh.
L'arconte Perun Veisi era presente con sua moglie, seduto a capotavola
alla sinistra del conte e di lord Luthor con i loro figli. E c'era Liada
Ogir, la cancelliera patrizia, situata accanto a una donna dal volto
duro nell'uniforme di gala nera della Legione con le mostrine a forma di
stelle e di quercia sui lati del colletto che la contrassegnavano come
tribuno di Armada: era la tribuno-cavaliere Smythe che aveva marciato
nell'arena durante l'Efebeia di Dorian.

Era presente anche il contingente della Cappellania, seduto vicino alla
coppia Veisi e quanto più vicino al trono lo permettesse l'etichetta.
Ero lieto che Gilliam e io fossimo dallo stesso lato del tavolo,
separati da parecchi scoliasti e dai capi della Compagnia Mercenaria del
Cavallo Bianco.

«Sei su Emesh da molto?» chiese Bassander; il tenente Lin, dovrei dire.
Mi rivolse un sottile sorriso, con il volto rasato rilassato ma pervaso
da una sottostante dignità che sconfinava nel formale. In lui c'era una
dura stanchezza, come se il banchetto fosse stato una trincea in cui si
era ritrovato, e fino a quel momento non aveva detto una sola parola,
concentrandosi sul cibo eccellente come se fosse stato un compito che
gli era stato assegnato. Il suo posto non era a quella tavola,
nonostante l'eleganza dell'immacolata uniforme color ebano.

Senza dubbio avrai sentito dire che ci siamo conosciuti come rivali e
abbiamo impegnato un duello per il comando del nostro esercito, ma non è
così. No, ho conosciuto la Fenice a tavola, in una tranquilla serata a
Borosevo. Bassander Lin. Il mio ultimo amico. Il mio nemico. L'eroe
della rotta a Perfugium, dove Hadrian il Semimortale ha fallito.
Veterano di cento battaglie, cavaliere, capitano, traditore. Sarebbe
stato tutte quelle cose, ma non lo era ancora. Quella notte era solo un
ospite a cena, proprio come me.

Non sapendo niente di tutto questo posai il bicchiere del vino stando
attento a non sporcare il formale completo grigio che avevo preso a
prestito per l'occasione. «Sono qui da alcuni anni... non da molto, e
sono entrato solo di recente al servizio del conte. Sono...» Un
prigioniero? Un pupillo? Un traduttore? «Un tutore.» Feci una breve
pausa per staccare un pezzo di pane dalla pagnotta bianca messa in
comune, poi domandai: «Cosa mi dici di te? Sei da molto nelle Legioni?»

Il tenente Lin fece una smorfia e si grattò il lato rasato della testa,
appena sopra l'orecchio. «Ecco, dipende da come misuri il tempo. Sono a
quasi diciotto anni di servizio attivo secondo il tempo di bordo, ma...»
Lasciò la frase in sospeso per bere un lungo sorso dalla sua coppa
d'acqua. Non beveva alcolici. «Se però calcoli anche il tempo sotto
ghiaccio... dèi dei miei padri, sono quasi due secoli.»

«Due secoli?» Dall'altro lato della tavola Valka per poco non si strozzò
con un boccone di peperone e formaggio di capra di importazione. «Non
puoi dire sul serio.» Da parte mia, pensai che si era concentrata sul
numero sbagliato... sul duecento e non sul diciotto. Il tenente era
quindi un palatino, o era di sangue patrizio al punto che la cosa non
faceva differenza. Un'attenta osservazione del suo volto non rivelò
tracce delle cicatrici che segnavano i lineamenti della cancelliera Ogir
o di Gilliam, o quelli della dama Camilla dallo sguardo duro, che sedeva
alle spalle del conte e del lord suo marito. I capelli del tenente erano
del colore del legno bruciato, senza tracce di grigio, e i suoi occhi di
un nero metallico erano più sorridenti che taglienti. Avrebbe potuto
avere davvero cento anni di vita, ma quelli che gravavano sulle sue ossa
non erano più di venti.

Bassander inclinò con deferenza il lato rasato della testa. «Sì,
signora.»

La capofazione della Gilda seduta accanto a Valka, una donna grande e
grossa che ricordava un bulldog, posò una mano inanellata sul braccio
della dottoressa. «Generalmente gli ufficiali imperiali si sottopongono
a lunghi periodi di sonno criogenico fra uno scongelamento e l'altro.»
Si protese in avanti con la mano che indugiava sul braccio di Valka...
cosa che non passò inosservata alla sua pallida ed esile moglie. «Il
buon tenente dovrebbe essere grato di non essere un coscritto. Sai, i
nostri legionari prestano servizio per un periodo di vent'anni, ma si
tratta di servizio attivo, senza contare gli anni di sonno criogenico.»

Bassander Lin confermò che la cosa era esatta. «Ho conosciuto un
centurione che si era arruolato durante il regno dell'imperatore
Aureliano III.»

«Ma è stato milleduecento anni fa!» esclamò la capofazione, in tono
sgomento.

Nello stesso tempo, Valka sibilò fra i denti una parola in panthai:
«\emph{Anaryoch}.» `Barbari.' Non mi vide, ma le sorrisi per esprimere
comprensione. Il cambiamento che imponeva il prestare servizio a vita
era subentrato solo di recente. Prima i soldati avevano prestato
servizio per periodi di vent'anni, alcuni anche fino a quattro. Molti
Casati palatini si erano opposti al cambiamento, cosa che avrei fatto
anch'io se la scelta fosse spettata a me.

In seguito appresi che l'indignazione di Valka era sincera perché il suo
popolo evitava come regola di avere eserciti stabili, preferendo fare
affidamento sui suoi terrori tecnologici per garantire una pace incerta.
La minaccia di un annientamento planetario -- della garanzia di una
distruzione reciproca -- teneva in riga i loro clan, e per lei era
preferibile che tutti gli uomini \emph{potessero} morire piuttosto che
ne \emph{morisse} anche uno soltanto. Pensai che era un principio che
potevo rispettare, per quanto barbaro.

La capofazione ridacchiò e rimosse la mano dal braccio di Valka per
battere un colpetto sul ginocchio dell'altra donna. «Ecco, penso che
quello di soldato sia più appropriatamente un lavoro da uomini. Non sei
d'accordo, tenente?»

Il tenente Lin parve riflettere, tamponandosi la fronte con il
tovagliolo. «Nella mia esperienza, signora, quello di soldato è un
lavoro per soldati.» E bevve un altro sorso d'acqua per sottolineare
quell'affermazione, con quella che mi parve una grazia derivante da
lunga pratica. Emerse da quella pausa rivolgendosi con disinvoltura a
Valka. «Se posso chiederlo, signora, considerato il tuo accento... sei
una Tavrosiana?»

Valka si spinse dietro un piccolo orecchio una ciocca di capelli. «Sì.»

«Una volta,» continuò il tenente, aggrappandosi a quell'argomento per
sfuggire al capofazione e alle sue domande imbarazzanti «tutta la mia
nave è stata data in prestito a una compagnia di Tavrosiani per la
Campagna di Mathuran. Vorrei aver potuto passare meglio il mio tempo
là.» Quel riferimento alla storia tavrosiana mi risultò incomprensibile
e abbassai la testa sul piatto perché non vedevo modo di entrare con
grazia nella conversazione. «Cosa ti porta su Emesh, mia signora?»
aggiunse {Bassander}.

«Dottoressa» lo corresse con disinvoltura Valka, pulendosi la bocca con
il tovagliolo bianco mentre socchiudeva gli occhi dorati nel fissare il
tenente. «Di fatto si tratta dei coloni.»

Quella parola chiave attirò l'attenzione di sir Elomas, che sedeva
dall'altro lato rispetto a lei e che interruppe la sua conversazione con
un logoteta del ministero della Previdenza Sociale. «Stiamo parlando
degli Umandh, vero?» Spinse indietro la massa di capelli neri e posò
coltello e forchetta con un tintinnio sbrigativo. Il vecchio cavaliere
si muoveva con una precisione da ragno, senza sprechi di energia, con
ogni più piccolo gesto calibrato al millesimo. Era il segno
caratteristico di qualcuno abituato a duellare da tutta una vita. «Hai
avuto occasione di esaminare i nativi... mmm...» Socchiuse gli occhi del
colore della spuma del mare per scrutare i gradi sul colletto di
Bassander. «Tenente?»

Da lì la conversazione si spostò per alcuni minuti sugli Umandh e io ne
approfittai per finire il mio cibo, segnalando a un cameriere di
rabboccarmi la coppa del vino e di mettere sul piatto di porcellana rosa
un'altra fetta di congrid.

Esaurito, o portato alla sua fine naturale, anche quell'argomento, il
tenente -- che sembrava affamato di conversazione e all'apparenza
riluttante a impegnarne una con lo scoliasta alla sua destra -- tornò a
rivolgersi a me. «E qual è il tuo ruolo qui, messere? Hai detto di
essere un tutore?»

«Hadrian» mi presentai, porgendo la mano come avevo imparato a fare al
colosseo, un gesto goffo a quella distanza così ravvicinata. «Hadrian
Gibson.»

«Bassander Lin» disse, stringendomi la mano.

«Il ragazzo parla la lingua dei Cielcin» commentò sir Elomas, con uno
strano bagliore negli occhi.

«Davvero?» Il tenente Lin inarcò le sopracciglia, con gli occhi così
dilatati da mostrare quasi solo il bianco.

Mi umettai le labbra, consapevole che la grande priora Vas e Gilliam
sedevano a pochi posti di distanza lungo la tavola, e abbassai il tono
della voce. «Sì.»

La grossa capofazione si protese in avanti. «Nel nome della Terra,
perché mai l'hai imparata?»

Mi sentii tentato di rispondere come avevo fatto con Gilliam, alcuni
giorni prima, nel cortile. `Per vedere con occhi non offuscati.'
Qualcosa però mi disse che in quel momento il mio senso del
melodrammatico non sarebbe stato apprezzato e ripiegai quindi su una
frase adatta ad Hadrian Gibson. «Mio padre... per essere mercanti siamo
stati molto fortunati, capisci... ha assunto uno scoliasta come tutore
per mio fratello e per me. Doveva insegnarci lo jaddiano, ma io ero
alquanto dotato, se posso dirlo di me stesso.»

«Dotato per le lingue, intendi?» chiese Bassander Lin, segnalando a uno
dei camerieri di versargli ancora da bere. La cameriera gli portò del
vino ma lui lo rifiutò e attese che gli servisse invece dell'acqua.

Annuii. «In qualche modo mi rimangono incollate qui.» Mi battei un
colpetto su un lato della testa. «Perfino il cielcin. Avevo sperato di
comunicare con i coloni locali, ma la dottoressa Onderra mi ha spiegato
che il loro è un linguaggio assolutamente impossibile.»

«Ma il cielcin...» interloquì la moglie minuta della capofazione, che
appariva ancora più pallida del suo solito. «Simili creature...
orribili. Demoni...» Quasi mi aspettai che tracciasse il segno del disco
del sole.

Guardai in modo espressivo in direzione di Valka. Una parte di me stava
ancora cercando di annullare l'impressione iniziale che aveva riportato
sul mio conto. «Non sono demoni, signora.» A quel punto lei tracciò il
segno del disco del sole, premendosi sulla fronte il cerchio formato da
pollice e indice. «Sai, quando ero bambino...»

La capofazione scoppiò a ridere, interrompendomi. «Devi scusare mia
moglie, messere. È molto devota.»

Rivolsi alle due donne il mio sorriso più incoraggiante, sentendomi in
qualche modo ridotto come un campione biologico su un vetrino. «Sono
certo che l'Impero ha bisogno di tutta la devozione possibile, signora.»
Bevvi un altro sorso del mio vino di Kandarene. «Quanto alla mia piccola
capacità, l'ho sempre considerata un investimento per il futuro.»

«Cosa intendi dire?» Bassander Lin si girò sulla sedia per vedermi
meglio, e in quel movimento qualcosa mi comunicò che era più vicino ai
vent'anni di quanto sembrasse, piuttosto che ai cento che erano
esteriormente possibili. Ne aveva forse quaranta? Non riuscivo a
smettere di pensare a lui come a un \emph{sottotenente}.

Allargai le mani. «Ecco, non possiamo combattere in eterno, giusto?»
chiesi, la stessa cosa che avevo detto a Adaeze Feng a tavola tanto
tempo prima. «E quando si finirà di combattere, qualcuno dovrà parlare
con loro.» Quando la moglie della capofazione fu sul punto di
commentare, indubbiamente con altre espressioni devote, sollevai una
mano per prevenirla: «Se non altro per garantire la loro resa.»

«Resa?» Al suono di quella voce alle mie spalle compresi che la donna
che mi sedeva di fronte non era stata sul punto di discutere ma di
avvertirmi. Ligeia Vas era lì in piedi, una strega incurvata dal tempo
avvolta in splendide vesti tinte del nero della Terra, con la spessa
treccia bianca che le cingeva due volte le spalle. Sul suo volto scorsi
echi di quello di Gilliam... quei due occhi azzurri gelidi come stelle
lontane erano identici a quello dello stesso colore dell'intus. Non ho
mai saputo da dove venisse l'occhio di colore nero, né mi importava
scoprirlo. I lineamenti che in lui erano una contorta parodia di umanità
apparivano invece finemente cesellati -- come nel marmo -- sul volto
segnato dal tempo della priora. «Noi non vogliamo la loro resa.» Detto
questo, si rivolse all'intero tratto della tavola alta, parlando a voce
abbastanza alta da essere sentita da lord Balian e da Luthor sui loro
seggi identici ad alto schienale. «I Cielcin devono essere spazzati via
dalla faccia della nostra galassia, epurati.» Scivolò in modo totale e
assoluto nei toni di un predicatore. «Nei Canti è scritto: `Vai nel Buio
e sottomettilo, e sottometti al tuo dominio tutto quello che è la fuori,
che si inchinerà a te.' Ed è anche scritto: `Non tollererai che i demoni
vivano.'»

Le volsi le spalle, inarcando le sopracciglia. La capofazione e sua
moglie avevano chinato entrambe il capo e la più minuta delle due donne
mormorava qualcosa rivolta al piatto che aveva toccato a stento.
Stupidamente, pensai che chiunque piluccasse in quel modo il cibo
migliore del pianeta doveva essere una pazza o una sciocca.
«Quell'ultima citazione è stata presa a prestito dagli scritti di un
antico culto» osservai. «Credo che la citazione originale si riferisse
alle streghe.» Irrigidii le spalle come un uomo che quasi si aspettasse
di sentire un coltello che gli penetrava fra le scapole. Aver voltato la
schiena era stata una mossa pericolosa da parte mia, ma ne valse la
pena, se non altro per il fugace, intenso sorriso negli occhi di Valka.

Sentii gemere il cuoio quando le mani di Ligeia Vas si serrarono sullo
schienale della mia sedia e giuro che la pelle color rame di Bassander
si fece bianca come la morte, accanto a me. «Tu sei quello da cui mi ha
messa in guardia il mio Gilliam... il ragazzo con la lingua di demone.»
`Con la lingua di demone.' Ligeia è stata la prima a definirmi così, ma
non sarebbe stata l'ultima. Sentii la moglie della capofazione mormorare
quelle parole facendo eco alla sacerdotessa. Il seme era stato piantato
e sarebbe fiorito a suo tempo. La sacerdotessa passò all'inglese
classico -- lingua tanto degli scoliasti quanto del rituale dell'alta
Cappellania -- citando direttamente dalla fonte a cui mi ero riferito,
un antico testo religioso appartenuto a uno dei culti endemici nelle
parti più antiche dell'Impero. «Non tollererai che una strega viva.» Poi
tornò al galstani e cominciò: «Significa...»

«So cosa significa» dissi, rispondendo in inglese e attirandomi sguardi
lungo la tavola, fin dai due lord.

Lord Mataro parve riconoscere la lingua e scoppiò a ridere. «Cosa ti
dicevo, Ligeia? Quel ragazzo è un vero talento.»

La priora sbuffò, ma non rimosse le mani dal mio schienale. «Questo era
lo schiavo del Colosso?»

«Non era uno schiavo, reverenza» interloquì il giovane Dorian,
intervenendo galantemente in mia difesa. «Era un mirmidone, e molto
bravo.»

Spostai la mia attenzione lungo la tavola, guardando al di là di
Bassander e dello scoliasta, oltre i foederati, Gilliam e la coppia
Veisi. Gilliam sorrideva, un taglio asimmetrico sul volto asimmetrico.
Ricordando la vecchia massima sul non far mai vedere al nemico se
sanguini, lo ricambiai con un sorriso altrettanto in tralice, e riuscii
a stento a mantenerlo in essere quando Ligeia dichiarò: «E un uomo che
parla la lingua dei demoni non è forse una strega?»

La temperatura della stanza -- già fresca per Emesh -- praticamente
gelò. Gilliam scoppiò a ridere, dannazione a lui, e le labbra di lord
Luthor si contrassero in un'ammissione del fatto che la vecchia aveva
segnato un punto a suo favore.

C'era tuttavia una risposta a un'accusa tanto assurda... mi ritirai a
mia volta nell'assurdo e con la bocca vicina alla coppa del vino
sussurrai: «Preferisco il termine `magio', se per te è lo stesso.»

Elomas scoppiò in una risata, un suono fasullo e metallico che comunque
invitò i suoi vicini di posto a unirsi a lui. Avrei potuto baciarlo. Era
stata una battuta goffa e scadente, ma anche la mia unica risorsa. «Hai
attributi degni di nota, ragazzo!» esclamò sir Elomas, continuando a
sogghignare.

Mi arrischiai a lanciare un'occhiata da sopra la spalla in direzione
della priora, il cui volto era gelido anche se l'atmosfera si era fatta
di colpo più calda nella sala da pranzo. «Stai attento, figlio mio.»

«Sempre, Vostra reverenza.» Chinai il capo e le volsi le spalle, fin
troppo consapevole delle sue mani simili ad artigli che serravano i pomi
d'ottone che decoravano gli angoli del mio schienale. «Anche se è
gentile da parte tua preoccuparti per me.»

Quel volto avvizzito simile a un cristallo di ghiaccio sembrava incapace
di manifestare emozioni mortali come preoccupazione o interesse.
Lanciando un'occhiata al volto irregolare di Gilliam decisi che non
c'era da meravigliarsi se quella creatura fosse risultata così
sgradevole, vittima di una simile madre innaturale.

Poi, come un raggio di luce, giunse la salvezza. Valka si schiarì la
gola. «Chiedo scusa, sacerdotessa, ma M Gibson stava per elargirci una
storia quando sei arrivata. Potrebbe continuare?» Una storia? Quale
storia? La mia mente si affannò in una folle danza per riordinarsi.
Cortesemente, Valka aggiunse: «Avevi detto: `Non sono demoni, signora.
Sai, quando ero bambino...' Poi la nostra amica qui presente ti ha
interrotto.» Indicò la capofazione. Era la mia immaginazione, oppure
Valka aveva imitato con precisione il tono e la cadenza della mia
precedente affermazione? La confusione di base della memoria scosse il
mio senso di dejà-vu e la fissai con aria accigliata.

Poi ricordai. Ero stato sul punto di raccontare una storia su come
Gibson mi avesse mostrato alcuni ologrammi di guerrieri cielcin, ma
pensai che non era il caso di parlarne. Un nuovo pensiero stava
prendendo forma nelle ombre del mio cranio, crescendo in fretta. Quella
parte interiore di me fece le capriole e si sfregò le mani mentre mi
lanciavo nella narrazione, dicendo: «Quando ero ragazzo...» \emph{Il mio
	littore? La mia guardia del corpo?}... «mio zio Roban mi portò a una
fiera dei liberi mercanti. A quel tempo una compagnia di Eudoriani era
in giro nel sistema e ricordo che c'era un uomo, un addestratore, che
faceva combattere gli animali fra loro mentre gli uomini scommettevano.»
Mi interruppi, permettendo alla cameriera di riempirmi il bicchiere. «Da
quel che ricordo era un tipo strano, non un omuncolo, ma con la pelle
blu. Mentre guardavamo, contrappose una mangusta a un serpente. Un
predatore contro un altro.»

«Cos'è una mangusta?» chiese un capitano dei foederati.

«Un tipo di mammifero terrestre non dissimile da un gatto, a quanto mi
hanno detto.» Tratto un profondo respiro continuai la storia, spostando
lo sguardo lungo la tavola in direzione dei lord e dell'arconte Veisi.
«Implorai mio zio di lasciarmi scommettere sull'esito della lotta, cosa
che feci a favore del serpente, sebbene lui insistesse perché
scommettessi a favore della mangusta. Cinque kaspum... la mia mancia di
un mese. Le creature lottarono in un canale su un lato della strada,
sotto gli occhi degli scommettitori che perlopiù se ne andarono
insoddisfatti. Riuscite a immaginare chi ha vinto?» Feci scorrere lo
sguardo lungo la tavola con le mani aperte in un gesto di invito.
«Nessuno risponde?»

Elomas -- che si stava dimostrando sempre più un vero sportivo anche se
era alquanto ignorante in fatto di ecologia terrestre -- fu il primo a
parlare. «Il serpente?» Questo generò un coro di cenni di assenso e di
fievoli approvazioni da parte degli ospiti.

Nell'angolo, il quartetto di archi passò a una melodia che mi era
leggermente familiare e sorrisi nel proseguire: «È quello che pensavano
gli spettatori, e in effetti era la scelta più ovvia, dato che la massa
del serpente era superiore a quella della mangusta nella misura di tre a
uno, per non parlare del veleno.» Mi presi un momento per spiegare, a
beneficio del capitano dei foederati e di alcuni altri, che i serpenti
terrestri avevano nelle zanne un potente veleno. «Tuttavia, è stata la
mangusta a vincere.» Bevvi un sorso di vino con una smorfia di
apprezzamento e mi girai a mezzo sulla sedia verso la priora ancora in
piedi. «Gli spettatori imprecarono contro l'addestratore dandogli del
truffatore, ma se ne andarono quando lui li minacciò con un altro dei
suoi serpenti.»

«La tua storia ha una qualche morale, M Gibson?» chiese Ligeia,
inarcando le sopracciglia sottilissime.

«In realtà no» risposi, rispolverando il mio sorriso in tralice. «A
parte un'osservazione che mio zio fece a quel tempo.» Le leggi di
un'adeguata abilità oratoria richiesero che rimanessi in silenzio per lo
spazio di alcuni secondi, limitandomi a inarcare le sopracciglia.
«Commentò che il comportamento degli animali non ci dovrebbe
sorprendere. Dopotutto, sono soltanto animali, e una tigre non può
cambiare le sue strisce, se mi perdonate la frase fatta. Le manguste
sono sempre state grandi cacciatrici di serpenti, come vi potrebbe dire
qualsiasi studente di letteratura. Di per sé, questo non è notevole, ma
risultò che l'Eudoriano aveva alterato la sua mangusta in modo da
renderla a prova di veleno di serpente.» Rivolsi alla priora un sorriso
smagliante, sperando che vedesse l'insulto per quello che era. «Vedi, il
problema è che l'Eudoriano era lui stesso una sorta di serpente che
truccava i combattimenti. Mio zio disse che il problema era proprio
quello, che non si sa mai quali uomini sono serpenti e quali manguste
finché non li mordi... o vieni morso.»

Il silenzio fluttuò come fumo, punteggiato dalle note sommesse dell'arpa
e della viola, e in qualche modo quella musica accentuò il silenzio
invece di romperlo. Dopo un momento distolsi lo sguardo dalla priora e
lo concentrai sul mio piatto. Valka mi stava sorridendo, una singola
fiamma di candela in un mare di volti cinerei, e già solo per questo
decisi che era valsa la pena di ricorrere a quello stratagemma,
qualsiasi altra cosa fosse successa. La banda -- veri soldati -- non
aveva saltato una singola nota durante quello spaventoso silenzio. Poi
il conte rise, battendo il bicchiere vuoto sulla superficie del tavolo
in un rauco applauso. «Un vero talento, non c'è che dire! Reverenza Vas,
sembra che non dobbiamo commettere l'errore di considerare il mirmidone
uno stupido.»

Con movimenti decisi, la grande priora si strinse la corda di lunghi
capelli bianchi intorno alle spalle incurvate dal tempo e dilatò le
narici ma sorrise, rivolgendomi quello che senza dubbio ritenne essere
un ottimo complimento. «Saresti un buon prete.»


