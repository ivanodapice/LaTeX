\chapter{Il Buio Esterno}

Le capsule criogeniche erano disposte lungo una parete della sezione
medica della nave e in esse c'era qualcosa -- forse il fatto che erano
messe come le colonne di una sala o il gelo vaporoso della stanza -- che
mi fece pensare al mausoleo dei miei antenati sotto il Riposo del
Diavolo. Erano dodici, ciascuna con il davanti formato da un cilindro di
vetro scuro e il telaio di metallo lucido che scintillava cupo, con le
luci degli indicatori che pulsavano lente di una luce rossa, verde o del
viola più cupo, senza un ritmo discernibile. I due all'estremità della
fila erano occupati, con il coperchio coperto di brina e i monitor dei
segni vitali che erano tutto uno scintillio di ologrammi fra il bianco e
l'azzurro. Gli altri erano vuoti, quiescenti. Ricordai di aver
trasportato un canopo di mia nonna, con i suoi occhi che mi fissavano
ciechi sospesi nel loro fluido azzurro, e sentii di nuovo lo sgocciolio
dell'acqua che scendeva lungo le stalattiti di calcare che aderivano al
soffitto sovrastante come perfette statue funebri nere.

Rabbrividii e mi strinsi le braccia sul petto. «Allora, come funziona?»

«Ecco, effettueremo il balzo a curvatura non appena ci saremo
allontanati dalle rotte di spedizione di Delos, poi ci sarà il viaggio
di cinque anni fino a Obatala, e quello di altri due fino a Siena, da
dove effettueremo il balzo finale che ci porterà a Teukros.» Demetri
calò una mano sulla copertura della capsula. «Tu però non ti accorgerai
di un accidente di niente. Queste bellezze erano tecnologia imperiale
per le Legioni, recuperate da un cacciatorpediniere che si è schiantato
su una delle lune di Bellos. In una di queste capsule potresti dormire
per mille anni senza ritrovarti con un solo capello grigio.»

Avanzai un poco nella stanza con passi cauti e la suola degli stivali
che scricchiolava sul sottile strato di brina che nessuno si era
preoccupato di rimuovere. «Allora tutto quello che devo fare è entrare
lì dentro? Adesso?»

«Tua madre non ha pagato per vitto e alloggio» replicò Demetri, con il
sorriso abituale che gli spiccava sul volto mentre si appoggiava alla
capsula più vicina. Non avevo idea di come non stesse morendo congelato
con quelle sue larghe vesti di seta. «D'altronde, non lo ha fatto
nessuno di noi. Nella stiva non abbiamo spazio per razioni per tredici
anni e non sono mai stato bravo con il giardinaggio, quindi entreremo
anche tutti noi subito dopo di te.» Controllò il suo terminale. «Sarà
il... 16149 secondo il tuo calendario imperiale la prossima volta che
respirerai di nuovo aria libera.»

Questo -- quel semplice dato di fatto -- mi bloccò. Non ero estraneo
agli aspetti tecnici dei viaggi spaziali, che erano conoscenze di
dominio pubblico presso la corte di qualsiasi nobile dell'Impero, ma
sentirseli esporre in quel modo, in termini tanto semplici e scevri da
qualsiasi emozione, sconvolse la mia mente ingenua ispirando attenzione.
A quei tempi il modo in cui un marinaio perdeva del tempo era chiamato
`scivolare' e forse lo è ancora. Tredici anni si sarebbero dissolti nel
nulla e io non me ne sarei neppure accorto.

Con lo sguardo fisso su quelle fredde macchine, segnalai che avevo
capito e alla fine protesi il mento in direzione delle due capsule
occupate. «Chi sono gli altri?»

«Eh?» Demetri si guardò alle spalle, con i capelli che quasi
scintillavano per quel movimento. «Ah, loro?» Accantonò la cosa con un
gesto. «Emigranti normanni: un tecnico di fattoria urbana e suo marito.
Sono a bordo da ventuno anni e ci lasceranno a Siena, quando ci
arriveremo.»

Da dove mi trovavo riuscivo a stento a distinguere due volti -- uno
pallido e uno dai vaghi toni ramati -- sotto il vetro scuro incrostato
di brina. Sospesi com'erano nell'oscurità, mi fecero pensare a campioni
biologici scuoiati e plastinati oppure sospesi nella formaldeide, in
salamoia come le cipolle e lasciati su uno scaffale del laboratorio di
un qualche scienziato pazzo. Sembravano morti, e in un certo senso lo
erano perché i loro processi vitali erano sospesi, spinti in avanti
verso un altro giorno. Avevo saputo che questo momento sarebbe arrivato,
e tuttavia niente mi avrebbe potuto preparare al suo orrore innaturale.
\phantomsection\label{fileintero-23.xhtml__idTextAnchor003}{}La paura è
la morte della ragione, dissi a me stesso, e di nuovo fu la voce di
Gibson a tranquillizzarmi con quelle parole familiari. La ragione è la
morte della paura. Quella era solo una fuga criogenica, un comune
procedimento di routine. Non sarei morto. Non qui. Non oggi.

Trassi un profondo respiro e quando esalai il fiato annuii. «Sono
pronto.»

«Bene!» La voce di Juno risuonò alle mie spalle e nel girarmi la vidi
entrare, precedendo un uomo baffuto dal volto cereo e dai flosci capelli
biondi raccolti in una coda. Con mio disgusto, il piccolo omuncolo entrò
dietro di loro, con le nocche che strisciavano letteralmente sul
pavimento. «Sarric, prepara la capsula.»

L'uomo biondo con i baffi chinò in silenzio la testa e si sfregò il
groviglio geometrico tatuato sulla sua fronte troppo alta, diamanti e
triangoli intrecciati. «Solo un momento.» L'uomo -- quel dottore che
Demetri Arello aveva menzionato di sfuggita in precedenza -- mi
oltrepassò con movimenti quasi del tutto silenziosi, esalando piccole
volute di vapore nell'aria gelida mentre armeggiava con la capsula più
vicina alle due già occupate.

«Ti rimettono nella bottiglia, eh?» chiese l'omuncolo, sollevando la
lunga treccia e drappeggiandosi quella disgustosa fune di capelli
intorno alle spalle come uno scialle. Ridacchiò. «Torni da dove sei
venuto.»

Juno gli assestò dietro un ginocchio un calcio deciso ma non troppo
violento.

Ignorai tanto il piccolo goblin quanto la donna. «Tu» disse il dottore,
battendo le mani per cancellare una serie di ologrammi. «Ti devi
spogliare.» Nel parlare non sollevò lo sguardo e si accoccolò per
esaminare uno schermo inserito nella parete accanto alla capsula.
«Abbiamo un armadietto per lui, capitano?» La sua voce era strana, roca
e crepitante. Decisi che era un tavrosiano nel ripensare alla lingua,
che mi pareva di aver sentito parlare dalle ragazze sporche che avevo
visto nel corridoio della nave appena prima del decollo. Quell'uomo era
uno dei membri dei clan della Demarchia di Tavros, il che spiegava il
tatuaggio. Anche Valka aveva tatuaggi del genere, che indicavano la
storia genetica e personale di una persona con un linguaggio simbolico
che non avevo mai imparato a leggere.

«Da questa parte.» Demetri indicò una serie di ammaccati armadietti di
metallo. «Metti tutto qui dentro.»

Mi immobilizzai, fissando il capitano sorridente, la sua splendida
moglie, il dottore dal volto cereo e il loro piccolo mostro domestico.
«Potrei avere un po' di privacy, allora?»

A parte il dottore risero tutti, e Saltus commentò: «Una volta che sarai
sotto ghiaccio potremo vedere tutto il tuo minuscolo pistolino, cugino.
Inutile fare il timido adesso.» La creatura snudò i suoi denti troppo
numerosi. Juno gli assestò un altro calcio e lui ricadde di lato contro
la parete con uno strillo.

«Lascia in pace il ragazzo, Salt» ingiunse, chinandosi ad afferrare
l'omuncolo per la collottola. «Ora vattene.» E un po' lo spinse e un po'
lo scaraventò verso la porta.

L'obbedienza alla necessità mi aveva già indotto a togliermi la giacca,
quella lunga giacca di cui Gibson mi aveva detto che non avrei avuto
bisogno. Demetri aprì uno degli armadietti e lo tenne aperto mentre
appendevo ordinatamente la giacca all'interno. Mentre la mettevo al suo
posto, la carta universale che avevo ottenuto dalla capofazione della
Gilda Mineraria cadde fuori e atterrò rumorosamente sul pavimento.
Scattai verso di essa, sperando di recuperarla prima che Demetri potesse
vedere di cosa si trattava e la riposi di nuovo nella fodera della
giacca, sorprendendo il capitano che mi osservava con le bianche
sopracciglia inarcate. «Fammi arrivare a Teukros ed è tua.» In ogni caso
non ne avrei avuto bisogno. «Lo giuro.»

Il dottore ci stava osservando. «Cos'era?»

«Una carta bancaria» replicò Demetri. «Quanto?»

«Parecchio.»

Infine mi ritrovai nudo e tremante nell'aria gelida, con i peli ritti
sulla pelle pallida e con entrambe le mani a coprire i genitali mentre
cercavo di non incontrare lo sguardo della donna e dei due uomini che
avevo intorno. Il dottore venne avanti e mi posò una mano secca sulla
spalla. «Ora vieni.» Mi guidò verso la capsula aperta e mi aiutò a
entrare. Mi tirai su con una mano, usando l'altra per coprirmi. Nel
vedere il mio anello, il dottore mi afferrò la mano libera. «Devi
toglierlo. Ti ustionerà.»

Scossi vigorosamente il capo. «Che lo faccia.» Guardai l'armadietto,
pensando alla carta universale. Mia madre aveva assoldato queste persone
e a quanto pareva Adaeze Feng le aveva raccomandate, ma questo non
significava che dovessi fidarmi di loro, e l'anello era tutto quello che
mi rimaneva del ragazzo che ero stato: un cerchio d'argento con
incastonata una corniola su cui l'incisione a laser del sigillo del
diavolo mascherava un cristallo in cui erano immagazzinati terabyte di
dati. Esso conteneva una copia dei contratti stipulati con la Gilda
Mineraria come pure ogni sorta di altri documenti, fra cui la mia
identificazione. Non intendevo separarmene.

Il dottor Sarric sbuffò. «Tipica stupidità dei barbari imperiali.»

«Lascia perdere, Sarric» intervenne Demetri, avvicinandosi con i pugni
piantati sui fianchi. «Non stiamo cercando di derubarti, ragazzo. Non
siamo pirati. I pirati ti avrebbero scaricato fuori dal portello stagno
nel momento stesso in cui abbiamo lasciato Delos.»

L'imbottitura di plastica bianca sul retro della capsula mi aderiva alla
pelle nuda e tremavo, lì in piedi nudo. «Non si tratta di questo,
capitano. È... è una cosa dei palatini.»

Questo lo fece ridere. «Non ti conviene avere indosso quell'anello
dentro la capsula.»

«Intendo conservarlo.» Serrai la mascella e appoggiai il capo sul
poggiatesta della capsula. «Procediamo.»

Il dottore lanciò un'occhiata al capitano, grattandosi la testa appena
sopra un piccolo orecchio. «Demetri?»

Il mercante jaddiano agitò una mano come per accantonare la cosa. «Il
ragazzo può fare come vuole, Sarric.»

Il dottore espirò con forza attraverso i denti gialli. «Come vuoi,
allora.» E senza preamboli mi applicò un sensore sul petto, seguito da
un altro e da un terzo, senza quasi guardarmi mentre lavorava. Fatto
questo tirò fuori da una fessura della capsula un ago autosterilizzante
che sibilò nel trapassarmi il braccio, poi mi passò la cinghia di
sicurezza intorno ai bicipiti. «Questo ti darà piuttosto presto un senso
di freddo.»

Lo stava già facendo. Il gelo strisciava dall'ago nel braccio, il sangue
si trasmutava, le cellule si indurivano senza lacerarsi. Il cervello
cominciò ad annebbiarmisi e come da molto lontano sentii il dottor
Sarric dire: «È pronto, sigillate la capsula.» Sentii, più che vedere,
il vetro scuro che si chiudeva su di me, intrappolandomi come in un
sarcofago, poi qualcosa di freddo e gelatinoso cominciò a salirmi
intorno alle caviglie mentre l'oscurità prendeva forma dietro i miei
occhi. Attraverso quell'oscurità percepii di nuovo le maschere funebri
dei miei antenati che mi fissavano con i loro occhi viola accusatori e
ostili, appese al di sopra della porta della camera del consiglio, sotto
la Cupola delle Incisioni Radiose.

Il gel conservante mi salì intorno mentre congelavo da dentro. Volevo
urlare, picchiare i pugni contro le pareti della capsula, ma avevo già
perso ogni forza. Stavo annegando: sapevo che stavo annegando e che non
c'era niente che potessi fare. Sarei morto in quella capsula. Poi arrivò
la parte peggiore.

La mia respirazione cessò. Il fluido ancora non mi arrivava neppure al
mento, ma smisi di respirare. Poi esso mi scivolò dentro, acqua nera
densa come olio che mi scendeva lungo la gola e mi entrava nel naso.
Quel Buio Esterno si impossessò di me e sprofondai nell'oscurità e nel
freddo.

Quando mi svegliai, il mio mondo era finito.

