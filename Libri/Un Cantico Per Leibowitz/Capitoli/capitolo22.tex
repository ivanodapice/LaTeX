	\chapter{\phantom{text}}

\lettrine{E}{ra} il giovedì dell\textquotesingle Ottava di Ognissanti.

In preparazione per la partenza, il Thon e il suo seguito dividevano gli
appunti e i documenti nel sotterraneo. Lo studioso aveva attirato un
piccolo pubblico di monaci, e prevaleva uno spirito di amicizia, ora che
il momento di andarsene si avvicinava.

In alto, la lampada ad arco continuava a scintillare abbagliante,
riempiendo l\textquotesingle antica biblioteca di una dura luce
biancazzurra, mentre la squadra di novizi azionava fiaccamente la dinamo
a mano. L\textquotesingle inesperienza del novizio che sedeva in cima
alla scaletta per regolare costantemente la distanza tra i due carboni
dell\textquotesingle arco provocava scintillii irregolari: quel novizio
aveva sostituito il precedente e più abile operatore, che in quel
momento era nell\textquotesingle infermeria, con compresse umide sugli
occhi.

Il Thon Taddeo aveva risposto a domande sul suo lavoro con minor
reticenza del solito, non più preoccupato, evidentemente, di argomenti
controversi come le proprietà di rifrazione della luce o le ambizioni
del Thon Esser Shon.

--- Ora, a meno che questa ipotesi sia senza senso --- stava dicendo ---
dovrebbe essere possibile confermarla in qualche modo mediante
l\textquotesingle osservazione. Io ho prospettato
l\textquotesingle ipotesi, con l\textquotesingle aiuto di alcune
nuove\ldots{} o meglio, di alcune antichissime forme matematiche
suggerite dal nostro studio dei vostri Memorabilia.
L\textquotesingle ipotesi sembra offrire una spiegazione più semplice
dei fenomeni ottici, ma francamente, non riuscivo a pensare ad alcun
metodo per sperimentarla, dapprima. E allora il vostro fratello Kornhoer
mi è stato di grande aiuto. ---. Fece un cenno con il capo in direzione
dell\textquotesingle inventore, sorridendo, e spiegò uno schizzo del
proposto apparecchio di prova.

--- Che cos\textquotesingle è? --- chiese qualcuno dopo un breve
intervallo di sbalordimento.

--- Ecco\ldots{} è una pila di lastre di vetro. Un raggio solare che
colpisce la pila con questo angolo sarebbe in parte riflesso e in parte
trasmesso. La parte riflessa sarà polarizzata. Ora, noi regoliamo la
pila per riflettere il raggio attraverso questo oggetto, che è
un\textquotesingle idea di frate Kornhoer, e facciamo cadere la luce su
questa seconda pila di lastre di vetro. La seconda pila è disposta in
modo di riflettere quasi tutto il raggio polarizzato e per non
trasmetterne quasi nulla. Guardando attraverso il vetro, difficilmente
vedremmo la luce. Tutto questo è stato sperimentato. Ma ora, se la mia
ipotesi è corretta, chiudendo questo interruttore sulla bobina di campo
di frate Kornhoer si dovrebbe provocare un improvviso ravvivamento della
luce trasmessa. Se non sarà così\ldots{} --- e scrollò le spalle ---
allora scarteremo l\textquotesingle ipotesi. --- Dovreste scartare la
bobina, invece --- propose modestamente frate Kornhoer. Non sono certo
che produrrà un campo abbastanza forte.

--- Io sì. Voi avete un istinto per queste cose. Io trovo molto più
facile sviluppare una teoria astratta che realizzare un metodo pratico
per provarla. Ma voi avete il dono straordinario di vedere tutto sotto
forma di viti, fili e lenti, mentre io sto ancora pensando a simboli
astratti.

--- Ma, tanto per cominciare, a me le astrazioni non verrebbero neppure
in mente, Thon Taddeo.

--- Noi due faremmo un\textquotesingle ottima squadra di ricerca,
fratello. Vorrei che voi accettaste di venire da noi al
\emph{collegium}, almeno per un certo tempo. Credete che il vostro abate
vi permetterebbe di partire?

--- Non ho la presunzione di indovinarlo --- mormorò
l\textquotesingle inventore, improvvisamente imbarazzato.

Il Thon Taddeo si rivolse agli altri. --- Ho sentito parlare di
``fratelli assenti''. Non è forse vero che qualche membro della vostra
comunità è impiegato altrove, temporaneamente?

--- Qualcuno soltanto, Thon Taddeo --- disse un giovane prete. --- Un
tempo, l\textquotesingle Ordine forniva impiegati, scrivani e segretari
al clero secolare, e alle corti reali ed ecclesiastiche. Questo,
tuttavia, fu durante i tempi delle maggiori ristrettezze, qui
all\textquotesingle abbazia. I fratelli che lavoravano altrove qualche
volta hanno salvato gli altri dalla morte per fame. Ma ora non è più
necessario, e avviene di rado. Naturalmente, vi sono alcuni fratelli che
studiano a Nuova Roma, adesso, ma\ldots{}

Ecco! --- disse il Thon con improvviso entusiasmo. --- Vi, offro di
studiare al \emph{collegium}, fratello. Stavo parlando al vostro abate,
ma\ldots{}

--- Sì? --- chiese il giovane prete.

--- Ecco, mentre non siamo d\textquotesingle accordo su alcune cose,
posso comprendere il suo punto di vista. Penso che uno scambio di
studenti potrebbe migliorare i nostri rapporti. Naturalmente vi verrebbe
assegnato uno stipendio, e io sono sicuro che il vostro abate ne farebbe
l\textquotesingle uso migliore.

Frate Kornhoer chinò il capo ma non disse nulla.

--- Suvvia! --- rise lo studioso. --- Non mi sembrate compiaciuto
dell\textquotesingle invito, fratello!

--- Ne sono lusingato, naturalmente. Ma non spetta a me decidere su
queste cose.

--- Certo, e io naturalmente lo comprendo. Ma non mi sognerei di
chiederlo al vostro abate, se l\textquotesingle idea vi dispiace.

Frate Kornhoer esitò. --- La mia vocazione è per la Religione --- disse
infine. --- Cioè\ldots{} per una vita di preghiera. Noi pensiamo che
anche il nostro lavoro sia una specie di preghiera. Ma quella\ldots{}
--- e indicò la sua dinamo --- mi sembra piuttosto un gioco. Tuttavia,
se don Paulo decidesse di mandarmi\ldots{}

--- Partireste con riluttanza --- concluse indispettito lo studioso. ---
Sono sicuro che potrei indurre il \emph{collegium} a mandare al vostro
abate almeno cento hannegan d\textquotesingle oro ogni anno, mentre voi
sarete presso di noi. Io\ldots{} --- Si interruppe, per guardare le
espressioni dei religiosi. --- Scusatemi, ho detto qualche cosa di
sbagliato?

A metà della scala, l\textquotesingle abate si fermò per osservare il
gruppo nel sotterraneo. Parecchi visi inespressivi erano rivolti verso
di lui. Dopo pochi secondi il Thon Taddeo notò la presenza
dell\textquotesingle abate e lo salutò con un cordiale cenno del capo.

--- Stavamo parlando proprio di voi, padre --- disse. --- Se avete
ascoltato, forse dovrei spiegare\ldots{}

Don Paulo scosse il capo. --- Non è necessario.

--- Ma mi piacerebbe discutere\ldots{}

--- Potete aspettare? In questo momento ho molta fretta.

--- Certamente --- disse lo studioso.

--- Tornerò prestissimo. --- Risalì le scale. Padre Gault lo stava
aspettando nel cortile.

--- Lo hanno già saputo, Domne? --- chiese cupo il priore.

--- Non l\textquotesingle ho chiesto, ma sono sicuro che non
l\textquotesingle hanno saputo --- rispose don Paulo. --- Stanno facendo
sciocche conversazioni, laggiù. Stanno parlando di portare frate
Kornhoer a Texarkana con loro.

--- Allora non l\textquotesingle hanno saputo, questo è certo.

--- Sì. E ora, dov\textquotesingle è?

--- Nella foresteria, Domne. C\textquotesingle è il medico, con lui. È
in delirio.

--- Quanti fratelli sanno che è qui?

--- Non più di quattro. Stavamo cantando Nona quando è arrivato alla
porta.

--- Dite a quei quattro di non parlarne con nessuno. Poi raggiungete i
nostri ospiti nel sotterraneo. Siate molto cortesi, e non fateglielo
sapere.

--- Ma dovrebbero esserne informati prima che partano.

--- Naturalmente. Ma lasciate che si preparino, prima. Sapete che questo
non impedirà loro di partire. Quindi, per ridurre al minimo
l\textquotesingle imbarazzo, aspettiamo l\textquotesingle ultimo minuto
per dirglielo. Ora, l\textquotesingle avete con voi?

--- No, l\textquotesingle ho lasciato con i suoi documenti nella
foresteria.

--- Andrò a vederlo. E adesso avvertite i fratelli, e raggiungete i
nostri ospiti.

--- Sì, Domne.

L\textquotesingle abate si avviò verso la foresteria. Quando entrò, il
frate farmacista stava uscendo dalla stanza del fuggitivo.

--- Vivrà, fratello?

--- Non posso saperlo, Domne. Torture, inedia, febbre da
sfinimento\ldots{} se Dio lo vuole\ldots{} --- E scrollò le spalle.

--- Posso parlargli?

--- Sono certo che non ha importanza. Ma non riesce a parlare con
lucidità.

L\textquotesingle abate entrò nella stanza e chiuse senza far rumore la
porta dietro di sé. --- Frate Claret?

--- No, basta! --- boccheggiò l\textquotesingle uomo disteso sul letto.
--- Per l\textquotesingle amore di Dio, basta\ldots{} vi ho detto tutto
ciò che so. Io l\textquotesingle ho tradito. E adesso lasciatemi\ldots{}
essere.

Don Paulo guardò con pietà il segretario del defunto Marcus Apollo.
Guardò le mani dello scrivano. C\textquotesingle erano soltanto piaghe
dove c\textquotesingle erano state le unghie.

L\textquotesingle abate rabbrividì e si voltò verso il tavolino accanto
al letto. In un mucchietto di carte e di effetti personali, trovò subito
il documento, rozzamente stampato, che il fuggitivo aveva portato con sé
da occidente:

\begin{center}
	\justify{HANNEGAN IL PODESTÀ per Grazia di Dio: Sovrano di Texarkana, Imperatore
		di Laredo, Difensore della Fede, Dottore delle Leggi, Capo dei Clan dei
		Nomadi e Vaquero Supremo delle Pianure, a TUTTI I VESCOVI, E PRELATI
		della Chiesa in tutto il Nostro Legittimo Reame, Salute e AVVISO, perché
		questa è la LEGGE \emph{videlicet} \& vale a dire:}
\end{center}
\leavevmode\\
\begin{center}
	\justify{1) Laddove un certo principe straniero, tale Benedetto XXII, Vescovo di
		Nuova Roma, presumendo di asserire un\textquotesingle autorità che non
		gli compete di diritto sul clero di questa nazione, ha osato tentare,
		primo, di porre la Chiesa Texarkana sotto sentenza di interdizione e,
		più tardi, di sospendere questa sentenza, creando conseguentemente
		grande confusione e spirituale abbandono fra tutti i fedeli, Noi, unico
		legittimo dominatore al di sopra della Chiesa in questo reame, agendo in
		concordia con un concilio di vescovi e di prelati, con la presente
		dichiariamo ai Nostri leali sudditi che il predetto principe e vescovo,
		Benedetto XXII, è un eretico, simoniaco, assassino, sodomita e ateo,
		indegno di qualsiasi riconoscimento da parte della Santa Chiesa nelle
		terre del Nostro regno, impero o protettorato. Chi lo serve non serve
		Noi.}
\end{center}
\leavevmode\\
\begin{center}
	\justify{2) Sia noto, pertanto, che sia il decreto di interdizione sia il decreto
		che lo sospende sono con la presente CANCELLATI, REVOCATI, DICHIARATI
		NULLI E PRIVI DI CONSEGUENZA, perché non ebbero mai alcuna validità
		originaria\ldots{}}
\end{center}
\leavevmode\\

Don Paulo gettò soltanto una breve occhiata al seguito. Non
c\textquotesingle era bisogno di leggere oltre. L\textquotesingle Avviso
podestarile ordinava che il clero texarkano si procurasse licenze per
esercitare il ministerio, proclamava la somministrazione dei Sacramenti
da parte di persone non autorizzate un crimine punibile secondo la
legge, faceva del giuramento di suprema lealtà al podestà una condizione
per ottenere autorizzazione e riconoscimento. Era firmato non soltanto
con la X del podestà, ma anche da parecchi ``vescovi'' i cui nomi erano
sconosciuti all\textquotesingle abate.

Ributtò il documento sulla tavola e sedette accanto al letto. Il
fuggitivo aveva gli occhi spalancati, ma si limitava a guardare il
soffitto e ad ansimare.

--- Frate Claret? --- chiamò, dolcemente. --- Fratello\ldots{}

Nel sotterraneo, gli occhi dello studioso brillavano
dell\textquotesingle esuberanza d\textquotesingle uno specialista che
invade il campo di un altro specialista allo scopo di chiarire una
grande confusione. --- In realtà, sì! --- disse, in risposta alla
domanda di un novizio. --- Io ho individuato una fonte, qui, che
dovrebbe, secondo me, essere di grande interesse per il Thon Maho.
Naturalmente, non sono uno storico, ma\ldots{}

--- Thon Maho? È quello che, ehm, sta cercando di correggere la Genesi?
-\/- chiese maliziosamente padre Gault.

--- Sì, è lui\ldots{} --- Lo studioso si interruppe, lanciando uno
sguardo un po\textquotesingle{} sorpreso a Gault.

--- Benissimo --- disse il prete, ridacchiando. --- Molti di noi pensano
che la Genesi sia più o meno allegorica. Che cosa avete scoperto?

--- Abbiamo individuato un frammento prediluviale che suggerisce un
concetto molto rivoluzionario, secondo me. Se interpreto correttamente
il frammento, l\textquotesingle Uomo non fu creato se non poco tempo
prima della caduta dell\textquotesingle ultima civiltà.

--- Cooosa? E allora da dove veniva la civiltà?

--- Non dall\textquotesingle umanità. Fu sviluppata da una razza
precedente che si estinse durante il \emph{Diluvium Ignis}.

--- Ma la Sacra Scrittura risale a migliaia di anni prima del
\emph{Diluvium}!

Il Thon Taddeo conservò un silenzio significativo.

--- Voi proponete --- disse Gault, improvvisamente sgomento ---
l\textquotesingle ipotesi che noi non siamo i discendenti di Adamo? Che
non siamo legati all\textquotesingle umanità storica?

--- Aspettate! Io propongo soltanto la congettura che la razza
prediluviale, che si definiva umana, riuscisse a creare la vita. Poco
prima della distruzione della loro civiltà, quegli esseri riuscirono a
creare gli antenati dell\textquotesingle umanità attuale\ldots{} ``a
loro immagine''\ldots{} come una razza di schiavi.

--- Ma anche se voi respingete completamente la Rivelazione, questa è
una complicazione assolutamente non necessaria, alla luce del semplice
buon senso! --- protestò Gault.

L\textquotesingle abate aveva sceso quietamente le scale. Si fermò
sull\textquotesingle ultimo pianerottolo e ascoltò, incredulo.

--- Potrebbe sembrare così affermò il Thon Taddeo --- fino a che non
considerate quante cose spiegherebbe, questa ipotesi. Voi conoscete le
leggende della Semplificazione. Assumono tutte un significato, mi
sembra, se si considera la Semplificazione come una ribellione
d\textquotesingle una razza schiava contro l\textquotesingle originale
specie dei creatori, così come suggerisce il frammento di cui parlo. E
spiegherebbe anche perché l\textquotesingle umanità di oggi sembra così
inferiore a quella antica, perché i nostri antenati precipitarono nella
barbarie, quando i loro padroni si estinsero, perché\ldots{}

--- Dio abbia misericordia di questa casa! --- gridò don Paulo,
avanzando verso l\textquotesingle alcova. --- Risparmiaci, o
Signore\ldots{} noi non sappiamo quello che facciamo.

--- Io dovrei saperlo --- mormorò lo studioso, rivolto a tutto il mondo
lontano.

Il vecchio prete avanzò come una nemesi verso il suo ospite. --- Dunque
noi siamo soltanto creature di altre creature, signor Filosofo? Siamo
stati fatti da dei inferiori a Dio, e di conseguenza comprensibilmente
meno che perfetti\ldots{} senza nostra colpa, naturalmente.

--- È soltanto una congettura che spiegherebbe molte cose --- disse
impettito il Thon, non disposto a cedere.

--- E che assolverebbe da molte colpe, non è vero? La ribellione
dell\textquotesingle Uomo contro i suoi creatori era, senza dubbio,
soltanto un giustificabile tirannicidio contro gli infinitamente malvagi
figli di Adamo, dunque.

--- Io non ho detto\ldots{}

--- Mostratemi, signor Filosofo, questo brano straordinario!

Il Thon Taddeo si affrettò a frugare tra i suoi appunti. La luce
cominciò ad ammiccare, quando i novizi che azionavano la dinamo si
tesero per ascoltare. Il piccolo pubblico dello studioso era rimasto in
uno stato di trauma fino a che l\textquotesingle ingresso tempestoso
dell\textquotesingle abate non aveva mandato in frantumi
l\textquotesingle ottuso sbigottimento degli ascoltatori. I monaci
sussurravano tra loro; qualcuno osò ridere.

--- Ecco qui --- annunciò il Thon Taddeo, porgendo a don Paulo parecchie
pagine.

L\textquotesingle abate gli lanciò una breve occhiata fulminante e
cominciò a leggere. Il silenzio era impacciato. ---
L\textquotesingle avete trovato nella sezione ``Non Classificati'',
vero? --- chiese, dopo pochi secondi.

--- Sì, ma\ldots{}

L\textquotesingle abate continuò a leggere.

--- Bene, penso che dovrei finire di preparare i bagagli --- mormorò lo
studioso, e ricominciò a dividere i documenti. I monaci si agitavano
irrequieti, come se desiderassero allontanarsi senza farsi notare.
Kornhoer meditava, tutto solo.

Dopo pochi minuti di lettura, don Paulo porse bruscamente gli appunti al
priore. --- \emph{Lege!} --- ordinò, burbero.

--- Ma cosa\ldots?

--- Un frammento di commedia, o un dialogo, sembra. L\textquotesingle ho
già visto prima. È qualcosa che parla di qualcuno che aveva creato
alcuni esseri artificiali come schiavi. E gli schiavi si rivoltano
contro il loro creatore. Se il Thon Taddeo avesse letto il \emph{De
	Inanibus} del venerabile Boedullus, avrebbe trovato questo frammento
classificato come ``probabile favola o allegoria''. Ma forse al Thon non
importerebbe molto la valutazione del venerabile Boedullus, quando può
darne una propria.

--- Ma che specie di\ldots{}

--- \emph{Lege!}

Gault si trasse in disparte con gli appunti. Paulo si volse di nuovo
verso lo studioso e parlò con tono educato, informativo, enfatico: ---
``A immagine di Dio Egli li creò: maschio e femmina Egli li creò''.

--- Le mie osservazioni erano pure congetture --- disse il Thon Taddeo.
--- La libertà di speculare è necessaria\ldots{}

--- ``E il Signore Iddio prese l\textquotesingle uomo, e lo mise nel
paradiso di delizie, perché lo curasse. E\ldots''

---\ldots{} per il progresso della scienza. Se volete intralciarci la
via con la cieca osservanza, con un dogma non ragionato, allora voi
preferite\ldots{}

--- ``E Dio lo comandò, dicendo: Di ogni albero del paradiso tu potrai
mangiare; ma non mangerai i frutti dell\textquotesingle albero della
conoscenza del bene\ldots''

---\ldots{} lasciare il mondo nella stessa ignoranza e superstizione
contro cui affermate che il vostro Ordine\ldots{}

--- ``\ldots{} e del male. Perché nel giorno in cui tu ne mangerai, tu
morirai.''

---\ldots{} ha lottato. Non potremmo neppure combattere la carestia, le
malattie, le malformazioni, o rendere il mondo un po\textquotesingle{}
migliore di quanto è stato per\ldots{}

--- ``E il serpente disse alla donna: Dio sa che nel giorno in cui voi
ne mangerete, i vostri occhi saranno aperti, e voi sarete come Dei, e
conoscerete il bene e il male.''

---\ldots{} dodici secoli, se ogni via di speculazione deve essere
sbarrata e ogni pensiero nuovo denunciato\ldots{}

--- Non è mai stato migliore, e non sarà mai migliore. Sarà soltanto più
ricco o più povero, più triste, ma non più saggio, fino
all\textquotesingle ultimo giorno.

Lo studioso scrollò le spalle, in segno di impotenza. --- Vedete? Sapevo
che vi sareste offeso, ma mi avevate detto\ldots. Oh, a che serve? Voi
avete la vostra versione!

--- La ``versione'' che io stavo citando, signor Filosofo, non era una
versione del modo della creazione, ma una versione del modo in cui la
tentazione portò alla Caduta. Questo vi è sfuggito? ``E il serpente
disse alla donna\ldots''

--- Sì, sì. Ma la libertà di speculare è essenziale\ldots{}

--- Nessuno ha tentato di privarvene. E nessuno si è offeso. Ma abusare
dell\textquotesingle intelletto per le ragioni
dell\textquotesingle orgoglio e della vanità, o per sfuggire alla
responsabilità, è il frutto dello stesso albero.

--- Mettete in dubbio l\textquotesingle onorabilità dei miei motivi? ---
chiese il Thon oscurandosi.

--- Qualche volta metto in dubbio i miei. Non vi accuso di nulla. Ma
chiedete questo a voi stesso: perché vi prendete diletto di balzare a
una simile bizzarra congettura da un trampolino così fragile? Perché
volete screditare il passato, fino al punto di disumanizzare
l\textquotesingle ultima civiltà? Perché non avete bisogno di imparare
dagli errori degli antichi? O forse non sopportate di essere soltanto un
``riscopritore'' e dovete convincervi di essere anche voi un
``creatore''?

Il Thon sibilò un\textquotesingle imprecazione. --- Questi documenti
dovrebbero essere affidati alle mani di persone competenti --- disse,
incollerito. --- Che ironia è questa!

La luce vacillò e si spense. L\textquotesingle interruzione non era
d\textquotesingle origine meccanica. I novizi che azionavano la dinamo
avevano smesso di lavorare.

--- Portate qualche candela --- gridò l\textquotesingle abate.

Furono portate le candele.

--- Scendete --- disse don Paulo al novizio che stava sulla scaletta.
--- E portate con voi quella cosa. Frate Kornhoer? Frate Korn\ldots{}

--- È andato in magazzino un momento fa, Domne.

--- Bene, chiamatelo. --- Don Paulo si volse di nuovo allo studioso,
porgendogli il documento che era stato trovato fra gli effetti di frate
Claret. --- Leggetelo, se riuscite a decifrarlo alla luce delle candele,
signor Filosofo!

--- Un editto del podestà!

--- Leggetelo, e rallegratevi della vostra diletta libertà.

Frate Kornhoer ritornò nella stanza. Reggeva il pesante crocifisso che
era stato tolto dall\textquotesingle archivolto per fare posto alla
nuova lampada. Porse la croce a don Paulo.

--- Come sapevate che volevo proprio questo?

--- Ho deciso che era venuto il momento, Domne. --- E scrollò le spalle.

Il vecchio sali sulla scaletta e riappese la croce al suo gancio di
ferro. Il corpo scintillò aureo nella luce delle candele.
L\textquotesingle abate si volse e gridò ai suoi monaci: --- Chi leggera
in questa alcova, d\textquotesingle ora innanzi, leggera \emph{ad Lumina
	Christi}!

Quando scese dalla scala, il Thon Taddeo stava già stipando gli ultimi
documenti in una grande cassa, per dividerli più tardi. Guardò
cautamente il prete, ma non disse nulla.

--- Avete letto l\textquotesingle editto?

Lo studioso annuì.

--- Se, per qualche improbabile eventualità, desideraste asilo politico
qui\ldots{}

Lo studioso scosse il capo.

--- Allora posso chiedervi di chiarire la vostra osservazione a
proposito dell\textquotesingle opportunità di porre i nostri documenti
in mani competenti?

II Thon Taddeo abbassò lo sguardo. --- L\textquotesingle ho detto nel
calore della discussione, padre. Lo ritiro.

--- Ma non avete smesso di pensarlo. L\textquotesingle avete sempre
pensato.

Il Thon non lo negò.

--- E allora sarebbe inutile ripetervi la mia supplica perché
intercediate in nostro favore\ldots{} quando i vostri ufficiali diranno
a vostro cugino che splendida guarnigione militare sarebbe questa
abbazia. Ma per il suo stesso bene, ditegli che quando i nostri altari e
i Memorabilia sono stati minacciati, i nostri predecessori non hanno
esitato a resistere con le armi. --- E fece una pausa. --- Partirete
oggi o domani?

--- Credo che sarebbe meglio oggi --- disse sottovoce il Thon Taddeo.

--- Ordinerò di preparare le provviste. --- L\textquotesingle abate si
voltò per andarsene, ma si fermò per aggiungere, gentilmente: --- Ma
quando ritornerete, riferite un messaggio ai vostri colleghi.

--- Naturalmente. Lo avete scritto?

--- No. Dite semplicemente che chiunque desideri studiare qui sarà il
benvenuto, nonostante la scarsa illuminazione. Il Thon Maho,
specialmente. O il Thon Esser Shon con i suoi sei ingredienti. Gli
uomini devono dibattersi per qualche tempo nell\textquotesingle errore
per separarlo dalla verità, io credo\ldots{} purché non afferrino
avidamente l\textquotesingle errore solo perché ha un sapore più
gradevole. Dite loro anche, figlio mio, che quando verrà il tempo, come
sicuramente verrà, in cui non soltanto i preti ma anche i filosofi
avranno bisogno di un rifugio\ldots{} dite loro che le nostre mura sono
robuste.

Fece un cenno di congedo ai novizi, poi risalì le scale, per chiudersi
da solo nel suo studio. Perché la Furia torceva di nuovo le sue viscere,
e sapeva che si avvicinava la tortura.

\emph{Nunc dimittis servum tuum, Domine\ldots{} Quia viderunt oculi mei
	salutare\ldots{}}

Forse questa volta si torcerà fino a staccarsi, pensò, quasi con
speranza. Desiderava chiamare padre Gault perché udisse la sua
confessione, ma decise che sarebbe stato meglio aspettare che gli ospiti
fossero partiti. Fissò di nuovo l\textquotesingle editto.

Un colpo alla porta interruppe la sua sofferenza.

--- Non potete tornare più tardi?

--- Temo che più tardi non sarò più qui --- rispose una voce soffocata
dal corridoio.

--- Oh, Thon Taddeo\ldots{} entrate, allora. --- Don Paulo si raddrizzò;
cercò di dominare la sofferenza, senza tentare di scacciarla, ma
soltanto di controllarla, come avrebbe fatto con un servitore
indisciplinato.

Lo studioso entrò e posò un fascio di carte sulla scrivania
dell\textquotesingle abate. --- Ho pensato che fosse giusto lasciarvi
questi --- disse.

--- Che cosa sono?

--- I disegni delle vostre fortificazioni. Quelli fatti dagli ufficiali.
Vi suggerisco di bruciarli immediatamente.

--- Perché avete fatto questo? --- mormorò don Paulo. --- Dopo quello
che ci siamo detti nel sotterraneo\ldots{}

--- Non fraintendetemi --- interruppe il Thon Taddeo. --- Li avrei
restituiti in ogni caso\ldots{} è una questione d\textquotesingle onore
non approfittare della vostra ospitalità per\ldots{} ma non importa. Se
avessi restituito prima i disegni, gli ufficiali avrebbero avuto il
tempo e la possibilità di prepararne un\textquotesingle altra serie
completa.

L\textquotesingle abate si alzò lentamente e tese la mano verso lo
studioso.

Il Thon Taddeo esitò. --- Non vi prometto alcun tentativo in vostro
favore\ldots{}

--- Lo so.

---\ldots{} perché sono convinto che ciò che avete qui dovrebbe essere
aperto a tutto il mondo.

--- Lo è, lo è sempre stato, e lo sarà sempre.

Si strinsero la mano, imbarazzati, però don Paulo sapeva che non era un
pegno di tregua, ma soltanto di reciproco rispetto fra avversari. Forse
non sarebbe mai stato altro.

Ma perché tutto doveva accadere di nuovo?

La risposta era a portata di mano: c\textquotesingle era ancora il
serpente che sussurrava: Perché Dio sa che nel giorno in cui ne
mangerete, i vostri occhi saranno aperti; e voi sarete come Dei. Il
vecchio padre delle menzogne era abile nel dire mezze verità: Come
conoscerete il bene e il male, fino a quando non ne avrete assaggiato?
Assaggiatene, e diventerete Dei. Ma né l\textquotesingle infinita
potenza né l\textquotesingle infinita sapienza poteva concedere la
divinità agli uomini. Perché sarebbe stato necessario anche
l\textquotesingle infinito amore.

Don Paulo mandò a chiamare il priore. Si avvicinava il momento di
andarsene. E presto sarebbe venuto un nuovo anno.

Fu l\textquotesingle anno di un torrente di pioggia senza precedenti nel
deserto, che fece fiorire semi sepolti da lungo tempo.

Fu l\textquotesingle anno in cui un vestigio di civiltà venne ai nomadi
della Pianura, e persino il popolo di Laredo cominciò a mormorare che
probabilmente tutto andava per il meglio. Nuova Roma non fu
d\textquotesingle accordo.

In quell\textquotesingle anno un temporaneo accordo fu concluso e
spezzato fra gli Stati di Denver e di Texarkana. Fu
l\textquotesingle anno in cui il Vecchio Ebreo ritornò alla sua
precedente vocazione di Medico e di Vagabondo, l\textquotesingle anno in
cui i monaci dell\textquotesingle Ordine Albertiano di Leibowitz
seppellirono un abate e si inchinarono al suo successore.
V\textquotesingle erano splendide speranze per il domani.

Fu l\textquotesingle anno in cui un re venne cavalcando
dall\textquotesingle Est, per sottomettere quella terra e possederla. Fu
un anno dell\textquotesingle Uomo.
