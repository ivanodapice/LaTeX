	\chapter{\phantom{text}}

\lettrine{E}{ra} stata cantata Compieta, ma l\textquotesingle abate rimase in chiesa,
inginocchiato, da solo, nella penombra della sera.

\emph{Domine, mundorum omnium Factor, parsurus esto imprimis eis filiis
	aviantibus ad sideria coeli quorum victus difficilior\ldots{}}

Pregò per il gruppo di frate Joshua\ldots{} per gli uomini che erano
partiti per salire su un\textquotesingle astronave e per scalare i
cieli, in una incertezza più grande di qualsiasi altra mai affrontata
dall\textquotesingle Uomo sulla Terra. Avevano grande bisogno che si
pregasse per loro; nessuno era più suscettibile del viaggiatore ai mali
che affliggono lo spirito per torturare la fede e per fare vacillare un
credo, inondando la mente di dubbi. In patria, sulla Terra, la coscienza
ha i suoi supervisori e i suoi maestri esterni, ma nello spazio la
coscienza era sola, divisa fra il Signore e
l\textquotesingle Avversario. ``Rendili incorruttibili'' pregò ``fai che
si attengano sinceramente alla via dell\textquotesingle Ordine.''

Il dottor Cors lo trovò in chiesa, a mezzanotte, e gli fece cenno di
uscire. Il medico sembrava imbarazzato, fuori di sé.

--- Ho appena infranto la mia promessa! --- dichiarò, in tono di sfida.

L\textquotesingle abate tacque. --- E ne siete orgoglioso? chiese
infine.

--- Non particolarmente.

Si avviarono verso l\textquotesingle unità mobile e si fermarono
nell\textquotesingle onda di luce azzurrina che filtrava
dall\textquotesingle ingresso. Il medico aveva il camice zuppo di
sudore; si asciugò la fronte con la manica. Zerchi lo osservò con la
pietà che si prova per i perduti.

--- Ce ne andremo subito, naturalmente --- disse Cors. --- Ho pensato
che dovevo dirvelo. --- E si girò per entrare nell\textquotesingle unità
mobile.

--- Aspettate un momento --- disse il sacerdote. --- Dovete dirmi il
resto.

--- Davvero? --- Di nuovo quel tono di sfida. --- Perché? Perché
possiate minacciare il fuoco dell\textquotesingle Inferno? È già
abbastanza sofferente, quella ragazza, e anche la sua bambina. Non vi
dirò niente.

--- Me l\textquotesingle avete già detto. So a chi alludete. Anche la
bambina, immagino.

Cors esitò. --- Malattia da radiazione. Ustioni di vampa. La donna ha
un\textquotesingle anca spezzata. Il padre è morto. Le otturazioni dei
denti della donna sono radioattive. La bambina, quasi splende, nel buio.
Il vomito è cominciato subito dopo l\textquotesingle esplosione. Nausea,
anemia, follicoli putrefatti. Cieca di un occhio. La bambina piange
continuamente perle ustioni. Come siano riuscite a sopravvivere è
difficile da capire. Non posso far nulla per loro, se non mandarle alla
squadra Eucrem.

--- Le ho viste.

--- E allora sapete perché ho infranto la mia promessa. Dovrò vivere
dopo! E non voglio vivere come il torturatore di quella donna e di
quella bambina!

--- È più piacevole vivere come il loro assassino, invece?

--- Con voi non si può fare una discussione ragionevole.

--- Che cosa avete detto a quella ragazza?

--- ``Se volete bene alla vostra bambina, risparmiatele questa
sofferenza. Andate a dormire, misericordiosamente, più presto che
potete.'' Ecco tutto. Ce ne andremo immediatamente. Abbiano finito con i
casi di radiazione e con gli altri casi più gravi. Agli altri non farà
male camminare per qualche chilometro. Non vi sono più casi di
dose-critica.

Zerchi si allontanò, poi si fermò e si voltò. --- Finite quel che dovete
fare --- gracchiò. --- Finite e poi andatevene. Se vi rivedrò
ancora\ldots{} ho paura di quello che potrei fare.

Cors sputò. --- Non mi piace stare qui più di quanto a voi piaccia
tenermi. Ce ne andremo subito, grazie.

Trovò la donna distesa su una branda, insieme alla bambina, nel
corridoio della foresteria sovraffollata. Erano ammucchiate insieme
sotto una coperta, e piangevano. L\textquotesingle edificio odorava di
morte e di antisettico. La ragazza alzò lo sguardo verso la vaga figura
che si profilava contro la luce.

--- Padre? --- La sua voce era spaventata.

--- Sì.

--- Siamo spacciate. Vedete? Vedete cosa mi hanno dato?

Non riuscì a vedere nulla, ma udì le dita di lei che facevano scattare
l\textquotesingle orlo del cartoncino. Il biglietto rosso. Non riuscì a
trovare voce per parlare. Si avvicinò alla branda. Si frugò nelle tasche
e ne tolse un rosario. Lei udì il tintinnio dei grani e tese la mano per
afferrarlo.

--- Sapete che cos\textquotesingle è?

--- Certamente, padre.

--- E allora tenetelo e usatelo.

--- Grazie.

--- Portatelo e pregate.

--- So quello che devo fare.

--- Non rendetevi complice. Per l\textquotesingle amore di Dio,
figliola, non\ldots{}

--- Il dottore ha detto\ldots{}

Si interruppe. Zerchi aspettò che finisse; ma lei continuò a tacere.

--- Non rendetevi complice.

Lei tacque ancora. Zerchi le benedisse, più in fretta che poté. La donna
aveva toccato i grani del rosario con dita che li conoscevano; non
poteva dirle nulla che lei non sapesse già.

"La conferenza dei ministri degli Esteri, a Guam, si è appena conclusa.
Non è stato ancora emesso alcun comunicato politico congiunto. I
ministri stanno ritornando alle rispettive capitali.
L\textquotesingle importanza di questa conferenza e
l\textquotesingle ansia con cui il mondo attende i risultati, inducono
questo commentatore a credere che la conferenza non sia ancora
terminata, ma soltanto sospesa perché i ministri degli Esteri possano
conferire per qualche giorno con i rispettivi governi. Un precedente
rapporto secondo il quale la conferenza sarebbe stata interrotta in
mezzo a feroci invettive è stato smentito dai ministri. Il primo
ministro Rekol ha rilasciato una sola dichiarazione alla stampa:
\emph{Torno} \emph{in patria per discutere con il Consiglio di Reggenza.
	Ma il tempo è stato bello, qui; può darsi che ci torni più tardi, a
	pescare}.

"Il periodo di tregua di dieci giorni scade oggi, ma si ritiene
generalmente che il cessate il fuoco continuerà a essere osservato.
L\textquotesingle unica alternativa sarebbe
l\textquotesingle annientamento reciproco. Due città sono state
distrutte, ma bisogna ricordare che nessuna delle due parti ha risposto
con un attacco in massa. I governanti asiatici affermano che si è
applicato il principio \emph{dell\textquotesingle occhio per occhio}. Il
nostro governo insiste che l\textquotesingle esplosione di Itu Wan non è
stata causata da un missile atlantico. Ma, per lo più, entrambe le
capitali mantengono uno strano silenzio riflessivo. Rare le voci che
chiedono una vendetta totale. Una specie di stolto furore, perché
l\textquotesingle assassinio è stato commesso, perché la pazzia regna e
prevale, ma nessuna delle due parti vuole la guerra totale. La Difesa
rimane all\textquotesingle erta. Lo Stato Maggiore ha emanato un
annuncio, quasi un appello, che dichiara che noi non ci spingeremo fino
in fondo, se l\textquotesingle Asia farà lo stesso. Ma
l\textquotesingle annuncio dice, più avanti: \emph{Se useranno il
	fallout sporco, risponderemo nello stesso modo, e con tale potenza che
	nessuna creatura potrà vivere in Asia per mille anni!}.

"Cosa strana, la nota più scoraggiante non viene da Guam ma dal.
Vaticano di Nuova Roma. Dopo la conclusione della Conferenza di Guam, si
è risaputo che papa Gregorio aveva smesso di pregare per la pace nel
mondo. Due messe speciali venivano cantate nella basilica: la
\emph{Exsurge quare obdormis}, che è la messa contro gli infedeli, e la
\emph{Reminiscere}, la messa in tempo di guerra. Successivamente, il
rapporto afferma che Sua Santità si è ritirato sulle montagne per
meditare e per pregare per la giustizia.

``E ora una parola da\ldots''

--- Spegnete --- ruggì Zerchi.

Il giovane prete che era con lui spense l\textquotesingle apparecchio e
fissò l\textquotesingle abate a occhi sbarrati. --- Non ci credo!

--- Che cosa? Ciò che hanno detto del papa? Non ci credo neppure io. Ma
l\textquotesingle avevo già sentito, prima, e Nuova Roma aveva avuto il
tempo di smentirlo. Ma non ha detto una parola.

--- Che cosa significa?

--- Non è ovvio? Il servizio diplomatico vaticano è al lavoro. È
evidente che hanno inoltrato un rapporto sulla conferenza di Guam. Ed è
evidente che quel rapporto ha inorridito il Santo Padre.

--- Che ammonimento! Che gesto!

--- È stato più di un gesto, padre. Sua Santità non dice Messe di
Battaglia solo per amore di un effetto drammatico. Inoltre, molta gente
penserà che egli intenda ``contro gli infedeli''
sull\textquotesingle altra riva dell\textquotesingle oceano, la
``giustizia'' della nostra parte. O, anche se la pensassero
diversamente, continueranno a crederlo. --- Si nascose la faccia fra le
mani, se la massaggiò. --- Dormire. Che cos\textquotesingle è dormire,
padre Lehy? Voi lo ricordate? Non ho visto un viso umano, in dieci
giorni, che non avesse cerchi neri sotto gli occhi. Io sono riuscito a
malapena a sonnecchiare, ieri notte, perché c\textquotesingle era
qualcuno che gridava, nella foresteria.

--- Lucifero non è l\textquotesingle uomo del sonno, lo sappiamo.

--- Cosa state guardando, fuori dalla finestra? --- chiese Zerchi con
voce tagliente. --- Ecco un\textquotesingle altra cosa. Tutti continuano
a guardare il cielo, a guardare e a pensare. Se la bomba arriva, non
avrete il tempo di vederla fino all\textquotesingle esplosione, e allora
farete meglio a non guardare. Smettetela. È malsano.

Padre Lehy si allontanò dalla finestra. --- Sì, Reverendo Padre. Ma non
stavo guardando per quello. Guardavo le poiane.

--- Le poiane?

--- Ce ne sono state moltissime, tutto il giorno. Dozzine di
poiane\ldots{} che volavano in cerchio.

--- Dove?

--- Sul campo della Stella Verde, sull\textquotesingle autostrada.

--- Allora non è un segno di cattivo augurio. È soltanto sano appetito
d\textquotesingle avvoltoio. Ah! Vado fuori a prendere un
po\textquotesingle{} d\textquotesingle aria.

Nel cortile incontrò la signora Grales. Portava un cesto di pomodori,
che posò in terra quando lo vide avvicinarsi. --- Vi ho portato
qualcosa, padre Zerchi --- gli disse. --- Ho visto che avete tolto la
scritta, e poi ho visto qualche povera ragazza dentro al portone, così
ho pensato che non vi sarebbe spiaciuta una visita della vecchia donna
dei pomodori, Ho portato un po\textquotesingle{} di pomodori, vedete?

--- Grazie, signora Grales. Abbiamo tolto la scritta per i profughi, ma
non importa. Dovrete parlare con frate Elton, però, per i pomodori. È
lui che fa gli acquisti per la cucina.

--- Oh, non li vendo, padre. Eh-eh! Ve li ho portati gratis. Avete tanta
gente da sfamare, tutti i poveracci che raccogliete qui. Così, sono
gratis. Dove devo metterli?

--- La cucina d\textquotesingle emergenza è nel\ldots{} ma no,
lasciateli lì. Li farò portare alla foresteria.

--- Li porto io, padre. Li ho portati fin qui. --- E riprese il
canestro.

--- Grazie, signora Grales. --- E si volse per andarsene.

--- Padre, aspettate --- chiamò la donna. --- Un minuto, vostro onore,
soltanto un minuto\ldots{}

L\textquotesingle abate represse un brontolio. --- Mi dispiace, signora
Grales, ma è come vi ho detto\ldots{} --- Si interruppe, fissò il viso
di Rachel. Per un momento, aveva immaginato\ldots{} Aveva ragione frate
Joshua? Ma no, sicuramente no. --- È\ldots{} è una cosa che riguarda la
vostra parrocchia e la diocesi, e non c\textquotesingle è nulla che io
possa\ldots{}

--- Oh, no, padre, non quello! --- disse la donna. --- Volevo chiedervi
qualcosa d\textquotesingle altro. --- (Ecco! Aveva sorriso! Ne era
certo!) --- Vorreste ascoltare la mia confessione,. padre? Vi chiedo
perdono per il disturbo, ma io sono pentita dei miei peccati, e vorrei
che foste voi ad assolvermi.

Zerchi esitò. --- Perché non padre Selo?

--- Vi dirò la verità, vostro onore, è che quell\textquotesingle uomo è
un\textquotesingle occasione di peccato per me. Vado da lui pensando
tutto il bene possibile, ma basta che lo guardi in faccia una volta, e
mi dimentico di me stessa. Dio lo ama, ma io non posso.

--- Se lui vi ha offesa, dovete perdonarlo.

--- Perdonare, è quello che faccio, è quello che faccio. Ma a buona
distanza. È un\textquotesingle occasione di peccato per me, vi dico,
perché io perdo la calma appena lo vedo.

Zerchi ridacchiò. --- D\textquotesingle accordo, signora Grales.
Ascolterò la vostra confessione, ma prima ho qualcosa da fare.
Aspettatemi nella Cappella di Nostra Signora fra
mezz\textquotesingle ora. Primo confessionale. Va bene?

--- Sì, e siate benedetto, padre! --- La donna chinò più volte il capo.
L\textquotesingle abate Zerchi avrebbe giurato che anche la testa di
Rachel ripeteva quei cenni, sia pure molto lievemente.

Scacciò quel pensiero e si diresse verso il garage. Un postulante gli
portò fuori la macchina. Vi salì, regolò il quadrante per fissare la
destinazione, e si abbandonò, stancamente, sui cuscini, mentre i
controlli automatici mettevano in moto i meccanismi e spingevano la
macchina verso il portone. Mentre varcava il portone,
l\textquotesingle abate vide la ragazza ritta sul ciglio della strada.
Aveva con sé la bambina. Zerchi premette il pulsante CANCELLATO. La
macchina si fermò. ``In attesa'' disse il comando automatico.

La ragazza aveva una ingessatura che la chiudeva dalla cintura al
ginocchio sinistro. Si appoggiava a un paio di grucce e ansimava. In
qualche modo era riuscita ad allontanarsi dalla foresteria e a varcare
il portone, ma era evidentemente incapace di andare oltre. La bambina si
aggrappava a una delle grucce e guardava il traffico
sull\textquotesingle autostrada. Zerchi apri la porta della macchina e
ne scese, lentamente. La ragazza lo guardò, poi distolse in fretta lo
sguardo.

--- Cosa state facendo fuori dal letto, figliola? --- mormorò. --- Non
dovreste alzarvi, con quell\textquotesingle anca spezzata. Dove
credevate di andare?

Lei spostò il proprio peso sull\textquotesingle altra gruccia, e il suo
viso si torse per la sofferenza. --- In città --- disse. --- Devo
andare. È urgente.

--- Non è così urgente che non possa andare qualcuno, al posto vostro.
Manderò frate\ldots{}

--- No, padre, no! Non può farlo nessuno, per me. Devo andare in città.

Mentiva. Zerchi era certo che mentiva. --- Benissimo, allora --- disse.
--- Vi porterò io in città. Ci stavo andando in ogni caso.

--- No, andrò a piedi! Io\ldots{} --- Fece un passo e boccheggiò. Zerchi
la sorresse prima che cadesse a terra.

--- Neppure se san Cristoforo vi tenesse le grucce, potreste arrivare
fino in città, figliola. Venite, su, vi riconduco a letto.

--- Devo andare in città, vi ho detto! --- gridò lei, incollerita. La
bambina, spaventata dalla collera della madre, cominciò un pianto
monotono. La ragazza cercò di calmarla, ma poi scattò.

--- E sta bene, padre. Mi portate in città?

--- Non dovreste andarci.

--- Ve l\textquotesingle ho detto, devo andare!

--- Benissimo, allora. Lasciate che vi aiuti a salire\ldots{} la
piccolina\ldots{} e adesso voi.

La bambina gridò istericamente quando il religioso la sollevò e la mise
nella macchina, accanto alla madre; le si aggrappò e ricominciò il suo
monotono singhiozzare. A causa dei vestiti sciolti e umidi e dei capelli
bruciati, era difficile stabilire il sesso della creaturina, ma
l\textquotesingle abate Zerchi sapeva che era una bimba.

Formulò di nuovo la destinazione. La macchina aspettò una pausa nel
traffico, poi svoltò nell\textquotesingle autostrada, nella corsia della
media velocità. Due minuti dopo, quando si avvicinarono
all\textquotesingle accampamento della Stella Verde, la fece dirigere
verso la corsia più lenta.

Cinque monaci stavano davanti all\textquotesingle accampamento, come un
solenne picchetto incappucciato. Camminavano avanti e indietro, in
processione, sotto l\textquotesingle insegna del Campo di Misericordia,
ma badavano a stare nella zona aperta al pubblico. I loro cartelli,
dipinti a fresco, dicevano:

\begin{center}
	LASCIATE OGNI SPERANZA
\end{center}

\begin{center}
	O VOI CHE ENTRATE
\end{center}

Zerchi aveva avuto intenzione di fermarsi per parlare con loro, ma
poiché aveva a bordo la ragazza si accontentò di guardarli mentre
passavano. Con i loro abiti, i cappucci e il lento procedere funebre, i
novizi creavano veramente l\textquotesingle effetto desiderato. Era
piuttosto improbabile che la Stella Verde si sentisse così imbarazzata
da spostare il campo lontano dal monastero, specialmente perché una
piccola folla di dimostranti si era presentata, qualche tempo prima, per
gridare insulti e scagliare sassi contro i cartelli portati dai
picchettanti. C\textquotesingle erano due macchine della polizia, ferme
sul fianco dell\textquotesingle autostrada, e parecchi agenti erano lì,
a osservare la scena con facce inespressive. Poiché la folla di
dimostranti era apparsa all\textquotesingle improvviso, e poiché le
macchine della polizia erano comparse subito dopo, in tempo per vedere
un dimostrante che cercava di strappare un cartello al picchettante, e
poiché un funzionario della Stella Verde era andato a procurarsi un
ordine del tribunale, l\textquotesingle abate sospettò che la
dimostrazione fosse stata accuratamente organizzata quanto i picchetti,
per mettere in grado il funzionario della Stella Verde di ottenere il
mandato. Probabilmente il mandato sarebbe stato concesso, ma fino a che
non fosse stato consegnato, l\textquotesingle abate Zerchi intendeva
lasciare i novizi al loro posto.

Guardò la statua che gli operai avevano eretto accanto
all\textquotesingle ingresso; e rabbrividì. La riconobbe per una delle
composite immagini umane derivate da sondaggi psicologici di massa, in
cui si sottoponevano ai soggetti disegni e fotografie di persone
sconosciute e si domandava: ``Quale preferireste conoscere?'' e ``Quale,
secondo voi, sarebbe il genitore migliore?'' oppure: ``Quale vorreste
evitare?'' o ``Quale è il criminale, secondo voi?''. Dalle fotografie
scelte come ``più'' o ``meno'' a seconda delle domande, dai risultati
del sondaggio di massa, era stata costruita, per mezzo di calcolatori,
una serie di ``facce medie'', ciascuna delle quali evocava un giudizio
di personalità a prima vista.

Questa statua, notò Zerchi con sbigottimento, aveva una notevole
somiglianza con alcune delle immagini più effeminate con cui gli artisti
mediocri o peggio che mediocri avevano rappresentato tradizionalmente la
personalità di Cristo. Il viso dolciastro, gli occhi vacui, le labbra
dalla posa affettata, le braccia tese in un gesto di abbraccio. I
fianchi erano larghi come quelli di una donna, e il petto recava un
accenno di seni\ldots{} a meno che non fossero pieghe della tunica.
``Caro Iddio del Golgota'' mormorò l\textquotesingle abate Zerchi ``è
così che la canaglia Ti immagina?'' Con qualche sforzo, poteva
immaginare quella statua dire ``Lasciate che i piccoli vengano a me'',
ma non poteva immaginare che dicesse ``Allontanatevi da me nel fuoco
eterno, maledetti'', o che scacciasse a sferzate i mercanti dal Tempio.
Quale domanda, si chiese, avevano rivolto ai soggetti, per evocare nella
mente della canaglia quel viso composito? Era solo, anonimamente, un
Cristo. La legenda sul piedistallo diceva CONFORTO. Ma senza dubbio la
Stella Verde doveva averne veduto la somiglianza con il tradizionale
Cristo aggraziato degli artisti meschini. Però l\textquotesingle avevano
caricata su un camion con una bandiera rossa legata
all\textquotesingle alluce, e sarebbe stato difficile provare che la
somiglianza era intenzionale.

La ragazza aveva posato una mano sulla maniglia della portiera; fissava
i comandi della macchina. Zerchi regolò i quadranti su CORSIA VELOCE. La
macchina si lanciò in avanti. La ragazza tolse la mano dalla portiera.

--- Ci sono molte poiane, oggi --- disse Zerchi, quietamente, guardando
il cielo, oltre il finestrino.

La ragazza rimase seduta, con il viso privo di espressione. Per un
momento l\textquotesingle abate studiò quel viso. --- Soffrite,
figliola?

--- Non importa.

--- Offrite la vostra sofferenza al Cielo, figliola.

Lei lo guardò freddamente. --- Credete che Dio ne sarebbe compiaciuto?

--- Se voi gliela offrite, sì.

--- Non riesco a comprendere un Dio che si compiace delle sofferenze
della mia bambina!

Il religioso rabbrividì. --- No, no! Non è della sofferenza che Dio si
compiace, figliola. Si compiace della perseveranza
dell\textquotesingle anima nella fede e nella speranza e
nell\textquotesingle amore, nonostante le afflizioni del corpo. La
sofferenza è come una tentazione negativa. Dio non si compiace delle
tentazioni che affliggono la carne; si compiace quando
l\textquotesingle anima si leva al di sopra delle tentazioni e dice
``Vai, Satana''. E lo stesso è per la sofferenza, che è spesso una
tentazione alla disperazione, all\textquotesingle ira, alla perdita
della fede\ldots{}

--- Risparmiate il fiato, padre. Non mi lamento. È la bambina che si
lamenta. La bambina non può capire il suo sermone. Può soffrire, però.
Può soffrire, ma non può capire.

``Che cosa posso risponderle?'' si chiese stordito il religioso. ``Devo
dirle di nuovo che un tempo l\textquotesingle Uomo aveva avuto in dono
una impossibilità preternaturale, e che la gettò via
nell\textquotesingle Eden? Che la bambina era una cellula di Adamo, e
che di conseguenza\ldots{} Era vero, ma quella ragazza aveva una figlia
ammalata, e lei stessa era malata, e non avrebbe ascoltato.''

--- Non fatelo, figliola. Non fatelo.

--- Ci penserò --- disse lei, freddamente.

--- Avevo un gatto, una volta, quando ero bambino --- mormorò lentamente
l\textquotesingle abate. --- Era un grosso gattone grigio con le spalle
come quelle d\textquotesingle un mastino e una testa e un collo
altrettanto solidi, e quella insolenza che li fa somigliare, qualche
volta, a creature dei Diavolo. Era un autentico gatto. Conoscete i
gatti?

--- Un po\textquotesingle.

--- Quelli che amano i gatti non li conoscono. Non è possibile amare
tutti i gatti se li si conosce, e quelli che potete amare se li
conoscete sono quelli che non piacciono alla gente che ama i gatti. Zeke
era un gatto di questo genere.

--- E questo ha una morale, naturalmente? --- La ragazza lo osservava,
insospettita.

--- Solo questa: che l\textquotesingle ho ucciso.

--- Tacete. Qualunque cosa stiate per dire, tacete.

--- Un camion lo investì, gli fracassò le zampe posteriori. Si trascinò
fino a casa e si infilò tra le cianfrusaglie del garage. Ogni tanto
emetteva un suono, come fanno i gatti quando litigano, e si agitava un
po\textquotesingle, ma per lo più se ne stava quietamente sdraiato, e
aspettava. ``Bisognerebbe ucciderlo'' continuavano a dirmi. Dopo qualche
ora, si trascinò fuori dal garage. Piangeva, per invocare aiuto.
``Bisognerebbe ucciderlo'' dicevano. Non volevo che lo facessero. E mi
risposero che era una crudeltà lasciarlo vivo. Così alla fine dissi che
l\textquotesingle avrei fatto, se era necessario. Presi un fucile e un
badile e lo portai sull\textquotesingle orlo del bosco. Lo stesi sul
terreno, mentre scavavo una fossa. Poi gli sparai alla testa. Era un
fucile di piccolo calibro. Zeke sussultò un paio di volte, poi si alzò e
cominciò a trascinarsi verso un cespuglio. Gli sparai ancora. Il colpo
lo stese secco, così pensai che fosse morto, e lo deposi nella fossa.
Dopo un paio di palate di terra, Zeke si alzò e si trascinò fuori della
buca e ricominciò a trascinarsi verso i cespugli. Io piangevo più forte
del gatto. Dovetti ucciderlo con il badile. Dovetti rimetterlo nella
fossa e usare la lama del badile come una mannaia, e mentre lo colpivo,
Zeke continuava ad agitarsi. Mi dissero che era solo un riflesso
spinale, ma io non lo credetti. Conoscevo quel gatto. Voleva arrivare a
quei cespugli e distendersi lì, ad aspettare. Desiderai di avergli
lasciato raggiungere quei cespugli, e morire come morirebbe un gatto se
lo si lascia in pace\ldots{} con dignità. Non lo dimenticai più. Zeke
era solo un gatto, ma\ldots{}

--- State zitto! --- sussurrò la ragazza.

---\ldots{} ma anche gli antichi pagani osservavano che la Natura non ci
impone niente che non ci abbia messo in grado di sopportare. Se questo è
vero per un gatto, allora non è forse ancora più vero per una creatura
dotata di volontà e di intelletto razionale\ldots{} qualunque cosa possa
pensare del Cielo?

--- State zitto, maledizione, state zitto! --- sibilò lei.

--- Se sono un po\textquotesingle{} brutale --- disse il religioso ---
lo sono verso di voi, non verso la bambina. La bambina, come dite voi,
non può capire. E voi, come avete detto, non vi lamentate. Di
conseguenza\ldots{}

--- Di conseguenza mi chiedete di lasciarla morire lentamente e\ldots{}

--- No! Non ve lo chiedo. Come prete di Cristo io vi comando, per
l\textquotesingle autorità di Dio Onnipotente, di non alzare la mano
sulla vostra bambina, di non offrire la sua vita in sacrificio a un
falso dio di sbrigativa misericordia. Io non vi consiglio, vi scongiuro
e vi comando in nome di Cristo Re. \emph{È chiaro?}

Don Zerchi non aveva mai parlato con quel tono, prima
d\textquotesingle allora, e la facilità con cui le parole gli venivano
alle labbra sorpresero persino lui. Mentre continuava a guardarla, lei
abbassò gli occhi. Per un momento, l\textquotesingle abate aveva temuto
che la ragazza gli ridesse in faccia. Quando la Santa Chiesa faceva
capire, di tanto in tanto, che considerava assoluta la propria autorità
sulle nazioni, superiore all\textquotesingle autorità degli Stati, gli
uomini, in quei tempi, tendevano a sghignazzare. Eppure
l\textquotesingle autenticità del comando poteva ancora essere sentita
da una ragazza amareggiata che aveva una figlia morente. Era stata una
brutalità tentare di ragionare con lei, e gli dispiaceva. Un semplice
comando diretto poteva ottenere ciò che non poteva la persuasione.
Adesso aveva bisogno della voce dell\textquotesingle autorità, più di
quanto avesse bisogno di persuasione. Lo poté capire dal modo in cui la
ragazza aveva sussultato, sebbene lui avesse formulato il comando con
tutta la dolcezza di cui la sua voce era capace.

Raggiunsero la città. Zerchi si fermò per impostare una lettera, si
fermò alla chiesa di san Michele per parlare qualche minuto con padre
Selo del problema degli sfollati, si fermò alla Difesa Interna di Zona
per prendere una copia delle ultime direttive in materia di difesa
civile. Ogni volta che ritornava alla macchina, quasi si aspettava di
non trovarvi più la ragazza, ma lei se ne stava tranquilla, stringendo
la piccina e fissando distratta nel vuoto.

--- Mi volete dire dove dovete andare, figliola? --- le chiese, alla
fine.

--- In nessun posto. Ho cambiato idea.

L\textquotesingle abate sorrise. --- Ma avevate bisogno urgente di
andare in città.

--- Lasciate perdere, padre. Ho cambiato idea.

--- Bene, allora torneremo a casa. Perché non lasciate che le sorelle si
prendano cura della bambina, per qualche giorno?

--- Ci penserò.

La macchina accelerò, sull\textquotesingle autostrada, verso
l\textquotesingle abbazia. Come si avvicinarono al campo della Stella
Verde, l\textquotesingle abate poté vedere che qualcosa non andava. I
picchettanti non stavano più marciando. Si erano raccolti in gruppo e
stavano parlando agli agenti --- o li ascoltavano --- e a un terzo uomo
che Zerchi non poté identificare. Diresse la macchina sulla corsia
lenta. Uno dei novizi vide la macchina, la riconobbe, e cominciò ad
agitare il cartello. Don Zerchi non aveva intenzione di fermarsi, poiché
aveva a bordo la ragazza, ma uno degli agenti uscì sulla corsia del
traffico lento e puntò la paletta verso i detector della macchina;
l\textquotesingle autopilota reagì automaticamente e fermò il veicolo.
L\textquotesingle agente fece cenno di portare la macchina sul ciglio
della strada. Zerchi non poteva disobbedire. I due agenti si
avvicinarono, annotarono il numero della patente e chiesero i documenti.
Uno di loro guardò, incuriosito, la ragazza e la bambina, notò i
biglietti rossi. L\textquotesingle altro fece un cenno in direzione
della fila, ora immobile, dei picchettanti.

--- Dunque siete voi che avete organizzato tutto, non è vero? ---
grugnì, rivolto all\textquotesingle abate. --- Bene, quel signore dalla
tunica bruna, laggiù, ha qualche notizia per voi. Credo che farete
meglio ad ascoltarlo. --- E indicò con il capo un tipo grassoccio che
avanzava pomposamente verso di loro.

La bambina aveva ricominciato a piangere. La madre si agitava,
irrequieta.

--- Agenti, questa ragazza e questa bambina non stanno bene. Accetterò
il processo, ma per favore, lasciateci ritornare
all\textquotesingle abbazia. Poi verrò qui da solo.

L\textquotesingle agente guardò di nuovo la ragazza. Signora\ldots{}

Lei si volse verso il campo e alzò gli occhi alla statua che torreggiava
all\textquotesingle ingresso. --- Io scendo qui--- disse, con voce
incolore.

--- Sarà bene che scendiate, signora ---disse l\textquotesingle agente
guardando di nuovo i biglietti rossi.

--- No! --- Don Zerchi l\textquotesingle afferrò per il braccio. ---
Figliola, vi proibisco\ldots{}

La mano dell\textquotesingle agente scattò per afferrare il polso del
religioso. --- Lasciatela andare! --- proruppe; poi, con voce sommessa:
--- Signora, siete affidata a lui o qualcosa di simile?

--- No.

--- E allora perché proibite alla signora di scendere? --- domandò
l\textquotesingle agente. --- Abbiamo già perduto un
po\textquotesingle{} la pazienza con voi, caro signore, e sarà meglio
che\ldots{}

Zerchi lo ignorò e parlò rapidamente alla ragazza. Lei scosse il capo.

--- La piccina, allora. Lasciate che porti la piccina alle sorelle.
Insisto\ldots{}

--- Signora, la bambina è vostra --- chiese l\textquotesingle agente. La
ragazza era già scesa dalla macchina, ma Zerchi teneva ancora la
bambina.

La ragazza annuì. --- È mia.

--- Quest\textquotesingle uomo l\textquotesingle ha tenuta prigioniera o
qualcosa di simile?

--- No.

--- Cosa volete fare, signora?

Lei si fermò.

--- Risalite in macchina --- le disse don Zerchi.

--- Abbassate quel tono di voce, caro signore --- abbaiò
l\textquotesingle agente. --- Signora, cosa decidete per la bambina?

--- Scendiamo qui, tutte e due --- disse lei.

Zerchi sbatté la portiera e cercò di rimettere in moto la macchina, ma
la mano dell\textquotesingle agente scattò dal finestrino, premette il
pulsante CANCELLATO e tolse la chiavetta.

--- Tentato rapimento? --- brontolò un agente, rivolto
all\textquotesingle altro.

--- Forse, disse l\textquotesingle altro, e aprì la portiera. --- Adesso
lasciate andare la bambina!

--- Perché venga assassinata qui? --- chiese l\textquotesingle abate.
--- Dovrete usare la forza per riprenderla!

--- Vai dall\textquotesingle altra parte della macchina, Fal.

--- No!

--- Adesso infila il bastone sotto l\textquotesingle ascella. Ecco,
tira! Benissimo, signora\ldots{} ecco la vostra piccina. No, credo che
non riuscirete a portarla, con quelle grucce. Cors?
Dov\textquotesingle è Cors? Ehi, dottore!

L\textquotesingle abate Zerchi intravide un viso familiare che si
avvicinava, in mezzo alla folla.

--- Portate via la piccina mentre noi teniamo questo matto, vi spiace?

Il medico e il religioso si scambiarono un\textquotesingle occhiata
silenziosa, poi la piccina fu tolta dalla macchina. Gli agenti
lasciarono i polsi dell\textquotesingle abate. Uno di essi si girò e si
trovò bloccato dai novizi con i cartelli levati. Interpretò i cartelli
come armi potenziali, e la mano gli cadde sulla pistola. --- Indietro!
--- urlò.

Sconvolti, i novizi indietreggiarono.

--- Scendete.

L\textquotesingle abate scese dalla macchina. Si trovò di fronte al
grassoccio ufficiale giudiziario. Quest\textquotesingle ultimo gli batté
su un braccio con una carta ripiegata. --- Il tribunale mi chiede di
leggervi e di spiegarvi la seguente ordinanza. Questa è la vostra copia.
Gli agenti testimoniano che vi è stata consegnata, così non potrete
opporre resistenza.

--- Oh, date qua.

--- Così va bene. Ora, il tribunale vi ordina quanto segue: ``Poiché è
stata presentata una lamentela, affermante che un grave turbamento
dell\textquotesingle ordine pubblico è stato\ldots''

--- Buttate i cartelli in quel barile, laggiù --- disse Zerchi ai novizi
--- a meno che qualcuno non faccia obiezione. Poi salite in macchina e
aspettate. --- Non prestò attenzione alla lettura ma si avvicinò agli
agenti mentre l\textquotesingle ufficiale giudiziario lo seguiva,
leggendo con voce monotona. --- Sono in arresto?

--- Ci stiamo pensando.

--- ``\ldots{} e presentarsi davanti a questo tribunale nella data
predetta per la causa, poiché una ingiunzione\ldots''

--- C\textquotesingle è qualche accusa particolare?

--- Potremmo elevare quattro o cinque accuse, se volete.

Cors ritornò. La donna e la piccina erano state scortate nel campo.
L\textquotesingle espressione del medico era seria, se non colpevole.

--- Ascoltate, padre --- disse. --- So cosa provate davanti a tutto
questo, ma\ldots{}

Il pugno dell\textquotesingle abate Zerchi colpì il medico in pieno
viso. Cors fu colto alla sprovvista, e cadde a sedere sul viottolo.
Sembrava sbalordito. Tirò su con il naso un paio di volte.
All\textquotesingle improvviso, cominciò a perdere sangue dal naso. I
poliziotti avevano bloccato il religioso per le braccia.

--- ``E di conseguenza non manchi di presentarsi'' continuò a blaterare
l\textquotesingle ufficiale giudiziario ``altrimenti un decreto pro
confesso\ldots''

--- Portalo alla macchina --- disse uno degli agenti.

La macchina verso la quale l\textquotesingle abate fu spinto non era la
sua ma quella della polizia.

--- Il giudice sarà un po\textquotesingle{} deluso da voi --- gli disse,
acido, l\textquotesingle agente. --- Adesso state qui e restate
tranquillo. Ancora una mossa e vi spedisco in prigione.

L\textquotesingle abate e l\textquotesingle agente attesero accanto alla
macchina mentre l\textquotesingle ufficiale giudiziario, il medico e
l\textquotesingle altro agente discutevano sul viottolo. Cors si premeva
sul naso un fazzoletto.

Parlarono per cinque minuti. Pieno di vergogna, Zerchi premette la
fronte contro il metallo della macchina e cercò di pregare. Gli
importava poco, per il momento, ciò che potevano decidere. Riusciva a
pensare soltanto alla ragazza e alla bambina. Era certo che lei era già
pronta a cambiare idea, aveva bisogno soltanto del comando ``Io, prete
di Dio, ti scongiuro'' e della grazia di ascoltarlo\ldots{} se non
l\textquotesingle avessero costretto a fermarsi dove lei aveva potuto
vedere ``il prete di Dio'' sommariamente sopraffatto ``dai poliziotti di
Cesare''. Anche a lui, la Regalità di Cristo non era mai sembrata così
lontana.

--- Benissimo, caro signore. Siete proprio un uomo fortunato.

Zerchi alzò lo sguardo. --- Cosa?

--- Il dottor Cors rifiuta di firmare una denuncia. Dice che se
l\textquotesingle è voluta lui. Perché l\textquotesingle avete colpito?

--- Chiedeteglielo.

--- Gliel\textquotesingle abbiamo chiesto. Sto solo cercando di decidere
se devo portarvi via o limitarmi a darvi la citazione.
L\textquotesingle ufficiale giudiziario dice che siete conosciuto, da
queste parti. Che cosa fate?

Zerchi arrossì. --- Questo non vi dice niente? --- E si toccò la croce
sul petto.

--- No, quando chi la porta prende a pugni qualcuno. Che cosa fate?

Zerchi ringoiò l\textquotesingle ultima traccia di orgoglio. --- Sono
l\textquotesingle abate dei frati di san Leibowitz,
all\textquotesingle abbazia che vedete là lungo la strada.

--- E questo vi autorizza ad aggredire la gente?

--- Mi dispiacer Se il dottor Cors vorrà ascoltarmi, mi scuserò con lui.
Se mi consegnate una citazione, prometto che mi presenterò.

--- Fal?

--- La prigione è già piena di sfollati.

--- Sentite, se dimentichiamo questa storia, starete alla larga da qui,
e terrete a freno la vostra banda?

--- Sì.

--- Benissimo. Andatevene. Ma se fate tanto di passare di qui e di
sputare, ve ne pentirete!

--- Grazie.

Un organetto stava suonando, nel parco, mentre si allontanavano. E,
guardandosi indietro, Zerchi vide che la giostra stava girando. Un
agente si asciugò la faccia, batté la mano sulla spalla
dell\textquotesingle ufficiale giudiziario, e tutti ritornarono alle
rispettive macchine e se ne andarono. Anche se sulla macchina
c\textquotesingle erano cinque novizi, Zerchi era solo con la sua
vergogna.
