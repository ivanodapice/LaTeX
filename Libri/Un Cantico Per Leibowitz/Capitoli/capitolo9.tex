	\chapter{\phantom{title}}

\lettrine{P}{ochi} mesi dopo la partenza di monsignor Aguerra, venne
all\textquotesingle abbazia, da Nuova Roma, una seconda carovana di
asinelli, con una scorta completa di religiosi e di guardie armate
contro il pericolo dei briganti, dei mutanti pazzi e dei favoleggiati
dragoni. Questa volta la spedizione era capeggiata da un monsignore con
minuscole corna e zanne appuntite, il quale annunciò che aveva avuto
l\textquotesingle incarico di opporsi alla canonizzazione del beato
Leibowitz, e che era venuto per indagare --- e anche per stabilire certe
responsabilità, fece capire --- a proposito di alcune dicerie
incredibili e isteriche che erano filtrate dall\textquotesingle abbazia
e che avevano purtroppo raggiunto perfino le porte di Nuova Roma. Fece
capire che non avrebbe tollerato alcuna romantica sciocchezza, come
forse aveva fatto un certo visitatore che l\textquotesingle aveva
preceduto.

L\textquotesingle abate l\textquotesingle accolse gentilmente e gli
offrì una branda di ferro in una cella esposta a sud, dopo essersi
scusato perché l\textquotesingle appartamento degli ospiti era stato
recentemente teatro di qualche caso di vaiolo. Il monsignore era servito
dalla sua scorta, e mangiava muschio ed erbe con i monaci nel
refettorio; poiché le quaglie e i galli selvatici erano incredibilmente
scarsi in quella stagione.

Questa volta, l\textquotesingle abate non ritenne necessario mettere in
guardia Francis contro un liberale esercizio della sua immaginazione.
Provasse a esercitarla, se ne aveva il coraggio. C\textquotesingle era
ben poco rischio che l\textquotesingle{}\emph{advocatus diaboli} desse
immediatamente credito alla verità, senza prima averla fatta a pezzi e
senza aver cacciato le dita nelle piaghe.

--- So che sei incline agli svenimenti --- disse monsignor Flaught
quando fu solo davanti a frate Francis e lo ebbe fissato con uno sguardo
che al monaco sembrò maligno. --- Dimmi, c\textquotesingle è stato
qualche caso di epilessia nella tua famiglia? Qualche caso di pazzia?
Qualche caso di mutazioni neurali?

--- Nessuno, Eccellenza.

--- Io non sono ``Eccellenza'' --- insorse il prete. --- Ora, vediamo di
ottenere la \emph{verità} da te. ---\emph{Andrebbe bene un piccolo
	intervento diretto di chirurgia}, sembrava sottintendere, \emph{visto
	che è necessaria soltanto una amputazione trascurabile}. --- Sai che i
documenti possono essere invecchiati artificialmente? --- domandò.

Frate Francis non lo sapeva.

--- Ti rendi conto che il nome Emily non appare nei documenti che hai
trovato?

--- Oh, ma\ldots{} --- Francis si interruppe, improvvisamente incerto.

--- Il nome che vi figura è Em, non è vero? Che \emph{potrebbe essere}
un diminutivo per Emily?

--- Io\ldots{} io credo che sia così, monsignore.

--- Ma potrebbe essere anche un diminutivo di Emma non è vero? E nella
cassetta appariva il nome Emma!

Francis tacque.

--- Ebbene?

--- Qual era la domanda, monsignore?

--- Non ci pensare! Ho solo inteso dirti che l\textquotesingle evidenza
suggerisce che ``Em'' sta per Emma, ed Emma non è un diminutivo di
Emily. Cosa ne dici?

--- Non avevo un\textquotesingle opinione su questo argomento,
monsignore, ma\ldots{}

--- Ma cosa?

--- Forse che marito e moglie spesso non badano molto al nome con cui si
chiamano?

--- \emph{Stai cercando di fare dell\textquotesingle ironia con me?}

--- No, monsignore.

--- Allora, di\textquotesingle{} la verità! Come hai scoperto il
rifugio, e cos\textquotesingle è questa favola fantastica
sull\textquotesingle apparizione?

Frate Francis tentò di spiegare. L\textquotesingle{}\emph{advocatus
	diaboli} l\textquotesingle interruppe con sbuffi e domande sarcastiche,
e quando ebbe finito il suo racconto, l\textquotesingle avvocato passò
l\textquotesingle intera storia con un rastrello semantico, fino a che
lo stesso frate Francis cominciò a chiedersi se aveva veramente veduto
il vecchio oppure se aveva immaginato l\textquotesingle episodio.

La tecnica del controinterrogatorio era spietata, ma Francis giudicò
quell\textquotesingle esperienza meno spaventosa di un colloquio con
l\textquotesingle abate. L\textquotesingle avvocato del diavolo non
poteva fare di peggio che farlo a pezzi un\textquotesingle unica volta,
e la certezza che l\textquotesingle operazione sarebbe finita presto
aiutò il paziente a sopportare il dolore. Quando era di fronte
all\textquotesingle abate, invece, Francis si rendeva sempre conto che
uno sbaglio poteva essere punito molte volte, poiché Arkos era il suo
superiore a vita e l\textquotesingle inquisitore perpetuo della sua
anima.

E monsignor Flaught sembrò giudicare la versione del monaco troppo
ingenua per meritare un attacco in piena regola, dopo aver osservato la
reazione di frate Francis all\textquotesingle aggressione iniziale.

--- Bene, fratello, se questa è la tua versione e tu la sostieni, non
credo che ti disturberemo. Anche se è vera\ldots{} il che non lo
ammetto\ldots{} è così trascurabile da essere sciocca. Te ne rendi
conto?

--- È quello che ho sempre pensato, monsignore --- sospirò frate
Francis, che da molti anni, ormai, cercava di sminuire
l\textquotesingle importanza che gli altri avevano dato al pellegrino.

--- Ebbene, è ora che tu lo dica! --- scattò Flaught.

--- Ho sempre detto che mi pareva che fosse \emph{probabilmente}
soltanto un vecchio.

Monsignor Flaught si copri gli occhi con la mano e sospirò pesantemente.
La sua esperienza con i testimoni incerti lo indusse a non dire altro.

Prima di lasciare l\textquotesingle abbazia,
l\textquotesingle{}\emph{advocatus diaboli}, come aveva fatto prima di
lui l\textquotesingle avvocato dei Santi, si fermò nella copisteria e
chiese di vedere la copia alluminata della \emph{blueprint} di Leibowitz
(``quella spaventosa incomprensibilità'' come la definì Flaught). Questa
volta le mani del monaco tremavano non di impazienza ma di paura, perché
era possibile che fosse costretto ad abbandonare di nuovo il suo
progetto. Monsignor Flaught osservò in silenzio la cartapecora. Deglutì
tre volte. Alla fine si costrinse ad annuire.

--- La tua immaginazione è vivida --- ammise. --- Ma questo lo sapevamo
tutti, no? --- Fece una pausa. --- È da molto tempo che vi stai
lavorando?

--- Da sei anni, monsignore\ldots{} a intermittenza.

--- Sì, bene, sembra che occorrano ancora altrettanti anni per finirlo.

Le corna di monsignor Flaught si accorciarono immediatamente di un paio
di centimetri, e le sue zanne scomparvero completamente. La stessa sera
se ne partì dal convento per tornare a Nuova Roma.

Gli anni passarono tranquillamente, segnando di rughe i visi dei giovani
e aggiungendo capelli grigi alle loro tempie. Il lavoro perpetuo
dell\textquotesingle abbazia continuò, aggredendo quotidianamente il
Cielo con l\textquotesingle inno sempre ricorrente
dell\textquotesingle Ufficio Divino, rifornendo quotidianamente il mondo
di un lento rivolo di manoscritti copiati e ricopiati, prestando di
tanto in tanto chierici e scribi all\textquotesingle episcopato, ai
tribunali ecclesiastici, e alle poche potenze secolari che potevano
permetterselo. Frate Jeris manifestò l\textquotesingle ambizione di
costruire un torchio da stampa, ma Arkos respinse il progetto non appena
ne udì parlare. Non c\textquotesingle era né carta sufficiente né
inchiostro adatto, e non v\textquotesingle era richiesta di libri poco
costosi, in un mondo che si vantava del proprio analfabetismo. La
copisteria continuò ad andare avanti con barattoli e pennelli.

Perla Festa dei Cinque Santi Folli, arrivò un messaggero vaticano con
liete nuove per l\textquotesingle Ordine. Monsignor Flaught aveva
ritirato tutte le obiezioni e stava facendo penitenza davanti a
un\textquotesingle icona del beato Leibowitz. La causa di monsignor
Aguerra era vinta: il papa aveva dato istruzioni perché venisse emesso
un decreto che raccomandava la canonizzazione. La data per la
proclamazione ufficiale era fissata per il prossimo Anno Santo, e doveva
coincidere con la convocazione del Concilio Generale della Chiesa allo
scopo di fare una prudente riformulazione della dottrina a proposito
delle limitazioni del \emph{magisterium} a materie di fede e di morale;
era un problema che sembrava essere stato risolto molte volte, nel corso
della storia, ma pareva ripresentarsi sotto nuova forma durante ogni
secolo, specie in quei periodi bui in cui la conoscenza umana del vento,
delle stelle e della pioggia era in realtà soltanto una semplice
credenza. Durante il tempo del concilio, il fondatore
dell\textquotesingle Ordine Albertiano sarebbe stato incluso nel
Calendario dei Santi.

L\textquotesingle annuncio fu seguito da un periodo di allegrezza,
nell\textquotesingle abbazia. Don Arkos, ormai raggrinzito
dall\textquotesingle età e prossimo al rimbambimento, chiamò alla sua
presenza frate Francis e gemette: --- Sua Santità ci invita a Nuova Roma
per la canonizzazione. Preparati a partire.

--- \emph{Io}, monsignore?

--- Tu solo. Il fratello farmacista mi proibisce di mettermi in viaggio,
e non sarebbe bene che il Padre Priore si allontanasse, mentre io sono
ammalato. E adesso non svenirmi di nuovo --- aggiunse don Arkos in tono
querulo. --- Probabilmente stai acquistando più credito di quanto
meriti, perché il tribunale ha accettato la data di morte di Emily
Leibowitz come provata in modo conclusivo. Ma Sua Santità il Papa ti
invita comunque. Il consiglio che ti posso dare è di ringraziare Dio e
di non pretendere merito.

Frate Francis vacillò. --- Sua Santità\ldots?

--- Sì. Ora, noi manderemo al Vaticano la \emph{blueprint} originale di
Leibowitz. Cosa ne dici di prendere con te la copia alluminata come dono
personale per il Santo Padre?

--- Uh --- fece Francis.
