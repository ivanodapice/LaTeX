	\chapter{\phantom{title}}

\lettrine{D}{opo} lo sfortunato incidente nel sotterraneo, l\textquotesingle abate
studiò tutti i modi immaginabili di fare ammenda per
quell\textquotesingle infelice momento. Il Thon Taddeo non mostrò segni
di rancore, e si scusò persino con gli ospiti per il suo giudizio
dell\textquotesingle incidente, dopo che l\textquotesingle inventore
dell\textquotesingle ordigno gli ebbe dato una spiegazione
particolareggiata della sua recente progettazione e costruzione. Ma
quelle scuse riuscirono soltanto a convincere ulteriormente
l\textquotesingle abate della gravità dell\textquotesingle errore
commesso. Infatti, metteva il Thon nella posizione d\textquotesingle un
alpinista che ha scalato una cima ``inviolata'' soltanto per trovare le
iniziali del rivale scolpite sulla sommità\ldots{} senza che il rivale
l\textquotesingle avesse avvertito in precedenza. Doveva essere stato
terribile, per lui, pensò don Paulo.

Se il Thon non avesse insistito (con una fermezza nata forse
dall\textquotesingle imbarazzo) che quella luce era di qualità
superiore, e sufficientemente brillante per consentire persino un
attento esame di documenti fragili, consunti dall\textquotesingle età,
pressoché indecifrabili alla luce delle candele, don Paulo avrebbe fatto
togliere immediatamente la lampada dalla cantina. Ma il Thon Taddeo
aveva insistito che quella lampada gli piaceva\ldots{} solo per
scoprire, poi, che era necessario tenere almeno quattro novizi o
postulanti impiegati ininterrottamente per far funzionare la dinamo e
per regolare la lampada ad arco; per cui chiese che la lampada fosse
tolta\ldots{} ma allora fu la volta di Paulo a insistere perché
rimanesse al suo posto.

Così fu che lo studioso cominciò le sue ricerche
all\textquotesingle abbazia, continuamente conscio della presenza dei
tre novizi che manovravano la ruota della dinamo e del quarto che
sfidava la cecità in cima a una scala, per tenere accesa e regolata la
fiamma\ldots{} una situazione che indusse il Poeta a scrivere versi
spietati sul demonio Imbarazzo e sugli oltraggi che perpetrava in nome
della penitenza e del quieto vivere.

Per parecchi giorni il Thon e il suo assistente studiarono la
biblioteca, gli scaffali, i documenti del monastero a eccezione dei
Memorabilia\ldots{} come se, accertando la validità
dell\textquotesingle ostrica, potessero stabilire la possibilità di
trovarvi la perla. Frate Kornhoer trovò l\textquotesingle assistente del
Thon inginocchiato all\textquotesingle ingresso del refettorio, e per un
istante ebbe l\textquotesingle impressione che stesse recitando qualche
devozione davanti all\textquotesingle immagine di Maria posta sopra la
porta, ma un tintinnio di strumenti pose fine alle sue illusioni..
L\textquotesingle assistente collocò una livella da carpentiere
attraverso la soglia e misurò la depressione concava scavata nelle
pietre del pavimento da secoli di sandali monastici.

--- Stiamo cercando un modo di fissare le date --- disse a Kornhoer,
quando questi l\textquotesingle interrogò. --- Questo sembra un punto
adatto per stabilire una media dell\textquotesingle usura, poiché è
facile calcolare il passaggio. Tre pasti per ogni uomo, ogni giorno, fin
da quando furono messe qui queste pietre.

Kornhoer non poté fare a meno di sentirsi impressionato da quella
precisione: quei calcoli lo sbalordirono. --- I documenti architettonici
della abbazia sono completi --- disse. --- Possono dirvi esattamente
quando fu aggiunto ogni edificio e ogni ala. Perché non risparmiate il
vostro tempo?

L\textquotesingle uomo alzò lo sguardo con aria innocente. --- Il mio
padrone ha un detto: ``Nayol non parla, e perciò non mente mai''.

--- Nayol?

--- Uno degli dei della Natura del popolo del Fiume Rosso. Lui
l\textquotesingle intende in modo figurato, naturalmente.
L\textquotesingle evidenza obiettiva è l\textquotesingle autorità
suprema. I documenti possono mentire, ma la Natura non ne è capace. ---
Notò l\textquotesingle espressione del monaco e aggiunse, in fretta: ---
Non intendevo offendere. È soltanto una dottrina del Thon che tutto
debba essere ricontrollato obiettivamente.

--- Una concezione affascinante --- mormorò Kornhoer, e si chinò per
osservare il disegno, fatto dall\textquotesingle uomo, della sezione
della concavità del pavimento. --- Ehi, ha una forma simile a quella che
frate Majek chiama una curva di distribuzione normale. Che strano!

--- Non è strano. La probabilità che un passo abbia deviato dalla linea
centrale tende a seguire la normale funzione
dell\textquotesingle errore.

Kornhoer era affascinato. --- Chiamerò frate Majek --- disse.
L\textquotesingle interesse dell\textquotesingle abate per gli esami
preliminari dei suoi ospiti era meno esoterico.

--- Perché --- chiese a Gault --- stanno facendo disegni
particolareggiati delle nostre fortificazioni?

Il priore si mostrò sorpreso. --- Non l\textquotesingle avevo saputo.
Volete dire che il Thon Taddeo\ldots{}

--- No. Gli ufficiali che sono venuti con lui. Stanno facendo proprio
questo, molto sistematicamente.

--- Come l\textquotesingle avete scoperto?

--- Me l\textquotesingle ha detto il Poeta.

--- Il Poeta! Ah!

--- Sfortunatamente, questa volta ha detto la verità. Aveva rubato uno
dei loro disegni.

--- L\textquotesingle avete voi?

--- No. Gliel\textquotesingle ho fatto restituire. Ma questa faccenda
non mi piace. Sa di malaugurio.

--- Immagino che il Poeta abbia chiesto di essere pagato, per questa
informazione.

--- Cosa molto strana, non l\textquotesingle ha chiesto. Ha preso
immediatamente in antipatia il Thon. Da quando sono arrivati, non ha
fatto altro che andare in giro brontolando fra sé.

--- Il Poeta ha sempre brontolato.

--- Ma non seriamente.

--- Perché pensate che stiano facendo quei disegni?

Paulo fece una smorfia triste. --- A meno che non riusciamo a scoprire
altri motivi, dovremo pensare che il loro interesse sia recondito e
professionale. Come cittadella fortificata, la nostra abbazia è stata un
vero successo. Non è mai stata conquistata per assedio o per attacco, e
forse ha destato la loro ammirazione professionale.

Padre Gault guardò pensieroso il deserto, verso oriente. --- Ora che ci
penso, se un esercito intendesse colpire a occidente, attraverso le
Pianure, dovrebbe probabilmente stabilire una guarnigione in questa
regione, prima di marciare su Denver. Rifletté per qualche attimo e
cominciò ad allarmarsi. --- E qui c\textquotesingle è una fortezza bella
e pronta!

--- Temo che ci abbiano pensato anche loro.

--- Ritenete che siano stati mandati qui a spiare?

--- No, no! Dubito che lo stesso Hannegan abbia mai sentito parlare di
noi Ma quegli uomini sono qui, sono ufficiali, e hanno continuato a
guardarsi intorno e a farsi venire idee nuove. Ora sì, molto
probabilmente, Hannegan sentirà parlare di noi.

--- Cosa intendete fare?

--- Non lo so ancora.

--- Perché non ne parlate al Thon Taddeo?

--- Gli ufficiali non sono suoi servitori. Sono stati mandati soltanto
come scorta, per proteggerlo. Cosa potrebbe fare?

--- È parente di Hannegan, e ha molta influenza.

L\textquotesingle abate annuì. --- Cercherò il modo di abbordarlo su
questo argomento. Ma prima dovremo tenere d\textquotesingle occhio ciò
che succede, per un certo tempo.

Nei giorni che seguirono, il Thon Taddeo completò il suo studio
dell\textquotesingle ostrica e, evidentemente certo che non si trattasse
di una vongola camuffata, dedicò la sua attenzione alla perla. Non era
un compito semplice.

Furono esaminate notevoli quantità di copie facsimili. Le catenelle
tintinnavano mentre i libri più preziosi scendevano dagli scaffali.
Quando gli originali erano parzialmente danneggiati o deteriorati,
sembrava poco prudente fidarsi dell\textquotesingle interpretazione data
dall\textquotesingle autore del facsimile. E allora venivano tirati
fuori i manoscritti che risalivano veramente ai tempi di Leibowitz e che
erano custoditi in cofani a tenuta stagna chiusi in speciali
sotterranei, per garantirne la conservazione per un tempo indeterminato.

L\textquotesingle assistente del Thon raccolse parecchi volumi di
appunti. Dopo il quinto giorno, il passo del Thon Taddeo divenne più
affrettato, e il suo contegno cominciò a riflettere
l\textquotesingle impazienza d\textquotesingle un segugio affamato che
sente l\textquotesingle odore di selvaggina saporita.

--- Magnifico! --- Vacillava fra il giubilo e una divertita incredulità.
--- Frammenti degli scritti d\textquotesingle un fisico del Ventesimo
secolo! Le equazioni sono persino consistenti.

Kornhoer sbirciò al di sopra della sua spalla. --- L\textquotesingle ho
visto --- disse, ansimando. --- Ma non vi ho mai capito nulla. È un
argomento importante?

--- Non ne sono ancora certo. La matematica è bella, \emph{bella}!
Guardate qui\ldots{} questa espressione\ldots{} notate la forma,
estremamente contratta. Questo, sotto il segno radicale\ldots{} sembra
il prodotto di due derivate, ma in realtà rappresenta una intera serie
di derivate.

--- E come?

--- Gli indici lo permutano in una espressione espansa; altrimenti, non
potrebbe rappresentare una linea integrale, come afferma invece
l\textquotesingle autore. È splendido! E guardate qui\ldots{} questa
espressione che sembra tanto semplice. La sua semplicità è ingannevole.
Ovviamente rappresenta non una sola, ma un intero sistema di equazioni,
in una forma molto contratta. Mi è occorso un paio di giorni per capire
che l\textquotesingle autore stava pensando alle relazioni non solo tra
quantità e quantità, ma tra sistemi e sistemi. Non conosco ancora tutte
le quantità fisiche che entrano in gioco, ma la sofisticazione della
matematica è veramente\ldots{} veramente superba! Se è un falso, è un
falso ispirato! Se è autentico, può darsi che siamo incredibilmente
fortunati. In ogni caso, è magnifico. Devo assolutamente vederne la
copia più antica!

Il frate bibliotecario gemette quando un altro cofano di piombo fu tolto
dal sotterraneo per essere dissigillato. Armbruster non era
impressionato dal fatto che lo studioso secolare, in due giorni avesse
risolto una parte del rompicapo che era rimasto lì, come enigma
assoluto, per dodici secoli. Per il custode dei Memorabilia, ogni
rottura di sigilli rappresentava un\textquotesingle altra riduzione
della probabile durata del contenuto dei cofani, e non tentò neppure di
nascondere la sua disapprovazione. Per il frate bibliotecario, il cui
compito, nella vita, era la conservazione dei libri, la principale
ragione dell\textquotesingle esistenza dei libri era la loro
conservazione perpetua. Il loro uso era secondario, e doveva essere
evitato, se ne minacciava la longevità.

L\textquotesingle entusiasmo del Thon Taddeo per il proprio lavoro
divenne più forte via via che i giorni passavano, e
l\textquotesingle abate respirò più liberamente quando vide
l\textquotesingle iniziale scetticismo del Thon diminuire a ogni nuovo
esame di qualche frammentario testo scientifico
dell\textquotesingle epoca prediluviale. Lo studioso non aveva fatto
dichiarazioni specifiche circa lo scopo delle sue indagini. Forse,
all\textquotesingle inizio, il suo scopo era stato vago, ma adesso
lavorava con la secca precisione di un uomo che segue un piano preciso.
Intuendo che si avvicinava l\textquotesingle alba di qualcosa, don Paulo
decise di offrire un trespolo al gallo perché vi cantasse, nel caso che
il gallo provasse l\textquotesingle impulso di annunciare
l\textquotesingle imminente levar del sole.

--- Tutta la comunità è molto curiosa dei vostri lavori --- disse allo
studioso. --- Ci piacerebbe sentirne parlare, se a voi non dispiace
discuterne. Naturalmente, noi tutti abbiamo sentito parlare del vostro
lavoro teorico nel collegio, ma è troppo tecnico perché molti di noi
possano comprenderlo. Sarebbe possibile, per voi, dircene qualcosa
in\ldots{} oh, in termini generali, in modo che anche i non specialisti
possano capire? La comunità brontola con me perché non vi ho invitato a
tenere lezioni: ma io pensavo che prima avreste preferito
familiarizzarvi con il luogo. Naturalmente, se invece preferiste
non\ldots{}

Lo sguardo del Thon sembrava fissare dei calibri sul cranio
dell\textquotesingle abate e misurarlo nelle sue sei linee principali.
Poi lo studioso sorrise, dubbioso. --- Vorreste che io spiegassi il
nostro lavoro nel più semplice linguaggio possibile?

--- Qualcosa del genere, se è fattibile.

--- È proprio questo. --- E rise. --- L\textquotesingle uomo che non ha
una preparazione specifica legge una relazione sulla scienza naturale e
pensa: ``Perché l\textquotesingle autore non l\textquotesingle ha
spiegato in un linguaggio semplice?''. Non si rende conto che ciò che ha
tentato di leggere era il più semplice linguaggio possibile\ldots{}
almeno per quell\textquotesingle argomento. In realtà, gran parte della
filosofia naturale è semplicemente un processo di semplificazione
linguistica\ldots{} uno sforzo di inventare linguaggi in cui mezza
pagina di equazioni possano esprimere un\textquotesingle idea che non
potrebbe essere esposta in meno di mille pagine scritte nel
``cosiddetto'' linguaggio semplice. Mi sono spiegato?

--- Credo di sì. Poiché vi siete spiegato, forse potreste parlarci di
questo aspetto della situazione. A meno che il mio suggerimento non sia
prematuro\ldots{} per quanto riguarda il vostro lavoro sui Memorabilia.

--- Ecco, no. Ora abbiamo un\textquotesingle idea chiara di dove stiamo
andando e su che cosa dobbiamo lavorare, qui. Naturalmente, questo
richiederà un tempo considerevole. I vari pezzi devono essere messi
insieme, e non appartengono allo stesso rompicapo. Non possiamo ancora
prevedere che cosa possiamo ricavarne, ma siamo abbastanza sicuri di ciò
che non possiamo ricavarne. Sono felice di dire che la situazione si
presenta carica di buone speranze. Non ho alcuna obiezione a spiegare lo
scopo generale del nostro lavoro, ma\ldots{} --- E ripeté il gesto
dubbioso.

--- Che cosa vi turba?

Il Thon si mostrò moderatamente imbarazzato. --- Solo
un\textquotesingle incertezza sul mio pubblico. Non vorrei offendere le
convinzioni religiose di nessuno.

--- Ma come potreste? Non è una questione di filosofia naturale? Di
scienza fisica?

--- Naturalmente. Ma le idee di molta gente sulla realtà del mondo si
sono colorate di sfumature religiose\ldots{} ecco, ciò che intendo dire
è\ldots{}

--- Ma se l\textquotesingle argomento di cui vi occupate è il mondo
fisico, come potreste offendere qualcuno? Specialmente questa comunità.
Abbiamo atteso per molto tempo di vedere il mondo interessarsi di nuovo
a se stesso. A rischio di mostrarmi presuntuoso, potrei osservare che
qui nel monastero abbiamo alcuni abili dilettanti di scienza naturale.
C\textquotesingle è frate Majek, e c\textquotesingle è frate
Kornhoer\ldots{}

--- Kornhoer! --- Il Thon levò lo sguardo verso la lampada ad arco, poi
lo distolse, battendo le palpebre. --- Non riesco a capire!

--- La lampada? Ma voi, senza dubbio\ldots{}

--- No, non la lampada. La lampada è abbastanza semplice, una volta che
abbiate superato il trauma di vederla funzionare veramente. Deve
funzionare. Funzionerebbe sulla carta, assumendo vari dati
indeterminabili e indovinandone altri che non sono disponibili. Ma il
netto balzo dalla vaga ipotesi a un modello funzionante\ldots{} --- Il
Thon tossì, nervosamente. --- È Kornhoer che non capisco.
Quell\textquotesingle arnese\ldots{} --- e indicò la dinamo con un dito
---\ldots{} rappresenta un balzo attraverso vent\textquotesingle anni di
esperimenti preliminari, cominciando dalla comprensione dei principi.
Kornhoer ha semplicemente eliminato i preliminari. Voi credete negli
interventi miracolosi? Io no, ma qui ne avete un esempio
\emph{autentico}. Ruote da carro! --- E rise. --- Cosa potrebbe fare, se
avesse un\textquotesingle officina? Non capisco che cosa faccia un uomo
come lui sepolto in un monastero.

--- Forse frate Kornhoer potrebbe spiegarvelo --- disse don Paulo,
cercando di escludere dalla sua voce una sfumatura di stizza.

--- Sì, ecco\ldots{} --- I calibri visivi del Thon Taddeo ricominciarono
a misurare il vecchio prete. --- Se pensate davvero che nessuno si
offenderebbe ascoltando idee non tradizionali, sarei lieto di discutere
il nostro lavoro. Ma parte di esso potrebbe contrastare con radicati
pregiu\ldots{} ehm\ldots{} con radicate opinioni.

--- Bene! Dovrebbe essere affascinante.

Fu stabilito un giorno, e don Paulo ne provò sollievo.
L\textquotesingle abisso esoterico fra il monaco cristiano e
l\textquotesingle indagatore secolare della natura si sarebbe
indubbiamente ridotto, con un libero scambio di idee, ne era certo.
Kornhoer aveva già ridotto per conto suo quell\textquotesingle abisso,
no? Una maggiore comunicazione era probabilmente la terapia migliore per
allentare ogni tensione. E il nebuloso velo di dubbio e di esitazione
diffidente sarebbe stato squarciato, non appena il Thon avesse veduto
che i suoi ospiti non erano affatto gli irragionevoli reazionari
intellettuali che egli sembrava sospettare. Paulo provò un
po\textquotesingle{} di vergogna per i suoi dubbi iniziali. ``Abbi
pazienza, o Signore, verso uno sciocco animato da buone intenzioni, ti
prego.''

--- Ma non dimenticate gli ufficiali e i loro disegni --- gli ricordò
Gault.
