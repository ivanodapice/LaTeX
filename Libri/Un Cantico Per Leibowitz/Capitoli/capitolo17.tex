	\chapter{\phantom{title}}

\lettrine{P}{osso} dirvelo senza alcun dubbio: vi sarà guerra --- disse il
messaggero di Nuova Roma. --- Tutte le forze di Laredo sono impegnate
sulle Pianure. Orso Pazzo ha tolto il campo. E una battaglia continua di
cavalleria, secondo le usanze dei nomadi, per tutte le Pianure. Ma lo
Stato di Chihuahua minaccia Laredo da sud. Quindi Hannegan si prepara a
mandare forze texarkane al Rio Grande\ldots{} per contribuire alla
``difesa'' della frontiera. Naturalmente, con la piena approvazione dei
laredani.

--- Re Goraldi è un tremendo sciocco! --- disse don Paulo. --- Non era
stato messo in guardia contro il tradimento di Hannegan?

Il messaggero sorrise. --- Il servizio diplomatico vaticano rispetta
sempre i segreti di Stato, se per caso li scopre. Per non essere
accusati di spionaggio, noi siamo sempre molto prudenti per ciò
che\ldots{}

--- Era stato messo in guardia? --- domandò di nuovo
l\textquotesingle abate.

--- Naturalmente. Goraldi disse che il legato papale gli mentiva; accusò
la Chiesa di fomentare i dissensi fra gli alleati della Sacra Sferza,
nel tentativo di affermare il potere temporale del papa.
Quell\textquotesingle idiota riferì persino a Hannegan
l\textquotesingle avvertimento del legato.

Don Paulo rabbrividì e sussurrò. --- E Hannegan che ha fatto?

Il messaggero esitò. --- Credo di potervelo dire: monsignor Apollo è
stato arrestato. Hannegan ha ordinato di sequestrare tutti i suoi
documenti diplomatici. A Nuova Roma si parla della possibilità di
scagliare l\textquotesingle interdetto sull\textquotesingle intero reame
di Texarkana. Naturalmente, Hannegan è già incorso nella scomunica
\emph{ipso facto}, ma pare che questo non turbi molto i texarkani. Come
certamente saprete, la popolazione è dedita a culti bizzarri nella
misura dell\textquotesingle ottanta per cento, e il cattolicesimo delle
classi dominanti è sempre stato una vernice molto sottile.

Questo per quanto riguarda Marcus --- mormorò tristemente
l\textquotesingle abate. --- E che notizie vi sono del Thon Taddeo?

--- Non capisco come possa pensare di attraversare le Pianure, adesso,
senza buscarsi qualche palla di moschetto. Le ragioni per cui non voleva
affrontare il viaggio sembrano chiare. Ma non so nulla di questo
viaggio, Padre Abate.

Il cipiglio di don Paulo era doloroso. --- Se il nostro rifiuto di
mandare il materiale in nostro possesso alla sua università dovesse
causare indirettamente la sua morte\ldots{}

--- Non permettete alla vostra coscienza di turbarsi per questo, Padre
Abate. Hannegan sa badare ai suoi. Non so come, ma sono sicuro che il
Thon arriverà qui.

--- Il mondo non potrebbe permettersi di perderlo, a quanto ho sentito.
Bene\ldots{} ma ditemi, perché siete stato mandato a riferire a noi i
piani di Hannegan? Noi siamo nell\textquotesingle impero di Denver, e
non capisco in che modo questa regione sia coinvolta\ldots{}

--- Ah, ma io vi ho raccontato solamente l\textquotesingle inizio.
Hannegan spera di riuscire a unificare il continente, alla fine. Dopo
aver imbrigliato saldamente Laredo, avrà rotto
l\textquotesingle accerchiamento che lo teneva in iscacco. Poi, la sua
prossima mossa sarà contro Denver.

--- Ma questo non comporterebbe l\textquotesingle invio di carovane di
rifornimenti attraverso il territorio dei nomadi? Mi sembra impossibile.

--- È estremamente difficile, ed è questo che rende certa la prossima
mossa. Le Pianure formano una naturale barriera geografica. Se fossero
spopolate, Hannegan potrebbe considerare sicura la frontiera
occidentale. Ma i nomadi rendono necessario per tutti gli Stati
limitrofi alle Pianure di mantenere forti presidi militari attorno al
loro territorio, per operazioni di contenimento. L\textquotesingle unico
modo per sottomettere le Pianure è controllare entrambe le fasce
fertili, a est e a ovest.

--- Ma anche così --- rifletté l\textquotesingle abate --- i
nomadi\ldots{}

--- Il piano di Hannegan nei loro confronti è diabolico. I guerrieri di
Orso Pazzo possono tenere facilmente testa alla cavalleria di Laredo, ma
non possono tener testa a una moria di bestiame. Le tribù delle Pianure
non lo sanno ancora, ma quando Laredo decise di punire i nomadi per le
loro scorrerie ai confini, i laredani spinsero avanti parecchie
centinaia di capi di bestiame infetto per mescolarli alle mandrie dei
nomadi. Fu un\textquotesingle idea di Hannegan. Il risultato sarà la
carestia, e allora sarà facile spingere una tribù contro
l\textquotesingle altra. Naturalmente noi non conosciamo tutti i
particolari, ma lo scopo è quello di costituire una legione nomade,
sotto un comandante-fantoccio, armata da Texarkana, leale a Hannegan,
pronta a buttarsi a ovest, verso le montagne. Se riesce a passare,
questa regione subirà le prime infiltrazioni.

--- Ma perché? Senza alcun dubbio Hannegan non crederà che i nomadi
siano truppe su cui poter fare conto, oppure che siano in grado di
tenere un impero, dopo che avranno finito di mutilarlo!

--- No, monsignore. Ma se le tribù nomadi verranno disperse,
l\textquotesingle impero di Denver crollerà. Poi Hannegan raccoglierà i
frammenti.

--- Per farne che? Non potrebbe essere un impero molto ricco.

--- No, ma sarà sicuro da ogni lato. Allora potrebbe essere in una
posizione migliore per colpire a est o a nordest. Naturalmente, prima
che si arrivi a questo, i suoi piani potrebbero fallire. Ma, falliscano
o no, questa regione corre il pericolo di essere invasa in un futuro non
lontano. Bisognerebbe prendere misure per rendere sicura
l\textquotesingle abbazia, nei prossimi mesi. Ho ricevuto istruzioni di
discutere con voi il problema di mettere al sicuro i Memorabilia.

Don Paulo senti che l\textquotesingle oscurità cominciava a addensarsi.
Dopo dodici secoli, una piccola speranza si era accesa nel mondo\ldots{}
e poi veniva un Principe analfabeta che la calpestava con
un\textquotesingle orda di barbari e\ldots{}

Il suo pugno esplose sul piano della scrivania. --- Li abbiamo tenuti
fuori dalle nostre mura, per mille anni --- brontolò --- e possiamo
tenerveli per mille anni ancora. Questa abbazia è stata assediata tre
volte durante l\textquotesingle epoca Bayring, e
un\textquotesingle altra volta durante lo scisma di Vissarion. Salveremo
i libri. Li terremo al sicuro per molto, molto tempo.

--- Ma in questi tempi c\textquotesingle è un rischio assai più grave,
monsignore.

--- E quale sarebbe?

--- Un\textquotesingle abbondante scorta di polvere da sparo e di
proiettili.

La Festa dell\textquotesingle Assunzione era venuta ed era trascorsa, ma
ancora non si aveva notizia della carovana proveniente da Texarkana. I
preti dell\textquotesingle abbazia cominciavano a offrire messe votive
per i viaggiatori e i pellegrini. Don Paulo aveva rinunciato persino
alle sue leggere colazioni, e si sussurrava che facesse penitenza per
avere invitato lo studioso, in considerazione del pericolo che
minacciava le Pianure.

Sulle torri di guardia c\textquotesingle era sempre qualche monaco. Lo
stesso abate saliva frequentemente sulle mura, per guardare verso est.

Poco prima dei Vespri, la festa di san Bernardo, un novizio riferì di
aver visto in lontananza una lieve striscia di polvere, ma stava
scendendo l\textquotesingle oscurità, e nessun altro riuscì a scorgerla.
Poco dopo furono cantati la Compieta e il \emph{Salve Regina}, ma
nessuno si presentò alle porte.

--- Può darsi che fosse uno dei loro uomini mandato in avanscoperta ---
suggerì il priore Gault.

--- Può anche darsi che sia stata soltanto
l\textquotesingle immaginazione del frate che stava di guardia ---
ribatté don Paulo.

--- Ma se si sono accampati a una decina di miglia o giù di lì, lungo la
strada\ldots{}

--- Vedremmo brillare il loro fuoco dalla torre. È una notte chiara.

--- Eppure, Domne, dopo che sarà sorta la luna, potremmo mandare
qualcuno, a cavallo\ldots{}

--- Oh, no. È il modo migliore per farsi uccidere per sbaglio. Se sono
veramente loro, probabilmente avranno tenuto il dito sul grilletto
durante tutto il viaggio, specialmente la notte. Possiamo aspettare fino
all\textquotesingle alba.

Era già mattino avanzato, quando l\textquotesingle atteso gruppo di
cavalieri apparve da lontano, a est. Dall\textquotesingle alto delle
mura, don Paulo batté le palpebre e socchiuse gli occhi, guardando al di
là del terreno caldo e arido, cercando di mettere a fuoco i suoi occhi
miopi, in distanza. La polvere sollevata dagli zoccoli dei cavalli stava
disperdendosi, verso nord. Il gruppo di cavalieri si era fermato per
discutere.

--- Mi pare di vedere venti o trenta persone --- si lagnò
l\textquotesingle abate, soffregandosi gli occhi. --- Sono veramente
così tanti? --- Approssimativamente, sì --- disse Gault.

--- E come potremo provvedere a tutti?

--- Non credo che dovremo prenderci cura di quelli che indossano le
pelli di lupo, Monsignor Abate disse, un po\textquotesingle{} rigido, il
prelato più giovane.

--- \emph{Pelli di lupo?}

--- Nomadi, monsignore.

--- Mandate uomini alle mura! Chiudete le porte! Abbassate le
saracinesche! Lanciate\ldots{}

--- Aspettate, non sono tutti nomadi, don Paulo.

--- Oh? --- Don Paulo si voltò, per guardare ancora.

La discussione era terminata. Alcuni uomini agitavano le braccia in
segno di saluto; il gruppo si divise in due. La parte più cospicua
ritornò galoppando verso est. Gli altri cavalieri li seguirono per un
poco con lo sguardo, poi girarono i cavalli e avanzarono al trotto verso
l\textquotesingle abbazia.

--- Sono sei o sette\ldots{} qualcuno è in uniforme --- mormorò
l\textquotesingle abate mentre si avvicinavano.

--- Il Thon e la sua scorta, sicuramente.

--- Ma in compagnia di nomadi? È bene che non vi abbia permesso di
mandare laggiù un uomo a cavallo, questa notte: Cosa stavano facendo, in
compagnia dei nomadi?

--- A quanto pare, sono venuti come guide --- disse cupamente padre
Gault.

--- È molto gentile, da parte del leone, sdraiarsi accanto
all\textquotesingle agnello!

I cavalieri si accostarono alle porte. Don Paulo deglutì a vuoto. ---
Bene, faremmo meglio ad andare a porgere loro il benvenuto, padre ---
sospirò.

Prima che i due ecclesiastici fossero discesi dalle mura, i viaggiatori
avevano tirato le redini, davanti al cortile. Un cavaliere si staccò
dagli altri, avanzò al trotto, smontò, e presentò le sue credenziali.

--- Don Paulo del Pecos, Abbas?

L\textquotesingle abate si inchinò. --- \emph{Tibi adsum}. Benvenuto in
nome di san Leibowitz, Thon Taddeo. Benvenuto in nome della sua abbazia,
in nome delle quaranta generazioni che hanno aspettato la vostra venuta.
Siamo ai vostri ordini. --- Quelle parole gli erano dettate dal profondo
del cuore; le parole che erano state serbate per tanti anni, in attesa
di quel magico momento. Udendo un semplice monosillabo mormorato in
risposta, don Paulo alzò gli occhi.

Per un attimo, il suo sguardo rimase fisso in quello dello studioso. E
sentì la sensazione di calore svanire rapidamente. Quegli occhi
gelidi\ldots{} grigi, freddi, indagatori. Scettici, avidi e orgogliosi.
Lo studiavano come avrebbero potuto studiare un oggetto curioso, privo
di vita.

Paulo aveva pregato con fervore perché quel momento potesse essere un
ponte gettato su un abisso di dodici secoli\ldots{} aveva pregato che,
per suo mezzo, l\textquotesingle ultimo scienziato martirizzato di
quell\textquotesingle età antica potesse stringere la mano del domani.
C\textquotesingle era veramente un abisso: questo era evidente.
L\textquotesingle abate intuì, all\textquotesingle improvviso, che egli
non apparteneva a questa età, che era stato sospinto su una barena di
sabbia nel fiume del Tempo, e che in realtà non vi era alcun ponte.

--- Venite --- disse gentilmente. --- Frate Visclair si occuperà dei
vostri cavalli.

Quando ebbe veduto gli ospiti sistemati nei loro alloggi e si fu
ritirato nell\textquotesingle intimità del suo studio, il sorriso sul
volto del santo di legno gli ricordò inopinatamente il sorriso del
vecchio Benjamin Eleazar, mentre diceva: ``Anche i figli di questo mondo
sono consistenti''.
