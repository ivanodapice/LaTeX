	\chapter{\phantom{title}}

\lettrine{U}{n} po\textquotesingle{} sconvolto dalla commozione che si era sparsa
nell\textquotesingle abbazia, frate Francis ritornò quello stesso giorno
nel deserto, per completare la sua vigilia quaresimale in una solitudine
piuttosto desolata. Aveva previsto che le reliquie avrebbero destato un
po\textquotesingle{} di eccitazione, ma l\textquotesingle eccessivo
interesse che tutti dimostravano per il vecchio pellegrino lo
sorprendeva. Francis aveva parlato del vecchio soltanto per la parte che
quello aveva avuto, per caso o per disegno della Provvidenza, nel
ritrovamento della cripta e delle reliquie. Il pellegrino era soltanto
un ingrediente minore, per quanto riguardava Francis, in un disegno
superiore al cui centro stavano le reliquie di un santo. Ma i suoi
confratelli novizi avevano dimostrato un interesse maggiore per il
pellegrino che per le reliquie, e persino l\textquotesingle abate lo
aveva convocato non per interrogarlo sulla cassetta, ma per chiedere
particolari sul conto del vecchio. Gli avevano rivolto centinaia di
domande sul pellegrino, domande cui sapeva rispondere soltanto: ``Non
l\textquotesingle ho notato'', oppure ``Non stavo guardando, in quel
momento'', oppure ``Non ricordo se lo ha detto'': e alcune delle domande
erano piuttosto bizzarre. Quindi interrogò se stesso: ``Avrei dovuto
notarlo? Sono stato sciocco a non osservare ciò che faceva? Non prestavo
abbastanza attenzione a ciò che diceva? Mi è sfuggito qualcosa di
importante perché ero stordito?''

Rimuginò nell\textquotesingle oscurità mentre i lupi si aggiravano
attorno al suo nuovo accampamento e riempivano le notti dei loro
ululati. Si accorse di meditare durante certi momenti del giorno che
dovevano essere dedicati alle preghiere e agli esercizi spirituali della
vigilia di vocazione; e lo confessò al priore Cheroki la prima volta che
il prete si presentò, durante il suo giro di visite domenicali.

--- Non dovresti permettere che le romantiche immaginazioni degli altri
ti turbino; hai già abbastanza guai con la tua immaginazione --- gli
disse il prete, dopo averlo rimproverato per aver trascurato gli
esercizi e le preghiere. --- Quelli non escogitano domande del genere
sulla base di ciò che potrebbe essere vero; le elaborano sulla base di
ciò che potrebbe essere sensazionale, se per caso fosse vero. È
ridicolo! Posso dirti che il Reverendo Padre Abate ha ordinato a tutti i
novizi di lasciar cadere l\textquotesingle argomento. --- Dopo un attimo
aggiunse, sfortunatamente: --- Non c\textquotesingle era proprio nulla,
in quell\textquotesingle uomo, che potesse far pensare al
soprannaturale\ldots{} vero? --- con una sola, lievissima inflessione di
speranzosa interrogazione nella voce.

Anche frate Francis cominciò a chiederselo. Se c\textquotesingle era
stato qualcosa che poteva far pensare al soprannaturale, non
l\textquotesingle aveva notato. Ma, giudicando dal numero di domande cui
non sapeva rispondere, in verità non aveva notato molte cose. La
profusione delle domande gli aveva dato l\textquotesingle impressione
che la sua pochezza nell\textquotesingle osservare fosse stata, in un
certo senso, colpevole. Era grato al pellegrino, poiché grazie a lui
aveva scoperto il rifugio. Ma non aveva interpretato gli eventi
interamente in termini del proprio interesse, spinto dal desiderio di
trovare qualche prova che la sua vocazione per le fatiche del monastero
era nata non tanto dalla sua spontanea volontà quanto dalla grazia che
dava forza a tale volontà, senza tuttavia costringerla, dirigendola
verso la scelta. Forse gli eventi avevano un significato più vasto che
gli era sfuggito, poiché se ne era lasciato assorbire totalmente.

``Cosa ne pensi della tua esecrabile vanità?''

``La mia esecrabile vanità è del tutto simile a quella del gatto delle
favole, che studiava ornitologia, monsignore.''

Il suo desiderio di professare i voti definitivi e perpetui\ldots{} non
era forse simile al movente del gatto che era diventato
ornitologo?\ldots{} in modo di poter glorificare la propria ornitofagia,
mangiando esotericamente \emph{Penthestes atricapillus} senza mangiare
mai cinciallegre. Poiché, come il gatto era chiamato dalla Natura a
essere ornitofago, così Francis era chiamato dalla sua stessa natura a
divorare famelicamente la conoscenza che poteva essere insegnata in quei
tempi e, poiché non c\textquotesingle erano altre scuole se non le
scuole monastiche, aveva indossato dapprima l\textquotesingle abito di
postulante e poi quello di novizio. Ma sospettare che Dio, come la
Natura, lo avesse chiamato a diventare un monaco professo
dell\textquotesingle Ordine?

Che altro poteva fare? Non poteva ritornare alla sua terra natale, lo
Utah. Da bambino era stato venduto a uno sciamano, che
l\textquotesingle avrebbe istruito per farsene un servo e un accolito.
Poiché era fuggito, non poteva ritornare, se non per affrontare la
terribile ``giustizia'' tribale. Aveva rubato una proprietà dello
sciamano (la sua persona) e mentre il furto era una professione
onorevole nello Utah, farsi cogliere in fallo era un reato capitale
quando la vittima del furto era lo stregone della tribù. E non gli
sarebbe neppure piaciuto ritornare alla vita relativamente primitiva di
un popolo di pastori analfabeti, dopo i suoi studi
all\textquotesingle abbazia. Ma che altro? Il continente era scarsamente
popolato. Pensò alla mappa appesa a una parete della biblioteca
dell\textquotesingle abbazia, alla distribuzione sparsa delle aree
tratteggiate, che erano regioni, se non di civiltà, almeno di ordine
civile, dove regnava una specie di sovranità legittima che trascendeva
la concezione tribale. Il resto del continente era popolato scarsamente
dai popoli della foresta e della pianura, che in maggioranza non erano
selvaggi, ma gente liberamente organizzata in piccole comunità, qua e
là, che viveva di caccia, del raccolto dei prodotti spontanei della
terra e di una agricoltura primitiva: il loro,tasso di natalità era a
malapena sufficiente (se non si contavano le nascite di mostri e di
anormali) per mantenere costante il numero della popolazione. Le
principali industrie del continente, a eccezione di poche regioni
costiere, erano la caccia, l\textquotesingle agricoltura, il
combattimento e la stregoneria\ldots{} quest\textquotesingle ultima era
l\textquotesingle industria più promettente per un giovane che aveva la
possibilità di scegliersi una carriera e che aveva in mente, come scopi
primari, la massima ricchezza e il massimo prestigio.

L\textquotesingle istruzione che Francis aveva ricevuto
all\textquotesingle abbazia non aveva nessun valore pratico in un mondo
buio, ignorante, che viveva giorno per giorno, in cui la cultura non
esisteva e in cui un giovane letterato non era di alcun valore per una
comunità, a meno che non sapesse anche coltivare la terra, combattere,
cacciare o mostrare qualche speciale attitudine per il furto
intertribale e per la rabdomanzia dell\textquotesingle acqua o del
metallo lavorabile. Persino nei rari domini in cui esisteva una forma di
ordine civile, la cultura di Francis non gli sarebbe stata di molto
aiuto, se doveva vivere al di fuori della Chiesa. Era vero che qualche
barone di poco conto aveva a volte alle sue dipendenze uno scriba o due,
ma quei casi erano abbastanza rari da essere trascurabili; e quei posti
erano occupati altrettanto spesso da monaci quanto da laici istruiti in
un monastero.

L\textquotesingle unica richiesta di scribi e segretari era creata dalla
stessa Chiesa, la cui tenue rete gerarchica si stendeva su tutto il
continente, e qualche volta fino a lidi lontanissimi, sebbene i capi
delle diocesi lontane fossero virtualmente autorità autonome, soggetti
alla Santa Sede in teoria ma di rado in pratica, poiché erano separati
da Nuova Roma non tanto da scismi quanto da oceani attraversati molto di
rado. Questa organizzazione poteva essere tenuta insieme soltanto da una
rete di comunicazioni. La Chiesa era divenuta, per caso e senza nessuna
intenzione di diventarlo, l\textquotesingle unico mezzo per trasmettere
le notizie da un luogo all\textquotesingle altro, attraverso tutto il
continente. Se nel Nord-est scoppiava una pestilenza, il Sud-ovest ne
veniva presto informato, come effetto secondario delle storie dette e
ridette dai messaggeri della Chiesa, che andavano e venivano da Nuova
Roma.

Se l\textquotesingle infiltrazione dei nomadi nel lontano Nord-ovest
minacciava una diocesi cristiana, ben presto al Sud e
nell\textquotesingle Est veniva letta dai pulpiti una enciclica che
avvertiva della minaccia ed estendeva l\textquotesingle apostolica
benedizione agli ``uomini di ogni condizione sociale, che fossero abili
nell\textquotesingle uso delle armi, che avessero i mezzi di compiere il
viaggio, fossero piamente disposti a farlo, per giurare fedeltà al
Nostro diletto figlio, N., legittima autorità di quel luogo, per il
periodo di tempo che possa sembrare necessario per il mantenimento di
guarnigioni in difesa dei Cristiani contro la minacciosa orda di
infedeli, la cui spietata ferocia è nota a molti e che, con Nostro
profondissimo dolore, torturarono, assassinarono e divorarono quei
sacerdoti di Dio che Noi mandammo fra loro a portare. la Parola divina,
affinché entrassero come agnelli nel grembo
dell\textquotesingle Agnello, del cui gregge Noi siamo il Pastore sulla
Terra: perché, mentre Noi non abbiamo mai disperato né cessato di
pregare che quei nomadi figli delle tenebre possano essere condotti alla
Luce e introdotti in pace nel Nostro regno (perché non è da pensare che
gli stranieri pacifici debbano essere respinti da una terra così vasta e
deserta: no, essi sarebbero i benvenuti se venissero in pace, anche se
fossero estranei alla Chiesa Visibile e al suo Divino Fondatore, purché
obbedissero alla Legge Naturale che è scritta nel cuore di tutti gli
uomini, legandoli in ispirito a Cristo, anche se essi ignorano il Suo
Nome), è tuttavia opportuno e prudente che la Cristianità, pur pregando
per la pace e per la conversione degli infedeli, si accinga alla difesa
del Nord-ovest, dove le orde selvagge si raccolgono e gli incidenti
provocati dalla ferocia degli infedeli sono recentemente aumentati; e su
ognuno di voi, dilettissimi figli, che può portare le armi e che si
dirigerà a nord-ovest per arruolarsi tra coloro che si preparano
giustamente a difendere le loro terre, le loro case e le loro chiese,
Noi estendiamo e con la presente concediamo, come pegno del Nostro
particolare affetto, l\textquotesingle apostolica benedizione''.

Francis aveva pensato per un po\textquotesingle{} di andare a
nord-ovest, se non fosse riuscito a trovare una, vocazione per
l\textquotesingle Ordine. Ma, sebbene fosse forte e abbastanza abile nel
maneggiare il coltello e l\textquotesingle arco, era piuttosto basso e
non molto robusto, mentre --- secondo le voci --- gli infedeli erano
alti tre metri. Non poteva garantire che quelle voci fossero vere, ma
non vedeva alcuna ragione per considerarle false.

Oltre a morire in battaglia, c\textquotesingle erano ben poche cose che
poteva pensare di fare della sua vita --- ben poche cose che sembrassero
degne di essere fatte --- se non poteva dedicarla
all\textquotesingle Ordine.

La certezza nella sua vocazione non era stata spezzata, ma solo
lievemente piegata, dalla bruciante lezione impartitagli
dall\textquotesingle abate, e dal pensiero del gatto diventato
ornitologo quando la Natura l\textquotesingle aveva chiamato soltanto a
essere un ornitofago. Quel pensiero lo rese abbastanza infelice da
permettergli di lasciarsi sopraffare dalla tentazione, così che, la
Domenica delle Palme, il priore Cheroki udì da Francis (o dal suo
residuo disseccato e bruciato dal sole, in cui l\textquotesingle anima
di Francis era rimasta in qualche modo incapsulata) pochi brevi gracidii
che costituivano ciò che era probabilmente la confessione più succinta
che Francis avesse mai fatto o che Cheroki avesse mai udito:

--- Beneditemi, padre, ho mangiato una lucertola.

Il priore Cheroki, che era stato per molti anni il confessore di
penitenti che praticavano il digiuno, scoprì che
l\textquotesingle abitudine gli aveva dato, come al becchino della
favola, una particolare e tranquilla disinvoltura, così che rispose con
perfetta equanimità, senza batter ciglio: --- Era giorno di astinenza,
ed è stata artificialmente preparata?

La Settimana Santa sarebbe stata meno solitaria delle precedenti
settimane di Quaresima, se gli eremiti non fossero stati ormai ridotti
in condizioni tali da non provare più alcun interesse; perché in parte
la liturgia della Passione veniva portata fuori dalle mura
dell\textquotesingle abbazia per toccare i penitenti nei loro
eremitaggi: due volte fu portata l\textquotesingle Eucarestia, e il
Giovedì Santo fu l\textquotesingle abate a fare personalmente il giro,
accompagnato da Cheroki e da tredici monaci, per compiere il Mandato a
ogni eremitaggio. Le vesti dell\textquotesingle abate Arkos erano
nascoste sotto una tonaca da frate, e il leone riuscì quasi a sembrare
un umile gattino mentre si inginocchiava, lavava e baciava i piedi dei
suoi sudditi digiunanti con la massima economia di movimenti e con la
minima retorica, mentre gli altri cantavano, le antifone.

\emph{``Mandatum novum do vobis: ut diligatis invicem\ldots''}

Il Venerdì Santo la Processione della Croce portò un crocifisso velato,
fermandosi a ogni eremitaggio per svelarlo gradualmente davanti al
penitente, sollevando il drappo un centimetro dopo
l\textquotesingle altro per l\textquotesingle Adorazione, mentre i
monaci cantavano i Rimproveri:

\emph{``O mio popolo, che ti ho fatto? O in quale tempo ti ho afflitto?
	Rispondi\ldots{} Io ti ho esaltato con il potere della virtù; e tu mi
	hai appeso al patibolo della croce\ldots''}

E poi, il Sabato Santo.

I monaci riportarono i novizi all\textquotesingle abbazia uno alla
volta\ldots{} affamati e deliranti. Francis era di quindici chili più
leggero e immensamente più debole di quanto lo fosse stato il Mercoledì
delle Ceneri. Quando lo misero in piedi nella sua cella, barcollò, e
prima di raggiungere la branda, cadde. I fratelli ve lo deposero, lo
lavarono, Io rasero, e unsero la sua pelle screpolata, mentre Francis
balbettava in delirio, parlando di qualcosa avvolta in un telo di sacco
e alla quale si indirizzava talvolta come a un angelo e talvolta come a
un santo, invocando spesso il nome di Leibowitz e cercando di scusarsi.

I suoi confratelli, cui l\textquotesingle abate aveva proibito di
parlare di quell\textquotesingle argomento, si limitarono a scambiarsi
occhiate significative e cenni misteriosi.

Qualche rapporto filtrò fino all\textquotesingle abate.

Conducetelo qui --- brontolò a un archivista, non appena seppe che
Francis era in grado di camminare. Il suo tono mise le ali ai piedi
all\textquotesingle archivista.

--- Neghi di aver detto queste cose? --- grugnì Arkos.

--- Non ricordo di averle dette, Monsignor Abate --- rispose il novizio,
sogguardando il righello dell\textquotesingle abate. --- Può darsi che
delirassi\ldots{}

--- Assumendo che tu delirassi\ldots{} le ripeteresti, adesso?

--- Dovrei dire che il pellegrino era il Beato? Oh, no, Magister meus!

--- E allora afferma il contrario.

--- Non \emph{penso} che il pellegrino fosse il Beato.

--- Perché non dici chiaro: \emph{non lo era}?

--- Ecco, non avendo mai visto personalmente il beato Leibowitz, io non
vorrei\ldots{}

--- Basta! --- ordinò l\textquotesingle abate. --- È troppo! Non voglio
più vederti o sentirti per molto, molto tempo! Fuori!
Un\textquotesingle altra cosa\ldots{} NON aspettarti di professare i
voti con gli altri, quest\textquotesingle anno. Non ne avrai il
permesso.

Per Francis fu come se un tronco l\textquotesingle avesse colpito allo
stomaco.
