	\chapter{\phantom{title}}

\lettrine{I}{l} tempo scorre lentamente nel deserto, e vi sono ben pochi cambiamenti
che segnino il suo passaggio.

Erano trascorse due stagioni da quando don Paulo aveva respinto la
richiesta venutagli da oltre le Pianure, ma la questione era stata
regolata soltanto poche settimane prima. Ma era stata veramente
regolata? Texarkana era evidentemente scontenta del risultato.

L\textquotesingle abate camminava lungo le mura
dell\textquotesingle abbazia, al tramonto, con la mascella spinta in
avanti, come un vecchio scoglio baffuto contro i possibili frangenti
usciti dal mare degli eventi. I capelli ormai radi svolazzavano come
gagliardetti bianchi nel vento del deserto, che avvolgeva strettamente
l\textquotesingle abito attorno al suo corpo piegato e lo rendeva simile
a un Ezechiele emaciato con una pancia stranamente arrotondata.

Infilò le mani nodose nelle maniche e guardò, accigliato, oltre il
deserto, verso il villaggio di Sanly Bowitts che appariva in lontananza.
La rossa luce del sole lanciava la sua ombra in movimento attraverso il
cortile, e i monaci che la incontravano, attraversando quello spiazzo,
alzavano lo sguardo, stupiti, verso il vecchio. In quegli ultimi tempi
il loro superiore era sembrato di cattivo umore, dedito a bizzarri
presentimenti. Si sussurrava che presto sarebbe venuto il momento in cui
un altro abate sarebbe stato nominato superiore dei fratelli di san
Leibowitz. Si sussurrava che il vecchio non stesse tanto bene, anzi che
non stesse bene affatto. Si sussurrava che, se l\textquotesingle abate
avesse udito tutti quei mormorii, i mormoratori avrebbero dovuto
scavalcare il muro in gran fretta.

L\textquotesingle abate aveva udito quei mormorii, in realtà, ma una
volta tanto preferiva non badarvi affatto. Sapeva bene che quelle voci
erano vere.

--- Rileggetemelo ancora --- disse bruscamente al monaco che gli stava
immobile al fianco.

Il cappuccio del monaco sussultò lievemente in direzione
dell\textquotesingle abate. --- Quale, Domne? --- chiese.

--- Sapete benissimo quale.

--- Si, monsignore. --- Il monaco si frugò in una manica, appesantita da
mezzo staio di documenti e di corrispondenza: ma dopo un attimo trovò il
documento che cercava. Affissa al rotolo c\textquotesingle era
l\textquotesingle etichetta:

\begin{center}
	{\large{SUB IMMUNITATE APOSTOLICA HOC SUPPOSITUM EST.}}
\end{center}

\begin{center}
	{\large{QUISQUIS NUNTIUM MOLESTARE AUDEAT,}}
\end{center}

\begin{center}
	{\large{IPSO FACTO EXCOMMUNICETUR.}}
\end{center}

{\begin{flushleft}
		\emph{Rev.dissimoo Domno Paulo de Pecos, 43, Abbati}
\end{flushleft}}

{\begin{flushright}
		(Monastero dei Frati Leibowitziani, nei dintorni del villaggio di Sanly
		Bowitts Deserto di Sudovest, Impero di Denver)
\end{flushright}}

{\begin{flushleft}
		CUI SALUTEM DICIT:
\end{flushleft}}

{\begin{flushright}
		Marcus Apollo
		
		Papatiae Apocrisarius Texarkanae
\end{flushright}}

~

--- Benissimo, è quello. Leggetelo, dunque --- disse impaziente
l\textquotesingle abate.

--- \emph{Accedite ad eum\ldots{}} --- Il monaco si fece il segno della
croce e mormorò la tradizionale Benedizione dei Testi, che veniva
recitata prima di mettersi a leggere e a scrivere con la stessa
puntigliosità con cui veniva recitata la benedizione prima dei pasti.
Perché la conservazione del sapere durante il millennio
d\textquotesingle oscurantismo era stata la missione dei frati di
Leibowitz, e tutti quei piccoli riti servivano a tenere vivo lo spirito
di quella missione.

Finita la benedizione, il monaco levò il rotolo contro il tramonto, fino
a che divenne trasparente. --- \emph{Iterum oportet apponere tibi crucem
	ferendam, amice\ldots{}}

La sua voce era una debole cantilena, mentre il suo sguardo sceglieva le
parole in mezzo a una foresta di svolazzi superflui.
L\textquotesingle abate si appoggiò al parapetto, in ascolto, mentre
osservava le poiane che volavano in cerchio sulla mesa
dell\textquotesingle Ultima Speranza.

--- Di nuovo è necessario importi una croce da portare, mio vecchio
amico e pastore di miopi topi di biblioteca --- cantilenò la voce del
lettore --- ma forse portare questa croce sarà un trionfo. Sembra che la
regina di Saba venga a Salomone, dopotutto, anche se probabilmente viene
per denunciarlo come ciarlatano.

``La presente è per notificarti che il Thon Taddeo Pfardentrott, Saggio
dei Saggi, Studioso degli Studiosi, Biondo Figlio Illegittimo di un
certo Principe, e Dono di Dio a una \textquotesingle Generazione che si
ridesta\textquotesingle{} si è finalmente deciso a farti visita, dopo
aver rinunciato alla speranza di trasportare i vostri Memorabilia in
questo felice reame. Arriverà verso la Festa
dell\textquotesingle Assunzione, se riuscirà a sfuggire ai banditi lungo
la strada. Porterà la sua sfiducia e una piccola scorta di cavalieri
armati, cortesia personale di Hannegan II, la cui corpulenta persona
incombe su di me mentre scrivo, e grugnisce e fa smorfie davanti a
queste righe, che Sua Supremazia mi ha ordinato di scrivere, e nelle
quali Sua Supremazia si aspetta che io acclami suo cugino il Thon, nella
speranza che tu lo onorerai convenientemente. Ma poiché il segretario di
Sua Supremazia è a letto con la gotta, io sarò assolutamente franco:
''Quindi, in primo luogo, permetti che ti metta in guardia contro questa
persona, il Thon Taddeo. Trattalo con la tua abituale carità, ma non
fidarti di lui. È uno studioso geniale, ma uno studioso secolare e un
prigioniero politico dello Stato. Qui, lo Stato è Hannegan. Inoltre, il
Thon è piuttosto anticlericale, mi sembra\ldots{} o forse è soltanto
anti-monastico. Dopo la sua nascita imbarazzante, fu mandato in un
monastero benedettino e\ldots{} ma no, interroga il corriere, a questo
proposito\ldots"

Il monaco levò gli occhi dalla lettura. L\textquotesingle abate stava
ancora osservando le poiane sull\textquotesingle Ultima Speranza.

--- Avete sentito parlare della sua infanzia, fratello? --- chiese don
Paulo.

Il monaco annuì.

--- Continuate a leggere.

La lettura continuò, ma l\textquotesingle abate smise di ascoltare.
Conosceva quasi a memoria la lettera, ma aveva ancora
l\textquotesingle impressione che Marcus Apollo avesse tentato di dirgli
qualcosa fra le righe, qualcosa che lui, don Paulo, non era ancora
riuscito a comprendere. Marcus stava cercando di avvertirlo\ldots{} ma
di che? Il tono della lettera era blandamente ironico, ma sembrava
carico di incongruenze mal auguranti che potevano essere designate
soltanto a sommarsi ad alcune buie congruenze, se soltanto gli fosse
stato possibile sommarle. Che pericolo poteva esservi, nel permettere a
uno studioso secolare di studiare nell\textquotesingle abbazia?

Il Thon Taddeo, secondo il corriere che aveva portato la lettera, era
stato educato nel monastero benedettino, dove era stato portato da
bambino, per evitare imbarazzo alla moglie di suo padre. Il padre del
Thon era lo zio di Hannegan, ma sua madre era una fantesca. La duchessa,
moglie legittima del duca, non aveva mai protestato per i vagabondaggi
sentimentali del marito, fino a che quella fantesca gli aveva dato il
figlio maschio che aveva sempre desiderato; poi gridò
all\textquotesingle ingiustizia. Gli aveva generato soltanto figlie, e
l\textquotesingle essere stata superata da una fantesca scatenò la sua
ira. Fece allontanare il bambino, fece battere e scacciare la fantesca,
e rafforzò il suo dominio sul duca. Decise di dargli un figlio maschio
per riaffermare il proprio onore: ma gli diede altre tre figlie. Il duca
attese pazientemente per quindici anni: quando la duchessa morì
d\textquotesingle aborto (di un\textquotesingle altra figlia) andò
subito dai Benedettini per reclamare il ragazzo e per farne il suo
erede.

Ma il giovane Taddeo degli Hannegan-Pfardentrott era diventato un tipo
difficile. Era cresciuto, dall\textquotesingle infanzia
all\textquotesingle adolescenza, in vista della città e del palazzo dove
il suo primo cugino veniva preparato per salire al trono: se la sua
famiglia l\textquotesingle avesse ignorato, tuttavia, sarebbe maturato
senza risentirsi per la sua condizione sociale. Ma tanto suo padre
quanto la fantesca che l\textquotesingle aveva partorito venivano a
visitarlo con frequenza sufficiente a ricordargli che egli era nato di
carne umana e non di pietra, e per fargli vagamente comprendere che era
stato defraudato dell\textquotesingle amore cui aveva diritto.

Poi, il Principe Hannegan era venuto nello stesso monastero per un anno
di istruzione, aveva signoreggiato al di sopra del suo cugino bastardo,
e l\textquotesingle aveva superato in tutto, tranne che nella prontezza
di mente. Il giovane Taddeo aveva odiato il principe con calmo furore, e
si era accinto a distanziarlo il più possibile, almeno in dottrina.
Quella gara si era rivelata inutile, tuttavia: il principe aveva
lasciato la scuola monastica l\textquotesingle anno seguente,
illetterato come vi era giunto, e nessuno aveva più pensato a
perfezionare la sua istruzione. Nel frattempo, il suo cugino esiliato
continuava da solo la gara, conquistando alti onori: ma la sua vittoria
era inutile, perché Hannegan non se ne curava. Il Thon Taddeo era giunto
a disprezzare l\textquotesingle intera corte di Texarkana ma, con
giovanile incoerenza, vi ritornò volentieri per esservi finalmente
legittimato come figlio di suo padre, disposto a perdonare tutti tranne
la morta duchessa che l\textquotesingle aveva esiliato e i monaci che in
quell\textquotesingle esilio avevano avuto cura di lui.

Forse pensa al nostro chiostro come a un luogo abietto, si disse
l\textquotesingle abate. Forse per lui vi sarebbero amari ricordi, mezzi
ricordi e forse qualche ricordo immaginario.

---\ldots{} semi di controversia nel letto della Nuova Sapienza ---
continuò il lettore. --- Perciò stai in guardia, e bada ai sintomi.

"Ma, d\textquotesingle altra parte, non soltanto Sua Supremazia ma anche
le leggi della carità e della giustizia esigono che io te lo raccomandi
come un uomo bene intenzionato, o almeno come un bambino privo di
malizia, come quasi tutti questi pagani istruiti e educati (e nonostante
tutto diventano pagani). Si comporterà bene se tu ti comporterai con
fermezza, ma sii prudente, amico mio. Ha una mente simile a un moschetto
carico, e può sparare in qualsiasi direzione. Io confido, tuttavia, che
andare d\textquotesingle accordo con lui per qualche tempo non sarà un
problema troppo grave per la tua ingegnosità e per la tua ospitalità.

\emph{``Quidarn mihi calix nuper expletur, Paule. Precamini ergo Deum
	lacere me fortiorem. Metuo ut hic peream. Spero te et fratres saepius
	oraturos esse pro tremescente Marco Apolline. Valete in Christo,
	amici''.}

\emph{``Texarkanae missum est Octava Ss Petri et Pauli, Anno Domini
	termillesimo\ldots''}

--- Vediamo ancora il sigillo --- disse l\textquotesingle abate.

Il monaco gli porse il rotolo. Don Paulo lo accostò al viso per guardare
la scritta confusa impressa in fondo alla pergamena, con un timbro di
legno male inchiostrato:

~
\begin{center}
	{\large{APPROVATO DA HANNEGAN II, PER GRAZIA DI DIO PODESTÀ}}
\end{center}

\begin{center}
	{\large{DOMINATORE DI TEXARKANA, DIFENSORE DELLA FEDE,}}
\end{center}

\begin{center}
	{\large{E VAQUERO SUPREMO DELLE PIANURE.}}
\end{center}

\begin{center}
	{\large{QUI SEGNO: X}}
\end{center}

~

--- Mi chiedo se Sua Supremazia si è fatto leggere la lettera da
qualcuno, più tardi --- si preoccupò l\textquotesingle abate.

--- E se così fosse, monsignore, credete che la lettera sarebbe stata
mandata?

--- Credo di no. Ma questa frivolezza sotto il naso di Hannegan solo per
beffare l\textquotesingle analfabetismo del podestà è insolita in Marcus
Apollo, a meno che non cercasse di dirmi qualcosa fra le righe\ldots{}
ma non riuscisse a trovare il modo sicuro di esprimersi.
L\textquotesingle ultima parte\ldots{} a proposito d\textquotesingle un
certo calice che teme non gli verrà allontanato\ldots{} E chiaro che è
preoccupato di qualcosa, ma che cosa? E insolito, in Marcus: è
completamente insolito.

Erano passate parecchie settimane dall\textquotesingle arrivo della
lettera: durante quelle settimane don Paulo aveva dormito male, aveva
sofferto per il riacutizzarsi del vecchio disturbo gastrico; aveva
riflettuto moltissimo sul passato, come per scongiurare il futuro. Quale
futuro? si chiedeva. Non pareva vi fosse alcuna ragione logica per
aspettarsi guai. La controversia tra i monaci e gli abitanti del
villaggio si era quietata. Nessun segno di turbolenza veniva dalle tribù
del Nord e dell\textquotesingle Est. L\textquotesingle imperiale Denver
non insisteva nei suoi tentativi di esigere tasse dalle congregazioni
monastiche. Non c\textquotesingle erano truppe nelle vicinanze.
L\textquotesingle oasi forniva ancora acqua. Non pareva esservi alcuna
minaccia di epidemia tra gli animali e gli uomini. Il grano cresceva
bene, quell\textquotesingle anno, nei campi irrigati. Il mondo mostrava
segni di progresso, e il villaggio di Sanly Bowitts aveva raggiunto una
percentuale fantastica di persone che sapevano leggere e scrivere: ben
l\textquotesingle otto per cento. E di questo gli abitanti del villaggio
avrebbero dovuto essere grati ai monaci dell\textquotesingle Ordine
Leibowitziano\ldots{} ma non lo erano.

Eppure, aveva qualche presentimento. Qualche minaccia innominata era in
agguato, all\textquotesingle angolo del mondo, non appena il sole fosse
sorto di nuovo. Quella sensazione lo rodeva, tormentosa come uno sciame
di insetti affamati che ronzassero attorno al viso d\textquotesingle un
pellegrino nel sole del deserto. C\textquotesingle era la sensazione di
qualcosa di imminente, di spietato, di irragionevole: si avvolgeva in
spire come un serpente a sonagli reso furioso dal calore e pronto a
colpire un ciuffo d\textquotesingle erba che rotolasse.

Era un demonio, quello che cercava di affrontare, decise
l\textquotesingle abate, ma era un demonio molto evasivo. Il diavolo
dell\textquotesingle abate era piuttosto piccolo: alto quanto il
ginocchio d\textquotesingle un uomo, ma pesava dieci tonnellate e aveva
la forza di cinquecento buoi. Non era spinto dalla malizia, come
l\textquotesingle immaginava don Paulo, quanto da una compulsione
frenetica, qualcosa che somigliava al furore d\textquotesingle un cane
idrofobo. Azzannava carne e ossa e unghie semplicemente perché si era
dannato, e la dannazione creava un appetito dannatamente insaziabile. Ed
era malvagio semplicemente perché aveva negato il Bene, e quella
negazione era diventata parte della sua essenza, o una falla in essa.

In qualche luogo, pensò don Paulo, stava guadando un mare di uomini,
lasciando dietro di sé una veglia funebre di uomini storpiati.

Che sciocchezza, vecchio!, si rimproverò. Quando si è stanchi di vivere,
ogni cambiamento sembra malvagio, non è così? Perché allora qualunque
cambiamento disturba la pace della noia di vivere, così simile alla
morte. Oh, c\textquotesingle è il diavolo, sì, ma non dobbiamo dargli
più credito del dovuto. Sei così stanco di vivere, vecchio fossile?

Ma i presentimenti continuarono.

--- Pensate che le poiane abbiano già divorato il vecchio Eleazar? ---
chiese una voce tranquilla, accanto a lui.

Don Paulo si girò con un sussulto, nella penombra. La voce era quella di
padre Gault, il suo priore e probabile successore. Se ne stava là,
toccando una rosa, e sembrava imbarazzato per aver disturbato la
solitudine del vecchio.

--- Eleazar? Volete dire Benjamin? Perché, avete avuto sue notizie, in
questi ultimi tempi?

--- Ecco, no, Padre Abate. --- Gault rise, imbarazzato. --- Ma mi pareva
che voi guardaste verso la mesa, e ho creduto che steste pensando al
Vecchio Ebreo. --- Guardò verso la montagna a forma di incudine,
profilata contro una fascia grigia di cielo, a occidente. ---
C\textquotesingle è un filo di fumo, lassù, quindi credo che sia ancora
vivo.

--- Non dovremmo limitarci a crederlo --- disse bruscamente don Paulo.
--- Andrò lassù, a fargli visita.

--- Parlate come se steste per partire questa notte --- ridacchiò Gault.

--- Partirò fra un giorno o due.

--- Sarà meglio che siate prudente. Dicono che scagli pietre contro
coloro che si avvicinano.

--- Non lo vedo da cinque anni --- confessò l\textquotesingle abate. ---
E mi vergogno. È molto solo. Andrò da lui.

--- Se è tanto solo, perché si ostina a vivere come un eremita?

--- Per sfuggire alla solitudine\ldots{} in un mondo giovane. Il giovane
prete rise. --- Questo è forse logico secondo lui, Domne, ma io non
capisco.

--- Capirete, quando avrete la mia età\ldots{} o la sua.

--- Non penso di diventare tanto vecchio. Afferma di avere parecchie
migliaia di anni.

L\textquotesingle abate sorrise, ricordando. --- E, sapete, non posso
discuterne con lui. Lo conobbi quando ero soltanto un novizio, cinquanta
e più anni or sono, e giurerei che sembrava vecchio quanto ora. Deve
avere superato i cent\textquotesingle anni.

--- Tremiladuecentonove anni, così sostiene lui. Qualche volta dice di
essere ancora più vecchio. E credo che ne sia convinto, anche. Una
interessante follia.

--- Non sono tanto sicuro che sia pazzo, padre. Soltanto anormale, nella
sua lucidità. Perché volevate parlarmi?

--- Per tre piccole questioni. Prima, come Faremo ad allontanare il
Poeta dalle stanze degli ospiti reali\ldots{} prima che arrivi il Thon
Taddeo? Deve arrivare fra pochi giorni, e il Poeta ha messo radici.

--- Tratterò io con il Poeta. Che altro?

--- I Vespri. Verrete in chiesa?

--- Non fino a Compieta. Pensateci voi. Che altro?

--- C\textquotesingle è una discussione nel sotterraneo\ldots{} per
l\textquotesingle esperimento di frate Kornhoer.

--- Chi e come?

--- Ecco, sembra che il nocciolo della questione sia questo: frate
Armbruster ha l\textquotesingle atteggiamento di \emph{vespero mundi
	expectundo}, mentre frate Kornhoer sostiene che siamo al mattino
dell\textquotesingle età dell\textquotesingle oro. Kornhoer sposta
qualcosa per fare posto a un pezzo della sua attrezzatura. Armbruster
grida \emph{Perdizione!} Frate Kornhoer grida \emph{Progresso!} e si
accapigliano di nuovo. Poi vengono da me, furibondi, perché risolva la
discussione. Io li rimprovero perché hanno perduto la calma. Quelli
diventano umili e per dieci minuti si sopportano a vicenda. Sei ore
dopo, il pavimento trema per le urla di ``Perdizione!'' lanciate da
frate Armbruster nella biblioteca. Io riesco a dominare le esplosioni,
ma mi pare che sia una questione fondamentale.

--- E una fondamentale offesa alla giusta condotta, direi. Cosa volete
che faccia? Che li escluda dalla mensa comune?

--- Non ancora, ma potreste ammonirli.

--- Benissimo, ci penserò io. È tutto?

--- È tutto, Domne. --- Gault fece per allontanarsi, ma si fermò\ldots{}
Oh, fra l\textquotesingle altro\ldots. pensate che il meccanismo di
frate Kornhoer funzionerà?

--- Spero di no! --- sbuffò l\textquotesingle abate.

Padre Gault si mostrò sorpreso. --- Ma allora, perché
permettergli\ldots{}

--- Perché all\textquotesingle inizio ero incuriosito. Tuttavia quel
lavoro ha destato tanto scompiglio, ormai, che mi dispiace di avergli
permesso di cominciare.

--- E allora perché non lo fermate?

--- Perché spero che si arrenderà davanti all\textquotesingle assurdità
senza bisogno d\textquotesingle aiuto da parte mia. Se
l\textquotesingle esperimento fallisce, fallirà proprio in tempo per
l\textquotesingle arrivo del Thon Taddeo. Questa sarebbe la
mortificazione più adatta per frate Kornhoer\ldots{} gli ricorderebbe la
sua vocazione, prima che cominciasse a pensare di essere stato chiamato
alla religione al solo scopo di costruire un generatore di essenze
elettriche nel sotterraneo del monastero!

--- Ma, Padre Abate, dovrete ammettere che sarebbe una grande conquista,
se l\textquotesingle esperimento avesse successo.

--- Non è necessario che lo ammetta --- disse seccamente don Paulo.

Quando Gault si fu allontanato, l\textquotesingle abate, dopo una breve
discussione con se stesso, decise di risolvere il problema del Poeta
prima del problema perdizione-contro-progresso.

La soluzione più semplice del problema del Poeta sarebbe stato
allontanarlo dall\textquotesingle appartamento reale, e possibilmente
anche dall\textquotesingle abbazia, dai dintorni
dell\textquotesingle abbazia, fuori dalla portata di vista, di udito e
di. pensiero. Ma nessuno poteva sperare che sbarazzarsi del Poeta fosse
una ``soluzione semplice''.

L\textquotesingle abate lasciò le mura e attraversò il cortile,
dirigendosi verso la foresteria. Si muoveva a memoria, poiché gli
edifici erano monoliti d\textquotesingle ombra sotto le stelle, e
soltanto poche finestre splendevano della luce delle candele. Le
finestre dell\textquotesingle appartamento reale erano buie; ma il Poeta
seguiva orari strani, e poteva darsi che fosse nel suo alloggio.

Entrato nell\textquotesingle edificio, l\textquotesingle abate cercò a
tentoni la porta di destra, la trovò, e bussò. Non vi fu alcuna risposta
immediata, ma solo un debole belato che poteva e non poteva provenire
dall\textquotesingle interno dell\textquotesingle appartamento.

Bussò di nuovo, poi provò ad aprire la porta. Si aprì. Un fievole
chiarore rossastro da un bruciatore a carbone addolciva
l\textquotesingle oscurità: la stanza odorava di cibo rancido.

--- Poeta?

Di nuovo si udì quel debole belato, ma più vicino. Si avvicinò al
bruciatore, ne tolse con le molle un carbone incandescente, se ne servi
per accendere un ramoscello. Si guardò intorno e rabbrividì, vedendo il
disordine della stanza. Era vuota. L\textquotesingle abate accese una
lampada a olio e andò a esplorare il resto
dell\textquotesingle appartamento. Sarebbe stato necessario pulirla e
fumigarla con ogni cura\ldots{} forse addirittura esorcizzarla, prima
che il Thon Taddeo vi entrasse. Sperò di indurre il Poeta a fare le
pulizie, ma sapeva che era una possibilità molto remota.

Nella seconda stanza, don Paulo ebbe all\textquotesingle improvviso la
impressione di essere osservato. Si fermò e si guardò intorno,
lentamente.

Un occhio lo fissava da un vaso pieno d\textquotesingle acqua posato su
uno scaffale. L\textquotesingle abate gli rivolse un familiare cenno di
saluto e proseguì.

Nella terza stanza, incontrò la capra. Fu il loro primo incontro.

La capra era ritta su di un armadio, e masticava foglie di rapa.
Sembrava una minuscola varietà di capra di montagna, ma aveva la testa
calva che, nella luce della lampada, era d\textquotesingle un azzurro
brillante. Senza dubbio era un capriccio della natura.

--- Poeta? --- chiese l\textquotesingle abate, con voce sommessa,
guardando la capra e toccandosi la croce pettorale.

--- Sono qui --- disse una voce assonnata, dalla quarta stanza.

Don Paulo sospirò di sollievo. La capra continuò a masticare le foglie.
Era stato veramente un pensiero terribile.

Il Poeta giaceva disteso sul letto, con una bottiglia di vino a portata
di mano; batté irritato la palpebra del suo unico occhio buono, davanti
alla luce.

--- Dormivo --- si lagnò, aggiustando la fascia nera
sull\textquotesingle occhio cieco e allungando la mano verso la
bottiglia.

--- E allora svegliatevi. Ve ne andrete immediatamente di qui. Questa
notte stessa. Spostate tutte le vostre cose nel corridoio, per dare aria
all\textquotesingle appartamento. Dormirete nella cella del mozzo di
stalla, da basso, se proprio dovete. Poi tornate qui domattina e
ripulite l\textquotesingle appartamento.

Per un attimo, il Poeta assunse un\textquotesingle aria da cane
bastonato, poi fece l\textquotesingle atto di prendere qualcosa sotto le
coperte. Serrò un pugno e lo fissò.

--- Chi si è servito di questo appartamento, per ultimo? --- domandò.

--- Monsignor Longi. Perché?

--- Mi chiedevo chi avesse portato le cimici. --- Il Poeta aprì il
pugno, prese qualcosa dal palmo, lo schiacciò fra le unghie, poi lo
gettò via. --- Può godersele il Thon Taddeo. Io non le voglio. Mi hanno
mangiato vivo da quando sono entrato qui. Stavo pensando di andarmene,
ma adesso che voi mi avete offerto la mia vecchia cella, sarò felice
di\ldots{}

--- Non intendevo\ldots{}

---\ldots di accettare ancora per un certo tempo la vostra gentile
ospitalità. Soltanto fino a che sarà finito il mio libro, naturalmente.

--- Quale libro? Ma non importa. Togliete di qui tutta la vostra roba.

--- Adesso?

--- Adesso.

--- Bene. Non credo che riuscirei a sopportare queste cimici per
un\textquotesingle altra notte. --- Il Poeta rotolò giù dal letto, ma si
fermò per bere.

--- Datemi il vino --- ordinò l\textquotesingle abate.

--- Sicuro. Prendetene pure. È una buona annata.

--- Grazie, poiché l\textquotesingle avete rubato dalle nostre cantine.
Si dà il caso che sia vino per il Sacramento. Ci avevate pensato?

--- Non era stato consacrato.

--- Mi stupisce che abbiate pensato a questo. --- Don Paulo prese la
bottiglia.

--- A ogni modo non l\textquotesingle ho rubato. Io\ldots{}

--- Non pensate più al vino. Dove avete rubato la capra?

--- Non l\textquotesingle ho rubata --- si lagnò il Poeta.

--- Si è\ldots{} materializzata?

--- È stato un dono, Reverendissime.

--- Di chi?

--- Di un caro amico, Domnissime.

--- Un caro amico \emph{di chi}?

--- Mio, signore.

--- Questo è un paradosso. Allora, dove\ldots{}

--- Benjamin, signore.

Un lampo di stupore attraversò il viso di don Paulo.

--- L\textquotesingle avete rubata al vecchio Benjamin?

Il Poeta rabbrividì a quella parola. --- Vi prego, non
l\textquotesingle ho rubata.

--- E allora?

Benjamin ha insistito perché l\textquotesingle accettassi in dono, dopo
che io ho composto un sonetto in suo onore.

--- La verità!

Il Poeta deglutì, umilmente. --- Gliel\textquotesingle ho vinta a morra.

--- Capisco.

--- È vero! Quel vecchio rudere mi aveva quasi ripulito, e poi rifiutò
di farmi credito. Dovetti mettere in gioco il mio occhio di vetro contro
la capra. Ma rivinsi tutto.

--- Portate quella capra fuori dell\textquotesingle abbazia.

--- Ma è una capra meravigliosa. Il suo latte ha un profumo celestiale e
contiene spezie. In realtà, è merito suo la longevità del Vecchio Ebreo.

--- In che misura?

--- In tutti i suoi 5408 anni.

--- Credevo ne avesse soltanto 3200\ldots{} --- Don Paulo si interruppe,
sdegnoso. --- Cosa stavate facendo, voi, all\textquotesingle Ultima
Speranza?

--- Giocavo a morra con il vecchio Benjamin.

--- Voglio dire\ldots{} --- L\textquotesingle abate si fece forza. ---
Lasciate perdere. E andatevene di qui. E domani riportate la capra a
Benjamin..

--- Ma l\textquotesingle ho vinta onestamente.

--- Bene, non discutiamone. Portate la capra nella stalla, allora.
Gliela restituirò io stesso.

--- Perché?

--- Non ci serve una capra. E non serve neppure a voi.

--- Oh, oh --- fece il Poeta con aria astuta.

--- Cosa intendete dire, prego?

--- Sta per arrivare il Thon Taddeo. Ci sarà bisogno
d\textquotesingle una capra, prima che tutto sia finito. Potete esserne
sicuro. E ridacchiò soddisfatto fra sé.

L\textquotesingle abate si voltò, irritato. --- Andatevene --- aggiunse,
con rassegnazione, poi andò a occuparsi della disputa nel sotterraneo,
in cui ora riposavano i Memorabilia.
