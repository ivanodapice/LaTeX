{\pagestyle{empty} % Blocca le pagine dal venire considerate nella numerazione

{
	\centering
	
	~
	
	\vspace{24pt}
	{\scshape\Huge Un cantico\\per Leibowitz \par}
}

\cleardoublepage

\newlength\drop
\makeatletter
\newcommand*\titleM{\begingroup% Misericords, T&H p 153
	\setlength\drop{0.08\textheight}
	\centering
	\vspace*{\drop}
	{\Huge\bfseries Un cantico per Leibowitz}\\[\baselineskip]
	{\scshape Walter M. Miller}\\[\baselineskip]
	\vfill
	{\large\scshape \phantom{the author}}\par
	\vfill
	{\large\scshape DiracEdizioni \img{Immagini/DiracEdizioniLogo.png}}\par
	\endgroup}
\makeatother
	
\titleM

\space\pagebreak

\null\vfill % Portiamo il testo del copy in basso

\begin{flushleft}
	\begin{justify}
		{
		\footnotesize \textit{Un cantico per Leibowitz}
		
		\bigskip
		
		COPYRIGHT © 1959 Walter M. Miller
		
		\bigskip
		
		Tutti i diritti riservati.\\
		Nessuna parte di questa pubblicazione può essere riprodotta,\\
		memorizzata o trasmessa in qualsiasi forma o con qualsiasi mezzo,\\
		elettronico, meccanico, di fotocopiatura, registrazione, scansione o altro\\
		senza il permesso scritto dell'editore. È illegale copiare questo libro, pubblicarlo\\ 
		su un sito web o distribuirlo con qualsiasi altro mezzo senza autorizzazione.
		
		\bigskip
		
		Questo romanzo è interamente un'opera di fantasia.\\
		I nomi, i personaggi e gli episodi in esso rappresentati\\
	    sono frutto dell'immaginazione dell'autore.\\
	    Qualsiasi somiglianza con persone reali, vive o morte, eventi o località\\
	    è del tutto casuale.
		
		\bigskip
		
		\textonesuperior Edizione, 2024
		
		\bigskip
		
		\begin{tabular}{rl}
			ISBN--10:& 0-4908-2330-0\\ 
			ISBN--13:& 978-2-0827-3855-2\\ 
		\end{tabular}	
		
		\bigskip
		
		Pubblicato da DiracEdizioni \img{Immagini/DiracEdizioniLogo.png}
		}
	\end{justify}
\end{flushleft}

\let\cleardoublepage\clearpage

\chapter*{\phantom{DEDICA}}


	\begin{dedica}
		Una dedica è solo
		
		grattarsi dove prude\ldots{}
		
		Per Anne, poi,
		
		sui cui seni riposa Rachel
		
		come una musa,
		
		ispiratrice della mia goffa canzone
		
		mentre lei sorride tra le righe.
		
		Con benedizione, Lass W.
	\end{dedica}
}

\frontmatter

{\chapterstyle{dowding} \chapter*{INTRODUZIONE}}
{\chapterstyle{bringhurst} \chapter*{Scienziato Cristiano\\{\footnotesize Ken MacLeod.}}}
"I fisici hanno conosciuto il peccato", disse J. Robert Oppenheimer nel 1947, a proposito della loro responsabilità per la bomba atomica, ‘e questa è una conoscenza che non possono perdere’. Questo è un romanzo su quel peccato, e su come la sua conoscenza possa essere persa attraverso la ripetizione del peccato su una scala catastroficamente più grande - e su come coloro che lo hanno dimenticato possano poi continuare a ripeterlo, e ripeterlo ancora. La catastrofe ricorrente è resa possibile da un peccato molto specifico: quello degli scienziati che mettono la loro conoscenza al servizio di qualsiasi governante che finanzi il loro lavoro, indipendentemente dall'uso che ne viene fatto. Ci vuole molto di più dell'abilità letteraria, di cui il libro è ricco - è una delizia da leggere e rileggere: una storia forte e misurata, maestosa nella sua portata, ma dal ritmo incalzante, che offre lungo il percorso un tesoro di osservazioni ironiche, di umorismo nero, di arguzia asciutta e di sensibilità umana. Ciò che ha reso questo romanzo intensamente cattolico un classico della fantascienza è il valore eroico che attribuisce alla ragione, alla conoscenza, alla scienza e all'umanissimo rifiuto di adagiarsi e morire. Scritto dall'interno della fede colta e travagliata di un maturo convertito, il romanzo non contrappone la religione alla saggezza del mondo, o il cristianesimo all'umanesimo: rimane aperto a una lettura che prende le dottrine della Chiesa come simboli di una storia che ha senso in termini secolari. Lo schema: per secoli, dopo che la nostra civiltà si autodistrugge in una guerra nucleare, solo la Chiesa conserva frammenti di conoscenza scientifica. Poco dopo la guerra, l'ingegnere bellico Leibowitz, convertito e pentito, fonda un ordine religioso per salvare i libri dall'ira delle folle che incolpano la scienza e l'apprendimento dei libri per le loro atroci sofferenze (una guerra nucleare da sola non basterebbe a distruggere tutto il sapere scientifico - solo dei sopravvissuti impazziti potrebbero provocare un simile Anno Zero). Nei monasteri desertici di un'America del Nord martoriata e distrutta, i monaci dell'Ordine Albertiano di Leibowitz copiano e ricopiano ogni scarto che i loro “librai”, condannati a dieci martiri, hanno pericolosamente salvato. Secoli dopo, questi antichi testi - i Memorabilia - contribuiscono a innescare un nuovo Rinascimento, per poi far ripartire una civiltà industriale più avanzata della nostra. Questa civiltà, a sua volta... Ma tu sei più avanti di me. Anche la Chiesa, che questa volta porta i suoi resti - e i Memorabilia, ampiamente ampliati - tra le stelle. La storia millenaria è raccontata in primo piano. Seguiamo i monaci che copiano con riverenza i diagrammi dei circuiti che non capiscono e custodiscono documenti che sappiamo essere detriti. Ascoltiamo litanie in cui Fallout è il nome di un demone. Le ironie drammatiche si moltiplicano quando uno dei primi nuovi scienziati disdegna la pittoresca congettura di un chierico sull'evoluzione come spiegazione dell'origine delle specie. Piccoli capi barbari espandono i loro imperi e, nel sangue e nel fuoco, forgiano nuovamente la civiltà. Dopo ogni battaglia e martirio, gli avvoltoi vincono, fino all'ultimo. Quando anche gli inardi perdono, la partita è finita. Miller - un veterano dei bombardamenti alleati in Italia, con il senso di colpa per aver partecipato alla distruzione dell'antico monastero di Monte Cassino - combinava l'amore per la tecnologia di un ingegnere con la consapevolezza che il progresso tecnologico poteva coesistere con il regresso sociale e morale, e addirittura peggiorarlo. Il suo rigore con la dottrina e la liturgia è bilanciato da un giocoso riconoscimento dell'in-evitabilità delle aggiunte leggendarie e delle richieste contrastanti che queste pongono ai guardiani della fede: da un lato tollerare e benedire le pie credenze della gente semplice; dall'altro, raschiare senza sosta i cirripedi e le erbacce della superstizione dallo scafo liscio e dalla chiglia dritta dell'arca dell'ortodossia Una figura familiare della leggenda cristiana si aggira per la storia fin dall'inizio. Vediamo personaggi ed episodi di una sezione diventare leggende in quella successiva e, alla fine, assistiamo a quello che può essere un miracolo o una mutazione - a noi la scelta, come sempre. Sebbene la civiltà che le è succeduta abbia un'astronave a propulsione più lenta della luce e colonie interstellari in fase di avvio, la struttura della sua vita quotidiana è un'estensione riconoscibile della nostra, o meglio degli anni Sessanta: automobili, televisione, satelliti per le comunicazioni, guerra fredda e tensioni sul nucleare. Scritto alla fine degli anni Cinquanta, il racconto si legge come se fosse stato scritto un decennio dopo, come se la crisi dei missili di Cuba, il Concilio Vaticano II, la controversia sulla morte di Dio e lo sbarco sulla Luna fossero già avvenuti. Questo è quanto di più lontano si possa vedere, o si debba vedere. La società rappresentata è sempre più laica e il conflitto morale tra Chiesa e Stato che Miller sceglie di mettere in evidenza nelle ultime ore è di un'importanza letteralmente straziante: l'eutanasia per coloro che sono destinati a morire, molto presto, con grande dolore. Sceglie bene la battaglia: questa è una di quelle in cui nessuno, proprio nessuno, si aspetta che la forza vacilli o che la roccia si muova. Allo stesso modo, l'allineamento internazionale - che contrappone le alleanze occidentali e orientali - riecheggia il nostro recente passato. Sebbene non sia plausibile come ricorrenza post-atomica, non si tratta necessariamente di un fallimento dell'immaginazione. Miller, come dimostrano i suoi numerosi racconti, era in grado di immaginare mondi lontani. Forse è meglio vederlo come un espediente drammatico che mette in evidenza l'imminenza sempre potenziale del Diluvio di Fiamme, che potrebbe ancora essere nel nostro futuro prossimo, come lo sarà sempre finché non accadrà o non cambieremo strada. Per dosare il circuito e rafforzare l'argomentazione del libro, questa inquietante risonanza contemporanea crea il collegamento che John Clute ha identificato come una delle radici immaginative della SF: rovine e futuro. Qualche anno fa mi sono fermato dove un antico impero si è fermato, a Cramond sul Firth of Forth, e ho alzato lo sguardo dalle mura sepolte di una caserma romana per osservare un aereo di linea in avvicinamento Di punto in bianco l'accostamento ha ricordato i capitoli finali del Cantico, e ho rabbrividito. Possiamo anche sorridere dei monaci dell'Ordine di Leibowitz, che faticosamente inchiostrano un progetto invece di limitarsi a copiare le linee, ma l'ultima risata spetta a noi. 

{\chapterstyle{bringhurst} \chapter*{Un capolavoro ritrovato\\{\footnotesize Giuseppe Lippi.}}}
Un cantico per Leibowitz costituisce uno dei migliori esempi della fantascienza americana moderna, aggettivo che usiamo volentieri perché da allora (1960) non ha perduto un grammo della sua potenza e originalità. È anche uno dei pochi romanzi di sf che si farebbero leggere a chiunque, per il suo intrinseco piacere letterario: non un'opera di genere ma trans-genere, come tutti i capolavori cui calzi la definizione. Averlo ritrovato non è un merito, vista la sua fama da cinque decenni: il merito, semmai, consisterebbe nel conservarlo, dandogli di nuovo la dignità di un'edizione libraria come all'epoca della prima apparizione nello Science Fiction Book Club. Leggendolo si assiste alla nascita di una nuova forma d'espressione che fiorisce sì nelle riviste di settore (in questo caso, “Fantasy and Science Fiction” diretta da Anthony Boucher) ma va ben al di là dello scopo di queste ultime: è la speculative fiction americana che, a partire dagli anni Sessanta, comincia a produrre capolavori maturi come da sempre accadeva in quella inglese, dai tempi di H.G. Wells ad Aldous Huxley, da Olaf Stapledon a George Orwell. È allora, quando il genere cessa di essere semplicemente “un genere” e l'immaginazione si allea alla capacità di scrittura e all'originalità del pensiero, che il risultato può essere un quadro del mondo come quello contenuto nel Leibowitz, apocalittico ma non desolato, avveniristico ma mai scontato. E nel futuro di cui parla Miller si avvertono gli echi di un passato nient'affatto sepolto, un passato come quello custodito nell'abbazia di Montecassino che, bombardata dagli alleati alla fine della Seconda guerra mondiale, resiste persino alle esplosioni aeree, alla furia della guerra tecnologica, preservando il suo alone di simbolica sapienza. Walter Miller partecipò al bombardamento, vi assisté: e l'operazione destruens gliene ispirò una construens, Un cantico per Leibowitz, appunto. L'opera di una vita, cui stava per dare un seguito quando morì nel 1996. Poco dopo, un altro romanziere, Terry Bisson, avrebbe dato alle stampe una propria versione del romanzo che Miller non era riuscito a completare, ma che aveva già abbozzato e a cui mancava la parte finale: si tratta di Saint Leibowitz and the Wild Horse Woman, lunghissimo seguito del capolavoro originale. In Italia, vista la sua mole debordante e la natura sempre un po' spuria di certe operazioni editoriali, è parso impubblicabile; ma il lettore può consolarsi andando a leggere i racconti del nostro geniale autore, una selezione dei quali è apparsa nel n. 150 dei “Classici Urania” con il titolo Visioni dal futuro.

{\chapterstyle{bringhurst} \chapter*{Il cantore perduto\\{\footnotesize Gianni Montanari}}}
“Per buone e valide ragioni personali, Walter M. Miller, Jr. si è ritirato come scrittore.” Con queste parole, incluse nel breve cappello introduttivo a un racconto di Miller ristampato nell'antologia A Wilderness of Stars pubblicata nel 1971, si sanciva la scomparsa dal campo della fantascienza (e della letteratura) di uno dei suoi talenti più ricchi e singolari. A dire il vero, più che di una sanzione si trattava di una tardiva spiegazione, poiché la “scomparsa” era avvenuta qualcosa come undici anni prima, nel 1960, in coincidenza con la pubblicazione del capolavoro indiscusso di Miller, Un cantico per Leibowitz. Autore di quelle brevi righe era il curatore dell'antologia, William F. Nolan, che probabilmente sapeva in proposito più di quanto volesse scrivere, ma i lettori dovettero accontentarsi: Walter Miller aveva deciso di sparire dal mondo della fantascienza e nessuno poteva convincerlo a ripensarci.
Quali potevano essere i suoi motivi? Perfino David N. Samuelson, autore del più esauriente saggio critico su questo autore, The Lost Canticles of Walter M. Miller, Jr., apparso su “Science-Fiction Studies” n. 8 (marzo 1976), si limita ad accennare a motivi di ordine letterario: forse il suo romanzo lo ossessionava, prosciugandolo di ogni attività creativa; forse gli imponeva un termine di paragone il cui livello era troppo difficile mantenere; o ancora, forse il romanzo esprimeva così bene i temi cari a Miller che il suo completamento lo lasciava senza altro da dire. Ma si tratta sempre e soltanto di forse. L'unica cosa certa, ancora oggi, è che Un cantico per Leibowitz rimane un'opera difficilmente eguagliabile, e che nella produzione di Miller non costituisce un'eccezione fortunata ma il risultato finale di una continua ricerca durata quasi un decennio.
Nato il 23 gennaio 1923 in Florida,, da genitori cattolici, Walter Michael Miller. Jr interrompe agli inizi della Seconda guerra mondiale gli studi di ingegneria per arruolarsi in aviazione; partecipa così a più di cinquanta missioni di volo sui Balcani e sull'Italia, e assiste alla distruzione dell'abbazia di Montecassino. Finita la guerra, si laurea all'Università del Texas e inizia a scrivere durante un periodo di convalescenza provocato da un incidente automobilistico. Il suo esordio avviene con il racconto “Secret of the Death Dome” sulle pagine di “Amazing Stories” nel gennaio 1951, e nei sette anni seguenti la sua intera produzione viene ospitata da riviste come “Astounding”, “Fantastic Stories”, “Galaxy”. Sono anni in cui l'America, emersa poco prima vittoriosa dalla guerra, incomincia a perdere la sua sicurezza e il suo ottimismo euforico alle prese con la guerra di Corea e con il maccartismo, e sono gli anni in cui la fantascienza americana sembra finalmente voler abbandonare tanti stereotipi avventurosi per prestare un po' di attenzione anche allo sviluppo dei personaggi e al loro contesto ambientale. Le storie di Miller cominciano subito a lasciare il segno, con il loro piglio estremamente sicuro fin dall'inizio e la loro capacità di mettere in scena, oltre a personaggi dotati di un insolito spessore psicologico, temi che di lì a poco sarebbero diventati di bruciante attualità: il relativismo culturale di razze diverse, la solitudine urbana, il controllo delle nascite, l'alienazione tecnologica, per citarne alcuni.
In tutto, Miller pubblica quarantuno fra racconti e romanzi brevi, in un arco di tempo compreso tra il 1951 e il 1957. Uno di questi, The Darfsteller (“Il mattatore”), gli fa conquistare un premio Hugo nel 1956: è il magistrale ritratto di un attore del futuro che sabotando un “collega” elettronico riesce a tornare un'ultima volta sulle scene. Ma ci sono altri tre romanzi brevi, apparsi fra il 1955 e il 1957 su “The Magazine of Fantasy \& SF” (A Canticle for Leibowitz, And the Light is Risen e The Last Canticle), che sembrano assorbire Miller in un infaticabile lavoro di revisione e di ampliamento. Sono le tre storie che nel 1960 appaiono finalmente in volume come Un cantico per Leibowitz, meritando a Miller un premio Hugo per il miglior romanzo dell'anno. In quest'opera, concedendo finalmente spazio a un interesse in precedenza solo sfiorato in alcuni racconti, Miller ha modo di affrontare in modo diretto e globale un tema che gli sta a cuore: la religione. Per la precisione, quella cattolica romana. E nel dipingere le tre pale del suo romanzo imperniato attorno all'abbazia del Beato Leibowitz, dove i monaci dell'ordine omonimo custodiscono (seicento anni dopo la Terza guerra mondiale) documenti e progetti scientifici del passato come memorabilia, senza minimamente comprenderne il significato, Miller si mantiene al largo da qualsiasi tono apologetico. Le sue figure minuziosamente connotate servono anche a intavolare discussioni sulla legittimità di certi usi del progresso scientifico, sulla validità delle vocazioni e su altri temi religiosi, ma l'occhio che le osserva crescere mantiene un garbato tono ironico, conscio del fatto che sotto un saio o sotto gli stracci di un mutante si trovano gli stessi uomini. Uomini che cercano di conservare al genere umano la stessa dignità che può valere per un credente o un brigante di strada, anche sotto gli occhi delle poiane che ormai formano un'inamovibile eredità del passato atomico.
