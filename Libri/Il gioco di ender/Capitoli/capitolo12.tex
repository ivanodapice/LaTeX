\chapter{BONZO}

{~}

{~}

{~}

{-- \emph{Prego, si sieda, generale Pace. Mi pare d'aver capito che lei
		sia venuto a parlarmi di un argomento ritenuto urgente.}}

{-- \emph{Proprio così. Fino ad oggi, colonnello Graff, mi sono fatto un
		dovere di non interferire con l'andamento interno della Scuola di
		Guerra. Le è stata garantita l'autonomia, e malgrado la nostra
		differenza di grado sono conscio di non poterle ordinare certe
		precauzioni, ma solo di consigliarle.}}

{-- \emph{Precauzioni?}}

{-- \emph{Non faccia l'ingenuo con me, colonnello Graff. Voi americani
		siete dei dilettanti in questo, confronto a certe dannate facce di
		bronzo sovietiche con cui ho spesso a che fare. Lei sa bene perché sono
		qui.}}

{-- \emph{Ah! Suppongo che Dap si sia deciso a fare quel suo rapporto.}}

{-- \emph{Lui ha tendenze\ldots{} paterne verso gli studenti che sono
		qui. Afferma che lei, trascurando una situazione potenzialmente letale,
		sia non solo negligente, ma \ldots uh, complice di una cospirazione
		mirante a provocare ferite gravi o mortali a uno dei vostri studenti.}}

{-- \emph{Questa è una scuola per ragazzi giovani, generale Pace.
		L'ultimo posto in cui il capo della Polizia Militare della F.I. possa
		avere un serio motivo per intervenire.}}

{-- \emph{Colonnello Graff, il nome di Ender Wiggin è già trapelato su
		per le alte sfere di comando. Abbiamo già dovuto mettere il bavaglio a
		un paio di servizi segreti troppo curiosi, e non le dico altro.
		Comunque, l'ho sentito definire a chiari termini come la nostra sola
		speranza di vittoria, contro l'imminente invasione. Quando siano in
		gioco la sua vita, o la sua salute, non mi sembra certo assurdo che la
		Polizia Militare cominci a pensare di dover proteggere il ragazzo, no?}}

{-- \emph{Dannazione a Dap e dannazione a lei, signore! Io so benissimo
		quello che sto facendo.}}

{-- \emph{Lo sa?}}

{-- \emph{Meglio di chiunque altro.}}

{-- \emph{Oh, questo è evidente, dal momento che nessun altro ha la più
		piccola idea di ciò che lei sta facendo. Lei sa da otto giorni che
		esiste una cospirazione, fra alcuni dei più incorreggibili di questi
		«ragazzi», mirante a sconfiggere e umiliare Ender Wiggin. E sa che
		alcuni membri di questa combriccola, particolarmente un certo Bonito de
		Madrid, detto Bonzo, sono molto propensi ad agire con bestiale violenza
		nell'attuale circostanza. Così Ender Wiggin, una risorsa internazionale
		d'inestimabile importanza, viene messo a rischio d'essere ritrovato col
		cervello spiaccicato su una parete della sua tranquilla scuola orbitale.
		E lei, pienamente conscio di questo pericolo, propone di non fare altro
		che\ldots{}}}

{-- \emph{Niente.}}

{-- \emph{Vede da sé quanto la cosa possa lasciarci perplessi, no?}}

{-- \emph{Ender Wiggin è già stato in situazioni di questo genere: sulla
		Terra, il giorno che gli fu tolto il monitor. E anche allora contro un
		gruppo di ragazzi più anziani che\ldots{}}}

{-- \emph{Non sono venuto qui senza essermi documentato. Ender Wiggin ha
		provocato Bonzo Madrid oltre le sue capacità di sopportazione. E lei non
		ha neppure un uomo della Polizia Militare pronto a intervenire o a far
		opera di prevenzione. È irragionevole.}}

{-- \emph{Quando Ender Wiggin avrà il controllo delle nostre flotte,
		quando dovrà prendere decisioni da cui dipenderanno la vittoria o la
		distruzione dell'umanità, ci saranno lì squadroni della Polizia Militare
		a cui potrà rivolgersi perché gli tolgano le castagne dal fuoco?}}

{-- \emph{Mi scusi, ma non vedo il nesso.}}

{-- \emph{È chiaro. Ma il nesso c'è. Ender Wiggin deve avere la certezza
		che qualunque cosa accada nessun adulto interverrà mai ad aiutarlo. Deve
		sentirlo nel cuore e nelle viscere. Se in lui non si formerà questo
		istinto, non raggiungerà mai il vertice delle sue possibilità.}}

{-- \emph{Non lo raggiungerà neppure se sarà morto, o dovesse restare
		invalido per sempre.}}

{-- \emph{Questo non accadrà.}}

{-- \emph{Perché non si limita a promuovere Bonzo? Ha quasi finito il
		corso.}}

{-- \emph{Ender sa che Bonzo progetta di ucciderlo. Se lo trasferissimo
		anzitempo al corso superiore, capirebbe che noi stiamo qui a
		proteggerlo. Inoltre non sarò io a mandare alla Scuola Ufficiali un
		ragazzo palesemente inadatto al comando.}}

{-- \emph{Che mi dice degli altri ragazzi? Li metterà in condizioni di
		aiutarlo?}}

{-- \emph{Staremo a vedere cosa accadrà. Questa è stata e continua a
		essere la mia decisione irrevocabile.}}

{-- \emph{Dio l'aiuti se risulta che ha torto.}}

{-- Se \emph{ho torto, Dio ci aiuti tutti.}}

{-- \emph{Graff, se succede qualcosa di brutto al ragazzo io stesso le
		organizzerò la corte marziale. E farò in modo che il suo nome sia
		disprezzato da un capo all'altro del pianeta.}}

{-- \emph{Mi sta bene. Ma ricordi: se avrò visto giusto lei mi dovrà
		proporre per una dozzina di medaglie.}}

{-- \emph{Sì? E con quale menzione ufficiale?}}

{-- \emph{Per essere riuscito a tenermi la Polizia Militare fuori dai
		piedi.}}

{~}

\begin{center}
	{* * *}
\end{center}

{~}

{Seduto in un angolo della sala di battaglia, con un braccio agganciato
	a un corrimano, Ender osservava Bean che faceva pratica con la sua
	squadra. Il giorno prima avevano lavorato su una tecnica di attacco
	senza pistole, disarmando gli avversari coi piedi. Ender li aveva
	aiutati suggerendo colpi e mosse di lotta. Molte cose avrebbero dovuto
	esser cambiate, ma lo scontro di due corpi in volo a gravità zero
	presentava qualche possibilità sfruttabile.}

{Quei giorno, comunque, Bean aveva un giocattolo nuovo. Era una treccia
	molecolare, uno di quei sottili e quasi invisibili fili usati
	nell'edilizia spaziale per trattenere oggetti e carichi nelle vicinanze.
	A volte erano lunghe alcuni chilometri. Quella non superava in lunghezza
	la diagonale della sala, e tuttavia, mollemente arrotolata intorno alla
	cintura di Bean, era più o meno invisibile. La svolse e ne consegnò un
	capo a uno dei suoi soldati. -- Giralo intorno a quella ringhiera una
	ventina di volte, che stia ben fisso -- ordinò, e si spinse verso
	l'altra estremità della sala.}

{Come filo di sbarramento non sarebbe servito a molto, decise Bean. Era
	abbastanza invisibile, ma c'erano poche possibilità che un avversario
	andasse a sbatterci contro e ne fosse deviato. Questo gli diede però
	l'idea di usare la treccia per cambiare direzione a mezz'aria. Se ne
	arrotolò metà intorno alla cintura, lasciò l'altro capo fissato alla
	ringhiera e balzò in volo. La treccia lo bloccò di colpo, lo fece
	roteare su se stesso e trasformò la sua traiettoria in un arco al
	termine del quale Bean sbatté con violenza in una parete.}

{Nella sala risuonarono le sue urla; ma a Ender occorse qualche momento
	per capire che non stava gridando di dolore: -- Avete visto a che
	velocità andavo? Avete visto come ho cambiato direzione?}

{Poco dopo tutti i Draghi smisero l'allenamento per guardare Bean che
	s'impratichiva con la treccia molecolare. I mutamenti repentini di
	direzione facevano effetto, specialmente a chi non si rendeva conto che
	era legato con quel filo invisibile. Quando lo usò per roteare in orbita
	attorno a una stella riuscì a raggiungere una velocità stupefacente.}

{Erano le 2140 quando Ender fischiò la fine dell'allenamento serale.
	Stanca ma soddisfatta d'aver visto qualcosa di nuovo, la sua orda
	s'incamminò lungo i corridoi interni verso la camerata. Ender s'avviò
	fra loro, ascoltandone in silenzio i commenti e le spiritosaggini. Forse
	stavano pagando la fatica, rifletté: una battaglia al giorno per più di
	quattro settimane, spesso in situazioni che avevano messo duramente alla
	prova ogni loro risorsa. Ma erano orgogliosi, soddisfatti, uniti. Non
	avevano mai perso, e avevano imparato a confidare l'uno nell'altro.
	Sapevano che i loro compagni si battevano bene e con tenacia, sapevano
	che i loro capibranco non li facevano sudare in manovre prive di scopo,
	e soprattutto sapevano che lui li preparava ad affrontare tutte le
	eventualità.}

{Mentre oltrepassavano il vasto bar automatico, Ender notò parecchi
	ragazzi anziani riuniti in gruppetti che sembravano far conversazione
	nelle diramazioni del corridoio e sulle scale. Alcuni passeggiavano
	pigramente nel corridoio principale, o con le spalle poggiate a una
	parete avevano l'aria di chi aspetta qualcosa. Doveva essere più che una
	semplice coincidenza, rifletté, il fatto che molti di loro portassero
	l'uniforme delle Salamandre, mentre tutti gli altri appartenevano alle
	orde i cui comandanti lo odiavano di più. Alcuni lo sbirciavano e poi
	distoglievano in fretta lo sguardo, altri cercavano di apparire
	rilassati ma in realtà erano tesi e nervosi. \emph{Cosa potrei fare se
		aggredissero la mia orda qui nel corridoio? I miei ragazzi sono giovani,
		tutti fisicamente inferiori, e per niente addestrati alla lotta in
		gravità normale. Se avessimo avuto il tempo di\ldots{}}}

{-- Ehi, Ender -- lo chiamò una voce femminile. Si volse e vide Petra,
	sulla soglia del piccolo museo dei voli spaziali in compagnia di
	un'altra ragazzina. -- Ender, posso parlarti un momento?}

{Lui si rese conto che se si fosse fermato la sua orda sarebbe passata
	oltre, svoltando intorno alla sala musica e lasciandolo solo. --
	Facciamo quattro passi. Porta anche la tua amica -- disse.}

{-- Soltanto una parola. Aspetta.}

{Lui girò l'angolo insieme ai compagni. Pochi secondi dopo sentì i passi
	di Petra raggiungerlo di corsa. -- D'accordo, verrò in là con te. --
	Quando Ender la ebbe accanto s'irrigidì involontariamente. Era anche lei
	una di loro, una di quelli che lo odiavano abbastanza da volergli fare
	del male?}

{-- Un amico mi ha chiesto di avvertirti. Ci sono dei ragazzi che
	vogliono ucciderti.}

{-- Che sorpresa! -- esclamò Ender. I suoi compagni drizzarono gli
	orecchi. Li vide scambiarsi alcuni sussurri, con aria fra accigliata e
	disgustata.}

{-- Ender, guarda che possono farlo. Lui mi ha detto che lo stanno
	progettando fin da quando sei stato promosso comandante e\ldots{}}

{-- E ancor di più da quando ho battuto le Salamandre, vuoi dire?}

{-- Anch'io ti ho odiato, quando hai sconfitto l'orda delle Fenici.}

{-- Non ti biasimo. Avrai avuto i tuoi motivi.}

{Lei sbatté le palpebre. -- Comunque, lui mi ha detto di prenderti da
	parte, oggi, appena uscito dalla sala di battaglia, e di avvisarti che
	domani dovrai stare molto attento perché\ldots{}}

{-- Petra, se tu mi avessi preso da parte poco fa, in corridoio c'erano
	almeno una dozzina di ragazzi che avrebbero potuto spingermi dentro una
	stanza vuota. Vuoi darmi a intendere che non te n'eri accorta?}

{D'improvviso il volto di lei avvampò. -- No. Come puoi pensare una cosa
	simile? Non sai neppure chi sono i tuoi amici? -- Bruscamente la ragazza
	spinse da parte un paio di Draghi, girò un angolo e s'allontanò su per
	la scala che portava al ponte superiore.}

{-- È vero quel che ha detto? -- lo interrogò Tom il Matto.}

{-- Cosa dovrebbe esser vero? -- Ender si fermò sulla soglia della
	camerata, azzitti due o tre ragazzi che cominciavano a far baccano e
	ordinò loro di andare a letto.}

{-- Che alcuni dei più anziani si sono messi d'accordo per ammazzarti.}

{-- Tutte chiacchiere -- borbottò lui. Ma pensava esattamente il
	contrario. Petra aveva saputo qualcosa di concreto, e ciò che lui aveva
	visto nei corridoi non era frutto della sua immaginazione.}

{-- Saranno chiacchiere, ma spero che tu capisca di cosa parlo quando
	dico che i tuoi cinque capibranco adesso ti scorteranno fino in camera.}

{-- È una passeggiata superflua.}

{-- Diciamo che ci va di fare quattro passi.}

{Ender strinse i denti, seccato, ma sarebbe stato uno sciocco a
	rifiutare. -- Fate come volete -- disse. Si volse e uscì. I capibranco
	si accodarono a lui. Uno corse avanti e andò ad aprire la sua porta.
	Dopo aver controllato che nella stanza non lo attendesse una sorpresa, i
	ragazzi si fecero promettere che avrebbe chiuso a chiave. Uscirono,
	pochi secondi prima che si spegnessero le luci.}

{Sul suo banco lo attendeva un messaggio:}

{~}

\begin{center}
	{NON RESTARE MAI SOLO - DINK}
\end{center}

{~}

{Sul volto gli comparve un sorriso. Dunque Dink era ancora suo amico.
	\emph{Non preoccuparti. Non mi faranno proprio niente. Io ho la mia
		orda.}}

{Ma nel buio della notte non aveva altri che se stesso. Confusamente
	sognò di Stilson, e fu stupito di vedere quanto fosse piccolo: un
	bambino di appena sei anni. Com'erano ridicole le sue pose da duro! E
	tuttavia in quel sogno furono Stilson e i suoi amici a sopraffarlo e a
	picchiarlo, e tutto ciò che lui aveva fatto al ragazzo nella realtà gli
	venne restituito con gli interessi nella fantasia onirica. Poi vide se
	stesso strillare e farfugliare come un idiota mentre tentava di dare
	ordini all'orda dei Draghi, ma dalla bocca non gli uscivano che parole
	prive di senso.}

{Si svegliò nelle tenebre, gelato da una paura senza nome. Per
	scacciarla si ripeté che gli insegnanti certo lo stimavano, altrimenti
	non l'avrebbero sottoposto a quella pressione. \emph{Loro non
		permetteranno che mi succeda nulla. Nulla di grave, almeno.}
	Probabilmente, anni prima, quando i ragazzi anziani l'avevano assalito
	in sala di battaglia, fuori c'erano stati degli insegnanti a sorvegliare
	l'andamento della situazione ed in caso di necessità sarebbero
	intervenuti. \emph{Forse avrei potuto perfino mettermi a sedere, senza
		reagire, e loro li avrebbero fermati. In queste gare mi stanno addosso
		come aguzzini, ma fuori dalla sala vogliono che io sia al sicuro.}}

{Con quella riflessione tranquillizzante ricadde nel sonno, e a
	svegliarlo fu solo il fruscio della porta, il mattino dopo, quando sul
	pavimento svolazzò la notifica della battaglia che lo attendeva quel
	giorno.}

{~}

\begin{center}
	{* * *}
\end{center}

{~}

{Vinsero, naturalmente, ma arrivare alla porta nemica fu un inferno. La
	sala di battaglia era così fittamente piena di stelle che in quel
	labirinto lo scontro si trasformò in una stressante caccia all'uomo
	della durata di 45 minuti. Di fronte avevano i Tassi, di Pol Slattery,
	ed essi combatterono furiosamente. Inoltre era stata introdotta una
	nuova difficoltà: quando i Draghi colpivano un avversario agli arti
	costui restava disabilitato per non più di cinque minuti. Soltanto
	quelli completamente congelati erano fuori in via definitiva. Ma lo
	scongelamento non funzionava per l'orda dei Draghi. Il primo ad
	accorgersi di quel che stava accadendo fu Tom il Matto, allorché
	cominciarono a vedersi attaccare alle spalle da gente che credevano
	d'avere già tolto di mezzo. E alla fine della battaglia Pol Slattery
	venne a stringere la mano a Ender e dichiarò: -- Sono contento che abbia
	vinto tu, Ender. Il giorno che ti batterò voglio farlo lealmente.}

{-- Usa quello che ti danno -- sospirò lui. -- Se ti trovi con un
	vantaggio sul nemico, tu usalo.}

{-- Oh, è quel che ho fatto -- sogghignò Slattery. -- Io sono
	cavalieresco soltanto prima e dopo una battaglia.}

{Usciti di sala constatarono che avrebbero saltato la colazione; la sala
	mensa aveva già chiuso, a quell'ora. Ender guardò i suoi soldati che si
	avviavano in corridoio esausti e accaldati, e disse: -- Per oggi ne
	avete avuto abbastanza. Niente addestramento. Prendetevi un po' di
	riposo, svagatevi. Chi ha un esame, studi. -- Ed ebbe la misura della
	loro stanchezza quando nessuno applaudì o rise; si limitarono a sfilare
	in camerata togliendosi di dosso le tute umide di sudore. Se lui lo
	avesse chiesto, avrebbero fatto l'addestramento; ma erano al limite
	delle loro forze e dover stare senza colazione a qualcuno sembrava già
	l'ultima goccia.}

{Ender avrebbe voluto farsi una doccia, ma si sentiva la schiena a
	pezzi. Si distese sul letto con la tuta da battaglia addosso, per quello
	che gli parve un minuto, e quando si svegliò era quasi l'ora di pranzo.
	Così svaniva l'idea di andare in videoteca a studiare qualcos'altro
	sugli Scorpioni. C'era appena il tempo di darsi una lavata, mangiare, e
	filare in classe per le lezioni pomeridiane.}

{Si sfilò la tuta, con una smorfia per l'odore corporeo che la
	impregnava. Aveva dolori muscolari e le articolazioni rigide. \emph{Non
		avrei dovuto mettermi a dormire dopo quella faticata. Sto cominciando a
		cedere. Mi sono ammosciato. E questo non posso permettermelo.}}

{Così andò a correre un poco in palestra, e prima di passare nelle docce
	si arrampicò tre o quattro volte sulle corde. Non rifletté che la sua
	assenza dalla mensa dei comandanti sarebbe stata notata, né che andando
	a far la doccia all'ora di pranzo, con la sua orda occupata a rifarsi
	dalla perdita della colazione, sarebbe stato completamente solo e
	inerme.}

{Anche quando li sentì entrare nel locale delle docce non prestò loro
	molta attenzione. Stava assaporando la sensazione dell'acqua che gli
	scorreva sulla faccia e sul corpo, e il rumore dei passi sembrava
	lontano e soffocato. \emph{Escono già dalla mensa}, \emph{} pensò.
	Ricominciò a insaponarsi distrattamente. \emph{O forse è qualcuno che ha
		finito tardi l'addestramento.}}

{O forse no. Riaprì gli occhi e si volse. Erano in sette, chi fermo
	presso la fila dei WC, chi appoggiato a uno dei lavandini, e lo stavano
	fissando. Davanti a tutti c'era Bonzo. Alcuni di loro esibivano un
	sorrisetto contorto, la smorfia soddisfatta che il predatore si prende
	il lusso di concedere alla sua vittima. Bonzo però non stava
	sorridendo.}

{-- Ohé! -- li salutò Ender.}

{Nessuno rispose.}

{Lui si volse e chiuse la doccia, anche se aveva sempre un bel po' di
	schiuma addosso; poi allungò una mano verso l'accappatoio. Non era più
	lì. Uno dei ragazzi ci stava giocherellando. Era Bernard. Perché i
	personaggi di quella scena fossero al completo ci mancavano soltanto
	Peter e Stilson. Fra loro non avrebbe guastato il freddo sorriso di
	Peter, e neppure la grossolana imbecillità di Stilson.}

{Ender seppe che l'accappatoio era la loro esca, la mossa d'apertura.
	Nulla lo avrebbe reso più ridicolo e debole che andare dall'uno
	all'altro alla caccia di quell'indumento. Era questo che volevano:
	umiliarlo e farlo strisciare. Un gioco che lui non avrebbe giocato.
	Rifiutando di sentirsi a disagio perché era bagnato, infreddolito e nudo
	si tenne eretto e li fronteggiò, con le mani sui fianchi. Fissò Bonzo
	negli occhi.}

{-- A te la prima mossa -- lo sfidò.}

{-- Questo non è un gioco, furbone -- disse Bernard. -- Ne abbiamo piene
	le scatole di te. Ma visto quanto sei bravo ti promuoviamo\ldots{} al
	ghiaccio eterno.}

{Ender non guardò Bernard. Era negli occhi di Bonzo che vedeva il
	desiderio di uccidere, anche se taceva. Gli altri si tenevano a
	distanza, come incerti se fuggire e scoprire fin dove avrebbero avuto il
	coraggio di arrivare. Bonzo sapeva dove voleva arrivare.}

{-- Bonzo -- disse sottovoce Ender. -- Tuo padre sarebbe orgoglioso di
	te.}

{Bonzo strinse le palpebre.}

{-- Quanto sarebbe compiaciuto nel vederti adesso. Vieni a cercare il
	tuo avversario nudo sotto la doccia, grande e grosso come sei, e ti
	porti dietro sei amici. Direbbe che ti stai facendo davvero onore, eh?}

{-- Siamo soltanto venuti a farti una proposta amichevole -- disse uno
	dei ragazzi. -- Ci basta che tu perda qualche battaglia, una ogni tanto.
	Magari quella che ti diremo noi.}

{-- Magari anche tutte -- aggiunse Bernard.}

{Gli altri risero, ma Bonzo non fece una piega, e neppure Ender.}

{-- Sarai fiero di te, Bonito, coraggioso soldato. Poi potrai tornare a
	casa e raccontare ai tuoi ammiratori: sì, sono stato io a spezzare la
	schiena a Ender Wiggin, che non aveva neppure dieci anni quando io ne
	avevo tredici. E pensare che avevo soltanto sei amici, ma siamo riusciti
	lo stesso a dargliele, perché per fortuna lui era nudo e bagnato e solo.
	E loro diranno: diavolo! Hai avuto un bel fegato ad affrontare quello
	spaventoso e terribile Ender Wiggin senza portarti dietro almeno altri
	duecento coraggiosi amici\ldots{}}

{-- Tappati quella latrina di bocca, Wiggin -- disse uno di loro.}

{-- Non siamo venuti per sentir parlare questo piccolo bastardo -- disse
	Bernard. -- Non perdiamo altro tempo. Avanti.}

{-- Voialtri state zitti -- disse Bonzo. -- Chiudete la bocca e state
	fuori dai piedi. -- Cominciò a togliersi la tuta. -- Nudo, bagnato e
	solo, Ender. Così saremo alla pari. Se sono più grosso di te non
	possiamo farci niente, no? Tu sei tanto intelligente che sai cavartela
	sempre. -- Si volse agli altri. -- Andate a sorvegliare la porta. Che
	nessuno entri.}

{Il locale delle docce non era molto vasto, e ovunque sporgevano infissi
	e tubature. Era stato lanciato in orbita in un sol pezzo, come un
	satellite, pieno fino al soffitto di equipaggiamenti e materiali di ogni
	tipo, e non aveva spazi morti o non sfruttati al massimo. Non a caso lo
	avevano cercato lì dentro, si disse, dove un cranio fratturato poteva
	esser fatto passare per un incidente.}

{Quando vide Bonzo mettersi in posa si sentì un groppo in gola. Doveva
	aver preso lezioni di lotta, e si muoveva come un esperto. Aveva un
	allungo maggiore del suo, era più forte di lui e pieno d'odio. \emph{Non
		farà le cose a metà. Vuole spaccarmi il cranio}, \emph{} pensò Ender.
	\emph{Cercherà di lasciarmi qui dentro con la testa rotta, e se andiamo
		per le lunghe ci riuscirà. La sua forza avrà la meglio. Se voglio uscire
		di qui con le mie gambe devo vincere alla svelta, e definitivamente.}
	Gli parve di risentire lo spiacevole rumore con cui il naso di Stilson
	s'era rotto, quando l'aveva colpito con un calcio. \emph{Ma stavolta
		sarà il mio corpo a spezzarsi, a meno che prima io non spezzi lui.}}

{Ender indietreggiò, diede un colpetto alla testa di una doccia per
	sollevarla più in fuori e aprì il rubinetto dell'acqua calda. Il getto
	uscì, in un alone di vapore. Svelto aprì le altre docce della fila.}

{-- Non ho paura dell'acqua bollente -- disse sottovoce Bonzo,
	muovendosi verso di lui.}

{Ma a Ender non interessava l'acqua. Voleva il vapore. Aveva addosso una
	patina di sapone secco, e l'umidità avrebbe reso il suo corpo più
	sdrucciolevole di quel che Bonzo poteva gradire.}

{Improvvisamente dalla porta venne un grido: -- Basta, fermati! -- Per
	un attimo Ender pensò che fosse un insegnante capitato lì per caso, ma
	invece era Dink Meeker. Gli amici di Bonzo lo avevano immobilizzato
	sulla soglia, schiacciandolo col petto contro il montante della porta.
	Il ragazzo girò la testa. -- Smettila, Bonzo! -- urlò. -- Guai a te, se
	gli fai del male. Non puoi!}

{-- E perché non posso, eh? -- disse Bonzo, e per la prima volta
	sorrise. \emph{Ah}, \emph{} pensò Ender, \emph{gli piace far vedere a
		qualcuno che ha la situazione in mano, che è il più forte.}}

{-- Perché lui è il migliore, ecco perché! Chi altro può combattere gli
	Scorpioni come lui? È solo questo che conta, maledetti idioti, gli
	Scorpioni!}

{Bonzo smise di sorridere. Se doveva esserci una ragione inconfessabile
	per il suo odio, forse era proprio il sapere che Ender contava qualcosa
	per altra gente, mentre di sé non poteva dire lo stesso. \emph{Con le
		tue parole mi hai condannato, Dink. Per Bonzo, l'idea che io possa farmi
		onore anche fuori dalla Scuola è veleno.}}

\emph{{Dove sono gli insegnanti? }}{pensò, irritato. \emph{Non hanno
		capito che fra noi basta una lotta di pochi secondi per portare al
		dramma? Qui non siamo in sala di battaglia con addosso una tuta
		imbottita. Qui c'è la gravità, spigoli e angoli dove basta un colpo ed è
		la fine. Fermateci ora, o non ci fermerete più.}}

{-- Se lo tocchi sei uno sporco amico degli Scorpioni! -- gridò Dink. --
	Sei un traditore. Se gli fai del male meriti di crepare! E io\ldots{}
	uch! -- Gli altri ragazzi gli fecero sbattere la faccia contro lo
	spigolo della porta, e lui si afflosciò con un gemito.}

{L'atmosfera del locale era annebbiata dal vapore e il corpo di Ender
	sudava, imperlato di umidità. \emph{Adesso, mentre sono ancora
		abbastanza scivoloso per le sue mani.}}

{Fece un passo indietro, lasciando che la paura gli affiorasse
	liberamente sul volto. -- Bonzo\ldots{} smettila, adesso -- disse. --
	Per favore, lasciami stare.}

{Era questo che l'altro aspettava: la certezza di averlo in suo potere.
	Altri ragazzi si sarebbero accontentati di umiliarlo, ma per Bonzo
	quello era solo il segno che la violenza sarebbe stata facile. Alzò una
	gamba come per sferrargli un calcio, ma all'ultimo istante poggiò il
	piede a terra e gli balzò addosso. Pur colto di sorpresa Ender si
	abbassò d'istinto, per evitare d'essere afferrato per il collo.}

{La faccia di Ender sbatté dolorosamente contro le robuste costole
	dell'avversario, poi un ansito gli uscì dai polmoni quando le mani di
	lui gli si abbatterono sulla schiena in cerca di una presa. Ma le dita
	di Bonzo scivolarono in vani tentativi di affondarglisi nella carne, e
	lui girò su se stesso all'interno delle sue braccia. Un attimo dopo gli
	voltava le spalle. A quel punto la mossa più classica sarebbe stata di
	scalciarlo all'inguine con un calcagno, ma era un colpo che doveva
	essere preciso, e Bonzo se lo aspettava già; s'era alzato in punta di
	piedi e spostava i fianchi all'indietro per tenere il ventre fuori dalla
	sua portata. Pur senza vederlo Ender sapeva dove si trovava in quel
	momento la faccia di lui: proprio dietro la sua testa. E invece di
	scalciare allargò saldamente i piedi sul pavimento, poi il suo corpo
	s'inarcò con un violento scatto di reni, e lo colpì con la nuca in pieno
	volto.}

{Le braccia di Bonzo lo lasciarono all'istante, e voltandosi Ender lo
	vide vacillare all'indietro fra due docce, a occhi sbarrati e grondando
	sangue dal naso. Per un attimo Ender fu tentato di approfittarne per
	uscire da li, così com'era già uscito dalla sala di battaglia dopo aver
	fatto sputar sangue a due o tre aggressori. Ma come allora, la cosa
	avrebbe avuto un seguito, ancora e ancora, finché la sete di violenza di
	Bonzo non si fosse spenta. L'unica soluzione era di colpire Bonzo in
	modo che la paura finisse col diventare più forte del suo odio.}

{L'avversario aveva appena urtato la schiena contro il muro che Ender lo
	raggiunse con una forte ginocchiata nei testicoli. Bonzo mandò un
	grugnito e si piegò in due, girandosi di lato, ma lui gli fece sbattere
	la testa contro il tubo della doccia, che vibrò da cima a fondo. Poi,
	usando i gomiti invece dei pugni, lo colpì ancora ripetutamente sullo
	stesso lato del cranio.}

{Bonzo non emise un gemito e non reagì. Non tentò neppure di
	raddrizzarsi, mentre la sua testa continuava a sbattere rumorosamente
	contro il tubo metallico. Ma ad un tratto crollò in avanti e rotolò al
	suolo, direttamente sotto il getto di una delle docce. Restò lì
	immobile, senza accennare a togliersi via dal micidiale fiotto d'acqua
	bollente.}

{-- Cristo! -- gridò una voce. Gli amici di Bonzo si precipitarono a
	chiudere il rubinetto. Ender barcollò da parte sotto una spinta, e
	scivolò, ma una mano lo aiutò a rialzarsi e qualcuno gli porse
	l'accappatoio. Era Dink, che perdeva sangue da un labbro. -- Andiamocene
	da qui -- disse il ragazzo. Prese Ender per un gomito e lo portò fuori
	in fretta. Da lì a poco sentirono i passi pesanti di qualche adulto che
	scendeva di corsa per le scale. Adesso gli insegnanti si sarebbero fatti
	vivi. L'ufficiale medico e l'infermiere, per prendersi cura del suo
	aggressore. \emph{Dov'erano prima dello scontro, quando ancora non
		c'erano ferite da medicare?}}

{Le illusioni erano state spazzate via dalla mente di Ender. Adesso
	sapeva che era stato uno sciocco a fidarsi di Graff. Quella gente lo
	avrebbe lasciato crepare. Interessati a lui, certo, perfino premurosi
	dietro la loro durezza, ma lo avrebbero lasciato crepare, lì nelle
	docce. Nessuno lo avrebbe mai aiutato. Peter poteva essere una carogna,
	ma da quel lato aveva visto giusto, spietatamente giusto: il potere di
	causare sofferenza era il solo che gli altri rispettavano. Il potere di
	uccidere e di distruggere, perché chi non sapeva uccidere era sempre
	alla mercé di chi poteva farlo, e nulla e nessuno lo avrebbe salvato.}

{Dink lo accompagnò in camera sua e lo fece stendere sul letto. -- Pensi
	di avere qualche frattura? -- gli chiese.}

{Ender scosse il capo.}

{-- L'hai conciato male. Quando l'ho visto lì, non avrei scommesso uno
	sputo su di te. Invece l'hai ridotto molto male. Se non fosse caduto,
	credo che avresti continuato fino ad ammazzarlo.}

{-- Lui voleva ammazzare me.}

{-- Lo so. Lo conosco bene. Nessuno sa odiare come Bonzo. Ma è
	congelato, ormai. Se non lo rispediscono dritto a casa, non riuscirà più
	neppure a guardarti negli occhi. Ne te né chiunque altro. È venti
	centimetri più alto di te, e l'hai ridotto uno straccio.}

{Ma nella mente di Ender era rimasto impresso soltanto il tremito che
	aveva scosso Bonzo quando la sua testa aveva sbattuto nel tubo. Lo
	sguardo vitreo e morto dei suoi occhi. \emph{Era già finito fin da
		allora, già incosciente. Stava in piedi a occhi aperti, ma senza pensare
		e senza reagire. Con quell'espressione vuota, terribile, quasi oscena.
		La stessa faccia che aveva Stilson quando lo lasciai là per terra.}}

{-- Lo congeleranno, comunque -- continuò Dink. -- Tutti sanno che ha
	cominciato lui. Io li ho visti alzarsi insieme e uscire dalla mensa. Ci
	ho messo qualche secondo ad accorgermi che tu non c'eri, e poi un paio
	di minuti per scoprire dov'eri andato. Te l'avevo detto di non restare
	solo.}

{-- Già. Mi spiace.}

{-- Saranno costretti a congelarlo. È un cercaguai. Lui e il suo
	puzzolente senso dell'onore.}

{E in quel momento, con sorpresa di Dink, Ender cominciò a piangere.
	Disteso sulla schiena, ancora bagnato d'acqua e di sudore, tirò su col
	naso e lasciò che le lacrime gli si disperdessero sulle guance velate da
	tracce di schiuma secca. Un singhiozzo uscì dalla sua gola come un
	rantolo.}

{-- Sei sicuro di non avere niente?}

{-- Non volevo fargli del male! -- ansimò Ender. -- Perché non è stato
	capace di lasciarmi in pace?}

{~}

\begin{center}
	{* * *}
\end{center}

{~}

{Sentì la porta aprirsi con un fruscio, poi richiudersi. Pur
	semiaddormentato seppe che era la notifica per la battaglia di quel
	giorno. Socchiuse gli occhi, aspettandosi di trovare il buio del primo
	mattino, invece le luci erano già accese. Era nudo, e quando si mosse
	scoprì che le lenzuola erano umide. Nei suoi occhi, gonfi, era rimasto
	il dolore del pianto. Accese il banco per avere l'ora. 18,20 fu la cifra
	che comparve. \emph{È sempre lo stesso giorno. Ho già fatto una
		battaglia oggi. Ne ho fatte due\ldots{} quei bastardi sanno cos'ho
		passato, e continuano a farmi questo.}}

{~}

\begin{center}
	{ORDA DEI DRAGHI - Comandante Ender Wiggin}

{Sala di Battaglia, ore 1900}

{~}

{ORDA DEI GRIFONI - Comandante William Bee}

{ORDA DELLE TIGRI - Comandante Talo Momoe}
\end{center}

{~}

{Tornò a sedersi sul letto. Il foglio tremava fra le sue dita.
	\emph{Questo non lo posso fare}, \emph{} disse in silenzio. E poi ad
	alta voce: -- Questo non lo posso fare.}

{Si rialzò, stordito, e guardò attorno in cerca della tuta da battaglia.
	Poi ricordò: l'aveva messa in un pulitore automatico prima di far la
	doccia. Era ancora là.}

{Col foglio in mano uscì dal suo alloggio. L'ora di cena era quasi
	trascorsa e nei corridoi c'erano pochi ragazzi, ma nessuno gli rivolse
	la parola; in compenso raccolse parecchi sguardi intimoriti, forse a
	causa di quel che era successo a mezzogiorno nelle docce, forse per
	l'espressione fosca che gli aveva contratto il viso. Molti dei suoi
	ragazzi erano in camerata.}

{-- Ehilà, Ender! Facciamo un po' di allenamento stasera?}

{Lui consegnò il foglio a Zuppa Cinese, che mandò un grugnito. -- Questi
	figli di puttana -- disse. -- Due alla volta?}

{-- Due orde! -- sbottò Tom il Matto.}

{-- Si pesteranno i calli l'una con l'altra -- disse Bean.}

{-- Io vado a lavarmi -- disse Ender. -- Preparate i branchi e uscite.
	Vi raggiungerò alla porta.}

{Uscì dalla camerata lasciando dietro di sé un tumulto di chiacchiere,
	ma fece in tempo a sentire Tom il Matto che gridava: -- Due fottute
	orde! E con questo? Gli frusteremo il culo!}

{Nelle docce non c'era nessuno. Il pavimento era stato lavato. Neppure
	una delle gocce di sangue che Bonzo aveva lasciato sulla parete e in
	terra. Ogni traccia cancellata. Lì non era mai accaduto nulla di
	spiacevole e di sporco.}

{Ender avanzò sotto il getto d'acqua tiepida e si sciacquò, lasciando
	che il sudore di quel combattimento se ne andasse giù per lo scarico.
	\emph{Tutto eliminato, salvo che lo ricicleranno, e domattina ognuno
		berrà la sua dose del sangue di Bonzo. Sangue ormai senza vita, ma pur
		sempre sangue, e con esso il mio sudore. Il tutto versato in nome della
		stupidità o della crudeltà o di qualunque cosa li abbia convinti a
		lasciarlo succedere.}}

{Si asciugò, indossò la tuta e s'avviò verso la sala di battaglia. La
	sua orda stava aspettando in corridoio, presso la porta ancora chiusa, e
	quaranta sguardi lo seguirono in silenzio mentre andava a fermarsi di
	fronte al campo di forza bianco grigiastro. Tutti sapevano già che
	genere di battaglia li attendeva al di là di esso; questo, e la loro
	stanchezza residua dello scontro di quel mattino, li tratteneva dal
	darsi la carica con le solite grida. Dover affrontare insieme i Grifoni
	e le Tigri avrebbe messo a terra il morale di chiunque.}

\emph{{Qualunque cosa, purché serva a sconfiggermi}}{, \emph{} pensò
	Ender. \emph{Qualunque stratagemma riescano a pensare, sovvertendo anche
		le regole, senza fermarsi davanti a nulla pur di battermi. Be', sono
		stanco di questi giochi. Nessun gioco vale il sangue di un ragazzo
		sparso sul pavimento delle docce. Congelatemi, rispeditemi a casa, io
		non ci sto più.}}

{La porta si dissolse. Soltanto tre metri più avanti c'erano quattro
	stelle unite insieme, che bloccavano completamente la vista della sala.
	\emph{Due orde non bastavano. Devono anche tappare gli occhi alla mia
		orda.}}

{-- Bean -- disse, -- prendi i tuoi ragazzi e guarda cosa c'è dietro
	questa stella.}

{Bean svolse la treccia molecolare, se ne fissò un capo intorno alla
	cintura, diede l'altro a uno dei soldati della sua squadra e balzò
	lievemente oltre la porta. I cinque compagni lo seguirono subito. Era
	una manovra che avevano già sperimentato parecchie volte, e in pochi
	secondi riuscirono ad agganciarsi sulla parete interna della stella.
	Bean si spinse fuori a gran velocità, su una linea parallela alla porta;
	quando poi fu quasi all'angolo della sala scalciò contro la parete
	proiettandosi verso gli avversari. Lampi di luce sull'altro lato del
	locale lo informarono che questi gli stavano sparando addosso. Ma poiché
	era legato alla corda la sua traiettoria divenne un arco di cerchio
	facendo di lui un bersaglio impossibile, un arco che oltretutto si
	stringeva mentre la sua squadra tirava la treccia per recuperarlo sul
	lato opposto della stella. Appena i suoi lo ebbero portato al riparo
	mosse le braccia e le gambe, mostrando a quelli rimasti in corridoio che
	il nemico non lo aveva colpito da nessuna parte.}

{Ender lo raggiunse oltrepassando la soglia con un saltello.}

{-- È piuttosto scuro -- disse Bean, -- ma c'è abbastanza luce da non
	poter seguire facilmente le loro mosse grazie alla fluorescenza delle
	tute. Il tipo di illuminazione peggiore. È tutto spazio aperto, da
	questa stella fino alla parete opposta. Ma là ci sono otto stelle
	riunite in un quadrato attorno alla loro porta. Non ho visto nessuno,
	salvo quelli che sporgevano la testa per sbirciare fuori. Ci aspettano
	standosene tutti quanti appostati là dietro.}

{Come a corroborare il rapporto di Bean, uno degli avversari gridò in
	tono sfottente: -- Ehi, Dragamosci! Se volete pescare, non fatelo con
	uno stronzo attaccato alla lenza. Attaccateci le vostre sorelle, che noi
	stiamo qui ad abboccarcele!}

{Alcuni Draghi sogghignarono, impazienti di uscire con la pistola in
	mano, ma Ender non sapeva cosa pensare. Era una cosa stupida. Che
	possibilità aveva contro un nemico due volte più numeroso che per di più
	lo attendeva dietro una barricata? -- In una guerra vera, ogni
	comandante con due grammi di cervello terrebbe indietro i suoi uomini
	anche lui.}

{-- All'inferno! -- disse Bean. -- È soltanto un gioco.}

{-- Ha smesso di essere un gioco quando gli insegnanti hanno cominciato
	a capovolgere le regole.}

{-- Allora capovolgile anche tu.}

{Ender sogghignò. -- D'accordo. Perché no? Vediamo come i nostri amici
	reagiscono davanti a una formazione.}

{Bean sbarrò gli occhi. -- Una formazione? Ma non ne abbiamo mai fatta
	una da quando ci hanno messo in quest'orda!}

{-- Abbiamo avuto la nostra prima battaglia dopo un mese di
	addestramento, cioè quando di solito si comincia a lavorare in
	formazione. Ormai è tempo che impariate anche questo. -- Si volse alla
	porta, che era tornata a essere un muro opaco penetrabile da un solo
	lato, e con le dita segnalò: «branco A, avanti». I ragazzi emersero
	dalla parete d'energia, e lui cominciò a metterli in posizione al riparo
	della stella. Tre metri di spazio non erano molti, e la metà di loro
	erano confusi e di malumore, così ci vollero cinque minuti buoni prima
	che capissero il senso di ciò che stavano per fare.}

{Le Tigri e i Grifoni ingannavano il tempo gridando in coro sberleffi
	spiritosi di buon effetto, mentre i loro comandanti discutevano sulla
	possibilità di attaccare l'orda dei Draghi prima ancora che uscisse da
	dietro la stella. Momoe insisteva per l'attacco immediato. -- Cristo, li
	superiamo per due a uno! -- ripeteva, mentre la tesi di Bee era: --
	Appostati qui non possiamo perdere. Uscendo rischieremmo di scoprire che
	ha trovato un dannato modo per batterci.}

{Così restarono strettamente raggruppati dov'erano, finché in quella
	fosca penombra non videro una larga massa oscura emergere da dietro la
	stella dei Draghi. Aveva esattamente la stessa forma, e la mantenne
	anche quando smise di scivolare lateralmente e si proiettò dritta verso
	il centro vuoto del riparo quadrangolare usato dagli ottantadue
	avversari in attesa.}

{-- Quella è bella! -- esclamò un Grifone. -- I Draghi che vengono
	avanti in formazione!}

{-- Uno scudo! -- brontolò Momoe. -- E ci hanno impiegato cinque minuti
	per metterlo insieme. Se li attaccavamo allora, li avremmo già fatti a
	pezzi.}

{-- Rifletti, Momoe -- sussurrò Bee. -- Hai visto il modo in cui quel
	ragazzino è volato fuori. Ha fatto un giro intorno alla stella senza
	toccare una parete. Forse hanno ottenuto l'uso dei radioganci, non
	credi? Devono avere qualcosa di nuovo, quelli là.}

{La formazione era comunque strana: un quadrato formato da corpi
	strettamente uniti, come un muro, sulla parte anteriore. Dietro di esso
	un cilindro, con la circonferenza fatta da sei ragazzi e una profondità
	di due; tutti quanti però completamente congelati e rigidi, cosicché non
	si capiva come riuscissero a tenersi uniti. E tuttavia qualcosa li
	teneva uniti quasi che fossero legati l'uno all'altro\ldots{} il che,
	infatti, era vero.}

{Dall'interno di quella formazione altri Draghi stavano sparando con
	rapidità raffiche di colpi, e per un poco le Tigri e i Grifoni furono
	costretti a restare dietro le loro stelle.}

{-- La parte posteriore di quell'affare lì è aperta -- stabilì Bee. --
	Appena saranno abbastanza vicini potremo aggirarli e\ldots{}}

{-- Non starne a parlare, fallo! -- esclamò Momoe. Senza perdere altro
	tempo ordinò ai suoi ragazzi di lanciarsi contro le pareti e rimbalzare
	dietro la formazione dei Draghi.}

{Nel caos della partenza delle Tigri, mentre i Grifoni si riunivano
	anch'essi lungo i bordi esterni del loro riparo, qualcosa mutò nella
	formazione dei Draghi: sia il cilindro che il muro frontale si aprirono
	in due, come se all'interno ci fosse stato un movimento molto energico,
	e all'istante essa invertì la direzione, tornando verso la porta da cui
	era partita. I Grifoni cominciarono a sparare, lanciandosi avanti,
	mentre la manovra aggirante delle Tigri riusciva perfettamente. I corpi
	dei Draghi pullulavano di cerchietti luminosi, centrati da decine e
	decine di colpi. Nella penombra Momoe mandava urla vittoriose.}

{Ma c'era qualcosa di strano, di sbagliato. Bee ci rifletté un momento e
	capì di cosa si trattava. Quella formazione non poteva aver invertito il
	volo a mezz'aria senza che qualcuno non si fosse spinto nella direzione
	opposta, e se questo qualcuno era partito con tanta forza da rimandare
	indietro la massa dei suoi compagni doveva esser schizzato via a gran
	\emph{velocità.} Ringhiando un'imprecazione Bee si volse.}

{Ed era là: sei ragazzi con l'uniforme dei Draghi, proprio attaccati
	alla porta dei Grifoni e delle Tigri. Non c'erano arrivati sani, però, e
	con sollievo Bee vide che almeno cinque erano parzialmente inabilitati;
	soltanto uno era ancora intatto. Niente di cui preoccuparsi, dunque, si
	disse Bee. Puntò la pistola su uno di loro, prese con calma la mira,
	tirò il grilletto e\ldots{}}

{Non accadde niente.}

{Le luci si accesero.}

{La battaglia era finita.}

{Anche se li aveva guardati e continuava a guardarli, Bee ci mise un po'
	per capire cos'era successo. Quattro Draghi avevano posto il casco a
	contatto degli angoli luminosi della porta. E un quinto ci era passato
	attraverso. Insomma, avevano compiuto il rituale dell'apertura della
	porta nemica, e nient'altro. La loro orda era praticamente distrutta,
	non avevano inflitto ai Grifoni e alle Tigri la minima perdita, e
	avevano avuto l'incredibile sfacciataggine di andare a compiere il
	rituale della vittoria, causando l'accensione delle luci e la fine della
	battaglia.}

{Soltanto allora nella mente di William Bee si fece strada il sospetto
	che l'orda dei Draghi avesse non solo posto fine alla partita: esisteva
	la possibilità che, stiracchiando le regole, l'avessero anche vinta.
	Dopotutto, qualunque cosa accadesse in quel locale, un'orda non veniva
	registrata come vittoriosa finché i superstiti non fossero riusciti a
	toccare contemporaneamente i quattro angoli della porta nemica, mentre
	un quinto passava oltre nel corridoio. Di conseguenza se ne poteva
	arguire che il rituale della vittoria \emph{fosse} la vittoria.
	Comunque, le apparecchiature automatiche della sala di battaglia avevano
	reagito a quel gesto, decretando la fine.}

{La porta degli insegnanti si aprì, e il maggiore Anderson fluttuò
	all'interno. -- Ender! -- chiamò, guardandosi attorno.}

{Uno dei Draghi completamente congelati mandò un mugolio all'interno del
	casco ermeticamente chiuso. Anderson usò il radiogancio per avvicinarlo
	e lo scongelò.}

{Ender stava sorridendo. -- L'ho sconfitta di nuovo, signore -- disse.}

{-- Questo è un controsenso, Ender -- rispose l'ufficiale, -- I tuoi
	avversari erano i Grifoni e le Tigri.}

{-- Fino a che punto crede che io sia stupido? -- chiese Ender.}

{Ad alta voce Anderson annunciò: -- Dopo questa\ldots{} uh, manovra,
	tutto il regolamento sarà revisionato, introducendo l'obbligo che ogni
	soldato nemico sia congelato o disabilitato prima che la porta possa
	essere riaperta.}

{-- Comunque, la cosa poteva funzionare soltanto una volta -- aggiunse
	Ender.}

{Anderson gli consegnò il radiogancio. Ender scongelò i ragazzi tutti
	insieme. \emph{Al diavolo il protocollo. Al diavolo tutto.} -- Ehi! --
	gridò poi, mentre Anderson usciva. -- Cosa farete la prossima volta? La
	mia orda chiusa in una gabbia e senz'armi, e con tutto il resto della
	scuola contro di noi? A quando uno scontro da pari a pari?}

{Nella sala si alzò un mormorio di consensi, e non soltanto da parte dei
	Draghi. Anderson non si prese la briga di voltarsi per replicare alla
	sfida di Ender. Fu William Bee a rispondergli: -- Ender, se tu sei con
	una delle due parti in lotta non sarà mai uno scontro pari, qualunque
	cosa studino quelli.}

{-- Proprio così! È vero! -- esclamarono i ragazzi. Molti di loro
	risero. Talo Momoe cominciò a battere le mani e a gridare: -- En-der!
	En-der! En-der! -- Le sue Tigri e i Grifoni lo imitarono quasi tutti,
	applaudendo e continuando a ridere divertiti.}

{Dopo aver stretto la mano a Bee e a Momoe, Ender uscì dalla porta
	nemica. I suoi soldati gli si accodarono, e il coro di quelli che
	continuavano a gridare il suo nome li seguì lungo i corridoi.}

{-- Ci alleniamo, stasera? -- domandò Tom il Matto.}

{Ender scosse la testa.}

{-- Domani mattina, allora?}

{-- No.}

{-- Be', quando?}

{-- Mai più, per quello che riguarda me.}

{Alle sue spalle si levarono dei mormorii.}

{-- Ehi, questo non è leale -- disse uno dei ragazzi. -- Non è colpa
	nostra se gli insegnanti stanno stravolgendo le gare. E non puoi
	smettere di insegnarci e di guidarci soltanto perché\ldots{}}

{Ender sbatté una mano aperta contro il muro e si volse di scatto. --
	Non mi importa più un accidente di queste gare! -- Il suo grido echeggiò
	lungo il corridoio delle camerate. Ragazzi di altre orde misero la testa
	fuori dalle loro porte. Nel silenzio la voce di lui suonò bassa e secca:
	-- Non me ne importa. Chiaro? È finito -- sussurrò. -- Il gioco è
	finito.}

{Senza guardare nessuno tornò in camera sua. Avrebbe voluto sdraiarsi,
	ma quando toccò il letto lo sentì ancora umido. Questo gli ricordò quel
	che gli era successo, e furioso strappò via le lenzuola e il materasso
	scaraventando tutto quanto nel corridoio. Poi arrotolò una tuta per
	farne un cuscino e si sdraiò sulla rete elastica del letto. Era scomoda,
	ma gli parve perfettamente intonata alle sue riflessioni.}

{Le stava rimuginando da non più di dieci minuti quando qualcuno bussò
	alla porta.}

{-- Andatevene -- borbottò. Ma chiunque fosse non lo udì, o non gli
	importava. Alla fine Ender gli disse di entrare.}

{Era Bean.}

{-- Vattene, Bean.}

{Il ragazzo annuì, ma non si mosse. Con aria imbarazzata si guardò le
	scarpe. Il primo impulso di Ender fu di mettersi a urlare, di maledirlo
	e di ordinargli di lasciarlo in pace. Poi notò l'aspetto teso e depresso
	di Bean, le sue spalle curve per la stanchezza, gli occhi cerchiati
	dalla mancanza di sonno; e tuttavia la sua pelle era liscia e quasi
	trasparente, la pelle di un bambino. Le guance tenere di un bambino, i
	fianchi snelli di un bambino. Non aveva neppure otto anni. Per quanto
	fosse brillante, volonteroso e deciso era un bambino. Era
	\emph{giovane.}}

\emph{{No, non lo è del tutto}}{, \emph{} si corresse Ender.
	\emph{Piccolo, certo. Ma sa già cosa significa battersi con una truppa
		che dipende da lui e dalla sua squadra, e ci ha dato la vittoria con la
		sua risolutezza. Non c'è niente di infantile in questo.}}

{Interpretando il silenzio e l'espressione di Ender come un consenso,
	Bean chiuse la porta e si avvicinò al suo letto. Solo in quel momento
	lui vide che aveva in mano un foglio.}

{-- Sei stato trasferito? -- gli chiese. Era incredulo, ma la voce che
	si sentì uscire di bocca era smorta e piatta.}

{-- All'orda delle Lepri.}

{Ender annuì. \emph{Naturalmente. Era ovvio. Se io ho un'orda che non
		può essere sconfitta, quelli devono togliermela.} -- Carn Carby è in
	gamba -- sospirò. -- Spero che sappia riconoscere i tuoi meriti.}

{-- Carn Carby è stato promosso oggi. Gliel'hanno fatto sapere poco fa,
	mentre eravamo in sala di battaglia.}

{-- Bene. Adesso chi è al comando dell'orda?}

{Bean allargò le braccia con aria rassegnata. -- Io.}

{Ender fissò lo sguardo sul soffitto e annuì. -- È naturale. Dopotutto
	sei soltanto quattro anni più giovane dell'età prevista.}

{-- Non mi sembra divertente. Non so cosa stia succedendo qui. Tutti
	quei cambiamenti nelle gare. E adesso questo. Io non sono il solo a
	essere trasferito, sai. Hanno promosso metà dei comandanti, e messo un
	bel po' di noialtri al comando delle loro orde.}

{-- Chi di noi?}

{-- Sembra che\ldots{} tutti i capibranco e i loro vice.}

{-- È chiaro. Se hanno deciso di indebolire la mia orda, quelli la
	radono al suolo. Qualunque cosa facciano, non la fanno mai a metà.}

{-- Tu vincerai ancora, Ender. Tutti ne siamo convinti. Tom il Matto ha
	detto: «Ma mi ci vedi a comandare un'orda che debba battere i Draghi?»
	Tutti sanno che sei il migliore. Non riusciranno a spezzarti ora,
	qualunque cosa\ldots{}}

{-- L'anno già fatto.}

{-- No, Ender. Oggi hai dimostrato che\ldots{}}

{-- Non m'importa più niente di questi giochi, Bean. Io non gioco più.
	Niente più addestramenti, niente più battaglie. Possono consegnarmi qui
	dentro tutte le notifiche che vogliono, ma io lascio perdere. L'ho
	deciso oggi prima di entrare in sala di battaglia. E se ho fatto di
	tutto per vincere è perché volevo andarmene con stile, solo per questo.}

{-- Avresti dovuto vedere la faccia di William Bee. Non ce la faceva a
	raccapezzarsi all'idea che tu avessi vinto con sei ragazzi mezzo
	congelati, mentre in sala c'erano ottantadue di loro ancora tutti sani.}

{-- Perché dovrei stare a pensare alla faccia di William Bee? Perché
	dovrei voler battere questo e quello? -- Ender si appoggiò le palme
	delle mani sugli occhi. -- Oggi ho fatto del male a Bonzo. Del male sul
	serio, Bean.}

{-- Se l'è cercata.}

{-- Non cadeva, e io continuavo a colpirlo. Stava in piedi come un pezzo
	di carne morta, e io gli sbattevo la testa nel muro\ldots{}}

{Bean non disse niente.}

{-- Volevo essere sicuro che non potesse mai più minacciarmi così.}

{-- Non lo farà -- disse Bean. -- Lo spediscono a casa.}

{-- Di già?}

{-- Gli insegnanti non hanno detto molto, come al solito. La notizia
	ufficiale è che l'hanno promosso, ma nello spazio dove scrivono
	l'assegnazione\ldots{} sai, Corso Piloti, o Scuola Armamenti, Corso
	Sottufficiali, o Specializzazioni Tecniche, questo genere di
	cose\ldots{} be', c'è scritto Cartagena, Spagna. È casa sua.}

{-- Sono contento che l'abbiamo promosso.}

{-- Diavolo, Ender, noi siamo contenti che sia fuori. Se avessimo saputo
	cosa voleva farti l'avremmo ammazzato a sangue freddo. È vero che ti ha
	aggredito con tutta una banda di altre carogne? Si dice che\ldots{}}

{-- No. Soltanto lui e io. E si è battuto onorevolmente. -- \emph{Se non
		fosse stato per il suo senso dell'onore, comunque, gli altri mi
		sarebbero venuti addosso tutti insieme. E avrebbero potuto ammazzarmi.
		Questo mi ha salvato la vita.} -- Io invece non sono stato a pensare al
	mio onore -- aggiunse sottovoce. -- Mi sono battuto per vincere.}

{Bean rise. -- E l'hai fatto. Gli hai mollato un calcio che lo farà
	filare a razzo fin sulla Terra.}

{Bussarono alla porta. Ma prima che Ender potesse rispondere questa si
	aprì. S'era aspettato qualcun altro dei suoi soldati, invece era il
	maggiore Anderson. E dietro di lui venne dentro il colonnello Graff.}

{-- Ender Wiggin -- disse Graff.}

{Lui si alzò. -- Sì, signore.}

{-- L'insolenza di cui hai dato prova oggi in sala di battaglia è stata
	eccessiva, e non deve ripetersi.}

{-- Sissignore -- disse Ender.}

{Bean era però ancora d'umore insubordinato, e quel rimprovero gli parve
	ingiusto. -- Signore, secondo me era tempo che qualcuno dicesse a un
	insegnante come la pensiamo su quello che avete fatto.}

{I due adulti lo ignorarono. Anderson porse a Ender un foglio di carta.
	Formato protocollo, non come quelli stampati dal computer che servivano
	per le comunicazioni interne. Era fitto di ordini e di istruzioni. Bean
	sapeva cosa significava: Ender era stato trasferito fuori dalla Scuola.}

{-- Promosso? -- gli chiese. Ender annuì. -- Perché ci hanno messo
	tanto? Sei solo di due o tre anni in anticipo sull'età minima. Comunque
	hai imparato a camminare, a parlare e a vestirti da solo. Cos'altro gli
	rimarrebbe da insegnarti?}

{Ender scosse il capo. -- Tutto ciò che so è che il gioco è finito. --
	Ripiegò il foglio. -- Mai troppo presto per me. Posso dirlo all'orda?}

{-- Non c'è tempo di girare per la Scuola in cerca dei tuoi conoscenti
	-- disse Graff. -- La tua navetta parte fra venti minuti. Questo rende
	tutto più facile.}

{-- Per noi o per lei? -- chiese Ender. Non si aspettava una risposta.
	Si volse a Bean, gli strinse forte la mano per un momento, poi andò alla
	porta.}

{-- Aspetta -- disse Bean. -- Dove sei stato assegnato? Specializzazione
	Tecniche? Corsi di Tattica? Armamenti?}

{-- Scuola Ufficiali -- rispose Ender.}

{-- Al corso di \emph{preparazione?}}

{-- Al corso ufficiali -- disse Ender, e uscì in corridoio. Anderson lo
	seguì. Bean prese il colonnello Graff per una manica. -- Ma nessuno va
	alla Scuola Ufficiali, prima dei sedici anni.}

{Graff si liberò dalla sua mano, uscì e chiuse la porta dietro di sé.}

{Bean rimase solo nella stanza, stentando ad afferrare il significato di
	quel che aveva udito. Nessuno s'era mai iscritto alla Scuola Ufficiali
	senza aver fatto i tre anni di corso preparatorio in una delle
	specializzazioni tecniche o logistiche. Del resto, nessuno era mai
	uscito dalla Scuola di Guerra prima d'aver completato i sei anni del
	corso, e Ender ne aveva fatti solo quattro.}

\emph{{Il sistema sta andando a rotoli. Non c'è dubbio, ormai. O
		qualcuno negli alti comandi è impazzito, o la guerra ha avuto una brutta
		svolta. La guerra vera, quella con gli Scorpioni. Per quale altro motivo
		avrebbero stravolto i regolamenti interni a questo modo, e inasprito le
		gare? Per quale altro motivo avrebbero messo un bambino come me al
		comando di un'orda?}}

{Bean continuò a rimuginare quelle domande mentre tornava in camerata.
	Le luci si spensero proprio mentre si fermava accanto alla sua cuccetta.
	Si spogliò nel buio, e irritato annaspò a lungo prima di riuscire ad
	appendere la tuta nell'armadietto. Non s'era mai sentito così depresso.
	Dapprima aveva pensato che la causa fosse l'inconscia paura di comandare
	un'orda, ma non era così. Sapeva che sarebbe stato un buon comandante.
	Ma continuava ad aver voglia di piangere. Non lo aveva fatto più dopo i
	primi giorni dal suo arrivo, quando la nostalgia di casa lo assaliva.
	Cercò di dare un nome al groppo che aveva in gola, alla sensazione che
	gli spingeva le lacrime agli occhi ad onta dei suoi sforzi per
	ricacciarle indietro. Si mise un pollice in bocca e lo morse, per
	sostituire un dolore noto a uno oscuro. Non funzionò. Non avrebbe mai
	più rivisto Ender, mai più.}

{Appena capì che la sua spina era quella, riuscì pian piano a levarsela
	dalla carne. Disteso sulla cuccetta fece esercizi di respirazione finché
	il bisogno di piangere scomparve. Poi si girò su un fianco e cercò di
	dormire, ma per qualche minuto il suo respiro continuò a essere rapido e
	secco, la fronte corrugata, gli orecchi tesi ai piccoli rumori notturni
	dei suoi compagni. Lui era lì con loro, ed era un soldato. Se qualcuno
	fosse venuto a chiedergli cos'avrebbe voluto fare da grande, non avrebbe
	avuto altra risposta da dargli.}

{~}

\begin{center}
	{* * *}
\end{center}

{~}

{Fu mentre si trasferiva sulla navetta che Ender notò per la prima volta
	i gradi sull'uniforme del maggiore Anderson. Erano diversi. -- Sì, ora è
	colonnello -- gli spiegò Graff. -- In effetti, oggi pomeriggio il
	colonnello Anderson è stato messo alla direzione della Scuola di Guerra.
	Io sono stato assegnato a un altro incarico.}

{Ender non gli chiese quale.}

{Graff fluttuò giù fra i braccioli della poltroncina accanto alla sua,
	dall'altra parte del passaggio centrale. In cabina c'era soltanto un
	altro passeggero, un uomo in borghese, dall'aria tranquilla, che gli era
	stato presentato come il generale Pace. Aveva con sé una valigetta
	portadocumenti, ma non più bagaglio di quello che avevano lui e Graff.
	Per qualche ragione, il fatto che anche Graff se ne andasse da lì a mani
	vuote gli parve consolante.}

{Durante il tragitto verso la zona più interna del sistema solare Ender
	parlò una volta sola. -- Perché stiamo tornando sulla Terra? -- domandò.
	-- Credevo che la Scuola Ufficiali fosse da qualche parte fra gli
	asteroidi.}

{-- È così -- annuì Graff. -- Ma la Scuola di Guerra non ha strutture di
	attracco per le astronavi di stazza e autonomia maggiore. Farai una
	breve tappa intermedia sulla Terra.}

{Ender fu tentato di chiedere se avrebbe rivisto la sua famiglia. Ma
	d'improvviso, al pensiero che fosse possibile, fu colto da una strana
	paura e tacque. Preferì chiudere gli occhi e cercare di farsi la sua
	nottata di sonno. Poco più indietro, il generale Pace continuava forse a
	scrutarlo e studiarlo, mosso da interessi che lui aveva subito
	rinunciato a immaginare.}

{Quando atterrarono in Florida, Ender scoprì che quello era un caldo
	pomeriggio di mezza estate. Era stato tanto a lungo lontano dalla luce
	del sole che se ne sentì subito pugnalare gli occhi. Strinse le
	palpebre, arricciò il naso al puzzo del carburante e desiderò entrare in
	qualche luogo chiuso. Tutto gli appariva lontano, steso su distanze
	eccessive e stranamente piatto. Il terreno, senza la curvatura all'insù
	dei pavimenti su cui aveva camminato per quattro anni, dava
	l'impressione di curvarsi in basso, e benché sapesse di essere su una
	pista piatta Ender si sentiva come sulla cima di una collinetta. Anche
	l'attrazione gravitazionale mancava di quella lieve spinta laterale
	dovuta alla rotazione centrifuga, e nel camminare i suoi piedi avevano
	un noioso sbandamento. Detestò quell'insieme di sensazioni estranee.
	Avrebbe voluto tornarsene a casa: alla Scuola di Guerra, l'unico posto
	dell'universo a cui il suo corpo sembrava appartenere.}

{~}

\begin{center}
	{* * *}
\end{center}

{~}

{-- \emph{Arrestato?}}

{-- \emph{Be', mi è parsa una deduzione logica. Il generale Pace è il
		capo della Polizia Militare. E c'è stato un morto alla Scuola di
		Guerra.}}

{-- \emph{Non mi hanno detto se il colonnello Graff sia stato promosso
		oppure deferito alla Corte Marziale. Soltanto trasferito, ecco quello
		che so, con l'ordine di mettersi a rapporto dal Condottiero.}}

{-- \emph{Questo è buon segno?}}

{-- \emph{Chi lo sa? D'altra parte Ender Wiggin non soltanto è
		sopravvissuto a ogni prova e test, ma ne è uscito in ottima forma.
		Bisogna pur dare Credito di questo al vecchio Graff. Per contro, c'è il
		quarto passeggero della navetta. Quello che ha viaggiato in una cassa.}}

{-- \emph{È soltanto il secondo nell'intera storia della Scuola. Almeno
		non si è trattato di un suicidio, stavolta.}}

{-- \emph{L'omicidio è per qualche verso preferibile, maggiore Imbu?}}

{-- \emph{Non è stato un omicidio, colonnello. Abbiamo registrazioni
		video prese da due diverse angolature. Nessuno può incolpare Ender.}}

{-- \emph{Ma potrebbero incriminare Graff. Quando tutto questo sarà
		finito, una commissione senatoriale potrebbe passarci al setaccio e
		stabilire chi ha commesso degli illeciti e chi no. E darci una medaglia
		se non abbiamo offeso il loro senso estetico, oppure toglierci la
		pensione o metterci in galera. Se non altro, comunque, hanno avuto il
		buon senso di non dire a Ender che il ragazzo è morto.}}

{-- \emph{Ed è la seconda volta.}}

{-- \emph{Già. Non gli hanno detto neppure di Stilson.}}

{-- \emph{Ma il ragazzo ha paura di se stesso.}}

{-- \emph{Ender Wiggin non è un killer. Lui si limita a vincere\ldots{}
		definitivamente. Se qualcuno deve averne paura, lasciamo che siano gli
		Scorpioni.}}

{-- \emph{Sembra quasi che lei li compatisca, al pensiero di mandargli
		addosso Ender Wiggin.}}

{-- \emph{L'unico per cui sono rattristato è Ender. Ma non lo sono
		abbastanza da suggerire che lo mettano da parte. Ho appena avuto accesso
		al materiale che finora Graff teneva chiuso col suo codice personale.
		Sui movimenti della flotta, cose di questo genere. Credo che potrò
		dimenticare cosa significa dormire sonni tranquilli.}}

{-- \emph{Il tempo stringe, vero?}}

{-- \emph{Non avrei dovuto parlargliene. Sono informazioni riservate.}}

{-- \emph{Lo so.}}

{-- \emph{Una cosa però posso dirgliela: non era troppo presto per
		trasferirlo alla Scuola Ufficiali. Anzi, forse è stato fatto con un paio
		d'anni di ritardo.}}

\phantomsection\label{Orsonux20Scottux20Cardux20-ux20Ilux20Giocoux20Diux20Enderux20-ux20BY_SLY70A1_split_015.htm}{}
