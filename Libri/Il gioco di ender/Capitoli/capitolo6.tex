\chapter{IL GIGANTE}

{~}

{~}

{~}

{-- \emph{In passato abbiamo avuto fin troppe delusioni. Ce li alleviamo
		per anni, li facciamo ballare sul filo del rasoio sperando ansiosamente
		che se la cavino, e poi loro non ce la fanno. Ma con Ender sarà tutto
		più semplice: sembra deciso a finire congelato entro i prossimi sei
		mesi.}}

\emph{{-- Ah!}}

{-- \emph{Non vede quello che sta succedendo? Si è fissato su uno dei
		test mentali, il Drink del Gigante. Il ragazzo ha per caso tendenze
		suicide? Lei non ne ha mai parlato.}}

{-- \emph{Tutti si cimentano col Gigante, una volta o l'altra.}}

{-- \emph{Ma Ender rifiuta di cedere. Come Pinual.}}

{-- \emph{Tutti reagiscono un po' come Pinual, prima o poi. Ma lui resta
		il solo che si è suicidato. E non credo che la cosa fosse collegata al
		Drink del Gigante.}}

{-- \emph{Lei ci sta scommettendo la mia carriera. E guardi cos'ha
		combinato col suo gruppo.}}

{-- \emph{Sa bene che non è stata colpa sua.}}

{-- \emph{Non m'importa, Colpa sua o meno, sta avvelenando quel gruppo.
		Si presume che i membri di un gruppo debbano sentirsi uniti, ma dove
		entra lui si aprono abissi larghi un miglio.}}

{-- \emph{Non progetto di lasciarlo lì a lungo, comunque.}}

{-- \emph{Allora, meglio che riveda i suoi progetti. Quel gruppo si è
		ammalato, e lui ne è stato il virus. Ma non è allontanandolo che potremo
		curare gli altri. Al contrario, deve restare.}}

{-- \emph{Sono stato io a causare questa malattia. L'ho isolato dagli
		altri, e l'effetto non è mancato.}}

{-- \emph{Gli dia tempo. Vediamo se riesce a sbrogliare la situazione.}}

{-- \emph{Di tempo non ne abbiamo.}}

{-- \emph{Dobbiamo averlo, visto che si tratta di capire se abbiamo per
		le mani uno che ha le stesse probabilità di diventare un genio militare
		oppure un mostro.}}

{-- \emph{Questo è un ordine?}}

{-- \emph{Stiamo registrando. Si registra tutto, qui. Il suo collo è ben
		protetto. E adesso vada all'inferno.}}

\emph{{-- Se si tratta di un ordine, io\ldots{}}}

{-- \emph{È un ordine. Lo lasci dov'è, e stiamo a vedere come se la cava
		col suo gruppo. Graff, lei mi farà venire l'ulcera.}}

{-- \emph{Non rischierebbe l'ulcera se lasciasse la Scuola a me, e
		andasse a occuparsi della Flotta lei personalmente.}}

{-- \emph{La Flotta ha bisogno di un comandante che sappia portarla in
		battaglia. Non c'è niente di cui occuparsi, finché lei non me ne darà
		uno.}}

{~}

\begin{center}
	{* * *}
\end{center}

{~}

{Entrarono nella Sala di Battaglia in fila per uno e con aria spaesata,
	come bambini condotti in piscina per la prima volta, tenendosi stretti
	ai corrimano lungo il perimetro. La gravità zero li metteva a disagio e
	li disorientava. Presto s'accorsero che le cose erano più facili se
	evitavano del tutto di usare i piedi.}

{Inoltre, dentro le tute si sentivano isolati. Era difficile compiere
	movimenti precisi, perché lo spesso tessuto si piegava male e opponeva
	più resistenza di qualunque altra cosa avessero mai indossato.}

{Ender si aggrappò alla ringhiera e fletté le ginocchia. Aveva già
	notato che malgrado lo spessore la tuta amplificava i movimenti in modo
	strano. Era difficile iniziarli, ma poi le gambe della tuta continuavano
	a muoversi, e con forza, anche dopo che i muscoli s'erano fermati.
	\emph{Fai un gesto con una certa forza, e lei te lo porta avanti con
		forza doppia. Per un po' sarò scoordinato. Meglio che stia attento.}}

{Così, senza mollare il corrimano, si diede un'energica spinta con i
	piedi.}

{All'istante le sue gambe balzarono in alto, ruotò intorno alla
	ringhiera e andò a urtare nel muro col fondo della schiena. Il rimbalzo
	fu ancora più forte, o così gli parve: le mani persero la presa e Ender
	volò via attraverso la sala di battaglia, sbattendo in ogni ostacolo che
	gli si parò davanti.}

{Per qualche terribile momento tentò di capire dove fossero l'alto e il
	basso, o meglio a tentarlo fu il suo corpo, in cerca di una gravità che
	non esisteva. Poi si costrinse a orientarsi su nuovi punti di vista.
	Stava andando a sbattere in una parete. Quello era il \emph{suo} basso.
	E non volava, si disse: cadeva, era a metà di un tuffo. Spettava a lui
	scegliere in che modo urtare su quella superficie.}

\emph{{Sto andando troppo veloce per cercare una presa e fermarmi, ma
		posso ammorbidire l'impatto. Posso calcolare il tempo della rotazione, e
		nell'istante dell'urto usare i piedi per\ldots{}}}

{La cosa non andò come aveva pronosticato. La velocità con cui roteava
	era fuori dalle sue possibilità di manovra, e non ebbe neppure il tempo
	di considerarne le conseguenze. Volò a sbattere in un'altra parete,
	stavolta troppo vicina perché potesse prepararsi all'urto. Ma del tutto
	casualmente scoprì l'esistenza di un principio di dinamica: avvolgendosi
	a palla ruotava più velocemente, distendendosi rallentava la rotazione
	inerziale. Adesso stava di nuovo attraversando l'immenso locale, in
	direzione dei suoi compagni ancora aggrappati al corrimano. Scoperto il
	segreto per ruotare lentamente calcolò che sarebbe riuscito ad
	aggrapparsi da qualche parte. L'angolazione con cui vedeva gli altri
	ragazzi era un po' folle, ma il suo orientamento s'era di nuovo
	riadattato e per quanto lo riguardava essi stavano ora distesi su un
	pavimento, non già in piedi lungo un muro, e lui non era più capovolto
	di quel che lo fossero loro.}

{-- Che vuoi fare, stai cercando di ammazzarti? -- gli chiese Shen.}

{-- Prova anche tu -- disse Ender, atterrandogli accanto. -- La tuta ti
	ripara dagli urti, e se giri su te stesso puoi rallentare a questo modo.
	-- Gli mimò il movimento, rannicchiandosi e distendendosi.}

{Shen scosse il capo. Non aveva la minima voglia di tentare folli balzi
	come quello che aveva appena visto. Ma uno dei ragazzi si spinse
	nell'aria, non con la velocità che il rimbalzo aveva conferito a Ender
	però abbastanza rapido anch'egli. Ender non ebbe bisogno di guardarlo in
	faccia per sapere che si trattava di Bernard. E subito dietro di lui
	partì in volo il suo migliore amico, Alai.}

{Ender li guardò allontanarsi nella vastità del locale. Bernard si
	contorceva per restare in posizione verticale rispetto a quello che
	continuava a vedere come un pavimento; Alai cedeva alla forza che lo
	faceva ruotare e si preparava al rimbalzo sulla parete opposta.
	\emph{Non c'è da meravigliarsi se Bernard si è rotto un braccio sulla
		navetta}, \emph{} pensò Ender. \emph{Quando vola s'irrigidisce come un
		legno. Il panico lo acceca.} Mise da parte quel dato di fatto per usarlo
	eventualmente in futuro.}

{E un'altra cosa valeva la pena di notare: Alai non s'era spinto via
	nella stessa direzione di Bernard. Aveva mirato a uno degli angoli alti
	della sala. I due seguivano percorsi divergenti, allontanandosi sempre
	più l'uno dall'altro, e dopo che Bernard fu andato goffamente a sbattere
	nella parete Alai fu costretto dalla conformazione dell'angolo a ben tre
	rimbalzi, l'ultimo dei quali lo spedì via a un'angolazione sorprendente.
	Il ragazzo mandò un grido d'eccitazione, e lo stesso fecero quelli che
	lo stavano guardando. Alcuni dimenticarono d'essere in gravità zero e
	batterono le mani, lasciando la ringhiera. Questo li fece fluttuare
	lentamente in varie direzioni, agitando le braccia come se tentassero di
	nuotare.}

\emph{{Ecco un altro problema}}{, \emph{} pensò Ender. \emph{Come se la
		cava uno che sta lì a galleggiare? Non c'è modo di spingersi di qua o di
		là.}}

{Fu tentato di fluttuare via pian piano per risolvere il problema
	attraverso prove ed errori. Ma poteva già vedere in atto i più diversi
	tentativi degli altri, tutti fallimentari, e non riuscì a pensare a
	nessun espediente in grado di funzionare meglio.}

{Tenendosi alla ringhiera con una mano si tastò distrattamente la tuta,
	e le sue dita incontrarono la fondina della pistola fissata sotto
	l'ascella sinistra. Questo gli fece tornare a mente i piccoli razzi a
	mano usati dai marines durante gli arrembaggi a una stazione spaziale
	nemica. Estrasse l'arma e la esaminò. Prima di uscire dalla camerata
	aveva già premuto quei pulsanti, e non era successo niente. Ma forse lì,
	nella sala di battaglia, la pistola avrebbe funzionato. Non c'erano
	istruzioni su di essa, né etichette presso i pulsanti. Il grilletto
	aveva un aspetto normalissimo, e il riflesso di premerlo era automatico
	in qualunque bambino avesse usato armi giocattolo. C'erano due pulsanti
	che sembravano fatti apposta per essere raggiunti col pollice, più
	alcuni altri sotto il fondo del calcio che però erano inaccessibili alle
	dita della mano che impugnava l'arma. Ovviamente i due pulsanti del
	pollice erano studiati per un uso rapido.}

{Puntò la pistola verso il pavimento e premette il grilletto.
	All'istante sentì che l'arma gli si scaldava fra le dita; quando lasciò
	il grilletto essa si raffreddò quasi subito. L'altro effetto fu la
	comparsa di un minuscolo disco di luce nel punto in cui aveva mirato.}

{Spinse col pollice il bottone superiore, quello rosso, e tirò ancora il
	grilletto. Stesso risultato.}

{Premette allora il pulsante bianco. Dalla canna nacque una luce assai
	brillante che illuminò una zona abbastanza vasta, ma con intensità
	minore. Stavolta le sue dita sentirono l'arma restare quasi fredda.}

{Il pulsante rosso la trasformava in un laser (ma non in un vero laser,
	come aveva detto Dap) mentre quello bianco ne faceva una lampada. Né uno
	né l'altro effetto potevano servire a qualcosa per la manovra in gravità
	zero.}

{Dunque tutto dipendeva dalla spinta iniziale, l'unico impulso su cui
	uno poteva poi contare. \emph{In altre parole, o diventiamo subito molto
		abili a controllare la forza di spinta e i rimbalzi, o finiremo a
		fluttuare nel mezzo del niente.}}

{Ender si guardò attorno. Soltanto pochi dei suoi compagni erano
	arrivati in vicinanza di una parete, e si sbracciavano in cerca di
	qualche appiglio. Quasi tutti gli altri rimbalzavano l'uno contro
	l'altro, ridendo e agitandosi senza scopo. Alcuni di loro s'erano presi
	per mano e roteavano galleggiando nella penombra. Appena due o tre erano
	rimasti alla ringhiera, come Ender, e osservavano con calma quella
	scena.}

{Notò che uno di essi era Alai. Il suo volo s'era concluso su un'altra
	parete, non troppo distante da lui. D'impulso Ender decise di parlargli,
	e con una spinta si proiettò rapidamente nella sua direzione. Mentre
	fluttuava nell'aria si chiese cos'avrebbe potuto dirgli. Alai era amico
	di Bernard. Finora non si erano mai rivolti la parola.}

{Comunque era tardi per cambiare percorso, così tenne lo sguardo dritto
	avanti a sé e sperimentò lievi movimenti delle braccia e delle gambe per
	mantenere l'assetto di volo. Soltanto all'ultimo momento s'accorse di
	aver mirato con troppa precisione sul bersaglio: non sarebbe atterrato
	\emph{accanto} ad Alai, stava per arrivargli dritto addosso.}

{-- Ehi, afferrami le mani! -- gridò Alai.}

{Ender allungò le braccia verso le sue. Alai riuscì così ad ammortizzare
	l'impatto, quindi lo aiutò a fermarsi senza danni alla parete.}

{-- Ottima mossa -- approvò Ender. -- Dovremmo fare pratica in questo
	genere di cosa.}

{-- Lo penso anch'io, solo che tutti quanti sono là che girano come
	trottole -- disse Alai. -- Mi chiedo cosa succederebbe se ci spingessimo
	avanti insieme. Dovremmo esser capaci di proiettarci l'un contro l'altro
	in direzioni opposte, no?}

{-- Sicuro.}

{-- D'accordo?}

{Era un'ammissione che fra loro le cose non erano state troppo lisce.
	Sei d'accordo che tu e io si possa fare qualcosa insieme? Per tutta
	risposta Ender prese Alai per un polso e si preparò al balzo.}

{-- Pronto? -- chiese Alai. -- Andiamo!}

{A causa della diversa energia con cui s'erano dati la spinta, i due
	cominciarono a ruotare l'uno intorno all'altro. Ender compì alcuni lievi
	movimenti col braccio libero, poi allungò una gamba. La rotazione
	rallentò. Ripeté la manovra ed essa s'interruppe. Ora stavano volando
	avanti in assetto stabile.}

{-- Hai una buona testa, Ender -- disse Alai. Quello era il suo
	complimento migliore. -- Procediamo alla spinta, prima di finire nel
	mucchio degli altri.}

{-- E poi troviamoci insieme in quell'angolo là. -- Ender aveva messo
	una testa di ponte in campo nemico, e non voleva vederla svanire.}

{-- L'ultimo che arriva paga all'altro dieci scorregge in una bottiglia
	del latte -- disse Alai.}

{Con lenta prudenza manovrarono fino a trovarsi faccia a faccia, mani
	unite e ginocchia a contatto.}

{-- Riusciremo a evitare gli altri? -- si preoccupò Alai.}

{-- Per tutto dev'esserci una prima volta -- disse Ender.}

{Distesero le braccia di scatto. La spinta diede loro più velocità di
	quel che s'aspettavano. Ender urtò in un paio di ragazzi, e fu deviato
	in una direzione inattesa. Gli occorse qualche istante per orientarsi
	rispetto all'angolo in cui avrebbe dovuto incontrare Alai, mentre
	l'altro già volava in quella direzione. In fretta calcolò un percorso
	che avrebbe incluso due rimbalzi, per evitare un folto gruppo di
	compagni.}

{Quando Ender giunse al traguardo, Alai s'era agganciato alle tre
	ringhiere dell'angolo e stava fingendo di dormire.}

{-- Hai vinto tu.}

{-- Mi aspetto le tue dieci scorregge migliori -- disse Alai.}

{-- Le ho già messe nel tuo armadietto. Non te ne sei accorto?}

{-- Credevo che fossero le mie calze.}

{-- Qui nessuno di noi porta più calze.}

{-- Ah, già. -- Qualcosa che ricordava loro quanto fossero lontani da
	casa. Parte della soddisfazione provata nel navigare abilmente nell'aria
	si dissolse.}

{-- Cosa succede se spari addosso a qualcuno? -- domandò Alai.}

{-- Non lo so.}

{-- Perché non cerchiamo di scoprirlo?}

{Ender scosse il capo. -- Potremmo ferire qualcuno.}

{-- Volevo dire, perché non ci spariamo l'un l'altro, magari in un
	piede, o qualcosa del genere. Io non sono Bernard, non ho mai torturato
	un gatto per vedere se si torce.}

{-- Ah!}

{-- Non può essere troppo pericoloso, altrimenti non avrebbero dato
	queste pistole a dei ragazzi.}

{-- Adesso ci considerano soldati.}

{-- Sparami in un piede.}

{-- No, spara tu a me.}

{-- Va bene, spariamoci a vicenda.}

{Fu quel che fecero, e all'istante Ender sentì la gamba della tuta farsi
	rigida, immobilizzandosi all'articolazione del ginocchio e della
	caviglia.}

{-- Sei congelato? -- chiese Alai.}

{-- Gamba dura come un legno.}

{-- Congeliamo qualcun altro -- propose Alai. -- La nostra prima azione
	bellica: tu e io contro tutti loro.}

{Sogghignarono, poi Ender disse: -- Meglio invitare anche Bernard.}

{Alai inarcò un sopracciglio. -- Oh?}

{-- E Shen.}

{-- Quello scodinzolante vermiciattolo nero?}

{Ender decise che Alai stava scherzando. -- Ehi, non tutti possiamo
	vantare dei genitori neri.}

{Alai mugolò: -- Mio nonno avrebbe potuto frustarti per una frase come
	questa.}

{-- O forse ci avrebbe bevuto sopra, e il mio anche.}

{-- D'accordo. Recuperiamo Bernard e Shen, e congeliamo questa frotta di
	Scorpioni.}

{Venti minuti più tardi tutti i ragazzi in sala erano congelati, salvo
	Ender, Bernard, Shen e Alai. I quattro si appollaiarono su una ringhiera
	e risero dello spettacolo che si presentava loro, finché nel locale non
	entrò Dap.}

{-- Vedo che avete appreso l'uso del vostro equipaggiamento -- disse.
	Poi azionò un piccolo apparecchio che aveva in mano. I ragazzi che
	fluttuavano qua e là cominciarono a spostarsi lentamente verso la parete
	in cui si aprivano gli ingressi. Dap si mosse fra i ragazzi congelati,
	toccandoli e rendendo di nuovo flessibili le loro tute. Ognuno
	protestava impermalito che Bernard e Alai avevano agito scorrettamente,
	colpendoli quando loro non erano pronti.}

{-- E perché non eravate pronti? -- chiese Dap. -- Vi siete messi le
	tute nello stesso momento. Ma voi avete perso tempo svolazzando attorno
	come polli senza testa. Piantatela di frignare e cominciamo a lavorare
	sul serio.}

{Ender notò che davano per scontato che i capi di quella battaglia
	fossero stati Bernard e Alai. \emph{Meglio così}, \emph{} pensò. Bernard
	sapeva che lui e Alai avevano imparato insieme l'uso delle pistole, e
	che dunque erano amici, perciò poteva dedurne che lui s'era unito al suo
	gruppo. Ma le cose stavano diversamente: Ender s'era aggregato a un
	nuovo gruppo. Quello di Alai. Un gruppo a cui anche Bernard s'era
	unito.}

{La cosa non risultò evidente a tutti; Bernard continuava a fare il
	capoccia e a dare ordini a questo e a quello. Ma Alai adesso aveva mano
	in ogni questione della camerata, e quando Bernard eccedeva era lui che
	interveniva per placarlo. Quando fu loro chiesto di scegliere il nome
	del capogruppo, la scelta fu quasi unanime in favore di Alai. Bernard
	brontolò scontrosamente per qualche giorno, poi si adattò, e i ragazzi
	trovarono una certa unità in quel nuovo schema. Il gruppo non era più
	suddiviso in fazione interna di Bernard, neutrali, e fuoricasta tipo
	Ender. Alai era il ponte fra di loro.}

{~}

\begin{center}
	{* * *}
\end{center}

{~}

{Ender sedeva sulla branda con il banco elettronico girato sulle
	ginocchia. I ragazzi stavano studiando ognuno per conto proprio, e lui
	aveva chiamato sullo schermo del desco una Partita Libera. Era un gioco
	di tipo strano e bislacco, nel quale il computer della Scuola inseriva a
	getto continuo elementi nuovi creando una sorta di labirinto che il
	giocatore doveva esplorare. Era possibile restare alle prese con
	situazioni a piacere, almeno per un poco, ma bastava lasciarle scorrere
	perché qualcos'altro prendesse il loro posto.}

{Talvolta erano cose divertenti, talaltra eccitanti, e lui era costretto
	a stare sempre sul chi vive per non essere \emph{ucciso.} Era già morto
	una gran quantità di volte, ma la cosa era normale, faceva parte del
	gioco: capitava d'essere uccisi ripetutamente, prima di scovare il modo
	di procedere oltre gli ostacoli.}

{La sua figura sullo schermo aveva cominciato in forma di ragazzino. Nei
	tentativi seguenti lo aveva trasformato in un orsacchiotto. Adesso era
	un grosso topo, con mani lunghe e delicate. Ender fece correre la figura
	sotto un gran mucchio di mobili sfasciati. L'aveva fatta competere a
	lungo contro un gatto, ma questo aveva finito per annoiarlo: troppo
	facile eluderlo, ora che conosceva tutti i segreti di quei mobili.}

\emph{{Non attraverso la tana del topo stavolta}}{, \emph{} disse a se
	stesso. \emph{Non ne posso più del Gigante. È una partita insensata, e
		non posso vincere mai. Qualunque sia la mia scelta, è sbagliata.}}

{Ma andò lo stesso fuori attraverso la tana del topo, e oltrepassò il
	ponticello nel giardino fiorito. Evitò i becchi dei paperi e i tuffi
	delle api-kamikaze; aveva provato a gareggiare con loro ma era stato
	tutto troppo facile, inoltre se superava il tempo limite concesso contro
	i paperi si ritrovava trasformato in un pesce, cosa che non gli piaceva.
	Fare il pesce gli ricordava troppo le occasioni in cui finiva congelato,
	nella sala di battaglia, rigido da capo a piedi, senza altro da fare che
	attendere la fine dell'esercitazione perché Dap lo rimettesse in
	movimento. Così, come al solito, scelse di proseguire e si diresse su
	per le colline tondeggianti.}

{Cominciò il tratto paludoso. Dapprima lui era affondato
	interminabilmente, trascinato indietro da rigurgiti di fango sanguigno
	che essudava da sotto ogni roccia. Adesso però s'era fatto svelto a
	correre su per i tratti liberi, evitando il fango e zigzagando verso
	l'alto.}

{Come sempre, quindi, la salita terminò fra i macigni. L'altipiano si
	aprì libero dinnanzi a lui, ma al posto del terreno c'era una distesa di
	pane bianco, tenerissimo, la cui pasta s'innalzava in fragili croste che
	si spezzavano e cadevano. La sua figura ci sprofondava come in una
	spugna e dovette rallentare l'andatura. E quando saltò giù dall'enorme
	pezzo di pane si trovò in piedi su una tavola. Colossali fette di pane
	dietro di lui, colossali cubetti di burro a destra e a sinistra. E di
	fronte il Gigante in persona, che col mento poggiato sulle mano lo
	scrutava. La figura di Ender era poco più alta del suo naso.}

{-- Credo che ti staccherò la testa con un morso -- disse il Gigante,
	come al solito.}

{Questa volta, invece di correre via o di saltare dietro il burro, Ender
	mosse la figura verso la faccia del Gigante e lo colpì al mento con un
	calcio.}

{Il Gigante sporse la lingua, che come il rosso tentacolo d'una piovra
	sbatté al suolo Ender.}

{-- Che ne dici di giocare agli indovinelli? -- chiese il Gigante.
	Dunque quella variante iniziale non faceva alcuna differenza:
	l'avversario insisteva nella sua immancabile proposta. \emph{Stupido
		computer. Milioni di possibili gare nella sua memoria, e il Gigante
		vuole solo giocare a questo stupido gioco.}}

{Come ogni volta, il Gigante piazzò due larghe coppe di vetro alte
	quanto il ginocchio di Ender fra loro, sul piano del tavolo. E come ogni
	volta esse erano colme di liquidi diversi. Il computer era abbastanza
	intelligente da far sì che quei liquidi non fossero mai gli stessi, per
	quante partite potesse giocare. Stavolta uno conteneva una spessa crema
	dall'aspetto semiliquido. L'altro gorgogliava e fumava.}

{-- Uno è velenoso e l'altro no -- disse il Gigante. -- Indovina il
	drink giusto e io ti porterò nella Terra delle Meraviglie.}

{Indovinare significa immergere la faccia in uno dei drink e
	assaggiarlo. Lui non l'aveva azzeccata mai. Talvolta la sua testa si
	dissolveva. Talvolta prendeva fuoco. Talvolta ci cadeva dentro e
	affogava. Talvolta schizzava indietro, diventava verde e andava in
	pezzi. La fine era sempre orribile, e il Gigante rideva sempre.}

{Ender sapeva che qualunque fosse stata la sua scelta sarebbe morto. Il
	gioco era truccato. Dopo la prima morte, la sua figura sarebbe riapparsa
	sul tavolo del Gigante per giocare ancora. Dopo la seconda morte sarebbe
	stata riportata indietro sul pendio fangoso. Poi sul ponticello del
	giardino. Poi nella tana del topo. E poi, se fosse tornato fin dinnanzi
	al Gigante per giocare e perdere ancora, il suo banco si sarebbe spento.
	«Fine della Partita Libera», questa scritta avrebbe lampeggiato sullo
	schermo, e a Ender non sarebbe rimasto che abbandonarsi indietro sulla
	branda, tremante ed esausto, in attesa che il sonno scendesse su di lui.
	Il gioco era truccato, però il Gigante continuava a parlare della Terra
	delle Meraviglie, qualche stupidissima e infantile Fantasyland dove
	probabilmente c'era una stupidissima Mamma Oca, o i Tre Porcellini, o
	Peter Pan, o comunque nulla che valesse la fatica di posarvi gli occhi
	sopra. Eppure lui doveva scoprire il modo di battere il Gigante e
	arrivare là.}

{Si chinò a bere la crema liquida. Immediatamente cominciò a gonfiarsi
	come un pallone. Scoppiò, il Gigante rise. Era morto un'altra volta.}

{Giocò la seconda partita, e stavolta il liquido divenne solido come il
	cemento mentre lo beveva, imprigionandogli la faccia. Il Gigante lo
	spaccò in due lungo la spina dorsale, lo aprì come un pesce e cominciò a
	divorarlo, staccandogli a morsi gambe e braccia.}

{Riapparve sul pendio fangoso e stabilì che non avrebbe proseguito.
	Lasciò perfino che la poltiglia rossa lo ricoprisse, facendolo affogare.
	Ma quando s'accorse che stava sudando, a denti stretti per la
	frustrazione, usò la vita successiva per risalire le colline fin
	sull'altopiano di pane. Poi saltò giù dalla fetta, e in piedi attese che
	il Gigante piazzasse le due grandi coppe di liquido davanti a lui.}

{Esaminò i drink. Quello di destra fumava, l'altro era increspato di
	onde simili a quelle del mare. Cercò di capire che razza di morte
	ciascuno dei due gli avrebbe dato. \emph{Magari da quel mare schizzerà
		fuori un pesce che mi mangerà. E quello che fuma probabilmente mi farà
		soffocare. Odio questo gioco. Non sa di niente. È stupido. È truccato.}}

{E invece di chinarsi a bere rovesciò con un calcio la coppa di
	sinistra, quindi l'altra, saltando qua e là per evitare le mani
	inferocite del Gigante che gridava: -- Imbroglione! Imbroglione! --
	Balzò su quell'enorme faccia, arrampicandosi sulle labbra e sul naso, e
	affondò un pugno nell'occhio destro dell'avversario. La cornea bianca
	schizzò attorno come ricotta fresca, e mentre il Gigante urlava la
	figura di Ender gli si aggrappò alla palpebra, scavando nel molle
	materiale con colpi ampi e violenti.}

{Il Gigante si rovesciò all'indietro e cadde. La visuale dello schermo
	tremò all'immenso urto, e quando il corpo del colosso giacque immobile
	sul terreno tutto attorno sorgevano alberi fitti ed intricati. Un
	pipistrello svolazzò avanti e atterrò sul naso del Gigante. Ender fece
	emergere la sua figura dall'occhio ridotto in poltiglia.}

{-- Come sei riuscito ad arrivare qui? -- chiese il pipistrello. --
	Nessuno viene mai da queste parti.}

{Ender era troppo sorpreso per rispondere. Si chinò, raccolse una
	manciata della sostanza di cui era fatto l'occhio del Gigante e la offrì
	al volatile.}

{Il pipistrello la ingoiò d'un colpo, quindi si alzò in volo. --
	Benvenuto nella Terra delle Meraviglie! -- gridò, mentre si
	allontanava.}

{Ce l'aveva fatta. Ora poteva esplorare. Ora poteva saltar giù dalla
	faccia del Gigante e guardare ciò che aveva finalmente ottenuto.}

{Invece spense lo schermo, spinse il banco nell'armadietto, si tolse la
	tuta da fatica e lentamente s'infilò sotto le coperte. Non aveva avuto
	intenzione di uccidere il Gigante. Quello avrebbe dovuto essere soltanto
	un gioco, non una scelta fra il morire in modo ripugnante e il
	commettere un omicidio ancor meno piacevole. \emph{Sono un assassino,
		perfino quando gioco. Peter sarebbe fiero di me.}}

\phantomsection\label{Orsonux20Scottux20Cardux20-ux20Ilux20Giocoux20Diux20Enderux20-ux20BY_SLY70A1_split_009.htm}{}
