\chapter{DRAGO}

{~}

{~}

{~}

{-- \emph{Adesso?}}

{-- \emph{Suppongo di sì.}}

{-- \emph{Devono esserci degli ordini, colonnello Graff. Un esercito non
		si muove solo perché un comandante dice di supporre che sia il momento
		di attaccare.}}

{-- \emph{Io non sono un comandante. Mi occupo di ragazzini, sono un
		insegnante.}}

{-- \emph{Colonnello, ammetto di esserle stato addosso, ammetto d'esser
		stato la spina nel suo fianco, ma è servito. Tutto ha funzionato come
		lei voleva. Nelle ultime settimane Ender è stato\ldots{} è
		stato\ldots{}}}

{-- \emph{Felice.}}

{-- \emph{Soddisfatto. Sta andando bene. Ha la mente lucida, il suo
		gioco è eccellente. Pur giovane com'è, non abbiamo mai avuto un ragazzo
		meglio preparato per il comando. In genere lo meritano a undici, ma a
		nove e mezzo lui è già all'optimum.}}

{-- \emph{Già, certo. Sa una cosa? Poco fa mi stavo chiedendo che genere
		d'uomo vorrebbe prendere un ragazzino ferito, curarlo alla meglio, e
		rispedirlo sul campo di battaglia. Un piccolo dilemma morale del tutto
		privato. Non ci faccia caso. Devo essere stanco.}}

{-- \emph{Salvare il mondo, ricorda?}}

{-- \emph{Lo chiami dentro.}}

{-- \emph{Stiamo facendo quel che dobbiamo fare, colonnello Graff.}}

{-- \emph{Andiamo, Anderson, lei sta morendo dalla voglia di vedere come
		se la caverà con tutti i nuovi stratagemmi del regolamento su cui le
		chiesi di lavorare. Scommetto che ci si è diabolicamente divertito.}}

{-- \emph{Questa è una bassa insinuazione di cui non la
		credevo\ldots{}}}

{-- \emph{Sicuro, sono un basso individuo. E poiché fra una bassezza e
		l'altra a volte ci incontriamo, non nego d'essere ansioso di vedere come
		se la caverà. Dopotutto, le nostre vite dipendono dal fatto che sia
		veramente abile. Mi sintonizza?}}

{-- \emph{Lei sta cominciando a usare i modi verbali dei ragazzi, eh?}}

{-- \emph{Lo faccia entrare, maggiore. Io registrerò i turni nel suo
		programma di lavoro, e gli fornirò un nuovo sistema di sicurezza. Quel
		che gli stiamo facendo non è tutto un peso per lui; avrà di nuovo la sua
		intimità.}}

{-- \emph{Isolamento, vuol dire.}}

{-- \emph{La solitudine del potere. Coraggio, lo chiami.}}

{-- \emph{Sì, signore. Quando avrò finito con lui, fra una ventina di
		minuti, lo condurrò nel suo ufficio.}}

{~}

\begin{center}
	{* * *}
\end{center}

{~}

{Ender aveva capito cosa c'era in ballo fin dall'istante in cui era
	stato convocato da Anderson. Tutti ormai si aspettavano che avrebbe
	avuto il grado di comandante. Forse non \emph{così} presto, ma da tre
	anni capeggiava la classifica dell'efficienza individuale, con molti
	punti di distacco sul secondo, e quello che faceva gli allenamenti extra
	con lui ogni sera era diventato il più prestigioso gruppo di soldati
	della Scuola. Alcuni si chiedevano perché gli insegnanti non si fossero
	ancora decisi.}

{Si domandò quale orda gli avrebbero dato. Tre comandanti, compresa
	Petra, avrebbero presto finito il corso, ma non poteva certo sperare che
	gli dessero l'orda delle Fenici: nessuno passava mai al comando della
	stessa orda in cui era stato un soldato fra i soldati.}

{Per prima cosa Anderson lo condusse nel suo nuovo alloggio. Questa era
	già una dichiarazione ufficiale: solo i comandanti avevano stanze
	private. Poi gli mostrò pile di uniformi nuove di zecca, accessori vari
	e tute da battaglia. Ender aprì il cellofan per scoprire il nome della
	sua orda.}

\emph{{Draghi}}{, \emph{} diceva l'etichetta su una delle tute. Non
	esisteva nessuna orda dei Draghi.}

{-- Non ho mai sentito parlare dell'orda dei Draghi, signore -- disse.}

{-- Perché da quattro anni è stata sciolta. Usiamo questo nome solo a
	intervalli, dato che c'è una\ldots{} uh, superstizione su di esso. Da
	quando è stata fondata la Scuola di Guerra, l'orda dei Draghi non ha mai
	vinto neppure un terzo delle battaglie. Era diventata oggetto di scherzi
	e di battute.}

{-- Be', perché adesso la rimettete in tabellone?}

{-- Abbiamo pile di uniformi. Dobbiamo pur usarle, no?}

{Seduto dietro la scrivania, Graff sembrava più grassoccio e stanco
	dell'ultima volta che Ender l'aveva visto. Consegnò a Ender il
	radiogancio, l'apparecchietto che i comandanti usavano per spostarsi a
	loro piacimento in sala di battaglia. Durante gli allenamenti serali
	Ender aveva spesso sospirato il possesso di un radiogancio, invece di
	dover rimbalzare sulle pareti prima di poter arrivare dove voleva. E ora
	che aveva imparato a farne a meno abbastanza bene, gliene davano uno.}

{-- Funzionerà soltanto durante le ore di addestramento programmate nel
	tuo orario di lavoro -- lo avvertì Anderson.}

{Visto che Ender contava di proseguire coi suoi allenamenti extra,
	questo significava che il radiogancio gli sarebbe servito solo per metà
	delle ore di lavoro. E la cosa spiegava anche perché pochi comandanti
	facessero pratica fuori orario, ovvero nei momenti in cui il radiogancio
	non era collegato alla sala di battaglia: se avevano l'impressione che
	esso fosse un simbolo di autorità, o di superiorità sui soldati,
	lavoravano meno volentieri allorché dovevano farne a meno. \emph{Perciò
		questo è un vantaggio che avrò su alcuni miei avversari}, \emph{} pensò
	Ender.}

{Il discorsetto con cui Graff gli conferì la nomina suonò trito e
	annoiato. Soltanto verso la fine l'ufficiale parve interessato a ciò che
	stava dicendo. -- Con l'orda dei Draghi abbiamo seguito una procedura
	insolita. Spero che a te non importi. Per metterla insieme si è dovuto
	promuovere anticipatamente una certa quantità di novellini, e ritardare
	nello stesso tempo la promozione di pochi veterani. Credo che sarai
	compiaciuto dei soldati da noi scelti. O meglio, spero che lo sarai,
	perché ti è proibito trasferire chiunque di loro.}

{-- Niente scambi? -- domandò Ender. Quello era sempre stato il metodo
	dei comandanti per eliminare i punti deboli, e favoriva anche i soldati
	stessi.}

{-- Nessuno. Vedi, sono ormai tre anni che porti avanti i tuoi
	addestramenti extra. Hai dei seguaci. Molti bravi soldati metterebbero
	in atto spiacevoli pressioni sui loro comandanti per farsi trasferire da
	te. Noi ti diamo un'orda che potrà, col tempo, diventare competitiva.
	Non abbiamo intenzione di lasciarti riunire il meglio delle altre.
	Questo non servirebbe a nessuno.}

{-- E se avrò dei soldati incapaci di andare d'accordo con me?}

{-- Prova ad andare d'accordo con loro. -- Graff abbassò gli occhi su
	alcuni fascicoli, e Anderson si alzò. Il colloquio era terminato.}

{Ai Draghi era stato assegnato il colore grigio, arancione, grigio.
	Ender andò a mettersi la tuta da battaglia, poi seguì la traccia
	luminosa fino alla camerata in cui erano stati trasferiti i suoi uomini.
	Li trovò già lì, che oziavano intorno all'ingresso, e non perse tempo in
	chiacchiere. -- Ordinatevi nelle cuccette secondo l'anzianità di
	servizio. I veterani in fondo alla camerata, i nuovi verso la porta.}

{Era una sistemazione diametralmente opposta alle usanze, e Ender lo
	sapeva benissimo. Sapeva anche che non intendeva agire come gli altri
	comandanti, i quali non vedevano neppure i novellini sempre un po'
	isolati in fondo al locale.}

{Mentre i ragazzi si comunicavano l'un l'altro le rispettive date di
	arrivo per ordinare i posti, Ender andò su e giù lungo il passaggio
	centrale. Quasi trenta dei suoi soldati erano dei novellini appena tolti
	dal gruppo con cui erano giunti alla Scuola, completamente privi di
	qualsiasi esperienza. Alcuni perfino sotto il limite minimo di età:
	quello più vicino alla porta era un soldo di cacio quasi patetico. Ender
	ricordò a se stesso che così doveva esser apparso anche lui a Bonzo
	Madrid, il giorno del suo arrivo. Tuttavia Bonzo s'era trovato con un
	unico soldato tanto giovane, e aveva avuto la possibilità di
	scambiarlo.}

{Nessuno dei veterani aveva mai fatto parte del gruppo che si allenava
	privatamente con lui. Nessuno era mai stato capobranco. Nessuno, in
	realtà, era più anziano dello stesso Ender, e questo significava che
	perfino i suoi veterani non avevano più di diciotto mesi di esperienza.
	Ricordava appena due o tre dei loro nomi, tanto scarsa era l'impressione
	che avevano destato in lui.}

{Naturalmente loro lo conoscevano bene, dato che era ormai il soldato
	più discusso della Scuola. E alcuni, notò Ender, lo guardavano senza la
	minima simpatia. \emph{Se non altro un favore me l'hanno fatto\ldots{}
		nessuno di questi ragazzi è più anziano di me.}}

{Appena ciascuno ebbe scelto la branda, Ender ordinò che indossassero la
	tuta da battaglia! -- Il nostro orario prevede l'addestramento al
	mattino, e ci metteremo al lavoro subito dopo colazione. Ufficialmente
	dovreste godere di un'ora di libertà, appena usciti dalla mensa. Ma di
	questa parleremo in seguito, quando avrò visto a che punto siete. -- Tre
	minuti dopo, benché molti di loro non fossero ancora del tutto pronti,
	ordinò loro di uscire in fila per uno.}

{-- Ma io sono nudo! -- si lamentò un ragazzino.}

{-- La prossima volta sarai più svelto. Tre minuti dal mio ordine al
	momento di uscire dalla porta, questa è la regola della settimana in
	corso. La settimana prossima la regola sarà di due minuti. Avanti,
	march! -- C'era il rischio che ben presto nel resto della Scuola
	circolasse la battuta che i Draghi erano dei tali pivelli da aver
	bisogno di esercizi perfino per imparare a vestirsi.}

{Cinque ragazzini erano completamente nudi, e sfilavano a passo di
	marcia nei corridoi tenendo la tuta in mano. Quelli del tutto vestiti
	erano una minoranza, e nel passare davanti alle porte spalancate delle
	aule l'orda attirò prevedibilmente l'irrispettosa attenzione delle
	scolaresche. Pochi avrebbero osato sfidare quella pioggia di commenti
	due giorni di fila.}

{Più tardi, nei corridoi che portavano alla sala di battaglia, Ender li
	fece correre rapidamente avanti e indietro, in modo che sudassero un
	po', mentre quelli nudi si vestivano. Poi li condusse alla porta
	superiore, quella che si apriva la centro della parete come nella sala
	dove si svolgevano le battaglie fra le orde. Ordinò a ciascuno di
	saltare in alto, aggrapparsi al corrimano superiore e usarlo per darsi
	la spinta in avanti. -- Riunitevi alla parete opposta -- disse. -- Come
	se andaste a conquistare la porta del nemico.}

{Già al momento di balzare, quattro alla volta, fuori dal corridoio i
	ragazzi gli mostrarono a che punto fossero. Quasi nessuno sapeva come
	procedere in linea retta verso l'obiettivo, e una volta arrivati alla
	parete opposta erano pochi quelli che riuscivano ad ancorarsi o a
	controllare il loro rimbalzo.}

{L'ultimo della fila era il più piccolo dell'orda, e per lui la
	ringhiera superiore era così lontana da richiedere un balzo di
	precisione.}

{-- Puoi usare il corrimano laterale, se vuoi -- disse Ender.}

{-- Un accidente! -- ringhiò il ragazzino. Saltò in alto, toccò la
	ringhiera appena con un dito e sbatté malamente nello stipite della
	porta, roteando via senza più controllo. Ender non seppe se ammirare
	quel piccoletto per aver rifiutato una facilitazione o irritarsi per la
	sua attitudine alla disubbidienza.}

{Quando finalmente riuscirono ad allinearsi lungo la parete, Ender notò
	che senza eccezione s'erano orientati con la testa volta dalla parte che
	nel corridoio era stata l'alto. Poggiò allora le mani su quello che i
	ragazzi consideravano il pavimento e si capovolse. -- Perché state tutti
	a testa in giù, soldati? -- domandò.}

{Alcuni di loro cominciarono a girarsi con ubbidienza.}

{-- Attenzione, voialtri! -- li fermò lui. -- Ho chiesto perché state a
	testa in giù.}

{Nessuno rispose. Non avevano capito il senso della sua domanda.}

{-- Ho chiesto \emph{il motivo} per cui ognuno di voi ha i piedi in aria
	e la testa verso il basso.}

{Dopo qualche istante uno si decise a rispondere: -- Signore, questa è
	la direzione di\ldots{} in cui siamo usciti dalla porta, cioè.}

{-- E questo ha forse qualche significato? Che differenza fa
	l'orientamento gravitazionale del corridoio? Pensate per caso di
	battervi nel corridoio? Qui dove stiamo c'è forza di gravità?}

{-- No, signore -- risposero alcuni, perplessi.}

{-- Da ora in poi dimenticherete l'esistenza della parola stessa ancor
	prima di saltar fuori da quella porta. La gravità scompare, non ha più
	senso. Mi capite? E in qualunque modo siate girati quando entrerete in
	sala, ricordate questo: la porta nemica è \emph{in basso.} I vostri
	piedi staranno puntati da quella parte. L'alto sarà invece verso la
	vostra porta. Il nord davanti, il sud di dietro, l'est a destra,
	l'ovest\ldots{} da che parte?}

{Le loro mani si alzarono a indicare.}

{-- Bene, vedo che sapete ragionare per eliminazione. Ma non vi
	consiglio di orientarvi col processo di eliminazione quando dovete
	andare al gabinetto d'urgenza. Cos'era quella specie di circo equestre
	che ho visto poco fa? Qualcuno aveva forse l'impressione di
	\emph{volare} davvero? Ora tutti quanti: lanciarsi e radunarsi in doppia
	fila sul soffitto. Scattare! Muoversi!}

{Come Ender s'era aspettato, un buon numero di loro si lanciò d'istinto
	non verso la parete della porta d'ingresso, bensì verso quella che lui
	aveva definito «nord», ovvero la direzione che aveva rappresentato
	l'alto quand'erano ancora nel corridoio. Naturalmente capirono quasi
	subito l'errore, ma era troppo tardi, e per porvi rimedio dovettero
	aspettare di poter rimbalzare sulla parete nord.}

{Nel frattempo Ender li stava suddividendo dentro di sé in due gruppi,
	in base alla loro rapidità nell'apprendere. Il ragazzino più piccolo,
	che aveva fatto la peggiore uscita dalla porta, fu il primo ad arrivare
	alla parete giusta e restò lì posizionandosi correttamente con la testa
	in \emph{alto.} Non lo avevano dunque promosso per caso, e avrebbe fatto
	una buona riuscita. Era però un galletto e un ribelle, anche se forse
	non aveva mandato giù il fatto d'esser stato costretto a marciare nudo
	nei corridoi.}

{-- Tu -- disse Ender, indicando il piccoletto. -- Da che parte è il
	basso?}

{-- Verso la porta nemica. -- La risposta era stata rapida. Ma anche un
	po' seccata, come a dire: OK, OK, adesso passiamo alle cose importanti.}

{-- Il tuo nome, ragazzo.}

{-- Questo soldato si chiama Bean }{{[}fagiolo N.d.T.{]}}{, signore.}

{-- Riferito alle dimensioni del corpo o a quelle del cervello? -- Gli
	altri ragazzi fecero udire qualche risatina, ma lui li azzittì subito.
	-- Non farci caso, Bean. Ho visto che sei svelto. Ora aprite bene gli
	orecchi, perché non mi ripeterò spesso. Nessuno esce da quella porta
	senza rischiare d'essere all'istante colpito e congelato. Ai vecchi
	tempi avreste avuto dieci, venti secondi prima di cominciare le
	ostilità. Adesso, se non schizzate fuori già pronti a colpire e a
	ripararvi, siete congelati. E cosa succede quando uno è congelato?}

{-- Non può muoversi -- rispose uno dei ragazzi.}

{-- Questo è ciò che la parola \emph{significa} -- disse Ender. -- Ma al
	soldato cosa \emph{succede}?}

{Fu Bean, per nulla intimidito dalla sua spiritosaggine di poco prima,
	che rispose correttamente: -- Continua ad andare dritto in quella
	direzione. Alla velocità con cui è partito.}

{-- Proprio così. Voi cinque, là in fondo alla fila, muovetevi!}

{Stupiti i ragazzi si guardarono l'un l'altro. Ender puntò la pistola e
	li colpì tutti. -- I cinque successivi, muoversi!}

{Si mossero. Ender sparò anche a ciascuno di loro, ma continuarono a
	volare allontanandosi verso le pareti. I primi cinque, invece, erano
	rimasti a fluttuare dove il raggio di luce li aveva raggiunti.}

{-- Guardate questi cosiddetti soldati -- disse Ender. -- Il loro
	comandante ha ordinato loro di muoversi e non l'hanno fatto. Primo
	errore. Adesso sono congelati ma, peggio ancora, sono congelati qui dove
	non possono servire a niente; mentre gli altri, visto che almeno si sono
	mossi, stanno andando a dar fastidio al nemico, ostacolandogli i
	movimenti e la visuale. Voglio sperare che almeno cinque di voi abbiano
	capito il punto. E non dubito che Bean sia uno di loro. Non è così,
	Bean?}

{Il ragazzo non gli rispose subito, ma Ender lo fissò finché si decise a
	dire: -- È così, signore.}

{-- Allora, qual è il punto?}

{-- Quando lei ordina di muoversi, il soldato si deve muovere in fretta.
	Così, se lo colpiscono, va a rimbalzare fra le posizioni nemiche invece
	di stare fra i piedi ai compagni.}

{-- Eccellente! Vedo che in quest'orda c'è almeno un soldato capace di
	usare l'immaginazione. -- Ender poté vedere il risentimento crescere
	nelle occhiate che gli altri si scambiavano, evitando di guardare Bean.
	\emph{Perché sto facendo questo? Cos'ha a che fare coi doveri di un buon
		comandante il trasformare un ragazzino in un bersaglio per gli altri?
		Dovrei farlo a lui soltanto perché l'hanno fatto a me?} Per un attimo fu
	tentato di far marcia indietro, di dire ai ragazzi che il piccoletto
	aveva bisogno del loro aiuto e della loro amicizia più di chiunque
	altro. Ma naturalmente non poteva farlo. Non il primo giorno. Quel
	giorno, perfino i suoi errori sarebbero stati visti come parte di un
	qualche brillante progetto di istruzione.}

{Col radiogancio Ender si trasse vicino alla parete; prese un ragazzo e
	lo fece scostare dagli altri. -- Stai rigido sull'attenti -- ordinò. Lo
	fece ruotare nell'aria finché i piedi di lui puntarono verso i compagni.
	Quando il ragazzo accennò a muoversi, Ender lo congelò. Gli altri
	risero. -- Quali parti del suo corpo potresti colpire? -- Domandò al
	soldato direttamente davanti ai piedi di quello congelato.}

{-- Tutt'al più le suole delle scarpe.}

{Ender si volse al ragazzo accanto. -- E tu?}

{-- Io posso vedere il suo corpo.}

{-- E tu, laggiù?}

{Un ragazzo a qualche distanza da lui rispose: -- Tutto il corpo.}

{-- I piedi non sono grandi. Non riparano molto, eh? -- Ender scostò da
	sé il soldato congelato. Poi ripiegò le gambe, come se fosse
	inginocchiato a mezz'aria, e sparò a ognuna di esse. All'istante i
	pantaloni della tuta s'irrigidirono, tenendogliele ferme in quella
	posizione.}

{Si spinse in alto, presentando loro le ginocchia unite. -- Adesso cosa
	vedete?}

{Molto di meno, fu la risposta.}

{Ender si piazzò la pistola fra i polpacci. -- Ma io vi vedo benissimo
	-- annunciò, e cominciò a sparare a quanti si trovava davanti. --
	Fermatemi! Colpitemi, se ci riuscite! -- gridò.}

{Alla fine lo congelarono, ma non prima che lui avesse colpito un terzo
	almeno di loro. Il suo pollice sinistro annullò l'effetto sfiorando un
	pulsante del radiogancio, poi usò l'apparecchio per scongelare gli altri
	soldati. -- Ora -- disse, -- dov'è la porta nemica?}

{-- Giù!}

{-- E qual è la vostra posizione di attacco?}

{Qualcuno fece per rispondergli a parole, ma Bean reagì spingendosi via
	dalla parete con le gambe ripiegate sotto di sé, dritto verso il lato
	opposto della sala e sparando con l'arma fra le ginocchia per tutta la
	strada.}

{Per un attimo Ender fu tentato di gridargli un rimprovero, di punirlo,
	poi scacciò quell'impulso abbastanza meschino. \emph{Perché dovrei
		essere così ingiusto con un bambino?} -- Bean è il solo che ha capito
	quello che dico? -- sbottò.}

{Immediatamente l'intera orda balzò in direzione della parete di fondo,
	tutti inginocchiati nell'aria, sparando all'impazzata fra le gambe e
	gridando con feroce entusiasmo. \emph{Potrà venire il giorno}, \emph{}
	pensò Ender, \emph{che mi sarà utile proprio una tattica di questo
		genere: quaranta ragazzi che urlano a squarciagola nel più disordinato
		degli assalti.}}

{Quando li vide fermi sull'altro lato gridò loro di attaccarlo, tutti
	insieme. \emph{Sì}, \emph{} rifletté, \emph{non c'è male. Mi hanno dato
		un'orda non addestrata, senza veterani di valore, ma almeno non è una
		torma di sciocchi. Potrò lavorare con loro.}}

{Appena li ebbe rimessi in fila, ancora ridacchianti ed esilarati,
	cominciò a darsi da fare con impegno. Ordinò a tutti di congelarsi le
	gambe nella posizione che ormai conoscevano. -- Ora sentiamo, a cosa vi
	servono le gambe in battaglia?}

{A niente, dissero alcuni.}

{-- Bean non la pensa così, no? -- suggerì Ender.}

{-- Servono a rimbalzare meglio via dalle pareti. A spingersi.}

{-- Giusto -- disse Ender.}

{Gli altri ragazzi protestarono che spingersi via era movimento, non
	combattimento.}

{-- Non c'è combattimento senza movimento -- li corresse Ender. Loro
	tacquero, e detestarono Bean un po' di più. -- Adesso, con le gambe
	congelate in questo modo, sapreste spingervi via dalla parete?}

{Nessuno osò rispondere, per paura di sbagliare.}

{-- Bean? -- chiese Ender.}

{-- Non ci ho mai provato, ma forse mettendosi fronte alla parete e
	piegandosi all'altezza della cintura\ldots{}}

{-- Giusto ma anche sbagliato. Guardate me. Ho la schiena al muro, le
	gambe congelate. Poiché sono in ginocchio ho i piedi contro la parete.
	Di solito, quando vi spingerete via dovrete spingervi in basso,
	lasciando il corpo dietro di voi, ovvero piegandovi all'indietro. Non in
	avanti, come ha detto Bean, altrimenti vi schiaccereste il
	\emph{fagiolo.} OK?}

{Tutti guardarono Bean e risero.}

{-- Dunque la tecnica è questa: arrivare contro la parete a gambe
	ripiegate, ammortizzare l'urto con esse e rotolare con la schiena a
	contatto dell'ostacolo. Poi spingersi via usando le spalle. Guardate
	me.}

{Ender si staccò dalla parete con quel metodo, quindi assunse la
	posizione di attacco e a gambe avanti volò fino al lato opposto della
	sala. Atterrò sulle ginocchia, rotolò sulla schiena e con un colpo di
	reni balzò via in un'altra direzione, girando su se stesso come una
	trottola. -- Sparatemi! -- gridò. Il suo percorso era quasi parallelo
	alla fila dei ragazzi, che gli indirizzarono addosso gragnuole di colpi,
	ma poiché stava roteando nessuno poté tenere il raggio sul bersaglio per
	il minimo tempo necessario.}

{Lui riammorbidì la tuta e col radiogancio si portò di nuovo fra loro.
	-- Adesso lavorerete una mezz'ora su questa tecnica. Metterà in funzione
	alcuni muscoli che non sapevate di avere. Imparate a usare costantemente
	le gambe come uno scudo, ed a controllare il rimbalzo per poter roteare.
	Contro i colpi a distanza ravvicinata girare su se stessi non serve a
	niente, ma quelli che vi arrivano addosso da lontano risulteranno
	innocui: a quella distanza il raggio deve star fermo sullo stesso punto
	per alcuni decimi di secondo, e se un corpo sta roteando questo non
	succede. Adesso ciascuno si congeli le gambe, e scattare via.}

{-- Non ci assegni un percorso? -- volle sapere un ragazzo.}

{-- Nossignore. Voglio che sbattiate l'uno contro l'altro e impariate a
	cavarvela negli urti imprevisti. Salvo che quando manovreremo in
	formazione, perché allora dovrete sbattere su un compagno o spingervi
	via da lui per scopi ben precisi. E ora scattare, ho detto!}

{Quando diceva \emph{scattare}, \emph{} se non altro, l'orda scattava.}

{Ender fu l'ultimo a uscire al termine dell'orario, perché s'era
	attardato in fondo alla sala per aiutare un paio dei più lenti a capire
	certi movimenti. Per i veterani era stato un gioco, ma tutti i novellini
	avevano annaspato come pulcini nella stoppa quando s'era trattato di
	fare due o tre cose nello stesso tempo. Roteare con le gambe congelate
	era facile per chi non soffriva di vertigini, nessuno aveva difficoltà a
	stabilizzarsi in assetto di volo; ma lanciarsi in una direzione e
	sparare in un'altra, girare su se stessi, rimbalzare in una parete e
	uscirne sparando a un bersaglio, volando nella direzione voluta\ldots{}
	questo era molto oltre le loro possibilità del momento. Esercizio
	fisico, rimbalzi e volo, questo era tutto ciò su cui Ender poteva farli
	sudare per i primi tempi. La strategia e le tattiche erano eccitanti, ma
	non se ne poteva neppure parlare finché l'orda non avesse imparato a
	muoversi in gravità zero.}

{A lui sarebbe servita un'orda pronta fin da quel momento. Come
	comandante era un novizio, inoltre gli insegnanti avevano cambiato non
	poche regole, non lo lasciavano fare scambi e gli avevano dato dei
	veterani che nessuno considerava delle cime. E nulla garantiva che gli
	avrebbero dato i soliti tre mesi di tempo per preparare l'orda, prima di
	metterla in cartellone per le battaglie con le altre.}

{Almeno, si disse, alla sera avrebbe avuto Alai e Shen per dargli una
	mano ad allenare i suoi nuovi ragazzi.}

{Era appena uscito dalla sala di battaglia quando, in corridoio, si
	trovò di fronte al piccolo Bean.}

\foreignlanguage{ngerman}{-- Ehilà, Bean.}

\foreignlanguage{ngerman}{-- Ehilà, Ender.}

\foreignlanguage{ngerman}{Una pausa.}

{-- \emph{Signore} -- lo corresse lui dolcemente.}

{-- Io lo so quello che stai facendo, Ender, signore, e voglio darti un
	avvertimento}

{-- Un avvertimento a me?}

{-- Io posso essere il miglior soldato che tu abbia, ma non fare
	giochetti con me.}

{-- Altrimenti?}

{-- Altrimenti sarò il peggiore. O l'uno o l'altro.}

{-- E cos'è che vuoi, complimenti e bacetti? -- Ender stava cominciando
	a irritarsi, adesso.}

{Bean lo fermò prendendolo per un gomito. -- Voglio un branco.}

{Lui si volse di scatto e lo fissò negli occhi. -- E cosa ti fa supporre
	che potresti mai averne uno in vita tua?}

{-- Perché io so cosa deve fare un branco.}

{-- Sapere cosa deve fare è una cosa -- disse Ender, -- farglielo fare è
	un'altra. Perché dei soldati dovrebbero seguire un poppante come te?}

{-- Mi hanno detto che un tempo chiamavano te a questo modo. E ho
	sentito che Bonzo Madrid lo fa anche adesso.}

{-- Ti ho fatto una domanda, soldato.}

{-- Io mi guadagnerò il loro rispetto, se non mi fermerai.}

{Ender sogghignò. -- Anzi, io ti sto aiutando.}

{-- All'inferno! -- disse Bean.}

{-- Nessuno ti avrebbe notato, se non per compatire il povero bambinetto
	magrolino. E oggi ho fatto in modo che tutti ti notassero. D'ora in poi
	ti terranno sotto il loro microscopio. Tutto ciò che ti resta da fare
	per essere rispettato, adesso, è di essere perfetto.}

{-- Così non avrò neppure una possibilità di imparare, prima d'essere
	giudicato.}

{-- Povero piccino! Nessuno vuol essere buono con lui! -- Ender lo prese
	per le spalle e lo tenne fermo contro il muro. -- Te lo dirò io come
	puoi avere un branco. Provami che sai diventare un ottimo soldato.
	Provami che sai come tenere in pugno altri soldati. E poi provami che
	qualcuno vorrebbe affidarsi ai tuoi ordini in battaglia. Allora ti darò
	il tuo branco. Ma potresti sputar sangue per riuscirci, bada.}

{Bean sorrise. -- Questo mi sta bene. Se tu lavori nel modo che hai
	detto, sarò capobranco entro un mese.}

{Ender lo afferrò per il petto e lo spinse contro il muro. -- Quando io
	dico che lavoro in un modo, Bean, allora quello è il modo in cui lavoro.
	Chiaro?}

{Bean si limitò a sorridere. Ender lo lasciò e si allontanò a lunghi
	passi. Quando fu nel suo alloggio si gettò disteso sul letto e strinse i
	denti, scosso da un tremito. \emph{Cosa sto facendo? Il mio primo
		addestramento con l'orda, e sto già soggiogando i ragazzi come faceva
		Bonzo. E Peter. Li sbatto di qua e di là. Prendo di mira un povero
		bambino per dare a tutti gli altri qualcosa da odiare. Le cose che più
		disprezzavo in un comandante; e io le sto facendo.}}

\emph{{È una legge della natura umana che uno debba diventare uguale al
		primo uomo che ha avuto autorità su di lui? Posso lasciar perdere tutto
		fin d'ora, se è così.}}

{Nella sua mente ripassarono più volte le cose che aveva detto e fatto
	in quella prima mattinata con la nuova orda. Perché non aveva parlato e
	agito come sempre faceva con i ragazzi del gruppo di allenamento serale?
	Nessuna autorità se non la capacità di eccellere. Nessuno aveva bisogno
	di dare ordini, soltanto suggerimenti. Ma questo non avrebbe funzionato,
	non con un'orda. Gli amici che si allenavano con lui non dovevano
	imparare a lavorare insieme. Non dovevano sviluppare l'istinto di
	gruppo, non dovevano imparare a vivere situazioni che in battaglia li
	avrebbero portati a sostenersi a vicenda, a confidare l'uno nell'altro.
	Non c'era bisogno che loro scattassero ai suoi comandi.}

{Avrebbe anche potuto andare all'estremo opposto, se avesse voluto:
	esibire lassismo e incompetenza come Rose de Nose. Fare errori stupidi e
	affidarsi a capibranco capaci di porvi rimedio\ldots{} ma no. No, lui
	voleva le capacità formative della disciplina, e questo significava
	pretendere -- e riuscire a ottenere -- ubbidienza rapida e
	incondizionata. Lui voleva un'orda ben addestrata, e questo voleva dire
	far allenare i soldati duramente, finché avessero padroneggiato una
	tecnica al punto di averla a noia, finché gli fosse penetrata nelle
	cellule del corpo tanto da divenire un riflesso condizionato.}

{Ma cos'era che lo aveva spinto ad agire così con Bean? Perché aveva
	messo gli occhi proprio sul più piccolo, più debole, e forse anche il
	più brillante di quei ragazzi? Perché aveva fatto a Bean ciò che era
	stato fatto a lui da comandanti che disprezzava?}

{Poi ricordò che la cosa non era cominciata con i suoi comandanti. Prima
	che Rose e Bonzo lo trattassero in modo sprezzante, era stato Bernard a
	creare quella situazione. Era stato Graff.}

{Sì, l'insegnante aveva fatto questo. E non certo per sbaglio. Ender ora
	lo capiva chiaramente. Era stata una strategia. Graff lo aveva
	deliberatamente isolato dagli altri ragazzi, rendendogli impossibile
	legare con loro. E adesso cominciava a sospettarne i motivi. Non era
	stato per unire il resto del gruppo, anzi la cosa li aveva divisi. Graff
	lo aveva isolato per vedere come reggeva sotto il torchio, per spingerlo
	a dimostrare non che era soltanto capace, ma che era migliore di tutti
	gli altri. Perché non gli era restato altro modo di ottenere rispetto e
	amicizia. E lo aveva reso un soldato migliore di quel che altrimenti lui
	sarebbe diventato. Aveva anche fatto di lui un ragazzo solo, spaventato,
	irritato, sfiduciato. E forse perfino queste caratteristiche s'erano
	sommate per renderlo un soldato migliore.}

\emph{{Questo è ciò che sto facendo a te, Bean. Ti ferirò perché tu
		diventi capace di sopportare le ferite. Ti costringerò a stare all'erta
		contro di me per svegliare il tuo ingegno. Ti insegnerò ad abituarti
		alla tensione. Ti terrò sempre sbilanciato, mai sicuro di quel che sta
		per succederti, in modo che tu sia pronto a ogni cosa, pronto a
		improvvisare, e deciso a vincere ad ogni costo. E ti farò anche sentire
		un misero reietto. Ecco il motivo per cui ti hanno messo con me, Bean:
		perché tu possa essere come me. Perché tu cresca camminando sulle mie
		stesse orme.}}

\emph{{Ed io\ldots{} si suppone che io debba crescere come Graff?
		Grassoccio e triste e indifferente, manipolando le vite di ragazzini per
		farli uscire perfetti da questa fabbrica, ufficiali e generali capaci di
		condurre le astronavi a difesa della patria. Tu devi aver gustato il
		piacevole senso di potere del burattinaio, nel costruirli. Finché non ti
		sei trovato ad avere un soldato migliore di qualsiasi altro. Ma non puoi
		avere anche questo. Distruggerebbe la simmetria della tua opera. Devi
		rimetterlo in riga allora; o schiacciarlo, isolarlo e colpirlo finché
		lui non si rimetterà in fila con tutti gli altri.}}

\emph{{Be', quel che oggi ti ho fatto, Bean, l'ho fatto. Ma ti terrò
		d'occhio con più comprensione di quel che credi, e quando verrà il
		momento giusto scoprirai che sono stato tuo amico, e che tu sei il
		soldato che volevi essere.}}

{Ender non andò in classe quel pomeriggio. Rimase disteso sul letto e
	mise per iscritto le sue impressioni su ognuno dei ragazzi dell'orda, le
	loro caratteristiche psicofisiche e i dettagli su cui questo o quello
	avrebbe dovuto lavorare di più. Agli allenamenti di quella sera avrebbe
	parlato con Alai, e insieme avrebbero studiato il modo di insegnare a un
	gruppo eterogeneo fino a portare i singoli allo stesso livello. Almeno
	in questa difficoltà non avrebbe dovuto agire da solo.}

{Ma quando quella sera arrivò in sala da battaglia, mentre quasi tutti
	erano ancora a mensa, trovò sulla porta il maggiore Anderson che lo
	aspettava. -- Ci sono state alcune modifiche al regolamento, Ender. Da
	ora in poi soltanto membri della stessa orda potranno lavorare insieme
	nelle ore libere, e di conseguenza le sale di battaglia dovranno essere
	frequentate secondo orari programmati. Da oggi il tuo turno è ogni
	quattro giorni.}

{-- Nessun altro sta facendo allenamenti extra.}

{-- Li hanno in progetto, Ender. Ora che tu comandi un'altra orda, i
	tuoi colleghi non vogliono che i loro ragazzi ti frequentino. E mi
	sembra comprensibile, no? Così ognuno condurrà i suoi programmi di
	allenamento.}

{-- Finora ho pur sempre fatto parte di orde loro avversarie. E mi hanno
	ugualmente mandato soldati da addestrare.}

{-- Ma non eri un comandante.}

{-- Voi mi avete dato un'orda completamente grezza, maggiore Anderson,
	signore\ldots{}}

{-- Hai un certo numero di veterani.}

{-- Non sono certo eccezionali.}

{-- Nessuno viene qui alla Scuola se non ha grosse doti, Ender. Impara a
	renderli migliori.}

{-- Ho bisogno di Alai e Shen per\ldots{}}

{-- È tempo che tu cresca e faccia le tue cose da solo, Ender. Non hai
	bisogno che questi altri ragazzi ti tengano la manina. Adesso sei un
	comandante. Perciò fammi il favore di agire di conseguenza.}

{Ender oltrepassò Anderson e proseguì verso la sala di battaglia. Poi si
	fermò. -- Dato che anche gli allenamenti serali sono ora regolarmente
	programmati, potrò usare il radiogancio come in quelli normali?}

{Era un sorriso quello di Anderson? No. Neppure una minima probabilità
	che lo fosse. -- Vedremo -- fu la risposta.}

{Ender si volse e andò in sala di battaglia. Da lì a poco arrivò la sua
	orda; ma nessun altro si fece vedere, sia perché Anderson fosse rimasto
	fuori a intercettare chi stava arrivando, sia che già nella Scuola si
	fosse sparsa la voce che le serate informali sotto la direzione di Ender
	erano un capitolo chiuso.}

{Fu un allenamento fruttuoso e i ragazzi fecero qualche passo avanti, ma
	al termine Ender era sfinito e si sentiva solo. C'erano ancora trenta
	minuti prima dell'ora di andare a letto. Non voleva accompagnare l'orda
	in camerata; aveva imparato da tempo che i migliori comandanti se ne
	stavano lontani, a meno che non avessero una buona ragione per
	addentrarsi fra le brande. I ragazzi dovevano avere la possibilità di
	starsene in pace, di rilassarsi, senza nessuno che fosse lì a farsi
	un'opinione di loro dal modo in cui parlavano o agivano fuori orario.}

{Così andò a bighellonare in sala giochi, dove qualche altro ragazzo
	stava sfruttando l'ultima mezz'ora prima della campanella per fare una
	scommessa o battere un punteggio fatto in precedenza. Nessuna delle
	macchine lo attirava, ma fece ugualmente una partita su una di quelle
	disegnate più che altro per i principianti. Annoiato, ignorò gli
	obiettivi del gioco e fece uso della figura mobile, un orso, per
	esplorare lo scenario animato che il programma conteneva.}

{-- Non vincerai mai a quel modo.}

{Ender sorrise. -- Ho sentito la tua mancanza stasera, Alai.}

{-- Io \emph{ero} in sala. Ma loro hanno fatto entrare la tua orda in
	qualche altro posto separato. Sembra che adesso tu sia diventato uno dei
	grandi, e che non potrai più giocare con noialtri piccoletti.}

{-- Tu sei almeno un cubito più alto di me.}

{-- Un cubito! Forse Dio ti ha ordinato di costruire una barca, o ti ha
	dato le misure per un tempio? O sei improvvisamente d'umore arcaico?}

{-- Non arcaico, forse arcano. Segreto, tortuoso e incomprensibile.
	Sento già la tua mancanza, volpone circonciso.}

{-- Non te l'hanno detto? Ora siamo nemici acerrimi. La prossima volta
	che ci incontreremo in battaglia dovrò darti una brutta strigliata.}

{Erano le solite battute, ma adesso c'era troppa verità dietro di esse.
	Sentendo Alai parlarne come se tutto fosse uno scherzo Ender si rese
	dolorosamente conto che quella nuova regola lo allontanava da un amico,
	e il suo malumore aumentò quando si chiese se Alai provava
	\emph{davvero} la tristezza che aveva cercato di comunicargli con quella
	frase.}

{-- Puoi sempre provarci -- disse Ender. -- Ti ho insegnato tutto quello
	che sai. Ma non ti ho insegnato tutto ciò che \emph{io} so.}

{-- Sapevo perfettamente che ti stavi tenendo da parte qualche
	trucchetto, Ender.}

{Una pausa. L'orso di Ender era nei guai, sullo schermo. Si arrampicò su
	un albero. -- No, Alai. Non mi tenevo da parte niente con te.}

{-- Lo so -- disse l'altro. -- Neppure io.}

{-- Salaam, Alai.}

{-- Non credo che ci sarà.}

{-- Che non ci sarà cosa?}

{-- La pace. È questo che \emph{salaam} significa. La pace sia con te.}

{Quelle parole risvegliarono un'eco nella memoria di Ender. La voce di
	sua madre che gli leggeva una storia, da bambino. \emph{Non illuderti
		che io sia venuto a portare la pace sulla Terra. Io non vengo a portare
		la pace, ma una spada.} E con la fantasia aveva visto Peter incedere sui
	cadaveri dei suoi nemici con uno spadone rosso di sangue fra le mani.
	Quelle parole e quell'immagine erano rimaste a lungo nella sua mente.}

{Senza un lamento l'orso morì. Fu una morte divertente, accompagnata da
	una musichetta allegra. Ender si volse e vide che Alai era già andato
	via. Ebbe l'impressione di aver perso una parte di se stesso, un
	sostegno interno che gli dava coraggio e sicurezza. Con Alai, assai più
	che con Shen, era giunto a provare un'affinità così forte che il
	\emph{noi} gli saliva alle labbra molto più facilmente della parola
	\emph{io.}}

{Ma Alai gli aveva lasciato qualcosa. Disteso a letto con gli occhi
	fissi nel buio Ender ci ripensò, e sentì ancora il bacio che Alai gli
	aveva dato sulla guancia mormorando la parola \emph{pace.} Quel momento,
	quel bacio e quella pace erano sempre lì con lui. \emph{Io sono i miei
		ricordi, e i miei ricordi sono me. Alai è già un ricordo così legato a
		me che nessuno potrà mai togliermelo. Come Valentine, il ricordo più
		forte di ogni altro.}}

{Il giorno dopo incrociò Alai in un corridoio, e si salutarono, si
	presero per mano, parlarono un poco; ma entrambi sapevano che adesso
	c'era un muro. Avrebbe potuto essere abbattuto, quel muro, in qualcuno
	degli anni a venire, ma per ora la sola vera comunicazione rimasta fra
	loro erano le radici già allargatesi profonde nel terreno, sotto il
	muro, dove chi l'aveva costruito non poteva tranciarle.}

{La cosa più raggelante, però, era la paura che quel muro fosse di un
	materiale indistruttibile, che Alai fosse lieto d'esser stato separato
	da lui e pronto per trasformarsi in un suo nemico. Perché ora che non
	potevano essere insieme erano infinitamente separati, e ciò che prima
	era stato certo e incrollabile adesso era fragile e impalpabile.
	\emph{Da ora in poi Alai diventerà uno sconosciuto ogni giorno di più,
		perché ha una vita che ormai si è staccata dalla mia. E questo significa
		che un bel momento ci incontreremo e scopriremo di non conoscerci l'un
		l'altro.}}

{Questo lo rese triste, ma non al punto di piangere. I suoi occhi non
	erano più capaci di tanto. Quando avevano trasformato Valentine in una
	sconosciuta, quando l'avevano usata come un utensile per lavorare su di
	lui, da quel giorno in poi nulla di quel che potevano fare sarebbe
	riuscito a farlo piangere. Ender era certo di questo.}

{E con quella rabbia in corpo decise che era forte abbastanza da
	resistere loro e da sconfiggerli. I suoi insegnanti. I suoi nemici.}

\phantomsection\label{Orsonux20Scottux20Cardux20-ux20Ilux20Giocoux20Diux20Enderux20-ux20BY_SLY70A1_split_013.htm}{}
