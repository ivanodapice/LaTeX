\chapter{LANCIO}

{~}

{~}

{~}

{-- \emph{Con Ender bisognerà fare un delicato gioco di equilibrio. Lo
		si dovrà isolare abbastanza da farlo restare creativo, altrimenti
		adotterà sistemi già in uso qui e lo avremo perduto. E nello stesso
		tempo dovremo assicurarci che sviluppi forti doti di comando.}}

{-- \emph{Non è così semplice. Mazer Rackham poteva tenere in pugno la
		sua piccola flotta e portarla all'obiettivo. Ma quando scoppierà il
		prossimo conflitto le complicazioni saranno eccessive, anche per un
		piccolo genio. Troppe astronavi, troppi equipaggi. Dovrà avere il guanto
		di velluto coi subordinati.}}

{-- \emph{Oh, Dio! Dovrà essere un genio e anche un simpaticone?}}

{-- \emph{Niente affatto. Un simpaticone ci lascerebbe fare a pezzi
		dagli Scorpioni.}}

{-- \emph{Così lei pensa di isolarlo.}}

{-- \emph{Ne farò un paria rispetto agli altri ragazzi, ancor prima che
		arrivino alla Scuola.}}

{-- \emph{Non ho dubbi che ci riuscirà. Anzi, ci conto. Ho esaminato il
		nastro di ciò che ha fatto al ragazzo Stilson. Quello che lei porterà
		qui non è precisamente un bambinetto sdolcinato.}}

{-- \emph{È qui che lei sbaglia. È più dolce di quel che sembra. Ma non
		si preoccupi, a questo sapremo metter rimedio alla svelta.}}

{-- \emph{Qualche volta penso che lei si diverta a spezzare la schiena a
		questi piccoli geni.}}

{-- \emph{Si tratta di un'arte, nella quale sono ormai molto esperto. Ma
		in quanto a divertirmi? Be', forse. In seguito, quando rimettono insieme
		i loro pezzi e si accorgono che tanto basta a farli star meglio.}}

{-- \emph{Lei è un mostro.}}

{-- \emph{Grazie. Significa che posso sperare in un aumento di paga?}}

{-- \emph{Al massimo una medaglia. I nostri fondi non sono illimitati.}}

{~}

\begin{center}
	{* * *}
\end{center}

{~}

{Li avevano avvertiti che l'assenza di peso poteva sfasare le percezioni
	fisiche, in specie nei bambini, il cui senso dell'orientamento non
	dispone ancora di parametri stabili. Ma Ender cominciò a sentirsi
	disorientato già prima di vedere la navetta che li avrebbe portati
	lontano dalla gravità della Terra.}

{Con lui c'erano altri diciannove ragazzini. Furono fatti scendere dal
	bus ed entrarono nell'ascensore, chiacchierando e ridendo, avidi di
	mostrarsi chi impavido e chi già esperto in materia. Ender mantenne un
	indifferente silenzio. Aveva notato che Graff e gli altri ufficiali li
	stavano osservando. Analizzando. \emph{Tutto ciò che facciamo significa
		qualcosa}, \emph{} si rese conto Ender. \emph{Loro ridono. Io non
		rido.}}

{Si trastullò con l'idea di comportarsi come gli altri ragazzini, ma non
	riuscì a trovare nessuna battuta da dire. Nessuna che fosse divertente,
	almeno. Da qualunque cosa avessero origine le loro risate, Ender non
	avrebbe mai potuto associarsi a quella reazione. Aveva paura, e la paura
	lo rendeva serio e rigido.}

{Gli avevano fatto indossare un'uniforme, una tuta d'un solo pezzo, e
	l'assenza della cintura intorno alla vita lo metteva un po' a disagio.
	In quell'indumento largo e rigonfio si sentiva nudo. C'erano delle
	telecamere puntate su di loro. Le portavano dei militari, tenendosele
	appollaiate su una spalla come animaletti attenti e curiosi. Gli uomini
	si spostavano con cautela felina per riprendere le immagini lentamente e
	senza sbalzi. Anche Ender si scoprì a muoversi lento e senza sbalzi.}

{Immaginò di apparire alla TV, in un'intervista. L'operatore puntava un
	microfono direzionale su di lui: come si sente, signor Wiggin?
	Abbastanza bene, grazie, appena un po' affamato. Affamato? Eh, sì, per
	affrontare il lancio bisogna essere a stomaco vuoto da venti ore. Questo
	è interessante, scommetto che i nostri spettatori non lo sapevano. Be'
	sì, siamo piuttosto affamati tutti quanti. E mentre si lasciava
	intervistare Ender, nell'immaginazione, camminava verso la navetta, con
	l'uomo della TV che al suo fianco procedeva di traverso per puntargli
	addosso la telecamera da spalla. Per la prima volta provò il bisogno di
	unirsi a quelle risatine. Sulle labbra gli comparve un sorriso. In quel
	momento i ragazzini che aveva accanto stavano ridendo anch'essi, per
	un'altra ragione. \emph{Penseranno che sorrido delle loro battute},
	\emph{} rifletté Ender. \emph{Ma è per qualcosa di molto più divertente,
		invece.}}

{-- Avviatevi su per la scala uno alla volta -- disse un ufficiale. --
	Appena sarete nel passaggio fra le poltroncine, sedete sulla più vicina
	che trovate vuota. Non ci sono posti a sedere accanto al finestrino.}

{Era una battuta. Gli altri ragazzini risero.}

{Ender era in fondo alla fila, ma non proprio l'ultimo, e le telecamere
	continuavano a riprenderli. \emph{Valentine mi potrà vedere mentre
		scompaio dentro la navetta?} Pensò che forse avrebbe potuto voltarsi a
	salutarla con la mano, oppure correre da uno degli operatori e chiedere:
	-- Posso dire addio a Valentine? -- Non sapeva però che se l'avesse
	fatto il nastro sarebbe stato censurato, perché ufficialmente si
	supponeva che i giovani diretti alla Scuola di Guerra fossero eroici e
	dignitosi. Non era previsto che sentissero la nostalgia di qualcuno.
	Ender era all'oscuro di questo tipo di censura. Tuttavia sapeva che
	correre a una delle telecamere sarebbe stato uno sbaglio.}

{Attraversò il ponte metallico e il portello della navetta, e notò che
	la paratia alla sua destra aveva la moquette come un pavimento. Lì si
	cominciava a esser disorientati sul serio. Nello stesso momento in cui
	s'accorse che quella parete era un pavimento ebbe la strana sensazione
	di camminare di traverso su un muro. Appoggiò le mani alla scaletta e
	vide che la superficie verticale dietro di essa era coperta di moquette.
	\emph{Mi sto arrampicando su per il pavimento. Mano dopo mano, passo
		dopo passo.}}

{Per gioco immaginò poi di arrampicarsi \emph{giù} per la paratia.
	Subito le sue percezioni mentali si capovolsero, a dispetto di quel che
	diceva la forza di gravità. Appena seduto si aggrappò tenacemente ai
	braccioli per non scivolare in alto, mentre invece il suo peso lo teneva
	incollato allo schienale.}

{Gli altri ragazzini s'erano accalcati alla rinfusa sulle poltroncine e
	facevano baccano chiamandosi l'un l'altro. Ender esaminò con attenzione
	le cinghie di sicurezza e cercò di capire come si agganciavano alla
	cintura, alle cosce e intorno alle spalle. Per un attimo ebbe
	l'impressione d'essere salito su una giostra che li avrebbe fatti girare
	intorno alla Terra, con la forza centrifuga a inchiodarli saldamente sui
	sedili. \emph{Ma non ci sarà peso lassù}, \emph{} pensò. \emph{Cadremo
		via da questo pianeta.}}

{Ancora non si rendeva pienamente conto di quella realtà. Soltanto più
	tardi, riesaminando quei momenti, si sarebbe accorto di aver pensato fin
	da allora alla Terra come a un pianeta, uno qualsiasi, non
	particolarmente il \emph{suo} pianeta.}

{-- Oh, hai già visto come si mettono le cinture -- disse Graff. S'era
	fermato accanto a lui, sulla scaletta.}

{-- Viene con noi? -- domandò Ender.}

{-- Di solito non torno a terra per i reclutamenti -- disse Graff. -- Io
	sono di servizio nello spazio, come amministratore della Scuola. Una
	specie di direttore. Ma stavolta mi hanno detto che avrei dovuto
	scendere, altrimenti mi avrebbero licenziato. -- Curvò le labbra in un
	sorriso.}

{Ender gli sorrise di rimando. Graff lo faceva sentire a suo agio. Graff
	era buono. Ed era il direttore della Scuola di Guerra. Ender si rilassò
	un poco. Lassù avrebbe avuto un amico.}

{Agli altri ragazzini, quelli che non avevano fatto come Ender, venne
	agganciata la cintura di sicurezza. Poi attesero un'ora, mentre uno
	schermo TV sulla paratia anteriore dello scompartimento illustrava il
	funzionamento dell'astronave, la storia dei voli spaziali, e quello che
	avrebbe potuto essere il loro futuro sulle grandi navi della F.I. Una
	cosa abbastanza noiosa. Ender aveva già visto filmati di quel genere.}

{Ma non era mai stato legato a una poltroncina sagomata nell'interno di
	una navetta. Quasi a testa in giù mentre stavano per scaraventarlo via
	dalla Terra.}

{Il lancio non fu duro. Soltanto un po' spiacevole. Ci furono degli
	scossoni, poi brevi momenti d'ansia al pensiero che quello avrebbe
	potuto essere il primo disastro aereo nella storia della F.I. Dai
	filmati non aveva mai capito esattamente quali sensazioni si potevano
	provare stando distesi sulla schiena, con la morbida imbottitura che
	cedeva sotto la pressione.}

{Poi essa parve invertirsi, e lui fu davvero appeso alle cinghie in una
	giostra, in totale assenza di gravità.}

{Ma dal momento che s'era già preparato a orientarsi su nuovi parametri
	non fu sorpreso nel vedere Graff tornare giù per la scaletta a testa in
	avanti, come se ora si arrampicasse verso il retro della navetta. Né si
	meravigliò quando l'uomo agganciò un piede a uno scalino e si diede una
	spinta con le mani, mettendosi in posizione eretta come se fosse in
	piedi fra i sedili di un normale aeroplano.}

{Per alcuni l'inversione del senso dell'equilibrio fu troppo. Un
	ragazzino rantolò, portandosi le mani alla bocca. Finalmente Ender capì
	perché avevano proibito loro di mangiare per venti ore prima del lancio.
	Vomitare a gravità zero sarebbe stato poco divertente per tutti.}

{Ma a Ender i movimenti di Graff in assenza di peso parvero divertenti.
	Si spinse più oltre con la fantasia, provando a immaginare che l'uomo
	camminasse a testa in giù sugli scalini e l'andatura che avrebbe potuto
	adottare procedendo sul soffitto e sulle paratie come una mosca.
	\emph{La gravità può attirare da qualsiasi parte}, \emph{} pensò.
	\emph{Dovunque io immagini di farla girare. Potrei far ruotare Graff a
		testa in giù e lui non si accorgerebbe neppure d'esser stato
		capovolto.}}

{-- Cos'è che ti sembra tanto divertente, Wiggin?}

{La voce di Graff era dura e seccata. \emph{Cos'ho fatto di sbagliato?}
	Pensò Ender. \emph{Che mi sia sfuggita una risatina?}}

{-- Ti ho fatto una domanda, soldato! -- abbaiò Graff.}

{Ah, sì. Quello era veramente l'inizio dell'addestramento alla vita
	militare. Ender aveva visto alla TV sceneggiati sull'arrivo delle
	reclute nei campi, e sapeva che i graduati le accoglievano latrando come
	cani rabbiosi prima che tutti, soldati e ufficiali, diventassero buoni
	compagni d'arme.}

{-- Sissignore -- rispose Ender.}

{-- Allora rispondi alla domanda!}

{-- Stavo pensando che lei potrebbe andare in giro capovolto. Questo mi
	è sembrato comico.}

{Ma sembrava soltanto stupido adesso, con Graff che lo squadrava
	freddamente. -- Suppongo cha \emph{a te} debba sembrare comico. C'è
	qualcun altro che trova la cosa comica, qui dentro?}

{Si levarono mormorii di diniego.}

{-- Nessuno, eh? E perché? -- Graff girò su di loro un'occhiata
	sprezzante. -- Un'imbarcata di teste di rapa, ecco cosa ci hanno
	affibbiato in questo lancio. Piccoli ritardati mentali. Uno solo di voi
	ha avuto l'intelligenza di capire che a gravità zero si può stare dritti
	in qualunque senso uno si metta. Riuscite a farvelo entrare in testa,
	reclute?}

{I ragazzini annuirono.}

{-- No che non ci riuscite, invece. È chiaro che non ci riuscite. Non
	solo stupidi, dunque, ma anche bugiardi. Di questa imbarcata c'è un
	unico ragazzo col cervello in grado di funzionare, ed è Ender Wiggin.
	Guardatelo bene, piccoli sciocchi. Lui avrà un posto di comando quando
	voi sarete ancora a ramazzare i pavimenti, lassù. E questo perché lui sa
	come bisogna pensare in gravità zero, mentre voialtri riuscite soltanto
	a vomitare l'anima.}

{Non era esattamente così che andava negli sceneggiati della TV. A
	regola, Graff avrebbe dovuto infierire su di lui, non metterlo su un
	piedistallo di fronte agli altri. A regola, lui e Graff avrebbero dovuto
	avere rapporti bruschi all'inizio, così più tardi fra loro avrebbe
	potuto istaurarsi quel rude e solido cameratismo.}

{-- Molti di voi finiranno congelati nello spazio. Cominciate a
	considerare questo pensiero fin d'ora, bambocci. Molti di voi non
	faranno altro che spaccarsi la faccia in Sala di Battaglia, perché non
	sapranno adattare il cervello alle tecniche di pilotaggio spaziale.
	Molti di voi non valgono neppure la spesa di trasportarli alla Scuola di
	Guerra, perché non hanno i requisiti necessari. Alcuni di voi potrebbero
	averli. Pochi di voi potrebbero servire a qualcosa per la razza umana.
	Ma non ci scommetterei un soldo. Su uno soltanto sono disposto a
	puntare.}

{D'un tratto Graff fece una piroetta all'indietro e afferrò la scala con
	le mani, proiettando i piedi in direzione opposta. Fino a un attimo
	prima gli scalini erano stati il suo pavimento; con quella mossa parve
	dichiarare che pavimento e soffitto erano la stessa cosa, dando ragione
	a Ender.}

{-- Sembra che tu sia ammanigliato bene, qui -- disse il ragazzino
	seduto davanti a lui.}

{Ender scosse il capo.}

{-- Ah, non vuoi abbassarti a parlare con me? -- disse il ragazzino.}

{-- Non gli ho chiesto io di dire quelle cose -- mormorò Ender.}

{Qualcosa lo colpì dolorosamente alla nuca. Poi lo colpì di nuovo.
	Dietro di lui ci furono alcune risatine. Il ragazzo seduto alle sue
	spalle doveva aver sganciato le cinture della poltroncina. Una scoppola
	gli scompigliò i capelli. \emph{Smettetela, per favore}, \emph{} pensò
	Ender. \emph{Io non vi ho fatto niente.}}

{Ancora un pugno nella nuca. I ragazzini ridacchiarono. Graff si stava
	accorgendo di questo? Non aveva intenzione di mettervi fine? Un altro
	pugno, più forte e stavolta davvero doloroso. Dov'era Graff?}

{Poi capì come stavano le cose. Graff aveva intenzionalmente provocato
	ciò che stava accadendo. Era ancor peggio delle soperchierie che si
	vedevano nei film. Quando un sergente percuote una recluta, gli altri
	solidarizzano col malcapitato. Ma quando la elogia, gli altri la
	odiano.}

{-- Ehi, mangiamerda -- sussurrò una voce dietro di lui. Gli arrivò una
	scoppola. -- Che ne dici di questo? Ehi, super-cervello, questo lo trovi
	comico? -- Ancora un pugno nella nuca, così violento che Ender mandò un
	gemito soffocato.}

{Se Graff lo aveva messo apposta in quella posizione, allora non poteva
	aspettarsi l'aiuto di nessuno. Aspettò finché fu sul punto di ricevere
	un altro pugno. \emph{Adesso}, \emph{} pensò. E infatti il pugno arrivò.
	Gli fece male, ma si costrinse a calcolare il ritmo dei colpi.
	\emph{Adesso.} E in quel preciso momento fu colpito. \emph{Stavolta ti
		tengo}, \emph{} si disse Ender.}

{Un attimo prima del colpo successivo Ender si volse di scatto, afferrò
	il polso del ragazzino con entrambe le mani e gli abbassò violentemente
	il braccio.}

{In gravità normale la mossa avrebbe attirato l'altro contro lo
	schienale del suo sedile, facendogli urtare il petto sullo spigolo. In
	assenza di peso il braccio funse da leva, il ragazzino fu sollevato dal
	suo posto e proiettato verso il soffitto. Ender non se l'era aspettato.
	Non aveva ancora capito quanto fosse facile spostare una massa a gravità
	zero. Il ragazzino volò obliquamente contro il soffitto, rimbalzò in
	basso addosso a un altro seduto nella poltroncina, e la spinta lo mandò
	a roteare avanti lungo il passaggio centrale finché con un grido di
	dolore urtò pesantemente nella paratia anteriore. Il suo braccio era
	piegato in modo anomalo quando rimbalzò ancora in alto.}

{La cosa era durata appena pochi secondi, ma Graff era già sbucato dalla
	cabina di pilotaggio, in tempo per intercettare al volo il ragazzino.
	Con una smorfia lo spinse verso un altro degli ufficiali. -- Braccio
	sinistro. Fratturato, direi -- fu il suo commento. Pochi minuti dopo al
	ragazzo era già stato iniettato un antidolorifico, e tenendolo sospeso a
	mezz'aria l'ufficiale gli arrotolò un bendaggio rigido attorno al
	braccio.}

{Ender si sentiva sgomento. Tutto ciò che aveva voluto era stato di
	fermare il braccio del ragazzino\ldots{} no, no, aveva voluto fargli
	male, e ci aveva messo tutta la sua forza. Non era stato nelle sue
	intenzioni dare il via a una scena di quel genere, e tuttavia il suo
	tormentatore si stava sorbendo esattamente quel che lui aveva voluto
	procurargli. L'assenza di gravità aveva giocato a suo sfavore, tutto
	qui. \emph{Io sono Peter. Sono proprio come lui}, \emph{} pensò Ender. E
	odiò se stesso.}

{Sulla soglia della cabina Graff si volse. -- Mi domando se non siate
	dei bambocci lenti di comprendonio. I vostri cervellini non hanno ancora
	capito questo semplice fatto? Siete stati portati qui per diventare dei
	\emph{soldati.} Forse nelle vostre famiglie o a scuola eravate
	considerati dei duri, magari perfino intelligenti. Ma noi scegliamo il
	meglio del meglio, e questo è il solo genere di compagni che
	incontrerete d'ora in avanti. Perciò, quando vi dico che Ender Wiggin è
	il migliore di questo lotto aprite gli orecchi, teste dure. Non
	prendetelo sottogamba. Alla Scuola di Guerra dei pivelli della vostra
	età ci hanno già lasciato la pelle in passato. Sono stato abbastanza
	chiaro?}

{Per il resto del volo nessuno aprì bocca. Il ragazzino seduto a fianco
	di Ender prestò scrupolosa attenzione a non sfiorarlo neppure.}

\emph{{Io non sono un killer}}{, \emph{} disse Ender a se stesso più
	volte. \emph{Non sono Peter. Qualunque cosa lui dica, io non lo sono e
		non voglio esserlo. Mi sono soltanto difeso. Avevo cercato di
		sopportare. E ho avuto pazienza. Non sono come lui ha detto.}}

{Una voce dall'interfono li informò che la navetta era in fase di
	avvicinamento alla Scuola. Occorsero venti minuti per la decelerazione e
	l'attracco. Ender si tirò avanti per la scaletta in coda al gruppo, e
	arrampicandosi nella direzione che alla partenza era stata il basso ebbe
	l'impressione che gli altri fossero quasi ansiosi di lasciarselo alle
	spalle. Al termine del corridoio flessibile che collegava la navetta
	alle strutture della Scuola c'era in attesa Graff.}

{-- Hai fatto buon viaggio, Ender? -- gli domandò gentilmente.}

{-- Credevo che lei fosse mio amico. -- A dispetto dei suoi sforzi Ender
	sentì che gli tremava la voce.}

{Graff parve sorpreso. -- E dove hai preso questa idea, Ender?}

{-- Perché lei\ldots{} -- \emph{Perché lei era stato buono con me, e
		onesto.} -- Lei non mi ha mai mentito.}

{-- E non voglio mentirti neppure adesso -- disse Graff. -- Il mio
	compito non è di essere tuo amico. È di formare quelli che dovranno
	essere i migliori combattenti del mondo. I migliori della storia. A noi
	serve un Napoleone. Un Alessandro. Salvo che Napoleone alla fine fu
	sconfitto, e Alessandro morì giovane dopo aver fiammeggiato come una
	meteora. O avremmo bisogno di un Giulio Cesare, senonché egli divenne un
	dittatore e per questo fu ucciso. Il mio compito è di formare un
	individuo di questo tipo, e tutti gli uomini e le donne di cui avrà
	bisogno per agire. E nel regolamento non è scritto che per arrivarci io
	debba essere un amico per voialtri ragazzini.}

{-- Lei li ha indotti a detestarmi.}

{-- Sul serio? E tu che pensi di farci? Nasconderti in un angoletto? O
	baciare il sedere a tutti quanti perché ricomincino a volerti bene? Hai
	un solo modo perché smettano di odiarti: diventare così bravo che
	nessuno ti possa ignorare. Io ho detto loro che sei il migliore. Adesso
	farai dannatamente bene a dimostrare che lo sei davvero.}

{-- E se non ci riuscissi?}

{-- Peggio per te. Senti, Ender, non mi rende felice pensare che tu
	abbia paura o ti senta solo. Ma là fuori ci sono gli Scorpioni. Dieci
	miliardi, cento miliardi, o per quel che ne sappiamo un milione di
	miliardi. Forse con altrettante astronavi. Con armi a noi del tutto
	sconosciute. E con la ferma volontà di usarle per spazzarci via. Non è
	in gioco la Terra, Ender. Soltanto noi, soltanto la razza umana. Per
	quel che riguarda il pianeta noi potremmo anche scomparire, e lui
	andrebbe avanti verso il prossimo passo nell'evoluzione della vita. Ma
	l'umanità non vuole estinguersi. Come specie, noi abbiamo il dovere e
	l'istinto della sopravvivenza. Un istinto che si crea nelle avversità e
	nel loro susseguirsi finché, come prodotto dallo sforzo di generazioni,
	la razza dà alla luce un genio. Quello che riesce a inventare la ruota,
	o la luce elettrica, o il volo. Quello che costruisce una città, una
	nazione, un impero. Capisci il senso di questo?}

{Ender rifletté che lo capiva, ma non era del tutto sicuro, così non
	disse niente.}

{-- No, naturalmente no. Allora sarò più chiaro. Gli esseri umani hanno
	il diritto di essere liberi, salvo quando l'umanità ha bisogno di loro.
	Forse l'umanità ha bisogno di te. Perché tu faccia qualcosa. Io penso
	che comunque abbia bisogno di \emph{me\ldots{}} per scoprire se quelli
	come te possono servire. Tanto tu che io potremmo dover fare cose poco
	commendevoli, Ender, ma se grazie ad esse l'umanità riuscirà a
	sopravvivere noi saremo stati dei buoni strumenti.}

{-- Soltanto questo? Nient'altro che strumenti?}

{-- Individualmente gli esseri umani sono degli strumenti, che altri
	hanno il diritto di usare per la sopravvivenza della razza.}

{-- Questa è una menzogna.}

{-- No, è soltanto metà della verità. Dell'altra metà potrai
	preoccupartene dopo che avremo vinto questa guerra.}

{-- Potremmo essere distrutti prima che io diventi grande -- disse
	Ender.}

{-- Spero che non accada -- borbottò Graff. -- Comunque, stando qui a
	parlare con me non fai i tuoi interessi. Gli altri penseranno che quel
	furbone di Ender Wiggin sta leccando le scarpe a Graff. E se corre voce
	che sei il pupillo del direttore, stai certo che ti succederà qualche
	incidente.}

\emph{{In altre parole, levati dai piedi e lasciami in pace. }}{--
	Arrivederci -- disse Ender. Una mano dopo l'altra si spinse lungo il
	corridoio nella direzione in cui gli altri erano scomparsi.}

{Graff lo seguì con lo sguardo.}

{Accanto a lui uno degli insegnanti disse: -- È lui quello su cui
	contiamo?}

{-- Lo sa Iddio -- mormorò Graff. -- Se non fosse lui, meglio che Ender
	ce lo faccia capire al più presto.}

{-- Forse non è nessuno di loro -- disse l'insegnante.}

{-- Forse. Ma se le cose stanno così, Anderson, vuol dire che il solo
	Dio è quello degli Scorpioni. E puoi citare le mie parole.}

{-- Lo farò.}

{Per un poco i due rimasero in silenzio.}

{-- Anderson\ldots{}}

{-- Mmh?}

{-- Il ragazzo sbaglia. Io \emph{sono} suo amico.}

{-- Lo so.}

{-- È intelligente. Te lo dico col cuore, ha del carattere.}

{-- Ho letto i rapporti.}

{-- Pensa a quel che gli stiamo facendo, Anderson.}

{L'altro lo fissò con aria di sfida. -- Stiamo cercando di farne il
	miglior comandante in campo della storia.}

{-- Per poi gettare sulle sue spalle il destino del mondo. Dovrei
	sperare che quello che cerchiamo non sia lui, per il suo bene. E lo
	spero.}

{-- Consolati, magari gli Scorpioni ci faranno fuori tutti prima ancora
	che dia gli esami.}

{Graff sorrise. -- Hai ragione. Sai una cosa? Le tue profezie sono
	ottime per tirare un uomo su di morale.}

\phantomsection\label{Orsonux20Scottux20Cardux20-ux20Ilux20Giocoux20Diux20Enderux20-ux20BY_SLY70A1_split_007.htm}{}
