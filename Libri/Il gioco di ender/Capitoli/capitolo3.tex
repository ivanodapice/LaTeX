\chapter{GRAFF}

{~}

{~}

{~}

{-- \emph{La sorella è il nostro punto più debole. Le vuol bene
		davvero.}}

{-- \emph{Lo so. Lei può bloccarci. Il ragazzo non vuole lasciarla.}}

{-- \emph{Perciò, cosa intendete fare?}}

{-- \emph{Lo persuaderemo che desidera venire con noi più di quanto
		voglia restare con lei.}}

{-- \emph{In che modo pensa di riuscirci?}}

{-- \emph{Gli mentirò.}}

{-- \emph{E se non funziona?}}

{-- \emph{Allora gli dirò la verità. Ci è concesso farlo, in caso di
		emergenza. Abbiamo linee di condotta pronte per ogni circostanza, lo
		sa.}}

{~}

\begin{center}
	{* * *}
\end{center}

{~}

{All'ora di colazione Ender non aveva un briciolo d'appetito. Stava
	cominciando a chiedersi come sarebbe stata, a scuola. Affrontare Stilson
	dopo la zuffa del giorno prima. Cos'avrebbero fatto gli altri della sua
	banda? Probabilmente nulla, ma di questo non poteva essere sicuro.
	Scoprì che non aveva voglia di andarci.}

{-- Ender, non hai ancora mangiato niente -- disse sua madre.}

{Peter entrò in cucina. -- Buongiorno, Ender. Grazie per aver lasciato
	tutti gli asciugamani bagnati, nella doccia.}

{-- Per te farei questo ed altro -- mormorò lui.}

{-- Andrew, devi mangiare.}

{Ender tese un braccio e le porse la parte interna del gomito, in un
	gesto che diceva: allora nutritemi attraverso un ago.}

{-- Molto divertente -- sospirò sua madre. -- Non c'è bisogno che io mi
	preoccupi per voi, vero? È bello avere figli tanto geniali.}

{-- Sono i tuoi geni che ci hanno fatti cosi geniali, mamma -- disse
	Peter. -- Per fortuna i geni di papà quel giorno erano in ferie.}

{-- Ti ho sentito -- borbottò suo padre, senza alzare gli occhi dal
	video-giornale acceso sul piano del tavolo.}

{-- In caso contrario la mia battuta sarebbe andata sprecata.}

{Il tavolo emise una nota musicale. Qualcuno era alla porta.}

{-- Chi può essere? -- chiese la donna al marito.}

{Lui sfiorò un pulsante della tastiera e sul video della cucina apparve
	la figura di un uomo, a mezzo busto. Indossava una uniforme, l'unica
	riconoscibile all'istante in tutto il pianeta: quella della Flotta
	Internazionale.}

{-- Credevo che con questa faccenda avessimo chiuso -- disse il padre.}

{Peter tacque, limitandosi a versare il latte nel suo piatto di cereali.
	Ma Ender s'era irrigidito. \emph{Forse oggi non dovrò andare a scuola,
		dopotutto.}}

{Suo padre batté il codice d'apertura per la porta e si alzò da tavola.
	-- Me ne occupo io -- disse. -- Voi fate colazione.}

{Gli altri annuirono, ma nessuno cominciò a mangiare. Qualche minuto
	dopo l'uomo riapparve sulla soglia e accennò alla moglie di seguirlo in
	soggiorno.}

{-- Sei nei guai fino al collo -- commentò Peter. -- Hanno scoperto quel
	che hai fatto a Stilson, e adesso sarai deportato sulla Cintura degli
	Asteroidi.}

{-- Ho soltanto sei anni, idiota. Sono troppo giovane.}

{-- Sei un Terzo, caccola. Voi non avete diritti civili.}

{Valentine fece il suo ingresso in cucina, insonnolita e coi capelli
	scompigliati intorno al volto. -- Dove sono mamma e papà? Oggi mi sento
	troppo male per andare a scuola.}

{-- Un altro esame orale, eh? -- chiese Peter.}

{-- Oh, taci, Peter -- disse Valentine.}

{-- Dovresti essere tranquilla e riderci sopra -- continuò Peter. --
	Potrebbe andarti peggio.}

{-- Non vedo come.}

{-- Potrebbe essere un esame anale.}

{-- Davvero spiritoso, proprio -- disse Valentine. -- Dove sono mamma e
	papà?}

{-- Stanno parlando con un tipo della F.I.}

{D'impulso lei guardò Ender. D'altronde ormai da anni si aspettavano che
	qualcuno venisse a dir loro che Ender aveva superato l'esame, che c'era
	bisogno di lui.}

{-- Certo, è giusto che tu pensi a lui -- annuì Peter. -- Ma potrebbe
	essere per me, lo sai. Loro potrebbero aver capito che a conti fatti io
	resto il migliore. -- Il suo tono era aspro, come sempre quando si
	sentiva ferito.}

{La porta fu aperta. -- Ender -- disse suo padre, -- meglio che tu venga
	un momento qui.}

{-- Condoglianze, Peter -- sorrise Valentine.}

{L'uomo si accigliò. -- Ragazzi, non è cosa su cui scherzare.}

{Ender lo seguì in soggiorno. L'ufficiale della F.I. si alzò nel vederli
	entrare, ma non accennò a porgere la mano al bambino.}

{Sua madre si stava tormentando nervosamente l'anello nuziale. -- Andrew
	-- mormorò, -- non avrei mai creduto che tu facessi il prepotente in una
	zuffa.}

{-- Il figlio degli Stilson è all'ospedale -- disse suo padre. -- L'hai
	fatta grossa, Ender. Non è esattamente cavalieresco prendere qualcuno a
	calci in faccia.}

{Ender scosse il capo. S'era aspettato che per la faccenda di Stilson
	venisse qualcuno della scuola, non certo un ufficiale della F.I. La cosa
	era ancora più seria di quanto avesse creduto. E tuttavia non capiva che
	altro di grave potesse aver fatto.}

{-- Hai una spiegazione per il tuo comportamento, giovanotto? -- domandò
	l'ufficiale.}

{Ender scosse ancora il capo. Non sapeva cosa dire, e temeva che
	spiegarsi lo avrebbe fatto apparire ancor più spregevole di quel che i
	fatti nudi e crudi rivelavano. \emph{Accetterò la punizione, qualunque
		sia}, \emph{} si disse. \emph{Anche questa passerà.}}

{-- Siamo propensi a considerare le circostanze attenuanti -- disse
	l'ufficiale. -- Ma è mio dovere sottolineare la gravità del caso.
	Colpirlo al ventre, e ripetutamente in faccia e al corpo mentre era a
	terra\ldots{} c'è da pensare che tu ci provassi gusto.}

{-- Io no, signore -- sussurrò Ender.}

{-- Allora perché l'hai fatto?}

{-- Con lui c'era la sua banda -- disse Ender.}

{-- E con ciò? Questo giustifica tutto?}

{-- No, signore.}

{-- Dimmi perché hai continuato a colpirlo. Avevi già vinto.}

{-- Buttandolo a terra avevo vinto solo il primo scontro. Io volevo
	vincere subito anche i prossimi, definitivamente, cosi mi avrebbero
	lasciato in pace. -- Ender non poté evitarlo, era troppo spaventato,
	troppo vergognoso di quel che aveva fatto: malgrado ogni tentativo di
	controllarsi scoppiò di nuovo in lacrime. Piangere non gli piaceva, e lo
	faceva di rado, ma ecco che adesso in meno di ventiquattr'ore gli
	succedeva per la terza volta. E la cosa più vergognosa era piangere così
	davanti ai suoi genitori e a quello sconosciuto in divisa. -- Voi mi
	avete levato il monitor -- ansimò. -- Dovevo cavarmela da solo, si o
	no?}

{-- Ender, avresti dovuto chiedere aiuto a un adulto\ldots{} -- cominciò
	a dire suo padre.}

{Ma l'ufficiale si alzò e attraverso il soggiorno, quindi porse la mano
	al bambino. -- Il mio nome è Graff, Ender. Colonnello Hyrum Graff. Sono
	il direttore dei corsi di addestramento alla Scuola di Guerra, nella
	Cintura. Sono venuto per invitarti a iscriverti alla Scuola.}

{Dopo tutto quel che era accaduto. -- Ma il monitor\ldots{}}

{-- Il passo conclusivo nel tuo esame consisteva nel vedere come avresti
	reagito una volta privo del monitor. Non sempre facciamo a questo modo,
	ma nel tuo caso\ldots{}}

{-- E ho superato l'esame?}

{Sua madre lo fissava, incredula. -- Dopo aver mandato il ragazzo
	Stilson all'ospedale? Che avreste fatto se Andrew l'avesse ucciso? Gli
	avreste dato una medaglia?}

{-- Non è ciò che ha fatto, signora Wiggin. È il perché. -- Il colonello
	Graff le porse una cartelletta piena di fogli. -- Qui c'è
	l'autorizzazione al prelievo legalizzato: vostro figlio è stato ritenuto
	idoneo dal Dipartimento Selezioni della F.I. Naturalmente abbiamo già il
	vostro consenso legale, firmato prima che vi venisse data
	l'autorizzazione a concepire il bambino. Fin da allora, se giudicato
	idoneo, lui appartiene a noi.}

{Il signor Wiggin non riuscì a nascondere un tremito nella voce. -- Non
	è stato bello da parte vostra lasciarci credere che non lo volevate, e
	poi venire qui a prelevarlo.}

{-- E poteva risparmiarsi quella sceneggiata sul ragazzo Stilson --
	disse sua moglie.}

{-- Non era una sceneggiata, signora Wiggin. Finché non sapevamo in base
	a quale motivazione Ender ha agito, come potevamo esser certi che non
	fosse un altro\ldots{} be', dovevamo conoscere la ragione del suo
	comportamento. O almeno, quella che Ender ritiene sia la ragione.}

{-- Deve proprio chiamarlo anche lei con quello stupido nomignolo? -- La
	donna cominciò a piangere.}

{-- Chiedo scusa, signora Wiggin, ma è così che lui si fa chiamare.}

{-- E adesso che intende fare, colonnello Graff? -- disse il signor
	Wiggin. -- Se ne va e lo porta via con sé, così sui due piedi?}

{-- Questo dipende -- disse Graff.}

{-- Da cosa?}

{-- Dal fatto che Ender voglia venire o no.}

{Fra le lacrime della donna ricomparve un sorriso. -- Oh, ma allora
	l'accettazione è volontaria, dopotutto! È così?}

{-- Per quanto riguarda voi, avete fatto la vostra scelta prima che il
	bambino fosse concepito. Ma Ender, personalmente, non ha fatto ancora
	nessuna scelta. L'arruolamento obbligatorio fornisce ottima carne da
	cannone, però al Corso Ufficiali possono accedere soltanto i volontari.}

{-- Corso Ufficiali? -- chiese Ender. Il tono della sua voce fece
	ammutolire i suoi genitori.}

{-- Sì -- disse Graff. -- La Scuola di Guerra addestra i futuri
	comandanti di astronave, i commodori di squadriglia e gli ammiragli di
	flotta.}

{-- Non gli faccia ballare questa carota davanti al naso -- disse
	irosamente il signor Wiggin. -- Quanti dei ragazzini entrati alla scuola
	di Guerra oggi sono al comando di un'astronave, eh?}

{-- Sfortunatamente, signor Wiggin, questa è un'informazione riservata.
	Ma posso dirle che nessuno dei nostri ragazzi usciti dal primo anno di
	addestramento ha mai mancato di ottenere un incarico come ufficiale. E
	nessuno ha mai fatto servizio con grado inferiore a quello di capitano
	di vascello su una nave interplanetaria. Perfino nei servizi a terra
	nella Difesa Strategica del sistema solare gli ufficiali usciti dalla
	Scuola occupano posizioni di tutto rispetto.}

{-- Quanti riescono a superare il primo anno? -- chiese Ender.}

{-- Tutti quelli che vogliono riuscirci -- disse Graff.}

\emph{{Io lo voglio}}{, \emph{} fu sul punto di dire Ender. Ma tenne a
	freno la lingua. Questo gli avrebbe risparmiato di tornare a scuola,
	però il pensiero gli sembrò stupido, perché quel problema si sarebbe
	risolto comunque in pochi giorni. La cosa lo avrebbe allontanato da
	Peter\ldots{} questo era più importante, questo poteva significare la
	vita stessa. Ma avrebbe dovuto lasciare mamma e papà, e soprattutto
	Valentine. E diventare un soldato. A Ender non piaceva combattere. Non
	gli piaceva farlo al modo di Peter, il forte contro il debole, e
	d'altronde neppure a modo suo, l'intelligente contro lo sciocco.}

{-- Credo che adesso -- disse Graff, -- Ender e io dovremmo parlare un
	po' in privato.}

{-- No -- disse il padre.}

{-- Non lo porterò via senza darvi la possibilità di parlare ancora con
	lui -- disse Graff. -- Comunque non potete impedirmelo, sia chiaro.}

{Il signor Wiggin fissò Graff in silenzio per qualche istante, poi si
	volse e lasciò la stanza. La madre di Ender si fermò a stringergli forte
	una mano. Subito dopo uscì e chiuse la porta.}

{-- Ender -- cominciò Graff, -- se vieni con me non potrai tornare qui
	per molto tempo. Alla Scuola di Guerra non ci sono vacanze. E non sono
	ammesse le visite. Il corso completo di addestramento durerà fino al tuo
	sedicesimo compleanno\ldots{} e potrai godere del primo periodo di
	libera uscita, a certe condizioni, solo quando avrai dodici anni. Puoi
	credermi quando ti dico che in sei anni, in dieci anni, la gente cambia,
	Ender. Tua sorella Valentine sarà una donna il giorno in cui potrai
	rivederla di nuovo, se verrai con me. Sarete due sconosciuti. Tu le
	vorrai bene ugualmente, Ender, ma non la riconoscerai neppure. Come
	vedi, non ti sto dicendo che sarà facile.}

{-- E mamma e papà?}

{-- Io ti conosco, Ender. Assai spesso ho consultato le registrazioni su
	disco del tuo monitor. Non proverai nostalgia per i tuoi genitori, non
	molto, e non a lungo. E neppure loro sentiranno per troppo tempo la tua
	mancanza.}

{Malgrado ogni sforzo Ender si sentì salire le lacrime agli occhi.
	Distolse il viso, ma non volle alzare una mano ad asciugarsele.}

{-- Essi ti \emph{amano}, \emph{} Ender. Però devi capire quel che sei
	costato loro. Sai bene che provengono da famiglie religiose. Tuo padre è
	stato battezzato col nome di John Paul Wieczorek. Cattolico. Il settimo
	di nove fratelli.}

{Nove figli. Questo era quasi incredibile. Criminale.}

{-- Be', sì, la gente fa strane cose per la religione. Tu conosci le
	sanzioni, Ender\ldots{} a quei tempi non erano dure, ma neppure lievi.
	Soltanto i primi due figli avevano diritto all'istruzione gratuita. E
	per ogni figlio in più si pagavano tasse maggiori. A sedici anni tuo
	padre si appellò alla Legge sulle Famiglie Dissidenti per separarsi
	dalla sua famiglia. Cambiò nome, rinunciò alla religione, e fece voto di
	non avere mai più figli dei due ufficialmente consentiti. Era una cosa
	in cui credeva. Tutta la vergogna e le persecuzioni che aveva dovuto
	sopportare da bambino\ldots{} giurò che questo non sarebbe mai accaduto
	a un figlio suo. Capisci?}

{-- Lui non mi voleva.}

{-- Be', nessuno \emph{vuole} veramente un Terzo. Non ci si può
	aspettare che sia felice. Ma tua madre e tuo padre erano casi speciali.
	Entrambi avevano rinunciato alla loro religione (tua madre era una
	mormone) ma in realtà avevano desideri un po' ambigui. Sai che significa
	ambigui?}

{-- Desideravano due cose opposte.}

{-- Si vergognano di provenire da famiglie dissidenti. E cercano di
	nasconderlo, al punto che tua madre rifiuta di ammettere con chiunque di
	essere nativa dello Utah, perché nessuno sospetti la verità. Tuo padre
	rinnega i suoi antenati polacchi, perché la Polonia è una nazione
	dissidente e sotto sanzioni internazionali a causa di questo. Così vedi
	bene che avere un Terzo, anche in obbedienza a esplicite istruzioni del
	governo, distrusse tutto ciò che avevano cercato di costruire.}

{-- Questo lo so.}

{-- Ma la cosa è ancora più complicata. Tuo padre ha voluto darti il
	nome di uno dei santi del calendario. Anzi, è giunto al punto di
	battezzarvi lui stesso tutti e tre quando foste portati a casa dopo la
	nascita. E tua madre non era d'accordo. Ogni volta litigarono, e non
	perché lei fosse contraria al sacramento ma perché non voleva che foste
	battezzati come cattolici. Nessuno dei due ha veramente abbandonato la
	sua religione. Ti guardano e vedono in te un motivo di orgoglio, perché
	sono riusciti ad aggirare la legge e ad avere un Terzo. Ma tu sei anche
	un emblema della loro vigliaccheria, perché non osano andare ancora più
	in là e praticare la dissidenza che nel loro intimo continuano a
	ritenere giusta. E sei anche il simbolo della loro vergogna sociale,
	perché la tua stessa presenza interferisce con gli sforzi che fanno per
	essere integrati nella normale società non dissidente.}

{-- Lei come fa a sapere tutto questo?}

{-- Abbiamo monitorato tuo fratello e tua sorella, Ender. E ti
	stupirebbe sapere quanto è sensibile quello strumento. Eravamo in
	collegamento diretto col tuo cervello. Sentivamo tutto quello che ti
	giungeva agli orecchi, che tu stessi ascoltando con attenzione o meno.
	E\ldots{} che tu lo capissi o meno, \emph{noi} lo capivamo.}

{-- Così i miei genitori mi amano o non mi amano?}

{-- Ti amano. La questione è se ti vogliono qui. La tua presenza in
	questa casa è un elemento di costante disgregazione. Una fonte di
	tensione. Capisci?}

{-- Non sono \emph{io} quello che causa tensione.}

{-- Non è quello che \emph{fai}, \emph{} Ender. È il fatto che esisti.
	Tuo fratello ti odia perché sei la prova vivente che lui non è stato
	abbastanza bravo. I tuoi genitori vedono in te tutto il passato da cui
	hanno cercato di fuggire.}

{-- Valentine mi vuole bene.}

{-- Con tutto il cuore, lealmente, appassionatamente. Lei ti è devota e
	tu l'adori. Te l'ho detto che non è cosa facile.}

{-- Come sarà, lassù?}

{-- Lavorerai duro. Studierai come qui a scuola, ma avrai un'istruzione
	ferrea in matematica e nei computer. In storia militare. In strategia e
	tattica. E soprattutto, la sala di battaglia.}

{-- Che cos'è?}

{-- Simulazione bellica. Tutti gli studenti sono inquadrati in piccoli
	eserciti. Ogni giorno combattono battaglie simulate. Nessuno resta
	ferito, ci sono soltanto vincitori e perdenti. Ognuno comincia come
	soldato semplice, sottoposto agli ordini. I ragazzi più anziani saranno
	i tuoi ufficiali, col dovere di addestrarti e guidarti in battaglia. Ma
	c'è più di questo. È come giocare a Scorpioni e Astronauti\ldots{} salvo
	che avrai armi funzionanti, e compagni che combatteranno al tuo fianco,
	perché il vostro futuro e quello dell'intera razza umana dipendono dalle
	vostre capacità di imparare e di affrontare la guerra. Ma è chiaro che
	con la tua mentalità, e con lo svantaggio d'essere un Terzo, non avresti
	comunque un'adolescenza normale.}

{-- Sono tutti maschi?}

{-- Ci sono anche delle femmine. Ma poche riescono a passare i test del
	reclutamento. Troppi secoli di evoluzione le ostacolano ancora. Nessuna
	di loro potrà essere per te una seconda Valentine, stanne certo. Ma
	troverai là dei fratelli, Ender.}

{-- Come Peter?}

{-- Peter non è stato accettato, Ender, e per la stessa ragione per cui
	si fa odiare da te.}

{-- Io non lo odio. Solo che\ldots{}}

{-- Ne hai paura. Be', Peter non è del tutto malvagio, lo sai. Lo
	giudicammo il migliore che avevamo visto fino a quel momento. Subito
	dopo chiedemmo ai tuoi genitori di avere una figlia femmina (l'avrebbero
	voluta comunque) sperando che Valentine sarebbe stata un Peter
	dall'animo più mite. Ma risultò troppo mite. Così chiedemmo loro di
	avere te.}

{-- Contando che fossi una via di mezzo fra Peter e Valentine?}

{-- Se tu avessi ereditato i cromosomi giusti.}

{-- E li ho?}

{-- Sì, per quanto ne possiamo dire. I tuoi test sono risultati molto
	buoni, Ender. Però essi non ci dicono tutto. In realtà anzi, quando si
	viene ai fatti, ci dicono assai poco. Ma sono meglio di niente. -- Graff
	si chinò e prese le mani di Ender fra le sue. -- Ender Wiggin, se si
	trattasse soltanto di scegliere per te il futuro migliore ti direi di
	restare qui a casa tua. Ti direi di amare i tuoi, di crescere, di farti
	una vita. Ci sono cose peggiori che essere un Terzo o avere un fratello
	maggiore che non riesce a decidere se essere una persona o un cane
	rabbioso. La Scuola di Guerra è una di queste cose peggiori. Però
	abbiamo bisogno di ragazzini come te. Può darsi che oggi gli Scorpioni
	ti sembrino una specie di gioco, Ender, ma il loro ultimo attacco è
	andato maledettamente vicino a spazzar via la razza umana. Ci avevano
	soverchiati, sia come numero che come mezzi e armamenti. La sola cosa
	che ci salvò fu la fortuna, perché proprio allora era in servizio il più
	brillante dei nostri generali. Chiamala fortuna, chiamala provvidenza
	divina, chiamalo un dannatissimo caso, noi avevamo Mazer Rackham.}

{«Ma adesso un Rackham non ce l'abbiamo, Ender. Si è dato fondo alle
	risorse di tutto il pianeta, e abbiamo una flotta al cui confronto
	quella che ci hanno mandato addosso l'ultima volta è una frotta di
	barchette a galla in una vasca da bagno. Ci sono anche alcune nuove
	armi. Ma questo potrebbe non essere abbastanza, perché negli ottant'anni
	trascorsi dall'ultima guerra loro hanno avuto lo stesso tempo per
	potenziarsi. Ci serve il meglio che possiamo avere, e ci serve adesso.
	Non so se tu voglia metterti a lavorare con noi o no, Ender, e non so
	dirti se ce la farai a resistere allo sforzo. Forse non otterrai altro
	che rovinare la tua vita, forse mi odierai per essere venuto oggi a casa
	tua. Ma se c'è una possibilità che arruolandoti nella Flotta tu possa
	contribuire alla sopravvivenza dell'umanità nella lotta contro gli
	Scorpioni\ldots{} allora è mio dovere chiederti di farlo, e di venire
	con me.}

{Gli occhi di Ender non mettevano più a fuoco il colonnello Graff.
	L'uomo gli appariva stranamente lontano, e così piccolo che ebbe
	l'impressione di poterlo raccogliere con un paio di pinzette e
	metterselo in tasca. Lasciare tutto ciò che aveva lì: andare in un posto
	duro e spiacevole, senza Valentine, senza mamma e papà.}

{E poi ripensò ai film sugli Scorpioni che tutti avevano occasione di
	vedere almeno una volta all'anno. La devastazione della Cina. La
	battaglia degli Asteroidi. E Mazer Rackham che con le sue brillanti
	manovre tattiche distruggeva una flotta nemica due volte più grossa
	della sua e con una doppia potenza di fuoco, mandando all'attacco quelle
	astronavi che sembravano così fragili e inermi. Come bambini che si
	battessero contro adulti grossi e minacciosi. E avevano vinto.}

{-- Ho paura -- disse Ender sottovoce, -- ma credo che verrò con lei.}

{-- Non devi avere dubbi -- disse Graff.}

{Lui scosse il capo. -- È per questo che sono nato, non è così? Se non
	venissi, che scopo avrebbe la mia vita?}

{-- Questo non è ancora un buon motivo -- osservò Graff.}

{-- Non voglio venire con lei -- disse Ender, -- ma verrò lo stesso.}

{Graff annuì. -- Puoi ancora cambiare idea. Fino al momento in cui
	salirai sulla mia auto, puoi cambiarla. Ma da allora in poi sarai
	sottoposto all'autorità della Flotta Internazionale. Lo capisci questo?}

{Ender accennò di sì.}

{-- Va bene. Dillo ai tuoi.}

{Sua madre pianse. Suo padre lo abbracciò strettamente. Peter gli
	strinse la mano e disse: -- Tu, piccolo fortunato stronzetto
	presuntuoso. -- Valentine lo baciò e gli lasciò le sue lacrime sulle
	guance.}

{Non c'erano valigie da fare. Nessun oggetto personale da potare con sé.
	-- La scuola provvederà a darti tutto quello che ti serve, dalle
	uniformi al rancio quotidiano. E per giocare\ldots{} avrai soltanto le
	simulazioni belliche.}

{-- Arrivederci -- disse Ender ai suoi familiari. Mise una mano in
	quella del colonnello Graff e uscì dalla porta al suo fianco.}

{-- Fai fuori un paio di Scorpioni per me! -- gli gridò Peter.}

{-- Non dimenticare che ti voglio bene, Andrew! -- disse sua madre.}

{-- Ti scriveremo! -- promise il padre.}

{E mentre saliva sull'auto che li attendeva nel corridoio esterno sentì
	la voce di Valentine rotta dai singhiozzi: -- Ritorna da me! Ritorna, io
	ti vorrò bene per sempre!}

\phantomsection\label{Orsonux20Scottux20Cardux20-ux20Ilux20Giocoux20Diux20Enderux20-ux20BY_SLY70A1_split_006.htm}{}
