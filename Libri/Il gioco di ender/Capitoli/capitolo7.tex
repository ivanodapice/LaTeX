\chapter{SALAMANDRA}

{~}

{~}

{~}

{-- \emph{Non è simpatico sapere che Ender riesce a fare
		l'impossibile?}}

{-- \emph{La morte di un giocatore ha deleteri effetti cumulativi sulla
		sua mente. Ho sempre pensato che il Drink del Gigante fosse il gioco più
		pericoloso da questo punto di vista. Ma accanirsi sul suo occhio a quel
		modo\ldots{} è questo il nostro miglior candidato al comando della
		Flotta?}}

{-- \emph{Non vedo cosa ci sia di male nell'aver vinto a un gioco
		truccato.}}

{-- \emph{Suppongo che adesso lei lo trasferirà.}}

{-- \emph{Stavamo aspettando di vedere cos'avrebbe fatto con Bernard. Se
		l'è cavata perfettamente.}}

{-- \emph{Così, appena riesce a risolvere una situazione lei lo mette di
		fronte a un'altra che non sa come affrontare. Non gli lascerà un po' di
		riposo?}}

{-- \emph{Avrà un mese o due, forse tre, di tranquillità col suo gruppo.
		È un periodo abbastanza lungo, nella vita di un bambino.}}

{-- \emph{Non hai mai l'impressione che questi non siano bambini? Io
		osservo quel che fanno, ascolto ciò che dicono, e non mi sembra che
		abbiano molto di infantile.}}

{-- \emph{Sono i più brillanti bambini del pianeta, ciascuno a suo
		modo.}}

{-- \emph{Ma non dovrebbero comportarsi come bambini? Non sono normali.
		Agiscono come\ldots{} personaggi storici. Napoleone e Wellington. Cesare
		e Bruto.}}

{-- \emph{Noi dobbiamo occuparci del destino del mondo, non di curare i
		cuori infranti. Lei è troppo compassionevole.}}

{-- \emph{Il generale Levy non aveva compassione per nessuno. Tutti i
		filmati ce lo confermano. Ma non faccia del male a questo ragazzino.}}

{-- \emph{Sta scherzando?}}

{-- \emph{Voglio dire, non gli faccia più male di quanto è necessario.}}

{~}

\begin{center}
	{* * *}
\end{center}

{~}

{A cena, Alai andò a sedersi di fronte a Ender. -- Finalmente ho capito
	come hai mandato quel messaggio. Quello firmato Bernard.}

{-- Io? -- si schermì Ender.}

{-- Avanti, e chi altro? Bernard non è stato di certo. E Shen non è un
	genio col computer. E io non l'ho fatto. Chi resta? Non importa. Ho
	capito che hai iscritto uno studente nuovo. Non hai fatto che aggiungere
	all'elenco un ragazzo di nome Bernard Zero-Zero, BERNARD-spento, in modo
	che il computer non possa né tenerlo presente nei programmi, né
	eliminarlo come un errore.}

{-- Sembra un'ipotesi che può funzionare -- disse Ender.}

{-- Sicuro che \emph{funziona.} Ma tu l'hai fatto praticamente il giorno
	del nostro arrivo.}

{-- Io o qualcun altro. Forse è stato Dap, per impedire a Bernard di
	diventare capogruppo.}

{-- Ho scoperto anche un'altra cosa. Non posso fare lo stesso con il tuo
	nome.}

{-- Ah, sì?}

{-- Qualsiasi messaggio con la parola \emph{Ender} viene cancellato
	appena scritto. E non sono neanche riuscito a farmi mandare sullo
	schermo il tuo fascicolo personale. Tu hai inserito un sistema di
	sicurezza.}

{-- Forse.}

{Alai sogghignò. -- Mettere le mani sui dati e sulle registrazioni
	altrui è fin troppo facile. E conosco altri che ci riescono. Io ho
	bisogno di proteggermi, Ender. Ho bisogno del tuo sistema.}

{-- Se ti do il mio sistema saprai come metterlo in atto, e saprai come
	ottenere e manipolare tutti i dati che riguardano me.}

{-- Vuoi dire io? -- finse di scandalizzarsi Alai. -- Il migliore amico
	che tu abbia qui dentro?}

{Ender rise. -- Studierò un altro sistema per te.}

{-- Adesso?}

{-- Se mi lasci finire di mangiare.}

{-- Tu non lo finisci mai quel vassoio.}

{Era vero. Dopo ogni pasto, sul vassoio di Ender avanzava sempre un po'
	di cibo. Lui guardò il piatto e decise d'essere già sazio. -- Va bene,
	andiamo.}

{Quando furono in camerata, Ender si gettò a sedere sulla sua cuccetta e
	disse: -- Stacca il tuo banco e portalo qui. Ti farò vedere cosa devi
	fare. -- Ma quando Alai fece ritorno con la sua scrivania elettronica
	Ender era sempre seduto nello stesso posto, e il suo armadietto era
	ancora chiuso.}

{-- Che c'è? -- domandò Alai.}

{Come tutta risposta Ender poggiò una mano sullo scanner
	dell'armadietto. Comparve una scritta: «Tentativo d'accesso non
	autorizzato». E lo sportello non si aprì.}

{-- Qualcuno ha imparato a ciurlarti nel manico, piccolo -- disse Alai.
	-- Qualcuno ti ha dato una fregatura.}

{-- Sei sicuro di volere ancora il mio sistema di sicurezza? -- brontolò
	lui. Si alzò dal letto e uscì in corridoio.}

{-- Ender -- lo chiamò l'altro.}

{Lui si volse. Alai gli stava mostrando un cartoncino rettangolare.}

{-- Che cos'è?}

{Alai si strinse nelle spalle. -- Non lo sai? Era sul tuo letto. Forse
	ci stavi seduto sopra.}

{Ender prese il cartoncino e lo lesse.}

{~}

\begin{center}
	{ENDER WIGGIN}

{ASSEGNATO ALL'ORDA DELLE SALAMANDRE}

{COMANDANTE BONZO MADRID}

{DECORRENZA IMMEDIATA}

{CODICE VERDE VERDE MARRONE}

{Gli oggetti personali}

{non saranno trasferiti}
\end{center}

{~}

{-- Tu sei in gamba, Ender, ma in sala di battaglia non sei affatto
	migliore di me.}

{Lui scosse il capo. Vedersi dare una promozione era la cosa più assurda
	che mai avrebbe potuto pensare. Nessuno veniva promosso prima di aver
	compiuto otto anni. Ender non ne aveva ancora sette. E di solito i
	novellini erano trasferiti in gruppi alle orde, molte delle quali
	aumentavano così gli effettivi contemporaneamente. Ma non c'erano ordini
	di trasferimento su nessuno degli altri letti.}

{Proprio quando le cose si stavano mettendo bene. Proprio quando Bernard
	cominciava a diventare sopportabile per tutti, perfino per lui. Proprio
	quando Alai si stava rivelando un vero amico. Proprio quando la sua vita
	diventava finalmente facile da vivere.}

{Ender fece scostare Alai dalla cuccetta, ma non si mise a sedere al suo
	posto.}

{-- L'orda delle Salamandre è in sala di battaglia, comunque -- disse
	Alai.}

{Ender era così infuriato per quel trasferimento così inopportuno che
	gli stavano salendo le lacrime agli occhi. \emph{Non devi piangere},
	\emph{} si disse.}

{Alai notò le sue palpebre inumidite, ma ebbe il tatto di ignorarle. --
	Sono delle teste di cavolo, Ender. Arrivano perfino al punto di non
	lasciarti portare via le tue cose.}

{Lui riuscì a trovare un sorriso che scacciò le lacrime. -- Dici che
	devo lasciare qui la tuta e andarmene nudo come un verme?}

{Anche Alai rise.}

{D'impulso Ender lo abbracciò strettamente, quasi come se fosse
	Valentine. E l'improvviso desiderio di rivederla gli fece desiderare
	d'essere a casa. -- Non voglio andarmene -- disse.}

{Alai gli restituì l'abbraccio. -- Io li capisco, Ender. Tu sei il
	migliore di noi. Forse hanno fretta d'insegnarti tutto il possibile.}

{-- Non so cosa vogliano insegnarmi -- mormorò Ender. -- So soltanto che
	volevo sapere cosa significa avere un amico.}

{Alai annuì gravemente. -- Amici una volta, amici per sempre --
	dichiarò. Poi sorrise. -- Vai a fare a fettine gli Scorpioni,
	d'accordo?}

{-- Sicuro. -- Ender gli restituì il sorriso.}

{Ad un tratto Alai lo baciò su una guancia, e mormorò: -- Salaam! --
	Poi, rosso in volto, si volse e tornò alla sua cuccetta in fondo al
	locale. Ender si disse che quel bacio e quella parola dovevano essere
	qualcosa di proibito. Una delle religioni soppresse, forse. Oppure la
	parola doveva contenere qualche segreto e potente significato per Alai.
	Ma qualunque cosa avesse inteso, Ender sapeva che l'amico la riteneva
	sacra e che gli aveva rivelato il suo animo, così come gli era accaduto
	una sera con sua madre, prima che gli mettessero il monitor nella nuca,
	quando credendolo addormentato s'era seduta sul bordo del suo letto e
	gli aveva poggiato le mani sulla testa, pregando sottovoce per lui.
	Ender non ne aveva mai fatto parola con nessuno, neppure con lei, ma ne
	aveva conservato un ricordo profumato di mistero sacro, la
	consapevolezza che sua madre lo amava così profondamente da non osare
	dirlo se non ai suoi occhi addormentati. Questo era ciò che Alai gli
	aveva dato; un dono così sacro che neppure a lui era concesso
	comprenderne il significato.}

{Dopo una cosa simile null'altro poteva essere detto. Alai si gettò
	sulla cuccetta e volse su di lui uno sguardo pacato. I loro occhi
	s'incontrarono come quelli di due fratelli. Poi Ender uscì.}

{Nessun sentiero verde verde marrone lo attendeva in quella zona della
	Scuola; avrebbe dovuto cercare i colori in uno dei locali più
	frequentati. Ma gli altri sarebbero usciti di mensa da lì a pochi
	minuti, e lui non se la sentiva d'incontrarli. La sala dei giochi invece
	doveva essere praticamente deserta.}

{Nell'umore in cui era, nessun gioco gli parve più molto attraente; così
	andò allo schermo di una delle scrivanie pubbliche in fondo al locale e
	lo accese, chiedendo la sua partita personale. Subito fece correre la
	sua figura fino alla Terra delle Meraviglie. Adesso, ogni volta che
	giungeva lì, il Gigante era un cadavere. Per scendere dal tavolo dovette
	saltare dapprima su una gamba dell'enorme sedia rovesciata, quindi si
	calò cautamente al suolo. Per un po' c'erano stati dei topi, occupati a
	rosicchiare il corpo del Gigante, ma Ender ne aveva ucciso uno con uno
	spillo tolto dall'abito sgualcito del colosso, e da allora lo avevano
	lasciato in pace.}

{Il corpo del Gigante era pressoché ai limiti della decomposizione. Ciò
	che poteva esser mangiato via dagli animali necrofori era consumato; i
	vermi avevano compiuto il loro lavoro negli organi interni; adesso non
	restava che una mummia disseccata dalle orbite vuote, coi denti scoperti
	in un sogghigno scheletrico e le dita come artigli ricurvi. Ender
	ripensò alla ferocia con cui gli aveva aggredito l'occhio quando era
	vivo, malizioso e intelligente. Irritato e frustrato come si sentiva,
	desiderò poterlo di nuovo attaccare e uccidere. Ma ormai il Gigante era
	divenuto parte di quel panorama, e odiarlo non aveva più alcun senso.}

{Ender era già stato oltre il ponte al castello della Regina di Cuori,
	dove c'erano da giocare partite abbastanza divertenti, ma in quel
	momento nessuna di esse lo attirava. Aggirò il cadavere del Gigante e
	seguì il ruscello controcorrente, fino al punto in cui emergeva dalla
	foresta. Là c'era un tipico parco giochi, con i toboga e le altalene, la
	pista di pattinaggio e alcune giostre, e dozzine di bambini stavano
	cicalando e ridendo. Ender si avvicinò e s'accorse che la sua figura
	aveva perso certe caratteristiche adulte trasformandosi in quella di un
	bambino. Anzi era ancor più piccola e giovane degli altri ragazzetti.}

{Si mise in fila per il toboga. Gli altri bambini lo ignorarono. Salì la
	scaletta fino in cima e attese che quello davanti a lui si fosse gettato
	giù lungo la liscia spirale che terminava al suolo. Poi sedette e si
	spinse in avanti.}

{Non stava scivolando neppure da un istante quando si trovò ad atterrare
	nella sabbia sotto l'incastellatura. Il toboga non lo voleva su di sé.}

{Anche le altalene rifiutavano la sua presenza. Poteva sedersi e
	cominciare a muoversi, ma appena l'oscillazione aumentava il sedile
	diventava incorporeo e lui cadeva. Il ponticello sullo stagno lo lasciò
	precipitare in acqua mentre attraversava. Provò una delle giostre, che
	partì normalmente; quando però essa cominciò a girare forte e Ender
	cercò di aggrapparsi le maniglie si smaterializzarono e la forza
	centrifuga lo scaraventò al suolo.}

{E gli altri bambini: le loro risate erano rauche, offensive. Fecero
	circolo intorno a lui, gli rivolsero gesti derisori e prima di tornare
	ai loro giochi lo insultarono beffardamente.}

{Ender provò l'impulso di colpirli, di afferrarli e gettarli nel
	ruscello. Invece si inoltrò nella foresta. Trovò un sentiero, che poco
	dopo si allargò in un'antica strada lastricata in pietra, aggredita
	dalle erbacce ma ancora praticabile. Su ambo i lati c'erano possibili
	buone partite da giocare, ma Ender non s'impegnò in alcuna di esse.
	Voleva vedere dove portava la strada.}

{Ciò che si trovò davanti fu una radura con un vecchio pozzo al centro,
	e su di esso un cartello che diceva: «Dissetati, viandante». Ender andò
	a guardare nel pozzo. In quell'istante udì un ringhio. Dalla foresta
	erano sbucati una dozzina di lupi avidi di sangue, ed avevano volti
	umani. Ender li riconobbe: erano i bambini che l'avevano deriso. Ma
	adesso avevano zanne fatte per sbranare, e senza un'arma con cui opporsi
	Ender fu subito sopraffatto e divorato.}

{La sua figura successiva apparve, come di regola, nello stesso luogo, e
	fu di nuovo fatta a pezzi dai lupi, benché Ender avesse tentato di
	gettarsi nel pozzo.}

{Nella partita che seguì venne riportato indietro nel parco giochi. I
	bambini stavano ridendo intorno a lui. \emph{Ridete pure finché volete},
	\emph{} pensò Ender. \emph{Ora so chi siete.} Agguantò una di loro. Lei
	lo seguì, irosamente, fino al toboga e si lasciò spingere in cima alla
	scaletta. Poi Ender si gettò giù con lei. Come in precedenza si ritrovò
	di colpo al suolo, ma anche la bambina era precipitata insieme a lui e
	al momento dell'impatto s'era trasformata in un lupo, che adesso giaceva
	stordito o morto sulla sabbia.}

{Uno dopo l'altro Ender trascinò i piccoli licantropi in quella
	trappola. Ma prima che avesse finito di eliminare l'ultimo i lupi
	ripresero vita, e non si mutarono in bambini. Ender fu sbranato
	nuovamente.}

{Questa volta, scosso e sudato, ritrovò la sua figura in piedi sul
	tavolo del Gigante. \emph{Potrei anche averne abbastanza}, \emph{} si
	disse. \emph{E dovrei presentarmi al comandante dell'orda.}}

{Ma invece fece scendere la figura sulla sedia e al suolo, aggirò il
	corpo del Gigante e si diresse al parco giochi.}

{Stavolta, non appena i bambini si mutarono in lupi sotto il toboga,
	Ender li trascinò via e li gettò nel ruscello. A ogni tuffo i corpi
	sfrigolavano come se l'acqua fosse acido. I lupi furono distrutti, e una
	grossa nuvola di fumo scuro fluttuò via dalla zona. Nello stesso modo
	dovette disfarsi di altri bambini, che avevano preso a inseguirlo verso
	l'antica strada. Nella radura non trovò lupi in agguato, cosicché entrò
	nel secchio del pozzo e usando la carrucola si calò fino in fondo.}

{Nella caverna aleggiava una penombra rosata nella quale sfavillavano
	mucchi di gioielli. Passò oltre, e notò che alle sue spalle degli occhi
	balenavano fra le gemme. Una tavola coperta di cibarie non destò il suo
	interesse. S'inoltrò fra numerose gabbie, appese al soffitto della
	grotta, ognuna contenente creature strane dall'aria abbastanza
	amichevole. \emph{Giocherò con voi più tardi}, \emph{} pensò Ender. Sul
	fondo si trovò davanti a una porta che recava inciso, in lettere verdi e
	scintillanti:}

{~}

\begin{center}
	{LA FINE DEL MONDO}
\end{center}

{~}

{Senza pensarci sopra spinse il battente e passò oltre.}

{Dovette fermarsi subito. Si trovava su uno stretto cornicione roccioso,
	alto sulla parete di un burrone, di fronte a un immenso panorama di
	boschi su cui stagnavano i colori dell'autunno, qua e là chiazzato
	dall'ocra scuro dei campi ormai mietuti. C'erano stradicciole, carri
	trainati da buoi, piccoli villaggi sonnolenti, e un castello che in
	distanza si stagliava contro il cielo, così alto che le nuvole
	s'infrangevano nei picchi rocciosi alla base delle sue mura. Alzò gli
	occhi e vide che il cielo era il soffitto di un'immensa caverna, dove
	nidi di cristalli luccicavano fra le stalattiti.}

{Dietro di lui la porta si chiuse. Ender studiò quello scenario con
	meraviglia. Era così bello che la sua perenne attenzione contro il
	pericolo si rilassò. Al momento gli importava poco di quali partite si
	potessero giocare in quel posto. L'aveva scoperto lui, e contemplarlo
	era il suo premio. Così, senza nessun timore per le conseguenze, saltò
	giù dal cornicione.}

{La mossa lo mandò a precipitare in picchiata verso le rapide
	spumeggianti di un torrente, fra cui si levavano rocce acuminate, ma una
	nuvola avanzò a interporsi fra lui e il disastro, lo raccolse e lo portò
	via. Quel singolare tappeto volante lo condusse fino alla terre del
	castello, e quindi direttamente dentro una delle finestre che vi si
	aprivano. Fu deposto al suolo in una stanza di pietra, priva di porte e
	senza botole sul soffitto o sul pavimento. L'unica uscita era la
	finestra, che offriva soltanto una mortale caduta da grande altezza.}

{Pochi momenti prima s'era tuffato in un burrone con cieca incoscienza,
	ma stavolta esitò.}

{Quello che era parso un pezzo di legno davanti al caminetto si svolse
	dalle spire, rivelandosi per un lungo serpente i cui denti scintillavano
	di veleno.}

{-- L'unica uscita dalla stanza sono io -- disse. -- La morte è la tua
	sola via di fuga.}

{Ender si stava guardando attorno in cerca di un'arma, quando
	all'improvviso lo schermo diventò nero. Su di esso lampeggiò una
	scritta:}

{~}

\begin{center}
	{SUBITO A RAPPORTO DAL COMANDANTE}

{SEI IN RITARDO}

{VERDE VERDE MARRONE}
\end{center}

{~}

{Seccato, Ender spense la scrivania, andò agli indicatori colorati
	accanto alla porta e premette la striscia verde verde marrone, poi seguì
	il sentiero che s'era acceso davanti a lui. Il verde chiaro, il verde
	smeraldo e il marrone terroso del nastro gli ricordarono l'autunno del
	regno che aveva appena scoperto. \emph{Devo ritornarci}, \emph{} disse a
	se stesso. \emph{Quel lungo serpente è come una corda, posso usarlo per
		calarmi dalla torre e trovare la soluzione di quel posto. Forse si
		chiama la fine del mondo perché è la fine della partita, perché io
		potrei entrare in uno di quei villaggi e diventare uno dei ragazzini che
		lavorano e giocano laggiù, senza nulla che mi possa uccidere e senza
		nulla da uccidere, soltanto per vivere là.}}

{Ma a quel pensiero non fu capace di immaginare cosa poteva significare
	per lui «soltanto vivere». Era un'esperienza che non gli sembrava di
	aver mai fatto prima. Comunque fosse, desiderava farla.}

{~}

\begin{center}
	{* * *}
\end{center}

{~}

{Le orde erano più numerose dei gruppi dei nuovi arrivati, e le camerate
	in cui risiedevano erano molto più grandi. Quella era di larghezza
	normale, ma così lunga che si poteva vedere la lieve curvatura verso
	l'alto del pavimento, il quale seguiva la circonferenza esterna della
	Scuola di Guerra.}

{Ender si fermò all'ingresso. Alcuni ragazzi al di là della porta gli
	gettarono un'occhiata, ma erano alquanto più grandi di lui e parve che i
	loro sguardi lo trapassassero senza vederlo. Proseguirono nella
	conversazione, in piedi o seduti sulle loro cuccette. Stavano discutendo
	di qualche battaglia, ovviamente. I ragazzi più anziani non parlavano di
	sciocchezze. Ed erano molto più alti di lui: quelli di dieci o undici
	anni lo sovrastavano, e lo stesso si poteva dire per i più giovani,
	quelli di otto anni. Ender non era certo alto per la sua età.}

{Cercò di capire chi di loro fosse il comandante, ma quasi tutti erano
	seminascosti oltre i letti a castello, alle prese con le loro tute da
	battaglia e con quelle che i soldati chiamavano «uniformi da notte»,
	calzamaglie che coprivano dalla testa ai piedi. Molti di essi avevano
	tirato fuori il loro banco, ma pochi erano occupati a studiare.}

{Ender fece un passo avanti. E nell'istante in cui oltrepassò la porta
	una mano si alzò a dargli l'alt.}

{-- Cosa cerchi? -- chiese il ragazzo che aveva la cuccetta superiore
	accanto all'ingresso. Era il più alto di tutti. Ender lo aveva già
	notato alla mensa: un giovane gigante con già qualche rado peluzzo sul
	mento. -- Tu non sei una salamandra, pivello.}

{-- Dovrei esserlo, invece, credo -- disse Ender. -- Verde verde
	marrone, giusto? Sono stato trasferito. -- Intuendo che il ragazzo aveva
	mansioni di guardia alla porta, gli mostrò il cartoncino.}

{La guardia allungò una mano. Ender lo ritrasse, appena fuori portata.
	-- Credo di doverlo consegnare a Bonzo Madrid.}

{Alla conversazione si unì un altro ragazzino, di statura inferiore agli
	altri ma sempre più alto di Ender. -- Non bahn-zoe, testa di rapa:
	Bon-zo. È un nome spagnolo. Bonzo Madrid. Aqui nosotros hablamos
	español, Señor Gran Fedor.}

{-- Bonzo sei tu, allora? -- chiese Ender, pronunciando correttamente il
	nome.}

{-- No. Io sono una poliglotta di luminoso talento. Petra Arkanian.
	L'unica femmina dell'orda delle Salamandre. Ma con più palle che
	chiunque altro in questa stanza.}

{-- Ha parlato Petra, la bocca di pietra -- esclamò un ragazzo. --
	Udite, udite, tutti voi!}

{Un altro ridacchiò. -- Petra, bocca di pietra, bocca di merda, parla di
	merda!}

{Soltanto pochi risero.}

{-- Resti fra me e te, ragazzo -- disse Petra, -- ma se dovessero fare
	un clistere alla Scuola di Guerra ficcherebbero la cannuccia nel verde
	verde marrone.}

{L'umore di Ender peggiorò. Aveva già parecchi svantaggi a suo carico:
	un addestramento scarsissimo, la giovane età, l'inesperienza, il rancore
	che altri avrebbero provato per la sua precoce promozione. E adesso, per
	soprammercato, si stava facendo la più sbagliata delle amicizie, una
	sorta di paria fra le Salamandre, la quale aveva visto in lui un altro
	possibile disadattato con cui fare coppia contro il resto dell'orda.
	\emph{Davvero una bella giornata di lavoro}, \emph{} pensò. Per un
	attimo, mentre girava lo sguardo su quei volti ironici e sogghignanti,
	gli parve di vederli coprirsi di peli fra cui biancheggiavano zanne
	pronte a mordere. \emph{Sono io l'unico essere umano qui dentro? Questi
		sembrano animali capaci soltanto di azzannare il prossimo.}}

{Poi ripensò ad Alai. In ogni orda, sicuramente, c'era almeno qualcuno
	che valeva la pena di conoscere.}

{In quel momento, benché nessuno l'avesse ordinato, le risate tacquero e
	nella camerata cadde il silenzio. Ender si volse alla porta. Sulla
	soglia c'era un ragazzo alto e snello, di pelle olivastra, con due
	splendidi occhi neri e labbra su cui aleggiava un sorrisetto
	sofisticato. \emph{Questo ragazzo ha del fascino}, \emph{} disse
	qualcosa in fondo alla mente di Ender. \emph{Vorrei vedere nel modo in
		cui i suoi occhi vedono.}}

{-- Chi sei? -- domandò il ragazzo a bassa voce.}

{-- Ender Wiggin, signore -- disse lui. -- Trasferito dal mio gruppo
	all'orda delle Salamandre. -- Gli porse il cartoncino.}

{Il ragazzo lo prese con un movimento fluido e sicuro, senza sfiorargli
	le dita. -- Quanti anni hai, Wiggin? -- chiese.}

{-- Quasi sette.}

{Sempre a bassa voce l'altro osservò: -- Ti ho chiesto quanti anni hai,
	non quanti non ne hai ancora.}

{-- Ho sei anni, nove mesi e dodici giorni.}

{-- Quanto hai lavorato in sala di battaglia?}

{-- Pochi mesi soltanto. Aspiro a migliorare.}

{-- Addestramento in manovre belliche? Hai mai fatto parte di un branco?
	Sei mai stato inserito in azioni di gruppo?}

{Ender non aveva neppure sentito parlare di cose simili. Scosse il
	capo.}

{Madrid lo guardò negli occhi. -- Capisco. Come avrai modo d'imparare
	presto, gli ufficiali in comando alla Scuola, e particolarmente il
	maggiore Anderson che sovrintende alle gare, appezzano l'arte di dare
	colpi bassi all'avversario. L'orda delle Salamandre si appresta ad
	emergere da un'indecorosa oscurità. Abbiamo vinto dodici delle nostre
	ultime venti gare. Abbiamo sorpreso i Topi, le Api e i Levrieri, e siamo
	pronti a batterci per ottenere la posizione di prestigio. Di
	conseguenza, ovviamente, mi è stato assegnato un peso morto, un elemento
	inutilizzabile e senza alcun addestramento, dal fisico sottosviluppato.
	Tu.}

{-- Neppure lui è entusiasta di conoscerti -- disse Petra con calma.}

{-- Taci, Arkanian -- disse Madrid. -- Alle nostre difficoltà ora se ne
	aggiunge un'altra. Ma qualunque ostacolo gli ufficiali vogliano sbattere
	sul nostro cammino, noi siamo ora e sempre\ldots{}}

{-- Le Salamandre! -- gridarono i soldati come un sol uomo.}

{D'istinto, la percezione che Ender aveva della camerata cambiò. Quello
	era uno schema di comportamento, un rituale. Madrid non stava cercando
	di ferire lui, bensì di prendere sotto controllo un avvenimento
	imprevisto e usarlo per rafforzare la sua autorità sull'orda.}

{-- Noi siamo il fuoco che li brucerà dalla testa ai piedi. Noi siamo
	cervello e cuore, molte fiamme, un solo fuoco.}

{-- Le Salamandre! -- urlarono gli altri.}

{-- Neppure questo pivello ci indebolirà.}

{Per un momento Ender si concesse un palpito speranzoso. -- Lavorerò
	sodo e imparerò in fretta -- disse.}

{-- Non ti ho dato il permesso di parlare -- disse Madrid. -- Ho
	intenzione di venderti al più presto possibile. Probabilmente sarò
	costretto a dar via insieme a te un elemento valido, ma piccolo come sei
	risultati peggio che inutile per me. Un congelato in più da sobbarcarsi
	durante ogni battaglia, ecco quello che sei inevitabilmente. E al punto
	in cui siamo ora, ogni soldato congelato può costituire la differenza
	decisiva per la sopravvivenza di una postazione. Niente di personale,
	Wiggin, ma sono certo che potrai fare il tuo addestramento a spese di
	qualcun altro.}

{-- Abbiamo un comandante tutto cuore, come vedi -- disse Petra.}

{Madrid fece un passo verso di lei e le sferrò un rapido manrovescio. Il
	rumore fu appena udibile, perché la colpì soltanto con le unghie. Ma
	lasciò sulla guancia di lei quattro strisce rosse, e quattro piccole
	ferite sanguinanti dove le unghie avevano colpito.}

{-- Queste sono le tue istruzioni, Wiggin. Voglio sperare che questa sia
	l'ultima volta che dovrò perder tempo a parlare con te. Quando ci
	alleneremo in sala di battaglia tu starai fuori dai piedi. Dovrai far
	atto di presenza, naturalmente, ma non apparterrai a nessun branco e non
	prenderai parte a nessuna manovra. Quando saremo chiamati a combattere,
	ti vestirai in fretta e ti presenterai alla porta come ogni altro. Ma
	non oltrepasserai la porta finché la battaglia non sarà cominciata da
	quattro minuti esatti, quindi resterai accanto all'uscita senza mai
	estrarre la pistola, in attesa che il tempo di gara sia scaduto.}

{Ender annuì. Dunque stava per diventare un niente. Sperò che lo
	vendessero a qualcun altro il più presto possibile.}

{Intanto aveva notato che Petra non aveva aperto bocca né battuto ciglio
	a quel ceffone, e neppure aveva alzato una mano a tastarsi la guancia,
	benché una striscia di sangue le scivolasse verso il mento. Bonzo Madrid
	s'era rivelato definitivamente ostile, ma in quanto alla ragazzina,
	paria o non paria che fosse in quell'orda, Ender sentì che avrebbe
	potuto diventarle amico.}

{Gli fu assegnata una cuccetta nell'angolo più lontano della camerata.
	Quella superiore, cosicché quando vi si distese scoprì di non riuscire
	neppure a vedere la porta: la curvatura del soffitto gli bloccava la
	visuale. Nelle sue vicinanze c'erano altri ragazzini, silenziosi e
	dall'aria triste e stanca, evidentemente gli ultimi nella valutazione
	del comandante. Nessuno di loro ebbe una parola di benvenuto da
	regalargli.}

{Ender poggiò una mano sullo scanner di un armadietto per aprirlo, ma
	non accadde niente. Soltanto allora si accorse che non c'erano
	serrature. I quattro stipi avevano una maniglia a forma di anello e
	basta. Nulla sarebbe dunque stato privato e personale, adesso che faceva
	parte di un'orda.}

{Nell'armadietto alto c'era una tuta. Non quella azzurro pallido dei
	nuovi arrivati, bensì l'uniforme verde scuro bordata di arancione
	dell'orda delle Salamandre. Notò che gli sarebbe andata larga.
	Probabilmente il magazzino non aveva mai dovuto fornire un'uniforme del
	genere a un ragazzo così giovane.}

{La stava tirando fuori quando si accorse che Petra veniva verso di lui,
	lungo il passaggio centrale. Scivolò giù dal letto e la attese in piedi
	accanto al montante metallico.}

{-- Riposo -- disse lei. -- Io non sono un ufficiale.}

{-- Sei un capobranco, non è così?}

{Qualcuno nelle vicinanze fece udire una risatina.}

{-- Cosa ti ha fatto venire quest'idea, Wiggin?}

{-- La tua cuccetta è vicino alla porta.}

{-- Mi è stata assegnata perché sono la miglior tiratrice dell'orda
	delle Salamandre, e perché Bonzo teme che se i capibranco non mi tengono
	sott'occhio io possa mettere in piedi una rivolta. Come se potessi
	combinare qualcosa di buono con elementi come \emph{questi.} -- Indicò i
	ragazzi dall'aria depressa sulle cuccette vicine.}

{Cosa stava cercando? Forse di rendergli le cose peggiori di quel che
	già erano? -- Sono tutti migliori di me -- disse Ender, per chiarire che
	si dissociava dal disprezzo di lei verso quei ragazzi, i quali dopotutto
	erano i suoi vicini di letto.}

{-- Io sono una femmina -- disse lei, -- e tu sei un piscione di sei
	anni. Dunque abbiamo qualcosa in comune. Perché non essere amici?}

{-- Guarda che non farò i tuoi compiti di scuola -- disse lui.}

{Lei capì all'istante che stava scherzando. -- Hu-hu -- annuì. -- Ma
	quando sei nelle gare, tutto è molto militaresco. La Scuola non è così
	per i nuovi arrivati. Storia e strategia e tattica e Scorpioni e
	compagni e stelle, queste sono le cose di cui hai bisogno per diventare
	un pilota o un comandante. Vedrai.}

{-- Così sei mia amica. Che ci guadagno? -- chiese Ender. Stava imitando
	l'eloquio di lei, fra sfrontato e indifferente.}

{-- Bonzo non ha intenzione di addestrarti. Ciò che farà è di ordinarti
	di portare il tuo banco anche in sala di battaglia, perché tu studi
	anche là. In un certo senso ha ragione\ldots{} non vuole che un
	marmocchio ignorante rovini la precisa meccanica delle sue manovre. --
	La sua voce si deformò nell'imitazione della parlata di chi non
	conosceva né l'inglese né l'interlingua: -- Bonzo, lui così
	pre-\emph{cizo.} Lui così \emph{curato.} Lui piscia dentro piatto senza
	che una goccia va fuori!}

{Ender sogghignò.}

{-- La sala di battaglia è aperta a orario continuato. Se ti va,
	potremmo andarci nelle ore in cui non c'è nessuno e ti insegnerò quello
	che so. Io non sono un gran soldato, però sono in gamba, e conosco un
	bel po' di cose che tu non sai.}

{-- Se hai tempo -- annuì Ender.}

{-- Domattina dopo colazione, allora.}

{-- E se qualcun altro sta usando la sala? Il mio gruppo ci andava
	sempre, dopo colazione.}

{-- Nessun problema. Le sale di battaglia sono sette.}

{-- Non mi avevano detto dell'esistenza delle altre.}

{-- Il locale d'ingresso è unico per tutte. Le sale di battaglia si
	trovano nel centro esatto della Scuola, al mozzo della ruota. E non
	ruotano con il resto della stazione. Ecco perché l'assenza di peso, lo
	zero-G, è totale. Niente impulso centrifugo, niente alto e basso. Le
	sette sale sono piazzate intorno al mozzo, che è il corridoio d'ingresso
	comune. Una volta lì dentro lo fanno girare, così alla porta ti si
	presenta la sala che desideri.}

{-- Ah!}

{-- Domani al termine della colazione, come ho detto.}

{-- D'accordo -- rispose Ender.}

{Lei cominciò ad allontanarsi.}

{-- Petra -- la fermò.}

{La ragazzina si volse a guardarlo.}

{-- Grazie.}

{Lei non disse nulla. Ebbe appena un cenno del capo e poi se ne andò a
	passi svelti.}

{Ender risalì sulla cuccetta e si tolse la tuta, poi giacque nudo sul
	materasso con il banco elettronico girato davanti a sé, riflettendo
	sulla possibilità che avessero messo le mani sui suoi codici d'accesso.
	Era quasi certo che il suo sistema di sicurezza fosse stato tolto di
	mezzo. Non poteva possedere niente lì, neppure il suo banco.}

{Le luci si abbassarono leggermente. Era quasi l'ora di dormire. Ender
	domandò dove fossero i gabinetti.}

{-- Esci e vai a sinistra -- disse il ragazzo della cuccetta accanto. --
	Li abbiamo in comuni coi Topi, i Condor e gli Scoiattoli.}

{Ender lo ringraziò e fece per avviarsi.}

{-- Ehi -- lo richiamò l'altro. -- Non puoi uscire a quel modo. Fuori
	dalla camerata l'uniforme è obbligatoria.}

{-- Anche per andare ai gabinetti?}

{-- Soprattutto questo. E non puoi neppure rivolgere la parola ai membri
	di un'altra orda. Né a mensa né ai gabinetti. A volte si può farlo in
	sala giochi, e naturalmente quando un insegnante te lo chiede. Ma se ti
	pesca Bonzo sei morto, capito?}

{-- Grazie.}

{-- E un'altra cosa: Bonzo ti mangia vivo se ti scopre a\ldots{} fare
	giochetti con Petra.}

{-- Eppure era nuda quando sono entrato, no?}

{-- Lei fa quel che vuole, ma tu devi vestirti. Ordini di Bonzo.}

{Era una stupidaggine. Petra aveva ancora l'aspetto di un ragazzino,
	perciò l'ordine era assurdo. \emph{Questo la isola, la rende diversa,
		divide l'orda. Stupido, stupido.} Come aveva fatto Bonzo a diventare
	comandante se non riusciva a pensarne una migliore? \emph{Alai sarebbe
		un comandante più capace di Bonzo. Lui sa come tenere unito un gruppo.}}

\emph{{E anch'io so come unire la gente in un gruppo}}{, \emph{}
	continuò a pensare Ender. \emph{Forse sarò comandante, un giorno o
		l'altro.}}

{Era nelle docce che si lavava le mani quando qualcuno gli rivolse la
	parola. -- Ehi, tu, non mi dire che adesso le Salamandre allevano
	poppanti!}

{Ender non rispose e andò ad asciugarsi le mani.}

{-- Ehi, guardate un po'! Le Salamandre arruolano anatroccoli. Quello
	potrebbe passarmi fra le gambe senza toccarmi le palle!}

{-- Questo è perché non le hai, Dink, ecco perché -- ridacchiò un
	altro.}

{Mentre Ender usciva dal locale sentì una terza voce dire: -- Lui è
	Wiggin. Quello che ha stracciato Waldrop in sala giochi, ricordi?}

{Allontanandosi lungo il corridoio s'accorse di sorridere. \emph{Lui è
		piccolo, certo, ma loro ricordano il suo nome. In sala giochi,
		naturalmente, perciò non significa nulla. Ma lo vedranno. Diventerà un
		buon soldato, anche. Presto tutti conosceranno il suo nome. Non
		nell'orda delle Salamandre, forse, ma abbastanza presto.}}

{~}

\begin{center}
	{* * *}
\end{center}

{~}

{Petra era già in attesa nel corridoio che portava alla sala di
	battaglia. -- Aspettiamo qui -- disse a Ender. -- L'orda delle Lepri sta
	arrivando proprio ora, e occorre qualche minuto per girare la porta
	sulla sala di battaglia successiva.}

{Ender sedette accanto a lei, per terra. -- C'è dell'altro circa le sale
	di battaglia, oltre a questo fatto di passare da una a quella che segue
	-- disse. -- Ad esempio, perché qui nel corridoio c'è la gravità, mentre
	appena oltre quella porta si va subito a zero-G?}

{Petra chiuse gli occhi. -- E se le sale di battaglia sono davvero
	isolate dal resto della stazione, cosa succede quando una viene
	collegata alla porta? Perché non comincia a muoversi secondo la
	rotazione della Scuola?}

{Ender annuì gravemente.}

{-- Questi sono i grandi misteri -- disse Petra in un drammatico
	sussurro. -- Non cercare di svelarli. Cose terribili accaddero
	all'ultimo soldato che osò ficcarci il naso. Fu ritrovato appeso per i
	piedi al soffitto del gabinetto, con la testa infilata nella tazza.}

{-- Allora non sono il primo che fa queste domande.}

{-- Una cosa devi tenere a mente, pivello. -- Detto da lei l'appellativo
	suonò amichevole, non più sprezzante. -- Loro non ti diranno mai più
	verità di quanto non siano costretti a fare. Ma perfino i bambini
	dell'asilo sanno che la scienza ha fatto grandi passi dai giorni del
	vecchio Mazer Rackham e della Flotta Vittoriosa. È ovvio che adesso
	possiamo controllare la gravità. Accenderla e spegnerla, cambiarne la
	direzione, forse rifletterla. .. ho pensato a un sacco di cose veramente
	forti che potresti fare, con armi antigravità e con motori
	gravitazionali sulle astronavi. E pensa a come potrebbero manovrare in
	vicinanza dei pianeti. Magari usando la gravità planetaria stessa per
	accelerare, oppure come energia per le apparecchiature. Ma loro non
	dicono niente.}

{Le riflessioni di Ender andavano più in là. Manipolare la gravità era
	una cosa basilare, ufficiali che tenevano segreti dei dati scientifici
	era una cosa grave, ma il messaggio che Petra gli stava inviando era
	questo: i nostri nemici sono gli adulti, non le altre orde. Loro non ci
	dicono la verità.}

{-- Avanti, pivello, la sala di battaglia è calda. La mano di Petra è
	salda. Davanti a noi il nemico si sfalda. -- Ridacchiò. -- Petra la
	poetessa, ecco come mi chiamano.}

{-- Dicono anche che sei matta come un cavallo.}

{-- E tu galoppa dietro di me, puledro -- esclamò lei, entrando nel
	vastissimo locale.}

{Ender la seguì. La ragazzina aveva un contenitore con dieci
	palle-bersaglio. Quando le tirò, ciascuna in una diversa direzione, lui
	si aggrappò alla ringhiera con una mano e la tenne ferma con l'altra,
	per impedirle di fluttuare via. In assenza di gravità le palle
	cominciarono a rimbalzare velocemente da tutte le parti.}

{-- Lasciami -- disse lei. Si diede una spinta, deliberatamente casuale;
	agitando un braccio si mise in assetto stabile, poi estrasse la pistola
	e la puntò su un bersaglio dopo l'altro. Quando colpiva una palla il suo
	colore da bianco diventava rosso. Ender sapeva che entro due minuti
	esatti i bersagli centrati sarebbero tornati al colore originale.
	Soltanto una delle palle era ridiventata bianca allorché Petra riuscì a
	colpire l'ultima.}

{La ragazzina eseguì un rimbalzo calcolato contro una parete e si spinse
	velocissima verso Ender. Lui ammortizzò il suo impatto e le impedì di
	rimbalzare ancora, una delle prime tecniche che aveva imparato col suo
	gruppo.}

{-- Sei brava -- le disse.}

{-- Nessuno è migliore di me. E tu stai per apprendere alcuni piccoli
	segreti del mestiere.}

{Come inizio Petra gli insegnò che il braccio armato andava tenuto
	dritto, per mirare con tutta la sua lunghezza. -- Una cosa che molti
	soldati non capiscono mai è che più il bersaglio è lontano, più a lungo
	devono tenervi puntato contro il raggio, perché pur ristretto esso si
	allarga a cono. La differenza in più è di pochi decimi di secondo, ma in
	una battaglia questo è un tempo lungo. Molti soldati credono di aver
	sbagliato mira dopo aver colpito il bersaglio, invece hanno solo
	distolto il raggio troppo presto. Così non puoi usare la tua pistola
	come una spada swish-swish-spaccali-in-due. Devi mirare colpo per
	colpo.}

{Premendo un pulsante richiamò le palle presso la porta, poi le rilanciò
	lentamente, una alla volta. Ender puntò e sparò. Le sbagliò tutte.}

{-- Benone -- disse lei. -- Vedo che non hai automatismi sbagliati.}

{-- Non ho neppure quelli buoni -- borbottò lui.}

{-- Quelli te li darò io.}

{Quella prima mattina non realizzarono molto. Per lo più parlarono: come
	puoi continuare a pensare mentre prendi la mira. La necessità di
	visualizzare il movimento dell'avversario e il tuo raffrontandoli
	incessantemente. Devi sempre tenere il braccio teso in avanti, imparando
	a mirare girando tutto il corpo, così se ti congelano riuscirai ancora a
	sparare. Calcola dove il grilletto scatta e tienilo sul filo di quel
	punto, così non sarai costretto a tirarlo del tutto se ti trovi davanti
	un nemico all'improvviso. Rilassati, impara a respirare, la tensione
	fisica causa errori di mira.}

{Fu il solo addestramento che Ender ebbe per quel giorno. Nel
	pomeriggio, durante le esercitazioni dell'orda, gli fu ordinato di
	portarsi dietro il banco e di fare i compiti di scuola seduto in un
	angolo della sala. Bonzo voleva l'orda al completo in sala di battaglia,
	ma non era tenuto a usare tutti i soldati.}

{Ender tuttavia lasciò perdere i compiti. Se non gli veniva dato
	l'addestramento militare, poteva approfittarne per studiare almeno le
	tattiche di Bonzo. L'orda delle Salamandre era divisa, come di regola,
	in quattro branchi di dieci soldati ciascuno. Alcuni comandanti li
	organizzavano in modo che il branco A fosse quello coi migliori
	combattenti, mentre nel branco D c'erano i peggiori. Bonzo li aveva
	mescolati, cosicché ognuno era composto di soldati abili e soldati
	scadenti.}

{Con la sola differenza che adesso il branco B aveva soltanto nove
	ragazzi. Ender si chiese chi mai fosse stato trasferito per lasciare il
	posto a lui. Presto gli fu chiaro che il capo del branco B era nuovo a
	quel compito. Nessuna meraviglia che Bonzo fosse così seccato: aveva
	perso un capobranco per vedersi arrivare Ender.}

{E Bonzo aveva ragione su un'altra cosa: Ender non era pronto. Tutto il
	tempo degli allenamenti era dedicato a lavorare sulle manovre. Branchi
	che non potevano vedersi l'un l'altro mettevano in atto operazioni
	coordinate con precisione cronometrica, o si regolavano sulla posizione
	altrui per effettuare imprevisti mutamenti direzionali senza scomporre
	la formazione. Da tutti questi soldati ci si aspettavano come scontate
	delle capacità che Ender non aveva. L'istinto di un atterraggio morbido
	e senza rimbalzi, la precisione di volo, la capacità di sfruttare come
	ripari i soldati congelati che fluttuavano a caso attraverso il locale.
	Roteare, spingersi via, schivare. Scivolare lungo le pareti, manovra
	questa difficile quanto preziosa, che consentiva il continuo contatto
	con una superficie utile.}

{E mentre si rendeva conto di quante fossero le cose che non sapeva,
	Ender ne vide altre che avrebbe potuto perfezionare. La manovra in
	formazioni prestabilite era un errore. Permetteva ai soldati di ricevere
	ed eseguire immediatamente gli ordini a voce, ma li rendeva anche molto
	più prevedibili. Inoltre ai singoli elementi era concessa poca
	iniziativa. Una volta che uno schema era ritenuto valido, c'era
	l'obbligo di seguirlo dall'inizio alla fine. Questo non lasciava spazio
	alle improvvisazioni, necessarie allorché il nemico si rivelava più
	capace del previsto. Ender analizzava le manovre di Bonzo come l'avrebbe
	fatto un comandante avversario, prendendo nota dei loro punti deboli.}

{Durante la partita libera di quella sera Ender chiese a Petra di
	giocare con lui.}

{-- No -- disse alla ragazzina. -- Io voglio diventare comandante un
	giorno o l'altro, perciò ho intenzione di cimentarmi solo in sala
	giochi.}

{Era convinzione comune che gli insegnanti spiassero elettronicamente le
	partite, e scegliessero lì i potenziali comandanti. Ender ne dubitava. I
	giocatori si esibivano su una macchina, i capibranco potevano mostrare
	sul campo le loro capacità di comando.}

{Ma non volle discutere con Petra. La sua offerta di fargli fare un po'
	di pratica era generosa. Tuttavia, quel breve allenamento dopo colazione
	non gli bastava. E non poteva esercitarsi da solo, salvo che in certe
	attività di base. Molte delle sue attività più complesse richiedevano un
	compagno o una squadra. Se soltanto avesse avuto Alai o Shen\ldots{}}

{Be', cosa gli \emph{impediva} di allenarsi con loro? Non aveva mai
	sentito di un membro di un'orda che andasse a far pratica coi novellini,
	però non c'erano regole che lo vietassero. Semplicemente, visto il
	generale disprezzo per i pivelli, nessuno s'era mai abbassato a tanto.
	Ender si disse che comunque l'orda avrebbe continuato a trattarlo come
	un pivello. A lui interessava avere qualcuno con cui esercitarsi, uno al
	quale avrebbe potuto dare in cambio ciò che apprendeva osservando
	l'orda.}

{-- Ehi, il grande soldato è di ritorno! -- fu il saluto con cui lo
	accolse Bernard, quando lo vide comparire sulla soglia della sua vecchia
	camerata. Mancava da appena ventiquattr'ore ma già gli sembrava che il
	posto avesse qualcosa di estraneo, e così anche i ragazzini con cui
	aveva vissuto fianco a fianco. Per un attimo fu tentato di voltarsi e di
	andarsene. Ma poi vide il volto di Alai, con cui aveva stretto un sacro
	patto di amicizia. Alai non era un estraneo.}

{Ender non si curò affatto di nascondere il modo in cui era trattato
	nell'orda delle Salamandre. -- E non hanno torto -- disse poi. -- Io
	servo loro come uno sternuto in una tuta spaziale. -- Alai rise, e altri
	del gruppo si fecero loro attorno. Ender propose il suo affare: partite
	libere ogni giorno, lavorando sodo in sala di battaglia sotto la sua
	direzione. Loro avrebbero appreso comportamenti e tecniche usate dalle
	orde in battaglia, lui si sarebbe impratichito nelle capacità militari
	che gli servivano. -- Potremo migliorare insieme. D'accordo?}

{I ragazzi che accettarono subito furono parecchi. -- A patto -- disse
	però lui, -- che veniate per lavorare. Chi ha soltanto voglia di
	svagarsi, è escluso. Io non ho tempo da gettar via.}

{Non fu gettato via, infatti, il tempo di quelli che lo seguirono in
	sala di battaglia. Ender ebbe delle difficoltà a far visualizzare loro
	gli addestramenti a cui aveva assistito, nuovi per tutti. Ma al termine
	della prima partita libera i ragazzi avevano imparato diverse cosette.
	Quando se ne andarono, sfiniti, già si eccitavano nel discutere questa o
	quella tecnica.}

{-- Dove sei stato? -- fu la domanda con cui lo accolse Bonzo.}

{Davanti alla cuccetta del comandante Ender si mise sull'attenti. -- A
	far pratica in sala di battaglia, signore.}

{-- Sì? Mi è stato detto che avevi con te alcuni dei tuoi ex compagni.}

{-- Non potevo esercitarmi da solo.}

{-- I soldati dell'orda delle Salamandre non devono far comunella con i
	novellini. E tu sei soldato, adesso.}

{Ender lo fissò senza aprir bocca.}

{-- Mi stai ascoltando, Wiggin?}

{-- Sì, signore.}

{-- Niente più trasgressioni con quei pidocchietti merdosi.}

{-- Posso parlarti privatamente? -- domandò Ender.}

{Era un genere di richiesta che i comandanti dovevano accogliere. Bonzo
	non nascose un'espressione irritata, ma precedette Ender nel corridoio
	esterno. -- Apri bene gli orecchi, Wiggin. Io non ti voglio, e sto
	cercando di liberarmi di te. Ma provati a darmi dei problemi e io ti
	faccio passare attraverso questo muro.}

\emph{{Un buon comandante}}{, \emph{} pensò Ender, \emph{non ha bisogno
		di fare queste stupide minacce.}}

{Seccato dal suo silenzio Bonzo emise un grugnito. -- Allora, mi hai
	fatto venire qui solo per rimirarmi? Sentiamo cos'hai da dire.}

{-- Comandante, hai fatto bene a non aggregarmi a un branco. Io non so
	far niente.}

{-- Non ho bisogno delle tue opinioni su quello che faccio, Wiggin.}

{-- Però io intendo diventare un buon soldato. Non voglio disturbare le
	vostre esercitazioni giornaliere, ma ho necessità di far pratica, e
	posso farla soltanto con quelli che accettano di esercitarsi con me. I
	miei ex compagni.}

{-- Tu farai quello che dico io, piccolo bastardo!}

{-- Certo, signore. Io eseguirò tutti gli ordini che sei autorizzato a
	darmi. Ma la partita libera è libera. Non possono essere imposte delle
	restrizioni. Nessuna. E da nessuno.}

{Il bel volto di Bonzo fu deformato da una smorfia di furore. Lasciarsi
	andare a emozioni così accese era uno sbaglio. Ender lo sapeva, ed era
	freddo, e sapeva come usare la sua freddezza. Bonzo prendeva fuoco, ed
	era la rabbia a usare lui.}

{-- Signore, questa carriera l'ho scelta liberamente. Non voglio
	interferire coi vostri allenamenti e le vostre battaglie, ma ho il
	diritto d'imparare. Non ho chiesto io d'essere assegnato alla tua orda,
	e tu stai cercando di vendermi al più presto. Però nessuno mi acquisterà
	se non so fare niente, no? Lasciami imparare qualcosa, e questo ti
	aiuterà a liberarti di me in minor tempo e a scambiarmi con qualcuno che
	ti sarà veramente utile.}

{Bonzo non era così sciocco da lasciare che l'ira gli impedisse di
	riconoscere un'osservazione logica e sensata. Ma questo non bastò a
	fargliela sbollire del tutto.}

{-- Chi indossa l'uniforme delle Salamandre non deve azzardarsi a
	discutere i miei ordini, bamboccio!}

{-- Alterare le partite libere di qualcuno può costare il congelamento.}

{Questo probabilmente non era vero. Ma era possibile. Certo, se Ender
	avesse fatto un esposto agli insegnanti, l'aver interferito con le sue
	partite libere poteva costare a Bonzo il grado di comandante. Inoltre
	era ovvio che gli ufficiali dovevano aver visto qualcosa in Ender, per
	avergli dato quella promozione. Forse Ender \emph{aveva} abbastanza
	influenza presso gli ufficiali da ottenere il congelamento di qualcuno.
	-- Bastardo! -- ringhiò Bonzo.}

{-- Non è colpa mia se mi hai dato quell'ordine davanti a tutti -- disse
	Ender. -- Ma se vuoi, adesso fingo di andarmene a letto con la coda fra
	le gambe. E domani potrai informarmi che hai cambiato idea.}

{-- Sei così presuntuoso da suggerire a me come mi devo comportare?}

{-- Non voglio che gli altri ti vedano costretto a far marcia indietro.
	Altrimenti non potresti conservare la tua autorità.}

{Quella cortesia Bonzo se la legò al dito come uno sgarbo, quasi che
	Ender gli avesse concesso a titolo di favore di non perdere la faccia
	con gli altri. Lo fissò con odio, conscio che pur dandogli una
	scappatoia quel novellino non gli lasciava scelta. E non stette a
	pensare che la colpa era sua, per avergli dato un ordine irragionevole.
	Sapeva solo che Ender lo aveva messo alle strette, e che adesso si
	degnava d'essere magnanimo con lui.}

{-- Un giorno o l'altro avrò le tue palle su un vassoio -- disse Bonzo.}

{-- Probabilmente -- annuì lui. Le luci si abbassarono e un cicalino
	ronzò il segnale della ritirata. Ender rientrò nel dormitorio a capo
	chino. Irritato. Mogio mogio. Gli altri ragazzi poterono trarne le ovvie
	conclusioni.}

{Il mattino successivo, mentre Ender si metteva in fila coi compagni
	diretti a far colazione, Bonzo gli ordinò di fare un passo avanti e
	disse, a voce alta: -- Ho cambiato idea, ragazzo. Forse far pratica con
	i tuoi vecchi compagni ti insegnerà qualcosa, e potremo imbrogliare
	l'orda a cui ti venderemo dicendo che almeno due soldi li vali.
	D'accordo?}

{-- Sissignore. Grazie, signore -- disse lui.}

{-- E spero -- sussurrò Bonzo, -- di vederti finire congelato.}

{Ender gli rivolse un sorriso di gratitudine e uscì con gli altri. Dopo
	colazione fece ancora pratica con Petra. Per tutto il pomeriggio assisté
	alle esercitazioni di Bonzo e ipotizzò metodi per distruggere la sua
	orda. Durante la partita libera lavorò con Alai e gli altri finché
	furono esausti. \emph{Posso farcela}, \emph{} si costrinse a pensare
	quella sera lasciandosi cadere sulla cuccetta. Aveva i muscoli a pezzi.
	\emph{Posso tenere in pugno questa cosa.}}

{~}

\begin{center}
	{* * *}
\end{center}

{~}

{Quattro giorni dopo l'orda delle Salamandre entrò in campo contro
	l'orda dei Condor. Ender sfilò nei corridoi con gli altri soldati,
	marciando al passo verso la sala di battaglia. Sulle pareti scorrevano
	due strisce luminose, la verde verde marrone delle Salamandre e la
	bianca nera bianca dei Condor. Nel corridoio centrale le due strisce si
	separarono, e le Salamandre seguirono i loro colori in una diramazione.
	Dopo un'ultima svolta a destra l'orda si fermò davanti a una parete
	nuda.}

{I branchi serrarono i ranghi in silenzio, mentre Ender restava in coda
	alla formazione. Bonzo mitragliava già i primi ordini: -- A, perdere per
	il corrimano e andare su, B a sinistra, C a destra, D in basso. --
	Controllò che gli uomini fossero pronti, poi si volse. -- Tu, pivello,
	aspetta quattro minuti poi entra e fermati a lato della porta. Non
	muoverti e non estrarre la pistola.}

{Ender annuì. Ad un tratto la parete davanti a Bonzo diventò
	trasparente. Non era un muro dunque, ma un campo di forza. Anche la sala
	di battaglia che vide era diversa. Nell'aria erano sospesi cassoni
	poligonali di colore marroncino, che ostruivano in parte la visuale.
	Dunque quelli erano gli ostacoli che i soldati chiamavano \emph{stelle.}
	Apparentemente erano distribuiti a caso. Bonzo sembrò non preoccuparsi
	della loro dislocazione, così Ender pensò che i soldati sapevano già
	quale uso fare delle stelle.}

{Ma quasi subito, mentre sedeva in corridoio a osservare l'inizio delle
	ostilità, gli fu chiaro che non lo sapevano affatto. Non erano capaci di
	compiere un atterraggio morbido su una di esse e sfruttarla per
	coprirsi, quando dovevano attaccarla per distruggere un avamposto nemico
	attestato sul retro. Non avevano il senso di quello che era al momento
	il valore strategico di una stella: insistevano ad attaccare anche
	quelle che avrebbero potuto lasciarsi alle spalle per conquistare
	posizioni più avanzate.}

{L'altro comandante stava approfittando delle manchevolezze strategiche
	di Bonzo. L'orda dei Condor invitava le Salamandre a effettuare attacchi
	che costavano loro un prezzo eccessivo, e dopo aver conquistato una
	stella erano sempre meno gli uomini non congelati che si spingevano
	verso la successiva. Dopo cinque o sei minuti soltanto fu evidente che
	l'orda delle Salamandre non poteva vincere insistendo in quell'attacco.}

{Ender oltrepassò la porta. In assenza di peso si spinse leggermente
	verso il basso. Le sale di battaglia in cui s'era esercitato avevano
	l'ingresso al livello del pavimento. Negli scontri fra orde questo era
	invece al centro di una parete, equidistante dalle altre quattro.}

{In pochi istanti il suo senso dell'orientamento cambiò come gli era
	accaduto la prima volta nella navetta. Quello che era stato il basso
	diventava a piacere l'alto, oppure un lato. A zero G non c'era motivo di
	restare orientato secondo i punti cardinali del corridoio, e poiché la
	porta era quadrata gli era già impossibile dire dov'era stato l'alto.
	Non che questo importasse. Ender aveva stabilito su quale parametro un
	soldato doveva regolarsi: la porta d'ingresso del nemico era giù.
	L'obiettivo della battaglia stava nel cadere verso le postazioni
	avversarie.}

{Con alcuni movimenti si orientò in quella nuova direzione. Invece di
	essere steso all'infuori con l'intero corpo esposto ai Condor, adesso
	presentava loro solo le suole delle scarpe. Era un bersaglio molto più
	ristretto.}

{Qualcuno lo vide. E non c'era da aspettarsi altro, dato che fluttuava
	indifeso all'aperto. D'istinto ripiegò le gambe sotto di sé. Nello
	stesso istante su di lui balenò un circoletto di luce, e le gambe della
	sua tuta si congelarono in quella posizione. Le braccia invece restarono
	libere, poiché se il colpo non giungeva in pieno corpo a subirne
	l'effetto erano solo gli arti che lo incassavano. Ender rifletté che se
	non si fosse messo per il lungo il Condor l'avrebbe colpito al corpo. E
	lui sarebbe rimasto del tutto immobilizzato.}

{Visto che Bonzo gli aveva ordinato di non estrarre la pistola Ender
	continuò a fluttuare senza muovere la testa né le braccia, come se
	avessero congelato anche lui. Il nemico lo ignorò, e concentrò il fuoco
	sui soldati che stavano sparando. La conclusione si prospettava amara.
	Ormai inferiore di numero l'orda delle Salamandre, pur tenace, stava
	cedendo terreno. La battaglia si frammentò in una dozzina di scontri
	isolati. Ma la disciplina imposta da Bonzo dava adesso i suoi frutti,
	perché ogni Salamandra colpita si portava dietro almeno un avversario.
	Nessuno fuggiva o si lasciava prendere dal panico: tutti conservavano la
	calma e sparavano finché non venivano sopraffatti.}

{La più micidiale fra i superstiti era Petra. I Condor erano stati
	costretti ad accorgersene, e un intero branco manovrava per toglierla di
	mezzo. Infine riuscirono a congelarle il braccio con cui sparava, e il
	torrente d'imprecazioni della ragazzina s'interruppe soltanto quando una
	gragnuola di colpi la immobilizzò completamente e la visiera del suo
	casco s'abbassò fino al mento. L'orda delle Salamandre non oppose più
	una valida resistenza, e pochi minuti dopo tutto era finito.}

{Ender notò compiaciuto che i Condor potevano appena mettere insieme
	cinque soldati, il numero minimo indispensabile per aprire la porta in
	caso di vittoria. Quattro di loro toccarono con l'elmetto i punti
	luminosi ai quattro angoli della porta delle Salamandre, ed il quinto
	passò oltre il campo di forza. Questo atto mise termine alla partita. Le
	luci tornarono alla massima luminosità, e Anderson entrò in sala dalla
	porta degli insegnanti.}

\emph{{Avrei potuto estrarre la pistola}}{, \emph{} pensò Ender mentre i
	Condor uscivano. \emph{Mi sarebbe bastato colpire uno di loro e
		sarebbero stati troppo pochi per aprire. La partita sarebbe finita in
		pareggio. Servono quattro uomini per consentire al quinto di
		oltrepassare la porta. E i Condor non avrebbero avuto la vittoria.
		Bonzo, razza di somaro, avrei potuto salvarti dalla disfatta. Forse
		perfino trasformarla in un successo, perché quei cinque erano bersagli
		facili e non avrebbero capito subito da dove sparavo. Sono già
		abbastanza bravo come tiratore.}}

{Ma gli ordini erano ordini, e lui aveva promesso di ubbidire. La sola
	soddisfazione l'ebbe pensando che nei documenti di gara delle Salamandre
	sarebbero stati registrati non quarantuno eliminati, bensì quaranta
	eliminati e uno parzialmente inabilitato. Bonzo non l'avrebbe saputo
	finché non avesse consultato il registro di Anderson e visto di chi si
	trattava. \emph{Inabilitato, Bonzo, capisci? Io potevo ancora sparare.}}

{S'era quasi atteso che Bonzo venisse a cercarlo e dicesse: -- La
	prossima volta che capita una cosa simile, sei autorizzato a sparare. --
	Ma lui non gli rivolse la parola fino al mattino successivo dopo
	colazione. Naturalmente Bonzo mangiava nella mensa dei comandanti, ma
	Ender era abbastanza certo che lo strano risultato della partita avrebbe
	causato là tante chiacchiere quante ne stava destando nella mensa
	comune. In ogni partita che non fosse terminata in pareggio tutti i
	soldati dell'orda perdente risultavano eliminati oppure completamente
	disabilitati, cioè non del tutto congelati ma privi della possibilità di
	sparare o infliggere danni al nemico. Le Salamandre erano l'unica orda
	che fosse riuscita a perdere con un uomo ancora nella categoria di
	quelli in grado di usare l'arma.}

{Ender s'era riproposto di tener la bocca chiusa, ma accanto a lui
	vennero a sedersi delle Salamandre che con aria grave pretesero una
	spiegazione. E quando i ragazzi gli chiesero perché non avesse ignorato
	gli ordini e sparato, lui rispose con calma: -- Io ubbidisco agli
	ordini.}

{Dopo colazione Bonzo lo fece chiamare. -- Le istruzioni che hai restano
	tali e quali -- disse. -- E bada a non sgarrare.}

\emph{{Questo continuerà a costarti caro, idiota. Forse non sarò un buon
		soldato, ma posso sempre essere d'aiuto e non c'è ragione che tu me lo
		proibisca.}}

{Ender non diede voce ai suoi pensieri.}

{Un interessante effetto collaterale della battaglia fu che il nome di
	Ender emerse in cima alla lista dei quozienti d'efficienza individuale.
	Dal momento che non aveva sparato un sol colpo, il computer gli
	conferiva un record perfetto: errori zero. E visto che non era mai stato
	eliminato né disabilitato, il quoziente d'efficienza risultava ottimo.
	Il secondo della lista era abbondantemente distanziato. Questo fece
	ridere molti dei ragazzi, mentre altri imprecarono contro l'imbecillità
	dei cervelli elettronici, ma restava il fatto che quei risultati
	conducevano a un premio, e che Ender era il primo in graduatoria.}

{Continuò ad assistere inattivo agli allenamenti dell'orda, e continuò a
	lavorare sodo per conto suo, con Petra al mattino e col gruppo di Alai
	alla sera. Altri dei novellini adesso si stavano unendo a loro, non per
	passatempo ma perché potevano vederne i risultati: imparavano a
	battersi, e questo era soddisfacente. Ender e Alai però erano sempre un
	passo più avanti degli altri. In parte perché Alai non la smetteva di
	ideare nuove varianti, cosa che forzava Ender a studiare nuove
	contromosse per rintuzzarle. In parte perché seguitavano a fare errori
	stupidi, per rimediare ai quali si adattavano ad azioni che nessun
	soldato ben addestrato e conscio della propria dignità avrebbe mai
	fatto. Molte delle tecniche che escogitarono si rivelarono
	inutilizzabili. Ma era pur sempre divertente, sempre eccitante, e le
	cose che funzionavano erano abbastanza da convincerli che non stavano
	perdendo tempo. La sera era il momento migliore delle loro giornate.}

{Le due battaglie successive furono vinte con facilità dalle Salamandre.
	Ender entrò in sala allo scadere dei quattro minuti e rimase intoccato
	dagli avversari sconfitti. Questo lo convinse che l'orda dei Condor, da
	cui erano stati battuti, era decisamente pregevole. Le Salamandre, per
	quanto le tattiche di Bonzo fossero stucchevoli, erano fra le orde
	migliori e consolidando la loro posizione in classifica stavano
	contendendo il terzo posto all'orda dei Topi.}

{Ender compì sette anni. Il calendario terrestre, con le sue date e
	festività, veniva ignorato alla Scuola di Guerra, ma lui aveva scoperto
	il modo di richiamare la data sullo schermo del banco e poté prender
	nota del suo compleanno. Anche il magazzino della Scuola aveva notato la
	data; gli presero le misure e gli consegnarono nuove tute da fatica,
	oltre a quella speciale da portarsi in sala di battaglia, con i colori
	sgargianti delle Salamandre. Tornò in camerata con la pila di indumenti
	sulle braccia. Nel provarli li aveva sentiti strani e larghi, come se la
	sua pelle stentasse ad adattarsi ad essi.}

{Gli sarebbe piaciuto fermarsi alla cuccetta di Petra e parlarle un poco
	di casa sua, di ciò che erano stati là i compleanni, oppure dirle
	semplicemente che quel giorno compiva gli anni in modo che lei facesse
	una battuta ironica sull'allegria di simili ricorrenze. Ma lì nessuno
	parlava dei compleanni. Era una cosa infantile. Torte e candeline erano
	roba che non usava quasi più neppure sulla Terra. Per il suo sesto
	compleanno Valentine aveva fatto una torta alla crema. Ma la pasta s'era
	rifiutata di lievitare. Nessuno sapeva più cucinare in casa, però quello
	era il genere di stravaganze tipico di Valentine. Tutti avevano
	biasimato sia lei che il sapore della torta, ma Ender ne aveva messo via
	una fetta avvolta nella stagnola. Poi gli avevano tolto il monitor, era
	partito, e per quel che ne sapeva la fetta era ancora là nel suo
	armadio, un pezzetto di roba gialla dura e polverosa. Nessuno parlava di
	casa, non fra i soldati; la vita prima della Scuola di Guerra era un
	periodo chiuso. Nessuno riceveva lettere, né le scriveva. Tutti
	fingevano di non interessarsi più al passato.}

\emph{{Ma a me importa}}{, \emph{} pensò Ender quella sera. \emph{La
		sola ragione per cui sono qui è perché gli Scorpioni non riescano mai a
		spegnere per sempre gli occhi di Valentine, a farla a pezzi coi raggi a
		esplosione come quei marines dei filmati ripresi durante le prime
		battaglie. Non le colpiranno la testa con quei raggi così ardenti che il
		cervello ribolle nel cranio e schizza fuori giallo quanto il budino di
		una pasta scoppiata, come succede nei mei incubi peggiori, nelle mie
		notti peggiori, quando mi sveglio tremante ma zitto\ldots{} zitto,
		perché non sentano che ho nostalgia della mia famiglia. Come vorrei
		essere a casa!}}

{Il mattino dopo si sentiva meglio. La casa era soltanto una lieve fitta
	di dolore in un angolo della sua memoria. Una luce grigia nei suoi
	occhi. Mentre si vestivano Bonzo entrò a lunghi passi. -- Tute da
	battaglia! -- ordinò. Li attendeva una partita, la quarta dall'arrivo di
	Ender.}

{L'avversario era l'orda dei Leopardi. Non si prevedevano difficoltà. I
	Leopardi erano un'orda nuova, messa in piedi soltanto sei mesi prima dal
	suo comandante, Pol Slattery, e stazionava nelle ultime posizioni della
	classifica. Ender indossò la sua tuta di battaglia fresca di magazzino e
	si allineò con gli altri; Bonzo lo spinse rudemente fuori dalla fila e
	lo spedì in coda a tutti. \emph{Non c'era bisogno che tu facessi così},
	\emph{} disse lui dentro di sé. \emph{Potevi lasciarmi in fila
		dov'ero.}}

{Dal corridoio osservò l'inizio delle ostilità. Pol Slattery era
	giovane, ma in gamba e pieno di idee nuove. Teneva i suoi soldati in
	perpetuo movimento facendoli balzare da stella a stella, o slittare
	lungo le pareti per arrivare sopra o dietro le stolide Salamandre. Ender
	sorrise. Quella tattica gettava Bonzo in uno stato di confusione, e così
	anche i suoi branchi. I Leopardi sembravano avere uomini piazzati
	dappertutto. Tuttavia lo scontro non era così squilibrato come poteva
	sembrare. Ender notò che i Leopardi stavano perdendo molti uomini,
	troppi\ldots{} la loro strategia basata sul movimento li portava di
	continuo allo scoperto. Ciò che faceva gioco, però, era il fatto che le
	Salamandre si \emph{sentivano} surclassate. Avevano perso completamente
	l'iniziativa. Pur dimostrando maggiori capacità individuali si
	stringevano assieme come gli ultimi superstiti di un massacro, come se
	sperassero che nel carnaio il nemico si dimenticasse di loro.}

{Ender scivolò lentamente dentro dalla porta, si girò in modo che la
	posizione del nemico fosse in basso rispetto a lui, e pian piano si
	spinse fino all'angolo di destra dove pochi avrebbero potuto notarlo.
	Nel fluttuare sparò alle sue stesse gambe, per tenere le ginocchia
	ripiegate nella posa che gli offriva la migliore protezione. A un occhio
	poco attento sarebbe parso uno fra i tanti soldati congelati che
	galleggiavano via ai margini della battaglia.}

{Appena fu chiaro che le Salamandre attendevano più o meno supinamente
	la sconfitta, i Leopardi s'impiegarono ferocemente in cerca della
	vittoria. Avevano ancor nove uomini attivi quando il fuoco delle
	Salamandre cessò. Questi si riunirono e s'accinsero ad aprire la porta
	degli avversari.}

{Col braccio teso in avanti come Petra gli aveva insegnato, Ender prese
	accuratamente la mira. Prima che gli altri capissero cosa stava
	succedendo, aveva congelato tre dei soldati che erano sul punto di
	poggiare il casco sugli angoli luminosi della porta. Poi alcuni dei
	superstiti lo individuarono e puntarono le armi\ldots{} ma i colpi
	giunsero a segno sulle sue gambe, già immobilizzate. Questo gli diede il
	tempo di centrare gli ultimi due di quelli che erano andati alla porta.
	Allorché Ender fu finalmente colpito al braccio e disabilitato, i
	Leopardi avevano soltanto quattro uomini non congelati. La partita era
	terminata in pareggio, e non lo avevano neppure mai colpito al corpo.}

{Pol Slattery era furibondo, ma nella cosa non c'era stato nulla di
	sleale. Tutti i Leopardi diedero per certo che lasciar fuori un uomo
	fino all'ultimo minuto era stata una mossa tattica di Bonzo. Nessuno
	poteva sospettare che Ender aveva sparato contravvenendo agli ordini. Ma
	le Salamandre sapevano come stavano le cose. Bonzo lo sapeva, e dal modo
	in cui lo guardava Ender constatò che il comandante lo odiava per
	avergli risparmiato la disfatta. \emph{Non me ne importa}, \emph{} si
	disse. \emph{Questo gli renderà più facile vendermi, e intanto i ragazzi
		non scenderanno troppo in classifica. Ma tu vendimi. Ho già imparato
		tutto quel che potevo da te: come perdere con faccia impassibile, ecco
		l'unica cosa che sai far bene, Bonzo.}}

\emph{{Cos'ho imparato di buono oggi? }}{Ender cercò di tirare i conti
	della giornata, mentre si spogliava accanto alla sua cuccetta. \emph{La
		porta del nemico è sempre giù. Usare le gambe come scudo in battaglia.
		Alcune riserve, tenute da parte fino al termine degli scontri, possono
		essere decisive. E il fatto che a volte i soldati sanno prendere
		decisioni più intelligenti degli ordini che hanno avuto.}}

{Era nudo e sul punto di arrampicarsi sul letto a castello quando Bonzo
	arrivò nel passaggio centrale, con faccia dura e ferma. \emph{Ho già
		visto quell'espressione in Peter}, \emph{} pensò Ender. \emph{Silenzio,
		e l'omicidio nello sguardo. Ma Bonzo non è Peter. Bonzo sa cos'è la
		paura.}}

{-- Finalmente ti ho venduto, Wiggin. Sono riuscito a persuadere l'orda
	dei Topi che il tuo incredibile posto nella lista dell'efficienza
	individuale non è soltanto un puro caso. Domani te ne vai.}

{-- Grazie, signore -- disse Ender.}

{Forse il suo tono fu eccessivamente grato. Bonzo si volse di scatto e
	lo colpì con un furibondo ceffone in piena faccia, che lo mandò a
	barcollare stordito contro il montante delle cuccette. Poi gli sferrò un
	pugno secco e calcolato al plesso solare. Ender cadde in ginocchio.}

{-- Questo perché hai disubbidito -- disse Bonzo ad alta voce, perché
	tutti sentissero. -- Un buon soldato non disubbidisce mai.}

{Ma anche mentre gemeva sul punto di vomitare Ender riuscì a sentire,
	con un acre fremito di soddisfazione, il mormorio che s'era levato nella
	camerata. \emph{Sei uno sciocco, Bonzo. Non hai rafforzato la
		disciplina, le hai dato un calcio. Loro sanno che ho trasformato io la
		sconfitta in un pareggio, e adesso hanno visto come mi ripaghi. Hai
		fatto la figura dell'idiota davanti a tutti. Quanta ne rimane della tua
		disciplina, ora?}}

{Il giorno dopo disse a Petra che per il suo bene le conveniva non
	dargli più lezioni di tiro al mattino. Per giungere ad atti estremi
	Bonzo non aspettava altro che vedersi sfidato, così lei avrebbe fatto
	meglio a tenersi alla larga da Ender per un po'. La ragazzina capì
	benissimo la situazione. -- Comunque -- gli disse, -- sei già sul punto
	di arrivare al massimo delle tue capacità di tiratore.}

{Lasciò il banco e la tuta da battaglia negli armadietti. Avrebbe tenuto
	addosso l'uniforme delle Salamandre finché non avesse potuto andare in
	magazzino a cambiarla con quella marrone e nera dei Topi. Non aveva
	oggetti personali; non avrebbe portato via nulla con sé. Tutto ciò che
	poteva affermare di possedere si trovava nel computer della Scuola,
	nella sua testa e nel suo cuore.}

{Usò una delle scrivanie pubbliche della sala giochi per registrare la
	richiesta di un corso personale di combattimento a gravità-Terra durante
	l'ora successiva alla colazione. Non intendeva vendicarsi di Bonzo. Ma
	non voleva che qualcuno potesse ancora colpirlo e metterlo a terra a
	quel modo.}

\phantomsection\label{Orsonux20Scottux20Cardux20-ux20Ilux20Giocoux20Diux20Enderux20-ux20BY_SLY70A1_split_010.htm}{}
