\chapter{L'ORDA E IL BRANCO}

{~}

{~}

{~}

{-- \emph{Colonnello Graff, finora le partite sono sempre state giocate
		con lealtà. Sia che la dislocazione delle stelle fosse casuale, sia che
		fosse simmetrica.}}

{-- \emph{La lealtà è una dote meravigliosa, maggiore Anderson. Non ha
		niente a che fare con la guerra.}}

{-- \emph{I risultati ne saranno compromessi. La classifica diventerà un
		dato privo di significato.}}

{-- \emph{Così sia.}}

{-- \emph{Ci vorranno mesi, anni, per attrezzare le nuove sale di
		battaglia e regolamentare le simulazioni belliche.}}

{-- \emph{È di questo che sono venuto a parlarle, infatti. Ricominci.
		Sia creativo. Pensi a ogni insolita o impossibile dislocazione delle
		stelle. Pensi ad altri modi in cui le regole possono essere aggirate.
		Aggiunga articoli, comma, eccezioni. Poi collaudi le simulazioni e veda
		qual è il loro grado di difficoltà. Vogliamo che qui ci sia una
		progressione calcolata. Vogliamo portare avanti il ragazzo.}}

{-- \emph{Quando ha intenzione di farne un comandante? A otto anni?}}

{-- \emph{No, naturalmente. Non ha ancora messo insieme la sua orda.}}

{-- \emph{Ah! Dunque mette sotto il torchio anche altri allo stesso
		modo?}}

{-- \emph{Lei sta dando troppa importanza alle gare, Anderson. Dimentica
		che si tratta di un addestramento e nient'altro.}}

{-- \emph{Dalle gare emergono lo stato sociale dell'individuo, i suoi
		scopi di vita, la sua identità. I bambini ne vengono fuori con una
		personalità formata. Se si pensasse che le gare possono essere oggetto
		di manipolazioni e imbrogli, la Scuola ne sarebbe scossa fin nelle
		fondamenta. Non sto esagerando.}}

{-- \emph{Lo so.}}

{-- \emph{Allora preghi che Ender Wiggin sia davvero il suo uomo, perché
		lei ha rovinato l'efficienza del suo metodo di addestramento e non potrà
		metterci una pezza per un bel po' di tempo ancora.}}

{-- \emph{Se Ender non è quello che spero, e se il momento in cui
		giungerà al meglio delle sue possibilità militari non coinciderà con
		l'arrivo delle nostre flotte al mondo d'origine degli Scorpioni, allora
		non avrà alcuna importanza quali metodi usiamo qui alla Scuola.}}

{-- \emph{Spero che lei mi perdoni, colonnello Graff, ma sento di dover
		riferire i suoi ordini e la mia opinione sulle loro conseguenze allo
		Stratega e all'Egemone.}}

{-- \emph{Perché non anche al nostro amato Condottiero?}}

{-- \emph{Tutti sanno che lei ce l'ha nella manica.}}

{-- \emph{Quanta ostilità, maggiore Anderson! E io che credevo fossimo
		amici.}}

{-- \emph{Lo siamo. E penso che lei possa aver ragione su Ender. Solo
		non credo che lei, e soltanto lei, debba decidere il destino del
		mondo.}}

{-- \emph{Io non penso neppure d'avere il diritto di decidere il destino
		del solo Ender Wiggin.}}

{-- \emph{Così non le importa se faccio un rapporto?}}

{-- \emph{Certo che m'importa, razza d'un dannatissimo ficcanaso. Questa
		è una cosa che va decisa da gente che sa quel che sta facendo, non da
		dei cacasotto di politicanti che hanno usato i loro quattrini per farsi
		dare una poltrona.}}

{-- \emph{Ma lei capisce perché devo farlo.}}

{-- \emph{Certo: perché lei è un piccolo bastardo di burocrate dalla
		vista corta e pensa soltanto a star bene ammanigliato nel caso che le
		cose vadano male. Be', se le cose andranno male tutti quanti saremo cibo
		per le larve degli Scorpioni. Così adesso abbia fiducia in me, Anderson,
		e non tiri sulle mie spalle tutta la dannata Egemonia. Quel che sto
		facendo è già abbastanza difficile anche senza di loro.}}

{-- \emph{Oh, che peccato! Qualcosa le rende dura la vita? Può farlo a
		Ender, ma non sopporta quando capita a lei, vero?}}

{-- \emph{Ender Wiggin è dieci volte più intelligente e robusto di me.
		Quello che gli faccio tirerà fuori la sua genialità. Se al suo posto ci
		fossi io, ne uscirei a pezzi. Maggiore Anderson, so che sto facendo
		naufragare le gare, e so che lei è più affezionato di me ad alcuni dei
		ragazzi che le giocano. Mi odi pure, se vuole, ma non mi fermi.}}

{-- \emph{Mi riservo il diritto di parlarne all'Egemone e allo Stratega
		quando vorrò. Ma per ora\ldots{} faccia quello che ritiene meglio.}}

{-- \emph{Grazie per la sua così spontanea fiducia.}}

{~}

\begin{center}
	{* * *}
\end{center}

{~}

{-- Ender Wiggin\ldots{} il piccolo mangiamerda che furoreggia nella
	grande graduatoria! Che piacere averti qui con noi! -- Il comandante
	dell'orda dei Topi giaceva spaparanzato su una delle cuccette inferiori,
	vestito solo del suo banco. -- Con te attorno, un'orda deve proprio
	mettercela tutta per perdere. -- Parecchi ragazzi della camerata risero
	forte.}

{Non avrebbero potuto esserci due orde più diverse delle Salamandre e
	dei Topi. Il locale era un caos di disordine, sporco e rumoroso. Dopo
	Bonzo, Ender avrebbe creduto che un po' d'indisciplina sarebbe stata un
	sollievo. Invece scoprì che s'era atteso quiete e ordine, e che quella
	baraonda lo metteva a disagio.}

{-- Le cose ci vanno già a tutto vapore, Enderello bello. Io sono Rose
	de Nose, un geniale comandante ebreo, e tu un testavuota buono a nulla
	di un \emph{goy.} Non scordarlo mai, e tutto ti andrà facile.}

{Fin da quando la F.I. era stata fondata, lo Stratega delle operazioni
	militari era sempre stato un ebreo. Questo per via del mito secondo cui
	un generale ebreo non perdeva mai una guerra. E fino a quel momento il
	mito non era stato smentito. Ciò conferiva prestigio a ogni ebreo della
	Scuola di Guerra fin dall'inizio, e gli faceva sognare di diventare
	Stratega. Era anche causa di rancori. Di conseguenza c'era chi chiamava
	i Topi «l'orda dei Giudei» o con titoli ancor meno gentili. Ma c'era
	anche chi ricordava volentieri che durante la Seconda Invasione il
	Presidente americano, un ebreo, era stato l'Egemone degli alleati, e un
	ebreo israeliano aveva ricoperto l'incarico di Stratega nella difesa a
	terra. E il Condottiero della Flotta era stato un ebreo d'origine per
	metà russa e per metà maori della Nuova Zelanda, Mazer Rackham,
	inizialmente sconosciuto e per due volte sottoposto a corte marziale, la
	cui leggendaria Forza d'Assalto aveva spezzato l'accerchiamento delle
	strapotenti astronavi nemiche per poi distruggere gli Scorpioni in una
	battaglia terribile presso Saturno.}

{E se Mazer Rackham era riuscito a salvare il mondo, allora non
	importava un fico se uno era ebreo o non lo era. Così diceva la gente.}

{Ma importava, e Rose de Nose lo sapeva. Si compiaceva di prendere in
	giro se stesso per prevenire i commenti sarcastici degli antisemiti
	(quasi tutti quelli che sconfiggeva in sala di battaglia diventavano,
	almeno per qualche giorno, dei mangiaebrei) ma nello stesso tempo si
	assicurava che tutti sapessero chi era. La sua orda occupava il secondo
	posto in classifica, e aspirava al primo.}

{-- Ti ho preso con me, goy, perché non mi va di sentir dire che vinco
	soltanto perché ho dei bravi soldati. Tutti devono vedere che perfino
	con un soldo di cacio di poppante come te posso sempre vincere. Noialtri
	qui abbiamo solo tre regole. Fai quello che dico io, e non pisciare a
	letto.}

{Ender annuì. Sapendo che Rose voleva sentirsi chiedere quale fosse la
	terza regola si rassegnò a domandarlo. L'altro strinse le palpebre.}

{-- Vuoi dire che quelle non erano tre? Be', ragazzo, non siamo molto
	bravi in matematica, qui.}

{Il messaggio era chiaro. Vincere contava di più di ogni altra cosa.}

{-- Le tue piccole esercitazioni con quei lattonzoli del tuo gruppo sono
	finite, Wiggin. Dimenticale. Sei in un'orda di ragazzi grandi, adesso.
	Ti faccio l'onore di arruolarti nel branco di Dink Meeker. Da ora in
	poi, per quello che ti riguarda, Dink Meeker è il tuo solo Dio. OK?}

{-- Allora tu chi sei?}

{-- Il boss a cui Dio viene a fare rapporto tutti i giorni. -- Rose
	sogghignò. -- E per cominciare ti è proibito usare ancora il banco
	finché non farai fuori due nemici nella stessa battaglia. L'ordine serve
	solo alla difesa di noi poverini. Corre voce che tu sia un seduttore di
	computer, e non voglio che tu metta le tue laide mani sul mio banco
	innocente.}

{Tutti scoppiarono a ridere, anche chi non poteva aver sentito, e Ender
	ci mise qualche momento a capirne il perché. Rose aveva programmato
	sullo schermo del suo banco un disegno animato, rappresentante un organo
	genitale maschile fornito di braccia e gambe che si toglieva l'uniforme
	delle Salamandre per indossare quella dei Topi. Nudo e con l'apparecchio
	poggiato sull'addome, lo aveva inviato ai banchi degli altri.
	\emph{Questo è proprio il tipo di comandante a cui Bonzo voleva
		vendermi}, \emph{} pensò Ender. \emph{Come riesce a vincere le partite
		un ragazzo che passa il suo tempo a questo modo?}}

{Ender trovò Dink Meeker in sala giochi, seduto a guardare un paio di
	compagni. -- Alla porta mi hanno detto che tu sei Meeker. Io sono Ender
	Wiggin.}

{-- Lo so -- annuì l'altro.}

{-- Faccio parte del tuo branco.}

{-- Lo so -- disse ancora lui.}

{-- Sono piuttosto inesperto.}

{Il ragazzo lo fissò. -- Senti, Wiggin, queste cose le so già. Perché
	credi che io abbia chiesto a Rose di assegnarti a me?}

{Non era stato affibbiato a qualcuno, era stato scelto, lo avevano
	chiesto. Meeker lo voleva. -- Perché? -- domandò.}

{-- Ho assistito a un paio dei tuoi allenamenti coi nuovi arrivati.
	Credo che tu abbia delle doti. Bonzo è uno stupido, e voglio che tu
	abbia un addestramento migliore di quello che può darti Petra. Tutto ciò
	che lei sa è come usare la pistola.}

{-- Avevo bisogno di far pratica di tiro.}

{-- Ti muovi ancora come se avessi paura d'inciampare nelle scarpe.}

{-- Allora insegnami.}

{-- Tu pensa a imparare.}

{-- Non ho intenzione di smettere l'allenamento nel mio tempo libero.}

{-- Io non ti ho chiesto di smetterla.}

{-- Lo ha fatto Rose de Nose.}

{-- Rose de Nose non può darti quest'ordine. D'altra parte, può
	impedirti di usare il tuo banco.}

{-- Credevo che i comandanti potessero ordinare qualsiasi cosa.}

{-- Potrebbero anche ordinare alla luna di diventare blu, ma questo non
	accadrebbe. Ascolta, Ender, i comandanti hanno esattamente l'autorità
	che tu gli permetti di avere. Più ubbidisci ciecamente, più potere
	avranno su di te.}

{-- Anche quello di prendermi a pugni a loro piacimento? -- chiese
	Ender, ricordando la punizione inflittagli da Bonzo.}

{-- Ho sentito dire che quello è stato a causa di una tua certa
	iniziativa non autorizzata.}

{-- Mi hai tenuto d'occhio sul serio, allora. È così?}

{Dink non rispose.}

{-- Non voglio che anche Rose mi prenda di mira. Voglio scendere in
	battaglia come gli altri, adesso. Sono stanco di star seduto fuori a
	guardare.}

{-- Nella classifica dell'efficienza personale andrai giù.}

{Stavolta fu Ender a non rispondere.}

{-- Ascolta, Wiggin, finché sarai parte del mio branco sarai parte della
	battaglia.}

{Lui ne capì presto il motivo. Dink addestrava il suo branco
	indipendentemente dal resto dell'orda dei Topi, con vigore e disciplina;
	non si consultava mai con Rose, e solo di rado l'orda eseguiva manovre
	d'insieme. Era come se Rose comandasse un esercito e Dink un altro molto
	più piccolo che per caso si allenava in sala di battaglia nelle stesse
	ore.}

{Dink diede inizio ai primi esercizi chiedendo a Ender di dare una
	dimostrazione della sua tecnica d'attacco a piedi in avanti. Agli altri
	ragazzi non piacque. -- Come si può andare all'attacco distesi sulla
	schiena? -- domandarono.}

{Con sorpresa di Ender, Dink non li corresse dicendo: -- Non state
	attaccando sdraiati sulla schiena, state cadendo giù verso di loro. --
	Aveva visto la posa in cui Ender agiva, ma non aveva capito il diverso
	orientamento che essa implicava. A Ender fu subito chiaro che per quanto
	Dink fosse esperto e molto in gamba, la tenacia con cui restava
	attaccato all'orientamento gravitazionale del corridoio anche in sala di
	battaglia limitava la sua mentalità.}

{Fecero pratica d'attacco contro una stella tenuta dal nemico. Prima di
	sperimentare il sistema di Ender a piedi in avanti, s'erano sempre
	spinti in volo in posizione «eretta», con l'intero corpo esposto ai
	colpi. A quel modo non ebbero difficoltà a conquistare la stella con una
	manovra agile ed efficiente. -- In alto, adesso! -- gridò Dink, e il
	branco balzò verso il «soffitto». A suo credito, tuttavia, volle far
	ripetere l'esercizio ordinando: -- A piedi in avanti, forza! -- Ma a
	causa del loro inconscio collegamento a parametri gravitazionali che non
	esistevano, i ragazzi eseguirono la manovra con goffaggine, come se il
	vuoto che avevano sotto i piedi desse loro le vertigini.}

{Detestavano quel modo di andare all'attacco. Dink insisté che era
	pratico e dovevano usarlo. E come risultato essi detestarono Ender. --
	C'è bisogno che venga un novellino a insegnarci a volare? -- brontolò
	uno di loro, a voce alta perché anche Ender sentisse. -- Pare di sì --
	rispose Dink. I ragazzi continuarono a lavorare.}

{E impararono. Nelle scaramucce pratiche cominciarono a capire quanto
	fosse più difficile colpire un avversario che arrivava a piedi in
	avanti. Non appena si furono convinti di questo, eseguirono le manovre
	molto più volentieri.}

{Quella era la prima sera in cui Ender usciva da un intero pomeriggio di
	addestramento. All'arrivo di Alai era stanco.}

{-- Ora che sei in un'orda -- osservò l'amico, -- non hai bisogno di far
	pratica con noi.}

{-- Da voi posso imparare cose che nessuno sa -- disse Ender.}

{-- Dink Meeker è il migliore. Ho sentito dire che sei nel suo branco.}

{-- Perciò diamoci da fare. Vi insegnerò quello che oggi ho imparato da
	lui.}

{Guidò Alai e due dozzine di altri attraverso le stesse esercitazioni
	che nel pomeriggio l'avevano sfibrato. Ma aggiunse particolari nuovi
	agli schemi; costrinse i ragazzi a tentare manovre con una gamba
	congelata, o con tutte e due, e ad usare la massa di un soldato già
	immobilizzato come appoggio per cambiare direzione.}

{A un certo punto, voltandosi, notò che Dink e Petra erano insieme sulla
	porta della sala e stavano guardando. Più tardi, quando si girò di
	nuovo, i due se n'erano andati.}

\emph{{Così mi stanno sorvegliando, e quel che faccio è risaputo}}{,
	\emph{} pensò. Non sapeva se Dink fosse suo amico o meno; supponeva che
	Petra lo fosse, ma non era certo di niente. Avrebbero potuto essere
	irritati nel vederlo indossare i panni di capobranco o addirittura di
	comandante intento ad addestrare i suoi uomini. Oppure offesi, trovando
	che un soldato preferiva la compagnia dei novellini. A disagio rifletté
	che i suoi rapporti coi ragazzi più anziani non sarebbero mai stati
	facili.}

{-- Credevo d'averti ordinato di tenere il tuo banco sotto naftalina,
	pupo -- disse Rose de Nose, fermandosi accanto alla sua cuccetta.}

{Ender non alzò lo sguardo. -- Sto finendo il compito di trigonometria
	per domani.}

{Rose appoggiò un ginocchio sullo schermo. -- Credi di poter prendere
	sottogamba i miei ordini?}

{Ender depose il banco sul letto e si alzò. -- Credo di aver bisogno
	della trigonometria più di quanto ho bisogno di te.}

{Rose era almeno venti centimetri più alto di lui, ma questo non lo
	preoccupava particolarmente. Non si sarebbe giunti alla violenza fisica,
	e anche il tal caso lui avrebbe potuto difendersi. Rose era un pigro, e
	non conosceva le tecniche di combattimento individuale.}

{-- Scenderai molto in classifica, ragazzo. -- Rose scosse il capo.}

{-- Era previsto. Stavo in cima alla lista solo perché l'orda delle
	Salamandre mi ha usato nel modo più stupido.}

{-- Stupido? La strategia di Bonzo gli ha fatto vincere una partita
	chiave.}

{-- La strategia di Bonzo non gli farebbe vincere una partita di
	ravanelli in scatola. Lui mi aveva messo fuori. Estraendo la pistola dal
	fodero ho violato i suoi ordini.}

{Rose non ne era stato al corrente. La rivelazione lo irritò. -- Così
	tutto quello che Bonzo ha detto di te era una bugia. Non sei né svelto
	né competente\ldots{} e inoltre disubbidisci agli ordini.}

{-- Ma ho trasformato una disfatta in un pareggio, e da solo.}

{-- Be', vedremo come fai a vincere una partita da solo, la prossima
	volta. -- Rose si allontanò.}

{Uno dei compagni di branco guardò Ender e scosse il capo. -- Solo lo
	sciocco sputa nel piatto dove mangia.}

{Ender si volse a controllare Dink, che stava disegnando sul proprio
	banco. Come se lo fosse aspettato Dink alzò gli occhi e gli restituì in
	silenzio uno sguardo fermo. Nessuna espressione, nessun cenno.
	\emph{Benissimo}, \emph{} pensò Ender. \emph{So prendermi cura di me
		stesso.}}

{Due giorni dopo ci fu una battaglia. Era la prima volta che Ender si
	batteva come parte di un branco, e questo lo rendeva nervoso. I ragazzi
	di Dink si allinearono sul lato destro del corridoio, e luì cercò di
	imitarne l'atteggiamento sicuro e noncurante. \emph{Almeno fingi},
	\emph{} si disse a denti stretti.}

{-- Wiggin! -- lo chiamò Rose de Nose.}

{Ender sentì la tensione bloccargli d'un tratto la gola, e una goccia di
	sudore gli scivolò lungo una guancia. Rose la notò.}

{-- Tremante? Sudato? Non bagnare la tua tuta nuova, pivello. -- Rose
	gli batté un dito sul calcio della pistola, poi lo spinse verso il campo
	di forza che celava alla vista l'interno della sala di battaglia. --
	Adesso vedremo quanto sai esser bravo, Ender. Appena questa porta si
	apre, tu schizzi dentro e fili dritto avanti verso la porta nemica. OK?}

{Un suicidio. Autodistruzione immotivata e senza significato. Ma lui
	doveva eseguire gli ordini, quella era una battaglia e non una seduta di
	allenamento. Per un attimo l'ira gli fece stringere i denti, poi si
	costrinse alla calma. -- Eccellente, signore. -- Annuì. -- La direzione
	in cui sparerò sarà quella del loro contingente principale.}

{Rose sorrise ampiamente. -- Sparare? Non ti daranno neppure il tempo di
	sputare, bambino.}

{Il muro d'energia svanì. Ender balzò in alto, si aggrappò al corrimano
	superiore e con una torsione puntò i piedi in «basso», poi si spinse
	verso la porta nemica.}

{Avevano di fronte l'orda dei Millepiedi, e i soldati stavano appena
	cominciando a uscire dalla loro porta quando Ender era già a mezza via
	nella sala di battaglia. Molti di loro furono svelti a saltare al riparo
	delle stelle, ma lui aveva ripiegato le gambe sotto di sé e, con la
	pistola fissa nel varco fra le ginocchia per assicurarsi la mira, sparò
	un colpo dopo l'altro centrando gli avversari al momento del loro
	ingresso nel locale.}

{Gli congelarono le gambe, cosa che servì soltanto a regalargli altri
	preziosi secondi prima d'arrivare sotto il fuoco di quelli allargatisi
	ai lati. Ne colpì ancora diversi, quindi allargò le braccia in croce,
	puntando quello armato verso il grosso dell'orda dei Millepiedi. Fece
	fuoco sui loro corpi in rapido spostamento, e subito dopo una gragnuola
	di colpi lo congelò.}

{Un secondo più tardi andò a sbattere in pieno sul campo di forza della
	porta nemica, che lo rispedì indietro roteante come una trottola. Ormai
	inerme finì in mezzo a un branco di avversari attestati dietro una
	stella, e uno di loro lo tolse di mezzo con un calcio che lo fece
	roteare ancor più velocemente. Per il resto della battaglia rimbalzò qua
	e là, mentre la frizione dell'aria lo faceva rallentare poco a poco. Non
	aveva modo di sapere quanti Millepiedi fosse riuscito a metter fuori
	causa, ma poté stabilire che l'orda dei Topi stava comunque vincendo,
	come al solito.}

{Dopo la battaglia Rose non gli disse verbo. Ender risultava sempre
	primo nella classifica dell'efficienza, poiché ne aveva congelati tre,
	disabilitati interamente due, e parzialmente altri sette. Non vi furono
	più accenni al suo comportamento insubordinato, né proibizioni di usare
	il banco. Rose restò nella sua zona della camerata e lo lasciò in pace.}

{Dink cominciò a sperimentare la tattica dell'uscita istantanea dal
	corridoio; l'attacco di Ender mentre il nemico era ancora in fase
	d'ingresso era stato giudicato devastante. -- Se un solo uomo può fare
	tanto danno, pensate cosa riuscirebbe a ottenere un branco. -- Dink
	convinse il maggiore Anderson a far aprire una porta nel centro di una
	parete, nelle sedute di allenamento, al posto di quella a livello del
	«pavimento», per esercitarsi alle uscite di slancio in condizioni di
	battaglia. La voce si sparse subito. Da quel giorno in poi nessuno
	avrebbe concesso ai suoi uomini di uscire in campo con tutta calma. Le
	gare erano cambiate.}

{Ci furono altre battaglie. Ender vi partecipò svolgendo il suo ruolo
	come parte del branco. Commise degli errori. Parecchi scontri lo videro
	soccombere. Nella classifica scese dapprima al secondo posto, poi al
	quarto. Ma più imparava come porre rimedio ai suoi sbagli, più si
	adattava e si affiatava al branco, e riuscì a risalire al terzo posto,
	quindi al secondo e di nuovo al primo.}

{~}

\begin{center}
	{* * *}
\end{center}

{~}

{Un pomeriggio, dopo gli allenamenti, Ender si trattenne in sala di
	battaglia. Aveva notato che Dink Meeker arrivava invariabilmente a cena
	con un po' di ritardo, e s'era detto che il capobranco si dedicava a un
	addestramento extra di qualche genere. Non aveva una gran fame, ed era
	curioso di sapere come Dink si allenava quando nessuno poteva vederlo.}

{Ma Dink non fece assolutamente nulla. Rimase fermo accanto alla porta,
	lo sguardo fisso su Ender.}

{Dal centro del vasto locale lui lo osservò in silenzio.}

{Nessuno dei due disse parola. Era chiaro che Dink aspettava l'uscita di
	Ender. E altrettanto chiaramente lui gli stava comunicando che non se ne
	sarebbe andato.}

{Dink allora gli volse le spalle, con gesti metodici si tolse la tuta da
	battaglia e poi si diede una spinta leggera, fluttuando via dal
	pavimento. Il suo volo lentissimo, fluido, lo portò attraverso la sala
	immersa nella penombra, col corpo quasi del tutto rilassato e le braccia
	mollemente distese quasi a cogliere il respiro delle inavvertibili
	correnti d'aria.}

{Dopo la fatica e la tensione degli esercizi, le imprecazioni, gli
	ordini e le manovre concitate, guardarlo galleggiare a quel modo era
	perfino riposante. Dink impiegò almeno dieci minuti per raggiungere la
	parete opposta. Infine si spinse indietro con uno scatto rapido, tornò
	dove aveva lasciato la tuta e la indossò.}

{-- Andiamo -- disse a Ender.}

{Tornati in camerata trovarono il locale vuoto, poiché tutti i ragazzi
	erano a mensa. I due andarono ai loro armadietti e misero tute da fatica
	pulite, quindi Ender ripassò accanto alla cuccetta di Dink e si fermò ad
	attendere che anch'egli fosse pronto.}

{-- Perché mi hai aspettato? -- domandò Dink.}

{-- Non ho molta fame.}

{-- Be', ora sai perché non sono un comandante.}

{Ender se l'era già chiesto.}

{-- In realtà mi hanno promosso, due volte, ma ho rifiutato.}

{Rifiutato? si stupirono gli occhi di Ender.}

{-- Mi hanno tolto ogni volta la cuccetta, gli armadietti e il banco, mi
	hanno assegnato una cabina da comandante e mi hanno dato un'orda. Ma io
	sono rimasto nel mio alloggio, finché non si sono rassegnati a
	rimandarmi di nuovo in un'orda come subordinato.}

{-- Perché?}

{-- Perché non voglio che mi manovrino fino a questo punto. Non credo
	che tu abbia già saputo guardare in fondo a questa situazione, Ender. Ma
	tu sei ancora ingenuo. Tutte le altre orde, non sono loro il nemico. I
	nostri nemici sono gli insegnanti. Riescono a farci combattere l'uno
	contro l'altro, a farci odiare l'un l'altro. Tutto è gara. Vincere,
	vincere, vincere. E dietro questo c'è il niente. Ci ammazziamo a
	vicenda, diventiamo matti per battere questo o quell'avversario, e per
	tutto il tempo quei vecchi bastardi ci sorvegliano, ci studiano,
	scoprendo i nostri punti deboli, decidendo se siamo \emph{abbastanza
		bravi} o no. Be', abbastanza bravi per cosa? lo avevo sei anni quando mi
	hanno portato qui. Cosa diavolo potevo sapere? \emph{Loro} decisero che
	io ero adatto al programma in corso, ma nessuno mi ha mai domandato se
	il programma era adatto a me.}

{-- Allora perché non torni a casa?}

{Dink ebbe un sorriso storto. -- Perché io non mi arrendo a metà gara.
	-- Palpeggiò il tessuto della sua tuta da battaglia, distesa sulla
	cuccetta. -- Perché amo tutto questo.}

{-- Se è così, perché non essere un comandante?}

{Dink scosse il capo. -- Mai. Guarda quel che ha fatto a Rose. Il
	ragazzo è matto. Rose de Nose. Dorme qui con noi invece che nella sua
	cabina. E sai perché? Perché ha paura della solitudine, Ender. Ha paura
	del buio.}

{-- Rose?}

{-- Ma loro lo hanno fatto comandante, e così deve comportarsi come se
	lo fosse \emph{davvero.} E non sa cosa sta facendo qui. Vince le
	partite, e questo lo spaventa più di qualunque altra cosa, dato che non
	sa \emph{perché} le vince, salvo che io ho qualcosa a che fare col
	risultato. Teme che da un momento all'altro qualcuno possa scoprire che
	lui non è una sorta di magico generale israeliano. Non si chiede neppure
	perché ci lasciano accanire tanto in queste gare. Nessuno se lo chiede.}

{-- Questo non significa che sia matto, Dink.}

{-- Lo so, tu sei qui da appena un anno e credi che questi ragazzi siano
	normali. Be', non lo sono. \emph{Noi} non lo siamo. Io frugo in
	biblioteca, e chiedo dei libri sul mio banco. Libri vecchi, perché non
	ci permettono di consultare roba recente; comunque mi è bastato per
	avere un'idea di ciò che è un ragazzino. E noi non siamo dei ragazzini.
	Quelli possono perdere qualche volta, e a nessuno importa. I ragazzini
	non vengono chiamati alle armi, non diventano comandanti, non
	spadroneggiano su più di quaranta altri della loro età. Questo supera
	ciò che chiunque possa sopportare senza diventare un po' pazzo.}

{Ender cercò di rammentare quali altri bambini, nella sua vecchia scuola
	e in città, erano di quel genere. Ma il solo a cui poté paragonarli fu
	Stilson.}

{-- Io avevo un fratello. Un tipo proprio normale. L'unica cosa che gli
	importava erano le ragazze. E il volo. Voleva volare. Gli piaceva anche
	giocare a pallone\ldots{} qualche partitella, far rimbalzare la palla
	contro il muro, dribblare e correre su e giù per i corridoi della città,
	finché un agente della quiete non gli sequestrava il pallone. Insieme ce
	la spassavamo. Mi stava insegnando a dribblare, quando fui arruolato.}

{Ender ripensò al proprio fratello, e non si trattò di un ricordo molto
	consolante.}

{Dink fraintese l'espressione del suo volto. -- Ehi\ldots{} so che qui
	nessuno parla di casa. Ma noi proveniamo da un \emph{luogo}, \emph{} no?
	La Scuola di Guerra non ci ha partorito. Semmai ci distrugge. E tutti
	quanti ricordiamo le cose di casa nostra. Forse non volentieri, a volte,
	ma le ricordiamo e poi davanti agli altri fingiamo che\ldots{} senti,
	Ender, perché fanno in modo che \emph{nessuno} parli \emph{mai} di casa?
	Questo non ti fa pensare che la cosa abbia un'importanza? Ci manovrano
	in modo che nessuno osa ammettere\ldots{} ah, al diavolo anche te!}

{-- No, aspetta -- lo corresse Ender. -- Stavo solo pensando a
	Valentine. Mia sorella.}

{-- Scusa. Non volevo metterti di cattivo umore.}

{-- Non fa nulla. Non ho pensato molto a lei, ultimamente, e proprio
	perché sto diventando\ldots{} come hai detto tu.}

{-- Già. E non piangiamo mai. Cristo, a questo non avevo mai pensato.
	Stiamo davvero mettendocela tutta per essere adulti. Come i nostri
	padri. Scommetto che tuo padre era come te, eh? Un bambino tranquillo,
	paziente, ma capace di\ldots{}}

{-- No, io non sono come mio padre.}

{-- Be' forse dico delle sciocchezze. Ma guarda Bonzo, il tuo ex
	comandante: si è praticato da solo un'overdose di antico onore spagnolo.
	Non può concedere a se stesso un attimo di debolezza. E chi riesce
	meglio di lui, lo sta insultando. Ma essere forte a quel modo è come
	tagliarsi le palle. Ecco perché ti odia: quando cercava di punirti tu
	non ne soffrivi. Così ti odia, e gli sembra normale desiderare di
	ammazzarti. È un pazzo. Tutti sono pazzi.}

{-- E tu no?}

{-- Sì, anch'io, ragazzino. Ma almeno, quando ho fatto un'indigestione
	di pazzia mi alzo in volo come un uccello nello spazio\ldots{} finché la
	pazzia non mi esce dalla pelle e va ad appiccicarsi ai muri. Ma il
	giorno dopo arrivano altre battaglie, e torme di ragazzi urlanti vanno a
	sbattere calci sulle pareti. E la pazzia ne schizza fuori e mi ritorna
	addosso.}

{Ender sorrise.}

{-- E anche tu sei pazzo -- disse Dink. -- Avanti, andiamo a mangiare.}

{-- Magari tu potresti essere un comandante senza essere un pazzo.
	Magari il fatto di conoscere questa pazzia ti impedirà di cascarci
	dentro.}

{-- Io non lascerò che quei bastardi mi manovrino, Ender. Sono riusciti
	a metterti sotto ben bene, e non hanno in programma di trattarti coi
	guanti. Guarda quello che ti hanno combinato finora.}

{-- Non mi hanno fatto niente, a parte darmi una promozione.}

{-- E questa ti ha reso la vita tanto dolce, eh?}

{Ender rise e scosse il capo. -- No, se la metti così.}

{-- Loro pensano di averti su un vassoio. Non permetterglielo.}

{-- Ma è per questo che sono venuto qui -- disse Ender. -- Per lasciare
	che mi trasformino in uno strumento. Per salvare il mondo.}

{-- Non mi capacito che tu creda ancora a queste cose.}

{-- Quali cose?}

{-- La minaccia degli Scorpioni. Salvare il mondo. Ascolta, Ender, se
	gli Scorpioni volessero tornare, sarebbero \emph{già qui.} Ma non ci
	stanno invadendo. Li abbiamo battuti, e loro se ne sono andati.}

{-- Ma i filmati che\ldots{}}

{-- Tutta roba della Prima e della Seconda Invasione. Quando Mazer li
	spazzò via, i tuoi nonni non erano ancora nati. Apri gli occhi. È tutta
	una commedia. Non c'è nessuna guerra, e la F.I. ci tiene qui per i suoi
	scopi.}

{-- Quali scopi?}

{-- Finché la gente avrà paura degli Scorpioni, la F.I. resterà in una
	posizione di potere, e finché deterrà il potere certe nazioni
	continueranno a esser governate come in passato. Ma guarda i
	telegiornali, Ender: presto la gente non vedrà più il motivo di questa
	alleanza, e ci saranno di nuovo guerre, forse anche quella che metterà
	fine a tutte le guerre. La minaccia è \emph{questa}, \emph{} Ender, non
	gli Scorpioni. E in \emph{questa} guerra, quando verrà, tu e io non
	saremo amici. Perché tu sei americano, proprio come i nostri cari
	insegnanti. E io non lo sono.}

{Andarono in sala mensa e cenarono, parlando d'altre cose. Ma Ender non
	poté impedirsi di continuare a riflettere su quel che Dink aveva detto.
	La Scuola di Guerra era un ambiente a tal punto chiuso, intorno a quei
	bambini così presi dalle gare, che lui dimenticava perfino l'esistenza
	del mondo esterno. Onore spagnolo. Guerre. Manovre politiche. Sì, la
	Scuola di Guerra era un posto ben piccolo al confronto.}

{Ma lui non poteva prendere per buone le conclusioni di Dink. Gli
	Scorpioni erano veri. La minaccia era reale. La F.I. controllava un
	sacco di cose, ma non la TV e la stampa. Non nella città in cui era
	nato. A casa di Dink, in Olanda, dopo tre generazioni di egemonia
	sovietica forse tutto era controllato. Ma suo padre aveva detto spesso
	che le bugie non potevano durare a lungo in America. E lui ci credeva.}

{Ci credeva, anche se il seme del dubbio era lì, ma del tutto inerte, e
	ogni tanto metteva fuori una piccola radice. Era un seme che nel
	crescere stava causando dei mutamenti. Lo rese più attento al
	significato dei discorsi altrui che alle loro parole. Lo rese più
	saggio.}

{~}

\begin{center}
	{* * *}
\end{center}

{~}

{Quella sera non c'erano molti ragazzi al solito allenamento, neppure la
	metà. -- Dov'è Bernard? -- s'informò Ender.}

{Alai si limitò a sogghignare. Shen alzò gli occhi al cielo e assunse
	un'aria di meditazione ispirata.}

{-- Non te l'hanno detto? -- intervenne un altro, un novellino di un
	gruppo arrivato un paio di mesi prima. -- Corre voce che chi viene a
	imparare con te poi non combina niente di buono nell'orda di qualcun
	altro. Dicono che i comandanti non vogliono soldati che siano stati
	rovinati dai tuoi allenamenti.}

{Ender annuì.}

{-- Ma io non me ne curo -- continuò il ragazzino. -- Voglio diventare
	il miglior soldato che ci sia, e allora un comandante che abbia un
	grammo di cervello pregherà per avermi. No?}

{-- Sicuro. -- Ender esibì un'aria convinta.}

{Cominciarono a lavorare di lena. Dopo circa mezz'ora, mentre stavano
	addestrandosi a manovrare i corpi congelati altrui per farsene scudo,
	numerosi comandanti vestiti di uniformi diverse entrarono in sala.
	Ostentando un'aria grave presero il nome a tutti.}

{-- Ehi! -- gridò Alai quando se ne andarono. -- Siete sicuri di aver
	scritto bene il mio nome?}

{La sera dopo i ragazzi presenti erano ancora meno. E agli orecchi di
	Ender stavano giungendo voci preoccupanti: bambini del gruppo appena
	arrivato gettati a terra nelle docce, presi a spinte in sala giochi,
	sottomessi a soprusi in qualche corridoio, e le registrazioni dei
	compiti di scuola nei loro banchi cancellate o rovinate da ragazzi più
	anziani che sapevano come inserirsi nel computer.}

{-- Stasera niente esercizi -- disse Ender.}

{-- Niente esercizi col cavolo! -- si oppose Alai.}

{-- Diamogli soddisfazione per qualche giorno. Non voglio che facciano
	del male a questi ragazzini.}

{-- Se la smettiamo, anche per una sola sera, si convinceranno che le
	prepotenze di questo genere funzionano. Proprio come se tu fossi rimasto
	zitto e buono quando Bernard ti prendeva a pugni in testa.}

{-- Inoltre -- aggiunse Shen, -- qualunque cosa facciano, noi non
	abbiamo paura. Perciò dobbiamo continuare. Abbiamo bisogno di pratica, e
	tu anche.}

{Ender ripensò a quel che aveva detto Dink. Le gare erano irrilevanti a
	confronto del resto del mondo. Perché qualcuno avrebbe dovuto regalare
	tutte le serate della sua vita a quello stupido, stupidissimo gioco?}

{-- Pochi come siamo, non concluderemmo molto in ogni modo -- disse
	Ender avviandosi all'uscita.}

{Alai lo prese per un gomito. -- Ti hanno messo paura? Ti hanno pestato
	nelle docce? Ti hanno ficcato la testa nel gabinetto? I ragazzi della
	tua orda ti sparano alla schiena quando nessuno li vede?}

{-- No -- disse Ender.}

{-- Sei ancora mio amico? -- chiese Alai sottovoce.}

{-- Sì.}

{-- Allora restiamo uniti, Ender. Io starò qui e mi allenerò con te.}

{I ragazzi più anziani tornarono a curiosare, ma pochi di loro erano
	comandanti di un'orda. Nel gruppo che venne dentro Ender vide alcune
	uniformi delle Salamandre, e anche un paio di Topi. Stavolta non presero
	nomi. Ridacchiarono, si diedero di gomito l'un l'altro e cominciarono a
	far battute pesanti ad alta voce, deridendo gli sforzi dei ragazzini più
	giovani che compivano esercizi coi loro muscoli non allenati. Non pochi
	di essi ne furono umiliati; qualcuno accennò a smettere.}

{-- Ascoltate bene quello che dicono -- intervenne Ender. -- Annotatevi
	le loro parole. Vi saranno utili per quando vorrete far uscire dai
	gangheri il vostro avversario. Noi invece sappiamo mantenere la calma,
	no?}

{Shen volle sviluppare quel concetto, e ad ogni comparsa dei
	sogghignanti spettatori preparò un gruppetto di novellini per ripetere
	in coro le frasi più offensive. Quando ci presero gusto e acquistarono
	ritmo, quel coro intercalato da ululati sarcastici divenne così
	sfottente che alcuni dei ragazzi più anziani si spinsero via dalla
	parete e vennero avanti per battersi.}

{Le tute da battaglia erano confezionate per combattimenti a impulsi
	luminosi; offrivano scarsa protezione nelle lotte corpo a corpo in
	gravità zero, oltre ad ostacolare molto i movimenti. Metà dei ragazzi di
	Ender, tuttavia, indossavano tute di quel genere e non potevano lottare
	a mani nude. Ma la rigidità del tessuto li rendeva potenzialmente utili.
	In fretta lui ordinò ai novellini di radunarsi in un angolo della sala.
	I ragazzi più anziani risero di quella mossa, e altri nel vedere che il
	gruppetto si ritirava lasciarono la parete per unirsi agli attaccanti.}

{Ender e Alai decisero di proiettare un soldato congelato in faccia a un
	avversario. Il ragazzo prescelto usò la pistola su se stesso, abbassò
	l'elmo sul volto, e i due lo scaraventarono avanti. L'avversario fu
	colpito dal casco in pieno petto, e rantolò di dolore.}

{Nessuno scherzava più, adesso. Il resto dei ragazzi anziani si lanciò
	in volo verso la zona della battaglia. Ender non aveva troppe speranze
	che i suoi compagni se la cavassero senza ferite, forse anche serie. Ma
	il nemico li aggrediva in disordine e senza alcuna coordinazione: non
	avevano mai lavorato insieme, mentre la piccola orda di Ender, benché
	composta da appena una dozzina di elementi, aveva già una serie di
	schemi pronti per le manovre di gruppo.}

{-- Quattro-Tre-Nova! -- gridò Ender. Gli avversari risero. I suoi
	ragazzi formarono tre gruppi, coi piedi uniti e tenendosi per mano,
	simili a piccole stelle a contatto della parete di fondo. -- Aggirare
	gli avversari e raggiungere la porta. Pronti\ldots{} adesso!}

{Al segnale le tre stelle esplosero, mentre ciascuno dei quattro
	componenti schizzava via in una direzione diversa per rimbalzare sulle
	pareti laterali e raggiungere la porta. I loro assalitori si trovavano
	al centro del locale, dove mutare direzione era assai più difficoltoso,
	e oltrepassarli fu una manovra facile.}

{Ender aveva calcolato la sua posizione in modo che la spinta lo
	portasse a raggiungere il ragazzo congelato che s'era lasciato usare
	come un missile. Ora la sua tuta s'era di nuovo ammorbidita, e appena
	l'ebbe preso Ender sfruttò il proprio momento d'inerzia per spedirlo
	verso la porta. Sfortunatamente l'inevitabile risultato fu che lui venne
	respinto dalla parte opposta, e a velocità ridotta. Isolato dai suoi
	soldati stava ora fluttuando in direzione del fondo della sala, dove gli
	avversari s'erano riuniti. Si girò e controllò che i suoi compagni
	fossero giunti senza danni nei pressi dell'ingresso.}

{Ma intanto gli altri, furibondi e disorganizzati, s'erano accorti di
	lui. Ender cercò di calcolare quanti secondi aveva a disposizione per
	arrivare alla parete e spingersi via. Non abbastanza. E parecchi
	avversari già rimbalzavano verso di lui. Per un attimo fu sgomento nel
	vedere fra i loro volti quello di Stilson. Poi, con un brivido, capì che
	s'era trattato di uno scherzo della fantasia. Ma la situazione non era
	poi troppo diversa, con la differenza che stavolta non poteva risolverla
	con un duello. Quei ragazzi non avevano un capobanda, almeno per quanto
	ne sapeva lui, ed erano tutti più grossi e più forti.}

{Tuttavia qualcosa aveva imparato sui combattimenti corpo a corpo in
	assenza di peso, e sulla meccanica degli oggetti in movimento inerziale.
	Nelle partite in sala di battaglia non c'era bisogno di quelle tecniche;
	un soldato non si gettava in mezzo a un gruppo di avversari non
	congelati per colpirli a mani nude. Così, nei pochi secondi che gli
	restavano, cercò d'assumere la posizione migliore per accogliere gli
	assalitori.}

{Per sua fortuna essi conoscevano la lotta a zero G ancor meno di lui, e
	i pochi che tentarono di prenderlo a pugni scoprirono che i colpi
	avevano ben scarso effetto, dal momento che i loro corpi si muovevano
	all'indietro nell'istante stesso in cui facevano scattare avanti un
	braccio. Ma alcuni stavano arrivando a gambe tese, chiaramente
	intenzionati a spaccargli una costola con una pedata, e Ender si disse
	che doveva togliersi via al più presto dal loro punto d'impatto.}

{Afferrò per un polso un ragazzo che gli aveva appena mollato una
	sventola e lo tirò con forza verso di sé. Lo strattone servì a farlo
	roteare fuori portata dagli avversari in avvicinamento, ma lo allontanò
	ancor di più dalla porta. -- State dove siete! -- gridò ai compagni, che
	si preparavano ad accorrere in sua difesa. -- Non muovetevi da lì!}

{Qualcuno lo afferrò per un piede. La stretta gli servì da leva, e
	riuscì a piazzare sull'orecchio destro del ragazzo un pedatone che gli
	strappò un grido. Se l'avversario l'avesse lasciato andare per tempo il
	colpo gli avrebbe causato assai meno danni. Invece volle essere
	testardo: il calcio gli lacerò l'orecchio facendone sprizzare gocce di
	sangue, e soltanto il dolore lo costrinse infine a mollare la presa.}

\emph{{Lo sto facendo di nuovo}}{, \emph{} pensò Ender. \emph{Faccio del
		male agli altri, soltanto per salvare me stesso. Perché non mi lasciano
		in pace? Perché devono costringermi a questo?}}

{Altri tre ragazzi stavano convergendo su di lui, e stavolta agivano di
	concerto. La loro intenzione era di ancorarsi a lui e di colpirlo
	tenendolo fermo. Ruotò su se stesso in modo da consegnare i suoi piedi a
	due di loro, e avere le mani libere per affrontare il terzo.}

{Come aveva previsto, i due avversari gli agguantarono subito le gambe.
	Ender prese l'altro per le spalle della tuta, lo trasse a sé e lo colpì
	con una testata in piena faccia. Ancora un gemito, ancora gocce di
	sangue che fluttuavano attorno. Gli altri due lo stavano percuotendo sui
	fianchi e cercavano di girarlo. Ender sbatté loro addosso il ragazzo che
	perdeva sangue dal naso, scalciò più volte e le sue gambe furono libere.
	Poi fu solo questione di usare lo stesso avversario come punto di
	appoggio, e spingendolo via si proiettò in direzione della porta. La
	manovra non fu pulita e veloce come quelle eseguite in allenamento, e lo
	fece roteare in modo antiestetico, ma poco importava. Nessuno lo stava
	inseguendo.}

{Alla porta si trovò in mezzo ai compagni. Dieci mani lo presero e lo
	dirottarono nel corridoio. I ragazzi ridevano sollevati e gli davano
	grandi manate sulle spalle. -- Dannato bastardo! -- lo complimentarono.
	-- Razza di volpone! In gamba! Sei andato forte, amico!}

{-- Be', basta con l'addestramento, per oggi -- disse Ender.}

{-- Domani quelli torneranno -- pronosticò Shen.}

{-- Non otterranno quel che sperano -- disse Ender. -- Se verranno senza
	tute, finirà come oggi. Se avranno le tute da battaglia, li batteremo
	sulla velocità.}

{-- Però -- disse Alai, -- scommetto che gli insegnanti non lo
	permetteranno.}

{Ender tornò a ripensare alle parole di Dink, e si disse che forse Alai
	aveva visto giusto.}

{-- Ehi, Ender! -- gli gridò dietro uno dei ragazzi anziani, mentre lui
	se ne andava. -- Tu non sei nessuno, pivello. Sei zero!}

{-- È il mio ex comandante, Bonzo -- sospirò Ender. -- Sembra che io non
	gli sia simpatico.}

{Quella sera Ender chiese sullo schermo del suo banco il rapporto
	dell'infermeria. Quattro ragazzi s'erano presentati per ricevere cure.
	Uno con una costola incrinata, uno con un testicolo dolorante, uno con
	l'orecchio destro lacerato, e uno col naso rotto e un incisivo spezzato.
	La causa riferita al medico era la stessa nei quattro casi:}

{~}

\begin{center}
	{COLLISIONE ACCIDENTALE IN GRAVITÀ ZERO}
\end{center}

{~}

{Se gli insegnanti avallavano quel palese falso nelle registrazioni
	ufficiali, era ovvio che non intendevano prendere provvedimenti contro
	chi aveva partecipato alla zuffa in sala di battaglia. \emph{Possibile
		che non facciano niente? Non gli importa quel che succede in questa
		scuola?}}

{Visto che era tornato in camerata prima del solito, chiamò la partita
	libera sul suo banco. Da un po' di tempo non la giocava più, e forse per
	quel motivo la sua figura non cominciò nel posto in cui l'aveva
	lasciata. La vide prender forma presso il corpo del Gigante. Soltanto
	che adesso era a stento identificabile come un corpo, a meno che uno non
	indugiasse a esaminarlo. La massa mummificata s'era trasformata in una
	collinetta su cui crescevano erbacce e rampicanti. Il cranio era invece
	ancora riconoscibile per i tratti di osso nudo e bianco, simile a roccia
	gessosa levigata dalla pioggia.}

{Ender proseguì, aspettandosi di dover eliminare i bambini licantropi,
	ma giunto al parco giochi ebbe la sorpresa di trovarlo vuoto. Forse una
	volta uccisi restavano morti per sempre. Questo lo rese un po' triste.}

{Attraversò la foresta, scese nel pozzo, uscì dalla caverna piena di
	gemme e si trovò sul cornicione che sovrastava il meraviglioso panorama
	campestre. Di nuovo si gettò nel vuoto, la nuvoletta lo prese al volo e
	lo trasportò nella stanza in cima alla torre del castello.}

{Il serpente cominciò a sciogliere le sue spire dinnanzi al focolare, ma
	stavolta Ender non esitò: balzò sulla testa del rettile e la schiacciò
	sotto i piedi. La bestiaccia si contorse furiosamente, costringendolo a
	calpestarla a lungo, ma finalmente giacque immobile. Ender sollevò il
	serpente e lo scosse per controllare che non potesse tornare in vita.
	Poi, trascinandoselo dietro, cominciò a cercare se c'era una via
	d'uscita.}

{Trovò invece uno specchio. E in esso vide comparire una faccia che
	riconobbe all'istante. Era Peter. Sul suo mento ruscellavano gocce di
	sangue, e da un angolo della bocca gli sporgeva la coda di un serpente.}

{Con un grido di spavento Ender respinse il banco. I pochi ragazzi che
	c'erano in camerata si volsero di scatto, allarmati, e lui dovette
	scusarsi spiegando che non era successo nulla. Ma quando trasse di nuovo
	il banco a sé gli tremavano le mani. La sua figura era sempre nella
	stanza, davanti allo specchio. La fece voltare e cercò di usare un
	mobile per rompere il cristallo, ma non riuscì a spostarlo. Inutile fu
	anche il tentativo di staccare lo specchio dal muro. Alla fine Ender vi
	scaraventò contro il serpente. Lo specchio andò in frantumi e dietro di
	esso comparve un foro sbrecciato nei mattoni. Dall'apertura guizzarono
	fuori dozzine di serpentelli che si gettarono sulla figura di Ender,
	mordendola dappertutto. Strappandosi i rettili di dosso con movimenti
	frenetici la figura barcollò, cadde morta e fu ricoperta da un viluppo
	di forme verdi che la nascosero.}

{Lo schermo diventò nero, e apparve una scritta:}

{~}

\begin{center}
	{GIOCHI ANCORA?}
\end{center}

{~}

{Ender spense il banco e lo mise nell'armadietto.}

{Il giorno dopo parecchi comandanti vennero a stringere la mano a Ender,
	o mandarono uno dei loro soldati a dirgli che erano solidali con lui.
	Alcuni dichiararono che i suoi allenamenti extra una buona idea e che
	dovevano continuare. Per esser sicuri che nessuno avrebbe tentato
	soprusi si dissero disposti ad affidargli quei loro soldati che avevano
	bisogno di migliorare. -- E i miei sono grossi come quegli Scorpioni che
	vi hanno attaccato l'altra sera -- disse uno di loro. -- Adesso dovranno
	pensarci due volte.}

{Quella sera invece di dodici ragazzi ce n'erano quarantacinque, più dei
	componenti di un'orda. E sia che fosse per la presenza di quelli che
	avevano affiancato Ender, sia che la sera prima ne avessero avuto
	abbastanza, nessuno dei loro provocatori si fece vivo.}

{Ender non chiamò più sul suo banco la partita libera. Ma essa
	continuava a svolgersi nei suoi sogni, mista al ricordo di come aveva
	ucciso il Gigante, alla ferocia con cui aveva schiacciato il serpente e
	affogato i licantropi, ai calci che aveva dato a Stilson,
	all'indifferenza con cui aveva rotto un braccio a Bernard. E terminava
	col volto di Peter che lo fissava orribilmente dallo specchio.
	\emph{Questo gioco sa troppe cose di me. Questo gioco dice delle sporche
		bugie. Io non sono Peter. Io non ho l'istinto omicida nel mio cuore.}}

{Ma restava la paura più raggelante, il sospetto di \emph{essere} un
	killer, e perfino migliore dello stesso Peter. Il pensiero che proprio
	quella sua dote compiacesse maggiormente gli insegnanti. \emph{È di
		killer che hanno bisogno contro gli Scorpioni. Gente che può prendere il
		nemico a calci nei denti e far schizzare il suo sangue per tutto lo
		spazio.}}

\emph{{Be', io sono il vostro uomo. Sono io il bastardo sanguinario che
		volevate quando avete autorizzato la mia nascita. Io sono il vostro
		strumento, e che differenza fa se odio la parte di me della quale avete
		più bisogno? Che differenza fa se quando i serpentelli della partita mi
		hanno ucciso io ero d'accordo con loro, e ne ero contento?}}

\phantomsection\label{Orsonux20Scottux20Cardux20-ux20Ilux20Giocoux20Diux20Enderux20-ux20BY_SLY70A1_split_011.htm}{}
