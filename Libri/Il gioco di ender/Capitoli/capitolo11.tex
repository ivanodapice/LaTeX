\chapter{VENI VIDI VICI}

{~}

{~}

{~}

{-- \emph{Lei non può pensare sul serio di mettere in programma queste
		battaglie.}}

{-- Sì, \emph{che lo penso.}}

{-- \emph{Lavora con la sua orda da sole tre settimane e mezzo.}}

{-- \emph{Gliel'ho già detto: abbiamo eseguito simulazioni
		computerizzate per stimare i probabili risultati. E qui c'è quello che
		il computer prevede che Ender farà.}}

{-- \emph{Noi siamo qui per dargli un'istruzione, non un esaurimento
		nervoso.}}

{-- \emph{Il computer lo conosce meglio di noi.}}

{-- \emph{Il computer non è certo famoso per essere compassionevole.}}

{-- \emph{Se lei voleva comporre elogi alla compassione, avrebbe dovuto
		ritirarsi in un monastero.}}

{-- \emph{Ehi, sta dicendo che questo non è un monastero?}}

{-- \emph{Inoltre, per Ender è meglio così. Lo stiamo portando al meglio
		del suo potenziale.}}

{-- \emph{Pensavo che gli avremmo dato un paio d'anni come comandante.
		Di solito facciamo loro fare una battaglia ogni due settimane, a
		cominciare dalla fine del terzo mese. Così tiriamo troppo la corda.}}

{-- \emph{E li abbiamo questi due anni da gettar via?}}

{-- \emph{Lo so. Ma non riesco a togliermi dalla mente questa immagine
		di Ender da qui a un anno: completamente inutilizzabile, bruciato, dopo
		esser stato sottoposto a tensioni che né lui né altri potrebbero
		sopportare.}}

{-- \emph{Abbiamo premesso al computer che c'era un obiettivo
		prioritario: il soggetto deve mantenere l'efficienza psicofisica dopo il
		programma di addestramento.}}

{-- \emph{Mantenerla, certo, ma per quanto tempo\ldots{}}}

{-- \emph{Senta, colonnello Graff, è stato lei a chiedermi di preparare
		questo programma, e malgrado le mie proteste, se ricorda bene.}}

{-- \emph{Lo so, ha ragione, non dovrei scaricarle addosso i miei
		problemi di coscienza. Ma la mia brama di sacrificare dei bambinetti in
		vista della salvezza della razza umana si è alquanto assottigliata. Il
		Condottiero è andato a parlare con l'Egemone. Sembra che i burocrati
		russi siano preoccupati: alcuni giornalisti delle reti video stanno già
		esaminando i modi in cui l'America potrebbe usare la F.I. per smembrare
		il Patto di Varsavia, dopo che gli Scorpioni saranno stati sconfitti.}}

{-- \emph{Mi sembrano ipotesi premature.}}

{-- \emph{A me sembrano folli. La libertà d'informazione è una cosa, ma
		incitare gli Alleati a rivalità nazionalistiche\ldots{} ed è per gente
		come quella, dalla vista corta e paranoica, che stiamo spingendo Ender
		sull'orlo della sopportazione umana.}}

{-- \emph{Penso che lei sottovaluti Ender.}}

{-- \emph{La mia paura è che stiamo sottovalutando la stupidità del
		resto della razza umana. Siamo davvero sicuri di dover vincere questa
		guerra?}}

{-- \emph{Signore, questa è una domanda da corte marziale.}}

{-- \emph{Sono passato all'umorismo nero.}}

{-- \emph{Be', non è divertente. Quando si parla degli Scorpioni,
		niente\ldots{}}}

{-- \emph{Niente è divertente, lo so.}}

{~}

\begin{center}
	{* * *}
\end{center}

{~}

{Ender Wiggin era disteso sul letto, gli occhi fissi in alto. Da quando
	lo avevano promosso comandante non dormiva mai più di cinque ore per
	notte, anche se le luci si spegnevano alle 2200 e non venivano riaccese
	che alle 0600. Qualche volta lavorava ugualmente al suo banco, o
	sforzava gli occhi usandone per altri scopi la debole luminosità. Ma di
	solito lasciava vagare lo sguardo nel buio del soffitto e rifletteva.}

{O gli insegnanti erano stati dopotutto abbastanza generosi, o lui era
	un comandante migliore di quel che credeva. Nei suoi pochi e scalcinati
	veterani, vissuti senza infamia e senza lode nelle loro precedenti orde,
	stavano sbocciando doti di prim'ordine. Così, invece dei consueti
	quattro branchi, lui ne aveva creati cinque, ciascuno con un capo e un
	vice; ogni veterano in una posizione di responsabilità.
	Nell'addestramento disponeva di cinque branchi di otto elementi oppure
	di dieci mezzi branchi, cosicché a un singolo comando in codice l'orda
	poteva condurre un massimo di dieci manovre separate all'interno di
	un'unica manovra tattica. Nessun'orda s'era mai frammentata in quel modo
	prima d'allora, ma d'altronde Ender non meditava di fare cose già note
	ad altri. Molte orde praticavano manovre di massa, strategie ampiamente
	collaudate. Ender non predeterminava i particolari, anzi addestrava i
	suoi capibranco perché usassero le loro piccole unità contro obiettivi
	limitati, senza aiuto, da soli, obbligati ad agire di propria
	iniziativa. Dopo la prima settimana aveva messo in scena battaglie dallo
	svolgimento confuso, selvaggi scontri scimmieschi che lasciavano esausti
	i soldati e senza più voce i capibranco. Ma ora sapeva, dopo neppure un
	mese di lavoro, che la sua orda aveva un potenziale umano capace di
	trasformarla nel miglior gruppo combattente fra quelli già in
	classifica.}

{Quanto di tutto questo era parte del piano degli insegnanti? Erano al
	corrente di avergli assegnato ragazzi oscuri ma eccellenti? Gli avevano
	gettato fra le braccia trenta novellini, molti dei quali sotto il limite
	d'età, perché sapevano quale rabbia repressa vi fosse nei pivelli
	invidiosi dei più grandi e bramosi di portarsi alla loro altezza? O
	questo era ciò che succedeva a ogni gruppo simile, se dato in mano a un
	comandante che sapeva cosa voleva da loro e come costringerli a
	impararlo?}

{Quegli interrogativi lo preoccupavano, perché non sapeva se stava
	confondendo le idee a degli insegnanti ostili oppure esaudendo le
	aspettative di insegnanti molto astuti.}

{Il solo elemento di cui era certo era la sua impazienza di battersi.
	Molte orde avevano bisogno di tre mesi per il solo fatto che dovevano
	memorizzare dozzine di elaborate tattiche. \emph{Be', noi siamo già
		pronti. Mandateci in battaglia.}}

{Nelle tenebre la porta si aprì silenziosamente. Ender tese gli orecchi:
	un passo soffocato. Il battente fu richiuso.}

{Rotolò fuori dal letto e si mosse lentamente verso il lato opposto
	della camera; ma soltanto due minuti dopo, quando avvertì la presenza di
	un foglio di carta sotto un piede, capì che non era entrato nessuno. Lo
	raccolse. Naturalmente non poté leggerlo, ma sapeva che genere di ordini
	conteneva. Battaglia. \emph{Ma quanto sono gentili. Esprimi un
		desiderio, e loro te lo realizzano all'istante.}}

{Quando le luci si accesero Ender indossava già la tuta da battaglia
	dell'orda dei Draghi. Uscì subito in corridoio, e alle 0601 era alla
	porta della camerata della sua orda.}

{-- Uomini, fra un'ora sapremo se stiamo qui dentro per suonare o per
	essere suonati. Abbiamo una battaglia contro l'orda delle Lepri alle
	sette in punto. Vi voglio scaldati a gravità normale e pronti a scendere
	in campo. Tutti nudi come vermi e dritti in palestra, con la tuta da
	battaglia sottobraccio. Ci vestiremo dopo la ginnastica. E la
	colazione?}

{-- Non vogliamo che qualcuno si metta a vomitare in sala di battaglia,
	no?}

{C'era almeno il tempo di andare a fare un po' d'acqua?}

{-- Non più di un decilitro ciascuno, razza di perditempo!}

{I ragazzi risero. Quelli che non avevano l'abitudine di dormire nudi si
	spogliarono; tutti arrotolarono la tuta da battaglia e seguirono Ender
	di corsa lungo i corridoi fino in palestra. Li fece passare due volte
	sul percorso a ostacoli, quindi alla scala svedese e alle parallele, ma
	con tutta calma. -- Non affaticatevi, andate in scioltezza, dovete
	soltanto scaldarvi -- ordinò, ma non temeva che si stancassero. Erano in
	buona forma, agili e leggeri, e il suo repertorio di frasi salaci una
	volta tanto li eccitava. Alcuni cominciarono spontaneamente a lottare;
	la palestra, di solito tediosa, nell'imminenza della battaglia diventava
	all'improvviso un posto divertente. \emph{La loro è l'euforica sicurezza
		di chi non ha mai sbattuto la faccia nelle delusioni di una battaglia, e
		pensano d'esser pronti. Be', perché non dovrebbero pensarlo? Lo sono. E
		io anche.}}

{Alle 0640 ordinò che indossassero le tute da battaglia. Parlò poi con i
	capibranco e i loro vice mentre si vestivano. -- Le Lepri sono per lo
	più veterani, ma Carn Carby è il loro comandante da soli cinque mesi e
	non ho mai combattuto contro quest'orda con lui alla guida. È stato un
	soldato di ottima levatura, e le Lepri sono ormai da anni nella zona
	alta della classifica. Ma mi aspetto di vederli manovrare in formazione
	standard, perciò non sono molto preoccupato.}

{Alle 0650 li fece stendere tutti sui materassini per un training
	autogeno basato sulla musica e sulla respirazione. Alle 0656 ordinò
	l'uscita e correndo con leggerezza sfilarono nei corridoi verso la sala
	di battaglia. Ogni tanto Ender balzava in alto a toccare il soffitto, e
	i ragazzi dietro di lui lo imitavano battendo una mano nello stesso
	punto esatto. La loro striscia di luce colorata girò a sinistra; l'orda
	delle Lepri era già passata di lì, svoltando a destra. E alle 0658
	furono davanti alla porta chiusa della sala di battaglia.}

{I branchi si allinearono su cinque colonne. A ed E erano pronti ad
	afferrare i corrimano esterni per proiettarsi ai lati. B e D avrebbero
	usato quello superiore per spingersi verso l'\emph{alto} nell'ambiente a
	gravità zero. Il branco C si sarebbe tuffato sul corrimano inferiore per
	balzare nella direzione opposta.}

{Su, giù, destra, sinistra; Ender li fronteggiò, stando fra due delle
	colonne per non essere d'ostacolo, e li orientò nel solito modo: --
	Soldati, dite al nostro Bean dov'è la porta del nemico. Da che parte?}

{-- In basso! -- gridarono tutti, ridendo. E in quel momento \emph{su}
	diventò nord, \emph{giù} diventò sud, e la \emph{destra} e la
	\emph{sinistra} divennero est e ovest.}

{La parete grigiastra davanti a loro si dissolse, e l'interno della sala
	di battaglia fu visibile. Non sarebbe stato un combattimento al buio, ma
	non si poteva dire che la luce fosse molta: c'era una fosca penombra in
	cui tutto sembrava nebuloso. In distanza Ender vide la porta nemica, da
	cui già scattavano fuori le forme appena fluorescenti degli avversari.
	Per un attimo questo lo fece sorridere: tutti avevano imparato la
	lezione, dopo che Bonzo aveva fatto un uso malaccorto di Ender Wiggin, e
	continuavano ad applicarla. Si sparavano fuori dalla porta come razzi,
	cosicché c'era soltanto il tempo di gridare il codice della formazione
	in cui manovrare. Le battaglie iniziavano senza che i comandanti
	avessero il tempo di pensare. Ma Ender ora voleva prendersi quel tempo,
	e confidava nella capacità dei suoi soldati di combattere con le gambe
	congelate perché uscissero dalla porta intatti, malgrado il ritardo.}

{Controllò la disposizione delle stelle con un'occhiata. Ce n'erano
	otto, scaglionate non diversamente dal solito, abbastanza grosse perché
	valesse la pena di sfruttarle. -- Prendiamo le stelle più vicine --
	ordinò. -- C, scivolare lungo la parete. Se funziona, A ed E seguiranno.
	Se no, deciderò da lì. Io sarò col D. Muoversi!}

{Tutti i soldati avevano sentito quale fosse la strategia, ma le
	decisioni tattiche dipendevano adesso dai capibranco. Le istruzioni di
	Ender non fecero ritardare più di una decina di secondi la loro uscita,
	mentre l'orda delle Lepri si stava già muovendo in un'elaborata danza
	aerea all'altro lato della sala. In qualunque orda precedente Ender
	avesse militato, in quel momento avrebbe dovuto preoccuparsi che i suoi
	compagni potessero coprirsi a vicenda in una formazione adatta ad
	arginare la manovra nemica. Invece l'unica cosa che i Draghi stavano
	pensando era di penetrare al di là di essa, disperdersi fra le stelle e
	negli angoli della sala e quindi spezzettare la formazione avversaria in
	gruppetti molti dei quali sarebbero stati privi di un capobranco. E con
	quattro settimane appena di lavoro collettivo il caos scimmiesco in cui
	combattevano sembrava a Ender l'unico modo intelligente, l'unico modo
	\emph{possibile.} Fu quasi sorpreso nel constatare che le Lepri erano
	ancorate a schemi tattici per lui già sorpassati.}

{Il branco C scivolò via lungo la parete, con le ginocchia ripiegate per
	farsi scudo contro gli avversari. Tom il Matto, il capo del branco C,
	aveva evidentemente già ordinato ai suoi di spararsi alle gambe. In
	quella cupa penombra non la si poteva definire un'idea malvagia, dal
	momento che una volta colpite le tute da battaglia perdevano la loro
	debole fluorescenza. Il suo branco parve svanire nella foschia, e per
	quanto seccato dal non vederli più Ender dovette approvare la cosa.}

{L'orda delle Lepri riuscì a rintuzzare l'attacco laterale del branco C,
	ma non prima che Tom il Matto e i suoi ragazzi avessero attraversato
	come proiettili la formazione più avanzata: gli otto Draghi fecero fuori
	una buona dozzina di Lepri prima d'esser costretti a ripararsi dietro
	una stella. Ed era una stella all'interno del territorio nemico, cosa
	che bastò a fermare l'avanzata ordinata da Carn Carby.}

{Han Tzu, soprannominato Zuppa Cinese, era il capo del branco C. Scivolò
	come un gatto dietro il bordo della stella dove s'era attestato Ender.
	-- Che ne dici di girare sulla parete nord e arrivargli in faccia a
	ginocchia in avanti?}

{-- Provaci -- annuì Ender. -- Io porto il B a sud e cerco di aggirarli.
	-- Poi gridò: -- A ed E, allargare sulle pareti! -- Puntellò i piedi sul
	bordo della stella e si proiettò oltre la stella tenuta dal branco B,
	ordinando loro di seguirlo. Pochi istanti dopo li stava precedendo verso
	la parete sud. Vi arrivarono quasi all'unisono, e il rimbalzo li portò
	al di là delle due stelle da cui i soldati di Carn Carby stavano
	sparando contro Tom il Matto. Fu come tagliare il burro con una lama
	rovente. L'orda delle Lepri perse tanti uomini che il resto sarebbe
	stato un semplice rastrellamento. Ender suddivise i branchi in
	mezzi-branchi per attaccare ogni soldato nemico che fosse stato soltanto
	parzialmente inabilitato, e tre minuti più tardi i capibranco gli
	segnalarono che la sala era stata ripulita. Appena uno dei suoi ragazzi
	era stato completamente congelato -- uno del branco C, che aveva
	sopportato il grosso del contrattacco nemico -- e solo cinque erano
	inabilitati alle braccia. Metà avevano le gambe congelate, ma si
	trattava per lo più di colpi auto-inflitti. Tirate le somme, la
	strategia «caos contro ordine» aveva dato risultati fin troppo superiori
	a quelli che Ender s'era atteso.}

{Riunì i suoi capibranco e diede loro il privilegio di conquistare la
	porta delle Lepri: quattro caschi a contatto degli angoli luminosi, e
	Tom il Matto al centro per godersi quell'onore. Assai di rado accadeva
	che un comandante, purché non fosse congelato anch'egli, potesse portare
	tutti i suoi capibranco alla porta del nemico. Ender avrebbe potuto
	scegliere i cinque uomini in un'orda che non aveva praticamente subito
	perdite. Una buona battaglia.}

{Le luci si accesero al massimo, e il maggiore Anderson uscì dalla porta
	degli insegnanti al lato sud della sala. Con gesto solenne consegnò a
	Ender il radiogancio, il cui uso spettava per tradizione al vincitore.
	Lui azionò il piccolo apparecchio sulle tute dei suoi soldati,
	scongelandole, quindi li recuperò uno dopo l'altro e mise in fila i
	branchi prima di andare a scongelare gli avversari. Ranghi ordinati e
	militareschi, questo voleva esibire al momento in cui Carn Carby e le
	Lepri avessero riavuto il controllo dei loro corpi. \emph{Potranno
		imprecare e dire che li abbiamo attaccati come un branco di scimmie
		urlanti, ma ricorderanno d'esser stati distrutti, e ricorderanno di
		averci visti così: vittoriosi e perfettamente allineati, usciti quasi
		senza perdite dalla prima battaglia. L'orda dei Draghi non resterà molto
		a lungo nel suo tradizionale ruolo di mediocrità.}}

{Carn Carby venne a stringere la mano a Ender appena fu scongelato. Era
	un ragazzo di dodici anni, che malgrado le sue doti era stato promosso
	comandante solo nel suo ultimo anno di permanenza alla Scuola, e forse
	questo gli aveva impedito di metter su arie da galletto come altri più
	precoci di lui. \emph{Quel che mi sta insegnando lo terrò a mente},
	\emph{} si disse Ender, \emph{per quando sarò io a perdere. Dignità.
		Saper fare omaggio al valore dell'avversario. Le sconfitte non sono una
		tragedia\ldots{} anche se spero di conoscerne il meno possibile.}}

{Poco dopo, appena le Lepri furono uscite per la porta da cui erano
	entrati i ragazzi di Ender, Anderson mise in libertà l'orda dei Draghi.
	In fila indiana oltrepassarono la soglia, oltre la quale il pavimento
	piastrellato ricordava loro da che parte attirasse la forza di gravità.
	Ogni soldato oltrepassò la porta nemica, atterrò con un saltello e corse
	avanti fermandosi in fila con i compagni nel corridoio.}

{-- Sono le sette e un quarto, uomini -- li apostrofò Ender. -- Questo
	significa che avete quindici minuti per la colazione, prima che l'orda
	si presenti in sala di battaglia per l'addestramento mattutino. -- Gli
	parve quasi di sentirli sospirare in silenzio: avanti, comandante!
	Abbiamo vinto, lasciaci fare un po' di festa! Ma conservò un'espressione
	impassibile. -- Inoltre, poiché entrerete a mensa subito dopo le Lepri,
	siete avvisati che chi non sogghigna con tutti e trentadue i denti verrà
	severamente sculacciato.}

{I ragazzi risero, si scambiarono allegre gomitate nelle costole e
	battute scherzose, poi lui li mise al passo di corsa. Ma sulla soglia
	della mensa prese da parte i capibranco e disse loro che i ragazzi
	avrebbero avuto mezz'ora per la colazione, e che quel mattino
	l'addestramento sarebbe finito prima per dar loro il tempo di farsi una
	doccia e rilassarsi un po' in camerata. Si trattava di un premio
	piuttosto striminzito, ma nel severo orario della Scuola era meglio che
	niente. Inoltre dava modo a Ender di mettere in atto la sua politica.
	\emph{Lascia che i ragazzi abbiano dai loro capibranco le buone notizie,
		e dal comandante solo frasi alquanto burbere. Così diventeranno nodi
		stretti e solidi nel tessuto che si tesse in questa fabbrica.}}

{Lui non fece colazione. Non aveva fame. Andò invece alle docce e si
	lavò senza fretta, dopo aver ficcato la tuta da battaglia in un pulitore
	automatico che gliel'avrebbe restituita fresca e stirata in pochi
	minuti. In piedi sotto la doccia lasciò che l'acqua portasse via il
	sapone e continuasse a scorrergli addosso, ad occhi chiusi. \emph{Ogni
		goccia viene riciclata, qui dentro. Qualcuno berrà un po' del mio sudore
		di oggi.} Gli avevano affibbiato un'orda priva d'addestramento, e aveva
	vinto, e in modo per di più indiscutibile. Aveva vinto con sei soli
	soldati fra congelati e disabilitati. \emph{Adesso vediamo per quanto
		tempo gli altri comandanti continueranno a basarsi sulle loro formazioni
		rigide, dopo aver visto cosa può fare una strategia flessibile.}}

{Stava fluttuando nel mezzo della sala di battaglia loro assegnata,
	quando i suoi soldati cominciarono a entrare. Nessuno venne a dirgli
	niente, come si aspettava. Sapevano che a parlare sarebbe stato lui,
	appena pronto a farlo e non prima.}

{Allorché furono allineati, Ender usò il radiogancio per passarli in
	rassegna e li osservò uno per uno. -- La nostra prima battaglia non è
	finita disastrosamente -- disse. Questo diede la stura ad alcune risate
	e al tentativo di inneggiare «Dra-dra-draghi! Ahyy-\emph{draghi}!» che
	lui azzitti. -- L'orda dei Draghi si è comportata bene contro le Lepri.
	Ma non illudetevi di trovare altri avversari tanto facili. Se quella
	fosse stata una \emph{buona} orda, gruppo C, il vostro attacco è stato
	così lento che vi avrebbero schiacciati contro la parete prima di
	lasciarvi attestare al riparo. Branco A, e branco E, la vostra mira è
	difettosa. Ognuno di voi ha sul suo cartellino un solo centro ogni
	quattro o anche cinque colpi sparati. Erano colpi a lunga distanza,
	certo, ma se le Lepri non avessero concentrato il fuoco sul branco C il
	loro bersaglio sareste stati voi. E vi avrebbero fatto pagar cara questa
	manchevolezza. Voglio che ogni branco si eserciti al tiro, da lontano,
	con bersagli fermi e in movimento. Ogni mezzo branco fungerà a turno da
	bersaglio. Io scongelerò i colpiti ogni tre minuti. Adesso muoversi,
	coraggio!}

{-- Non potremmo avere una stella su cui piazzarci? -- chiese Zuppa
	Cinese. -- Per tener ferma la mira mentre spariamo, voglio dire.}

{-- No. Non dovete abituarvi ad avere un punto d'appoggio per la
	pistola. Se ti trema il braccio, congelati il gomito. Ai vostri posti!}

{I capibranco organizzarono subito il tiro al bersaglio, e Ender si
	mosse da un gruppo all'altro per dare suggerimenti e aiutare quei
	soldati che avevano qualche particolare difficoltà. I ragazzi s'erano
	già accorti che il loro comandante poteva essere brutalmente secco
	quando si rivolgeva ai branchi, ma che nel lavorare con il singolo
	individuo era assai paziente, dava suggerimenti più che ordini,
	ascoltava ogni domanda e ogni problema con sincera attenzione. Ma non
	rideva mai quando essi accennavano a coinvolgerlo in qualcosa di
	scherzoso, e presto avevano smesso di provarci. Lui era il comandante in
	ogni momento che trascorrevano insieme. Non ebbe bisogno di
	ricordarglielo: semplicemente \emph{lo era.}}

{Quel mattino i ragazzi avevano ancora in bocca il sapore della vittoria
	e lavorarono volentieri, chiacchierando e appassionandosi a quel che
	facevano, finché un'ora prima del pranzo uscirono per andare a far la
	doccia. Ender trattenne i capibranco e analizzò con loro la tattica che
	avevano usato ed il rendimento dei singoli soldati. Poi andò in camera
	sua e senza fretta si mise una tuta da riposo pulita, ascoltando la
	registrazione di una lezione tecnica del giorno prima. Aveva idea di
	entrare nella mensa dei comandanti con circa dieci minuti di ritardo.
	Una legge non scritta imponeva ai comandanti di non entrare nel locale
	da pranzo riservato ad essi fino alla loro prima vittoria, perciò lui
	non ne aveva mai visto l'interno né sapeva quale comportamento ci si
	attendeva da un comandante a mensa con i colleghi. Ma sapeva che
	intendeva entrare soltanto quando i punteggi delle squadre che avevano
	combattuto quel mattino sarebbero apparsi sul tabellone.}

{Il suo ingresso non destò la minima sensazione. Ma quando alcuni dei
	presenti notarono quanto fosse giovane, e videro l'emblema del Drago
	sulle maniche della sua uniforme, lo seguirono con lo sguardo. Lui andò
	a riempirsi un vassoio al distributore automatico, e nel sedersi a un
	tavolo libero s'accorse che le conversazioni in sala erano cessate.
	Cominciò a mangiare, lentamente e con cura, fingendo di non rendersi
	conto degli occhi puntati su di lui. Pian piano le chiacchiere e i
	rumori normali ripresero, e soltanto allora poté rilassarsi e girare
	attorno qualche rapido sguardo.}

{Una delle pareti era occupata per intero da un grande schermo, su cui
	il computer proiettava classifiche e dati di vario genere. I soldati
	erano tenuti al corrente delle prestazioni delle orde, mentre qui
	c'erano graduatorie relative a quelle dei singoli comandanti.}

{E di nuovo, grazie agli scherzi delle percentuali, Ender risultava in
	testa con un 100\% di vittorie, mentre il suo distacco era ancor
	maggiore in altre categorie: media dei superstiti sani/disabilitati,
	media degli avversari congelati, tempo medio per ottenere una vittoria,
	e media delle perdite per battaglia.}

{Aveva quasi finito di mangiare quando una mano gli si poggiò su una
	spalla. -- Posso sedermi?}

{Ender non ebbe bisogno di voltarsi per riconoscere Dink Meeker. -- Ehi,
	Dink! -- si compiacque. -- Certo, accomodati.}

{-- Allora, bastardo placcato d'oro -- disse allegramente l'altro. --
	Qui stiamo tutti cercando di decidere se i tuoi punteggi sono un
	miracolo o un maledetto imbroglio.}

{-- Un'abitudine -- disse Ender.}

{-- Una vittoria non è un'abitudine -- lo rimbrottò Dink. -- Non
	montarti la testa. Quando sei nuovo ti mettono contro squadre
	materasso.}

{-- Carn Carby non è precisamente un materasso. -- Era vero. Carby si
	trovava presso il centro delle graduatorie.}

{-- Lui è in gamba -- disse Dink, -- considerando la sua scarsa
	esperienza di comando, è una promessa. Tu non sei una promessa. Sei una
	minaccia.}

{-- Minaccia di cosa? Ti danno da mangiare di meno, se io vinco di più?
	E pensare che sei stato proprio tu a dirmi che questo è uno stupido
	gioco, senza nessuna importanza.}

{A Dink non fece piacere vedersi ritorcere contro le sue stesse parole,
	non in quelle circostanze. -- Però tu sei quello che mi ha convinto a
	seguire fino in fondo il loro piano. Bene\ldots{} ma non fare il gioco
	dei tre bussolotti con me, Ender. Non vinceresti.}

{-- Probabilmente no -- disse lui.}

{-- Io ti ho insegnato molte cose.}

{-- Le più importanti -- ammise Ender. -- Adesso le sto solo risuonando
	a orecchio.}

{-- Congratulazioni -- disse Dink.}

{-- È bello avere un amico, qui. -- Ma Ender non era troppo sicuro che
	Dink fosse sempre veramente suo amico, e quell'impressione era
	reciproca. Dopo una ventina di secondi di silenzio Dink tornò al suo
	tavolo.}

{Ender poggiò le posate sul piatto e si guardò attorno. Qua e là in sala
	si cominciava a far conversazione. Gettò un'occhiata a Bonzo, che adesso
	era uno dei comandanti più anziani. Rose de Nose aveva terminato il
	corso. Petra era con un gruppetto di colleghi, in un angolo, e non aveva
	ancora guardato una volta verso di lui. Poiché molti degli altri di
	tanto in tanto si voltavano a osservarlo, inclusi quelli con cui Petra
	stava parlando, Ender fu abbastanza certo che la ragazza evitava
	deliberatamente il suo sguardo. \emph{Ecco il guaio di chi vince fin
		dall'inizio}, \emph{} sospirò fra sé. \emph{Perdi gli amici.}}

\emph{{Diamogli qualche settimana perché si abituino. Per il giorno in
		cui avrò la mia prossima battaglia, le acque si saranno un po'
		calmate.}}

{Carn Carby si fece un punto d'onore di fermarsi a salutare Ender, poco
	prima che scadesse l'ora del pranzo. Fu di nuovo un placido gesto
	formale, e tuttavia, a differenza di Dink, Carby esibì modi molto
	spontanei. -- Pare che io sia in disgrazia -- disse con franchezza. --
	Nessuno ha voluto credermi, quando ho detto che hai messo in atto una
	strategia mai vista prima. Così spero che tu batta il prossimo di questi
	altezzosi signorini che ti capiterà davanti. Te lo chiedo come un
	favore.}

{-- Farò del mio meglio, -- annuì Ender, serio. -- E grazie per esserti
	fermato a parlare con me.}

{-- Oggi ti hanno trattato in modo indegno. Per tradizione i nuovi
	comandanti vengono complimentati da tutti, la prima volta che vengono
	alla mensa. Ma di solito nessuno arriva qui senza aver già ingoiato
	diverse sconfitte. Io ci sono entrato per la prima volta appena un mese
	fa. E se qualcuno merita complimenti, sei tu. Ma così è la vita. Fagli
	mangiare la polvere.}

{-- Sei davvero\ldots{} sì, ci proverò. -- Carby uscì, e Ender
	mentalmente lo aggiunse alla sua lista privata di membri a pieno merito
	della razza umana.}

{Quella notte Ender dormì meglio di quanto gli accadeva ormai da mesi. E
	il suo sonno fu così profondo che a destarlo fu soltanto il cicalino,
	quando le luci si riaccesero. Scendendo dal letto notò che si sentiva a
	meraviglia; andò di corsa a fare una doccia rapida, tornò in camera e
	tolse dall'armadio una tuta da fatica. Soltanto allora, quando l'aria
	spostata dall'indumento lo fece svolazzare, si accorse che sul pavimento
	c'era un foglio di carta. Lo raccolse e lo lesse.}

{~}

\begin{center}
	{ORDA DEI DRAGHI - comandante Ender Wiggin}

{sala di battaglia, ore 0700}

{~}

{ORDA DELLE FENICI - comandante Petra Arkanian}
\end{center}

{~}

{Era la sua vecchia orda, quella che aveva lasciato soltanto quattro
	settimane prima, e ne conosceva le tattiche di battaglia come il palmo
	della sua mano. Grazie anche ai suoi apporti tecnici era la più
	flessibile fra le orde, capace di rispondere bene e in fretta alle
	situazioni impreviste. Le Fenici sarebbero stati i più abili
	nell'adattarsi all'attacco fluido e non schematico dei Draghi. Gli
	insegnanti erano determinati a rendergli la vita piuttosto
	interessante.}

{0700 diceva il foglio, ed erano le 0630. Alcuni dei suoi ragazzi
	dovevano essersi già avviati sbadigliando a far colazione. Ender rimise
	l'uniforme nell'armadio, afferrò la tuta da battaglia e pochi secondi
	dopo entrava a lunghi passi nella camerata della sua orda.}

{-- Signori, spero che ieri abbiate imparato cos'è una battaglia, perché
	oggi siamo gentilmente attesi nella stessa sala.}

{Occorse loro qualche momento per capire che stava parlando di una
	battaglia, e non dell'addestramento. Doveva esserci un errore, dissero
	alcuni. Nessuno aveva mai affrontato due avversari in due giorni
	consecutivi.}

{Lui porse il foglio a «Mosca» Molo, il capo del branco A, che
	all'istante sbraitò: -- Tute da battaglia! Controllare la batteria delle
	pistole! Oggi i Draghi bruceranno la coda a certi tipi che si fanno
	chiamare Fenici, uomini! Scattare!}

{-- Perché non ce lo hai detto prima? -- chiese Zuppa Cinese. Zuppa
	aveva un modo di fare domande a Ender che nessun altro osava imitare.}

{-- Volevo lasciarvi il tempo di fare una bella doccia -- disse lui. --
	Ieri le Lepri dicevano che abbiamo vinto perché la nostra puzza li ha
	messi fuori combattimento.}

{I soldati che erano nelle vicinanze risero.}

{-- Non hai trovato quel foglio finché non sei tornato dalla doccia,
	comandante. Vero? -- disse una voce.}

{Ender si volse in cerca di chi aveva parlato. Era Bean, già in tuta da
	battaglia e con uno sguardo insolente negli occhi. \emph{Cerchi
		l'occasione di restituire pan per focaccia, eh, Bean?}}

{-- Proprio così -- disse, esibendo un'annoiata pazienza. -- Io non sono
	vicino al pavimento quanto te, caro bambino.}

{Altre risate. Bean arrossì di rabbia.}

{-- È chiaro che ci troviamo di fronte a schemi tattici troppo rigidi --
	disse Ender. -- Ma di volta in volta diversi, e io voglio che le
	iniziative dei capibranco siano lucide. Non posso far finta che mi
	piaccia il modo in cui ci mettono sotto il torchio, però so di poter
	dire una cosa: che ho un'orda capace di farcela.}

{Dopo quella frase, se avesse chiesto loro di seguirlo sulla luna senza
	tuta spaziale i Draghi ci avrebbero provato.}

{Petra non era Carn Carby. Fin dall'inizio le Fenici si dimostrarono
	capaci di una manovra corale molto flessibile, e arginarono bene i
	velocissimi e imprevedibili attacchi dei Draghi. Tuttavia Petra faceva
	agire uniti i suoi quattro branchi, e quando s'accorse che di fronte a
	lei c'erano dieci mezzi branchi ognuno dei quali scatenato in
	un'iniziativa diversa ne fu confusa. Ordinò allora di attestarsi a
	difesa di posizione fisse, e come risultato venne presa alle spalle.
	Mentre sparava come un'indemoniata, Bean la congelò con un preciso colpo
	al corpo. La battaglia terminò con tre soli Draghi congelati e nove
	parzialmente inabilitati. Le Fenici uscirono dalla sala di pessimo
	umore, e Petra passò accanto a Ender evitando ostentatamente di
	stringergli la mano. L'ira nei suoi occhi sembrava dire: io ero tua
	amica, e invece di limitarti a sconfiggermi mi hai addirittura
	umiliata.}

{Ender finse di non notare la sua ostilità e le fece ugualmente un cenno
	di saluto, sapendo che la ragazza aveva un carattere acceso. Si disse
	che dopo qualche altra battaglia Petra avrebbe constatato che le Fenici
	avevano inferto ai Draghi più perdite di ogni altro, e questo la avrebbe
	placata. Inoltre da lei stava ancora imparando qualche cosetta, e
	nell'addestramento di quel giorno avrebbe insegnato ai capibranco come
	fronteggiare un paio di manovre puramente difensive che le aveva visto
	mettere in pratica. Presto sarebbero stati di nuovo amici.}

{O così sperava.}

{~}

\begin{center}
	{* * *}
\end{center}

{~}

{Alla fine della settimana l'orda dei Draghi aveva combattuto sette
	battaglie, una al giorno. Sul tabellone risultavano 7 vittorie e zero
	sconfitte. Ender ebbe le perdite maggiori nello scontro con le Fenici, e
	in due occasioni i Draghi terminarono senza neppure un soldato congelato
	o disabilitato. Nessuno poteva più dire che a metterlo in cima alle
	classifiche erano gli scherzi matematici delle percentuali: troppe erano
	le orde che aveva battuto con margini di punteggio mai visti prima.}

{Per gli altri comandanti non era più possibile ignorarlo. Alcuni di
	loro presero a sedersi al suo tavolo, cercando cautamente di sondarlo
	per capire come aveva sconfitto i suoi ultimi avversari. Lui lo rivelò
	liberamente, dicendo a se stesso che pochi avrebbero saputo come
	allenare i loro soldati e i loro capibranco per ottenere un'orda di quel
	genere. E mentre i più cordiali facevano conversazione con lui, un
	gruppo molto più numeroso si radunava intorno ai comandanti da lui
	sconfitti, nel tentativo di scoprire un sistema capace di batterlo.}

{Ciò che scoprì lui fu il numero, imprevisto, dei ragazzi che lo
	odiavano. Ognuno aveva un suo motivo particolare: chi perché era troppo
	giovane per avere un'orda, chi perché era troppo bravo, chi perché aveva
	fatto impallidire i successi altrui, e chi perché scambiava la sua
	riservatezza per superbia. Ender cominciò a leggere quei sentimenti in
	faccia anche a ragazzi che non conosceva, incrociandoli nei corridoi.
	Poi notò che a mensa alcuni si alzavano ostentatamente e cambiavano
	posto, quando lui veniva a sedersi troppo vicino. Nello stesso periodo
	ci fu un'escalation dei piccoli incidenti e degli atti ostili: gomiti
	che gli si affondavano casualmente nelle costole in sala giochi, piedi
	che lo facevano inciampare mentre passava in mezzo a un gruppetto fermo
	su una porta, sputi o palle di carta bagnata che gli arrivavano alle
	spalle quando faceva un po' di corsa lungo i corridoi. Qualcuno agiva
	anche apertamente, quasi a dirgli che se era invincibile in sala di
	battaglia nel resto della Scuola restava un ragazzino qualunque. Ender
	li disprezzava. Ma segretamente, così segretamente che forse neppure lui
	lo sapeva, aveva paura di loro. Era proprio con quelle vessazioni che
	Peter lo aveva sempre tenuto sulle spine, e in lui stavano rinascendo le
	angosce fatte di rabbia e d'impotenza che avevano accompagnato la sua
	infanzia.}

{Tuttavia erano soperchierie di scarso peso, e si disse che poteva
	accettarle come riconoscimenti d'altro genere. Alcune orde cominciavano
	già ad imitarlo. Molti soldati attaccavano con le gambe ripiegate in
	avanti, le manovre corali lasciavano il posto a quelle frammentate, e
	molti comandanti mandavano i branchi a scivolare via lungo le pareti.
	Nessuno sembrava incline a suddividere l'orda in cinque parti;
	affermavano che quattro branchi forti potevano prevalere contro cinque
	deboli\ldots{} e questo continuava a dargli un vantaggio.}

{Stava insegnando a tutti quanti, ex novo, le tattiche di combattimento
	a gravità zero. Ma dove poteva rivolgersi per tirar fuori idee sempre
	originali ed efficaci?}

{Cominciò a frequentare la videoteca, piena di filmati propagandistici
	su Mazer Rackham e altri famosi comandanti che avevano combattuto
	durante la Prima e la Seconda Invasione. Interruppe le lezioni di
	addestramento pratico un'ora prima e incaricò i capibranco di fare
	istruzione a loro piacimento in sua assenza. Di solito mettevano su
	delle scaramucce, branco contro branco; Ender restava qualche minuto a
	vedere cosa inventavano e poi andava a studiarsi le antiche battaglie.}

{Molti di quei video erano pura perdita di tempo. Musica eroica, sfilate
	di ufficiali, consegna di medaglie al valore, scene confuse di marines
	che prendevano possesso di istallazioni costruite dagli Scorpioni. Ma
	qua e là trovò dei filmati utili: astronavi simili a punti di luce che
	manovravano nel buio dello spazio, o meglio ancora gli schermi di bordo
	sui quali si leggeva lo svolgimento di una battaglia schematizzata dai
	computer. Era difficile interpretare a tre dimensioni quei video
	bidimensionali, e spesso le scene erano troppo corte o senza un commento
	esplicativo. Ma lo sorprese notare l'abilità con cui gli Scorpioni
	attaccavano su traiettorie apparentemente casuali per creare confusione,
	o usavano esche e false manovre di ritirata per attirare le navi della
	F.I. in trappola. Alcune battaglie erano spezzettate in scene diverse e
	sparse fra bobine di documentari, telegiornali e registrazioni
	archiviate dalla F.I. ma rimontandole e proiettandole da solo Ender
	riuscì a ricostruirle per intero. Cominciò a vedere particolari che i
	commentatori ufficiali avevano ignorato. Costoro non facevano che
	sviolinate e tiritere sul coraggio dei comandanti umani e sulla nera
	perversità degli Scorpioni in questa o quella circostanza, ma a Ender
	interessava il lato tecnico, e si chiedeva come fosse mai stato
	possibile ottenere certe vittorie inesplicabili. Le astronavi terrestri
	erano rozzi prodotti di catene di montaggio costrette a lavorare in
	fretta; le flotte rispondevano al verificarsi di circostanze nuove con
	incredibile lentezza e inefficienza, mentre le navi degli Scorpioni
	agivano con un sincronismo impensabile e sembravano capaci di mutare
	obiettivi e manovre con istantanea precisione. Ovviamente, durante la
	Prima Invasione le astronavi terrestri erano state delle bagnarole lente
	e inadatte a vere e proprie campagne belliche nello spazio profondo, ma
	gli Scorpioni avevano portato nel sistema solare vascelli non troppo
	superiori ad esse. Era stato soltanto nella Seconda Invasione che i
	mezzi spaziali s'eran fatti potenti, veloci, e le armi veramente
	mortali.}

{Così fu dagli Scorpioni, e non dai terrestri, che Ender si accorse
	d'imparare la strategia. E questo gli fece provare un senso di vergogna
	e di colpa, perché quegli esseri chitinosi erano il più terribile
	pericolo mai piombato addosso all'umanità, un nemico odioso e mortale.
	Ma erano anche maledettamente abili in ciò che facevano. Almeno fino a
	un certo punto. Ogni volta sembravano seguire una sola strategia
	basilare: riunire il maggior numero possibile di navi nella zona chiave
	della battaglia. Non facevano nulla di sorprendente, niente che
	rivelasse fra loro la presenza di comandanti geniali o con tendenze
	individualistiche. La disciplina era ferrea, pari alla loro efficienza.}

{E c'era una cosa strana. Nella marea di discorsi su Mazer Rackham, i
	filmati relativi alla sua flotta brillavano per la loro assenza. Al più
	esistevano registrazioni sulle premesse di una battaglia, con le piccole
	e scarse astronavi di Rackham quasi patetiche al confronto della
	strapotenza degli Scorpioni. Questi avevano già fatto a pezzi la
	principale flotta terrestre oltre l'orbita di Plutone, spazzato via gli
	avamposti esterni del sistema solare, e s'erano fatti gioco dell'intera
	strategia messa in atto per impedir loro di avvicinarsi alla Terra. Di
	questa fase abbondavano le registrazioni, recuperate dalle rovine e
	montate in film per alimentare l'orrore e l'odio della gente. Poi gli
	Scorpioni erano giunti a contatto della piccola flotta riunita da Mazer
	Rackham presso Saturno. L'avvicinamento, la disparità di forze, e
	quindi\ldots{}}

{L'unica ripresa, fatta dall'interno del piccolo incrociatore di
	Rackham, mostrava una nave nemica che esplodeva. Nelle bobine non c'era
	altro. Dozzine di filmati in cui i marines si aprivano la strada entro
	le buie astronavi nemiche, dozzine di riprese mostranti i corpi degli
	Scorpioni sparsi ovunque. Ma nulla in cui si vedesse uno Scorpione
	ucciso in combattimento, o comunque in atto di combattere, salvo in
	brani filmati chiaramente ripresi durante la Prima Invasione. Era
	frustrante scoprire che proprio la vittoria di Rackham fosse stata così
	censurata dalla F.I. Alla Scuola di Guerra gli studenti avrebbero avuto
	molto da imparare da Mazer Rackham, eppure le registrazioni di quella
	battaglia non c'erano. I servizi segreti non avevano certo fatto un
	favore a quei ragazzi che si accingevano a emulare le sue imprese
	belliche.}

{Come c'era da aspettarsi, appena si sparse la voce che Ender Wiggin
	studiava i filmati delle vecchie battaglie molti cominciarono a
	frequentare la videoteca. Per lo più erano i comandanti, e questi
	esaminavano le stesse registrazioni consultate da lui assumendo l'aria
	di chi ha capito cosa c'è di interessante e di cui prendere
	doverosamente nota. Ender li osservò darsi da fare senza dir parola.
	Anche quando un ragazzo, dopo aver proiettato alcuni video di diversa
	fattura, si volse a chiedergli: -- Secondo te, questi riguardano tutti
	la stessa battaglia? -- Lui si limitò a scrollare le spalle come se la
	cosa fosse irrilevante.}

{Fu durante l'ultima ora d'addestramento del settimo giorno, poche ore
	dopo che l'orda di Ender aveva vinto la sua settima battaglia, che il
	maggiore Anderson in persona entrò in videoteca. Consegnò un documento a
	uno dei comandanti seduti davanti agli schermi e poi si volse a Ender:
	-- Il colonnello Graff vuole vederti subito nel suo ufficio.}

{Ender si alzò e tenne dietro ad Anderson lungo i corridoi. Il maggiore
	poggiò una mano sullo scanner della porta che separava i quartieri degli
	studenti da quelli degli ufficiali, e poco dopo furono davanti a Graff,
	che li attendeva seduto su una sedia girevole imbullonata al pavimento.
	Lo stomaco rigonfio metteva a dura prova le cuciture della sua uniforme,
	e Ender sbatté le palpebre nell'osservarlo. Graff non gli era parso
	particolarmente grasso la prima volta che l'aveva visto, appena quattro
	anni addietro. L'età e la tensione non erano state molto gentili con il
	direttore della Scuola di Guerra.}

{-- Sono trascorsi sette giorni dalla tua prima battaglia, Ender --
	disse Graff.}

{Il ragazzo non fece commenti.}

{-- E tu hai vinto sette battaglie, una al giorno.}

{Ender annuì.}

{-- Inoltre, i tuoi punteggi sono insolitamente alti.}

{Lei sbatté appena le palpebre.}

{-- Comandante, a cosa attribuisci i tuoi notevoli successi?}

{-- Mi avete dato un'orda che riesce a fare qualunque cosa io pensi di
	farle fare.}

{-- E cos'hai pensato di farle fare?}

{-- Ci orientiamo come se la porta del nemico fosse in basso e usiamo le
	gambe come uno scudo. Evitiamo di manovrare in formazione e ci basiamo
	sulla mobilità. È stata d'aiuto anche la suddivisione in cinque branchi
	di otto elementi, invece che in quattro di dieci. Inoltre, i nostri
	avversari non hanno ancora avuto il tempo di adattarsi validamente alle
	nuove tecniche, e i primi li abbiamo sconfitti usando sempre gli stessi
	stratagemmi. Questa situazione perciò non ci aiuterà a lungo.}

{-- Dunque non ti aspetti di continuare a vincere.}

{-- Non con gli stessi metodi.}

{Graff annuì. -- Siedi, Ender.}

{Lui e Anderson presero due poltroncine. Graff guardò il collega, e fu
	questi a fare la domanda successiva: -- In che condizioni è la tua orda,
	dopo tutte queste battaglie consecutive?}

{-- Oggi si possono considerare tutti veterani.}

{-- Ma come reagiscono? Sono stanchi?}

{-- Se lo sono, rifiutano di ammetterlo.}

{-- Le loro capacità e i loro riflessi sono ancora al meglio?}

{-- Siete voi a controllare i giochi che il computer gioca con la loro
	mente. Dovreste dirlo voi a me.}

{-- \emph{Noi} sappiamo già quello che sappiamo. Ciò che vogliamo sapere
	è quello che sai \emph{tu.}}

{-- Questi sono bravi soldati, maggiore Anderson. Sono certo che hanno
	dei limiti, ma ancora non li hanno raggiunti. Alcuni dei più giovani
	hanno ancora difficoltà a padroneggiare certe tecniche di base, ma
	lavorano sodo e migliorano. Cosa vuole che le dica, che hanno bisogno di
	riposo? È ovvio che un paio di settimane senza battaglie non gli
	farebbero male. I loro studi sono andati alla malora; nessuno di noi
	combina molto quando si va in aula. Ma questo voi lo sapete, e sembra
	chiaro che non v'importa, così perché dovrei preoccuparmene io?}

{Graff e Anderson si scambiarono un'occhiata. -- Ender, perché ti sei
	messo a studiare i video delle guerre contro gli Scorpioni?}

{-- Per aggiornarmi in strategia, naturalmente.}

{-- Quei filmati sono stati fatti a scopi propagandistici. Tutta la
	nostra strategia ne è stata tagliata via.}

{-- Lo so.}

{Graff e Anderson tornarono a guardarsi. Graff tambureggiò con le dita
	sulla scrivania. -- Non giochi più la partita di fantasia -- disse.}

{Ender non rispose.}

{-- Dimmi perché hai smesso di giocarla.}

{-- Perché ho vinto.}

{-- Tu non hai vinto tutto in quella partita. C'è sempre dell'altro.}

{-- Ho vinto tutto.}

{-- Ender, noi vorremmo aiutarti a sentirti realizzato il più possibile,
	ma se tu\ldots{}}

{-- Voi volete fare di me il miglior soldato possibile. Andate giù a
	dare un'occhiata alle classifiche. Confrontatele con quelle di altri
	dalla fondazione della Scuola in poi. Non c'è dubbio che con me avete
	fatto un lavoro eccellente. Congratulazioni. Ora, quando intendete farmi
	combattere contro una buona orda?}

{Le labbra rigide di Graff si piegarono in un sorrisetto, e il suo
	stomaco sussultò un attimo a una risata silenziosa.}

{Anderson consegnò a Ender un foglio. -- Adesso -- lo informò.}

{~}

\begin{center}
	{ORDA DEI DRAGHI - Comandante Ender Wiggin}

{Sala di battaglia, ore 12}

{~}

{ORDA DELLE SALAMANDRE - Comandante Bonzo Madrid}
\end{center}

{~}

{-- L'inizio è fra dieci minuti -- disse Ender. -- I miei soldati hanno
	appena finito l'addestramento; saranno tutti nelle docce.}

{Graff sorrise. -- Allora meglio che ti sbrighi, ragazzo.}

{~}

\begin{center}
	{* * *}
\end{center}

{~}

{Cinque minuti più tardi piombò nella camerata dei Draghi. Quasi tutti
	si stavano vestendo dopo aver fatto la doccia, alcuni erano già andati
	in sala giochi o in videoteca ad aspettare l'ora del pranzo. Lui mandò
	tre dei più giovani a richiamarli, e fece indossare agli altri la tuta
	da battaglia il più in fretta possibile.}

{-- Questo è uno scontro duro, e siamo a corto di tempo -- disse. --
	Hanno mandato l'avviso a Bonzo almeno venti minuti fa, il che significa
	che quando arriveremo in sala di battaglia loro saranno dentro già da
	cinque minuti.}

{I ragazzi erano offesi, e se ne lamentarono ad alta voce nel linguaggio
	che solitamente in presenza del comandante evitavano. -- Cosa Cristo li
	morde, quei figli di puttana? Vogliono vederci con culo in terra? O si
	sono fottuti il cervello tutti quanti?}

{-- Lasciate perdere. Avremo tempo stasera per imprecare. Siete
	stanchi?}

{«Mosca» Molo fece una smorfia. -- Abbiamo lavorato duro fino adesso.
	Per non parlare della batosta che abbiamo dato ai Furetti stamattina.}

{-- Nessuno ha mai fatto due battaglie nello stesso giorno -- disse Tom
	il Matto.}

{Ender replicò nello stesso tono: -- Nessuno ha mai sconfitto i Draghi,
	però. Questa è la vostra grossa occasione. Volete gettarla via? -- La
	sua dura sfida era la risposta alle loro lamentele: prima vincere, e le
	recriminazioni farle in seguito.}

{Adesso in camerata c'erano tutti, e stavano finendo di vestirsi. --
	Muoversi, uomini! -- gridò Ender, e i ragazzi lo seguirono di corsa nei
	corridoi che portavano alla sala di battaglia, chi allacciandosi la tuta
	e chi controllando la pistola. Molti di loro avevano il fiato grosso;
	brutto segno, l'orda era troppo stanca per quella battaglia. Trovarono
	la porta già aperta, e nell'interno non era visibile nessuna stella: uno
	spazio del tutto vuoto, e l'illuminazione della sala era abbagliante.
	Niente ripari e niente penombra per nascondersi.}

{-- Madre mia! -- esclamò Tom il Matto. -- Neppure le Salamandre sono
	uscite dalla loro porta.}

{Ender si portò un dito alle labbra, ordinando il silenzio. Con la porta
	aperta gli avversari avrebbero potuto udire ogni loro parola. Con un
	dito indicò attorno alla porta, dove senza alcun dubbio l'orda nemica
	era andata ad appostarsi, a ridosso della parete e pronta a far fuori
	all'istante chiunque fosse emerso in sala.}

{Ender li spinse tutti indietro di una dozzina di passi. Poi fece uscire
	dai ranghi alcuni dei ragazzi più alti, incluso Tom il Matto, e sussurrò
	loro di accovacciarsi, non in ginocchio ma con le gambe tese in avanti,
	in modo che formassero una L con il corpo. Li congelò con un colpo
	ciascuno. L'orda lo fissava in silenzio. Scelse il ragazzo più piccolo,
	gli consegnò anche la pistola di Tom il Matto e lo fece inginocchiare
	sulle gambe congelate di Tom. Poi mise le mani di Bean, ognuna armata di
	pistola, sotto le ascelle dell'altro.}

{Adesso i ragazzi cominciavano a capire: Tom era uno scudo,
	un'astronave, e Bean l'addetto alle batterie di bordo. Il piccoletto non
	era certo invulnerabile, ma avrebbe avuto il tempo di sparare.}

{Ender assegnò due ragazzi al compito di scaraventare Tom e Bean fuori
	dalla porta, ma segnalò loro di aspettare. Passò attraverso i compagni e
	in fretta li suddivise in gruppi di quattro: uno scudo, un tiratore, e
	due addetti al lancio. Poi, quando tutti furono chi congelato, chi
	armato e chi pronto a dare la spinta, segnalò a questi ultimi di
	sollevare il loro carico, scagliarlo oltre la porta e quindi di balzare
	in sala, anch'essi abbracciati in modo che almeno uno avesse riparo.}

{-- Pronti\ldots{} \emph{scattare!} -- gridò Ender.}

{L'orda dei Draghi scattò. Due alla volta le coppie scudo-tiratore
	volarono fuori dalla porta, posizionate in modo che lo «scudo» volgesse
	la schiena al nemico. Le Salamandre aprirono il fuoco all'istante, ma la
	maggior parte dei colpi intercettava soltanto il ragazzo congelato. E
	nel frattempo, con due pistole al lavoro e i loro bersagli pulitamente
	allineati lungo la parete spoglia, i Draghi riuscirono a fare il tiro a
	segno su degli avversari immobili e del tutto scoperti. Sbagliare era
	quasi impossibile.}

{Ma furono i lanciatori a giocare lo scherzo più sporco alle Salamandre,
	perché «Mosca» Molo e il suo compagno ebbero l'idea di uscire con un
	lieve saltello appena, e poi si respinsero l'un l'altro, volando di lato
	rasente alla parete della porta. I successivi li imitarono, mirando alle
	Salamandre da un'angolazione diversa, cosicché gli uomini di Bonzo non
	seppero se sparare alle coppie scudo-tiratore in allontanamento o a
	quelli che li assalivano lungo le loro stesse file.}

{Per il momento in cui anche Ender balzò in sala, la battaglia era già
	finita. Era trascorso sì e no un minuto dal momento in cui era uscito il
	primo Drago a quello in cui le pistole avevano taciuto. L'orda dei
	Draghi contava venti fra congelati e del tutto disabilitati, e solo
	dodici ragazzi ancora intatti. Quella era la loro peggiore percentuale
	di danni subiti, ma avevano vinto.}

{Quando il maggiore Anderson uscì a consegnargli il radiogancio, Ender
	non poté nascondere la sua rabbia. -- Credevo che ci avreste dato la
	possibilità di affrontare un'orda in uno scontro leale!}

{-- Congratulazioni per la vittoria, comandante.}

{-- Bean! -- sbottò Ender. -- Se il comandante dell'orda delle
	Salamandre fossi stato tu, cos'avresti fatto?}

{Nei pressi della porta nemica dov'era finito, colpito alle braccia ma
	non del tutto congelato, Bean gridò: -- Avrei messo delle vedette per
	guardare dentro la porta dei Draghi. E non sarei stato così idiota da
	tener fermi i miei uomini, visto che il nemico sapeva dove trovarli.}

{-- Visto che vi date all'imbroglio -- disse Ender a Anderson, -- perché
	non insegnate alle altre orde a imbrogliare con intelligenza?}

{-- Ti suggerisco di scongelare i tuoi soldati -- disse Anderson.}

{Ender fece uso dell'apparecchietto per rimettere in attività entrambe
	le orde contemporaneamente. -- Draghi in libertà! -- ordinò subito dopo.
	Non ci sarebbe stata nessuna formazione schierata per salutare l'uscita
	degli avversari sconfitti. Quella non era stata una battaglia leale,
	perché se pure avevano vinto era chiaro che gli insegnanti avevano
	mirato a farli soccombere, ed era stata soltanto l'inettitudine di Bonzo
	a salvarli. Non c'era né onore né gloria in cosa simile.}

{Fu solo mentre usciva dalla sala di battaglia che Ender, ripensando
	all'espressione di Bonzo, capì che l'altro non gli avrebbe neppure
	riconosciuto il diritto d'essere adirato con gli insegnanti. L'onore
	spagnolo. Bonzo avrebbe masticato veleno inchiodato a ben altri
	pensieri: era stato sconfitto quando le probabilità erano tutte a suo
	favore, era stato sconfitto da dei novellini perdendo la faccia di
	fronte agli altri comandanti, era stato sconfitto da un avversario che
	prima di uscire non gli aveva neppure teso la mano per salvare almeno le
	apparenze. Se Bonzo non lo avesse già odiato per altri motivi, questo
	sarebbe bastato; ma poiché lo odiava ora quel sentimento si sarebbe
	mutato in una rabbia omicida. \emph{Ha cominciato a detestarmi quando le
		sue soperchierie non mi umiliavano abbastanza}, \emph{} pensò Ender.
	\emph{Queste sono cose che uno come Bonzo non dimentica.}}

{E certo non aveva dimenticato il giorno in cui s'era unito ad altri
	ragazzi anziani per aggredire i novellini che si allenavano con lui.
	Quei veterani se l'erano legata al dito. \emph{Se loro bramano la
		vendetta, Bonzo avrà addirittura sete di sangue.} Per un po' Ender si
	trastullò con l'idea di tornare da lui per scusarsi di non avergli
	stretto la mano, ma con due battaglie alle spalle nello stesso giorno si
	sentiva seccato e stanco, assillato dalla mancanza di tempo, e scrollò
	le spalle. \emph{Gli insegnanti mi hanno messo in questa situazione},
	\emph{} si disse, \emph{penseranno loro a controllarne le conseguenze.}}

{~}

\begin{center}
	{* * *}
\end{center}

{~}

{Bean si lasciò cadere sulla cuccetta con un sospiro esausto. Metà dei
	suoi compagni erano già addormentati, e c'erano ancora quindici minuti
	prima che le luci si spegnessero. Stancamente tirò fuori il suo banco
	dall'armadietto e lo accese. L'indomani c'era un esame di geometria che
	l'avrebbe trovato miseramente impreparato. Se avesse avuto qualche ora
	in più sarebbe riuscito a sfangarsela in qualche modo, e aveva letto
	Euclide prima ancora di compiere i cinque anni, ma l'esame aveva un
	limite di tempo e la necessità di pensare in fretta lo avrebbe fatto
	affogare. Era lì per studiare e stava affogando nell'ignoranza, nella
	fretta, nella stanchezza. E l'esame sarebbe stato un disastro. Ma quel
	giorno avevano vinto due volte, e questo lo faceva sentire a posto.}

{Appena lo schermo si accese, tuttavia, ogni pensiero sull'esame svanì.
	Al centro di esso era comparso un messaggio:}

{~}

\begin{center}
	{VOGLIO VEDERTI SUBITO - ENDER}
\end{center}

{~}

{L'orologio segnava le 2150, solo dieci minuti all'ora in cui spegnevano
	le luci. Da quanto tempo era arrivato il messaggio di Ender? Comunque
	fosse, non poteva ignorarlo. Poteva esserci un'altra battaglia il
	mattino dopo (il pensiero lo fece gemere) e di qualunque argomento Ender
	volesse parlargli la cosa andava fatta subito. Così Bean si trascinò giù
	dalla cuccetta e continuando a sospirare percorse i corridoi deserti
	fino alla camera di Ender. Bussò alla porta.}

{-- Vieni dentro -- fu invitato.}

{-- Ho visto adesso il tuo messaggio.}

{-- Bene -- disse Ender.}

{-- Fra poco spengono le luci.}

{-- Ti aiuterò a ritrovare la strada al buio. OK?}

{-- È solo che non so se sapevi che ore erano quando\ldots{}}

{-- \emph{Io} so sempre che ore sono.}

{Bean si tenne in bocca il mugolio che avrebbe voluto emettere. Erano
	alle solite. Qualunque conversazione avesse con Ender, sempre si tornava
	su quel tasto. Quello che Bean odiava. Lui era pur capace di riconoscere
	la genialità di Ender, e per questo lo stimava. Perché Ender non
	riusciva mai a vedere niente di buono in lui?}

{-- Ricordi quattro settimane fa, Bean? Quando mi hai chiesto di
	diventare capobranco?}

{-- Uh-uh.}

{-- Io ho nominato cinque capibranco e cinque vice, da allora. E nessuno
	di loro sei tu. -- Ender inarcò un sopracciglio. -- Ho sbagliato?}

{-- Nossignore.}

{-- Secondo te, come ti sei comportato in queste otto battaglie?}

{-- Oggi mi hanno disabilitato per la prima volta, ma il computer mi ha
	assegnato undici avversali congelati prima che mi colpissero. Non ne ho
	mai messi fuori gioco meno di cinque, in ogni battaglia. E ho sempre
	portato a termine la manovra che mi era stata assegnata.}

{-- Perché ti hanno preso a fare il soldato così giovane, Bean?}

{-- Non più giovane di quel che eri tu.}

{-- Sì, ma perché?}

{-- Non lo so.}

{-- Lo sai, invece, come lo so io.}

{-- Ho tentato d'immaginarlo, ma sono solo ipotesi. Tu sei\ldots{} molto
	bravo. Loro lo sapevano, e sapevano che dandoti qualche spinta\ldots{}}

{-- Dimmi il \emph{perché}, \emph{} Bean.}

{-- Perché hanno bisogno di noi, ecco perché. -- Bean sedette sul
	pavimento e fissò i piedi di Ender. -- Perché hanno bisogno di qualcuno
	che sconfigga gli Scorpioni. Questa è l'unica cosa che a loro importa.}

{-- È necessario che tu lo sappia, Bean. Perché in questa scuola molti
	ragazzi pensano che le battaglie in sala siano importanti \emph{di per
		sé}, \emph{} mentre non è così. Servono ad aiutarli a trovare ragazzi
	che possano essere avviati a posti di comando, nella guerra vera. In
	quanto alle gare, le renderanno più dure. Stanno dando un giro di vite
	al sistema.}

{-- Divertente. Credevo che lo stessero dando a noi.}

{-- La prima battaglia di un'orda, nove settimane in anticipo. Poi una
	battaglia al giorno. E adesso due nello stesso giorno. Bean, io non so
	cosa stiano facendo gli insegnanti, ma la mia orda è stanca, io comincio
	a essere stanco, e a loro sembra che non importi neppure che le gare
	abbiano un regolamento. Ho cercato nel computer le registrazioni più
	vecchie: nessuno ha mai vinto tanto e con tante poche perdite, fin da
	quando esistono le gare di battaglia.}

{-- Tu sei il migliore, Ender.}

{Lui scosse il capo. -- Forse. Ma non è per caso che mi hanno dato i
	soldati di cui dispongo. Novellini, scarti di altre orde, ma falli
	lavorare insieme e il peggiore di loro potrebbe essere un ottimo
	capobranco in qualunque orda. Finora s'erano limitati a rendermi dura la
	vita, ma adesso stanno indurendo tutto il sistema contro di me. Bean,
	vogliono spezzarci la schiena.}

{-- Non possono spezzare te.}

{-- Ti sorprenderebbe, se ci riuscissero? -- Ender emise un sospiro
	secco, a denti stretti, come a un'improvvisa fitta di dolore. Bean
	scrutò il suo volto e s'accorse che l'impossibile stava accadendo: lungi
	dal perseguitarlo ed esasperarlo, Ender Wiggin si stava confidando con
	lui. Non molto. Ma un po' sì. Ender era un essere umano, e a lui veniva
	concesso di saperlo.}

{-- Forse ne saresti sorpreso tu -- disse Bean.}

{-- C'è un limite alle idee nuove o intelligenti che io posso tirar
	fuori ogni giorno. Qualcuno può sempre rivolgere contro di me
	stratagemmi che non ho mai neppure lontanamente immaginato, e allora non
	saprei come fronteggiarli.}

{-- Il peggio che può succederti è che avrai perso una battaglia.}

{-- Il peggio che può succedermi è proprio questo. Io non posso perdere
	nessuna battaglia. Perché se perdessi\ldots{}}

{Tacque, senza spiegarsi meglio e Bean non fece domande.}

{-- Ho bisogno che tu faccia lavorare il cervello, Bean. Voglio che tu
	pensi alla soluzione di problemi che ancora non ci siamo mai trovati di
	fronte. Voglio che tu tenti cose che nessuno ha mai tentato perché sono
	assolutamente stupide.}

{-- Perché io?}

{-- Perché anche se nell'orda dei Draghi ci sono alcuni soldati migliori
	di te\ldots{} non molti, ma alcuni sì\ldots{} non c'è nessuno che riesca
	a pensare meglio e più in fretta di te. -- Bean non disse niente.
	Entrambi sapevano che era vero.}

{Ender gli indicò lo schermo del suo banco. Su di esso c'erano dodici
	nomi di ragazzi dell'orda. -- Scegli cinque di questi -- disse. -- Uno
	da ogni branco. Saranno una squadra speciale, e tu li addestrerai. Solo
	durante gli allenamenti extra della sera. Mi farai un rapporto sulle
	cose che insegnerai loro. Non dedicare troppo tempo a ciascuna di queste
	cose. Per tutto il resto dell'orario di lavoro, tu e la tua squadra
	tornerete a far parte dell'orda, ciascuno col suo branco. Ma in
	battaglia, quando ci sarà qualcosa che soltanto tu e i tuoi potrete
	fare, sarete la mia squadra speciale.}

{-- Questi sono tutti giovani -- disse Bean. -- Nessun veterano.}

{-- Dopo quest'ultima settimana, Bean, tutti i nostri soldati sono
	veterani. Non ti sei accorto che nella classifica dell'efficienza
	individuale tutti e quaranta i nostri soldati sono fra i primi
	cinquanta? E che devi scendere al diciassettesimo posto per trovarne uno
	che \emph{non sia} un Drago?}

{-- E se non riuscissi a pensare a niente?}

{-- Allora mi sarò sbagliato su di te.}

{Bean sogghignò. -- Non ti sei sbagliato.}

{Le luci si spensero.}

{-- Saprai ritrovare la strada al buio, Bean?}

{-- Facile che domani mi trovino addormentato in qualche corridoio.}

{-- Allora resta qui. E se terrai gli orecchi aperti potrai sentire la
	buona fata che stanotte passerà a lasciarci il nostro solito regalo
	quotidiano.}

{-- Diavolo, non vorranno farci combattere anche domattina\ldots{} o
	sì?}

{Ender non rispose. Bean lo sentì distendersi sul letto. A tentoni
	estrasse dalla parete la cuccetta di riserva e si sdraiò anch'egli.
	Prima che il sonno avesse la meglio riuscì a pensare a una dozzina di
	nuove idee. Ender ne sarebbe stato compiaciuto: ognuna di esse era
	stupida.}

\phantomsection\label{Orsonux20Scottux20Cardux20-ux20Ilux20Giocoux20Diux20Enderux20-ux20BY_SLY70A1_split_014.htm}{}
