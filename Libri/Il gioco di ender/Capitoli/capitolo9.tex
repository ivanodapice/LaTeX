\chapter{LOCKE E DEMOSTENE}

{~}

{~}

{~}

{-- \emph{Non l'ho chiamata qui per parlare del tempo. Come diavolo è
		possibile che un computer faccia questo?}}

{-- \emph{Non saprei.}}

{-- \emph{Come può aver ottenuto una foto del fratello di Ender, per poi
		inserirla nella grafica di questa Terra delle Meraviglie?}}

{-- \emph{Colonnello Graff, io non c'ero quando è stato programmato.
		Tutto ciò che so è che non aveva mai portato nessuno tanto avanti in
		quella partita. La Terra delle Meraviglie è già abbastanza strana, ma
		lui l'ha attraversata ed è andato oltre. In un posto al di là della Fine
		del Mondo. E\ldots{}}}

{-- \emph{Conosco il nome di quei posti. Solo non so che significato
		abbiano.}}

{-- \emph{La Terra delle Meraviglie è stata programmata qui. Viene
		menzionata in varie registrazioni. Ma quel che c'è oltre la Fine del
		Mondo non risulta da nessuna parte. E non abbiamo alcuna esperienza di
		questo.}}

{-- \emph{Non mi piace che il computer giochi così con la mente di
		Ender. Peter Wiggin è l'individuo col maggior potenziale della sua
		generazione, a parte forse la loro sorella Valentine.}}

{-- \emph{E la partita mentale è stata programmata per aiutarli a
		formarsi, e a trovare mondi in cui si trovino a loro agio.}}

{-- \emph{Lei non ha capito, maggiore Imbu, eh? Io non voglio che Ender
		si trovi a suo agio con la fine del mondo. Il nostro compito qui non è
		di essere a nostro agio con la fine del mondo!}}

{-- \emph{La Fine del Mondo, in una partita, non è necessariamente la
		fine dell'umanità nella guerra contro gli Scorpioni. Ha per Ender un
		significato del tutto personale.}}

{-- \emph{Bene. Quale significato?}}

{-- \emph{Non lo so, signore. Io non sono quel ragazzo. Lo domandi a
		lui.}}

{-- \emph{Maggiore Imbu, è a lei che lo sto domandando.}}

{-- \emph{Potrebbero esserci mille significati diversi.}}

{-- \emph{Sentiamone uno.}}

{-- \emph{Lei ha isolato il ragazzo. Forse ciò che desidera è la fine di
		questo mondo, la Scuola di Guerra. O forse riguarda la fine del mondo in
		cui è cresciuto, casa sua. Oppure è qualcosa circa il suo modo di
		competere così duramente con gli altri. Ender è un ragazzino molto
		sensibile, lo sa, e ha fatto fisicamente del male a parecchi compagni.
		Forse desidera la fine di quel sistema di cose.}}

{-- \emph{Oppure niente di tutto questo.}}

{-- \emph{La partita mentale è un rapporto fra il ragazzo e il computer.
		Insieme creano delle vicende. E si tratta di vicende reali, nel senso
		che riflettono la realtà della vita del ragazzo. Questo è tutto ciò che
		so.}}

{-- \emph{Ora le dirò ciò che so io, maggiore Imbu. Quella foto di Peter
		Wiggin non può esser stata tolta dai nostri archivi qui alla Scuola. Non
		abbiamo niente su di lui, né documenti né registrazioni elettroniche.
		Inoltre la foto è alquanto successiva all'arrivo di Ender qui.}}

{-- \emph{È trascorso appena un anno è mezzo, signore. Cosa glielo fa
		credere? Un ragazzo non può essere molto cambiato.}}

{-- \emph{Adesso si taglia i capelli in modo del tutto diverso. La sua
		bocca è stata modificata da un intervento odontoiatrico. Mi sono fatto
		spedire delle foto recenti dalla Terra e le ho confrontate. L'unico modo
		in cui il computer che abbiamo qui, alla Scuola di Guerra, può aver
		avuto quella foto è tramite richiesta radio a un computer situato sulla
		Terra. E non uno connesso a quelli della F.I. Questo presume la
		conoscenza di una chiave d'accesso, e un'autorizzazione. Non è in nostro
		potere contattare la Contea di Guilford nel North Carolina e pescare una
		foto dai loro archivi scolastici. È stato qualcuno in questa Scuola a
		prendere l'iniziativa?}}

{-- \emph{Lei non capisce, signore. Il computer della Scuola è collegato
		alla rete della F.I. Se vogliamo una foto dobbiamo, in teoria, chiedere
		un'autorizzazione; ma se il programma della partita mentale stabilisce
		che la foto è necessaria\ldots{}}}

{-- \emph{Può farsela mandare.}}

{-- \emph{Non è cosa di ogni giorno. Soltanto se è per il bene del
		ragazzo.}}

{-- \emph{OK, è per il suo bene. Ma perché? Suo fratello è pericoloso,
		lo abbiamo rifiutato dopo aver chiarito che ha una mentalità follemente
		distorta. Perché ha tanta importanza per Ender? Perché, dopo tutto
		questo tempo?}}

{-- \emph{Onestamente, signore, non lo so. E il programma della partita
		mentale non è strutturato in modo da potercelo rivelare. Esiste la
		possibilità che non lo sappia neppure lui. Si tratta di un terreno
		ancora poco esplorato.}}

{-- \emph{Sta dicendo che il programma si autocostruisce mentre va
		avanti?}}

{-- \emph{Possiamo metterla anche così.}}

{-- \emph{Non sa lui stesso dove sta mettendo i piedi, eh? Be', questo
		mi fa sentire un po' meglio. Pensavo d'essere io il solo.}}

{~}

\begin{center}
	{* * *}
\end{center}

{~}

{Valentine celebrò da sola l'ottavo compleanno di Ender, nel piccolo
	bosco dietro la loro nuova casa di Greensboro. Liberò una striscia di
	terreno dagli aghi di pino e dalle foglie, e lì scrisse il nome di lui
	con un bastoncino. Poi costruì un cono di ramoscelli in un cerchio di
	sassi e accese un fuoco. Il fumo passò fra i rami e gli aghi del pino
	sopra di lei e spiraleggiò nel cielo. \emph{Sali su nello spazio, sempre
		più in alto}, \emph{} gli augurò in silenzio. \emph{Fino alla scuola di
		Ender, fra le stelle.}}

{Non avevano ricevuto da lui una sola lettera, e per quanto ne sapevano
	quelle scritte da loro non lo avevano mai raggiunto. Nel periodo
	successivo alla sua partenza, ogni pochi giorni Mamma e Papà s'erano
	seduti davanti alla tastiera e avevano battuto lunghe lettere. Poi
	gliene avevano mandata una alla settimana, sempre in attesa di una
	risposta che non veniva mai, e quindi una al mese. Adesso erano
	trascorsi due anni e nessuno parlava più di lettere, nessuno aveva
	ricordato il suo compleanno. \emph{Lui è morto}, \emph{} si disse
	tristemente, \emph{perché lo abbiamo dimenticato.}}

{Ma Valentine non lo aveva dimenticato. Senza farne parola con i
	genitori, e soprattutto attenta che Peter non lo intuisse, aveva
	continuato a pensare a lui ed a scrivergli lettere pur sapendo che non
	avrebbe ricevuto nessuna risposta. E quando Mamma e Papà li avevano
	informati che avrebbero lasciato la città sotterranea per trasferirsi
	nel North Carolina, Valentine aveva capito che non si aspettavano di
	rivedere Ender mai più. Se ne andavano dall'unico posto dove lui avrebbe
	saputo rintracciarli. Come avrebbe più potuto trovarli lì, fra quegli
	alberi, sotto quel cielo pesante e mutevole? Per tutta la vita Ender
	aveva vissuto nella luce artificiale dei corridoi, e adesso, chiuso
	nella Scuola di Guerra, aveva ancor meno contatto con la natura. In
	tutto quel verde si sarebbe perso.}

{Valentine sapeva perché s'erano trasferiti lì. Era stato per Peter,
	affinché il vivere fra gli alberi e i piccoli animali, in quella che
	Mamma e Papà pensavano fosse la sana natura primordiale, avesse
	un'influenza positiva su quel loro figlio così preoccupante e strano. In
	un certo senso l'aveva avuta. Peter era molto immerso in quell'ambiente.
	Lunghe passeggiate all'aria aperta, tagliando per i boschi e i campi;
	assenze che duravano a volte un giorno intero, con un coltello a
	serramanico in tasca e sulla schiena lo zaino contenente il suo banco e
	un paio di sandwich.}

{Ma Valentine sapeva. Lei aveva visto lo scoiattolo spellato vivo,
	legato per le zampe a quattro bastoncini conficcati nel fango. E con
	l'immaginazione continuava a vedere Peter catturarlo, metterlo in croce,
	e poi spellarlo con fredda e pensosa attenzione per portare allo
	scoperto l'intreccio dei muscoli rossi e pulsanti. Quanto ci aveva messo
	lo scoiattolo a morire? Fissando il focherello le parve di vedere Peter
	seduto con la schiena poggiata all'albero dove forse l'animaletto aveva
	fatto il nido, occupato a giocare col suo banco mentre la vita dello
	scoiattolo sgocciolava via.}

{Quella sera rientrò in casa ancor così inorridita che non poté mandar
	giù un boccone, e guardò Peter mangiare con grande appetito
	chiacchierando piacevolmente. Ma più tardi ci ripensò, giungendo alla
	conclusione che forse per Peter quello era stato una sorta di rito
	magico, come il suo piccolo fuoco: un sacrificio teso a placare gli
	oscuri Dei che davano la caccia alla sua anima. Meglio torturare
	scoiattoli che gli altri ragazzi. Peter era sempre stato un coltivatore
	di dolore: lo piantava, lo annaffiava, e giunto a maturazione lo
	divorava avidamente. Meglio che se ne nutrisse con piccoli sporchi
	bocconi di quel genere che con crudeltà compiute ai danni dei suoi
	compagni di scuola.}

{-- Uno studente modello -- dicevano i suoi insegnanti. -- Vorrei averne
	altri cento come lui. Studia assiduamente, non termina mai un compito in
	ritardo. Ha desiderio d'imparare.}

{Ma Valentine sapeva che era una mistificazione. Peter voleva imparare,
	certo, ma non erano gli insegnanti a dargli un'istruzione. Lui studiava
	per suo conto a casa, sul banco, collegando lo schermo a biblioteche e
	banche di dati; seguiva programmi suoi e, soprattutto, parlava con
	Valentine. A scuola poi si comportava come se le puerili lezioni del
	giorno lo eccitassero. «Oh! Ah! Non sapevo che le rane, dentro, fossero
	fatte così!» era capace di dire. Ma a casa studiava la struttura intima
	delle cellule e le attività chimiche del DNA. Peter era un maestro della
	mistificazione, e i suoi insegnanti ci cascavano.}

{Ciò malgrado lo si poteva definire un buono. Non si batteva con
	nessuno, non faceva il bullo, andava d'accordo con tutti quanti. Era un
	nuovo Peter.}

{Così tutti credevano. Mamma e Papà lo dicevano così spesso che
	Valentine finiva per mugolare spazientita. \emph{Non è un nuovo Peter! È
		il vecchio Peter, solo più sottile!}}

{Quanto sottile? \emph{Più sottile di te, Papà. Più sottile di te,
		Mamma. Più sottile di chiunque abbiate conosciuto.}}

\emph{{Ma non più sottile di me.}}

{-- Sto ancora cercando di decidere -- disse Peter, -- se assassinarti o
	cos'altro.}

{Valentine s'appoggiò al tronco rugoso del pino e sospirò sulle ceneri
	sparse del suo focherello. -- Anch'io ti amo tanto, Peter.}

{-- Sarebbe talmente facile. Fai sempre questi stupidi piccoli fuochi.
	Un colpo alla nuca, un tizzone fra le vesti, ed ecco costruito lo
	sfortunato incidente. Così periscono le falene.}

{-- E io ho pensato di castrarti nel sonno.}

{-- No, non è vero. Tu pensi cose come queste soltanto quando sei con
	me. Perché io porto fuori il meglio di te. No, credo che non ti
	ucciderò, Valentine. Ho deciso che puoi essermi d'aiuto.}

{-- Io, eh? -- Qualche anno prima le minacce di Peter l'avrebbero
	terrorizzata. Ora invece non le facevano più molto effetto. Non che ne
	dubitasse: era capace di ucciderla davvero. Non riusciva a pensare a un
	solo delitto, per quanto terribile, che lui non avrebbe potuto
	commettere. Sapeva anche che non era pazzo, nel senso che non perdeva il
	controllo di se stesso. Era l'individuo più controllato che lei
	conoscesse. Salvo forse lei stessa. Peter sapeva rimandare l'esaudimento
	di un desiderio per tutto il tempo che gli era necessario; riusciva a
	mascherare qualsiasi emozione. Perciò Valentine era certa che non
	l'avrebbe mai uccisa in un accesso di rabbia. L'avrebbe fatto solo se i
	rischi fossero stati inferiori ai vantaggi. E non lo erano. In un certo
	senso si fidava di Peter più che di altri proprio per questo: sempre, e
	invariabilmente, le sue azioni erano calcolate in base ai suoi
	interessi. E così, per tutelare se stessa, le bastava accertarsi che
	Peter trovasse più vantaggioso lasciarla in vita.}

{-- Valentine, i nodi stanno venendo al pettine. Ho scoperto dei
	movimenti di truppe in Russia.}

{-- Di cosa stiamo parlando?}

{-- Del mondo, Val. Hai sentito parlare della Russia? L'impero
	sovietico? Il Patto di Varsavia? Quelli che tengono in pugno l'Eurasia
	dall'Olanda al Pakistan?}

{-- Non rendono pubblici i loro movimenti di truppe, Peter.}

{-- Naturalmente no. Ma le ferrovie sovietiche pubblicano mensilmente il
	numero di passeggeri trasportati sulle varie linee. Io ho fatto
	analizzare al mio banco queste statistiche, per estrapolare quando treni
	contenenti truppe potrebbero muoversi sulle stesse linee. Già da tre
	anni sto dietro a questa cosa. Negli ultimi sei mesi su certe linee ci
	sono stati i mutamenti di orario e la diminuzione dei convogli
	passeggeri da me previsti. Si stanno preparando alla guerra. Sul
	territorio, almeno.}

{-- E che mi dici degli Alleati? E degli Scorpioni? -- Valentine non
	sapeva a cosa lui stesse mirando. Ma spesso Peter la attirava in
	discussioni di quel genere, in tono pratico e sugli eventi del mondo. La
	usava per mettere alla prova le proprie idee, e perfezionarle. In quel
	procedimento anche lei perfezionava le sue opinioni. Aveva scoperto che,
	mentre di rado si trovava d'accordo con Peter sul \emph{come} il mondo
	avrebbe dovuto essere, spesso concordava con lui su ciò che il mondo
	effettivamente \emph{era.} Erano diventati abbastanza esperti
	nell'estrapolare informazioni plausibili dai servizi filmati o stampati
	dei giornalisti, spesso ignoranti e quasi sempre superficiali. I
	manovali della notizia, come li chiamava Peter.}

{-- Il Condottiero è russo, no? E lui sa cosa sta succedendo nelle alte
	sfere della Flotta, sia che gli Scorpioni non siano più considerati una
	minaccia, sia che s'avvicini una grossa battaglia. In un caso o
	nell'altro la guerra con gli Scorpioni sta per concludersi. E i russi si
	preparano per quel che accadrà dopo.}

{-- Se spostano truppe, dev'essere sotto la direzione dello Stratega.}

{-- È tutto interno ai confini del Patto di Varsavia.}

{Questo era preoccupante. La facciata della pace e della collaborazione
	continuava indisturbata fin dall'inizio delle ostilità con gli
	Scorpioni. Ciò che Peter aveva scoperto contrastava gravemente con
	quella situazione. E lei aveva un quadro mentale, chiaro come un
	ricordo, dei comportamenti delle nazioni prima che gli extraterrestri le
	costringessero a unirsi. -- Dunque tutto sta tornando com'era un tempo.}

{-- Con pochi cambiamenti. Lo scudo spaziale continuerà a impedire il
	lancio di missili e di armi atomiche, così dovremo ammazzarci l'un
	l'altro a migliaia invece che a milioni. -- Peter sogghignò. -- Ora come
	ora sono in attività la Flotta e le forze armate, e l'Egemonia è in mano
	agli Stati Uniti. Una volta finita la guerra con gli Scorpioni tutto
	questo potere si dissolverà, perché è tenuto insieme da una paura
	comune. E quando ci guarderemo intorno scopriremo tutto a un tratto che
	le vecchie alleanze sono svanite o moribonde. Salvo una: il Patto di
	Varsavia. E l'economia del dollaro resterà sola contro cinque milioni di
	laser. Noi avremo la Cintura degli Asteroidi, ma loro avranno la Terra,
	e lassù non ci vuol molto a finire le scorte di sedano e di uva passa,
	senza la Terra.}

{Quel che seccò maggiormente Valentine fu il vedere che Peter non
	sembrava per nulla preoccupato. -- Senti, perché mi sta venendo l'idea
	che tutto questo potrebbe essere un'opportunità dorata per Peter
	Wiggin?}

{-- Per me e per te, Val.}

{-- Peter, tu hai dodici anni. C'è una parola per quelli della nostra
	età: ci chiamano bambini, e ci trattano di conseguenza\ldots{} se non
	sgarriamo.}

{-- Ma noi non pensiamo come gli altri bambini. Giusto, Val? Non
	parliamo come bambini. E soprattutto non \emph{scriviamo} come bambini.}

{-- Per una chiacchierata cominciata con minacce di morte, Peter, mi
	pare che siamo andati alquanto fuori argomento. -- Tuttavia Valentine si
	accorse d'essere eccitata. Scrivere era una cosa che faceva meglio di
	Peter. Entrambi lo sapevano. Ne aveva perfino parlato una volta, quando
	aveva dichiarato che lui riusciva a capire ciò che gli altri odiavano di
	più in se stessi, per poi tormentarli, mentre lei intuiva quello che in
	loro li compiaceva di più e se ne serviva per adularli. Era un modo
	cinico di vedere la cosa, ma era vero. Valentine sapeva far accettare
	agli altri ì suoi punti di vista; riusciva a convincerli che
	desideravano ciò che lei voleva che desiderassero. Peter, per contro,
	poteva indurli a temere quel che voleva che temessero.}

{Quando lui glielo aveva fatto notare, lei se n'era impermalita. Le
	piaceva pensare d'esser brava a convincere la gente perché aveva
	ragione, non perché era più svelta di mente. Ma per quanto dicesse a se
	stessa che non avrebbe mai manovrato qualcuno nel modo esposto da Peter,
	la rallegrava sapere che, a suo modo, avrebbe potuto controllare gli
	altri. E non soltanto ciò che facevano. Lei riusciva a controllare ciò
	che \emph{volevano} fare. Provar piacere per quella capacità le
	rimordeva la coscienza, ciò malgrado talvolta s'era scoperta a usarla.
	Far sì che gli insegnanti agissero come lei voleva, e così gli altri
	studenti. Far sì che Mamma e Papà vedessero una cosa dal suo punto di
	vista. A volte era capace di persuadere perfino Peter. Questa era la
	cosa più terribile di tutte: capire Peter a tal punto e avere con lui
	un'empatia così profonda da entrare nella sua testa a quel modo. Dentro
	di lei c'era più \emph{Peter} di quanto sopportasse di ammettere, benché
	ogni tanto riuscisse a esaminarsi fino a quel livello. E mentre Peter
	parlava, in lei tornò quel pensiero: \emph{Tu sogni il potere, Peter. Ma
		a mio modo io sono più potente di te.}}

{-- Ho studiato a fondo la storia -- disse Peter, -- e ho imparato molto
	sulla meccanica del comportamento umano. Ci sono periodi in cui il mondo
	si torce per mutare se stesso, e in quei momenti chiave la parola giusta
	può cambiarne il destino. Pensa a ciò che fece Pericle ad Atene, e
	Demostene\ldots{}}

{-- Sì, e portarono Atene alla rovina.}

{-- Pericle, sì, ma Demostene aveva ragione su Filippo\ldots{}}

{-- O non fece che provocarlo\ldots{}}

{-- Vedi? Questo è proprio ciò che fanno gli storici. Chiacchierano
	sulle cause e sugli effetti, quando il punto è: ci sono periodi in cui
	il destino è fluido, e la giusta voce nel giusto luogo può muovere il
	mondo. Thomas Paine e Ben Franklin, ad esempio. Bismark. Lenin.}

{-- Non sono esattamente casi paralleli, Peter. -- Adesso gli stava
	dando torto a bella posta. Vedeva dove lui stava mirando, e pensò che la
	cosa era possibile.}

{-- Non mi aspettavo che \emph{tu} capissi. \emph{Tu} credi ancora che
	la scuola possa insegnarti qualcosa di valido.}

\emph{{Io capisco più di quel che credi}}{, \emph{} Peter. -- Così, vedi
	te stesso come un Bismark?}

{-- Io vedo in me stesso uno che sa come inserire idee nella mente delle
	masse. Non ti è mai successo di trovarti a dire una frase intelligente,
	un'opinione azzeccata e poi, magari un mese dopo, sentire un adulto che
	la ripete a un altro adulto, tutti e due a te sconosciuti? O di sentirla
	in un filmato o in una trasmissione TV?}

{-- Sì, ma ho sempre pensato di averla anch'io sentita in precedenza, e
	di non aver fatto altro che ripeterla.}

{-- E ti sbagliavi. Ci sono forse due o tremila persone al mondo
	intelligenti quanto noi, sorellina. Per la maggior parte si stanno
	sudando la vita da qualche parte. Insegnando, i poveri bastardi, o
	facendo ricerche. Pochi di loro sono attualmente in posizioni di
	potere.}

{-- Quei pochi fortunati, suppongo, siamo tu e io.}

{-- Divertente come un coniglio con le gambe rotte, Val.}

{-- Dei quali, senza dubbio, ce se saranno molti in questo boschi.}

{-- Quando nevica possono sempre sciare sulla pancia.}

{Valentine rise di quella ridicola immagine, e odiò se stessa per aver
	pensato che fosse comica.}

{-- Val, noi possiamo dire le parole che ogni altro ripeterà dopo un
	paio di settimane. Possiamo farlo. Non siamo tenuti ad aspettare
	d'essere cresciuti e tranquillamente inquadrati in qualche professione.}

{-- Peter, tu hai \emph{dodici} anni.}

{-- Non per i mezzi di comunicazione di massa. Sui sistemi
	computerizzati io posso usare lo pseudonimo che preferisco, e tu anche.}

{-- Sulle reti di computer noi siamo chiaramente identificati come
	studenti, e non possiamo interferire con l'opinione pubblica se non
	sotto questa veste, il che significa che non potremmo o non ci
	lascerebbero dire nulla di effettivo.}

{-- Io ho un piano.}

{-- Tu hai sempre un piano. -- Val fingeva un'ironica indifferenza, ma
	ascoltava con attenzione.}

{-- Possiamo inserirci sulle reti computerizzate come adulti a pieno
	titolo, e con qualsiasi nome vogliamo adottare, \emph{se} Papà ci lascia
	usare il suo codice d'accesso di cittadino.}

{-- E perché dovrebbe farlo? Abbiamo già i nostri codici di studenti.
	Cosa pensi di dirgli: Papà, ho bisogno di un codice da adulto, così
	potrò impadronirmi del mondo?}

{-- No, Val. \emph{Io} non gli dirò niente. \emph{Tu} andrai a dirgli
	quanto sei preoccupata per me. Quanto sudo e soffro per andare bene a
	scuola. E dirai d'esser certa che sto scivolando nella pazzia perché non
	trovo nessuno abbastanza intelligente da parlare con me, e che tutti mi
	zittiscono perché sono così giovane, e che non sono mai riuscito a
	conversare coi \emph{miei pari.} Tu puoi dimostrargli che questo stress
	mi sta facendo uscire di cervello.}

{Valentine ripensò al corpicino dello scoiattolo nel bosco, e capì che
	perfino quella scoperta rientrava nel piano di Peter. O almeno, dopo
	essersi divertito lui l'aveva inclusa nel suo piano.}

{-- Dunque devi convincerlo a lasciarci usare il suo codice. E ad
	assumere nuove identità così che la gente possa darci il rispetto che
	meritiamo.}

{Valentine poteva contrastare le sue idee, ma non affermazioni di quel
	genere. Non se la sentiva neppure di chiedergli: cosa ti fa pensare che
	meriti rispetto? Aveva letto molto su Adolf Hitler. Si domandò che tipo
	fosse stato a dodici anni. Non così intelligente, non simile a Peter, ma
	certo altrettanto avido di riconoscimenti. E se da bambino fosse stato
	travolto da una falciatrice, quali ne sarebbero stati gli effetti sul
	resto del mondo?}

{-- Val -- disse Peter, -- so cosa pensi di me. Io non sono una persona
	amabile, questo pensi.}

{Valentine gli tirò un grosso ago di pino. -- Una freccia nel tuo cuore
	nero, Jago!}

{-- Ci ho pensato a lungo prima di venire a parlarti di questo. E avevo
	paura.}

{Lei si mise un altro ago di pino fra le labbra e lo soffiò avanti. Le
	cadde quasi in grembo. -- Ancora un colpo fallito. -- Perché fingeva di
	mostrarle qualche debolezza?}

{-- Val, avevo paura che tu non mi credessi. O che non volessi credere
	che io posso farlo.}

{-- Peter, io credo che potresti fare di tutto. E probabilmente lo
	farai.}

{-- Ma la mia paura maggiore era che tu mi credessi e cercassi di
	fermarmi.}

{-- Avanti, adesso minacciami ancora di un'orrida morte, Peter. --
	Credeva davvero che lei si lasciasse abbindolare dal personaggio del
	bambino umile e contrito?}

{-- Merito il tuo sarcasmo, va bene. Mi spiace. Ma adesso parlo sul
	serio, ho bisogno del tuo aiuto.}

{-- Sei proprio ciò di cui il mondo ha bisogno: un bambino di dodici
	anni per risolvere tutti i nostri problemi.}

{-- Non è colpa mia se proprio in questo momento ho dodici anni. E non è
	colpa mia se proprio questo è il momento in cui l'opportunità si apre.
	Il momento in cui io posso dar forma agli eventi. Il mondo è ancora a
	regime democratico, in un periodo fluido, e a un uomo basta
	l'intelligenza per aver successo. Tutti pensano che Hitler sia andato al
	potere grazie alle sue camicie brune e alla loro violenza, e questo è in
	parte vero, perché nella bruta realtà il potere è sempre costruito sulla
	capacità di minacciare qualcuno. Ma più che altro lui trasse potere
	dalle parole. Le parole giuste al momento giusto.}

{-- Pensavo or ora che fra te e lui ci sono dei punti in comune.}

{-- Io non odio gli ebrei, Val. Non voglio sterminare nessuno. E non
	voglio neppure la guerra. Desidero che il mondo sia unito. È un'idea
	tanto malvagia? Non dobbiamo tornare ai vecchi sistemi. Cos'hai letto
	sulle guerre mondiali?}

{-- Abbastanza.}

{-- Potremmo ricadere nello stesso sbaglio. O peggio. Potremmo trovarci
	a far parte anche noi del Patto di Varsavia. Ecco un pensiero poco
	divertente.}

{-- Peter, noi siamo bambini, non lo capisci? Stiamo andando a scuola,
	stiamo crescendo\ldots{} -- Ma anche mentre gli resisteva, voleva che
	lui la convincesse. Lo aveva voluto fin dal principio.}

{Peter però non sapeva d'aver già vinto. -- Se credessi in un futuro di
	questo genere, se lo accettassi, non dovrei che starmene seduto e
	lasciar svanire l'opportunità, perché quando saremo adulti sarà troppo
	tardi. Dammi ascolto, Val. So quel che provi e hai sempre provato per
	me. Ma io non ti odio. Vi amavo entrambi, solo che dovevo essere\ldots{}
	dovevo avere il \emph{controllo.} Lo capisci questo? È la cosa più
	importante per me, ed è la mia dote migliore: io vedo dove sono i punti
	deboli, vedo come arrivare ad essi e manovrarli, e sono cose che
	\emph{vedo} senza neppure sforzarmi. Volendo potrei diventare un uomo
	d'affari in qualche grossa ditta e arricchirmi più di chiunque altro, ma
	cos'avrei ottenuto alla fine? Niente. Quello che io voglio è governare,
	Val, e avere il controllo reale delle cose. Ma voglio anche avere
	qualcosa che meriti di essere governato. E voglio portare a termine
	progetti veramente grossi. Una Pax Americana sull'intero pianeta. In
	modo che se venisse qualcun altro, dopo che avremo battuto gli
	Scorpioni, se altre razze aliene cercassero di attaccarci, scoprirebbero
	che ci siamo già sparsi su mille pianeti, forti, pacifici e impossibili
	da distruggersi. Capisci? Io voglio salvare la razza umana
	dall'autodistruzione.}

{Non lo aveva mai visto parlare con quella sincerità. Senza un filo di
	sarcasmo né ombra di menzogna nella voce. Stava imparando a vivere il
	ruolo che recitava. O forse a recitare nel ruolo in cui credeva. --
	Così, un ragazzo di dodici anni e la sua sorellina stanno per salvare il
	mondo?}

{-- Alessandro quanti anni aveva? Non presumo certo di farlo nel giro di
	una notte. Soltanto, devo cominciare adesso. Se mi aiuti.}

{-- Non credo che quello che hai fatto allo scoiattolo fosse parte di
	una commedia. Penso che tu l'abbia torturato perché ci provavi gusto.}

{D'un tratto Peter si coprì il volto con le mani e pianse. Val diede per
	scontato che fingesse, ma se ne stupì ugualmente. Era possibile che lui
	le volesse bene davvero? No, si disse, ma trovandosi dinnanzi a quella
	che vedeva come la sua grande opportunità forse desiderava umiliarsi di
	fronte a lei per conquistare il suo affetto. \emph{Mi sta manipolando},
	\emph{} pensò \emph{ma questo non significa che non sia sincero.} Le
	guance di lui erano umide quando abbassò le mani, e aveva gli occhi
	gonfi. -- Lo so -- mormorò. -- È questo a spaventarmi di più: che io sia
	davvero un mostro. Io non voglio essere un killer, solo che non so cosa
	farci.}

{Non lo aveva mai sentito ammettere così le sue debolezze. \emph{Sei
		così abile, Peter! Hai messo da parte anche le lacrime per poterle usare
		al momento giusto su di me.} E tuttavia questo non la commosse, perché
	dimostrava che era vero almeno in parte che lui non era un mostro, e
	dunque lei poteva lasciar spazio al suo stesso e non diverso amore per
	il potere senza la paura di diventare mostruosa anche lei. Sapeva che
	Peter stava agendo secondo un calcolo preciso, ma era certa che le aveva
	concesso di gettare uno sguardo sulla sua anima. Era nascosta sotto
	strati e strati di fredda pietra, e doveva essergli costato caro
	riportarla alla luce.}

{-- Val, se non mi aiuti io non so cosa diventerò. Ma se tu mi starai
	accanto, mia compagna in tutto quel che faremo, potrai impedirmi di
	cadere nel baratro, quello dove finiscono i dannati.}

{Lei annuì. \emph{Stai solo fingendo di voler dividere il potere con
		me}, \emph{} pensò. \emph{Ma in realtà io ho potere su di te, anche se
		non lo sai.} -- Lo farò. Ti aiuterò.}

{~}

\begin{center}
	{* * *}
\end{center}

{~}

{Appena il loro padre li autorizzò a usare il suo codice d'accesso con
	gli schermi di casa, Valentine e Peter cominciarono a tastare il
	terreno. Si tennero alla larga dalle reti di video-giornali con cui era
	richiesto l'uso del nome vero, cosa non difficile poiché la firma
	autentica era legata solo alla necessità di ricevere un compenso. Loro
	non avevano bisogno di denaro. Avevano bisogno di rispetto, per
	guadagnarne altro ancora. Con un nome falso, e sui video-giornali che
	accettavano interventi esterni specie quando gratuiti, potevano essere
	chiunque: uomini anziani, casalinghe di mezz'età, professionisti o
	piccoli politicanti locali, finché stavano attenti allo stile con cui
	scrivevano. Trasmettere un articolo di commento politico o culturale a
	un quotidiano a diffusione regionale costava circa quanto ricevere lo
	stesso quotidiano sullo schermo del tavolo la mattina dopo. E tutto quel
	che la gente avrebbe visto di loro sarebbero state le loro parole, le
	loro idee.}

{Per i primi articoli, che furono ben accetti, usarono nomi diversi, con
	le identità che Peter aveva già programmato di rendere famose e
	influenti. Ovviamente nessuno li contattò per invitarli a collaborare ai
	grandi video-giornali nazionali e internazionali; rispetto a questi essi
	potevano soltanto esser parte del pubblico. Ma potendo attingere al
	conto del padre riuscivano a tenersi aggiornati, leggevano gli articoli
	firmati dai commentatori politici più famosi sul loro banco personale,
	collegato al computer di casa, e su di esso assistevano ai dibattiti
	televisivi più pregnanti.}

{E sui piccoli quotidiani locali, dove anche la gente comune interveniva
	per discutere questioni nazionali e internazionali, cominciarono a
	inserire regolarmente i loro articoli. Fin dall'inizio Peter insisté che
	fossero deliberatamente provocatori. -- Non possiamo capire in che
	misura il nostro stile funziona, finché non otterremo delle
	risposte\ldots{} e a opinioni blande nessuno risponde mai.}

{Non furono blandi, e la gente rispose. Le risposte che ebbero sulle
	reti di video-giornali furono acide. Quelle che furono indirizzate loro
	per videoposta, servizio a cui Peter e Valentine accedevano con un
	codice personale da abbonati, o erano entusiaste o grondavano veleno.
	Presto appresero quali particolari dei loro saggi risultavano
	bambineschi o immaturi, e cominciarono a fare di meglio.}

{Quando Peter fu sicuro che entrambi sapevano come fingersi adulti,
	misero da parte le prime identità sperimentali e si apprestarono a
	destare attenzione su più larga scala.}

{-- Nessuno dovrà sospettare alcun collegamento fra noi. Scriveremo su
	argomenti diversi e in momenti diversi. Ognuno eviterà riferimenti
	all'altro. Tu lavorerai in prevalenza con le reti della costa
	occidentale, io nel meridione. E per fingere d'essere di casa lì, non
	trascureremo i video-quotidiani locali.}

{Quel lavoro li assorbì completamente. Mamma e Papà talvolta si
	preoccupavano nel vederli trascurare ogni altra cosa, sempre insieme, ma
	i loro volti erano buoni e sembrava chiaro che Valentine aveva
	un'influenza positiva sul fratello. Del resto, anche lei mostrava d'aver
	assunto attitudini nuove. Nei giorni di bel tempo andavano a sedersi
	insieme nei boschi; quando pioveva s'appartavano in un locale pubblico
	tranquillo o nei parchi coperti, e componevano i loro articoli politici.
	Peter aveva disegnato con cura le due personalità, in modo che fossero
	diverse sia nelle idee che nel modo di esporle; c'erano anche alcune
	identità spicciole che usavano per lasciar cadere qua e là opinioni di
	un terzo genere, o attacchi alle prime due. -- Lasciamo che ciascuna di
	esse trovi dei seguaci, se può -- disse Peter.}

{Un giorno, stanca di scrivere e riscrivere finché il fratello fosse
	soddisfatto, Val esclamò disperata: -- Scrivilo tu stesso, allora!}

{-- Non posso -- rispose lui. -- Mai. Rischieremmo di mescolare i due
	stili. Non scordare che un giorno saremo abbastanza famosi da indurre
	qualcuno a fare delle analisi su di noi. Dobbiamo risultare persone
	diverse, articolo per articolo.}

{E Valentine s'impegnò al meglio. Il suo nome di battaglia sui
	video-giornali, il principale, era ormai Demostene. L'aveva scelto
	Peter, e in quanto a lui si firmava Locke. Erano ovvi pseudonimi, ma
	anche ciò faceva parte del piano. -- Con un po' di fortuna, la gente
	comincerà a cercar di scoprire chi siamo.}

{-- Se diventiamo troppo famosi, i servizi segreti non ci metteranno
	niente a scoprirlo. Per il solo piacere di avere un dossier su di noi.}

{-- Quando accadrà, saremo già troppo in alto per soffrirne un danno. La
	gente riceverà un colpo nell'apprendere che Demostene e Locke sono due
	ragazzini, ma tutti quanti saranno già abituati ad ascoltarci.}

{Cominciarono a comporre dibattiti per le loro personalità fittizie.
	Valentine avrebbe studiato un articolo d'apertura, e Peter avrebbe usato
	un nome provvisorio per replicarle. La sua risposta avrebbe dovuto
	essere intelligente e rappresentare una parte dell'opinione pubblica,
	condita con acuto sarcasmo e una buona dose di retorica. Valentine aveva
	un intuito per le allitterazioni che rendeva memorabili certe sue frasi.
	Poi anche Locke sarebbe entrato nel dibattito, separato da un
	ragionevole lasso di tempo. Se altri giornalisti avessero interposto i
	loro commenti, Peter e Val li avrebbero ignorati, o avrebbero modificato
	i propri articoli per adeguarli a ciò che era stato già scritto.}

{Peter teneva un accurato elenco di tutte le loro frasi più originali, e
	di tanto in tanto vagliava la pagina politica dei video-giornali per
	vedere se una di esse sbucava fuori qua e là. Non tutte ebbero presa, ma
	molte furono ripetute anche da reti a larga diffusione, e alcune ebbero
	tanto successo da comparire sulla stampa internazionale, citate o
	parafrasate in vari modi. -- Ci stanno leggendo -- approvò Peter. -- Le
	idee stanno mettendo radici.}

{-- Le frasi celebri, comunque.}

{-- Quelle ne sono solo il veicolo. Guarda, cominciamo ad avere una
	certa influenza. Nessuno ancora cita i nostri nomi, certo, ma discutono
	gli argomenti che abbiamo messo in campo. Alcuni sono già all'attenzione
	del governo. È lì che dobbiamo mirare.}

{-- Ci muoviamo per entrare in un grosso video-giornale?}

{-- No. Aspetteremo che ce lo propongano.}

{Stavano lavorando da soli sette mesi quando una rete di distribuzione
	della costa occidentale contattò Demostene per videoposta. Era
	un'offerta per una colonna settimanale su uno dei quotidiani più letti.}

{-- Non posso fare una colonna settimanale -- disse Valentine. -- Non ho
	ancora neppure un periodo mensile.}

{-- Le due cose non sono collegate -- disse Peter.}

{-- Per me sì. Io sono ancora una bambina.}

{-- Rispondi che accetti, e che siccome preferisci non rivelare la tua
	identità desideri esser pagata addebitando loro ogni secondo che passi
	collegata alla videostampa, tramite un nuovo codice d'accesso avallato
	dai loro computer.}

{-- Così, quando i servizi segreti mi scopriranno\ldots.}

{-- Sarai soltanto una persona che si inserisce nelle reti di
	distribuzione facendo pagare alla CalNet. Il codice di cittadino di Papà
	non ci sarà più coinvolto. Quello che non riesco a capire è perché hanno
	voluto Demostene prima di Locke.}

{-- Che abbiano fiuto per il talento?}

{Vista come una sfida, era divertente. Ma a Valentine non piacevano
	certe posizioni che Peter imponeva a Demostene. Questi infatti cominciò
	a sviluppare un'ostilità paranoica verso il Patto di Varsavia. A
	preoccuparla c'era il fatto che Peter era il solo a sapere come destare
	una paura strisciante nei lettori; questo la costringeva a ricorrere a
	lui sia per la tecnica che per le idee spicciole. Intanto Peter,
	firmandosi Locke, propugnava strategie moderate più adatte a lei. Era un
	particolare studiato a bella posta, ma il suo effetto principale fu di
	legarla ancor più inestricabilmente a Peter. Val non avrebbe potuto
	rendersi indipendente da lui e usare Demostene per i suoi scopi: non
	sapeva come farne uso, da sola. Ma il legame funzionava nei due sensi,
	perché neppure Peter poteva far parlare Locke senza di lei. O avrebbe
	potuto?}

{-- Pensavo che l'idea fosse di unificare il mondo. Se scrivo questo
	articolo come vuoi tu, sembrerà che io invochi la guerra contro il Patto
	di Varsavia.}

{-- Niente guerra, bensì apertura delle reti di comunicazione
	internazionale, abolizione della censura sovietica e libertà
	d'informazione. Ossequienza alle stesse regole cui ubbidiscono gli
	Alleati, per la salvezza comune.}

{Senza volerlo Valentine replicò nello stile che usava per i suoi
	articoli, pur esprimendo un'opinione diversa da quella di Demostene. --
	Tutti sanno che fin dall'inizio il Patto di Varsavia fu considerato come
	una singola entità, per quanto riguarda il rispetto di quelle regole. La
	circolazione delle notizie è aperta, in campo internazionale. Ma nelle
	nazioni del Patto di Varsavia è una questione interna. Soltanto grazie a
	questo accordo essi permisero la supremazia americana fra gli Alleati.}

{-- Stai recitando la parte di Locke, Val. Ascolta me: tu devi invocare
	la cessazione di queste regole interne al Patto di Varsavia. Devi
	sollevare l'ira e il disgusto dei lettori contro di esse. Poi, in
	futuro, quando comincerai a riconoscere la necessità di certi
	compromessi\ldots{}}

{-- Non mi ascolteranno, perché li avrò già portati al punto che la
	guerra sembrerà l'unica soluzione.}

{-- Val, abbi fiducia. Io so quel che sto facendo.}

{-- Come puoi dir questo? Non sei più intelligente di me, e inoltre non
	hai mai avuto un'esperienza diretta in cose tanto complesse.}

{-- Io ho tredici anni, e tu dieci\ldots{}}

{-- Quasi undici.}

{-- E so come funzionano queste cose.}

{-- Va bene, farò a tuo modo. Ma niente retorica tipo «o la libertà o la
	morte».}

{-- Dovrai fartela piacere, invece.}

{-- E un giorno, quando ci avranno scoperti e ti chiederanno perché tua
	sorella è una tale guerrafondaia invelenita, dirai che sei stato tu a
	impormelo. Eh? Ci scommetto proprio!}

{-- Sei sicura di non avere le mestruazioni, signorina?}

{-- Peter Wiggin, io ti odio.}

{Ma ciò che sfumò di angoscia le preoccupazioni di Valentine fu quando
	la sua colonna fu venduta dalla California Network ad altre reti
	regionali, e Papà cominciò a leggerla sullo schermo del tavolo la
	mattina a colazione. -- Finalmente un uomo con un po' di buon senso! --
	esclamò il signor Wiggin. E commentò con entusiasmo alcuni dei paragrafi
	che Valentine, scrivendoli, aveva detestato di più. -- È stato bello
	lavorare con quegli imperialisti rossi finché c'erano gli Scorpioni là
	fuori, ma dopo che avremo vinto io non me la sento di lasciare metà del
	mondo civile imbavagliata e coi paraocchi. Per il nostro e per il loro
	stesso bene. Non è così, cara?}

{-- Credo che tu stia prendendo la cosa troppo seriamente -- rispose
	Mamma.}

{-- Questo Demostene mi piace. Guarda al futuro in modo giusto. È
	sorprendente che non sia pubblicato dalle reti internazionali\ldots{}
	l'ho cercato nei video a diffusione planetaria, e non l'ho ancora
	trovato. È un vero peccato. Per fortuna, la CalNet\ldots{}}

{Valentine perse ogni appetito e si alzò da tavola. Dopo qualche minuto
	Peter la raggiunse, in soggiorno.}

{-- E va bene, non ti piace mentire a Papà -- le disse. -- E con questo?
	Tu non stai mentendo \emph{a lui.} Lui non sa che Demostene sei tu, e
	Demostene non sta scrivendo ciò che tu pensi in realtà. Queste due
	menzogne si cancellano l'un l'altra, perciò.}

{-- Questo è proprio il tipo di ragionamento che fa di Locke un vero
	asino. -- Ma ciò che la angosciava non era il fatto di mentirgli quanto
	il vedere che Papà era d'accordo con Demostene. Finallora aveva creduto
	che soltanto gli sciocchi potessero condividere le sue idee.}

{Pochi giorni dopo Locke venne richiesto da un grosso videogiornale del
	New England, col preciso incarico di fornire punti di vista in contrasto
	con la colonna settimanale di Demostene. -- Niente male, per due
	ragazzini ancora più o meno impuberi, eh? -- commentò Peter.}

{-- Ci corre un bel pezzo di strada fra scrivere un articolo e governare
	il mondo -- gli rammentò Valentine. -- Ed è una strada così lunga che
	nessuno è mai riuscito a farla.}

{-- C'è chi l'ha fatta. Moralmente, intendo, non in senso politico. E
	nella mia prima colonna mi accingerò a fare a pezzi Demostene.}

{-- Be', Demostene non si è mai neppure accorto dell'esistenza di
	Locke.}

{-- Per ora.}

{Con le loro identità fittizie adesso supportate dai computer della
	videostampa, non ebbero più bisogno del codice d'accesso del padre salvo
	che per far uso di altre identità provvisorie. Mamma li rimproverò che
	trascorrevano troppo tempo attaccati agli schermi. -- Sole di vetro e
	aria di fessura, mena presto alla sepoltura -- ricordò a Peter. --
	Dovresti andare un po' a svagarti, ogni tanto.}

{Lui esibì una rassegnata mestizia. -- Se credi che io possa frequentare
	quegli sciocchi della mia età, e smettere di istruirmi, forse stavolta
	ce la farò senza sentirmi impazzire. Posso provarci.}

{-- No, no -- disse Mamma. -- Non voglio che tu smetta d'istruirti.
	Soltanto\ldots{} abbi cura di te, ecco tutto.}

{-- Io ho molta cura di me, Mamma.}

{~}

\begin{center}
	{* * *}
\end{center}

{~}

{Nulla era diverso, nulla era cambiato in quell'ultimo anno. Ender se lo
	ripeteva spesso, e tuttavia gli sembrava che ogni cosa avesse perduto
	sapore. Era sempre in vetta alla classifica dell'efficienza individuale,
	e adesso nessuno dubitava che lo meritasse. A nove anni di età era
	capobranco nell'orda delle Fenici, con Petra Arkanian come comandante.
	Dirigeva ancora gli allenamenti extra della sera, e ad essi partecipava
	ora un gruppo scelto di soldati nominati dai loro comandanti, benché
	qualunque novellino fosse il benvenuto fra essi. Anche Alai era
	capobranco, in un'altra orda, e continuava ad essere per lui un buon
	amico.}

{Shen non aveva il grado di capobranco, ma questo non era un ostacolo
	fra loro. Dink Meeker aveva finalmente accettato un comando ed era
	succeduto a Rose de Nose alla guida dell'orda dei Topi. \emph{Tutto sta
		andando bene, più che bene. Non potrei chiedere qualcosa di
		meglio\ldots{}}}

\emph{{Allora perché detesto la mia vita?}}

{Addestrarsi con l'orda e combattere in sala di battaglia era
	divertente. Gli dava soddisfazione istruire i ragazzi del suo branco, e
	loro lo seguivano lealmente. Aveva la stima di tutti, e negli
	allenamenti serali lo ascoltavano quasi con deferenza. I comandanti
	studiavano le sue tecniche. Soldati di altre orde, a mensa, si
	avvicinavano al suo tavolo e chiedevano il permesso di sedersi solo per
	ascoltarlo parlare. Perfino gli insegnanti erano rispettosi con lui.}

{Si vedeva così dannatamente rispettato che avrebbe voluto urlare.}

{Osservava i ragazzini appena arruolati nelle varie orde, ancora freschi
	dei loro ricordi di casa; guardava i loro giochi, il modo in cui si
	facevano beffe dei comandanti quando essi non erano nelle vicinanze.
	Vedeva il cameratismo dei ragazzi ormai legati da anni di vita in comune
	lì alla Scuola di Guerra, che rivangavano battaglie ormai vecchie e nomi
	di soldati e comandanti da tempo giunti al termine del corso.}

{Ma con i \emph{suoi} vecchi amici non c'erano giochi di quel genere, né
	risate, né tempo da dedicare ai ricordi. Soltanto lavoro. Soltanto
	tattica e strategia, ed eccitazione durante le battaglie, ma niente al
	di là di questo. E una sera, al termine degli allenamenti, la cosa lo
	colpì più di quel che aveva creduto. Stava discutendo con Alai certi
	particolari della manovra negli spazi aperti, quando Shen si avvicinò ad
	ascoltare. Per qualche minuto il ragazzo non disse nulla, poi una frase
	lo fece ridacchiare; d'improvviso afferrò Alai per le spalle e gridò: --
	Quattro-Tre-Nova! -- Anche Alai scoppiò a ridere, e per un poco Ender li
	ascoltò rammentarsi l'un l'altro la battaglia dove quella manovra era
	stata fin troppo reale, quando avevano aggirato i ragazzi più anziani e
	poi\ldots{}}

{D'un tratto i due ricordarono che lì c'era anche lui. -- Scusa, Ender
	-- disse Shen.}

{Scusa. Per che cosa? Per essere amici? -- Quel giorno c'ero anch'io, lo
	sai -- disse Ender.}

{E i due gli chiesero ancora scusa. Di nuovo al lavoro. Di nuovo al
	\emph{rispetto.} Così Ender capì che ai suoi compagni non era venuto in
	mente di includerlo nelle loro risate, nella loro amicizia.}

\emph{{E come avrebbero potuto pensare che io ne ero parte? Ho forse
		riso? Ho rivangato episodi? Me ne sono rimasto lì a guardare, come un
		insegnante della Scuola. È già a questo modo che mi vedono. Insegnante.
		Soldato leggendario. Non come uno di loro. Non come uno che hai
		abbracciato per sussurrargli «salaam» all'orecchio. Questo è durato
		finché Ender sembrava ancora una vittima, ancora un bambino
		vulnerabile.}}

{Adesso capeggiava una classifica, era un esperto. Ed era completamente,
	inevitabilmente solo.}

\emph{{Compiangi pure te stesso, Ender. }}{Quella sera, disteso sulla
	cuccetta, lasciò che le sue dita scrivessero sul banco: POVERO ENDER.
	Poi rise di quelle parole e le cancellò. \emph{Non c'è un ragazzo o una
		ragazza qui a scuola che non vorrebbero essere al mio posto.}}

{Chiamò sullo schermo la partita mentale. Come aveva fatto altre volte
	s'incamminò attraverso il villaggio che gli gnomi avevano edificato
	entro il collinoso scheletro del Gigante. Era facile costruire strani
	muri distorti seguendo la curvatura delle costole, aprendo finestre nei
	varchi fra esse. Il Torace era stato suddiviso in piccole abitazioni
	fissate a quelle travature ossee. L'anfiteatro per le riunioni era
	scavato a gradini nella coppa delle ossa iliache, e fra le gambe del
	Gigante c'erano cortili ed orti. Ender non aveva mai saputo a cosa
	mirassero gli gnomi con le loro attività, ma nel vederlo passare lungo
	il villaggio non lo avevano mai aggredito e in cambio lui li lasciava in
	pace.}

{Scavalcò l'osso pubico all'estremità dell'anfiteatro e si avviò fra gli
	orti. C'erano dei piccoli pony al pascolo, e nel vederlo scapparono. Lui
	non li inseguì. Non capiva più quale fosse il funzionamento della
	partita. Ai vecchi tempi, quando per primo aveva raggiunto la Fine del
	Mondo, tutto era combattimenti o enigmi da risolvere: sconfiggi
	l'avversario prima che lui uccida te, o escogita uno stratagemma per
	superare l'ostacolo. Adesso invece nessuno lo attaccava, non c'era da
	battersi, e dovunque andasse non si trovava davanti nessun ostacolo.}

{Salvo che, naturalmente, nella stanza del castello oltre la Fine del
	Mondo. Quello era rimasto l'unico luogo pericoloso. E Ender, benché
	avesse più volte giurato di non farlo più, continuava a ritornare là,
	continuava ad uccidere il serpente, e a guardare in faccia suo fratello.
	E ogni volta, qualunque azione intraprendesse, era morto lì dentro.}

{Neppure quella sera la cosa fu troppo diversa. Cercò di usare il
	coltello che c'era sul tavolo per scavar via la calcina ed estrarre una
	delle pietre del muro. Appena vi fu riuscito dal varco schizzò fuori un
	getto d'acqua, e a Ender non rimase che guardare lo schermo mentre la
	sua figura, ormai fuori controllo, si agitava follemente per restare in
	vita. La finestra della stanza era scomparsa; l'acqua salì e la sua
	figura annegò. Per tutto il tempo la faccia di Peter Wiggin rimase
	visibile nello specchio, con gli occhi fissi su di lui.}

\emph{{Sono intrappolato qui}}{, \emph{} pensò Ender. \emph{In trappola
		alla Fine del Mondo senza una sola via d'uscita.}}

{E seppe, infine, cos'era il triste senso d'inutilità che provava
	malgrado tutti i suoi successi lì alla Scuola di Guerra. Era
	disperazione.}

{~}

\begin{center}
	{* * *}
\end{center}

{~}

{C'erano uomini in uniforme all'ingresso della scuola, quando Valentine
	arrivò. Non avevano l'aria d'essere di guardia, anzi si sarebbero detti
	in attesa di qualcuno entrato un momento negli uffici. Portavano
	l'uniforme dei Marines della F.I. le stesse che tutti avevano sempre
	visto nei sanguinosi filmati di guerra o nei film della TV, e questo
	stava conferendo all'edificio scolastico un'aura inaspettatamente
	romantica e avventurosa. Tutti gli studenti erano piuttosto eccitati.}

{Valentine non lo fu per niente. Dapprima quella novità la fece pensare
	a Ender; poi ebbe paura. Qualcuno aveva appena pubblicato un saggio
	molto critico sull'insieme degli articoli di Demostene. Il saggio, e di
	conseguenza il lavoro di lei, erano stati discussi in un dibattito
	televisivo aperto a interventi internazionali, e alcuni dei più
	importanti personaggi della stampa e della politica avevano chi
	attaccato e chi difeso Demostene. Ciò che l'aveva più preoccupata era
	stato il commento di un inglese: -- Che provochi ostilità o consensi,
	Demostene non potrà godersi l'incognito per sempre. Ha oltraggiato
	troppi uomini illustri e sedotto troppi sciocchi perché glielo si
	permetta. Ma sia che si tolga la maschera da solo per assumere la guida
	dell'esercito di imbecilli che lo approvano, sia che lo smascherino i
	suoi avversari, non si può negare che sappia destare effetti di massa
	ben appropriati al suo pseudonimo.}

{Come c'era da aspettarsi, Peter ne era rimasto deliziato. Ma Valentine,
	rendendosi conto di quante persone potenti detestavano Demostene, aveva
	paura che cominciassero a indagare. La F.I. poteva farlo ufficialmente,
	ed era risaputo che sebbene fosse proibito i servizi segreti sapevano
	mettere le mani su qualunque dato. E adesso c'erano militari della F.I.
	tutto intorno alla Western Guilford Middle School, dentro e fuori. E non
	erano certamente lì per fare propaganda, perché il servizio di
	reclutamento dei Marines non ne aveva bisogno.}

{Così non fu sorpresa nel trovare il suo banco acceso e un messaggio che
	la attendeva in un angolo dello schermo.}

{~}

\begin{center}
	{PER FAVORE SI RECHI IMMEDIATAMENTE}

{NELL'UFFICIO DEL DR. LINEBERRY}
\end{center}

{~}

{Valentine attese nervosamente nell'anticamera del Preside, finché la
	porta dell'ufficio non si aprì e il Dr. Lineberry la invitò ad entrare.
	I suoi ultimi dubbi svanirono quando vide l'uomo alto e robusto, in
	uniforme da colonnello della F.I., che sedeva in una delle comode
	poltrone della stanza.}

{-- Lei è Valentine Wiggin -- disse l'uomo, alzandosi.}

{-- Sì -- mormorò lei, restituendogli debolmente la stretta di mano.}

{-- Io sono il colonnello Graff. Ci siamo già incontrati.}

{Già incontrati? Quando mai lei aveva avuto a che fare con la F.I.?}

{-- Venni a parlare ai suoi genitori, privatamente, per suo fratello.}

\emph{{Oh, allora non è per me, pensò lei. Loro hanno Peter\ldots{} ma
		cos'è successo? Che abbia fatto qualcosa di male? Credevo che avesse
		smesso di comportarsi bizzarramente. O forse\ldots{}}}

{-- Valentine\ldots{} posso chiamarla per nome, vero? Valentine, lei
	sembra spaventata. Non c'è alcun motivo di esserlo. Per favore, si
	sieda. Le assicuro che suo fratello sta bene. Ed è stato più che
	all'altezza delle nostre aspettative.}

{Soltanto allora, mentre la sua angoscia cominciava a sciogliersi, lesse
	negli occhi di Graff che era venuto lì per Ender. Ender. Non sarebbe
	stata interrogata e punita. La cosa riguardava Ender, il suo fratellino,
	che se n'era andato via ormai da tanto tempo, che non aveva più parte
	nei pensieri e nelle manovre di Peter. \emph{Sei stato tu il fortunato,
		Ender. Te ne sei andato prima che Peter potesse invischiarti nei suoi
		progetti.}}

{-- Cosa prova lei per suo fratello, Valentine?}

{-- Per Ender?}

{-- Naturalmente.}

{-- Lei cosa pensa che provi? Non l'ho più visto né sentito da quando
	avevo otto anni.}

{-- Dottor Lineberry, prego, vuole scusarci?}

{Seccato, Lineberry si avviò alla porta.}

{-- Un momento, dottore. Ripensandoci, credo che la signorina Wiggin e
	io avremo una conversazione più produttiva se facciamo due passi. Fuori.
	Lontano dai dispositivi d'ascolto che il suo segretario si è affannato a
	piazzare in questa stanza.}

{Era la prima volta che Valentine vedeva il Preside Lineberry restare
	senza parole. Il colonnello Graff andò a staccare un quadro dal muro e
	strappò via una membrana fonosensibile con la relativa microspia.}

{-- Economica ma efficiente -- annuì Graff. -- Inoltre ci sono i
	collegamenti col vostro computer, vero?}

{Lineberry girò dietro la sua scrivania, spense un interruttore
	mimetizzato e si lasciò cadere pesantemente in poltrona. Graff condusse
	fuori Valentine.}

{All'esterno si avviarono lungo il campo da football. I marines li
	seguirono discretamente a distanza, allargandosi intorno allo spazio
	erboso per tener d'occhio una zona il più ampia possibile.}

{-- Valentine, abbiamo bisogno del suo aiuto. Per Ender.}

{-- Che genere di aiuto?}

{-- Non siamo sicuri neppure di questo. Vorremmo anzi che lei ci
	aiutasse a capire come potrebbe aiutarci.}

{-- Be', cosa c'è che non va?}

{-- Questo è un altro lato dello stesso problema. Non lo sappiamo.}

{Valentine non poté impedirsi di scoppiare a ridere. -- Io non l'ho
	visto una volta in tre anni! E voi l'avete tenuto sotto controllo per
	ogni secondo in tutto questo tempo!}

{-- Valentine, farmi viaggiare avanti e indietro fra qui e la Scuola di
	Guerra costa al Governo più di quel che suo padre guadagna in una vita
	di lavoro. E io non viaggio per diporto.}

{-- Il Re aveva fatto un sogno -- disse Valentine, -- ma se n'era
	dimenticato il contenuto, così disse ai suoi saggi che dovevano
	interpretare quel sogno, pena la morte. Soltanto Daniele vi riuscì,
	perché era un profeta.}

{-- Lei legge la Bibbia?}

{-- Non quest'anno. Stiamo studiando i classici della letteratura
	medievale. Comunque, io non sono un profeta.}

{-- Vorrei poterle dire tutto sulla situazione di Ender, ma ci
	vorrebbero ore, forse giorni, e alla fine dovrei metterla in isolamento
	protettivo perché molto di questo è classificato strettamente
	confidenziale. Perciò vediamo cosa si può fare con le informazioni che
	posso darle, eh? Dunque, c'è una partita che i nostri studenti giocano
	con il computer della Scuola\ldots{} -- E proseguì, parlandole poi della
	Fine del Mondo, e della stanza chiusa, e della foto di Peter nello
	specchio.}

{-- È stato il computer a mettere lì la foto, non Ender. Perché non lo
	domandate al computer?}

{-- Il computer non lo sa.}

{-- E si suppone che io lo sappia?}

{-- Da quando Ender è con noi, questa è la seconda volta che la sua
	partita arriva a un punto morto. A una sfida che sembra senza sbocco.}

{-- La prima l'ha risolta?}

{-- Certo.}

{-- Allora dategli tempo, e probabilmente risolverà anche questa.}

{-- Non ne sono sicuro. Valentine, suo fratello è un ragazzo infelice.}

{-- Perché?}

{-- Non lo so.}

{-- Lei non sa molte cose, le pare?}

{Per un momento Valentine pensò che l'uomo stesse per bestemmiare.
	Invece Graff decise di riderci sopra. -- No, non molte. Valentine,
	perché suo fratello dovrebbe vedere Peter nello specchio?}

{-- Non dovrebbe. È una cosa stupida.}

{-- Stupida perché?}

{-- Perché se qualcuno è l'opposto di Ender, questi è Peter.}

{-- In che senso?}

{Valentine non riuscì a pensare una risposta che non contenesse elementi
	pericolosi. Spiegare troppo su Peter avrebbe potuto portare a
	conseguenze spiacevoli. Conosceva abbastanza la gente per sapere che
	nessuno avrebbe preso sul serio le sue ambizioni di dominio, e i suoi
	piani. Ma accennare alla sua personalità avrebbe potuto convincere
	quell'ufficiale a raccomandarlo per un trattamento psichiatrico.}

{-- Lei si sta preparando a dirmi una bugia -- osservò Graff.}

{-- Io mi sto preparando a dirle che non posso dirle niente.}

{-- E ha paura. Cos'è che la preoccupa?}

{-- Non mi piace parlare dei miei familiari. Lasciamo la mia famiglia
	fuori da questa faccenda.}

{-- Valentine, io voglio evitare di coinvolgere la sua famiglia. Sono
	venuto da lei per non dover sottoporre Peter a una batteria di test, e
	non seccare i vostri genitori con un interrogatorio. Sto cercando di
	risolvere il problema adesso con la persona che Ender ama di più, forse
	l'unica persona al mondo di cui si fida ciecamente. Se non riusciamo a
	farcela in questo modo temo che sequestreremo tutta la famiglia e i
	nostri psichiatri vi rivolteranno dentro e fuori. Questa non è una
	questione secondaria per noi, e non me ne andrò senza averla risolta.}

{L'unica persona che Ender amava e di cui si fidava. Valentine provò una
	cocente fitta di dolore, di rimorso, di vergogna al pensiero d'essere
	invece così vicina a Peter. Peter, che era diventato il centro della sua
	vita. \emph{Per te, Ender, accendo un focherello una volta all'anno. Per
		Peter e per i suoi sogni lavoro invece dalla mattina alla sera.} -- Non
	ho mai pensato che lei tenesse alla simpatia altrui. Non lo pensai
	quando venne a portar via Ender, e non m'illudo che ora sia cambiato.}

{-- Non finga d'essere una fanciulletta ignorante. Io ho visto i
	risultati dei test fatti quando era bambina, e oggi come oggi non ci
	sono molti professori universitari che potrebbero starle alla pari.}

{-- Ender e Peter si odiano l'un l'altro.}

{-- Questo lo sapevo. Lei li ha definiti opposti. Perché?}

{-- Peter\ldots{} può essere tutto odio, a volte.}

{-- È pericoloso, vuol dire?}

{-- Meschino, voglio dire. Odiare significa compiere atti meschini.}

{-- Valentine, per il bene di Ender, mi dica cosa può fare quando è in
	questo stato d'animo.}

{-- Minaccia di uccidere questo o quello. Non che lo faccia, beninteso.
	Ma quando eravamo piccoli Ender e io avevamo paura di lui. Progettava
	espedienti per ucciderci. In realtà ce l'aveva soprattutto con Ender.}

{-- Il monitor ci ha già informati di questo.}

{-- Parte della responsabilità l'aveva il vostro monitor.}

{-- Tutto qui? Mi dica qualcosa di più su Peter.}

{Valentine dovette dirgli dei compagni di classe in ogni scuola che
	Peter aveva frequentato. Non li colpiva mai fisicamente, ma sapeva
	ferirli in modo peggiore. Scopriva la cosa di cui si vergognavano di più
	e la faceva sapere alla persona di cui desideravano maggiormente il
	rispetto. Scopriva la cosa di cui avevano più paura, e faceva in modo
	che se la trovassero davanti di continuo.}

{-- Si comportava a questo modo anche con Ender?}

{Valentine scosse il capo.}

{-- Ne è sicura? Ender non aveva un punto debole? Una paura segreta, o
	qualcosa di cui si vergognava?}

{-- Ender non ha mai fatto nulla di cui dovesse vergognarsi. -- E d'un
	tratto, sprofondando nella vergogna per aver dimenticato e tradito
	Ender, Valentine scoppiò in lacrime.}

{-- Che c'è, adesso?}

{Lei scosse il capo. Non avrebbe mai potuto spiegare cosa provava nel
	pensare al suo fratellino, che era così buono, che lei aveva protetto
	fin dalla nascita, né dire cosa significava essere ora l'alleata di
	Peter, la sua aiutante, la sua serva in uno schema di eventi su cui lei
	non aveva il minimo controllo. Ender non s'era mai arreso a Peter, ma
	lei l'aveva fatto, fino al punto di divenire parte di lui. -- Ender non
	ha mai ceduto -- disse.}

{-- A cosa?}

{-- A Peter. Alla tentazione di essere come lui.}

{In silenzio girarono lungo la linea di fondocampo.}

{-- Come avrebbe potuto Ender essere come Peter?}

{Valentine ebbe un fremito. -- Gliel'ho già detto.}

{-- Ma Ender non ha mai fatto quel genere di cose. Era soltanto un
	bambino.}

{-- Ma sia lui che io avremmo voluto farle. Entrambi
	desideravamo\ldots{} uccidere Peter.}

{-- Ah!}

{-- No, non è così. Non ne parlammo mai. Ender non ha mai detto che
	sarebbe stato capace di farlo. Solo che io\ldots{} l'ho pensato. Io, non
	Ender. Lui non ha mai detto che gli sarebbe piaciuto vederlo morto.}

{-- Cosa desiderava, allora?}

{-- Niente. Ma non voleva essere\ldots{}}

{-- Essere cosa?}

{-- Peter tortura gli scoiattoli. Li inchioda a terra e li spella vivi,
	poi resta seduto a guardarli finché muoiono. È una cosa che adesso non
	fa più, però in passato lo faceva. Se Ender lo avesse saputo, se lo
	avesse visto, credo che avrebbe potuto\ldots{}}

{-- Che cosa? Salvare gli scoiattoli? Cercare di curarli?}

{-- No, a quel tempo non osavamo\ldots{} disfare ciò che Peter aveva
	fatto, o attraversargli la strada in quelle cose. Ma Ender amava gli
	scoiattoli che c'erano nei parchi della città sotterranea. Era uno dei
	pochi che riuscivano a farli avvicinare per nutrirli. Ma a quel modo
	diventavano docili, e \ldots per Peter era più facile catturarli. --
	Valentine riprese a piangere. -- Capisce? qualunque cosa uno faccia,
	questo aiuta Peter. Tutto gli serve, tutto lo aiuta, non importa cosa
	uno possa escogitare.}

{-- Lei sta aiutando Peter? -- domandò Graff.}

{Lei non rispose.}

{-- Suo fratello maggiore è davvero così malvagio, Valentine?}

{Lei accennò di sì.}

{-- Crede che Peter sia il peggior individuo del mondo?}

{-- Potrebbe esserlo? Non lo so. È solo il peggiore che io conosca.}

{-- Tuttavia lei e Ender siete suoi fratelli. Avete avuto la stessa
	eredità genetica, la stessa educazione, dunque come può Peter essere un
	tale\ldots{}}

{Valentine si volse di scatto e gridò, come se l'uomo la stesse
	torturando a morte: -- Ender non è come Peter! Non ha niente in comune
	con lui! Salvo che è intelligente, e che è suo fratello. Ma non per
	questo lei deve osare\ldots{} no! Lui non ha niente, niente, niente di
	Peter! Ha capito? Niente!}

{-- Vedo -- disse Graff.}

{-- So cosa sta pensando\ldots{} lei, bastardo! Lei pensa che io
	vaneggi, e che Ender sia uguale a Peter. Be', forse \emph{io} ho
	qualcosa di Peter, ma Ender no. Neppure lontanamente. E quando era
	piccolo e lo vedevo piangere glielo dicevo e glielo ripetevo, decine di
	volte: tu non sei come Peter, tu non hai mai fatto male agli altri, tu
	sei gentile e buono e diverso da lui in tutto e per tutto!}

{-- E questo è vero.}

{L'acquiescenza di lui la calmò. -- È maledettamente vero, infatti. Ci
	può scommettere che è vero.}

{-- Valentine, lei aiuterà Ender?}

{-- Non c'è nulla che io possa fare per lui, adesso.}

{-- Una cosa c'è, esattamente la stessa che lei faceva in passato.
	Niente di più che confortarlo e dirgli che far del male alla gente non
	gli piace, che è buono e gentile, e che in lui non c'è nulla di Peter.
	Questa è la cosa più importante: che non ha qualcosa di Peter dentro di
	sé.}

{-- Posso vederlo?}

{-- No. Voglio che lei gli scriva una lettera.}

{-- E questo servirebbe? Ender non ha mai risposto a una sola delle
	lettere che gli ho spedito.}

{Graff si schiarì la voce. -- Ha risposto a\ldots. uh, ogni lettera che
	ha ricevuto.}

{Valentine trasalì a quell'ammissione. -- Vuol dire che voi\ldots{}
	figli di puttana!}

{-- L'isolamento è, per certe cose, l'ambiente in cui meglio si sviluppa
	la creatività. E noi volevamo le sue idee, non il\ldots{} ma lasciamo
	perdere. Non sono tenuto a giustificarmi con lei.}

\emph{{E cos'altro sta facendo? }}{avrebbe voluto borbottare Val.}

{-- Comunque, si è arenato. O ha mollato. Noi vorremmo spingerlo avanti,
	ma se lui non vuole è inutile.}

{-- Forse farei a Ender un favore migliore se la mandassi a farsi
	friggere.}

{-- Lei mi ha già dato un aiuto. Può fare di più. Gli scriva.}

{-- Prometta che non taglierà una sola parola.}

{-- Non sono autorizzato a promettere niente a nessuno.}

{-- Allora se ne dimentichi.}

{-- Nessun problema. Scriverò io la sua lettera. Possiamo far uso delle
	lettere precedenti per lo stile e i particolari. Semplicissimo.}

{-- Voglio vederlo.}

{-- Avrà la sua prima libera uscita a diciott'anni.}

{-- Lei disse che l'avrebbe avuta a dodici.}

{-- Abbiamo cambiato il regolamento.}

{-- Perché dovrei aiutarvi?}

{-- Non noi, ma Ender. Che le importa se nel farlo aiuterà anche noi?}

{-- Che razza di cose terribili e odiose gli state facendo, lassù?}

{Graff ebbe una risatina. -- Mia cara signorina Wiggin, le cose
	terribili sono ancora tutte da venire.}

{~}

\begin{center}
	{* * *}
\end{center}

{~}

{Ender era già alla quarta riga quando s'accorse che quella era una
	lettera, e non un messaggio mandatogli da un compagno della Scuola di
	Guerra. Gli era arrivata nel solito modo, una nota che lo aveva
	informato: POSTA IN GIACENZA appena aveva acceso il banco. Con un
	sussulto, il suo sguardo corse alla firma. Poi tornò alla prima riga e
	semidisteso sulla cuccetta lesse e rilesse più volte ogni parola.}

{~}

{~}

\begin{flushleft}
	{\footnotesize {ENDER,

{I BASTARDI NON TI HANNO MAI CONSEGNATO UNA DELLE}

{LETTERE CHE TI HO SPEDITO FIN'ORA. TI AVRÒ'}

{SCRITTO CENTO VOLTE, MA TU DEVI AVER CREDUTO CHE}

{NON LO ABBIA MAI FATTO. IO NON TI HO}

{DIMENTICATO. RICORDO OGNI TUO}

{COMPLEANNO. RICORDO OGNI COSA.}

{QUALCUNO POTREBBE PENSARE CHE}

{POICHÉ' STAI FACENDO IL SOLDATO}

{ADESSO TU SIA DIVENTATO CRUDELE E SPIETATO,}

{UNO A CUI PIACE FAR DEL MALE E COLPIRE,}

{COME I MARINES DEI FILM,}

{MA IO SO CHE QUESTO NON È VERO. TU}

{NON SEI AFFATTO COME CHI-SAI-TU.}

{LUI SEMBRA PIÙ BUONO MA INVECE}

{DENTRO DI SÉ È SEMPRE UNA CAROGNA.}

{FORSE TI SEI FATTO PIÙ DURO, MA QUESTO}

{NON PUÒ INGANNARE ME. SEMPRE PAGAIANDO}

{SULLA VECCHIA CANNA,}

{TUTTO IL MIO AMORE E UN GROSSO BACIO}

{~}

{VAL}

{~}

{NON MI SCRIVERE. PROBABILMENTE LORO}

{FAREBBERO LA SCHIFANALISI ALLA TUA LETTERA.}

{~}

}


}
\end{flushleft}

{Ovviamente era stata scritta con la piena approvazione degli
	insegnanti. Ma non c'era dubbio che la mittente fosse Val. La
	deformazione della parola \emph{psicanalisi}, \emph{} l'epiteto
	\emph{carogna} per Peter, il vecchio scherzo di pronunciare \emph{canna}
	invece di \emph{canoa}, \emph{} erano tutte cosette che nessuno poteva
	sapere salvo Val.}

{E tuttavia quegli espedienti apparivano forzati, come se qualcuno
	avesse voluto studiarli per far sì che la lettere avesse un tocco di
	autenticità in più. Perché avrebbero dovuto esserne tanto preoccupati,
	se la lettera era vera?}

\emph{{Perché non è vera comunque. Anche se lei l'avesse scritta col suo
		sangue non sarebbe una cosa vera, dato che gliel'hanno fatta scrivere
		loro. Mi ha mandato tante lettere, e le hanno intercettate tutte. Quelle
		avrebbero potuto essere vere, lo erano, ma questa le è stata ordinata.
		Questa fa parte delle loro manipolazioni.}}

{E quell'oscura oppressione lo sommerse di nuovo. Ora ne conosceva il
	motivo. Ora sapeva quali cose odiava. Non aveva alcun controllo sulla
	sua stessa vita. Loro programmavano tutto. Facevano tutte le scelte.
	Soltanto la partita libera era lasciata a lui, nulla di più; ogni altra
	cosa apparteneva a loro, dai regolamenti ai giochi, dalle lezioni ai
	programmi a lunga scadenza, e preso in quell'ingranaggio lui non poteva
	che continuare o cedere. L'unica cosa reale, l'unica preziosa realtà che
	gli restava era il ricordo di Valentine, la persona che lo amava da
	prima che si mostrasse abile in quei giochi bellici, che lo avrebbe
	amato anche se non ci fosse stata da vincere nessuna guerra contro gli
	Scorpioni. Ed essi avevano allungato le mani anche su di lei, l'avevano
	portata al loro fianco. Era una di loro, adesso.}

{Odiava quella gente e i loro giochi. Li odiava al punto che non seppe
	frenare le lacrime, con gli occhi fissi sulla lettera fatta su
	ordinazione. E i ragazzi dell'orda delle Fenici che se ne accorsero
	distolsero lo sguardo. \emph{Ender Wiggin} che piangeva? Questo era
	preoccupante. Stava accadendo qualcosa di terribile. Il miglior soldato
	di tutte le orde disteso in lacrime sulla sua cuccetta. Nella camerata
	scese un silenzio profondo.}

{Ender cancellò la lettera, la spazzò via dalla sua memoria e richiamò
	sullo schermo la partita libera. Non sapeva bene cosa lo rendesse tanto
	ansioso di riprendere il gioco, di tornare alla Fine del Mondo, ma agì
	in modo da arrivarci senza sprecare tempo. Soltanto quando spinse lo
	sguardo sui colori autunnali di quel fiabesco mondo pastorale, soltanto
	allora capì cos'aveva detestato di più nella lettera di Val. Tutto ciò
	che diceva era in relazione con Peter, puntualizzava il fatto che lui
	non era come Peter: parole che Valentine aveva detto così spesso quando
	lo abbracciava per confortarlo mentre lui tremava di rabbia o di paura o
	di disgusto per i tormenti che il fratello gli aveva inflitto. Questo
	era più o meno tutto il contenuto della lettera.}

{E questo era ciò che loro avevano ordinato. I bastardi ne erano
	informati, e sapevano di Peter nello specchio della stanzetta di pietra,
	sapevano tutto, e per loro Val era soltanto uno strumento da usare per
	controllare lui, un altro trucco da mettere in atto. Dink aveva ragione:
	il nemico erano loro, e non amavano nessuno, e nulla gli importava, e
	perciò lui non avrebbe fatto quel che volevano, e di questo avrebbero
	potuto stare maledettamente certi. Lui aveva avuto un solo ricordo degno
	d'essere ricordato, una sola cosa buona, e quei bastardi l'avevano preso
	e mescolato al resto del loro concime\ldots{} e così lui era finito, e
	avrebbe messo fine al gioco.}

{Come sempre nella stanza in cima alla torre c'era ad attenderlo il
	lungo serpente, e al suo arrivo srotolò le spire davanti al caminetto.
	Ma stavolta Ender non lo schiacciò sotto i piedi. Stavolta allungò le
	mani a prenderlo, gli si inginocchiò davanti, e dolcemente, molto
	dolcemente attirò la bocca scagliosa del rettile alle sue labbra.}

{E lo baciò.}

{Non aveva avuto intenzione di farlo. Voleva lasciare che il serpente lo
	mordesse sulla bocca. O forse aveva inconsciamente desiderato mangiarlo
	vivo, come il Peter dello specchio doveva aver fatto col rettile la cui
	coda sanguinante gli emergeva pendula dalle labbra. Invece lo aveva
	baciato.}

{E fra le sue mani il corpo del serpente s'ingrossò, assumendo un'altra
	forma. Le sue sembianze si fecero umane, femminili. Era Valentine, e la
	sorella gli restituì il bacio.}

{Il serpente non poteva essere Valentine. Lo aveva ucciso troppe volte
	perché ora si rivelasse per sua sorella. Era insopportabile!}

{Era questo che volevano ottenere quando gli avevano fatto leggere la
	lettere di Valentine? Non che gliene importasse molto.}

{Lei si alzò dal pavimento della stanza della torre e si mosse verso lo
	specchio. Ender fece alzare anche la sua figura e la affiancò. Si
	fermarono davanti allo specchio, dove al posto dell'orrido riflesso di
	Peter c'erano ora un drago e un unicorno. Ender tese una mano e toccò il
	cristallo: la parete cadde in polvere, rivelando la presenza di una
	grande scalinata che curvava verso il basso, fitta di personaggi che
	gridavano e acclamavano invitandoli festosamente a scendere. Tenendosi
	sotto braccio lui e Valentine s'avviarono giù per le scale. Ender aveva
	gli occhi pieni di lacrime per il sollievo d'aver infine trovato
	l'uscita da quella torre di pietra alla Fine del Mondo. E a causa delle
	lacrime non notò che ogni persona di quella folla eterogenea aveva la
	faccia di Peter. Riusciva soltanto a pensare che dovunque fosse andato
	in quel mondo Valentine sarebbe stata con lui.}

{~}

\begin{center}
	{* * *}
\end{center}

{~}

{Valentine lesse la lettera che il Preside Lineberry le aveva appena
	consegnato. «Gentile signorina Wiggin» diceva. «Le siamo grati per gli
	sforzi da lei fatti in favore dello sforzo bellico. Abbiamo il piacere
	di notificarle che le è stata conferita, a nome degli Alleati e
	dell'intera umanità, la Stella del Valor Civile di Prima Classe, ovvero
	la più alta decorazione militare di cui possa fregiarsi un civile.
	Sfortunatamente il Servizio di Sicurezza della F.I. ci proibisce di
	render pubblica la decorazione fino alla vittoriosa conclusione delle
	operazioni in corso, ma privatamente mi pregio farle sapere che il suo
	atto si è risolto in un completo successo. Distinti saluti, generale
	Shimon Levy, Stratega».}

{Dopo che l'ebbe letta due volte, il Dr. Lineberry gliela sfilò dalle
	dita. -- Mi è stato ordinato di fartela leggere, e poi di distruggerla.
	-- Prese un accendisigaro da un cassetto e diede fuoco alla lettera,
	lasciandola incenerire in un portacenere. -- Erano buone o cattive
	notizie? -- domandò poi.}

{-- Ho venduto mio fratello -- disse Valentine, -- e loro mi hanno
	pagato i trenta denari.}

{-- Questo mi sembra un po' melodrammatico, Valentine, no?}

{Valentine non rispose e tornò in classe. Quella sera Demostene scrisse
	una graffiante denuncia delle leggi per la limitazione delle nascite. La
	gente aveva il sacro diritto di mettere al mondo quanti figli voleva, e
	la popolazione in eccesso avrebbe potuto esser inviata a colonizzare
	altri pianeti, per spargere la razza umana così lontano nella galassia
	che nessun disastro, nessuna invasione, avrebbe potuto farle rischiare
	l'estinzione. «Il titolo più nobile che un bambino possa avere --
	scrisse Demostene, -- è Terzo!»}

\emph{{Per te, Ender}}{, \emph{} disse a se stessa mentre spediva
	l'articolo.}

{Peter rise divertito quando lo lesse. -- Questo farà raddrizzare
	orgogliosamente le spalle a tanti poveri figli di mamma. Terzo! Un
	nobile titolo! Oh, in che sottile sarcasmo sai intingere la penna.}

\phantomsection\label{Orsonux20Scottux20Cardux20-ux20Ilux20Giocoux20Diux20Enderux20-ux20BY_SLY70A1_split_012.htm}{}
