\chapter{GIOCHI}

{~}

{~}

{~}

{-- \emph{Lei ha tutta la mia ammirazione. Un braccio rotto.., questo è
		stato un colpo da maestro.}}

{-- \emph{È stato un incidente.}}

{-- \emph{Sul serio? E io che le ho già fatto ampi elogi nel rapporto
		ufficiale!}}

{-- \emph{La cosa è andata oltre il limite. Ha trasformato in una specie
		di eroe quell'altro piccolo bastardo. E potrebbe aver guastato parecchi
		di loro ancor prima dell'addestramento. Credevo che avrebbe chiamato
		aiuto.}}

{-- \emph{Chiamare aiuto? Via, ero convinto che fosse questo a renderlo
		prezioso ai suoi occhi: il fatto che lui risolve da solo i suoi
		problemi. Quando sarà fuori, con attorno a sé una flotta nemica, che
		chiami aiuto o meno dovrà sfangarsela da solo.}}

{-- \emph{Chi avrebbe immaginato che quel piccolo imbecille si sarebbe
		sganciato le cinture? E come se non bastasse, è andato a sbattere nella
		paratia nel modo peggiore.}}

{-- \emph{È soltanto un ulteriore esempio della stupidità militare. Chi
		ha un grammo di cervello cerca di far carriera in un altro campo, magari
		nelle assicurazioni sulla vita.}}

{-- \emph{Se è così, c'è da stare poco allegri.}}

{-- \emph{Dobbiamo soltanto accettare il fatto che lei e io siamo dei
		subordinati, infine. Col destino dell'umanità nelle nostre mani. Questo
		dà un delizioso senso di potere, no? Specialmente al pensiero che se
		perdiamo stavolta non resterà più nessuno per criticarci.}}

{-- \emph{Non ho mai visto la cosa in questo modo. Ma non dobbiamo
		perdere.}}

{-- \emph{Stiamo a vedere come Ender se la cava. Se cedesse, se
		dovessimo rinunciare a lui, chi resta? Chi altro c'è?}}

{-- \emph{Compilerò una lista di nomi.}}

{-- \emph{Nel frattempo cerchi il modo di tenere Ender sulla giusta
		strada.}}

{-- \emph{Gliel'ho detto. Bisogna isolarlo e tenerlo isolato. Non dovrà
		mai presumere che qualcuno può venire in suo aiuto, mai. Se pensasse per
		una sola volta che questa è la via d'uscita più facile, sarebbe
		rovinato.}}

{-- \emph{Lei ha ragione. Sarebbe terribile se sospettasse per un solo
		istante che ha un amico.}}

{-- \emph{Gli amici non gli mancheranno. Ciò che non avrà più sono i
		genitori.}}

{~}

\begin{center}
	{* * *}
\end{center}

{~}

{Gli altri ragazzini avevano già scelto le loro cuccette quando Ender li
	raggiunse. Si fermò sulla soglia della camerata, e i suoi occhi
	cercarono l'unica branda rimasta libera. Il soffitto era così basso che
	alzandosi in punta di piedi avrebbe potuto toccarlo. Era un dormitorio
	per bambini, e la cuccetta inferiore dei letti poggiava sul pavimento.
	Quasi tutti lo stavano osservando senza parere. Ender si disse che senza
	dubbio la cuccetta in basso, a destra della porta, era la sola lasciata
	vuota. Per un momento fu costretto a riflettere che permettendo loro di
	affibbiargli il posto peggiore avrebbe incoraggiato futuri soprusi.
	Tuttavia costringere qualcun altro a cedergli il suo era fuori
	discussione.}

{Così girò intorno un ampio sorriso. -- Ehi, grazie, ragazzi! --
	esclamò, senza alcun sarcasmo. Il suo tono suonò sincero come se gli
	avessero riservato il posto migliore. -- Credevo che avrei dovuto pagare
	per ottenere la cuccetta inferiore accanto alla porta.}

{Poggiò un ginocchio al suolo e guardò nell'interno dell'armadietto
	aperto fissato ai piedi del lettuccio. Allo sportello era incollato un
	cartoncino che diceva:}

{~}

{Poggia una mano sullo scanner}

{collegato alla tua cuccetta}

{e pronuncia il tuo nome due volte.}

{~}

{Ender trovò lo scanner, una piastra di plastica opaca. Vi applicò la
	mano sinistra e disse: -- Ender Wiggin. Ender Wiggin.}

{Per un secondo la piastra brillò di una luce verde. Ender chiuse lo
	sportello e provò a riaprirlo. Non ci riuscì. Allora mise la mano sullo
	scanner e disse: -- Ender Wiggin. -- La serratura si aprì con uno
	scatto. Lo stesso accadde quando collaudò gli altri tre armadietti
	personali.}

{Uno di essi conteneva quattro tute da fatica uguali a quella che
	indossava, ed una bianca. In un altro c'era un banco elettronico,
	estraibile, simile a quello che aveva avuto a scuola. Dunque non l'aveva
	finita con le fatiche dello studio.}

{Fu nell'armadietto alto e stretto che trovò l'oggetto più interessante.
	A un primo sguardo gli parve una vera tuta spaziale, completa di elmo a
	pressione e guanti. Ma non aveva i collegamenti per le bombole d'aria.
	Tuttavia era studiata per contenere e proteggere il corpo, con spesse
	imbottiture. Al tatto la sentì un po' rigida.}

{E insieme ad essa c'era una pistola. Un'arma laser a giudicare dalla
	forma della canna, che terminava con un solido cilindro vitreo. Ma senza
	dubbio non avrebbero permesso a dei ragazzini di maneggiare armi così
	mortali\ldots{}}

{-- Non è un laser -- disse una voce d'uomo. Ender si voltò. Era un
	giovanotto che non aveva mai visto prima, simpatico e di bell'aspetto.
	-- Però emette un raggio di luce polarizzata, molto ristretto.
	Proiettato su un muro a centro metri di distanza forma un disco luminoso
	largo appena venti centimetri.}

{-- A cosa serve? -- domandò Ender.}

{-- Nelle partite che combattiamo in sala di battaglia. Qualcun altro ha
	gli armadietti aperti? -- L'uomo si guardò attorno. -- Voglio dire,
	avete seguito le istruzioni e codificato la voce e l'impronta palmare?
	Non potrete usare le serrature finché non lo fate. Questa camerata sarà
	la vostra casa almeno per il primo anno, qui alla Scuola di Guerra,
	perciò scegliete una cuccetta e tenetevela. Di regola lasciamo che le
	reclute scelgano un capo-camerata, il quale prende la cuccetta inferiore
	accanto alla porta, ma sembra che questo posto sia già stato occupato.
	Ormai è impossibile ricodificare le serrature. Decidete quale compagno
	intendete scegliere. La cena è fra sette minuti. Seguite le tracce
	luminose sulla pavimentazione. Il vostro colore-codice è rosso giallo
	giallo\ldots{} vale a dire che quando avrete ordine di seguire un
	percorso assegnato questo sarà rosso giallo giallo, tre linee
	affiancate, e andrete nella direzione che esse indicano. Qual è il
	vostro colore-codice, ragazzi?}

{-- Rosso giallo giallo.}

{-- Molto bene. Il mio nome è Dap. Nei mesi che ci attendono io sarò la
	vostra mammina.}

{I ragazzi risero.}

{-- Ridete finché volete ma fate quello che vi dico. Se vi perderete
	nelle strutture della Scuola, cosa che a volte capita, non andate in
	giro ad aprire tutte le porte che vedete. Alcune si apriranno\ldots{} ma
	soltanto sul vuoto. -- Ci furono altre risate. -- Quando capita, dite al
	più vicino inserviente che la vostra mammina è Dap, e mi chiameranno. O
	dite il vostro colore, e loro accenderanno un percorso per rimandarvi a
	casa. Se avete dei problemi, venite a parlarne con me. Ricordate sempre
	che qui dentro io sono l'unica persona pagata apposta per essere
	simpatico con voi. Ma simpatico fino a un certo punto. Fatemi uno sgarbo
	e io vi romperò la faccia. D'accordo?}

{Di nuovo tutti risero. Dap aveva una camerata piena di amici. I
	ragazzini spaventati sono facili da conquistare.}

{-- Qualcuno di voi sa dirmi da che parte è il basso?}

{Le loro voci gli risposero in coro.}

{-- Certo, è proprio così. Ma questa direzione indica soltanto
	l'esterno. La stazione sta ruotando, e l'effetto fa sì che questo sia il
	basso. In realtà il pavimento su cui state è curvo. Se lo seguite
	finirete per ritrovarvi nello stesso posto da cui siete partiti. Ma non
	provateci, perché in quella direzione ci sono le stanze degli
	insegnanti, e in quella opposta le camerate dei ragazzi più grandi. E ai
	ragazzi più grandi non piace ritrovarsi fra i piedi voialtri pivelli.
	Potrebbero farvi qualche brutto scherzo. Anzi, i brutti scherzi \emph{vi
		saranno fatti.} E quando questo accadrà non venite a piangere da me.
	Capito? Questa è la Scuola di Guerra, non un asilo infantile.}

{-- Ma allora cosa dovremo fare? -- chiese un ragazzino, un soldo di
	cacio dalla pelle nera che occupava la cuccetta superiore accanto a
	quella di Ender.}

{-- Se c'è qualcuno a cui i soprusi non piacciono, pensi lui stesso al
	modo di difendersene. Ma vi avverto: l'assassinio è tassativamente
	proibito. E così anche le ferite inferte deliberatamente. Mi è stato
	detto che fra voi c'è già stato un tentativo di omicidio. Un braccio
	rotto. Se una cosa del genere capita di nuovo, qualcuno finirà
	congelato. Mi avete inteso?}

{-- Cosa significa congelato? -- domandò il ragazzino col braccio
	immobilizzato nella steccatura.}

{-- Congelato. Sbattuto fuori nello spazio. Rimandato sulla Terra.
	Comunque, con la Scuola di Guerra avrà chiuso.}

{Nessuno guardò dalla parte di Ender.}

{-- Così, pivelli, se qualcuno di voi sta pensando di andare in cerca di
	guai, qui dentro, farà meglio a darsi una regolata. Chiaro?}

{Dap uscì. Gli occhi dei ragazzi continuarono a evitare Ender.}

{Ma d'improvviso lui aveva sentito la mano gelida della paura
	attanagliarlo allo stomaco. Il ragazzo a cui aveva spezzato il
	braccio\ldots{} non provava alcun rimorso per averlo fatto. Era un altro
	Stilson. E come Stilson stava già radunando una piccola banda attorno a
	sé. Un pugno di ragazzini, quelli fra i più robusti. Stavano ridendo fra
	loro sul fondo della camerata, e ogni tanto uno si voltava a guardare
	Ender.}

{Lui sentì un desiderio struggente di tornare a casa. Cos'avevano a che
	fare loro col fatto di salvare il mondo? Non c'erano monitor, adesso.
	Era di nuovo lui, da solo, contro una banda di ragazzini, con la
	differenza che ora li aveva proprio nella sua stanza. Di nuovo Peter, ma
	senza Valentine.}

{La paura continuò a tormentarlo durante la cena, quando nessuno sedette
	accanto a lui nella sala mensa. Gli altri chiacchieravano di varie cose:
	il grande schermo a una delle pareti, il cibo, i ragazzi più grandi.
	Ender, nel suo isolamento, poté soltanto guardarsi attorno.}

{Sullo schermo apparivano i nomi delle squadre in gara. Le vittorie, le
	sconfitte e i punteggi raggiunti. Gli parve che alcuni dei ragazzi più
	grandi avessero fatto delle scommesse sulle ultime competizioni. Due
	squadre, le Mantidi e le Vipere, non avevano punteggi recenti ma i loro
	nomi lampeggiavano. Ender decise che stavano gareggiando proprio in quel
	momento.}

{Aveva già notato che i ragazzi più anziani erano suddivisi in gruppi, a
	seconda delle uniformi che indossavano. Alcuni con uniformi diverse
	parlavano fra loro, ma in generale ciascun gruppo disponeva di una
	propria zona. I pivelli (i suoi compagni, più due o tre gruppi appena di
	poco più anziani) portavano tute di un uniforme colore azzurro. Ma i
	ragazzi più grandi, quelli suddivisi in squadre, esibivano indumenti
	multicolori e sgargianti. Ender cercò di capire da essi quali fossero i
	loro nomi. Api e Ragni erano facili da indovinarsi. E così anche le
	Fiamme e le Onde.}

{Un ragazzo più alto venne a sedersi al suo fianco. Non era soltanto più
	alto: dimostrava dodici o tredici anni. La sua corporatura era già
	quella di un adulto.}

{-- Ehilà -- disse.}

{-- Ehilà -- rispose Ender.}

{-- Io mi chiamo Mick.}

{-- Io Ender.}

{-- Di cognome?}

{-- No. Fin da piccolo mia sorella mi chiamava così.}

{-- Non è un nome malvagio, qui dentro. Ender\ldots{} quello che
	finisce, eh?}

{-- Così spero.}

{-- E sei tu lo Scorpione del tuo gruppo, Ender?}

{Lui scrollò le spalle.}

{-- Mi sono accorto che ti fanno mangiare da solo. Ogni mandata di
	pivelli ne ha uno così. Uno che tutti gli altri scansano. A volte penso
	che gli insegnanti lo facciano apposta. Qui gli insegnanti non sono
	esattamente dei simpaticoni. Lo avrai notato.}

{-- Già.}

{-- Allora, sei tu lo Scorpione?}

{-- Così credo.}

{-- Ehi! Non è il caso di piangerci sopra, ti pare? -- Prese il budino
	di Ender, e gli diede in cambio la sua \emph{brioche.} -- Tutta roba
	nutriente, qui. Vogliono farci crescere robusti. -- Mick attaccò di
	gusto il budino.}

{-- Tu di che gruppo sei?}

{-- Io? Di nessuno. Sono uno stronzo nell'impianto dell'aria
	condizionata.}

{Ender cercò di sorridere volonterosamente.}

{-- Divertente, già, ma non è uno scherzo. Io non combino niente di
	buono qui dentro. E ora sono cresciuto. Molto presto mi spediranno alla
	mia prossima scuola. Solo che non sarà la Scuola Ufficiali, ci puoi
	scommettere. Non sono mai stato portato a comandare, capisci? E soltanto
	chi ha doti di comando ha una possibilità di arrivare là.}

{-- Come si fa per dimostrare doti di comando?}

{-- Eh! Se lo sapessi, ti pare che sarei ancora dove sono? Quanti
	ragazzi alti come me vedi, qui dentro?}

{Non molti, notò Ender, ma non lo disse.}

{-- Pochi, vero? Non sono io il solo già mezzo congelato, cibo da
	Scorpioni. Siamo in pochi. Gli altri bei tipi\ldots{} quelli sono tutti
	al comando di un'orda. Tutti quelli arrivati qui con me adesso hanno la
	loro orda. Io no.}

{Ender annuì.}

{-- Ascoltami, ragazzino. Voglio insegnarti una cosa. Fatti degli amici.
	Diventa un capo. Lecca pure il culo a qualcuno, se dovrai farlo, ma
	attento che se gli altri cominciano a disprezzarti\ldots{} capisci cosa
	voglio dire?}

{Ender annuì di nuovo.}

{-- Adesso tu non sai niente di niente. Voi pivelli siete uno uguale
	all'altro. Non sapete niente. Cervelli vuoti come lo spazio, con niente
	dentro. E qualunque cosa vi troviate davanti, ci sbattete la faccia.
	Perciò, se farai la mia stessa fine, ricorda che qualcuno ti aveva
	avvisato. Queste parole sono l'ultimo favore che ti viene fatto qui,
	amico.}

{-- Perché mi dici questo, allora? -- chiese Ender.}

{-- Cosa sei, un chiacchierone? Taci e mangia.}

{Ender tacque e mangiò. Mick non gli piaceva. E si disse che non c'era
	la minima possibilità che anche lui facesse quella fine. Forse era
	proprio ciò che gli insegnanti avevano progettato, ma lui non intendeva
	affatto adeguarsi ai loro piani.}

\emph{{Io non sarò lo Scorpione del mio gruppo}}{, \emph{} pensò.
	\emph{Non ho lasciato Valentine e mamma e papà solo per venire qui e
		finire congelato.}}

{Mentre si portava la forchetta alla bocca gli parve di sentire la
	presenza della sua famiglia intorno a lui, vicini come gli erano sempre
	stati. Sapeva che se avesse girato la testa in un certo modo e alzato lo
	sguardo avrebbe visto il viso di sua madre, occupata ad ammonire
	Valentine di non sbrodolarsi. Sapeva esattamente dove si trovava suo
	padre, intento a leggere il suo video-giornale preferito ed a fingere di
	prender parte alle solite chiacchiere dell'ora di colazione. Mentre
	Peter, con due stuzzicadenti ficcati nelle narici, imitava un
	tricheco\ldots{} anche Peter sapeva essere spiritoso.}

{Pensare a loro fu uno sbaglio. Sentì un singhiozzo che tentava di
	salirgli in gola e lo ricacciò giù. Non riusciva neppure a vedere il suo
	piatto.}

{Ma non poteva mettersi a piangere. Era da escludersi che qualcuno gli
	avrebbe mostrato un po' di compassione. Dap non era Mamma. Ogni segno di
	debolezza avrebbe informato gli Stilson e i Peter che quel ragazzino
	poteva essere schiacciato. Ender fece ciò che faceva ogni volta che
	Peter accennava a tormentarlo. Cominciò a contare raddoppiando: uno,
	due, quattro, otto, sedici, trentadue, sessantaquattro. E via di seguito
	finché riuscì a visualizzare i numeri nella mente: 128, 256, 512, 1024,
	2048, 4096, 8192, 16384, 32768, 65536, 131072, 262144. Quando giunse a
	67108864 cominciò a essere incerto: aveva sbagliato una cifra? Avrebbe
	dovuto essere sempre nelle decine di milioni oppure nelle centinaia di
	milioni, o forse ancora sotto i dieci milioni? Cercò di raddoppiare
	nuovamente e perse il conto. 1342 e qualcos'altro. 16? O 17738?
	\emph{Bestia che sei. Ricomincia daccapo. Raddoppia con più attenzione.}
	Il dolore se n'era andato. Le lacrime non cercavano più di uscire. Non
	avrebbe pianto.}

{Non fino a quella notte, dopo che le luci furono abbassate, quando di
	lontano udì i singhiozzi soffocati di qualche ragazzino che piangeva per
	sua madre, o suo padre, o per il cane che non avrebbe visto più. Le
	labbra di lui formarono il nome di Valentine. Gli parve di sentire la
	sua voce ridere da qualche parte, giù in soggiorno. Vide Mamma passare
	davanti alla sua porta, fermandosi a sbirciare dentro per esser certa
	che tutto andasse bene. Sentì la voce di Papà commentare qualcosa,
	davanti alla TV. Era tutto ancora così nitido in lui, eppure non sarebbe
	mai più accaduto. \emph{La prossima volta che li rivedrò sarò diventato
		vecchio, dodici anni a dir poco. Perché ho detto di sì? Cosa mi ha fatto
		fare questa sciocchezza?} Andare a scuola sarebbe stato niente in
	confronto. E anche dover affrontare Stilson. E Peter. Erano due
	cacasotto, Ender non aveva più paura di loro.}

\emph{{Voglio andare a casa}}{, \emph{} sussurrò nel buio.}

{Ma il suo sussurro era quello che gli usciva di bocca quando Peter lo
	costringeva a gemere di dolore. Era un sussurro che non andava più
	lontano dei suoi stessi orecchi, e talvolta non giungeva neppure a
	quelli.}

{E le indesiderate lacrime poterono scivolare sulle sue guance,
	accompagnate da singhiozzi così lievi che non destavano un fremito nelle
	molle del letto, così silenziosi che nessuno li avrebbe uditi. Ma il
	dolore era lì, chiuso nella sua gola e rigido nella smorfia del viso,
	caldo nel suo petto e liquido sotto le palpebre tremanti. \emph{Voglio
		tornare a casa!}}

{Quella notte Dap entrò nella camerata e si mosse lento fra le cuccette,
	toccando una fronte qua, una mano là. Al suo passaggio i pianti
	divenivano più intensi, invece di smorzarsi. Quel tocco di gentilezza in
	un posto così freddo e sconosciuto bastava a spingere i ragazzini oltre
	l'orlo delle lacrime. Non Ender, però. Quando Dap gli fu accanto i suoi
	singhiozzi erano spenti, il suo volto asciutto. Era il volto bugiardo
	che lasciava vedere a Mamma e a Papà, quando non osava far loro capire
	che Peter era stato crudele con lui. \emph{Grazie per questo, Peter. Per
		gli occhi asciutti e i singhiozzi silenziosi. Tu mi hai insegnato come
		nascondere ciò che sento. E adesso ho bisogno di questo più che mai.}}

{~}

\begin{center}
	{* * *}
\end{center}

{~}

{C'era sempre una scuola. Ogni giorno ore ed ore da trascorrere in
	classe. Letture. Numeri. Storia. Filmati di battaglie sanguinose
	avvenute nello spazio, coi marines che spargevano le loro budella sulle
	paratie delle navi degli Scorpioni. Olografie di nitide manovre belliche
	della Flotta, e astronavi che si trasformavano in sbuffi di luce mentre
	gli equipaggi uccidevano e venivano uccisi nella profonda notte cosmica.
	Molte cose da imparare. Ender lavorò duro come ogni altro, e tutti loro
	dovettero per la prima volta nella vita impegnarsi al massimo, perché
	per la prima volta erano in competizione con compagni di classe
	intelligenti almeno quanto loro.}

{Ma i giochi\ldots{} era questo ciò per cui vivevano. Ciò che riempiva
	le loro ore fra il mattino e la sera della stazione spaziale.}

{Dap li condusse nella sala dei giochi fin dal secondo giorno. Era in
	uno dei ponti superiori, piuttosto in alto rispetto al livello in cui i
	ragazzini vivevano e lavoravano. Si arrampicarono lungo scale dove la
	gravità diminuiva gradatamente, e in una grande caverna metallica videro
	lampeggiare le policrome luci dei giochi.}

{Alcuni erano giochi che avevano già fatto a casa loro, altri erano
	sconosciuti. C'erano quelli facili e quelli difficili. Ender oltrepassò
	la fila dei giochi sugli schermi bidimensionali e cominciò a osservare
	quelli dei ragazzi più grandi, i giochi olografici con gli oggetti che
	si spostavano nell'aria. Ben presto fu il solo del suo gruppo ad
	aggirarsi in quella zona della sala, e ogni tanto inciampava in uno dei
	giocatori, che trovandoselo troppo vicino non esitava a spingerlo via.
	Tu che stai facendo qui? Sparisci, pivello. Vola via. E naturalmente le
	spinte lo facevano volare, lì in quella gravità così bassa. I suoi piedi
	si staccavano dal suolo e lui roteava altrove, finché non andava a
	sbattere in qualcosa o in qualcuno.}

{Ogni volta, tuttavia, si districava dall'ostacolo e tornava indietro,
	non sempre nello stesso posto esatto, per studiare il gioco da
	un'angolazione diversa. Era troppo piccolo per arrivare a veder bene i
	pannelli di controllo, da cui le partite erano regolate. Questo non gli
	era d'ostacolo. Ne esaminava i risultati nel campo visivo
	tridimensionale. Studiava la tecnica con cui il giocatore scavava tunnel
	nella tenebra, tunnel di luce, a caccia dei quali le navi nemiche si
	sarebbero gettate per poi seguirli spietatamente fino a trovare quella
	del giocatore. Il vascello cacciato poteva lasciare trappole dietro di
	sé, mine, missili automatici, falsi percorsi che costringevano la nave
	inseguitrice a girare in tondo interminabilmente. Alcuni giocatori erano
	molto abili. Altri perdevano la gara fin dall'inizio.}

{Quello che però appassionava Ender erano le partite in cui due ragazzi
	si battevano fra loro, non contro la macchina. In tal caso ognuno poteva
	usare i tunnel dell'altro, e presto diveniva chiaro chi dei due stava
	usando la strategia più efficace.}

{Quel gioco in particolare cominciò a sembrargli insipido dopo appena un
	paio d'ore, tanto gli era bastato per capirne le regole. O meglio, capì
	le regole secondo cui funzionava il computer, e quindi fu certo che una
	volta appreso l'uso dei comandi sarebbe riuscito a sventare fatalmente
	le manovre dell'avversario. Spirali quando la nave nemica avanzava in un
	certo modo, circoli chiusi quando si spostava in un altro. Fingere di
	cadere in alcune delle trappole, farle scattare a vuoto giocando sugli
	impulsi di vicinanza per le prime sei, trasformare la settima in una
	falsa trappola con un espediente tecnico. Non si trattava di una sfida
	vera e propria, era soltanto questione di giocare finché il computer
	diventava così veloce da superare i riflessi umani. Ma col computer non
	era divertente. A lui interessava competere con un avversario umano. Con
	quei ragazzi talmente addestrati a battersi contro la macchina che anche
	durante le sfide reciproche tentavano di emulare il computer. E che
	pensavano come una macchina invece che come un ragazzo.}

\emph{{Potrei batterli con questo sistema. Potrei batterli con
		quest'altro.}}

{-- Mi piacerebbe fare una partita con te -- disse al giocatore che
	aveva appena vinto.}

{-- Santo cielo, e questo cos'è? -- esclamò il ragazzo. -- Una piattola
	che parla con voce umana?}

{-- Hanno appena tirato a bordo un'infornata di lattonzoli -- gli
	rispose un altro.}

{-- Ma questo \emph{parla.} Chi gli ha tolto il ciucciotto dalla bocca?}

{-- Ho capito -- annuì Ender. -- Hai paura di giocare con me. Due
	partite su tre, se te la senti.}

{-- Stracciarti sarebbe più facile che pisciare nel lavandino, bimbo.}

{-- E divertente neanche la metà -- aggiunse l'altro ragazzo.}

{-- Io sono Ender Wiggin.}

{-- Apri l'audio, piattola. Tu sei nessuno. Ricevuto? Tu sei esattamente
	un nessuno, sintonizzati su questo. E resterai un nessuno finché non
	avrai ammazzato il tuo primo qualcuno. Chiudi pure l'audio e fila.}

{Il gergo dei ragazzi più grandi aveva un suo ritmo. Ender non mancò di
	apprezzarlo. -- Se io sono nessuno, come va che tu hai paura di giocare
	a due su tre con me?}

{Adesso gli altri stavano emettendo grugniti d'impazienza. -- Regalati
	dieci secondi per far fuori questa piattola, e leviamocela dai piedi.}

{Fu così che Ender prese posto ai comandi, a lui del tutto sconosciuti.
	Le sue mani erano piccole, ma leve e tasti avevano uno schema abbastanza
	semplice. Gli bastò sperimentare i pulsanti per accertarsi di quali armi
	comandavano. I controlli dei movimenti erano riuniti in una leva di tipo
	standard. Dapprima i suoi riflessi furono lenti e incerti. L'altro
	ragazzo, che non gli aveva ancora detto il suo nome, procedette invece
	con inflessibile rapidità. Ma Ender apprese ciò che non sapeva, e prima
	che la partita fosse terminata stava andando molto meglio.}

{-- Soddisfatto, pivello?}

{-- Abbiamo detto due partite su tre.}

{-- Qui non usa né il due su tre, né il tre su cinque, bimbo.}

{-- Sei stato bravo a battermi la prima volta che tocco questa macchina
	-- disse Ender. -- Se riesci a battermi anche la seconda, ammetterò che
	puoi farlo sempre.}

{Cominciarono a giocare di nuovo, e stavolta Ender fu abbastanza svelto
	da riuscire a mettere in atto alcune manovre che il ragazzo, ovviamente,
	non aveva mai visto prima. I suoi schemi attacco-difesa non poterono
	competere con esse. Ender dovette sudare per vincere, ma ce la fece.}

{I ragazzi più grandi smisero di ridacchiare e di fare commenti
	spiritosi. La terza partita si svolse nel più completo silenzio. Ender
	la vinse con grande rapidità ed efficienza.}

{Quando fu finita uno dei presenti emise un grugnito. -- Fra ieri e oggi
	devono aver modificato questa macchina. Qualcuno l'ha adattata perché
	anche i lattonzoli possano giocare e vincere.}

{Non una parola di congratulazione. Un freddo silenzio fu il solo saluto
	che seguì Ender mentre se ne andava.}

{Non andò molto lontano. Pochi passi più in là si fermò accanto a
	un'altra macchina, e con la coda dell'occhio sbirciò per vedere se i
	successivi due giocatori cercavano di mettere in atto i metodi che aveva
	appena mostrato loro. \emph{Lattonzolo, eh?} Ender sorrise dentro di sé.
	\emph{Quel che hanno visto non lo dimenticheranno.}}

{Adesso si sentiva meglio. Aveva vinto qualcosa, e contro dei ragazzi
	più grandi. Probabilmente non i migliori fra gli allievi, e tuttavia
	questo bastava per liberarlo dalla sensazione terrorizzante d'essere un
	pesce fuor d'acqua, troppo inferiore alle esigenze della Scuola di
	Guerra. Ora non doveva far altro che osservare i giochi, capire come
	funzionavano e poi usare il sistema più adatto. O una variazione
	migliore.}

{Fu il fatto di attendere e di osservare che venne a costargli un
	prezzo. Perché in quel periodo ci furono cose che dovette sopportare. Il
	ragazzino a cui lui aveva rotto un braccio aveva giurato vendetta. Il
	suo nome, come Ender apprese subito, era Bernard. Parlava con chiaro
	accento francese, poiché i francesi, col loro arrogante Separatismo,
	affermavano che l'insegnamento dello Standard non doveva cominciare fino
	ai quattro anni di età, e per allora i bambini avevano già assimilato
	profondamente la lingua madre. Il suo accento lo rendeva un tipo esotico
	e dunque interessante; il suo braccio rotto aveva fatto di lui un
	martire; il suo sadismo lo trasformò in un capo naturale per tutti
	quelli a cui piaceva veder soffrire gli altri.}

{Ender fu il loro primo nemico.}

{Piccole cose. Un calcio che gli disfaceva il letto ogni volta che
	entravano e uscivano dalla porta. Uno sgambetto mentre andava al tavolo
	col vassoio del pranzo. Pestoni sulle mani quando salivano le scale a
	pioli. Ender ci mise poco a imparare che non doveva lasciare niente di
	suo fuori dagli armadietti, e dovette anche imparare a stare all'erta
	per non finire a gambe all'aria d'improvviso. «Sbadatroccolo» lo chiamò
	una volta Bernard, per trovargli un soprannome sprezzante.}

{Ci furono momenti in cui Ender conobbe il tormento della rabbia. Ma
	contro Bernard, naturalmente, la sola rabbia non bastava. Era il tipo di
	ragazzo che era: un torturatore. Ciò che irritava Ender era il vedere
	con quale acquiescenza gli altri si associavano a lui. Senza dubbia essi
	dovevano capire che la sua voglia di vendicarsi era ingiusta. Senza
	dubbio sapevano che era stato lui a colpire per primo Ender sulla
	navetta, e che Ender s'era limitato a rispondere a un sopruso. Ma se lo
	sapevano, agivano come se le cose stessero al contrario. E anche quelli
	che non lo sapevano avrebbero dovuto capire da una sua sola parola che
	Bernard era un serpente velenoso.}

{Ma Ender non costituiva il suo unico bersaglio. Ciò che Bernard stava
	costruendo era un piccolo regno, con una sua piccola corte.}

{Sempre ai bordi del gruppo, isolato dai compagni, Ender assistette alle
	manovre di Bernard che stabiliva il rango dei suoi cortigiani. Alcuni
	ragazzi gli erano utili, e lui li ricopriva di melassa. Altri avevano
	l'istinto di servire, e gli ubbidivano ciecamente anche quando li
	maltrattava sprezzantemente.}

{Ma c'era anche chi s'irritava ai modi di Bernard.}

{Bastava osservarlo per vedere chi altri prendeva di mira. Shen era
	piccolo, ambizioso e molto suscettibile. Questo particolare era stato
	notato subito da Bernard, che aveva preso a soprannominarlo Verme.}

{-- Solo perché è così sottile, si capisce -- spiegò Bernard, -- e
	perché \emph{serpeggia.} Guardate come fa andare i fianchi quando
	cammina.}

{Shen gli diede un'occhiataccia e si allontanò con andatura rigida, ma
	questo fece ridere gli altri ancor di più. -- Guardate il suo
	\emph{culo.} Striscia, Verme!}

{Ender non disse nulla a Shen. Una mossa così scoperta avrebbe fatto
	pensare che cercava di riunire una sua banda, avversa all'altra. Restò
	chino sulla sua piccola scrivania elettronica, mostrandosi indifferente
	e dedito allo studio.}

{Ma non stava studiando. Stava cercando di regolare la scrivania perché
	mandasse un messaggio durante le lezioni, appena cominciate. Il
	messaggio doveva essere breve e diretto a tutti. La difficoltà
	consisteva nel celare l'identità del mittente, cosa che il computer
	consentiva soltanto all'insegnante. Alle frasi battute dagli alunni
	veniva automaticamente accluso il loro nome. Ender non era ancora
	riuscito a inserirsi sulla linea usata dagli insegnanti, dunque non
	poteva fingere di essere uno di loro. Ma conosceva il modo di costruire
	un fascicolo nuovo per un alunno inesistente, e una volta inseriti i
	dati, in un impulso di stravaganza, diede a questo alunno il nome
	\emph{Dio.}}

{Soltanto quando il messaggio fu pronto per partire si permise di
	cercare lo sguardo di Shen. Come altri ragazzi anch'egli stava prestando
	meno attenzione all'insegnante di matematica che ai compagni di Bernard:
	ridacchiavano, scambiandosi spiritosaggini sull'insegnante, che ogni
	tanto interrompeva un'operazione a metà per guardarsi attorno con l'aria
	perplessa di chi è uscito dall'autobus e non capisce a quale stazione
	l'hanno fatto scendere.}

{Da lì a poco tuttavia Shen si volse. Ender gli fece un cenno, indicò la
	superficie del banco e sorrise. Shen lo fissò senza capire. Lui batté
	ripetutamente l'indice sul banco. Finalmente Shen abbassò gli occhi sul
	suo, e in quell'istante Ender mandò il messaggio. Vide Shen leggerlo con
	tanto d'occhi, poi rialzare il capo e scoppiare a ridere. Il ragazzino
	tornò a fissare Ender con un'espressione che chiedeva: sei stato tu?
	Ender ~scosse il capo e si strinse nelle spalle, come a dire: no di
	certo, e non so proprio chi possa esser stato.}

{Shen rise ancora, e parecchi dei ragazzi non facenti parte del gruppo
	di Bernard seppero dai suoi cenni che sui loro banchi c'era qualcosa. Il
	messaggio appariva ogni trenta secondi, girava svelto lungo il perimetro
	degli schermi e poi spariva. Una quindicina di alunni scoppiarono a
	ridere contemporaneamente.}

{-- Cos'è che li diverte tanto? -- chiese Bernard. Ender badò bene a
	restare perfettamente serio quando il ragazzo girò attorno lo sguardo
	fosco con cui spaventava i più timidi. Shen, invece, ghignò in modo
	apertamente derisorio. I compagni di Bernard smisero di far battute
	sull'insegnante e osservarono i loro banchi, su cui correva la scritta:}

{~}

\begin{center}
	{COPRITEVI IL CULO. BERNARD VE LO GUARDA.}
\end{center}

\begin{flushright}
	{- DIO}
\end{flushright}

{~}

{Bernard s'imporporò per la rabbia. -- Chi è stato? -- gridò.}

{-- Dio, sembra -- lo informò Shen. -- Perché guardi me?}

{-- So perfettamente che non sei stato tu -- sbottò Bernard. -- Per far
	questo occorre molto più cervello di quello che ha un Verme!}

{Da lì a cinque minuti Ender fece svanire il messaggio. Dopo un po' al
	centro del suo banco ne apparve un altro:}

{~}

\begin{center}
	{SO CHE SEI STATO TU.}
\end{center}

\begin{flushright}
	{- BERNARD}
\end{flushright}

{~}

{Ender non rialzò lo sguardo, e si comportò come se non avesse ricevuto
	alcun messaggio. \emph{Bernard sta solo cercando di scoprire se ho la
		faccia del colpevole. Ma non lo sa.}}

{Naturalmente non importava nulla che sapesse o meno. Bernard avrebbe
	cercato di fargliela pagare, per il solo fatto che non poteva
	permettersi di perdere la faccia. L'unica cosa che non riusciva a
	sopportare era che gli altri ridessero di lui. Doveva far capire a tutti
	chi era il capo. Fu così che quel mattino Ender finì faccia a terra nel
	locale delle docce. Uno dei compagni di Bernard attese che l'inserviente
	si voltasse e gli piantò un ginocchio nell'addome. Ender mandò giù il
	rospo in silenzio. Stava ancora osservando e aspettando, e non intendeva
	mostrare agli insegnanti che fra lui e l'altro c'era guerra aperta.}

{Ma nell'altra guerra, quella che si svolgeva sui banchi, aveva già
	messo in opera l'attacco successivo. Quando tornò dalle docce trovò
	Bernard in preda alla rabbia; stava prendendo a calci le cuccette e
	gridava ai compagni: -- Non sono stato io a scriverlo! State zitti!}

{Sul banco di ogni ragazzo era in marcia un messaggio luminoso:}

{~}

\begin{center}
	{AMO I VOSTRI BEI CULETTI. LASCIATEMELI BACIARE.}
\end{center}

\begin{flushright}
	{- BERNARD}
\end{flushright}

{~}

{-- Ho detto che non sono stato io a scriverlo! -- strillò Bernard. Dopo
	qualche minuto quelle urla fecero apparire Dap sulla soglia della
	camerata.}

{-- Cos'è questo baccano? -- li apostrofò.}

{-- Qualcuno che usa il mio nome sta mandando attorno delle scritte! --
	brontolò imbronciato Bernard.}

{-- Quali scritte?}

{-- Non importa quali!}

{-- A me importa. -- Dap si accostò al banco del ragazzo che aveva la
	cuccetta accanto a quella di Ender. Lesse il messaggio, la sua bocca
	parve curvarsi in un mezzo sorriso, poi spinse il banco
	nell'armadietto.}

{-- Interessante -- disse.}

{-- Adesso indagherà per scoprire il colpevole? -- volle sapere
	Bernard.}

{-- Oh, lo conosco già -- rispose Dap.}

\emph{{Sì}}{, \emph{} rifletté Ender. \emph{È stato troppo facile
		inserirmi nel programma. Loro sanno che si può far questo col computer,
		forse anzi ci contano. E sanno che l'intrusione è venuta dal mio
		banco.}}

{-- Be', allora chi è? -- sbottò Bernard.}

{-- Stai gridando con me, recluta? -- chiese dolcemente Dap.}

{All'istante l'atmosfera della camera cambiò. Se gli amici di Bernard
	avevano fatto commenti rabbiosi, e da parte degli altri c'erano state
	risatine ironiche, tutti tacquero. L'autorità stava facendo sentire la
	sua voce.}

{-- Nossignore -- disse Bernard.}

{-- Tutti sanno che il programma inserisce automaticamente il nome del
	mittente.}

{-- Io non ho scritto quella roba! -- replicò Bernard.}

{-- Allora perché ti agiti tanto, marmocchio? -- disse Dap.}

{-- Ieri qualcuno ha mandato in giro un messaggio firmato DIO --
	aggiunse Bernard acremente.}

{-- Sul serio? -- chiese Dap. -- Guarda, guarda. Non sapevo che Dio
	fosse inserito nei programmi. -- Gli volse le spalle e uscì, e la
	camerata fu piena di risa divertite.}

{Il tentativo di Bernard d'eleggersi a piccolo duce del loro gruppo si
	sfasciò così nel ridicolo: soltanto pochi gli rimasero fedeli. Ma erano
	i più pervicaci. E Ender seppe che finché si fosse limitato a osservare
	e attendere per lui sarebbe stata dura. Tuttavia quel giochetto col
	computer aveva ottenuto un risultato. Bernard era stato rimesso a posto,
	e tutti i ragazzi che avevano qualche buona qualità erano liberi dalla
	sua influenza. Ma soprattutto, Ender c'era arrivato senza mandarlo
	un'altra volta in mano al medico. \emph{Molto meglio a questo modo},
	\emph{} pensò.}

{Poi si dedicò al difficile compito d'inserire un migliore sistema di
	sicurezza nel suo banco, visto che quelli previsti dal normale programma
	erano evidentemente inadeguati. Se un ragazzino di sei anni poteva farvi
	breccia, era chiaro che li avevano predisposti per eseguire una routine
	senza garanzie di riservatezza. \emph{Soltanto un altro gioco che gli
		insegnanti hanno studiato per noi. Ed è un gioco a cui sono bravo.}}

{-- Come ci sei riuscito? -- gli chiese Shen, a colazione.}

{Ender prese nota con calma che per la prima volta un ragazzino della
	sua classe veniva a sedersi a tavola accanto a lui. -- Riuscito a far
	cosa? -- domandò.}

{-- A mandare un messaggio con un nome falso. E poi con quello di
	Bernard! È stata grande. Adesso lo soprannominano Il Guardaculi. Davanti
	all'insegnante lo chiamano solo Il Guarda, ma tutti sanno che cosa
	guarda.}

{-- Povero Bernard -- mormorò Ender. -- Pensare che è così sensibile.}

{-- Avanti, Ender. Tu ti sei inserito nel programma. Come hai fatto?}

{Ender scosse il capo e sorrise. -- Grazie per aver pensato che io sia
	tanto abile da riuscirci. L'ho soltanto visto per primo, questo è
	tutto.}

{-- D'accordo, non sei costretto a dirmelo -- annuì Shen. -- Comunque è
	stata grande. -- Per un poco mangiò in silenzio. -- Sul serio faccio
	ondeggiare il sedere quando cammino?}

{-- Ma no -- disse Ender. -- Appena un poco. Solo, bada a non fare quei
	passi così lunghi, e sarai a posto.}

{Shen annuì.}

{-- L'unico che l'abbia notato è stato Bernard.}

{-- È un maiale -- disse Shen.}

{Ender scosse le spalle. -- Evita i maiali e non ne sentirai il puzzo.}

{Shen rise. -- Hai ragione. Io pure li individuo a naso.}

{Risero entrambi, guardandosi, e altri due ragazzini del loro gruppo
	vennero a sedersi accanto ad essi. L'isolamento di Ender era finito. La
	guerra era soltanto nella sua fase iniziale.}

\phantomsection\label{Orsonux20Scottux20Cardux20-ux20Ilux20Giocoux20Diux20Enderux20-ux20BY_SLY70A1_split_008.htm}{}
