\chapter{IL MAESTRO DI ENDER}

{~}

{~}

{~}

{-- \emph{Se l'è presa comoda, eh, Graff? Il viaggio ha richiesto il suo
		tempo, ma tre mesi di vacanza mi sembrano un po' troppi.}}

{-- \emph{Preferisco consegnare al cliente un prodotto non
		deteriorato.}}

{-- \emph{Ci sono uomini che semplicemente non hanno il senso
		dell'urgenza. Oh, be', dopotutto c'è in gioco soltanto il destino del
		mondo. Ma abbia pazienza con me. Qui dentro si rischia di diventare
		stupidamente ansiosi, no? Sempre attaccati all'ansible, sempre in
		ascolto dei rapporti delle nostre astronavi, giorno dopo giorno di
		fronte al costante avvicinarsi della guerra. Sempre che li si possano
		chiamare giorni. Bene, ho visto che si tratta di un ragazzino davvero
		molto giovane.}}

{-- \emph{C'è una certa grandezza in lui. Una grandezza spirituale.}}

{-- \emph{E anche l'istinto del killer, voglio sperare.}}

\emph{{-- Sì.}}

{-- \emph{Abbiamo programmato un corso di studi su misura per lui. Il
		tutto condizionato al suo beneplacito, naturalmente.}}

{-- \emph{Ci darò un'occhiata. Non pretendo però di conoscere le materie
		in oggetto, ammiraglio Chamrajnagar. Io sono qui solo perché conosco
		Ender. Così abbandoni pure il timore che io modifichi gli studi da lei
		programmati. Tutt'al più il loro ritmo.}}

{-- \emph{Ci sono argomenti che lei consiglia?}}

{-- \emph{Non fategli sprecare il suo tempo con la fisica dei viaggi
		interstellari.}}

{-- \emph{E per quanto riguarda l'ansible?}}

{-- \emph{Gli ho parlato anche di questo, e delle nostre flotte. Sa che
		arriveranno a destinazione fra cinque anni.}}

{-- \emph{Sembra che ci abbiate lasciato ben poco da dirgli.}}

{-- \emph{Dovrete spiegargli come funzionano i sistemi d'arma. Per
		prendere decisioni efficaci bisogna che conosca i particolari tecnici.}}

{-- \emph{Dunque serviremo anche noi a qualcosa, infine. È bello
		saperlo. Abbiamo riservato uno dei cinque simulatori per suo uso
		esclusivo.}}

{-- \emph{Cosa mi dice degli altri?}}

\emph{{-- Gli altri simulatori?}}

{-- \emph{Gli altri ragazzi.}}

{-- \emph{Lei è stato assegnato qui per occuparsi di Ender Wiggin.}}

{-- \emph{Pura curiosità. Non dimentichi che sono stati tutti miei
		studenti.}}

{-- \emph{E adesso sono tutti miei. Si stanno addentrando nei misteri
		della flotta, colonnello Graff; misteri ai quali lei, come soldato, non
		è mai stato introdotto.}}

{-- \emph{Ne parla come se fosse una religione.}}

{-- \emph{Con un Dio, e con dei sacerdoti. Anche quelli di noi che
		comandano a mezzo ansible conoscono la sublime grandezza del volo fra le
		stelle. Vedo che lei sembra trovar biasimevole il mio misticismo. Le
		assicuro che la sua disapprovazione nasce soltanto dall'ignoranza. Ben
		presto anche Ender Wiggin conoscerà ciò che conosco io; danzerà
		anch'egli la dolce e spettrale danza fra le costellazioni, e se in lui
		c'è grandezza essa scaturirà dal suo spirito, rivelata, affinché il
		resto dell'universo ne apprenda la nobiltà. Lei ha un'anima di pietra,
		colonnello Graff, ma anche la pietra può assumere forma, sfiorata dallo
		scalpello della verità. Adesso può andare nel suo alloggio e sistemare
		il bagaglio.}}

{-- \emph{L'unico bagaglio che mi sono portato dietro ce l'ho indosso.}}

{-- \emph{Vuol dire che non possiede niente?}}

{-- \emph{La vile moneta che mi pagano viene misticamente raccolta dai
		sacerdoti del denaro, sulla Terra, nei sacri recessi di qualche banca.
		Non ho mai avuto bisogno di niente, a parte un abito civile per la mia
		recente\ldots{} vacanza.}}

{-- \emph{Un potenziale francescano. E tuttavia lei è disgustosamente
		ingrassato. Ascetismo e ghiottoneria dunque? Quale contraddizione!}}

{-- \emph{Quando sono teso, io mangio. Là dove voi, evidentemente,
		reagite alla tensione espellendo dal corpo rifiuti mistici.}}

{-- \emph{Lei mi piace, colonnello Graff. Penso che finiremo per
		intenderci.}}

{-- \emph{Non si aspetti da me l'identico sforzo, ammiraglio
		Chamrajnagar. Io sono venuto qui per Ender. E né lui né io siamo venuti
		qui per lei.}}

{~}

\begin{center}
	{* * *}
\end{center}

{~}

{Ender detestò Eros fin dal momento in cui scese dalla navetta di
	collegamento. Era stato abbastanza a disagio sulla Terra, dove ogni
	pavimentazione era piatta, ma nell'asteroide c'era di peggio. Si
	trattava di un blocco di roccia lungo e affusolato, largo soltanto sei
	chilometri e mezzo nel punto più stretto. Poiché la superficie del
	pianetino era interamente coperta da fotocellule che trasformavano in
	energia la radiazione solare, tutti abitavano in ambienti dalle pareti
	lisce collegati da tunnel che si ramificavano nelle viscere rocciose.
	Vivere in uno spazio chiuso non era certo un problema per Ender; ciò che
	lo colpì fu la constatazione che tutti i tunnel si piegavano
	visibilmente verso il basso. Fin dall'inizio questo gli diede una
	spiacevole e vertiginosa sensazione, specialmente quando passava nel
	tunnel che girava lungo la circonferenza più esterna di Eros. Il fatto
	che la gravità fosse metà di quella terrestre non gli era affatto
	d'aiuto, anzi incrementava l'illusione ottica d'essere sul bordo di un
	lungo precipizio.}

{C'era qualcosa di molto antipatico anche nelle proporzioni dei locali:
	i soffitti erano troppo bassi per la loro ampiezza, e i tunnel troppo
	stretti. Non era un posto costruito a misura d'uomo.}

{Ma la cosa meno sopportabile di tutte era l'affollamento. Ender non
	aveva esperienze di vita in città terrestri di medie o grosse
	dimensioni. Il suo concetto di località abitata s'era formato alla
	Scuola di Guerra, dove aveva conosciuto almeno di vista ogni persona.
	All'interno di quella roccia vivevano invece oltre diecimila anime. Non
	che mancasse lo spazio, malgrado la gran quantità di macchinari e di
	attrezzature di supporto. Ciò che infastidiva Ender era il vedersi
	costantemente d'attorno facce sconosciute.}

{Il tempo di farsi delle conoscenze non gli era concesso. Vedeva una
	gran quantità di altri studenti della Scuola Ufficiali, ma poiché il suo
	programma lo costringeva a migrare da un corso all'altro essi restavano
	soltanto dei volti. Assisteva a una lezione qui e a una conferenza là,
	tuttavia di solito l'uno o l'altro insegnante si occupava privatamente
	di lui; oppure a mostrargli tecniche e procedimenti era uno studente
	anziano che prima e dopo quella particolare circostanza non incontrava
	mai più. Pranzava da solo o con il colonnello Graff. La sua sola
	ricreazione era in palestra, e anche lì difficilmente vedeva due volte
	di seguito le stesse persone.}

{Sapeva bene che lo stavano isolando ancora, non più con la tattica di
	renderlo inviso agli altri studenti, ma piuttosto privandolo
	dell'opportunità di farsi degli amici. Difficilmente, comunque, avrebbe
	potuto legare con la maggior parte di loro: quasi tutti gli allievi
	s'erano già lasciati alle spalle la prima adolescenza.}

{Così s'immerse nello studio, desiderando solo imparare presto e bene.
	L'astrogazione e la storia militare erano materie che assorbiva come
	acqua, la matematica nelle sue forme più astratte gli dava delle
	difficoltà, ma quando doveva risolvere problemi che coinvolgevano lo
	spazio e il tempo scopriva che il suo intuito era più affidabile anche
	dei calcoli, e spesso vedeva la soluzione prima di poterla provare sotto
	forma di noiose e snervanti equazioni.}

{E per il suo divertimento c'era il simulatore, il più perfezionato
	videogame a cui avesse mai giocato. Studenti e insegnanti lo guidarono
	passo per passo entro le complessità di quei programmi. Dapprima, non
	conoscendo le impressionanti potenzialità delle partite, aveva giocato
	soltanto a livello tattico, controllando un singolo astrocaccia in
	continue manovre tese alla ricerca e alla distruzione del nemico.
	L'avversario, controllato dal computer, era sempre astutissimo e
	potente, e qualunque tattica escogitasse Ender scopriva, da lì a pochi
	minuti, che il computer sapeva rivolgerla contro di lui.}

{La partita era giocata in un campo olografico tridimensionale, ed il
	suo astrocaccia era rappresentato da un piccolo punto luminoso. Il
	nemico era una lucciola di colore diverso, ed entrambi volavano e
	combattevano entro un cubo di spazio di dieci metri per dieci. I comandi
	davano ampie possibilità: il campo cubico poteva esser fatto ruotare su
	se stesso, in modo che il giocatore lo osservasse da un angolo visivo a
	suo piacimento, e lo spazio contenuto all'interno si spostava
	automaticamente o su ordinazione per trasferire il duello in zone sempre
	nuove.}

{Pian piano, mentre s'impratichiva nel controllo tecnico
	dell'astrocaccia e nelle possibilità d'impiego delle sue armi, le
	partite si fecero più complesse. Poteva trovarsi di fronte due o più
	navi nemiche; c'erano ostacoli tipo detriti cosmici di cui si doveva
	calcolare la rotta o campi gravitazionali dei quali prevedere
	l'attrazione, e il giocatore era costretto a chiedere ed elaborare
	questi dati su dei monitor ausiliari. Per buona parte del combattimento
	ci si doveva preoccupare delle scorte di carburante o di energia, o di
	improvvisi guasti nei sistemi d'arma. Il computer cominciò ad
	assegnargli anche compiti particolari da portare a termine, come manovre
	di soccorso o di retroguardia eseguite in condizioni anomale o
	disagevoli.}

{Quando ebbe padroneggiato a dovere l'astrocaccia gli fu data la
	conduzione di una squadra di quattro incrociatori. Poteva parlare a
	voce, sgranando raffiche di ordini vuoi ai piloti vuoi agli addetti alle
	batterie, o al resto del personale delle quattro grosse navi. E invece
	di eseguire le istruzioni di un ipotetico comando supremo gli era
	concesso di determinare lui stesso la strategia, stabilendo quale dei
	molti obiettivi era il più importante da raggiungere. Aveva la
	possibilità di programmare tre degli incrociatori e lasciarli agire da
	soli, mettendosi personalmente alla manovra del quarto, e dapprima
	scelse spesso questo modo d'agire. In tali circostanze tuttavia i tre
	incrociatori facevano in breve una brutta fine, e la partita diventava
	assai più dura, cosicché dovette lavorare settimane e mesi per imparare
	il controllo dello squadrone al completo. Quando cominciò a riuscirci le
	sue vittorie divennero più frequenti.}

{Al termine del suo primo anno alla Scuola Ufficiali fu in grado di
	usare il simulatore a ciascuno dei quindici livelli di difficoltà,
	ovvero dal controllo di un singolo astrocaccia al comando di un'intera
	flotta. Già da tempo s'era accorto che il simulatore della Scuola
	Ufficiali aveva scopi analoghi a quelli della sala di battaglia alla
	Scuola di Guerra. Le lezioni teoriche avevano la loro importanza, ma le
	cognizioni ottenute e le capacità personali del singolo erano
	controllabili solo al momento in cui egli si applicava al simulatore.}

{Di tanto in tanto s'accorgeva che dietro di lui, nei posti riservati
	agli spettatori, qualcuno lo osservava giocare. Gli studenti o l'altro
	personale non aprivano mai bocca, ma a volte un insegnante interveniva,
	se aveva qualcosa di specifico di cui informarlo. Ender imparò a
	ignorare quel piccolo pubblico che in silenzio lo guardava affrontare
	complesse situazioni simulate ed infine se ne andava senza alcun
	commento. \emph{Ebbene, vi siete divertiti?} avrebbe voluto chieder
	loro. \emph{Mi avete giudicato? Avete deciso se vi fidereste a far parte
		di una flotta comandata da me? Ma ricordate che non ho chiesto io
		d'essere il candidato al comando supremo.}}

{Trovava che gran parte dei concetti da lui sviluppati in sala di
	battaglia potevano essere trasferiti al simulatore, e ogni pochi minuti
	faceva ruotare del tutto il campo olografico per non restare
	intrappolato in un'orientazione alto-basso, riesaminando costantemente
	la sua posizione anche dal punto di vista del nemico. Era esilarante
	avere quel controllo esterno su una battaglia, partecipandovi sia dal
	ponte di comando di un'astronave che da qualsiasi luogo al di fuori di
	essa.}

{Ed era frustrante avere nello stesso tempo ai suoi ordini astronavi
	così limitate, perché quelle che metteva sotto il controllo del computer
	diventavano oggetti computerizzati. Non avevano iniziativa, ma soltanto
	una programmazione. Non avevano intelligenza. Cominciò a provare molta
	nostalgia dei suoi capibranco dell'orda dei Draghi, ragazzi che avrebbe
	potuto piazzare al comando dei vari squadroni e che avrebbero agito bene
	senza bisogno della sua costante supervisione.}

{Alla fine del primo anno stava vincendo ogni battaglia sul simulatore,
	e giocava come se i comandi e i monitor fossero estensioni del suo
	corpo. Un giorno, mentre mangiava insieme a Graff, gli domandò: -- Ciò
	che fa il simulatore è tutto qui?}

{-- Tutto in che senso?}

{-- Il modo in cui va il gioco. Troppo liscio. Da tempo non trovo più
	ostacoli e difficoltà nuove.}

{-- Ah!}

{Graff parve indifferente alla cosa. Ma Graff aveva una maschera
	indifferente per costituzione. Il giorno dopo le cose cambiarono: Graff
	non si fece vedere, e al suo posto Ender si vide dare un nuovo
	compagno.}

{~}

\begin{center}
	{* * *}
\end{center}

{~}

{L'uomo era già in camera sua quando Ender si svegliò, quel mattino. Era
	piuttosto anziano, e stava seduto sul pavimento a gambe incrociate. Il
	ragazzo si sfregò le palpebre e lo fissò in silenzio, aspettando che
	dicesse qualcosa. Lui non aprì bocca. Ender si alzò, si lavò la faccia e
	si vestì, lasciando che lo sconosciuto mantenesse il silenzio finché gli
	faceva piacere. Aveva già appreso da tempo che quando c'era in corso
	qualcosa d'insolito, qualcosa che era parte dei piani di qualcun altro e
	non dei suoi, otteneva maggiori informazioni aspettando piuttosto che
	chiedendo. Quasi tutti gli adulti perdevano la pazienza assai prima di
	lui.}

{Tuttavia l'uomo non aveva ancor detto verbo quando Ender fu pronto per
	uscire e andò alla porta. La porta non si aprì. Lui si volse a fissare
	il vecchio che sedeva sul pavimento. Dimostrava più di sessant'anni, ed
	era di gran lunga il più anziano di quelli che Ender aveva visto
	finallora su Eros. Aveva la barba di un giorno, una spolverata di
	peluzzi bianchi che insieme ai corti capelli spettinati dava alla sua
	faccia un aspetto ispido. Aveva guance un po' cascanti e occhi
	circondati da una rete di rughe. Rispose allo sguardo di Ender con
	un'espressione che lui trovò completamente apatica.}

{Si volse alla porta e cercò ancora di aprirla. Invano.}

{-- E va bene -- si rassegnò a dire. -- Perché è chiusa?}

{Il vecchio continuò a fissarlo con occhi vuoti.}

\emph{{Così questa è la partita di oggi}}{, \emph{} pensò Ender.
	\emph{Bene. Se vogliono che io vada in classe apriranno la porta. Se non
		lo fanno è segno che non vogliono. Per me fa lo stesso.}}

{Non gli era nuovo giocare a un gioco le cui regole sembravano
	evanescenti ed i cui obiettivi erano noti solo a qualcun altro. Ma
	rifiutò d'irritarsi per questo. Appoggiato con le spalle al battente
	fece alcuni esercizi di respirazione e poco dopo fu di nuovo calmo. Il
	vecchio non faceva altro che guardarlo, impassibile.}

{I minuti trascorsero e divennero ore, e più Ender s'intestardiva nel
	tener chiusa la bocca più lo sconosciuto sembrava tramutarsi in una
	statua priva di mente. Il ragazzo dovette chiedersi se non si trovasse
	davanti a un pazzo, un anormale sfuggito alla sorveglianza medica di
	Eros e che ora stava vivendo qualche sua insana fantasia lì nella sua
	camera. Ma più la situazione si prolungava, mentre nessuno veniva a
	bussare alla porta o a cercare di lui, e più si convinceva d'essere a
	confronto con un'azione deliberatamente tesa a sconcertarlo. E non
	voleva dare a quel vecchio la soddisfazione di spuntarla. Per ingannare
	il tempo cominciò a fare ginnastica. Gli esercizi più duri erano
	impossibili senza l'equipaggiamento della palestra, ma altri,
	specialmente quelli di preparazione alla lotta corpo a corpo, non
	richiedevano nessun attrezzo.}

{Gli esercizi lo portavano qua e là per la stanza. Stava facendo pratica
	di calci e colpi col taglio della mano. Una mossa lo costrinse a passare
	di fronte allo sconosciuto, e non era la prima volta che ciò accadeva,
	ma stavolta una delle vecchie braccia scattò di lato e lo colpì dietro
	il ginocchio d'appoggio proprio a metà di un passo. Ender perse
	l'equilibrio e cadde pesantemente al suolo.}

{All'istante balzò in piedi, furibondo, e si mise in guardia. Il vecchio
	sedeva calmissimo a gambe incrociate e il suo respiro non s'era
	accelerato di un filo, come se non si fosse mai mosso. Ender era pronto
	a battersi, ma l'immobilità dell'altro gli rendeva impossibile
	attaccarlo. \emph{Che faccio? Gli stacco la testa con un calcio? Sai che
		divertimento poi dover dire a Graff: il vecchio bastardo mi ha colpito,
		e ho dovuto reagire.}}

{Riprese gli esercizi, standogli a distanza. E il vecchio continuò a
	fissarlo.}

{Infine, stanco e irritato, prigioniero nella sua stessa camera, tornò
	accanto al letto a prendere il suo banco. Mentre si chinava per estrarlo
	dallo scomparto sentì una mano robusta afferrarlo rudemente fra le
	cosce, e un'altra per i capelli. Un attimo più tardi era stato sbattuto
	faccia a terra. Le ginocchia del vecchio gli premevano dietro le spalle
	schiacciandogli il petto e il volto contro il pavimento, aveva la
	schiena piegata all'indietro, e le sue gambe erano strette in un
	abbraccio che gliele sollevava dal suolo. In quella posizione non
	riusciva a usare le braccia, e alcune vane contorsioni lo informarono
	che non avrebbe potuto neppure scalciare. In meno di due secondi il
	vecchio lo aveva completamente sconfitto e immobilizzato.}

{-- Va bene! -- ansimò Ender. -- Hai vinto tu.}

{Le ginocchia gli affondarono dolorosamente nella schiena. -- E da
	quando -- disse l'uomo con voce bassa e rauca, -- il tuo nemico ha
	bisogno che sia tu a dirgli che ha vinto?}

{Ender rimase in silenzio.}

{-- Ti ho già colto di sorpresa una volta, Ender Wiggin. Perché non mi
	hai distrutto immediatamente dopo? Solo perché avevo un'aria innocua? Mi
	hai voltato le spalle. Idiozia. Non hai imparato niente. Non hai mai
	avuto un maestro.}

{Ender ebbe un impeto di rabbia, e non fece nulla per controllarlo o
	nasconderlo. -- Di insegnanti ne ho avuti fin troppi! Perché diavolo
	avrei dovuto immaginare che lei si sarebbe rivelato un lurido\ldots{}}

{-- Un nemico, Ender Wiggin -- sussurrò il vecchio. -- Io sono il tuo
	nemico. Il primo che sia più astuto, svelto e intelligente di te. Non
	c'è nessun insegnante salvo il nemico. Nessuno, salvo il nemico, ti
	lascerà mai capire ciò che il nemico sta per fare. Nessuno, salvo il
	nemico, t'insegnerà come distruggere e conquistare. Soltanto il nemico
	ti mostrerà i tuoi punti deboli. E le sole regole del gioco sono i colpi
	che gli puoi dare e quelli che puoi impedirgli di darti. Da ora in poi
	io sono il tuo nemico. Da ora in poi io sono il tuo maestro.}

{Il vecchio lasciò ricadere le gambe di Ender. Poiché la sua faccia era
	ancora schiacciata al suolo il ragazzo non poté compensare il movimento,
	e quando piedi e ginocchia sbatterono sul pavimento con un tonfo, dai
	polmoni gli scaturì un ansito di dolore. Poi l'altro gli si tolse di
	dosso.}

{Lentamente Ender ritirò le gambe sotto di sé, lasciando che una fosca
	smorfia sofferente gli affiorasse sul volto. Per qualche istante restò a
	quattro zampe, e riprese fiato. Poi il suo braccio destro scattò per
	colpire all'inguine il nemico. Con un saltello il vecchio balzò fuori
	portata, e mentre la mano di Ender annaspava nel vuoto l'altro avventò
	un piede per scalciarlo alla testa.}

{La testa di Ender non era più lì. S'era girato svelto sulla schiena, e
	nell'istante in cui il vecchio oscillava, sbilanciato dal calcio andato
	a vuoto, lui replicò con un altro calcio dietro il ginocchio. Con un
	grugnito l'uomo cadde, ma abbastanza vicino da riuscire a sferrargli un
	pugno in faccia. Ender gli balzò addosso, però l'altro si divincolava
	così furiosamente che gli fu impossibile afferrargli saldamente un
	braccio o una gamba, e nel frattempo una grandine di botte gli
	tempestava sulla schiena. Lui era più piccolo di statura; quando capì
	che l'avversario sapeva sfruttare troppo bene quel vantaggio lo scalciò
	via da sé, rotolò fino alla porta e con uno scatto di reni si rialzò.}

{Il vecchio s'era di nuovo seduto a gambe incrociate, ma adesso la sua
	espressione apatica era sparita. Stava sorridendo. -- Meglio, stavolta,
	ragazzo. Ma sei lento. Dovrai imparare a manovrare una flotta meglio di
	come manovri il tuo corpo, o nessuno sarà al sicuro sotto il tuo
	comando. Capito la lezione?}

{Lentamente Ender annuì. Il suo corpo era tutto un dolore.}

{-- Bene -- disse il vecchio. -- Non avremo più battaglie di questo
	genere. Tutte le altre saranno col simulatore. A programmartele adesso
	sarò io, con il computer; io svilupperò la strategia del tuo nemico, e
	tu imparerai a essere svelto e a scoprire quali trucchi il nemico ha in
	serbo per te. E ficcati in capo questo, ragazzo: d'ora in poi hai un
	nemico più veloce di te. D'ora in poi hai un nemico più forte di te.
	D'ora in poi sarai sempre sull'orlo della disfatta.}

{Il sorriso svanì dalla faccia del vecchio. -- Sarai sempre a un pelo
	dalla sconfitta, Ender, ma dovrai batterti per vincere. E se dentro di
	te ci sarà questa forza, io ti insegnerò a farlo.}

{Il maestro si alzò. -- In questa scuola esiste la tradizione che uno
	studente anziano scelga uno studente giovane. I due diventano compagni,
	e il più anziano insegna al più giovane tutto ciò che sa. Studiano
	insieme, combattono insieme e uno contro l'altro. Io ho scelto te.}

{Ender lo vide alzarsi e andare alla porta. -- Lei è troppo vecchio per
	essere uno studente.}

{-- Nessuno è troppo vecchio per studiare il nemico. Io ho imparato
	dagli Scorpioni. Tu imparerai da me.}

{Mentre la mano destra dell'uomo si poggiava sullo scanner della
	serratura, Ender saltò avanti a piè pari e lo colpì con un doppio calcio
	alle reni. Malgrado la forza del rimbalzo riuscì a restare in posizione
	eretta; l'altro invece mandò un rantolo e piombò in ginocchio.}

{Il vecchio trovò la maniglia della porta e si tirò faticosamente in
	piedi, il volto contratto dal dolore. Sembrava incapace di reagire, ma
	Ender non si fidò. Tuttavia, malgrado la sua diffidenza, la velocità con
	cui l'avversario si mosse lo sorprese con la guardia abbassata. E un
	momento dopo si trovò a terra sul lato opposto della stanza, col naso e
	un labbro che perdevano sangue. Quando s'aggrappò al bordo del letto e
	si volse vide il vecchio sulla soglia, occupato a massaggiarsi le reni
	indolenzite. Sulla sua bocca c'era un sogghigno.}

{Ender sorrise di rimando. -- Maestro -- disse, -- lei ha un nome?}

{-- Mazer Rackham -- rispose lui. E scomparve nel corridoio.}

{~}

\begin{center}
	{* * *}
\end{center}

{~}

{Da quel giorno in poi Ender fu in compagnia di Mazer Rackham oppure
	solo. Il vecchio non parlava molto, ma era sempre lì: ai pasti, durante
	le lezioni, al simulatore, e sull'altro letto della sua camera la notte.
	Qualche volta Mazer lo lasciava lì, ma invariabilmente per tutto il
	tempo della sua assenza la porta restava chiusa, e nessun altro entrava
	fino al suo ritorno. Ender non la prendeva molto docilmente, e un giorno
	cominciò a chiamarlo Carceriere Rackham. Il vecchio rispondeva però al
	soprannome senza batter ciglio, né più né meno che se fosse stato il suo
	nome di battesimo, e dopo una settimana Ender ci rinunciò.}

{C'erano anche i lati positivi. Mazer gli mostrò i filmati delle vecchie
	battaglie della Prima Invasione, e la disastrosa disfatta della F.I.
	durante la Seconda. Non erano frammenti tolti dai telegiornali
	censurati, ma registrazioni complete. Poiché le battaglie più importanti
	erano state riprese da molti operatori, studiarono la strategia e la
	tattica degli Scorpioni da diverse angolazioni. E per la prima volta in
	vita sua Ender ebbe un insegnante capace di mostrargli particolari che
	da solo non avrebbe saputo notare. Per la prima volta aveva trovato una
	mente e una personalità che sentiva di poter ammirare.}

{-- Perché lei non è invecchiato e morto come tutti? -- gli chiese
	Ender. -- Sono trascorsi settant'anni dalla guerra, eppure lei non passa
	di molto la sessantina.}

{-- I miracoli della relatività -- disse Mazer. -- Vent'anni dopo la
	fine della guerra mi mandarono qui, anche se li avevo pregati e
	scongiurati di darmi il comando di una delle astronavi lanciate contro
	il pianeta natale degli Scorpioni e le loro colonie. Poi\ldots{} si
	resero conto di alcune cose circa il comportamento dei militari nello
	stress della battaglia.}

{-- Quali cose?}

{-- Non ti hanno insegnato abbastanza psicologia perché tu possa capire.
	Basti dire questo: il Comando constatò che non avrei potuto comunque
	comandare l'attacco della flotta, per il semplice motivo che sarei morto
	di vecchiaia qui su Eros prima del suo arrivo. E tuttavia io ero la sola
	persona vivente capace di capire e prevenire il comportamento degli
	Scorpioni. Ero, così si dissero, l'unico ad averli sconfitti con
	l'intelligenza, piuttosto che grazie a circostanze fortunate. E avevano
	bisogno che io fossi stato qui quando si fosse trattato di addestrare la
	persona destinata a comandare la flotta.}

{-- Così l'hanno imbarcata su un'astronave, spedendola via a velocità
	relativistica, e\ldots{}}

{-- E al termine di quel giro tornai a casa. Un viaggio disgustosamente
	noioso, Ender. Per la Terra io passai cinquant'anni nello spazio. Per me
	gli anni furono solo otto, ma mi parvero ottocento. E tutto perché
	potessi prendere a calci qualcuno abbastanza da farne il nostro futuro
	comandante.}

{-- Dovrò essere io quell'uomo, allora?}

{-- Diciamo che al momento sei la nostra punta di diamante.}

{-- Ci sono altri che si stanno preparando?}

{-- No.}

{-- Questo fa di me la sola carta da giocare. Possibile?}

{Mazer scrollò le spalle.}

{-- Perché io solo? Lei ha già vinto una volta.}

{-- Io non posso assumere il comando, per ragioni diverse e comunque
	sufficienti.}

{-- Maestro, mi faccia vedere in che modo ha sconfitto gli Scorpioni.}

{La faccia di Mazer divenne imperscrutabile.}

{-- Mi ha già mostrato tutte le altre battaglie almeno sette volte. Ho
	visto com'è possibile contrastare il modo in cui gli Scorpioni
	combattevano \emph{in passato}; ma lei non mi ha ancora detto una parola
	sulla tattica che usò per sconfiggerli nell'ultima battaglia.}

{-- Quelle registrazioni video sono top secret, Ender.}

{-- Lo so. Ho messo insieme pezzi e bocconi di quelle rese pubbliche:
	lei con la sua piccola flotta di riserva, l'avvicinarsi della loro
	enorme formazione, quelle colossali navi panciute da cui schizzavano
	fuori sciami di astrocaccia, poi la nostra ammiraglia che colpiva una
	delle loro, e un'esplosione. Qui la ripresa s'interrompe. Tutte le
	successive mostrano i nostri che si aggirano nei meandri delle loro
	astronavi, trovando Scorpioni già morti dappertutto.}

{Mazer sogghignò. -- Già anche troppo. Ma queste scene avevano avuto
	centinaia di testimoni diretti e censurarle era inutile. Bene\ldots{}
	diamo un'occhiata a quelle che furono tagliate, coraggio.}

{Poco più tardi, quando furono soli in videoteca e Mazer ebbe chiuso
	ermeticamente la porta poggiando una mano sullo scanner, Ender lo vide
	inserire nel computer una complessa chiave in codice. -- Ecco qua.
	Osserva pure, ragazzo.}

{Ciò che passò sullo schermo era esattamente la stessa sequenza che
	Ender aveva rimesso insieme. L'ammiraglia di Mazer avanzò con coraggio
	suicida verso il cuore della formazione nemica, riuscì a colpire una
	loro astronave, e poi\ldots{}}

{Niente. L'incrociatore di Mazer proseguì sulla stessa rotta, investito
	dalla nube di frammenti e radiazioni della nave esplosa. Ma non un
	raggio né un missile gli venne indirizzato addosso. La flotta nemica
	parve continuare come per forza d'inerzia, all'esterno degli anelli di
	Saturno. Dopo venti minuti due delle loro navi si urtarono ed esplosero;
	una collisione assurda che perfino un pilota ubriaco avrebbe evitato. E
	a parte le stelle che scorrevano sullo sfondo della formazione non ci
	furono altri movimenti.}

{Mazer accelerò lo scorrimento del filmato, fermandolo a tratti. --
	Aspettammo per tre ore -- disse. -- Nessuno voleva crederci. -- Poi si
	videro le astronavi della F.I. accostare quelle degli Scorpioni. I
	marines cominciarono le operazioni di aggancio e di abbordaggio. A
	questo seguirono le riprese che mostravano gli Scorpioni morti ai loro
	posti.}

{-- E così ora sai -- disse Mazer, -- che avevi già visto tutto ciò che
	c'era da vedere.}

{-- Cos'è successo?}

{-- Nessuno lo sa. Io ho la mia opinione personale. Ma fin troppi
	studiosi mi hanno fatto notare che non sono qualificato abbastanza da
	avere delle opinioni.}

{-- Lei è quello che ha vinto la battaglia.}

{-- Credevo che questo mi desse anche il diritto di commentarla, ma tu
	sai come vanno le cose. Gli xenobiologi e xenopsicologi non possono
	accettare, in coscienza, l'idea che un rozzo comandante d'astronave gli
	insegni il mestiere. E penso che gli esperti convocati dalla F.I.
	finirono con l'odiarmi, perché dopo aver visionato queste scene
	dovettero trascorrere il resto della vita qui su Eros. Misure di
	sicurezza, come sai. Non vissero felici e contenti.}

{-- Mi dica la sua opinione.}

{-- Gli Scorpioni non parlano. Pensano insieme, ed è una cosa
	istantanea, come l'effetto filotico. Come l'ansible. Ma molta gente
	suppone che questo significhi solo una comunicazione codificata, come un
	linguaggio: io invio un pensiero a te, e tu trasmetti la risposta a me.
	Però non ho mai creduto che la cosa funzionasse così. Il loro modo di
	darsi risposte era troppo \emph{immediato.} Hai visto i filmati. Fra
	loro non poteva esserci nessuno scambio di venute per stabilire il corso
	di un'azione. Ogni nave agiva come parte di un organismo singolo. Erano
	sempre coordinate come il tuo corpo quando stai lottando: parti diverse,
	ciascuna con il suo automatismo, collegate da un unico pensiero. Loro
	non hanno conversazione mentale, come fra due o più creature dai diversi
	processi psichici. Tutti i loro pensieri sono presenti, insieme e nello
	stesso istante.}

{-- Un unico individuo, di cui ogni Scorpione è come una mano o un
	piede?}

{-- Sì. Non fui io il primo a suggerirlo, ma a crederci davvero eravamo
	in pochi. Sembrava un'idea così semplicistica che gli xenobiologi furono
	molto cortesi e pazienti, dopo la battaglia, quando dovettero spiegarmi
	perché non poteva funzionare. Ma gli Scorpioni sono \emph{insetti.} Come
	le api e le formiche da cui sembrano discendere: lo sciame, la regina,
	gli operai, i combattenti. Questa organizzazione l'avevano forse un
	milione di anni fa, ma è così che cominciarono, con uno schema sociale
	dal funzionamento perfetto. Ed è accertato che nessuno degli Scorpioni
	da noi sezionato aveva i mezzi per riprodursi. Dunque, quando evolsero
	la capacità di pensare insieme, perché non avrebbero dovuto tenere la
	loro efficientissima regina? Perché non avrebbero dovuto continuare ad
	accentrarsi intorno a questa meravigliosa macchina vivente? Perché
	avrebbero dovuto cambiare?}

{-- Così è la regina che controlla l'intero gruppo.}

{-- Ce n'è la prova. Non è una prova che tutti possano accettare, perché
	nella Prima Invasione non c'era una regina. Ma quella era una missione
	esplorativa. La Seconda Invasione invece doveva impiantare una nuova
	colonia; costruire un alveare, o qualcosa di analogo.}

{-- E perciò si portarono dietro una regina.}

{-- Vediamo i filmati della Seconda Invasione, quando distrussero la
	nostra flotta fuori dal sistema solare. -- Mazer fece apparire le
	immagini sullo schermo e gli indicò la formazione nemica. -- Mostrami la
	nave della regina.}

{Era problematico. Per un bel po' Ender non riuscì a distinguerla. Le
	astronavi degli Scorpioni continuavano a spostarsi secondo schemi
	complessi, nei quali non c'era un centro evidente né una nave con palesi
	caratteristiche da ammiraglia. Ma pian piano, mentre Mazer continuava a
	far scorrere le stesse immagini, Ender cominciò a intuire che ogni
	spostamento era focalizzato attorno a un punto che risultava il meglio
	difeso. Il punto mutava posizione in modo ingannevole, tuttavia
	concentrando l'attenzione su di esso si capiva che quella nave
	particolare continuava ad essere l'occhio della flotta. Ender la
	indicò.}

{-- L'hai vista! -- esclamò Mazer. -- L'hai vista! E con te fanno due,
	fra tutti quelli che hanno esaminato i film, che sono riusciti a
	identificarla. Gli Scorpioni la mimetizzano con un'astuzia maledetta.}

{-- Riescono a farla manovrare come fosse una nave qualsiasi.}

{-- Ma non dimenticano un istante che quello è il loro punto debole.}

{-- Lei ha ragione. Quella è la nave della regina. In tal caso, però,
	vien da pensare che quando lei la prese di mira le altre avrebbero
	dovuto focalizzare a sua difesa tutto il loro potere distruttivo.
	Avrebbero dovuto scaraventarvi fuori dallo spazio.}

{-- Lo so. Ed è questo che ancora non capisco. Non è che non tentassero
	di fermarmi: fra raggi e missili me ne inviarono addosso una gragnuola.
	Ma fu come se non riuscissero a capacitarsi, finché non fu troppo tardi,
	che io avrei veramente \emph{ucciso} la regina. Forse nel loro mondo è
	inconcepibile che una regina possa essere aggredita, o catturata, o
	addirittura soltanto infastidita. Io feci qualcosa che loro si
	aspettavano dal nemico in via soltanto teorica, irreale.}

{-- E quando lei morì, tutti gli altri morirono.}

{-- Niente affatto. Morì solo la loro capacità di pensare. Sulla prima
	nave di cui salimmo a bordo gli Scorpioni erano ancora vivi.
	Organicamente. Ma non si muovevano, non rispondevano a nessuno stimolo,
	non reagirono neppure quando i nostri scienziati ne presero alcuni per
	vivisezionarli. Dopo qualche ora morirono tutti. Niente più volontà. Non
	resta niente in quei loro cervelli, quando il contatto con la regina è
	troncato.}

{-- Perché lei non è stato creduto?}

{-- Perché non trovammo nessuna regina.}

{-- Se era finita in pezzi\ldots{}}

{-- Fortune e sfortune della guerra. Io capisco chi agisce senza tener
	presente la mia opinione. La sopravvivenza impone certi comportamenti, e
	le ipotesi biologiche vengono in seconda linea. Ma altri si sono
	avvicinati alle mie deduzioni. Non si può vivere su questo asteroide
	senza sbattere la faccia su certe prove.}

{-- Che prove possono mai esserci, qui su Eros?}

{-- Ender, guardati attorno. Non sono stati gli esseri umani a scavare
	questo posto. Noi abbiamo bisogno di soffitti più alti, per dirne una.
	Questo era l'avamposto degli Scorpioni. E l'affitto ci è costato caro.
	Oltre mille marines morirono per ripulirlo stanza dopo stanza, tunnel
	dopo tunnel. Gli scorpioni ne difesero ferocemente ogni palmo.}

{Ora Ender capiva perché le dimensioni dei locali gli erano apparse
	sbagliate. -- Sentivo che questo posto aveva qualcosa d'inumano.}

{-- Per noi fu lo scrigno del tesoro. Se avessero sospettato che il loro
	tentativo si sarebbe concluso con la nostra vittoria, probabilmente non
	avrebbero mai costruito e attrezzato questo posto. Noi imparammo a
	manipolare la gravità perché qui trovammo apparecchiature capaci di
	controllarla. Imparammo a sfruttare appieno l'energia solare perché
	furono loro a oscurare questo pianetino. E in realtà fu proprio questo
	che ci consentì di scoprirli: in un periodo di tre giorni Eros scomparve
	gradualmente da tutti i telescopi. Mandammo un rimorchiatore a indagare,
	e subito se ne capì il motivo. Le telecamere trasmisero alla Terra
	dozzine di scene, incluse quelle che accaddero quando gli Scorpioni
	abbordarono il rimorchiatore e fecero a pezzi l'equipaggio. Continuarono
	a funzionare mentre gli invasori esaminavano macchine e locali, e si
	spensero soltanto quando smantellarono infine tutto quanto. Il loro fu
	uno sbaglio\ldots{} non possedevano attrezzature studiate per
	trasmettere segnali, e una volta morto l'equipaggio certo non
	sospettarono che qualcuno continuava a osservarli.}

{-- Perché sterminarono l'equipaggio?}

{-- E perché dovevano avere scrupoli? Per loro, perdere alcuni membri
	del gruppo è come per noi tagliarci le unghie. Niente di tragico e di
	immorale. Probabilmente pensarono che ammazzandoli non facevano altro
	che interrompere le loro comunicazioni con noi. Non eliminavano un
	essere senziente, indipendente, con un suo personale diritto al futuro e
	alla procreazione. L'assassinio deve avere scarso peso per loro.
	Soltanto l'uccisione di una regina è un delitto, perché eliminandola si
	annienta sia l'alveare che il suo intero bagaglio genetico.}

{-- Così si può dire che loro non sanno quello che fanno.}

{-- Non cercargli delle scuse, Ender. Solo perché non capiscono cosa
	significa uccidere essere umani, non vuol dire che siano degli
	innocentini. Abbiamo il sacrosanto diritto di difenderci con ogni mezzo,
	e la sola difesa buona è di annientarli prima che annientino noi. Così
	tu la devi pensare. In tutti gli scontri che abbiamo avuto, loro hanno
	ucciso migliaia e migliaia di esseri viventi. E quel che noi abbiamo
	fatto in due guerre è stato di ammazzarne uno solo.}

{-- Se lei non avesse eliminato la regina, Mazer, avremmo perso la
	guerra?}

{-- Direi che avremmo avuto non più di tre o quattro probabilità su
	dieci. Sono ancora convinto che avrei ridotto molto male quella flotta,
	prima che spazzassero via le mie navi. Loro avevano maggiore
	coordinazione tattica e un'enorme potenza di fuoco, ma anche noi avevamo
	qualche vantaggio. Ogni nostra nave, ogni piccolo astrocaccia, contiene
	un essere umano che pensa col suo cervello. Ognuno di noi può trovare
	soluzioni diverse e originali allo stesso problema tattico. Loro possono
	intervenire con una sola soluzione alla volta. Gli Scorpioni pensano in
	fretta, ma è il pensiero di un'unica creatura contro molte. Anche quando
	alcuni dei nostri comandanti, durante la Prima Invasione, persero delle
	battaglie a causa dell'indecisione e della stupidità, molti dei loro
	subordinati riuscirono a infliggere grosse perdite alla flotta degli
	Scorpioni.}

{-- E quando la nostra flotta d'attacco arriverà sui bersagli previsti?
	Attaccheremo ancora le loro regine?}

{-- Gli Scorpioni non sono certi arrivati ai viaggi interstellari grazie
	alla loro stupidità. Ci sono strategie che funzionano soltanto una
	volta. Io sono portato a credere che non riusciremo mai ad avvicinarci a
	una regina, a meno che non si arrivi ad attaccarla sul suo pianeta
	natale. Dopotutto una regina non è tenuta a \emph{partecipare} alla
	battaglia spaziale che loro attuano. Basta che sia presente nelle menti
	dei membri del suo alveare. La Seconda Invasione portava una colonia;
	una regina veniva a popolare la Terra. Ma questa volta\ldots{} no, non
	funzionerebbe. Dovremo batterli, una flotta dopo l'altra. E poiché
	possono attingere alle risorse di dozzine di sistemi solari, scommetto
	che in ogni battaglia loro saranno molto più numerosi di noi.}

{Ender ripensò al giorno in cui aveva dovuto affrontare due orde
	contemporaneamente. \emph{E li ho accusati di aver imbrogliato. Quando
		si verrà alla guerra vera, ogni battaglia sarà come quella.}}

{-- Un paio di vantaggi li avremo, Ender. Non ci sarà bisogno di mirare
	con gran precisione. Le nostre armi hanno vasta capacità distruttiva.}

{-- Non useremo i missili a testata atomica, come in passato?}

{-- Il Dr. Device è molto più potente. Le armi nucleari potevano essere
	sperimentate sulla superficie terrestre; questo sarebbe impossibile con
	il nostro Little Doc. Mi sarebbe piaciuto averne uno durante la Seconda
	Invasione.}

{-- Come funziona?}

{-- Io non ne capisco abbastanza da costruirne uno. Comunque si tratta
	di due raggi convergenti, al cui punto focale si crea un campo nel quale
	le molecole perdono la forza di coesione. L'energia infratomica si
	inverte. Quanto ne sai di fisica, a questo livello?}

{-- Per di più ci occupiamo di astrofisica, in classe. Ma ne so
	abbastanza da capire il concetto.}

{-- Il campo si dilata in uno sferoide vastissimo, ma infine si
	indebolisce e si ferma. Salvo quando arriva a contatto di un sostanziale
	ammasso di molecole, e in questo caso continua a espandersi con enorme
	violenza. Più grossa è l'astronave colpita, più il campo di inversione
	energetica si allarga.}

{-- Così, ogni volta che va a contatto di un'altra nave, da essa si
	espande un nuovo sferoide\ldots{}}

{-- E se le navi nemiche sono abbastanza vicine, si crea una catena che
	le spazza via tutte. Poi il campo s'indebolisce e scompare, le molecole
	disgregate cominciano a riunirsi di nuovo, e dove prima c'era
	un'astronave adesso hai una nuvola di molecole, prevalentemente di
	ferro, carbonio, ossigeno e idrogeno. Niente radioattività, niente
	detriti. Soltanto un pulviscolo. Nelle prime battaglie dovremmo riuscire
	a coglierli insieme, ma loro imparano in fretta. In seguito terranno una
	distanza maggiore fra una nave e l'altra.}

{-- Se il Dr. Device non è un missile, non può inseguire il nemico nelle
	sue evoluzioni -- rifletté Ender, pensando al simulatore.}

{-- Esatto. Nella battaglia spaziale pura e semplice il missile è
	un'arma ormai superata. Non dimentichiamo però che anche loro hanno
	imparato qualcosa da noi. Come creare lo scudo energetico, ad esempio.}

{-- Little Doc può penetrare lo scudo?}

{-- Come se non ci fosse neppure. Non è possibile \emph{vedere}
	attraverso lo scudo di energia per mettere a fuoco i raggi sul
	bersaglio, ma poiché il generatore si trova al suo centro esatto è
	facile calcolarne la posizione.}

{-- Perché non sono stato addestrato a questi armamenti?}

{-- Lo sei stato. Abbiamo fatto sempre in modo che il simulatore ti
	portasse a situazioni simili. Il tuo lavoro è di delineare la strategia,
	quindi la tattica per andare sul bersaglio. E il computer di
	un'astronave è molto più abile di te a dirigere su di esso Little Doc.}

{-- Il dottor Device. Perché questo nome?}

{-- Il prototipo era stato chiamato Molecular Detachment Device. M.D.
	Device.}

{Ender ancora non capiva.}

{-- M.D. sono le iniziali di Medical Doctor. Di conseguenza ecco il Dr.
	Device, o Little Doc, anche. Tanto per scherzarci sopra. -- Ma Ender
	continuò a non vedere nulla di divertente nella cosa.}

{~}

\begin{center}
	{* * *}
\end{center}

{~}

{Avevano modificato il simulatore. Ender poteva ancora controllare la
	prospettiva ottica e i dettagli visuali del campo olografico, ma non
	c'erano più i comandi delle astronavi. Al loro posto erano stati messi
	dei nuovi pannelli, oltre a una cuffia fornita di visore ottico,
	auricolari e un piccolo microfono.}

{Il tecnico che li aveva attesi in sala gli spiegò in breve come
	indossare la cuffia.}

{-- Ma come potrò controllare le astronavi? -- domandò Ender.}

{Mazer glielo spiegò: non avrebbe più controllato direttamente nessuna
	nave. -- Hai raggiunto un'altra fase del tuo addestramento. Hai fatto
	esperienze strategiche a ogni livello, ma ora è tempo che ti concentri
	sul comando di un'intera flotta. Come alla Scuola di Guerra lavoravi con
	i tuoi capibranco, adesso dovrai condurre dei comandanti di squadrone.
	Ti sono stati assegnati tre dozzine di futuri ufficiali da addestrare.
	Devi insegnare loro i tuoi accorgimenti tattici, devi costringerli a
	usare al meglio le loro capacità e a riconoscere i loro punti deboli,
	devi fare di loro un unico gruppo affiatato.}

{-- Quando arriveranno qui?}

{-- Ciascuno è già stato presentato al proprio simulatore. Parlerai con
	loro in cuffia. I nuovi comandi sui tuoi pannelli ti danno il modo di
	osservare dalla stessa prospettiva di ognuno dei tuoi comandanti di
	squadrone. Questo imiterà più da vicino le condizioni che incontrerai in
	una vera battaglia, dove ti limiterai a supervisionare l'azione di ogni
	singola astronave.}

{-- Come posso lavorare con dei comandanti di squadrone che non ho modo
	di vedere?}

{-- E perché dovresti aver bisogno di vederli?}

{-- Per sapere chi sono, come pensano\ldots{}}

{-- Saprai chi sono e come pensano dal modo in cui lavoreranno con il
	simulatore. Ma non credo che dovrai preoccuparti troppo di questo. Ti
	stanno ascoltando, proprio adesso. Metti la cuffia e collegati col primo
	della serie.}

{Ender si aggiustò la cuffia sulla testa.}

{-- Salaam -- sussurrò una voce nei suoi orecchi.}

{-- Alai! -- esclamò Ender.}

{-- E anch'io, il tuo fagiolino.}

{-- Bean!}

{E poi risposero Petra, e Dink, e Tom il Matto, Shen, Zuppa Cinese,
	Mosca Molo, e via via tutti i migliori allievi di cui Ender era stato
	compagno d'orda o avversario, tutti ragazzi che alla Scuola di Guerra
	aveva imparato a stimare. -- Non sapevo che foste qui -- disse. -- Non
	mi hanno detto che stavate arrivando.}

{-- Ci hanno già fatto sudare su questi simulatori per tre mesi -- disse
	Dink.}

{-- Ti accorgerai che sono ancora la pistola più veloce della scuola --
	disse Petra. -- Dink ci prova ancora con me, ma non la spunta.}

{Così cominciarono a lavorare insieme, ogni comandante di squadrone alla
	direzione del suo gruppo di astronavi, e Ender a coordinare l'insieme.
	Appresero a collaborare a diversi livelli, poiché il computer forniva
	loro diverse situazioni da risolvere. Talvolta il simulatore dava loro
	una flotta più numerosa, e Ender li suddivideva in tre o quattro branchi
	ciascuno dei quali composto da tre o quattro squadroni. Talvolta il
	simulatore dava loro un incrociatore leggero con i suoi dodici
	astrocaccia, e lui sceglieva tre comandanti assegnando a ognuno quattro
	dei piccoli e veloci apparecchi.}

{Era un piacere, ed era un gioco. Il nemico, controllato dal computer,
	era potente ma non troppo brillante, ed essi vincevano sempre a dispetto
	dei loro errori di valutazione o della scarsa intesa. Ma dopo tre
	settimane Ender era giunto a conoscerli molto più a fondo: Dink, abile
	esecutore di ordini però lento a improvvisare; Bean, in difficoltà nel
	controllare contemporaneamente molte navi ma capace di manovrare in modo
	micidiale il suo squadrone, velocissimo a reagire in ogni situazione
	insolita proposta dal computer; Alai, che come abilità strategica gli
	stava alla pari e poteva occuparsi di metà della flotta senza troppo
	bisogno di istruzioni.}

{Più Ender li metteva alla prova e più si rendeva conto dei loro
	difetti, il che lo aiutava a valorizzarne al meglio le doti. Le sedute
	di addestramento cominciavano con il simulatore che presentava una certa
	situazione bellica nel campo olografico. Ender prendeva subito atto
	della consistenza della sua flotta e di come stava manovrando quella
	nemica. Pochi minuti gli bastavano poi per sgranare ordini ai comandanti
	di squadrone, assegnando a chi una nave e a chi gruppi di navi, ciascuno
	con istruzioni generiche o particolareggiate sui compiti da svolgere.
	Mentre si sviluppava la battaglia poteva quindi balzare dall'uno
	all'altro dei punti di vista dei suoi uomini, chiedendo e dando
	suggerimenti, o modificando la tattica in caso di necessità. Poiché gli
	altri osservavano la situazione soltanto dal loro posto di
	combattimento, spesso si sentivano dare ordini che non erano in grado di
	capire appieno, ma anch'essi imparavano a fidarsi della sua direzione
	strategica. Se Ender diceva loro di ritirarsi, si ritiravano, rendendosi
	conto o d'essere pericolosamente isolati oppure che la manovra avrebbe
	convinto il nemico a osare più di quanto poteva permettersi. Quando
	qualcuno agiva di sua iniziativa e non si sentiva subito arrivare
	consigli e ordini, sapeva che la manovra aveva l'approvazione di Ender e
	che il suo silenzio era un invito a darci dentro a fondo. Ognuno sapeva
	che se le sue capacità personali fossero state inadatte alla situazione
	in cui era, Ender non lo avrebbe scelto per risolverla.}

{La fiducia reciproca era completa, la loro flotta si comportava in modo
	deciso e responsabile. E alla fine delle prime tre settimane Mazer
	mostrò a Ender una rielaborazione delle loro più recenti battaglie, con
	la differenza che stavolta erano osservate dal punto di vista del
	nemico.}

{-- Così è come vi vedono quando li attaccate. Come giudichi la tua
	flotta? La vedi veloce e coordinata?}

{-- Direi di sì. Proprio come una flotta degli Scorpioni.}

{-- Infatti qui siete arrivati a eguagliarli. Reagite con la loro stessa
	rapidità. E ora qui\ldots{} guarda questo.}

{Ender studiò i suoi squadroni che filavano contro obiettivi diversi,
	ciascuno nella sua situazione particolare, tutti inseriti nella
	strategia generale preordinata da Ender, ma chi osando di più, chi di
	meno, chi improvvisando varianti, e ognuno capace di agire con
	un'iniziativa personale sconosciuta alle astronavi degli Scorpioni.}

{-- La mente-alveare degli Scorpioni è abilissima, ma può concentrarsi
	su una sola cosa alla volta. I tuoi squadroni si dedicano invece a vari
	obiettivi contemporaneamente, e la loro manovra è coordinata da un
	cervello sveglio. Vedi dunque che qui tu hai un vantaggio. Armamento
	superiore, anche se non di troppo, identica velocità di manovra e un
	serbatoio d'intelligenza a cui puoi attingere molto meglio. Ecco dove
	sarai superiore. L'inconveniente, invece, è che sarai sempre in netta
	inferiorità numerica, e che dopo ogni battaglia il nemico ti conoscerà
	meglio; saprà come combatti, e le sue contromisure saranno immediate.}

{Ender attese un commento conclusivo.}

{-- Da ora in poi -- disse invece Mazer, -- ricominceremo daccapo il tuo
	addestramento. Abbiamo programmato il computer per simulare il genere di
	situazioni che potremo ragionevolmente aspettarci una volta giunti a
	contatto del nemico. Come base useremo gli schemi tattici che gli
	abbiamo visto mettere in atto nella Seconda Invasione. Ma invece di
	usarli contro di te meccanicamente, al controllo della simulazione del
	nemico ci sarò io. Dapprima ti troverai in situazioni in cui ci si
	aspetta che tu vinca a mani basse. Impara da esse, perché io sarò sempre
	lì, un passo più avanti di te, a programmare maggiori difficoltà e
	tattiche più evolute per spingerti ai limiti delle tue capacità.}

{-- E anche oltre?}

{-- Il tempo stringe. Devi imparare più in fretta che puoi. Quando mi
	imbarcai per quel viaggio a velocità relativistica, in modo da poter
	esser vivo negli anni cruciali del nostro attacco, lasciai dietro di me
	mia moglie e i miei figli. Al mio ritorno era già morti da un pezzo, e
	restavano soltanto dei nipoti già della mia età. Non avremmo avuto molto
	da dirci in ogni caso. Ero stato tagliato fuori da tutto ciò che
	conoscevo e dalle persone che amavo, e fui costretto a vivere in questa
	catacomba extraterrestre senza di meglio da fare che insegnare a uno
	studente dopo l'altro\ldots{} tutti bravi ragazzi pieni di speranza.
	Anche tu sei molto promettente, come già tanti altri prima di te, e come
	loro potresti avere nella mente o nel cuore il germe del fallimento. Il
	mio compito è di scoprirlo\ldots{} distruggendoti, se dovrò farlo. E
	credimi, Ender, se tu sei nato per essere schiacciato io ti schiaccerò.}

{-- Così, non sono il primo.}

{-- Naturalmente no. Che ti aspettavi? Ma sei l'ultimo. Se non impari,
	non ci sarà tempo di cercare nessun altro. E se io spero in te è solo
	perché non c'è rimasto nessun altro in cui sperare.}

{-- E gli altri, i miei comandanti di squadrone?}

{-- Chi di loro è tagliato per sostituirti?}

{-- Alai.}

{-- Sii sincero.}

{Ender non seppe cosa rispondergli e tacque.}

{-- Io non sono un uomo felice, Ender. La razza umana non ci ha promesso
	nessuna felicità. E in cambio ci obbliga a mettere tutte le nostre
	facoltà al suo servizio. Prima per la sua sopravvivenza, poi per la sua
	sicurezza e comodità. Perciò, ragazzo, spero che durante l'addestramento
	tu non venga a seccarmi l'anima lagnandoti che non sei felice. Prendi il
	piacere che puoi nei tuoi rari momenti liberi, ma prima di questo dovrà
	venire il tuo lavoro, la tua istruzione, la tua capacità di vincere. La
	vittoria è tutto, perché senza di essa non ci sarà più niente. Solo se
	tu fossi in grado di ridarmi mia moglie e i miei figli, solo allora
	avresti il diritto di venire a lamentarti di quanto ti costa tutto
	questo.}

{-- Non sto cercando di scaricarmi di nessun peso.}

{-- Ma vorrai poterlo fare, Ender. Perché io ho intenzione di
	stritolarti nella polvere, se ci riuscirò. Ti colpirò con tutti i mezzi
	che potrò immaginare, e non avrò pietà, perché quando affronterai gli
	Scorpioni loro ti aggrediranno in modi che io \emph{non posso}
	immaginare. E hanno meno pietà e lealtà dell'insetto che ci ha costretti
	a dar loro questo nome.}

{-- Lei non può stritolarmi, Mazer.}

{-- Oh, non posso? Guarda, e perché?}

{-- Perché io sono più forte di lei.}

{Mazer sogghignò. -- Ne riparleremo quando morderai la polvere, Ender.}

{~}

\begin{center}
	{* * *}
\end{center}

{~}

{Mazer lo tirò giù dal letto molto prima del solito. L'orologio segnava
	0340 quando Ender si avviò in corridoio, stordito e insonnolito, alle
	spalle del vecchio. -- Presto a letto e presto alzato -- recitò Mazer,
	-- dell'uomo sano ne fa un malato.}

{Ma Ender non si lamentò della levataccia; aveva sognato che gli
	Scorpioni lo stavano vivisezionando. Solo che invece di tirargli fuori
	le budella gli estraevano i ricordi dal cranio con un paio di pinze,
	appendendoli poi ad asciugare come fotografie e cercando di analizzarne
	il significato. Era stato un vero e proprio delirio onirico, e non
	riuscì a scacciarlo del tutto dalla mente neppure lungo il tunnel che
	portava alla sala del simulatore. Gli Scorpioni lo tormentavano durante
	il sonno, e da sveglio Mazer non gli dava un attimo di requie. Fra gli
	uni e l'altro, le sue giornate erano un calvario. Si costrinse a
	svegliarsi un po' di più. Evidentemente Mazer lo intendeva alla lettera
	quando s'era detto deciso a schiacciarlo, perché portarlo a combattere
	ancora mezzo instupidito dal sonno era proprio il genere di sleale
	espediente che c'era da aspettarsi da lui. \emph{Be', oggi il trucco non
		funzionerà, signor mio.}}

{Sedette davanti al simulatore e appena ebbe la cuffia in testa scoprì
	che i suoi comandanti di squadrone erano già sulla breccia, in attesa.
	Il nemico non c'era ancora, cosicché li divise in due gruppi e cominciò
	una finta battaglia, limitandosi a guardare come se la cavavano lasciati
	a se stessi. All'inizio ebbero qualche incertezza, ma presto stabilirono
	tattiche precise e si batterono con decisione.}

{Poi il campo olografico del simulatore si spense, le astronavi
	scomparvero e la scena cambiò completamente. Sul lato più vicino del
	campo gli allievi poterono scorgere le forme, azzurrine nella luce
	polarizzata, di tre incrociatori terrestri, ciascuno dei quali capace di
	lanciare dodici astrocaccia. Il nemico, ovviamente conscio della loro
	presenza, aveva già formato un globo con una singola nave al centro.
	Ender non s'illudeva di certo che questa portasse a bordo una regina.
	Gli Scorpioni erano superiori per due a uno, ma s'erano raggruppati in
	una formazione insolitamente stretta. Il Dr. Device avrebbe fatto loro
	molti più danni di quel che s'aspettavano.}

{Ender scelse un incrociatore, ne fece lampeggiare l'immagine olografica
	e inserì il microfono. -- Alai, questo è tuo. Assegna Petra e Vlad agli
	astrocaccia, a tuo giudizio. -- Nominò gli altri due comandanti, poi
	distolse un astrocaccia da ciascuno dei tre incrociatori e li affidò a
	Bean. -- Gira al largo e portati sotto di loro, Bean. Se cercano di
	avvicinarti rientra immediatamente nelle nostre linee, altrimenti
	piazzati in qualche posto da cui io possa farti intervenire in fretta.
	Alai, punta dritto su di loro tenendo presso di tre gli astrocaccia. Non
	far fuoco finché non te lo dico. Cerchiamo di non farli allargare.}

{-- Non ci sono difficoltà, Ender -- disse Alai.}

{-- E perché non essere cauti ugualmente? Voglio avere il minor numero
	possibile di perdite.}

{Ender separò gli altri due incrociatori e li mandò dietro ad Alai a
	distanza di sicurezza. Bean era già fuori dal campo del simulatore, e
	durante l'avvicinamento lui continuò a contattarlo per avere la sua
	posizione. Era Alai, però, a giocare il ruolo più delicato contro un
	avversario stranamente immobile e sospettoso. La sua formazione a cuneo
	giunse a un migliaio di chilometri da quella sferica e cominciò a essere
	a portata delle armi avversarie. Ma al suo avvicinarsi le navi degli
	Scorpioni retrocedevano, come per attirarlo verso quella centrale. Alai
	deviò lateralmente, e la successiva nave nemica che si trovò a portata
	indietreggiò senza far fuoco, tornando a riassumere il suo posto dopo il
	passaggio dei terrestri.}

{Finte, ritirate, deviazioni all'esterno, ancora attacchi soltanto
	accennati, e infine Ender ordinò: -- Vai là dentro, Alai.}

{La formazione a cuneo scattò avanti, e Alai disse con calma: -- Sai che
	vogliono lasciarmi passare solo per poi circondarmi e mangiarmi vivo,
	no?}

{-- Tu limitati a ignorare la nave al centro.}

{-- Tutto quello che vuoi, boss.}

{La formazione sferica cominciò a contrarsi. Ender portò avanti il resto
	della flotta, e le astronavi nemiche si spostarono sulla circonferenza
	del loro globo schematico per far fronte ai due incrociatori in arrivo.
	-- Attaccali qui, dove si stanno concentrando! -- disse Ender.}

{-- Questo manda a gambe all'aria quattromila anni di storia militare --
	commentò Alai, mandando avanti gli astrocaccia. -- Si suppone che il
	nemico vada attaccato nel suo punto debole, non è così?}

{-- In questa battaglia è ovvio che loro non sanno ciò che possono fare
	le nostre armi. Funzionerà una volta sola, ma voglio che funzioni in
	modo spettacolare. Fuoco a volontà.}

{Alai puntò i due raggi convergenti. Il simulatore ne costruì l'effetto
	creando un globo azzurrino dove essi si toccarono: prima una nave, poi
	due, quindi una dozzina, e infine la maggior parte di quelle nemiche si
	disgregarono in vampate di pulviscolo man mano che il globo si espandeva
	in quella stretta formazione. -- State alla larga -- ordinò Ender.}

{Le astronavi sul lato opposto della formazione non furono colpite dalla
	reazione a catena, ma la loro resistenza risultò inutile e la nuova arma
	le distrusse. Bean attaccò poi l'ultima, fuggita quasi nella sua
	direzione, e la battaglia finì. Era stata molto più facile di tutte le
	loro più recenti esercitazioni.}

{Quando Ender glielo fece osservare, Mazer scosse le spalle. -- Quella
	era la simulazione della loro prima strategia di attacco. Doveva essere
	una battaglia in cui non sapevano quali fossero le nostre possibilità.
	Ora comincerai a impegnarti di più. E cerca di non fare troppo il
	gradasso quando avvisti il nemico: presto avrai pane per i tuoi denti.}

{Ender portò a dieci le ore di pratica giornaliera con i comandanti di
	squadrone, dando loro un intervallo pomeridiano di tre ore per riposare.
	Le battaglie simulate in cui Mazer supervisionava il nemico avvenivano
	ogni due o tre giorni, e com'era previsto si fecero sempre più
	difficili. Gli Scorpioni non tentarono più di accerchiare le sue navi, e
	impararono quale distanza tenere fra le loro per evitare le esplosioni a
	catena. Ogni volta c'era qualcosa di nuovo e situazioni inaspettatamente
	dure. Talvolta Ender disponeva soltanto di un piccolo incrociatore e di
	otto astrocaccia, talaltra il nemico lo costringeva a battersi in una
	cintura di asteroidi, oppure poteva capitare che gli Scorpioni lo
	lasciassero avvicinare a planetoidi fortificati che d'improvviso
	esplodevano, o a campi minati che sfuggivano ai rilevamenti e
	distruggevano alcune delle astronavi terrestri. -- Tu non puoi
	permetterti queste perdite! -- sbraitò Mazer dopo una di queste
	battaglie. -- In una vera guerra non avrai il lusso di infiniti
	rimpiazzi con cui affrontare lo scontro successivo. Avrai quello che ti
	sarai portato dietro, e nient'altro. Devi imparare a combattere senza
	inutili perdite.}

{-- Inutili è una parola dura -- replicò Ender. -- Ma non posso vincere
	una battaglia se il terrore di perdere una nave o due mi impedisce di
	affrontare ogni rischio.}

{Mazer sorrise. -- Molto bene, Ender. Stai cominciando a imparare. Ma in
	una vera campagna bellica avrai alle spalle gli alti comandi, per non
	parlare della popolazione civile che ti strillerà le stesse cose. Ora,
	se oggi ti fossi trovato di fronte un nemico molto abile, ti avrebbe
	colpito \emph{qui} annientando lo squadrone di Tom. -- Ripassarono
	insieme la registrazione della battaglia. Nell'addestramento successivo
	Ender avrebbe dovuto ripetere ogni correzione di Mazer ai suoi
	comandanti e far sì che imparassero a metterle in atto consci del loro
	significato.}

{Se avevano pensato d'essere un gruppo ben preparato, affiatato,
	soddisfatto dei risultati del proprio lavoro, quando il simulatore diede
	loro la sensazione reale d'essere uniti contro gli Scorpioni seppero che
	combattere insieme poteva essere esilarante. Era l'euforia semplice di
	chi ha un ideale alle spalle e un nemico odiato di fronte, ma questo li
	portava a cercare i limiti delle loro capacità. Già in quei giorni molti
	allievi e ufficiali di Eros affollavano i posti a sedere nelle sale dei
	simulatori, per osservarli. Ender pensò a come sarebbe stato avere i
	suoi amici lì accanto a sé, ridere con loro, vederli tesi e rigidi
	durante le azioni pericolose o soddisfatti dopo un attacco ben riuscito.
	Talora si diceva che sarebbe stata una sciocca distrazione, ma altre
	volte non poteva impedirsi di desiderarlo con tutto il cuore. Anche nei
	giorni in cui aveva oziato nel laghetto fra le colline non era mai stato
	così solo. Mazer Rackham era il suo compagno, il suo maestro, ma non era
	suo amico.}

{Tuttavia non ne fece parola. Mazer gli aveva detto che lì non c'era
	posto per la compassione, e la sua infelicità personale non significava
	niente per nessuno. Per la maggior parte del tempo non significava
	niente neppure per lui. Concentrava sul lavoro ogni sua facoltà,
	spremendo il massimo di informazioni dalle battaglie simulate, e non si
	limitava a imparare passivamente questa o quella lezione ma cercava di
	estrapolare ciò che gli Scorpioni avrebbero fatto se fossero stati più
	abili, e come lui avrebbe reagito a questi loro miglioramenti. Dentro di
	lui continuavano a svolgersi le ultime battaglie e si svolgevano già
	quelle che si aspettava nei giorni successivi, sia che dormisse o fosse
	sveglio, e metteva alla frusta i suoi comandanti di squadrone con una
	durezza che di tanto in tanto li induceva a reagire.}

{-- Devo osservare che sei un po' troppo mite con noi -- lo provocò un
	giorno Alai. -- Perché non fai mai fucilare chi non è al massimo della
	sua genialità bellica? Coccolandoci a questo modo finirai col
	rovinarci.}

{Qualcuno degli altri rise nel suo microfono. Ma l'ironia era troppo
	scoperta, e Ender si limitò a rispondere con un lungo silenzio. Infine
	decise che gli conveniva ignorare quel tipo di commenti. -- Ricominciamo
	daccapo -- ordinò. -- Stavolta senza che io sia costretto a pensare che
	qualcuno di voi ha bisogno d'essere sostituito. -- La serie di manovre
	fu ripetuta senza più errori.}

{Ma mentre il loro rispetto per le sue doti di comandante si accresceva,
	l'amicizia che li aveva uniti a lui nella vecchia Scuola di Guerra
	svaniva pian piano. Era fra loro che formavano un gruppo, era fra loro
	che si scambiavano confidenze. Ender rappresentava soltanto una fonte di
	ordini, un insegnante, una voce negli orecchi, ed era distante da loro
	come Mazer lo era da lui. E non meno esigente.}

{Questo accresceva la loro efficienza in battaglia. E aiutava Ender a
	concentrarsi sul suo lavoro.}

{Di giorno, se non altro, e la sera dopo cena, quando tornava in camera
	con gli avvenimenti del simulatore che gli scorrevano nella mente. Ma
	nel sonno erano altre le immagini da cui non sapeva liberarsi. Spesso
	rivedeva il corpo del Gigante in stato di avanzata putrefazione, ma non
	come sullo schermo del banco: era reale, solido, e torreggiava su di lui
	emanando l'orrido puzzo della carne morta. Il piccolo villaggio nato nei
	meandri della sua ossatura era adesso abitato da Scorpioni, ed essi
	salutavano il suo passaggio sollevando una chela, come gladiatori che
	onorassero il pretore romano prima di morire per il suo divertimento. In
	quei sogni non provava odio per gli Scorpioni, e anche quando capiva che
	gli stavano nascondendo la loro regina non si metteva a cercarla.
	S'allontanava svelto dal corpo del Gigante, e allorché giungeva sul
	parco dei giochi i bambini erano sempre lì, lupeschi e ghignanti. E
	avevano facce a lui ben note. Talora Peter, talaltra Bonzo, a volte
	Stilson e Bernard; ma abbastanza spesso fra quelle selvagge creature
	c'erano Alai e Shen, Dink e Petra, e non di rado la stessa Valentine;
	tuttavia nel sogno lui la gettava nel torrente come gli altri e
	aspettava che affogasse, tenendola sotto a viva forza. Fra le sue mani
	lei si divincolava, lottava per riemergere, e alla fine si abbandonava
	inerte. Lui la tirava fuori dal lago e la stendeva sulla zattera, poi
	contemplava il suo volto contratto nel rictus vacuo della morte. Allora
	gemeva e piangeva su di lei, gridando e continuando a gridare che quello
	era un gioco, un gioco, un gioco!\ldots{}}

{Poi una mano lo scuoteva, strappandolo dall'incubo. -- Stavi gridando
	nel sonno -- diceva la voce di Mazer Rackham.}

{-- Uh\ldots{} scusi -- borbottava Ender.}

{-- Non fa niente. È ora di alzarsi. Oggi c'è battaglia.}

{Il ritmo di lavoro si faceva sempre più intenso. Passarono a due
	battaglie al giorno, e Ender dovette ridurre al minimo le ore di
	addestramento. Mentre poi gli altri studiavano le registrazioni degli
	ultimi scontri simulati lui restava in silenzio a meditare sui suoi
	punti deboli, a ipotizzare quel che avrebbero potuto costargli in
	futuro. A volte era già preparato ad affrontare le innovazioni del
	nemico, a volte no.}

{-- Credo che lei stia imbrogliando -- disse un giorno a Mazer.}

{-- Io?}

{-- Lei può vedere tutte le mie sedute di preparazione, e si studia
	quello su cui sto lavorando, eh? Mi sembra stranamente pronto a
	contrastare certi miei stratagemmi.}

{-- Quello che ti trovi di fronte è per la maggior parte simulazione
	computerizzata -- replicò Mazer. -- E il computer è programmato per
	rintuzzare le tue tattiche, dopo che ne hai fatto uso una volta.}

{-- Allora è il computer più subdolo che ci sia, perché riesce a
	imbrogliare la sua stessa programmazione.}

{-- Ender, tu hai bisogno di dormire di più.}

{Ma l'insonnia cominciava a tormentarlo. Ogni notte restava sveglio più
	a lungo, per poi cadere in un sonno che non lo riposava affatto. E nel
	buio si destava spesso, senza capire se era per l'inconscio bisogno di
	ripensare da sveglio al lavoro oppure soltanto per sfuggire ai sogni.
	Era come se qualcuno dirigesse il suo sonno dall'esterno, costringendolo
	a vagare entro i suoi ricordi peggiori ed a riviverli in modo distorto
	ma realistico. Alcune delle sue notti riuscivano a essere perfino più
	reali dei giorni. Cominciò a rendersi conto che la tensione aveva un
	prezzo, e che al simulatore la sua lucidità era in ribasso. All'inizio
	di ogni battaglia c'era sempre un afflusso di adrenalina che lo
	stimolava, ma poi era tutta una discesa. E se le sue capacità mentali
	avessero avuto delle pause, si chiedeva, chi lo avrebbe notato?}

{Stava lentamente scivolando. Erano lontani i giorni in cui poteva
	vincere una battaglia perdendo soltanto pochi astrocaccia. Adesso il
	nemico riusciva a mettere in evidenza i suoi punti deboli, forzandolo
	sulla difensiva; oppure prolungava lo scontro in una sorta di guerra
	d'attrito dove la vittoria finiva per essere una questione di fortuna
	più che di abilità. E in quei casi Mazer gli faceva riesaminare la
	registrazione con una smorfia di disgusto. -- Guarda come perdi questo
	incrociatore! -- brontolava. -- E questa manovra\ldots{} volevi fare un
	favore al nemico? -- E Ender tornava alla preparazione,
	all'addestramento, sforzandosi di tenere alto almeno il morale degli
	altri. Ma non di rado gli sfuggivano rabbiose imprecazioni ai loro
	errori, in specie quando capiva che dietro di essi c'era una stanchezza
	maggiore della sua.}

{-- Stiamo facendo troppi sbagli -- disse un giorno un sussurro di Petra
	nei suoi auricolari. Era una richiesta d'aiuto.}

{-- Chi non fa, non falla -- borbottò Ender. Se la ragazza aveva bisogno
	di comprensione, non l'avrebbe avuta da lui. Il suo compito era di
	addestrarla; che cercasse i suoi amici fra gli altri allievi ufficiali.}

{Poi ci fu una battaglia che per poco non finì in un disastro. Petra
	lasciò le sue astronavi troppo lontano dall'azione, e in un momento in
	cui Ender non era con lei scoprì d'essere attaccata dalla retroguardia
	degli Scorpioni. In pochi secondi aveva perduto tutte le sue navi salvo
	due astrocaccia. Ender tornò su di lei e le ordinò di metterli su una
	rotta di fuga; la ragazza non rispose; i due astrocaccia non si mossero.
	Dieci secondi dopo una gragnuola di missili li facevano esplodere.}

{All'istante Ender si rese conto d'averla spinta all'esaurimento
	nervoso: il coraggio e la freddezza di Petra lo avevano indotto a
	utilizzarla più spesso degli altri, e in situazioni sempre fra le più
	dure. Ma non ebbe il tempo di preoccuparsi di Petra, o di sentirsi in
	colpa per ciò che le avevano fatto. Incaricò Tom il Matto di spostarsi
	per impedire alla retroguardia nemica di trasformarsi in un'ala tattica,
	e cercò di salvare il salvabile. Ma Petra aveva occupato una posizione
	chiave, e adesso la sua strategia era andata a rotoli. Se il nemico
	fosse stato soltanto un po' più rapido a sfruttare il varco creato
	dall'allontanamento di Tom, Ender avrebbe perso. Invece gli Scorpioni
	attaccarono in quel punto stando troppo vicini l'uno all'altro, e Shen
	riuscì ad annientare quell'intera formazione con una singola reazione a
	catena. Tom il Matto dovette lottare, preso fra due fuochi, e Shen fece
	rotta in suo soccorso. Un quarto d'ora dopo, quando entrambi avevano
	perduto quasi tutte le loro navi, Mosca Molo riuscì a intervenire e
	grazie a lui ottennero una faticosissima vittoria.}

{Alla fine della battaglia poté sentire Petra piangere in un sottofondo
	di voci, probabilmente già lontana dal suo microfono. -- Ditegli che mi
	dispiace\ldots{} ero stanca -- gemette la ragazza. -- Non riuscivo più a
	pensare. Non ci riuscivo. Dite a Ender che sono mortificata, ma\ldots{}}

{Petra non partecipò alla battaglia nei dieci giorni successivi, e
	quando infine tornò al lavoro non era più né svelta né salda di nervi
	come in passato. Molto di ciò che aveva fatto di lei un'ottima
	comandante di squadrone era perduto. Ender lo vide ed evitò di tenerla
	in prima linea, affidandole solo missioni ausiliarie e di copertura. La
	ragazza non si lasciò menare per il naso; sapeva quel che stava
	succedendo. Ma sapeva anche che Ender non aveva altra scelta, e si
	rassegnò.}

{Restava il fatto che aveva ceduto, e non era certo che la più fragile
	dei comandanti di squadroni. Ender lo prese come un avvertimento: non
	doveva spingere gli altri al limite delle risorse umane. Da quel
	momento, invece di sfruttare la loro abilità come parte integrante delle
	sue tattiche, avrebbe dovuto pensare a risparmiarli. Cominciò a
	sostituirli, e questo lo costrinse ad affrontare le battaglie con
	comandanti di squadrone di cui si fidava un po' meno. Ma rilassare la
	pressione su di loro significò vederla aumentare su se stesso.}

{Una notte si svegliò mugolando di dolore. Aveva in bocca il sapore del
	sangue, e il suo cuscino era bagnato, appiccicoso. Sollevò le mani,
	tremanti, e capì d'essersi morso le dita nel sonno. Il sangue continuava
	a ruscellargli giù per i polsi. -- Mazer! -- chiamò. Rakham si alzò e
	fece subito arrivare un medico.}

{Mentre la ferita gli veniva curata e bendata, Mazer lo fissò. -- Non mi
	preoccupa molto ciò che mangi, Ender, ma devi spingere
	l'auto-cannibalismo ben più oltre se vuoi essere escluso dalla Scuola
	Ufficiali.}

{-- Stavo dormendo -- disse lui. -- Ma se pensa che io sia il tipo che
	per uscire si spara in un piede, Mazer, è lei ad avere bisogno del
	medico.}

{-- Bene.}

{-- Gli altri, quelli che non ce l'hanno fatta\ldots{}}

{-- Di cosa stai parlando?}

{-- Quelli prima di me. I suoi allievi che non hanno superato
	l'addestramento. Cosa ne è successo?}

{-- È successo che non ce l'hanno fatta. Nient'altro. Credevi che gli
	avessimo sparato alla nuca? Sono finiti altrove.}

{-- Come Bonzo.}

{-- Bonzo?}

{-- L'hanno rimandato a casa.}

{-- No, non come Bonzo.}

{-- E allora cosa? Che gli è successo quando hanno fallito?}

{-- Che importanza ha questo, Ender?}

{Lui non rispose.}

{-- Nessuno di loro ha fallito a \emph{questo} punto del corso, Ender.
	Con Petra hai fatto uno sbaglio. Pian piano si riprenderà. Ma Petra è
	Petra, e tu sei tu.}

{-- Parte di lei è in me. Anche lei ha fatto di me quello che sono.}

{-- Tu non fallirai, Ender. Non così presto. Spesso hai dovuto sfangarla
	dura, ma hai sempre vinto. Dunque ancora non sai quali sono i tuoi
	limiti; ma se li avessi già raggiunti saresti molto più delicato di quel
	che m'era parso.}

{-- Sono morti?}

{-- Chi?}

{-- Quelli che hanno fallito.}

{-- No, non sono morti. Per Cristo, ragazzo, quelle che stai giocando
	sono battaglie simulate!}

{-- Credo che Bonzo sia morto. L'ho sognato l'altra notte. Ricordo lo
	sguardo che aveva quando l'ho colpito al volto con la nuca. Penso di
	avergli spinto le ossa nasali nel cervello. Il sangue gli usciva dagli
	occhi. Credo che sia morto in quel momento\ldots{}}

{-- È stato soltanto un sogno.}

{-- Mazer, non voglio continuare a sognare queste cose. Ho perfino paura
	di dormire. Sono costretto a ripensare a cose che non voglio ricordare.
	Tutto il passato mi ripassa nella testa, come se io fossi un
	registratore e qualcuno mi accendesse per tirarne fuori le cose più
	terribili della mia vita.}

{-- Possiamo anche imbottirti di tranquillanti, se è questo che chiedi.
	Mi rattrista molto che tu faccia brutti sogni. Vuoi che ti compri un
	orsacchiotto da tenere fra le braccia?}

{-- Non mi prenda in giro! -- protestò Ender. -- Ho paura che finirò per
	impazzire.}

{Il dottore aveva fissato il bendaggio e si alzò. Mazer lo ringraziò e
	attese che fosse uscito. -- È davvero questa la tua paura? -- chiese.}

{Lui ci pensò sopra e non seppe cosa rispondere. -- Nei sogni che faccio
	-- mormorò, -- non sono neppure sicuro d'essere me stesso.}

{-- I sogni strani sono una valvola di sfogo, Ender. Io ti ho messo
	sotto pressione, ed è un momento critico nella tua vita. La psiche
	reagisce alla tensione, e nient'altro. Ora non sei più un bambino, ed è
	tempo che tu la smetta di aver paura la notte.}

{-- Saggio consiglio -- annuì Ender. E decise che non avrebbe mai più
	parlato dei suoi sogni a Mazer.}

{I giorni si susseguirono, e le battaglie richiesero sempre più energia
	psicofisica, finché Ender seppe d'essere sul binario vertiginoso
	dell'autodistruzione. Cominciò ad avere forti dolori allo stomaco. Il
	dottore gli prescrisse una dieta, ma presto perse completamente
	l'appetito. -- Mangia -- gli ordinava Mazer, e lui si portava
	meccanicamente il cibo alla bocca. Se però nessuno era lì a incitarlo
	non mangiava neppure un boccone.}

{Altri due comandanti di squadrone ebbero collassi nervosi uguali a
	quello di Petra, e le responsabilità che gravavano sui rimanenti si
	appesantirono. In ogni battaglia adesso il nemico li superava per tre o
	quattro a uno; inoltre s'era fatto più svelto a ritirarsi quando era in
	pericolo, e riusciva a prolungare di molto lo scontro. Talvolta
	occorrevano ore e ore di inseguimenti stressanti prima che l'ultima nave
	nemica fosse finalmente distrutta. Ender si decise a far ruotare i
	comandanti di squadrone durante il corso di una stessa battaglia,
	mettendo ragazzi più freschi e riposati al posto di quelli che
	cominciavano a diventare tardi di riflessi.}

{-- Sai una cosa? -- gli disse una volta Bean, sostituendo Zuppa Cinese
	al comando dei suoi restanti astrocaccia. -- Questo gioco non è più
	molto divertente.}

{Poi un pomeriggio, mentre era in piena seduta di addestramento, Ender
	vide le luci offuscarsi e precipitò nel buio. Quando si risvegliò, steso
	sul pavimento, qualcuno stava dicendo che s'era spaccato un labbro e un
	sopracciglio contro il quadro dei comandi.}

{Lo portarono a letto, e per tre giorni non ebbe la forza di alzare un
	dito. Dormì quasi sempre, a tratti svegliandosi da sogni in cui
	ricordava d'aver visto delle facce, ma più che facce di persone vere gli
	erano parse maschere imperfette portate da misteriosi personaggi
	onirici. Sognò, o credette di sognare, a volte Valentine e a volte
	Peter, o i suoi amici della Scuola di Guerra, o gli Scorpioni che lo
	vivisezionavano. Una volta ebbe un sogno molto realistico in cui vide il
	colonnello Graff chino su di lui, che gli parlava dolcemente come un
	padre. Il mattino del quarto giorno aprì gli occhi e vide che nella
	stanza c'era il suo nemico, Mazer Rackham.}

{-- Sono sveglio -- disse Ender.}

{-- Così sembra -- annuì Mazer. -- Hai riposato abbastanza. Oggi c'è una
	battaglia.}

{Quando ebbe scoperto che riusciva a stare in piedi, Ender andò a
	lavorare al simulatore e vinse lo scontro. Ma quel giorno non ci fu una
	seconda battaglia, e Mazer lo mandò a letto presto. Spogliandosi era
	debole e gli tremavano le mani.}

{Durante la notte gli parve di sentire qualcuno che gli sfiorava il
	volto con leggerezza. Dita lievi e gentili, un tocco affettuoso. Sognò
	di udire delle voci.}

{-- Avrebbe potuto essere più comprensivo con lui.}

{-- La comprensione non è in programma.}

{-- Quanto crede che possa resistere? Sta cedendo.}

{-- Ce la farà. È quasi finita ormai.}

{-- Così presto?}

{-- Pochi giorni e tutto sarà concluso.}

{-- Come crede che si comporterà, nelle condizioni in cui è?}

{-- Bene. Oggi ha combattuto perfino meglio del solito.}

{Nel sogno le voci erano quelle del colonnello Graff e di Mazer Rackham.
	Ma nei sogni accadono cose strane e incredibili, e quello non faceva
	eccezione, perché poi una delle voci disse: -- Non sopporto più di
	vedere quello che gli stiamo facendo. -- E l'altra rispose: -- Lo so.
	Anch'io gli voglio bene. -- Subito dopo i due personaggi diventarono
	Valentine e Alai, che lo stavano seppellendo con palate di terra, ma il
	suo corpo crebbe fino alle dimensioni di una collina, si coprì di
	cespugli e la pioggia lo scarnificò, e come fra le costole del Gigante
	gli Scorpioni costruirono tane dentro di lui.}

{Sogni e ancora sogni. Se al mondo c'era qualcuno desideroso di dare
	riposo al suo corpo mortale, questo succedeva solo nei sogni.}

{Si svegliò, combatté un'altra battaglia e vinse. Poi tornò a letto,
	lasciò che i sogni scorressero su di lui finché fu di nuovo il momento
	di destarsi, e ancora una battaglia, ancora una vittoria, ancora una
	notte in cui sogno e realtà continuavano a confondersi. Non che questo
	gli importasse più, ormai.}

{Nessuno glielo aveva detto, ma quello che lo attendeva sarebbe stato il
	suo ultimo giorno alla Scuola Ufficiali. Quando si svegliò, Mazer
	Rackham non era in camera. Si lavò, tirò fuori un'uniforme pulita e
	attese che Mazer tornasse ad aprirgli la porta. Dieci minuti dopo,
	poiché il vecchio non si faceva vedere, tentò la maniglia. La porta si
	aprì.}

{Possibile che Mazer fosse stato così distratto da lasciarlo libero e a
	se stesso, quel mattino? Nessuno a dirgli che era l'ora di mangiare, che
	era l'ora di andare al lavoro, o che era l'ora di riposare un po'.
	Libertà. Il guaio era che non sapeva bene cosa farsene di quella novità.
	Per un momento pensò di andare a cercare i suoi comandanti di squadrone
	e parlare con loro faccia a faccia, ma non aveva idea di dove
	alloggiassero. Magari a venti chilometri da lì, per quel che ne sapeva.
	Così, dopo aver vagabondato una mezz'ora per i tunnel più frequentati
	tornò alla mensa. Fece colazione seduto allo stesso tavolo di alcuni
	marines che parlavano di sesso, argomento su cui lui aveva soltanto
	informazioni teoriche. Seccato da questa riflessione andò al simulatore
	per distrarsi un po'. Libero o non libero, non gli veniva in mente altro
	che fare un paio d'ore di pratica.}

{Quando entrò in sala la prima persona che vide fu Mazer. Con scarso
	entusiasmo Ender ubbidì al suo cenno e andò verso di lui. Poi tolse di
	tasca una pillola e la ingoiò; si sentiva poco energico e alquanto
	ottuso di mente.}

{Mazer lo fissò accigliato. -- Pensi d'essere sveglio, Ender?}

{I posti degli spettatori erano pieni di ufficiali dei due sessi, in
	divisa, e c'era anche qualche civile. Ender si domandò chi fossero, ma
	non si prese la briga di chiederlo; ben difficilmente, comunque,
	qualcuno sarebbe stato così affabile da presentarsi. Andò ai comandi del
	simulatore e sedette, preparandosi a cominciare.}

{-- Ender Wiggin -- disse Mazer Rackham, -- fammi la gentilezza di
	voltarti un momento. La battaglia di oggi necessita di qualche
	spiegazione.}

{Ender ruotò sulla poltroncina girevole e gettò un'occhiata alla gente
	seduta nella penombra. Molti avevano l'espressione scaltrita ed
	impenetrabile dei burocrati, specialmente i civili; ma fra loro vide
	Anderson. La sua presenza lo sorprese, e si chiese chi si stesse
	prendendo cura della Scuola di Guerra in sua assenza. Vide anche Graff,
	e lo sguardo dell'uomo gli ricordò momenti migliori, il lago, la villa
	che da qualche tempo nella sua memoria aveva sapore di casa.
	\emph{Portami a casa}, \emph{} disse silenziosamente a Graff. \emph{Nel
		sogno hai detto che mi volevi bene. Portami a casa.}}

{Ma Graff si limitò ad annuire; un cenno di saluto, non una promessa. E
	Anderson lo guardava come se non lo conoscesse affatto.}

{-- Per favore, Ender, un po' d'attenzione. Quello di oggi è l'esame,
	l'ultimo, con cui si conclude il tuo corso qui alla Scuola Ufficiali.
	Questi osservatori sono la commissione che valuterà il tuo grado di
	preparazione. Se preferisci che non stiano in sala, potranno esaminarti
	tramite un altro simulatore collegato.}

{-- Restino pure, prego. -- Esame finale. Dal giorno successivo forse si
	sarebbe goduto un po' di riposo.}

{-- Affinché questo sia un test probante, non come quelli che già
	conosci ma di un genere che sia una sfida alla tua abilità, la battaglia
	odierna introdurrà un nuovo elemento. Avverrà intorno a un pianeta.
	Questo influirà sulla strategia del nemico e ti costringerà a
	improvvisare. Sei pregato di concentrarti senza badare al pubblico.}

{Ender gli accennò di farsi più vicino e sottovoce chiese: -- Sono il
	primo allievo arrivato a questo punto?}

{-- Se oggi vinci, Ender, sarai il primo studente a superare questo tipo
	di esame. Più di così non sono autorizzato a dirti.}

{-- Non pretendo che lo dica. Può anche rispondermi a cenni.}

{-- Domani ti permetterò d'essere petulante e irrispettoso, ragazzo.
	Oggi, però, apprezzerei che tu badassi all'esame. Non gettare via tutto
	quello che hai fatto finora. Dunque, come pensi di agire rispetto al
	pianeta?}

{-- Dovrò considerarlo un elemento interno al campo di battaglia, non un
	obiettivo da raggiungere solo in caso di vittoria.}

{-- Vero.}

{-- Inoltre in un campo gravitazionale il consumo di carburante sarà
	maggiore, mentre si presume che il nemico potrà ottenere rifornimenti in
	orbita o soccorsi dal suolo.}

{-- Già.}

{-- Qual è l'effetto di Little Doc sulla massa di un pianeta?}

{Il volto di Mazer si fece rigido. -- Ender, gli Scorpioni non hanno
	attaccato la popolazione terrestre nelle loro due Invasioni. Devi
	decidere fino a che punto è saggio adottare una strategia che
	provocherebbe ritorsioni della stessa entità.}

{-- Il pianeta è l'unico elemento nuovo?}

{-- Ricordi forse qualche battaglia in cui io ti abbia fornito un solo
	elemento nuovo? Dai pure per scontato che oggi non sarò affatto più
	leale con te. Ho delle responsabilità verso la Flotta, e non posso
	regalare la promozione ad allievi poco affidabili. Oggi farò del mio
	meglio per mandarti a sbattere col sedere in terra. Comunque, se terrai
	a mente le possibilità dei tuoi uomini e ciò che sai degli Scorpioni,
	potrai giocare al meglio le tue carte.}

{Mazer si volse e uscì dalla sala.}

{Ender inserì il microfono. -- Siete ai vostri posti?}

{-- Tutti in riga -- confermò Bean. -- È un po' tardi per cominciare
	l'addestramento, stamattina, no?}

{Dunque non avevano detto niente ai suoi comandanti di squadrone. Ender
	si trastullò con l'idea di rivelare loro quanto fosse importante quella
	battaglia, ma decise che dar loro una preoccupazione in più non lo
	avrebbe favorito. -- Spiacente -- disse, -- non ce la facevo a levarmi
	dal letto.}

{Gli giunsero alcune risatine. Nessuno ci credeva.}

{In attesa che giungessero le immagini li fece scaldare con alcune
	manovre in un campo olografico standardizzato. Gli occorse più tempo del
	solito per schiarirsi la mente e concentrarsi sulle attività dei
	subordinati, ma dopo un poco cominciò a sentirsi pronto di riflessi e
	lucido delle decisioni. \emph{O almeno}, \emph{} disse a se stesso,
	\emph{convinto d'essere lucido. E tanto dovrà bastarmi.}}

{Il campo olografico del simulatore cancellò le immagini e si spense,
	poi ci furono delle scariche elettrostatiche. Ender attese che apparisse
	la zona prefissata per la battaglia. \emph{Cosa succederà se passo
		l'esame? Mi manderanno a un altro corso? Ancora un anno o due di
		addestramento massacrante? Ancora un anno di isolamento, di gente che mi
		torchi in questo o in quel modo, di assoluta mancanza di controllo sulla
		mia stessa vita?} Cercò di ricordare quanti anni aveva. Undici, passati.
	Ma passati da quanti anni? O da quanti giorni? Da quanto tempo non si
	preoccupava più di conoscere la data? L'ultimo compleanno gli era
	sfuggito del tutto. Nessuno lo aveva certo ricordato, salvo Valentine.}

{E con gli occhi fissi nel campo ancora vuoto del simulatore desiderò
	semplicemente alzarsi e andarsene, uscire di sala così sfacciatamente da
	costringerli a sbatterlo fuori, come Bonzo, anche con disonore. Bonzo
	almeno aveva rivisto il cielo di Cartagena. Lui si sarebbe accontentato
	della polvere di Greensboro. Vincere significava continuare, andare
	avanti. Fallire significava un biglietto di ritorno per casa sua.}

\emph{{No, non è così}}{, \emph{} si disse. \emph{Loro hanno bisogno di
		me, e se fallisco non avrò più nessuna casa a cui tornare.}}

{Ma non ne era convinto. Con la sua mente conscia lo sapeva, ma in altri
	posti più profondi, più oscuri, dubitava che quella gente avesse bisogno
	di lui. Tutta l'urgenza di Mazer, ad esempio, \emph{un altro trucco, un
		altro modo per spingermi a fare quello che vogliono.} Un'altra catena
	per legarlo, per impedirgli di riposare, di vivere, implacabilmente e
	senza requie.}

{La formazione nemica apparve, e la stanca apatia di Ender si trasformò
	bruscamente in disperazione.}

{Il nemico era superiore alle sue forze per mille a uno; l'intero campo
	del simulatore brillava di puntolini verdi. Gli Scorpioni erano
	raggruppati in una dozzina di formazioni diverse che continuavano a
	spostarsi ed a cambiare aspetto, muovendosi in schemi apparentemente
	casuali entro un'enorme area di spazio. Non vide alcuna via possibile
	per oltrepassare quello schieramento: varchi che sembravano aperti si
	chiudevano d'improvviso e ne comparivano altri, mentre formazioni che
	apparivano deboli da lì a poco s'infittivano di panciute astronavi. Il
	pianeta si trovava sul lato opposto del campo, e per quel che Ender ne
	sapeva avrebbero potuto esserci altrettante navi al di fuori della zona
	inquadrata nel simulatore.}

{In quanto alla sua flotta, essa consisteva in venti vecchi incrociatori
	della classe «Icaro», ciascuno con appena quattro Angeli Neri nella
	stiva. Conosceva bene quel tipo di nave fornita di quattro astrocaccia.
	Erano incrociatori solidi, ma antiquati e poco agili, e il loro Little
	Doc aveva una portata non superiore alla metà della versione più
	moderna. Ottanta Angeli Neri, contro almeno cinquemila o forse anche
	diecimila navi da battaglia nemiche.}

{Sentì i suoi comandanti di squadrone respirare pesantemente; poté anche
	udire, fra le file degli osservatori alle sue spalle, un'imprecazione
	soffocata. Era consolante che almeno uno degli adulti notasse che non si
	trattava di un esame molto corretto. Non che questo facesse differenza.
	La correttezza non faceva parte del gioco, era ovvio. Nessuno si
	azzardava a dargli una sia pur remota possibilità di successo.
	\emph{Tutto quello che mi hanno fatto passare, e adesso farebbero carte
		false pur di non promuovermi.}}

{Per un attimo rivide Bonzo e il suo perverso manipolo di amici, venuti
	a spaventarlo e a minacciarlo. Per convincere Bonzo a battersi da solo
	aveva fatto leva sulla sua vergogna. Ma adesso la psicologia non gli
	sarebbe servita a niente. E non poteva illudersi di sorprendere il
	nemico come aveva fatto con i ragazzi anziani, in sala di battaglia,
	perché Mazer conosceva le sue capacità dentro e fuori.}

{Gli osservatori alle sue spalle cominciarono a tossicchiare, a muoversi
	nervosamente. Qualcuno di loro doveva aver già capito che Ender non
	sapeva cosa fare.}

\emph{{Non è che me ne importi molto}}{, \emph{} pensò lui. \emph{Potete
		prendervi questa battaglia e ficcarvela dove dico io. Se non mi date
		neppure una sola misera possibilità, perché dovrei giocare?}}

{Come l'ultima volta in sala di battaglia, alla Scuola di Guerra, quando
	avevano messo due orde contro di lui.}

{E mentre l'episodio gli tornava in mente anche Bean di certo pensò a
	qualcosa di simile, perché in cuffia la sua voce disse: -- Ricordate,
	ragazzi, la porta nemica è \emph{in basso.}}

{Molo, Zuppa Cinese, Vlad, Dumper e Tom il Matto risero. Non avevano
	dimenticato neppure loro.}

{Anche Ender rise. La cosa \emph{era} divertente. Gli adulti prendevano
	i loro giochi da adulto con adulta serietà, e i ragazzi ci stavano e
	accettavano di giocarli, finché a un certo punto gli adulti passavano il
	limite, si strappavano la maschera e lasciavano indovinare che la loro
	serietà era fatta di regole abbastanza sporche. \emph{Lascia perdere,
		Mazer. Non ci tengo molto a passare il tuo esame, e non ci tengo per
		nulla a giocare con le tue regole. Se ti piace imbrogliare, lo stesso
		posso fare io. Non lascerò che la slealtà sia l'arma con cui mi
		batti\ldots{} io sarò ancora più sleale di te.}}

{Nell'ultima battaglia alla Scuola di Guerra lui aveva vinto ignorando
	il nemico, ignorando le proprie perdite; s'era mosso contro la porta del
	nemico.}

{E la porta del nemico era in basso.}

{Se infrango le regole anche qui, non mi daranno mai un posto di
	comando. Questa gente non ama stabilire dei precedenti pericolosi. Non
	mi daranno mai più un simulatore in mano. E questa sarà la mia
	vittoria.}

{In fretta sussurrò alcuni comandi nel microfono. Gli squadroni si
	raggrupparono e si strinsero in una formazione cilindrica e compatta, un
	proiettile puntato al centro della vasta massa di navi nemiche. Gli
	Scorpioni, lungi dal farsi avanti, sembrarono dargli il benvenuto, ben
	contenti di circondarlo e mostrargli che era condannato a morte ancora
	prima di cominciare a farlo a pezzi. \emph{Mazer sta almeno prendendo
		nota del fatto che in qualche modo hanno imparato a rispettarmi},
	\emph{} pensò Ender. \emph{E questo mi darà tempo.}}

{Fece muovere la sua formazione in basso, poi a destra e a sinistra,
	mostrandosi spaurito e indeciso sul da farsi ma avvicinandosi sempre più
	al pianeta nemico. Gli Scorpioni gli si addensavano attorno
	inesorabilmente, finché lo ebbero a portata dei grossi laser da
	battaglia. In quel momento la flotta di Ender sembrò esplodere in tutte
	le direzioni, come se fosse impazzita e in preda al caos. Gli ottanta
	Angeli Neri non seguirono alcuno schema tattico: cominciarono a sparare
	all'impazzata salve di missili, schizzando qua e là e cercando ognuno di
	aprirsi a caso una via di fuga nelle viscere dell'immensa formazione
	nemica.}

{Dopo qualche minuto di battaglia, tuttavia, Ender diede un altro ordine
	e una dozzina fra incrociatori e astrocaccia superstiti tornarono a
	riunirsi. Ma adesso erano al di là di uno dei più consistenti gruppi di
	navi nemiche; pur subendo perdite disastrose erano riusciti a
	oltrepassarlo, e avevano coperto più della metà della distanza che li
	separava dal pianeta.}

\emph{{Gli Scorpioni hanno aperto gli occhi, ora}}{, \emph{} pensò
	Ender. \emph{Sicuramente Mazer ha capito cosa sto per fare.}}

\emph{{O forse Mazer non può credere che io voglia farlo. Be', tanto
		meglio per me.}}

{La sua piccola flotta fece delle diversioni qua e là, evitando i laser
	che cercavano il metallo degli scafi e dando massima energia agli scudi
	per respingere i missili, mentre gli Angeli Neri fingevano qualche
	attacco per riunirsi subito dopo agli incrociatori. Le navi nemiche
	continuavano a riunirsi, e per i nove decimi sul lato esterno, come per
	tagliare fuori i terrestri da un possibile ritorno nello spazio aperto.
	\emph{Bene}, \emph{} pensò Ender. \emph{Intrappolateci pure.}}

{Mormorò un ordine nel microfono, e le astronavi terrestri accelerarono
	alla massima velocità verso la superficie del pianeta. Sia gli
	incrociatori che gli astrocaccia stavano andando alla distruzione,
	perché i loro scafi non avrebbero sopportato il surriscaldamento dopo
	l'ingresso nella stratosfera. E rallentare avrebbe significato finir
	preda dei laser da battaglia da cui l'unica difesa era la velocità di
	spostamento. Ma Ender non intendeva neppure avvicinarsi alla
	stratosfera. Fin dall'inizio di quella manovra ognuna delle sue
	astronavi stava mettendo a fuoco i raggi convergenti del suo Little Doc
	su una cosa sola: il pianeta stesso.}

{Il fuoco delle navi da battaglia che chiudevano verso di loro era
	infernale. In quell'incubo di raggi roventi come il cuore di una stella
	un incrociatore terrestre esplose, per altri due, e un quarto, tre
	astrocaccia svanirono in una nube atomica, e quindi ancora un
	incrociatore, e un altro\ldots{} era un massacro, e continuava ad
	esserci l'incognita: quante navi sarebbero sopravvissute abbastanza da
	giungere a portata di tiro? Sarebbero bastati pochi attimi, una volta
	che i due raggi dell'arma avessero potuto convergere in corrispondenza
	della superficie. \emph{Un secondo con il Dr. Device, questo è tutto ciò
		che chiedo.} Ender rifletté che forse il computer non era neppure
	equipaggiato con un programma che mostrasse le conseguenze dell'attacco
	di Little Doc a una massa planetaria. \emph{Cosa posso fare, allora?
		Dire «Bang! Siete morti»?}}

{Ender si appoggiò allo schienale della poltroncina e restò a osservare
	quel che avrebbero fatto i suoi uomini, o meglio i pochi piloti e gli
	addetti ai sistemi d'arma superstiti. C'era un solo incrociatore,
	adesso, e osservato dalla sua prospettiva il pianeta distava meno di
	cinquantamila chilometri. L'astronave filava verso di esso come una
	bomba. \emph{Sicuramente siamo a portata, ora, pensò Ender. Ci
		siamo\ldots{} i raggi sono andati a fuoco. E vediamo adesso come se la
		cava il computer.}}

{Poi la superficie verde e azzurra di quel mondo striato di nuvole, che
	occupava una buona metà del campo del simulatore, cominciò a ribollire.
	D'un tratto ci fu un'esplosione di lava ardente, che schizzò fin nello
	spazio investendo l'astronave da cui Ender osservava la scena. Era vano
	cercar d'immaginare cosa succedeva sotto le nubi di vapore, ma si vedeva
	balenare l'azzurro del campo di disgregazione molecolare. Lo sferoide
	crebbe come un'apocalittica bolla d'energia, trasformando in polvere
	inerte perfino la lava che scaturiva dalle viscere squarciate di quel
	mondo. Nubi di atomi invadevano lo spazio.}

{Nel giro di altri tre secondi il pianeta cessò di essere una cosa
	solida e divenne un globo di foschia luminosa il cui diametro aumentava
	a incredibile velocità. L'astronave terrestre fu la prima a trasformarsi
	in una sventagliata di molecole quando ne fu investita, e a quel punto
	il simulatore trasferì automaticamente la prospettiva visuale a un
	astrocaccia, probabilmente l'unico superstite degli Angeli Neri dispersi
	all'inizio dell'azione, che stava filando via nello spazio in cerca di
	salvezza. Era a circa trecentomila chilometri dal pianeta, e da lì si
	vedeva soltanto un'immagine sferica in espansione, più veloce delle navi
	degli Scorpioni, le quali tuttavia sembravano aver rinunciato ad
	allontanarsi. Da lì a poco anche l'immensa flotta fu assorbita da Little
	Doc, e uno dopo l'altro i puntini di luce che erano i loro propulsori si
	spensero, polverizzati nell'alone azzurro che li inghiottiva.}

{Soltanto al perimetro della zona mostrata dal simulatore il campo di
	disgregazione molecolare s'indebolì. Due o tre navi nemiche ne erano
	rimaste fuori, e neppure l'astrocaccia che fungeva da punto di vista ne
	fu colpito. Ma dove prima c'erano migliaia di astronavi e il pianeta che
	esse avevano protetto, non restava più nulla di concreto. La sua massa
	però non aveva cessato di esistere, e al centro di quel campo
	gravitazionale già la polvere tornava ad infittirsi: i detriti si
	riunivano, cominciavano a surriscaldarsi e a fondersi, e in qualche
	settimana di tempo in quel luogo si sarebbe formato un nuovo pianeta
	primordiale, un po' più piccolo di quello ormai svanito.}

{Ender si tolse la cuffia, nei cui auricolari cicalavano le voci dei
	suoi comandanti di squadrone, e soltanto allora si accorse che il
	pubblico seduto dietro di lui faceva un gran chiasso. Gli ufficiali in
	uniforme si stavano abbracciando l'un l'altro, gridando e ridendo;
	alcuni piangevano; altri s'erano inginocchiati a mani giunte, e
	stupefatto Ender si accorse che stavano pregando. Non riuscì a capirne
	il perché. C'era qualcosa di sbagliato. Avrebbero dovuto essere seccati
	e irritati.}

{Il colonnello Graff lasciò gli altri e si avvicinò a lui. Aveva il
	volto rigato di lacrime, ma sorrideva. Afferrò Ender per le spalle, lo
	tirò in piedi e con sua grande sorpresa lo abbracciò strettamente. --
	Grazie, Ender! -- balbettò, commosso. -- Grazie a te, e grazie a Dio,
	Ender!}

{Dietro di lui vennero subito tutti gli altri, chi per stringergli la
	mano, chi per congratularsi, e un paio di ufficialesse lo baciarono
	sulle guance con trasporto. Per qualche minuto non riuscì a trovare
	alcun senso nel loro comportamento. Forse che, dopotutto, era riuscito a
	superare l'esame? Era la \emph{sua} vittoria, non la loro, e per di più
	una vittoria di scarso significato tecnico, ottenuta con l'imbroglio.
	Perché mai agivano come se avesse vinto rispettando onorevolmente le
	regole?}

{La piccola folla si aprì e fra essi comparve Mazer Rackham. Il vecchio
	avanzò dritto su di lui e gli strinse la mano. -- Hai fatto la scelta
	più dura, ragazzo. O tutto o niente. La loro fine o la nostra. Ma Dio sa
	che non avevi altro modo di agire. Congratulazioni. Li hai battuti, e
	definitivamente distrutti.}

{Battuti. Distrutti. Ender si accigliò confuso. -- Io ho battuto
	\emph{lei.}}

{Mazer rise forte, divertito ma con una nota stridula che fece ridere
	anche gli altri. -- Ender, tu non hai mai giocato con me. Fin da quando
	io sono diventato il tuo nemico, tu \emph{non hai mai giocato} una sola
	volta.}

{Ender non capì dove stesse lo scherzo. Quel che sapeva era di aver
	sudato sangue ed innumerevoli battaglie sul simulatore, fino a rovinarsi
	la salute. Il sogghigno di Mazer cominciò a irritarlo.}

{Il vecchio allungò una mano a toccargli una spalla ma lui si scostò,
	scuro in volto. Mazer si fece serio, esitò un poco e disse: -- Ender,
	negli ultimi mesi tu sei stato il comandante delle nostre flotte
	d'attacco. Questa era la Terza Invasione. Non hai mai giocato; le
	battaglie erano vere, e il solo nemico che hai affrontato erano gli
	Scorpioni. Tu hai vinto ogni battaglia, e finalmente oggi li hai
	attaccati nel loro mondo di origine, dove si erano rifugiate le loro
	regine\ldots{} sì, tutte le loro regine, fuggite dalle colonie per
	evitare il nostro attacco, erano riunite lì e tu le hai distrutte dalla
	prima all'ultima. Non minacceranno mai più noi né nessun altro. E sei
	stato tu a fare questo. Tu.}

{Reale. Non era un gioco. Ender era troppo stordito per rendersi conto
	del significato di quelle parole. Quei puntini di luce ripresi da uno
	schermo e che il simulatore riproponeva a tre dimensioni\ldots{} non
	erano puntini di luce, erano vere astronavi, macchine possenti che lui
	aveva affrontato e distrutto. Ed era un vero pianeta quello che lui
	aveva cancellato dalla faccia dell'universo. Si avviò verso l'uscita
	evitando la gente, ignorando le loro mani e le loro frasi entusiaste,
	senza guardare in faccia nessuno. Quando fu in camera sua gettò al suolo
	i vestiti, si distese a letto e quasi subito si addormentò.}

{~}

\begin{center}
	{* * *}
\end{center}

{~}

{A svegliarlo fu una mano che lo scuoteva. Gli occorse qualche istante
	per riconoscere i due uomini. Graff e Rackham. Volse loro le spalle.
	\emph{Lasciatemi dormire.}}

{-- Ender, abbiamo bisogno di parlarti -- disse Graff.}

{Con un grugnito lui si volse a guardarli.}

{-- È tutta la notte e tutto il giorno che la nostra stazione sta
	trasmettendo alla Terra i filmati della battaglia di ieri.}

{-- Ieri? -- Doveva aver dormito quasi ventiquattr'ore.}

{-- Sei un eroe, Ender. La gente ha visto quello che avete fatto, tu e
	gli altri. Credo che non ci sia nazione che non ti abbia già conferito
	le più alte decorazioni.}

{-- Li ho uccisi tutti, non è vero? -- chiese Ender.}

{-- Tutti chi? -- Graff sbatté le palpebre. -- Gli Scorpioni? Già, pare
	di sì.}

{Mazer si piegò su di lui. -- È per questo che abbiamo fatto la guerra.}

{-- Tutte le loro regine, i piccoli. Dunque ho sterminato la loro
	razza\ldots{} ora e per sempre.}

{-- Se lo sono voluto loro, quando ci hanno attaccati. Non è certo colpa
	tua. Doveva accadere.}

{Ender afferrò Mazer per il petto dell'uniforme e vi si appese,
	costringendolo a chinarsi faccia a faccia con lui. -- Io non volevo
	ucciderli tutti. Non volevo uccidere nessuno! Non sono un killer! Voi
	non avevate bisogno di me, voialtri bastardi, ma di Peter. E invece lo
	avete fatto fare a me, con un inganno mostruoso! -- Stava piangendo e
	tremava, incapace di controllarsi.}

{-- È ovvio che ti abbiamo ingannato. Tutto era imperniato su questo --
	disse Graff. -- Doveva essere un trucco, altrimenti non l'avresti fatto.
	Eravamo prigionieri di questa constatazione. Ci occorreva un comandante
	capace di tale empatia da saper pensare come gli Scorpioni, per capirli
	e anticiparli. Capace d'immedesimarsi con loro fino ad amarli, più o
	meno consciamente, perché immedesimarsi era vitale. Ma una persona così
	sensibile non avrebbe mai potuto essere il killer che ci serviva. Mai
	sarebbe andato in battaglia deciso a vincere a tutti i costi. Se tu
	avessi saputo, non l'avresti fatto. Se tu fossi il genere d'individuo
	capace di uccidere a mente fredda, invece, ti sarebbe mancata la
	comprensione necessaria a vincere gli Scorpioni.}

{-- E doveva essere un ragazzo giovane, Ender -- aggiunse Mazer. -- Tu
	eri più veloce di me. Migliore di me. Io sono troppo vecchio e cauto. Un
	essere umano normale che sappia già cosa sia la guerra non può andare in
	battaglia con molto entusiasmo. Ma tu non lo sapevi. Abbiamo fatto di
	tutto perché tu non sapessi certe cose. Eri entusiasta e determinato,
	giovane e brillante. Ed eri nato per questo.}

{-- C'erano equipaggi umani sulle nostre navi. Non è così?}

{-- Sì.}

{-- Io ho ordinato a quei piloti di andare a morire, e non lo sapevo
	neppure\ldots{}}

{-- \emph{Loro} lo sapevano, Ender, e hanno attaccato. Sapevano per cosa
	stavano combattendo.}

{-- Non avete neanche provato a chiedermelo. Non avete mai tentato di
	dirmi una frazione della verità.}

{-- Tu dovevi essere un'arma, Ender. Come una pistola, come il Dr.
	Device, dal funzionamento perfetto ma all'oscuro del bersaglio su cui
	eri puntato. \emph{Noi} abbiamo preso la mira. Noi siamo i responsabili.
	Se c'è qualcuno che deve avere la coscienza sporca, siamo noi.}

{-- Andatevela a lavare da un'altra parte -- disse Ender. Si voltò e
	chiuse gli occhi.}

{Mazer Rackham lo scosse. -- Non è il momento di dormire. Apri gli
	orecchi, è importante.}

{-- Voialtri avete finito con me -- borbottò lui. -- Ora lasciatemi in
	pace.}

{-- Noi\ldots{} loro non l'hanno affatto finita con te -- sospirò Mazer.
	-- È questo che sto cercando di dirti. Laggiù sulla Terra sono usciti di
	cervello, stanno per dare il via a una guerra. Gli americani accusano il
	Patto di Varsavia di esser pronto ad attaccarli, e il Patto dice la
	stessa cosa dell'Egemonia. La guerra con gli Scorpioni non è finita da
	ventiquattr'ore e il mondo è già sul punto di scatenarne un'altra,
	peggiore delle precedenti. Inoltre tutti dichiarano d'essere preoccupati
	per te. E tutti quanti ti vogliono. Ogni esercito vuole alla sua testa
	il più grande comandante in campo della storia. Gli americani.
	L'Egemonia. Tutte le nazioni salvo quelle del Patto di Varsavia, le
	quali invece ti vogliono morto.}

{-- Peggio per me -- disse Ender.}

{-- Dobbiamo portarti via da qui. Eros è pieno di marines russi, perfino
	il Condottiero è russo. Potrebbe esserci uno spargimento di sangue da un
	momento all'altro.}

{Ender gli volse di nuovo le spalle. Stavolta i due non lo toccarono, ma
	la sonnolenza gli era passata. Li ascoltò parlare fra loro.}

{-- Era proprio questo che temevo, Rackham. Lei lo ha spremuto troppo.
	Alcuni dei loro avamposti avrebbero potuto aspettare. Poteva dargli
	qualche giorno di riposo.}

{-- Anche lei ci si mette, Graff? Anche lei mi taglierà i panni addosso
	col senno di poi? Non possiamo sapere cosa sarebbe successo se non ci
	fossimo impegnati in un attacco totale. Nessuno lo sa. È andata così e
	ha funzionato. Soprattutto questo: ha funzionato. Si tenga a mente
	questa giustificazione, Graff. Anche lei potrebbe vedersi costretto a
	usarla.}

{-- Mi scusi.}

{-- Riesco a capire cosa gli abbiamo fatto. Il colonnello Liki dice che
	potrebbe non riprendersi mai dal trauma, ma io non ci credo. È troppo
	forte. Vincere significa molto per lui, e ha vinto.}

{-- Non venga a parlarmi di forza. È un ragazzino di undici anni.
	Lasciamolo riposare, Rackham. La situazione non è ancora esplosa.
	Possiamo mettere un paio di sentinelle davanti alla porta.}

{-- O metterle davanti a un'altra porta, e fingere che quella sia la
	sua.}

{-- Cerchiamo il capo della sorveglianza.}

{I due uomini uscirono. Quasi subito Ender ricadde nel sonno.}

{~}

\begin{center}
	{* * *}
\end{center}

{~}

{Il tempo scivolò via attorno a Ender senza che la realtà esterna lo
	sfiorasse, salvo che per brevi e spiacevoli intervalli. Una volta si
	svegliò per qualche minuto tormentato da una dolorosa pressione nella
	carne di un braccio. Con un gemito mosse l'altra mano e si toccò: c'era
	un ago, conficcato in una sua vena. Cercò di levarselo ma le sue deboli
	dita annasparono invano sul nastro adesivo. Un'altra volta riaprì gli
	occhi nelle tenebre e sentì non distante da lui gente che mormorava e
	che imprecava. Nei suoi orecchi vibrava un rumore intenso, quello che lo
	aveva svegliato, ma non fu capace d'indentificarlo. -- Accendi un po'
	quella luce -- disse una voce sconosciuta. E un'altra volta gli parve
	che qualcuno piangesse sottovoce, accanto a lui.}

{Avrebbe potuto esser trascorso un giorno, come anche una settimana; ma
	per i sogni che fece avrebbero potuto essere dei mesi. E in quei sogni
	gli parve di vivere un'intera vita. Di nuovo affrontò il Drink del
	Gigante, i bambini licantropi, la continua violenza, l'omicidio come
	unica e continua soluzione. Nella foresta udì una voce sussurrare:
	«Dovevi uccidere quei bambini per arrivare alla Fine del Mondo». E lui
	cercò di rispondere che non voleva uccidere nessuno, e che non gli era
	mai stato chiesto se desiderava uccidere qualcuno. Ma la foresta rise di
	lui. E quando si tuffò nel burrone alla Fine del Mondo, a raccoglierlo
	non fu una nuvoletta bensì un astrocaccia che lo portò a distanza di
	sicurezza dal pianeta degli Scorpioni, in modo che potesse osservare a
	lungo, interminabilmente, la morte che ribolliva qua e là sulla
	superficie. E poi più vicino, sempre più vicino, finché poté vedere gli
	Scorpioni che si torcevano e scoppiavano, trasformandosi in polvere che
	gli roteava attorno. E la regina, circondata dai suoi piccoli, soltanto
	che la regina era Mamma ed i bambini erano Valentine e tutti quelli che
	lui aveva conosciuto alla Scuola di Guerra. Uno di loro aveva il volto
	di Bonzo, con gli occhi e il naso pieni di sangue, e diceva «Tu non hai
	onore». E come sempre il sogno finiva con uno specchio, o una superficie
	metallica, o una polla d'acqua in cui vedeva riflessa la sua faccia.
	Dapprima c'era stata solo la faccia di Peter, con la coda del serpente e
	il rivolo di sangue che gli uscivano di bocca. Nei sogni successivi
	invece vi trovò la propria faccia, vecchia e triste, con occhi entro i
	quali c'era il peso di miliardi e miliardi di delitti\ldots{} ma erano
	pur sempre i suoi occhi, e non poteva ridar loro uno sguardo luminoso e
	innocente.}

{Questo fu il mondo in cui Ender abitò e visse durante i cinque giorni
	della Guerra dei Due Blocchi.}

{Quando si risvegliò scoprì d'essere disteso nel buio. In distanza si
	udivano dei tonfi soffocati simili a esplosioni. Per un poco tese gli
	orecchi a quei rumori. Poi accanto a lui ci fu uno scalpiccio.}

{Si girò e protese le braccia, per fermare chiunque stesse cercando di
	colpirlo. Le sue mani incontrarono un vestito. Con un ansito rauco diede
	uno strattone di lato, e un corpo umano gli piombò addosso.}

{-- Ender, sono io! Sono io!}

{Riconobbe quella voce. Usciva dai suoi ricordi come da un baratro
	profondo un milione di anni.}

{-- Alai\ldots{}}

{-- Salaam, dannato pivello. Stai tentando di strangolarmi?}

{-- Sì. Credevo che tu volessi strangolare me.}

{-- Io stavo solo cercando di non svegliarti. Be', almeno ti è rimasto
	l'istinto di sopravvivenza. Da quel che dice Mazer, sei avviato a
	diventare una specie di vegetale.}

{-- Già, ci stavo provando. Cosa sono questi colpi?}

{-- Scontri armati in corso. La nostra sezione è tenuta al buio per
	misura precauzionale.}

{Ender mise fuori le gambe e si tirò a sedere, ma non ce la fece. Un
	dolore lancinante alla testa lo costrinse a stendersi di nuovo. Mandò un
	gemito.}

{-- Non cercare di alzarti, Ender. Va tutto bene. Sembra che possiamo
	vincere. Non tutte le nazioni del Patto di Varsavia si sono unite al
	Condottiero. Parecchie si sono alleate con noi, quando lo Stratega ha
	detto che tu eri fedele alla F.I.}

{-- Io stavo dormendo.}

{-- Be', ha mentito? Forse nei sogni hai complottato per tradirci? Spero
	di no. Anche molti russi stanno con noi, e hanno riferito che quando il
	Condottiero ha ordinato di trovarti e ucciderti alcuni di loro per poco
	non l'hanno ammazzato. Qualunque cosa provino per l'altra gente, Ender,
	loro ti amano. Il mondo intero ha visto le nostre battaglie, la TV le
	trasmette giorno e notte. Anch'io ne ho rivisto alcune, complete della
	tua voce che dà gli ordini. Niente censura, c'è proprio tutto. Roba
	interessante. Come attore farai molta strada.}

{-- Non credo -- disse Ender.}

{-- Stavo scherzando. Ehi, ci crederesti? Noi abbiamo vinto la guerra.
	Eravamo così impazienti di diventare adulti e di combattere, e già lo
	stavamo facendo tutto il tempo. Voglio dire, noialtri siamo dei
	ragazzini. Ma lo abbiamo fatto noi, Ender. -- Alai rise. -- Lo hai fatto
	tu, comunque. Sei stato in gamba, boss. Non so come tu abbia potuto
	portarci attraverso quell'inferno, ma lo hai fatto. Eri formidabile.}

{Ender notò quel verbo al passato. \emph{Ero formidabile.} -- E cosa
	\emph{sono} adesso, Alai?}

{-- Sempre in gamba.}

{-- Per fare cosa?}

{-- Per\ldots{} tutto. C'è un milione di soldati che ti seguirebbero
	fino ai confini dell'universo.}

{-- Io non voglio andare alla fine dell'universo.}

{-- Be', loro ti seguiranno. Dove vuoi andare?}

\emph{{Voglio andare a casa}}{, \emph{} pensò Ender. \emph{A casa. Ma
		non so dove sia.}}

{I colpi lontani tacquero.}

{-- Ascolta, c'è qualcuno -- disse Alai.}

{In corridoio si udivano dei passi. La porta si aprì, e dopo
	un'esitazione i passi entrarono nella stanza. -- È finita -- disse una
	voce. Era Bean. Come a comprovare quell'affermazione, le luci si
	accesero.}

{-- Ehi, Bean!}

{-- Come va, Ender?}

{Petra e Dink vennero dentro anch'essi, tenendosi per mano. Si fermarono
	ai piedi del letto. -- Ehi, l'eroe si è svegliato -- disse Dink.}

{-- Chi ha vinto? -- chiese Ender.}

{-- Noi, Ender -- rispose Bean. -- C'eri anche tu là.}

{-- Non è \emph{così} rimbecillito, Bean. Vuol dire chi ha vinto adesso.
	-- Petra prese una mano di Ender. -- Sulla Terra c'è una tregua. In
	realtà stavano negoziando da giorni. Finalmente si sono messi d'accordo
	sulla Proposta Locke.}

{-- Ender non può sapere della Proposta Locke.}

{-- È piuttosto complessa, ma in sintesi significa che la F.I. può
	continuare a esistere, senza che il Patto di Varsavia ne faccia parte.
	Così i marines del Patto rientreranno a casa loro. Credo che i russi si
	siano decisi a questo accordo perché le nazioni dell'Europa Orientale
	gli si stavano rivoltando contro. I morti sono stati molti, dappertutto.
	Qui almeno cinquecento, ma sulla Terra è stato abbastanza peggio.}

{-- L'Egemone si è dimesso -- disse Dink. -- Sono una manica di idioti
	laggiù. Vadano al diavolo.}

{-- Tu stai bene? -- chiese Petra, sfiorandogli la fronte. -- Eravamo
	preoccupati. Dicevano che sei diventato pazzo. Noi abbiamo risposto che
	i pazzi erano \emph{loro.}}

{-- Certo, che sono pazzo -- disse Ender. -- Ma sto meglio, credo.}

{-- Quando te ne sei accorto? -- domandò Alai.}

{-- Quando ho creduto che tu volessi ammazzarmi e ho deciso che prima ti
	avrei strangolato. Penso d'essere un killer fino in fondo all'anima.
	Però preferisco vivere che lasciarmi uccidere.}

{Gli altri sorrisero e si dissero d'accordo con lui. Poi Ender scoppiò
	in lacrime e abbracciò Bean e Petra, che erano i più vicini. -- Ho
	sentito la vostra mancanza -- ansimò. -- Avrei voluto essere con voi.}

{-- Sei sempre stato con noi -- disse Petra. Lo baciò sulle guance.}

{-- E tu sei stata magnifica -- disse Ender. -- Quelli di cui avevo più
	bisogno, li ho torchiati di più. Poco saggio da parte mia.}

{-- I ragazzi stanno benone, adesso -- lo informò Dink. -- Nulla che
	cinque giorni di letto, in una stanza oscurata e nel bel mezzo di una
	guerra, non possa curare.}

{-- Non sarò mai più il vostro comandante, eh? -- sospirò Ender. -- Non
	ho intenzione di comandare niente, d'ora in poi.}

{-- Nessuno può obbligarti -- disse Dink. -- Però tu sarai sempre il
	nostro comandante, per noi.}

{Per un poco rimasero in silenzio.}

{-- Così, che ci resta da fare, adesso? -- domandò poi Alai. -- La
	guerra con gli Scorpioni è finita, quella sulla Terra anche, e qui non
	si combatte più. Cos'altro resta da fare, per noi?}

{-- Siamo degli adolescenti -- rifletté Petra. -- Probabilmente ci
	rimanderanno a scuola. È la legge. La frequenza è obbligatoria fino a
	diciassette anni.}

{A quel pensiero tutti risero. E continuarono a ridere finché ebbero la
	voce rauca e le guance umide di lacrime.}

\phantomsection\label{Orsonux20Scottux20Cardux20-ux20Ilux20Giocoux20Diux20Enderux20-ux20BY_SLY70A1_split_017.htm}{}
