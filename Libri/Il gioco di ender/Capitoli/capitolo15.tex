\chapter{L'ARALDO DEI DEFUNTI}

{~}

{~}

{~}

{Il lago era immobile; non spirava un alito di vento. I due uomini
	occupavano un paio di sedie a sdraio affiancate sul moletto. A un anello
	rugginoso era ormeggiata una piccola zattera. Graff aveva allungato un
	piede sulla corda e ogni tanto tirava la zattera verso di sé, la
	spingeva via, poi la attirava di nuovo.}

{-- Lei è alquanto dimagrito.}

{-- Ci sono tensioni che fanno ingrassare, altre che fanno dimagrire. Io
	sono un ammasso di semplici reazioni chimiche.}

{-- Dev'essere stata dura per lei.}

{Graff scosse le spalle. -- Non poi troppo. Sapevo che sarei stato
	assolto.}

{-- Alcuni di noi non lo erano altrettanto. Maltrattamenti di minori,
	negligenza, due casi di morte violenta\ldots{} Quei filmati di Bonzo e
	di Stilson hanno fatto un brutto effetto. Vedere un ragazzo che ne
	uccide un altro\ldots{}}

{-- Se non altro, credo che abbiano salvato me. Il pubblico ministero li
	aveva tagliati, ma noi abbiamo presentato l'intera registrazione. È
	stato dimostrato che il provocatore non era Ender. Fatto ciò, si è
	trattato solo di ribadire certi concetti. Io ho affermato che lo
	consideravo necessario per la salvezza della razza umana, e ha
	funzionato; il giudice ha dichiarato che l'accusa doveva provare oltre
	ogni dubbio che Ender avrebbe vinto la guerra \emph{senza}
	l'addestramento particolare che gli abbiamo dato. Il resto è stato
	semplice. Le necessità della guerra.}

{-- Comunque sia, Graff, per noi è stato un sollievo. So che abbiamo
	dovuto costituirci anche noi come parte lesa, e che l'accusa ha usato
	nastri di nostre conversazioni contro di lei. Ma io ero già convinto che
	lei fosse nel giusto, e mi offersi di testimoniare a suo favore.}

{-- Lo so, Anderson. I miei avvocati me lo dissero.}

{-- E adesso cosa farà?}

{-- Non lo so. Per ora mi rilasso. Ho parecchi anni di stipendio
	accumulato in banca, e potrei vivere con gli interessi. Forse mi darò
	all'ozio.}

{-- Potrebbe essere un'idea. Ma io non ne sarei capace. Ho già rifiutato
	la presidenza di tre diverse università, offertami in base all'ipotesi
	che io sia un educatore. Nessuno mi crede quando dico che alla Scuola di
	Guerra tutto ciò che m'interessava erano le battaglie. Penso che
	accetterò quell'offerta di cui le dicevo.}

{-- Allenatore?}

{-- Ora che le guerre sono finite, il campionato attirerà più pubblico.
	Ma per me sarà una specie di vacanza: soltanto ventotto squadre in serie
	A. E dopo anni trascorsi a guardare ragazzi che volano e rimbalzano, il
	rugby mi farà l'effetto di un pomeriggio dedicato a contare le lumache
	in giardino.}

{I due risero. Graff sospirò e spinse via la zattera col piede.}

{-- Questo natante. Difficile che possa sostenere lei.}

{Graff scosse il capo. -- Lo ha costruito Ender.}

{-- Già, è vero, è qui che lei lo portò.}

{-- La proprietà è stata intestata a lui. Mi sono accertato che il
	governo non fosse avaro. Ha più denaro di quel che potrà mai spendere.}

{-- Sempre che gli permettano di tornare a spenderlo.}

{-- Non lo faranno.}

{-- Con Demostene che invoca il ritorno in patria dell'eroe?}

{-- Demostene ha chiuso con la videostampa.}

{Anderson inarcò un sopracciglio. -- Questo che significa?}

{-- Demostene si è ritirato. Definitivamente.}

{-- Lei sa qualcosa, eh, vecchio lupaccio? Lei sa chi è Demostene.}

{-- Chi era.}

{-- Be', me lo dica!}

{-- No.}

{-- Via, adesso lei non è più divertente, Graff!}

{-- Non lo sono mai stato.}

{-- Almeno potrebbe dirmi \emph{perché.} Molti di noi erano disposti a
	giurare che un giorno Demostene avrebbe potuto diventare Egemone.}

{-- Non esisteva nemmeno la più pallida possibilità. No, neppure tutti
	gli asini che trottano dietro Demostene potrebbero ragliare abbastanza
	da convincere l'Egemone a riportare Ender sulla Terra. Ender è troppo
	pericoloso.}

{-- Ha soltanto undici anni. Dodici, adesso.}

{-- Perciò chiunque potrebbe controllarlo facilmente, il che lo rende
	ancor più pericoloso. In ogni angolo del globo il nome di Ender può far
	muovere la gente: il Dio-Bambino, la Guida-Miracolosa, il Liberatore, Lo
	Stregone\ldots{} qualunque aspirante tiranno potrebbe metterlo alla
	testa di un esercito e avrebbe vinto prima di sparare un sol colpo. E
	qualunque uomo savio e giusto, avendo Ender dalla sua parte, lo
	sfrutterebbe per ottenere il potere assoluto. Se tornasse sulla Terra
	sarebbe per venire qui, vivere tranquillo, salvare ciò che resta della
	sua adolescenza. Ma non glielo permetterebbero mai.}

{-- Capisco. E questo è stato spiegato a Demostene?}

{Graff sorrise. -- È stato Demostene a spiegarlo a qualcun altro.
	Qualcuno che di Ender saprebbe farne l'uso più completo, per unificare
	il mondo e governarlo con mano di ferro.}

{-- Chi?}

{-- Locke.}

{-- Locke è quello che ha scritto di più sulla necessità di lasciare
	Ender su Eros.}

{-- Il che dimostra che le cose non sono mai quello che sembrano.}

{-- È troppo complicato per me, Graff. Mi dia una buona squadra, ecco la
	politica che capisco: regole chiare, arbitri onesti, e vincitori e
	perdenti che alla fine della partita si stringono la mano e se ne
	tornano a casa dalle loro donne.}

{-- Mi faccia avere qualche biglietto di tanto in tanto, d'accordo?}

{-- Non vorrà davvero starsene qui ad ammuffire, eh?}

{-- No.}

{-- Mi sembra d'aver capito che l'Egemone le ha offerto una poltrona.}

{-- Una nuova di zecca. Quella di Ministro delle Colonie.}

{-- Dunque è a questo che stanno puntando.}

{-- Appena ci arriveranno i rapporti sui mondi che erano stati
	colonizzati dagli Scorpioni. Voglio dire, sono lì che ci aspettano,
	fertili e pronti, con strade e industrie e abitazioni già edificate, e i
	loro vecchi padroni tutti morti. Assai conveniente. Potremo modificare
	le leggi sul controllo della popolazione\ldots{}}

{-- Che tutti odiano.}

{-- \ldots{} e tutti i Terzi e i Quarti e i Quinti avranno astronavi per
	cercare il loro destino su mondi conosciuti e sconosciuti.}

{-- Crede che la gente ci andrà?}

{-- La gente ci prova sempre. Sempre. Niente può togliere dalla testa a
	un uomo che forse su un altro mondo può trovare una vita migliore.}

{-- All'inferno, magari è davvero così!}

{~}

\begin{center}
	{* * *}
\end{center}

{~}

{Nei primi tempi Ender aveva creduto che lo avrebbero riportato sulla
	Terra, non appena la situazione si fosse stabilizzata. Ma le cose si
	erano stabilizzate da un pezzo, da un anno ormai, e adesso cominciava a
	capire che nessuno aveva interesse a farlo, e che se la sua immagine
	pubblica poteva essere utilmente usata la sua presenza in carne e ossa
	sarebbe stata soltanto una seccatura per tutti.}

{Aveva potuto farsi un'idea di come andavano le cose già durante il
	processo intentato contro il colonnello Graff. L'ammiraglio Chamrajnagar
	aveva cercato d'impedirgli di assistere alle udienze, quasi tutte
	teletrasmesse, ma non c'era riuscito: Ender era stato promosso
	ammiraglio, e quella volta aveva insistito per veder rispettati i
	privilegi che spettavano al suo grado. Rigido e silenzioso aveva
	assistito alla proiezione di un filmato riguardante Stilson e della
	registrazione del suo combattimento con Bonzo, aveva visto le fotografie
	dei loro corpi, aveva ascoltato gli psicologi e gli avvocati discutere
	di dove finiva l'autodifesa e cominciava l'eccesso di difesa. Lui aveva
	le sue opinioni in merito, ma nessuno gliele aveva chieste. Durante
	tutto il processo si era sentito personalmente in stato di accusa. Il
	pubblico ministero era troppo conscio degli umori del pubblico per
	imputargli qualcosa, ma aveva insinuato che la sua mente fosse quella di
	un malato, di un pervertito con tendenze omicide, di un criminale.}

{-- Non farci caso -- aveva commentato Mazer Rackham. -- I politicanti
	ti temono, ma non possono ancora distruggere la reputazione che ti sei
	fatto. A questo ci penseranno gli storici, fra una trentina d'anni.}

{A Ender non importava molto della sua reputazione. Aveva assistito a
	quelle trasmissioni televisive con faccia impassibile, ma in realtà con
	un certo stupore. \emph{In guerra ho ucciso decine di miliardi di
		Scorpioni, creature vive e intelligenti, forse non peggiori di noi e che
		comunque non avevano lanciato un terzo attacco contro di noi, e nessuno
		lo ha definito un crimine.}}

{La morte di Stilson e quella di Bonzo non erano un peso più leggero né
	più grave dei delitti che già sopportava la sua coscienza.}

{E così, oppresso da quelle ombre, per vuoti e interminabili mesi aveva
	atteso che il mondo da lui salvato decidesse di richiamarlo a casa.}

{Uno dopo l'altro i suoi amici, pur riluttanti, s'erano separati da lui
	per tornare alle loro famiglie, ciascuno atteso da una città che lo
	avrebbe salutato come un eroe. Aveva visto alla televisione quelle
	cerimonie di benvenuto, e s'era commosso nel sentirli tessere a lungo
	gli elogi di Ender Wiggin che, affermavano, aveva insegnato loro tutto
	ciò che sapevano e li aveva condotti alla vittoria. Ma se avevano speso
	qualche parola per invocare il suo ritorno sulla Terra, quei tratti
	erano stati censurati e nessun altro aveva potuto udirli.}

{Per un po' di tempo su Eros non c'era stato altro da fare che riparare
	i danni causati dalla Guerra dei Due Blocchi, e ricevere i rapporti
	delle astronavi rimaste in grado di esplorare i numerosi pianeti che
	avevano attaccato.}

{Ma adesso su Eros c'era più attività che mai in passato, e più
	affollamento, perché molti coloni erano stati trasferiti lì in attesa di
	partire verso i silenziosi mondi degli Scorpioni. Ender diede una mano a
	riattrezzare gli interni di alcuni incrociatori, lavorando più di quel
	che gli ufficiali e i tecnici avrebbero desiderato. Nessuno di loro
	sembrava pensare che quel ragazzo di dodici anni poteva essere utile in
	un'attività pacifica quanto lo era stato in guerra; ma lui sopportava
	pazientemente la loro tendenza a ignorarlo, e quando aveva proposte o
	suggerimenti validi ne parlava coi pochi adulti disposti ad ascoltarlo,
	lasciando che poi le presentassero come fossero idee loro. Non si
	preoccupava di ottenere credito, ma solo di far bene il suo lavoro.}

{L'unica cosa che non poteva sopportare era la venerazione dei coloni.
	Imparò a evitare i tunnel dov'erano acquartierati, dopo aver fatto
	esperienza della confusione che destava se solo gli capitava di passare
	fra quella gente. Il suo volto era ormai troppo noto. Le donne e le
	ragazze correvano ad abbracciarlo, gli uomini volevano stringergli la
	mano, le madri insistevano per fargli baciare i loro bambini, non pochi
	dei quali erano già stati battezzati con il suo stesso nome, e poi si
	commuovevano nel vederlo così giovane. E non mancavano quelli che gli
	giuravano di non poterlo biasimare per i suoi delitti, non \emph{loro},
	\emph{} perché lui infine era un \emph{ragazzino} e non ne aveva
	colpa\ldots{}}

{Ender faceva il possibile per tenersene alla larga.}

{Ma fra i coloni giunse qualcuno che non poté evitare.}

{Quel giorno non era all'interno di Eros. Era uscito con una navetta per
	andare all'Attracco I.S. dove stava imparando a lavorare sullo scafo
	esterno delle astronavi. Chamrajnagar gli aveva fatto osservare che la
	carpenteria meccanica non si confaceva alla dignità di un ammiraglio, ma
	Ender aveva replicato che non essendoci al momento eccessiva richiesta
	di esperti in guerre stellari gli sembrava saggio imparare un altro
	lavoro.}

{La radio del suo casco emise un ronzio e la voce di una centralinista
	lo informò che una persona chiedeva di vederlo nell'interno di quella
	stessa astronave. Ender non aveva idea di chi potesse essere, e non si
	affrettò particolarmente. Finì di installare l'antenna di riserva
	dell'ansible, poi si agganciò a uno dei cavi usati dagli operai e
	trecento metri più avanti, al portello della camera stagna, chiese il
	permesso di entrare.}

{Lei lo stava aspettando fuori dal deposito degli scafandri. Per un
	attimo lo seccò vedere che consentivano a dei coloni di disturbarlo
	perfino lì, sul lavoro. Poi la ragazza si volse, sentendolo arrivare, e
	lui ebbe un fremito.}

{-- Valentine!}

{-- Salve, Ender.}

{-- Che stai facendo qui?}

{-- Demostene ha dato le dimissioni. Adesso parto, vado alla più vicina
	delle colonie.}

{-- Ci vogliono cinquant'anni per arrivare là\ldots{}}

{-- Soltanto due, se sei a bordo della nave.}

{-- Ma se un giorno tornerai, tutti quelli che conoscevi sulla Terra
	saranno morti da un pezzo e\ldots{}}

{-- Proprio questo avevo in mente. Ho la speranza, tuttavia, che
	qualcuno di quelli che conosco su Eros venga con me.}

{-- Io non me la sento di andare su uno dei mondi che abbiamo rubato
	agli Scorpioni. Ciò che voglio è tornarmene a casa.}

{-- Ender, tu non tornerai mai più sulla Terra. Ho fatto in modo io
	stessa che fosse così, prima di partire.}

{Lui la fissò senza riuscire ad aprir bocca.}

{-- Ho preferito dirtelo subito, così se questi sono i tuoi sentimenti
	potrai cominciare a odiarmi fin dall'inizio.}

{Poco dopo, in una delle cabine già attrezzate per i coloni, la ragazza
	si spiegò meglio. -- Peter sta lavorando per farti richiamare sulla
	Terra, sotto la protezione del Consiglio dell'Egemonia -- disse. -- E
	può riuscirci. Nella situazione che si sta evolvendo, Ender, questo ti
	metterebbe a tutti gli effetti sotto il controllo di Peter, perché già
	metà dei consiglieri fanno quel che vuole lui. E quelli che non sono
	anima e corpo con Locke, li può intimidire o ingannare in altri modi.}

{-- Sanno chi è Locke in realtà?}

{-- Sì. La cosa non è ancora pubblica, ma nelle alte sfere della
	finanza, della F.I. e della politica lo conoscono bene. Ha troppo potere
	perché qualcuno stia a pensare alla sua età. Ha fatto cose incredibili,
	Ender.}

{-- Ho notato che il trattato firmato un anno fa portava il nome di
	Locke.}

{-- Quella è stata la sua mossa decisiva. Ha avanzato la Proposta Locke
	facendola avallare dai più grossi proprietari di video-giornali, e ad
	essa si è accodato anche Demostene. Era il momento che aveva atteso:
	usare l'influenza di Demostene sulle masse e quella di Locke sugli
	intellettuali per raggiungere un risultato di prestigio. Ed è riuscito a
	individuare una forma di accordo che, per motivi diversi, andava bene
	all'Est come all'Ovest, evitando una guerra che poteva essere
	terribile.}

{-- Ha deciso di mettersi l'aureola dello statista?}

{-- Così credo. Ma un giorno in cui era di buonumore, vale a dire in
	vena di fare il cinico, mi ha detto che se avesse permesso all'Egemonia
	di sfasciarsi avrebbe dovuto conquistare il mondo pezzo per pezzo.
	Finché l'Egemonia sta in piedi, invece, lo può conquistare in un solo
	boccone.}

{Ender annuì. -- Questo è il Peter che conoscevo.}

{-- Divertente, no? Peter che salva milioni di vite.}

{-- Mentre io ne stermino miliardi.}

{-- Non volevo alludere a questo.}

{-- Così pensa di potermi usare?}

{-- Lo pensava. Aveva dei piani per te, Ender. Voleva attendere il tuo
	arrivo per rivelare pubblicamente la sua identità, incontrandoti di
	fronte alle telecamere: il fratello maggiore di Ender Wiggin, che oltre
	a ciò è anche il grande Locke, l'architetto della pace. Accanto a te
	sarebbe apparso più maturo, e la somiglianza fisica fra voi è notevole,
	oggi. Stava già rastrellando denaro dappertutto. Col tuo stesso cognome,
	e col tuo appoggio, avrebbe potuto arrivare dovunque.}

{-- Perché lo hai fermato?}

{-- Ender, non ti piacerebbe trascorrere il resto della vita come una
	marionetta di Peter.}

{-- Perché no? Finora sono sempre stato la marionetta di qualcuno.}

{-- Anch'io. Ho mostrato a Peter del materiale che avevo messo insieme,
	abbastanza da provare all'opinione pubblica che è un maniaco omicida.
	Fra le altre cose alcune sue foto mentre tortura degli scoiattoli, varie
	conversazioni registrate, e altre registrazioni dei tempi in cui avevi
	il monitor e che lo mostrano mentre ti tormenta con ferocia. Quando ci
	ha riflettuto sopra mi ha chiesto che prezzo chiedevo. E il prezzo che
	ho chiesto è stato la tua libertà, e la mia.}

{-- Vivere in casa di qualcuno che ho assassinato non è precisamente la
	mia idea di libertà.}

{-- Ender, ciò che è fatto è fatto. Adesso i loro mondi sono vuoti, e il
	nostro è affollato. E possiamo portare lassù cose che non ci sono mai
	state: gente che vive una sua vita personale, individuale, che si ama o
	che si odia per ragioni soltanto sue. In tutti i pianeti degli Scorpioni
	c'è sempre stata soltanto una persona, una vita, una storia; quando li
	abiteremo noi saranno pieni di vite e di storie, di animali e bambini.
	Ender, la Terra appartiene a Peter, e se tu non vieni via con me lui ti
	avrà, ti userà, ti tormenterà finché maledirai il giorno in cui sei
	nato. Adesso, e con me, hai l'unica possibilità di fuga.}

{Ender non disse nulla.}

{-- So cosa stai pensando, Ender. Pensi che io desidero soltanto
	controllarti, non troppo diversamente da Peter o da Graff o altri.}

{-- Chi ti dice che non mi stai già controllando?}

{-- Benvenuto nella razza umana, allora -- sorrise lei. -- Nessuno ha il
	pieno controllo della sua vita. Il meglio che puoi fare è di lasciare un
	po' di questo controllo a qualcuno che sia in gamba, o che ti vuol bene.
	Io non sono venuta qui perché sogno la vita del colono. Sono qui perché
	finora ho vissuto con un fratello che odio. Ora voglio una possibilità
	di conoscere il fratello che amo, prima che sia troppo tardi, prima che
	la nostra infanzia sia svanita.}

{-- È già troppo tardi per questo.}

{-- Sbagli, Ender. Ti senti cresciuto e logoro e stanco di tutto, ma nel
	tuo cuore sei un ragazzino, e io sono ancor più giovane di te. Lo
	terremo gelosamente segreto. E quando tu governerai la colonia e io
	scriverò di filosofia e politica, nessuno saprà che la sera giochiamo a
	dama imbrogliando dispettosamente e poi facciamo le battaglie coi
	cuscini.}

{Ender rise, ma aveva notato un paio di cosette gettate lì troppo
	casualmente per essere casuali. -- Governare?}

{-- Io sono Demostene, Ender. Ho lasciato la terra su ali di fiamma: un
	pubblico annuncio in cui dichiaravo che credevo tanto nella nostra
	missione colonizzatrice da partire con la prima astronave. Nello stesso
	tempo il Ministro delle Colonie, un certo ex colonnello Graff, rivelava
	che il pilota di questa astronave sarebbe stato il grande Mazer Rackham,
	mentre la carica di governatore della colonia era stata affidata a Ender
	Wiggin.}

{-- Qualcuno avrebbe potuto prendersi il disturbo di chiedermelo.}

{-- Te lo sto chiedendo io.}

{-- Dopo che tutto è già stato annunciato?}

{-- A dire il vero queste registrazioni saranno trasmesse domani, se tu
	accetti. Mazer si è detto d'accordo qualche ora fa, su Eros.}

{-- Rivelerai a tutti che Demostene sei tu? Una ragazza di quattordici
	anni?}

{-- Si dirà soltanto che Demostene parte con i coloni. Lasciamo pure che
	i curiosi trascorrano i prossimi cinquant'anni a ruminare sulla lista
	dei passeggeri, cercando d'immaginare chi di loro è il grande demagogo
	che pestò i calli a Locke.}

{Ender rise e scosse il capo. -- Sembra proprio che tutto questo ti
	diverta molto, Val.}

{-- Non vedo perché non dovrebbe.}

{-- Va bene -- disse Ender. -- Verrò. Forse proverò anche a fare il
	governatore, se tu e Mazer sarete lì a darmi una mano. Al momento la mia
	sola genuina dote di politicante è un'ignoranza assoluta di quello che
	dovrò fare.}

{Lei mandò un gridolino e lo abbracciò, con tutte le manifestazioni
	d'entusiasmo tipiche di una fanciulla a cui il suo fratellino minore ha
	appena fatto il regalo più bello.}

{-- Val -- disse lui, -- voglio solo che una cosa sia chiara: non vengo
	perché me lo hai chiesto tu, né per essere governatore, né perché qui mi
	annoio. Vengo perché conosco gli Scorpioni meglio di chiunque altro, e
	forse là riuscirò a capirli meglio. Io ho rubato loro il futuro; posso
	riparare soltanto cercando di studiare e conservare il loro passato.}

{~}

\begin{center}
	{* * *}
\end{center}

{~}

{Il viaggio fu lungo. Prima che giungesse al termine, Val aveva finito
	il primo volume della sua storia delle guerre contro gli Scorpioni e il
	testo fu trasmesso alla Terra con la firma di Demostene. Ender si era
	guadagnato qualcosa di più che l'adulazione dei passeggeri; la gente che
	aveva imparato a conoscerlo gli voleva bene e lo rispettava.}

{Sul nuovo pianeta s'impegnò nell'organizzazione della colonia e lavorò
	con la stessa energia degli altri per mettere in piedi un'economia
	autosufficiente. Ma il compito alla lunga più importante, come tutti
	furono d'accordo, consisteva nell'esplorare ciò che gli Scorpioni
	avevano costruito: strutture e macchinari, fattorie, depositi e miniere,
	cercando di apprendere cose nuove e annotando tutto quel che vi era di
	utilizzabile per gli esseri umani. Non si trovarono libri; gli Scorpioni
	non avevano mai avuto materiale scritto o registrato. Con tutta la loro
	scienza immagazzinata nella memoria collettiva, con tutte le
	informazioni tecniche presenti nei ricordi da cui potevano attingere,
	quando quella razza era morta la sua cultura era scomparsa con lei.}

{Tuttavia ogni oggetto racconta la sua storia. Dalla robustezza dei
	tetti delle fattorie, dalle spesse mura delle stalle e dalle dimensioni
	delle dispense e dei depositi di foraggio, Ender seppe che lì gli
	inverni erano duri, con pesanti nevicate. Dai recinti armati con punte
	aguzze rivolte all'infuori, seppe che vi erano predatori molto insidiosi
	per gli animali domestici. Dai mulini seppe che il destino dei frutti
	oblunghi dei malridotti frutteti era di venir macinati e trasformati in
	tonde forme di pane verdastro. E dagli slittini che gli adulti usavano
	per tirarsi dietro la prole anche nei campi apprese che, sebbene gli
	Scorpioni non avessero una vera mente individuale, curavano teneramente
	i loro piccoli.}

{La vita si stabilizzò, e gli anni trascorsero. I coloni abitavano in
	case di legno, e usavano i tunnel delle città-alveare come magazzini o
	per impiantarvi fabbriche. A governarli c'era adesso un Consiglio di
	ministri che venivano eletti, cosicché Ender, benché la gente
	continuasse a chiamarlo «governatore», più che altro si occupava del
	tribunale e dell'ordine pubblico. Crimini e beghe non mancavano, anche
	fra coloni la cui vita si fondava sull'amicizia e sulla collaborazione;
	la gente si amava e si odiava, era contenta o infelice, e da questo
	nascevano conseguenze che facevano di quel pianeta un mondo umano.
	Nessuno era molto interessato alle trasmissioni che giungevano via
	ansible, anche se l'apparecchio era sempre in funzione per la
	corrispondenza in arrivo o in partenza, e i nomi saliti alla ribalta
	sulla Terra significavano poco per i coloni. L'unico che conoscessero
	era quello di Peter Wiggin, l'Egemone della Terra, e le sole notizie
	diramate in diretta dalla TV locale parlavano di pace, di prosperità, di
	grandi astronavi che lasciavano le sponde del sistema solare per
	ripopolare i mondi degli Scorpioni. Presto vi sarebbero state altre
	colonie sul Mondo di Ender, e la gente che avrebbe fondato quelle nuove
	cittadine qua e là sul pianeta era già a metà strada, ma nessuno se ne
	preoccupava. Gli emigranti sarebbero stati ben accolti e istruiti sulle
	caratteristiche del pianeta, però gli argomenti che importavano al
	colono medio erano ben altri: chi riuscirà a sposare l'ardente Juanita
	Cruz, da che malattia è affetto il giovane Kristopoulos, questo terreno
	non è adatto per le mele ma è un miracolo per le banane, e perché dovrei
	pagarlo quando quel maledetto vitello è morto tre settimane dopo che me
	l'ha dato.}

{-- Sono diventati gente di campagna -- disse Valentine un giorno. -- A
	nessuno interessa sapere che Demostene oggi spedisce il settimo volume
	della sua Storia. Nessuno lo leggerà, qui.}

{Ender sfiorò un pulsante e il suo banco passò a mostrargli la pagina
	successiva. -- Vai molto a fondo nei particolari, Valentine. Quanti
	volumi ancora pensi di scrivere?}

{-- Uno soltanto. La storia di Ender Wiggin.}

{-- E cosa pensi di fare? Aspetterai che io sia invecchiato e morto?}

{-- No. Comincerò dalla tua infanzia e arrivata al momento presente
	concluderò.}

{-- Io ho un'idea migliore. Metti la parola fine al giorno dell'ultima
	battaglia. Da allora in poi non ho fatto nulla che meriti d'esser messo
	per iscritto.}

{-- Forse farò così -- disse Valentine. -- E forse no.}

{~}

\begin{center}
	{* * *}
\end{center}

{~}

{L'ansible aveva riferito che l'astronave dei nuovi coloni era ancora a
	un anno di viaggio da lì. Il loro rappresentante chiamò Ender
	all'apparecchio e gli chiese di trovare per loro un buon insediamento,
	abbastanza vicino da poter comunicare e commerciare senza difficoltà, ma
	abbastanza lontano da esser governato separatamente. Ender si fece
	assegnare un elicottero e ne approfittò per esplorare oltre i confini
	del territorio meglio conosciuto. Come aiutante prese con sé un
	ragazzino, un undicenne sveglio di nome Abra, che all'arrivo
	dell'astronave aveva soltanto tre anni e non conosceva altro mondo che
	quello. Ender e il suo passeggero partirono al mattino e volarono verso
	est fino al tramonto, poi atterrarono per trascorrere la notte in tenda,
	con l'idea di esplorare a piedi la zona i giorni successivi.}

{Fu il mattino del terzo giorno che, d'improvviso, Ender cominciò ad
	avere la spiacevole sensazione d'essere già stato in quel posto. Si
	guardò attorno: era una nuova terra, del tutto sconosciuta ai suoi
	occhi. Si volse a chiamare Abra.}

{-- Ehi, Ender! -- rispose il ragazzino agitando le braccia. Era sulla
	cima di una piccola altura cespugliosa. -- Vieni a vedere!}

{Ender si avviò su per il pendio, sprofondando con gli stivali nel
	terreno molle e fangoso. Abra gli stava indicando qualcosa più in basso,
	dalla parte opposta. -- Guarda qui. Ci avresti creduto?}

{La collinetta era spaccata in due. Nel mezzo c'era una profonda
	depressione che l'allargava in una caverna oscura, sul cui fondo
	stagnava l'acqua, e le pareti apparivano concave, stranamente regolari.
	A sud l'altura si abbassava e si separava in due costoni, che
	l'allargavano a V; a nord invece campeggiava un enorme blocco di roccia
	bianca, simile al cranio di uno scheletro sogghignante, nella cui bocca
	aveva messo radici un albero.}

{-- È come se un gigante fosse caduto morto qui -- disse Abra, -- e la
	terra si fosse ammucchiata sulla sua carcassa.}

{Adesso Ender sapeva perché quell'immagine gli era entrata dritta nel
	subconscio. Il corpo del Gigante. Da bambino aveva giocato lì troppe
	volte per non riconoscere il posto. Ma questo era impossibile. Il
	computer della Scuola di Guerra non avrebbe mai potuto disporre di dati
	relativi a quel pianeta. Si portò il binocolo agli occhi e d'istinto
	scrutò verso est, già tremando all'incredibile sospetto di ciò che
	avrebbe potuto vedere sullo sfondo dei boschi.}

{E là, sulla riva di un ruscello, altalene e piccole giostre, un toboga.
	Il tutto arrugginito e sepolto fra le erbacce, ma non c'era possibilità
	di sbagliarsi sulle forme di quegli oggetti.}

{-- Qualcuno deve aver costruito, dentro questa collinetta -- disse
	Abra. -- Guarda il teschio, e i denti\ldots{} non è roccia. È cemento.}

{-- Lo so -- mormorò Ender. -- Loro l'hanno costruito per me.}

{-- \emph{Cosa?}}

{-- Conoscevo già questo posto, Abra. Gli Scorpioni l'hanno costruito
	per me.}

{-- Gli Scorpioni erano tutti morti cinquant'anni prima che arrivassimo
	qui.}

{-- Hai ragione, non è possibile. Ma io so quello che so. Abra, non
	avrei dovuto portarti con me. Potrebbe esserci un pericolo qui. Se mi
	conoscevano addirittura fino al punto di aver costruito questo posto,
	forse progettavano di\ldots{}}

{-- Di pareggiare i conti con te.}

{-- Per averli uccisi.}

{-- Allora vattene, Ender. Se questa è una trappola devi andartene!}

{-- Se quel che volevano era preparare la vendetta, Abra, non me ne
	importa. Ma forse non era questa la loro intenzione. Forse ciò che
	vediamo era quel che avevano di più vicino a una forma di
	linguaggio\ldots{} per lasciarmi scritto un messaggio.}

{-- Ma non sapevano neppure cosa significasse leggere o scrivere.}

{-- Forse stavano imparando, prima di morire. Meglio che tu vada via.}

{-- All'inferno! Io non torno al campo mentre tu esplori di qua e di là.
	Vengo con te.}

{-- No. Sei troppo giovane per rischiare di\ldots{}}

{-- Giovane un corno! Tu sei \emph{Ender Wiggin}, \emph{} perciò non
	dire a me cosa può fare e non può fare un ragazzo di undici anni!}

{Stabilirono di prendere l'elicottero, quindi tornarono sorvolando il
	corpo del Gigante, il parco giochi e la boscaglia, individuando la
	radura col pozzo. E poco più avanti c'era uno strapiombo, alla sommità
	del quale videro un cornicione su cui si apriva quella che era senza
	dubbio una porta di legno, esattamente dove avrebbe dovuto essere la
	Fine del Mondo. E all'orizzonte, sfumato nella foschia e tuttavia ben
	visibile sulla cima di un dirupo, c'era il castello. Con la torre.}

{Fu alla base delle mura corrose dal tempo che Ender atterrò. Scese
	dall'elicottero e ordinò ad Abra di mettersi ai comandi. -- Qualunque
	cosa accada non seguirmi. Se non torno, decolla e torna a casa.}

{-- Ah, tappati la bocca, Ender!}

{-- Tappatela tu, pivello, o te la riempio di fango.}

{Malgrado il tono scherzoso di Ender, un lampo nei suoi occhi informò
	Abra che diceva sul serio, così si strinse nelle spalle.}

{In muro esterno della torre aveva pietre così sporgenti che sembravano
	fatte apposta per arrampicarsi. Capì che avevano voluto proprio questo.}

{La stanzetta in cui entrò scavalcando il davanzale della finestra era
	proprio come doveva essere, mobili compresi. D'istinto Ender si volse al
	caminetto, aspettandosi di vedere il serpente, ma c'era soltanto un
	tronco d'albero con un'estremità scolpita a testa di rettile.
	Un'imitazione simbolica, non un duplicato, e per essere delle creature
	che non conoscevano l'arte la cosa era fin troppo ben fatta. Dovevano
	aver preso quelle immagini della sua stessa mente, contattandola ed
	esplorandone le fantasie oniriche attraverso l'immensità degli
	anni-luce. Ma perché? Per suggerire al suo inconscio di venire fin lì,
	naturalmente. Lì dove c'era un messaggio per lui. Un messaggio\ldots{}
	ma dov'era? E di che genere poteva mai essere? L'arcano stupore che
	s'era impossessato di lui continuava a dargli la pelle d'oca.}

{Lo specchio era fissato alle pietre della parete di fondo. Era una
	lastra di metallo opaco, nella quale era stata incisa rozzamente
	l'immagine di un volto umano. Il suo? \emph{Hanno cercato di riprodurre
		ciò che io vedo quando mi guardo allo specchio.}}

{Fissò quel metallo senza capire. Ma in lui tornavano i ricordi: lo
	specchio scalzato dal muro, la cavità, i serpenti che ne balzavano fuori
	e lo attaccavano, affondando i loro denti velenosi sulla sua figura che
	infine cadeva al suolo uccisa e sconfitta.}

\emph{{Quanto dovevano conoscermi bene! }}{si meravigliò Ender.
	\emph{Abbastanza bene da sapere che ho affrontato tante volte questo
		genere di morte da non averne più paura\ldots{} abbastanza da sapere
		che, se anche avessi paura, questo non m'impedirebbe di staccare lo
		specchio dal muro.}}

{Si avvicinò alla lastra metallica, sollevò il bordo inferiore e notò
	che veniva via come un coperchio. Ma niente balzò fuori ad aggredirlo.
	Ciò che Ender si trovò a fissare era una cavità dalle pareti lisce, sul
	fondo della quale riposava un ovoide di materiale bianco come la seta da
	cui, qua e là, pendevano stralci d'aspetto fibroso. Un uovo? No, non si
	trattava di un uovo: era una pupa, la larva di una regina degli
	Scorpioni, già fertilizzata dai maschi della sua specie e pronta a dare
	alla luce centinaia di migliaia di Scorpioni, compresi alcune altre
	regine ed altri maschi. Gli occhi di Ender stavano captando immagini che
	non facevano parte dei suoi ricordi, né della sua mente, né del suo
	mondo: le immagini dei maschi degli Scorpioni, molli e biancastri, che
	uscivano dall'oscurità di un tunnel. Dalla parte opposta due grosse
	femmine stavano introducendo la regina neonata nella sua stanza nuziale.
	Ognuno dei maschi si fece avanti, compì l'atto della penetrazione sulla
	regina larvale, tremò sconvolto da una breve estasi, cadde al suolo e
	morì, disseccandosi e accartocciandosi rapidamente. Poi la nuova regina
	fu deposta dinnanzi a un'anziana e magnifica creatura avvolta in due
	morbide ali scintillanti, un essere che aveva da molto tempo perso la
	capacità di volare ma era ancora aureolato di un maestoso potere. La
	vecchia regina si chinò a baciare la nuova, addormentandola con una
	droga lievemente venefica che le uscì dalle labbra cornee, quindi
	l'avvolse con i bianchi filamenti prodotti dal suo addome e nel farlo le
	comandò di diventare quel che lei era stata: una nuova creatrice, una
	nuova città, un nuovo mondo, una fonte da cui sarebbero emerse altre
	regine per popolare altre città e altri mondi\ldots{}}

\emph{{Come posso sapere tutto questo }}{si chiese Ender. \emph{Come
		posso vedere cose che non sono mai state nella mia memoria?}}

{Quasi in risposta a quella domanda nuove immagini lo sommersero, e
	riconobbe quelle della prima battaglia contro una flotta degli
	Scorpioni. Le stesse che aveva osservato sul simulatore, ma capovolte,
	perché ora le vedeva come le aveva viste la regina di quell'alveare,
	attraverso moltissimi occhi diversi. Vide gli Scorpioni assumere la loro
	formazione globulare, sentì la loro sorpresa quando i terribili
	incrociatori terrestri sbucarono come lampi imprevedibili dalle tenebre;
	quindi vi furono i bagliori azzurri del distruttore molecolare che
	faceva esplodere in polvere le navi dell'alveare.}

{Ender provò le sensazioni che la regina aveva provato e trasmesso ad
	altre, mentre attraverso gli occhi delle sue operaie/combattenti vedeva
	piombare sulla flotta una morte troppo rapida perché fosse possibile
	evitarla. Non erano state sensazioni di paura o di dolore, tuttavia. Ciò
	che quella regina aveva sentito era stata una grande tristezza, una cupa
	rassegnazione all'ineluttabile. Non aveva pensato quelle parole, mentre
	vedeva l'attacco dei terrestri decisi ad uccidere, ma fu in parole che
	Ender poté tradurre la sua riflessione: \emph{Loro non ci hanno
		perdonato}, \emph{} aveva pensato quella regina. \emph{Di certo noi
		moriremo, adesso.}}

{-- E come puoi riavere la vita? -- chiese Ender.}

{La regina racchiusa nel suo bozzolo di seta non aveva parole da
	offrirgli, ma quando lui fissò accigliato quell'oggetto, di nuovo da
	esso parvero scaturire delle immagini mentali: l'atto di deporre il
	bozzolo in un luogo fresco, un luogo oscuro, dove scorresse acqua per
	dargli umidità\ldots{} no, non semplice acqua, bensì acqua mista alla
	linfa di un certo albero, e tenerlo tiepido cosicché alcune reazioni
	potessero avvenire nel suo interno. Poi attendere. Giorni e settimane,
	per dare alla pupa il tempo di completare la metamorfosi. E poi,
	allorché il bozzolo avrebbe assunto un polveroso colore
	marroncino\ldots{} Ender vide se stesso nell'atto di aprirlo, e di
	aiutare la piccola e fragile regina ad emergerne. Vide se stesso
	sorreggerla per gli arti anteriori e aiutarla a camminare dal bozzolo
	squarciato a un nido fatto di sabbia e foglie secche. \emph{Allora sarò
		viva}, \emph{} fu il pensiero/sensazione che lui captò. \emph{Allora
		sarò sveglia. Allora partorirò i miei diecimila figli.}}

{-- No! -- disse Ender. -- Non posso farlo.}

{Angoscia.}

{-- I tuoi figli, oggi, sono i mostri dei nostri incubi. Se io ti
	portassi alla luce, sarebbe soltanto per destinarti al massacro.}

{Dentro di lui lampeggiarono dozzine di immagini di esseri umani che
	venivano uccisi dagli Scorpioni, ma insieme ad esse scaturì un flusso di
	dolore così intenso che Ender non poté sopportarlo. Sentì le lacrime
	scorrergli sul volto, calde e veloci.}

{-- Sì\ldots{} se puoi far provare agli altri quel che fai provare a me,
	forse sapranno perdonare e dimenticare. Forse.}

\emph{{Soltanto io}}{, \emph{} rifletté. \emph{Mi hanno trovato
		attraverso l'ansible, seguendolo e scivolando nella mia mente.
		Penetrando in quei miei sogni tormentosi sono arrivati a conoscermi,
		proprio quando trascorrevo le giornate combattendoli e distruggendoli
		hanno scoperto le mie paure, e soprattutto hanno scoperto che non ero
		consapevole di sterminarli veramente. In quelle poche settimane che
		restavano loro da vivere hanno costruito questo posto per me, e il corpo
		del Gigante, e il precipizio alla Fine del Mondo, in modo che i miei
		occhi mi conducessero fin qui. Io sono il solo che essi conoscano, e
		così riescono a parlare soltanto a me e attraverso di me.}}

{Noi siamo come te.}

\emph{{Noi siamo come te}}{, \emph{} fu il pensiero che prese forma
	nella sua mente. \emph{Non volevamo uccidere. E quando abbiamo capito,
		non siamo più tornati al vostro mondo. Noi credevamo d'essere le uniche
		creature intelligenti dell'universo, finché non abbiamo incontrato voi.
		Ma non avremmo mai supposto che il pensiero cosciente potesse nascere in
		animali solitari che non condividevano i loro sogni. Come avremmo potuto
		saperlo? Noi avremmo potuto vivere in pace con voi. Credimi. Credimi.
		Credimi.}}

{Allungò le mani nella cavità e sollevò il bozzolo. Era
	sorprendentemente fragile, per un oggetto che conteneva tutto il futuro
	e tutte le speranze di una razza di esseri senzienti.}

{-- Ti porterò con me -- disse Ender, -- di pianeta in pianeta, finché
	troverò un luogo dove tu possa svegliarti in sicurezza. E racconterò la
	vostra storia alla mia gente, cosicché per quel giorno possano avervi
	perdonato. Così come voi avete perdonato me.}

{Avvolse il bozzolo della regina nella blusa e tornò alla finestra, poi
	si calò fino alla base della torre.}

{-- Che c'era là dentro? -- chiese Abra.}

{-- La risposta -- disse Ender.}

{-- A cosa?}

{-- Alla domanda che mi hai fatto. -- E questo fu tutto ciò che gli uscì
	di bocca sull'argomento. Continuarono l'esplorazione per altri cinque
	giorni, e infine scelsero una località molto a meridione del castello.}

{Qualche settimana dopo domandò a Valentine di leggere un saggio che
	aveva scritto. Lei batté il codice di quella registrazione, se la fece
	mandare su uno schermo dal computer dell'astronave, e lesse.}

{Era stato scritto come se la narratrice fosse l'ultima regina degli
	Scorpioni, che esponeva ciò che la sua razza aveva desiderato fare e ciò
	che aveva fatto. Parlava dei loro successi e dei loro fallimenti, e fra
	questi ultimi annoverava l'incontro con gli esseri umani. «Non volevamo
	farvi del male. Non consapevolmente» diceva, «e vi perdoniamo per averci
	uccisi».}

{Dagli albori della loro civiltà alla guerra che aveva spazzato via il
	loro pianeta natale, Ender ne riassumeva la storia come fosse un
	racconto tramandato oralmente dall'antichità. Quando arrivò a parlare
	della Grande Madre, l'unica regina riconosciuta nella sua epoca, colei
	che per prima aveva stabilito di allevare e istruire le giovani regine
	invece di ucciderle per non avere rivali, rallentò il ritmo della
	narrazione e disse di quante volte ella era stata costretta a
	distruggere quei frutti del suo corpo, le piccole regine che d'istinto
	le si rivoltavano contro, finché non ne partorì una che capiva il
	significato profondo dell'armonia e della collaborazione.}

{Questa era stata una novità rivoluzionaria per il loro mondo: due
	regine che si amavano e si aiutavano l'un l'altra invece di battersi
	furiosamente. Sotto di loro gli alveari si moltiplicarono, divennero
	forti e civili; prosperarono ed ebbero figlie capaci di vivere in pace.
	Quello era stato l'inizio di un regno destinato ad evolversi su molti
	pianeti.}

{«Ah, se soltanto avessimo saputo comunicare con voi!» sospirava
	l'immaginaria regina della storia di Ender. «Ma poiché ciò non accadde,
	vi chiediamo solo questo: che ci ricordiate, noi regine e operaie che vi
	combattemmo, non come nemiche ma come sventurate e tragiche sorelle, a
	cui Dio o il Fato o l'Evoluzione aveva dato una forma ahimè diversa
	dalla vostra. Se fossimo riusciti a stringerci la mano, ci saremmo
	apparsi l'un l'altro come creature uguali. E invece ci siamo uccisi a
	vicenda. Ma nonostante ciò i nostri spiriti vi danno il benvenuto, oggi,
	come ospiti onorati. Venite sui nostri mondi, amici della Terra; abitate
	i nostri tunnel, ridate la vita ai nostri campi, e ciò che non è più
	fatto dalle nostre mani siano le vostre a farlo in pace. Germogliate per
	loro, alberi e fiori. Sole, scalda questi nostri fratelli. E tu, buona
	terra, sii fertile per loro. Purché la vita continui, questa è l'eredità
	che gli lasciamo, e sia per sempre la loro casa.»}

{Il libro che Ender aveva scritto non era lungo, comunque conteneva
	tutti i fatti buoni o malvagi che erano a conoscenza della regina non
	ancora nata. E non lo firmò col suo nome, bensì con un titolo che aveva
	voluto darsi:}

{~}

\begin{center}
	{L'ARALDO DEI DEFUNTI}
\end{center}

{~}

{Sulla Terra il libro fu pubblicato senza molto scalpore, ma ne furono
	distribuite tante copie che già pochi mesi dopo era difficile credere
	che qualcuno non ne conoscesse il contenuto. Molti lo trovarono
	interessante; una ristretta minoranza prese alcuni dei suoi aspetti fin
	troppo sul serio. Questi diedero inizio a un culto basato sulla
	fratellanza universale e sul principio che, quando uno di essi moriva,
	aveva il diritto di avere accanto a sé un altro confratello, l'Araldo
	dei Defunti, il quale narrava la vita e le opere dello scomparso con le
	parole che lui stesso avrebbe usato, ma con spietata verità e senza
	celare i difetti né sottolineare le virtù. Quelli che si dedicarono a
	simili servizi funebri destarono spesso sconcerto e disagio fra i
	parenti del defunto, ma vi fu anche chi ritenne che la sua vita dovesse
	servire d'insegnamento a qualcun altro, anche per gli errori in essa
	contenuti, e s'impegnò a lasciarla scritta affinché alla sua conclusione
	vi fosse un Araldo che dicesse la verità come per la sua stessa bocca.}

{Sulla Terra essa rimase una religione fra le tante. Ma per quelli che
	avevano attraversato lo spazio per abitare nei tunnel delle regine degli
	alveari, e per coltivare i campi un tempo appartenuti agli alveari,
	spesso questa fu la sola religione. E non ci fu colonia che non avesse
	il suo Araldo dei Defunti.}

{Nessuno seppe, e nessuno in realtà volle sapere, chi fosse stato il
	primo degli Araldi. Ender preferì non dirlo.}

{All'età di venticinque anni Valentine finì l'ultimo volume della sua
	storia delle guerre contro gli Scorpioni. Ad esso accluse il testo
	completo del piccolo libro di Ender, senza però rivelare il nome
	dell'autore.}

{Fu allora che l'anziano Egemone della Terra, Peter Wiggin, ormai
	settantasettenne e sofferente di gravi disturbi cardiaci, si mise in
	contatto con lei, via ansible.}

{-- Io so chi l'ha scritto -- le disse il fratello. -- Ebbene, se lui
	può dar voce alle parole degli Scorpioni, sicuramente potrà farlo anche
	per me.}

{Ender parlò così con lui a mezzo ansible, e Peter gli raccontò la
	storia della sua vita senza omettere nessuno dei suoi crimini né le
	azioni che avevano portato vantaggi a qualcun altro. E quando Peter
	morì, Ender scrisse un secondo volume ancora a firma dell'Araldo dei
	Defunti. I due libri, insieme, vennero chiamati \emph{La Regina
		dell'Alveare} e \emph{l'Egemone}, \emph{} e furono considerati scritti
	sacri.}

{-- Coraggio, Val -- disse un giorno a sua sorella. -- Voliamo via, e
	andiamo a vivere per sempre.}

{-- Non ci è concesso -- rispose Valentine. -- Ci sono miracoli che
	neppure la velocità relativistica può fare, Ender.}

{-- Dobbiamo andarcene. Sento che qui potrei perfino trovare la
	felicità.}

{-- Allora rimani.}

{-- Ho vissuto troppo a lungo col mio dolore. Non voglio sapere che
	persona sarei senza di esso.}

{Così si imbarcarono su un'astronave e viaggiarono di pianeta in
	pianeta. Dovunque si fermarono lui fu soltanto Andrew Wiggin, Araldo
	itinerante dei defunti, e lei fu soltanto una storica di nome Valentine,
	che metteva per iscritto le opere dei vivi mentre lui dava voce alle
	storie dei defunti. E in ognuno di quei luoghi Ender portò sempre con sé
	il prezioso bozzolo di seta bianca, in cerca del mondo in cui la regina
	dell'alveare avrebbe potuto risvegliarsi e crescere, e vivere in pace.}

{La sua fu una lunga ricerca.}

\begin{center}
	\textbf{\emph{{FINE}}}
\end{center}

\newpage\blankpage