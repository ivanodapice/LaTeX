\chapter{VALENTINE}

{~}

{~}

{~}

{-- Ragazzini?}

{-- \emph{Fratello e sorella. Si erano nascosti sotto cinque diversi
		strati di precauzioni difensive, nelle reti computerizzate dei
		video-giornali\ldots{} lavorando per compagnie che pagano bene i loro
		articolisti. Per rintracciarli ci è voluta una maledetta quantità di
		tempo.}}

{-- \emph{Cosa stanno nascondendo?}}

{-- \emph{Potrebbe essere qualsiasi cosa. La più ovvia da nascondere,
		comunque, è la loro età. Il ragazzo ha quattordici anni, la femmina
		dodici.}}

{-- \emph{Chi di loro è Demostene?}}

{-- \emph{La ragazza. La dodicenne.}}

{-- \emph{Mi scusi. Non penso affatto che ci sia qualcosa di divertente,
		ma non ho potuto fare a meno di ridere. Tutto il tempo che abbiamo
		trascorso a roderci l'anima\ldots{} tutta la fatica che stiamo facendo
		per convincere i russi a non prendere sul serio Demostene! Siamo
		arrivati al punto di portare Locke come prova che gli americani non sono
		tutti dei paranoici guerrafondai. E loro sono fratello e sorella. Due
		adolescenti!}}

{-- \emph{E il loro cognome è Wiggin.}}

{-- \emph{Ah! Una coincidenza?}}

{-- \emph{Il Wiggin è un Terzo. Loro sono il primo e la seconda.}}

{-- \emph{Ah, andiamo bene! Adesso i russi non crederanno mai e poi mai
		che\ldots{}}}

{-- \emph{Che Demostene e Locke non sono affatto sotto il nostro
		controllo, come lo è} il \emph{Wiggin.}}

{-- \emph{Che sia una cospirazione? Che qualcuno li stia controllando?}}

{-- \emph{Siamo riusciti a stabilire che non esiste nessun contatto fra
		questi due adolescenti e qualsiasi adulto che potrebbe dirigerli.}}

{-- \emph{Questo non significa che qualcuno non abbia escogitato un
		sistema in cui non avete potuto penetrare. È difficile convincersi che
		due ragazzini\ldots{}}}

{-- \emph{Ho avuto un colloquio con il colonnello Graff, quando è
		tornato dalla Scuola di Guerra. È sua ferma opinione che nulla di quanto
		hanno fatto i due ragazzini era al di là delle loro capacità. Queste
		sono virtualmente uguali capacità di\ldots{} del Wiggin. Soltanto il
		loro carattere è diverso. Ciò che lo ha sorpreso, tuttavia, è
		l'orientamento dei due personaggi pubblici: Demostene è infatti la
		ragazza, non c'è dubbio, però Graff dice che lei fu rifiutata dalla
		Scuola di Guerra perché era troppo pacifica, troppo conciliante, e
		soprattutto troppo capace di empatia.}}

{-- \emph{L'esatto contrario di Demostene.}}

{-- \emph{Mentre il ragazzo ha l'anima di uno sciacallo.}}

{-- \emph{Non è stato Locke a esser premiato di recente come «il
		giornalista americano di più larghe vedute»?}}

{-- \emph{È difficile capire cosa sta succedendo. Ma Graff ha
		raccomandato, e io sono d'accordo, di lasciarli fare. Senza
		smascherarli. L'ordine è di non fare nessun rapporto, salvo una nota in
		cui si dichiara che Locke e Demostene non hanno contatti esteri né con
		gruppi interni, a parte i legami pubblicamente dichiarati nei loro
		articoli.}}

{-- \emph{In altre parole, dichiararli innocui e lasciargli mano
		libera.}}

{-- \emph{So che Demostene sembra pericoloso, forse perché lui, o lei,
		ha un seguito così vasto. Ma ritengo significativo il fatto che il più
		ambizioso dei due ha scelto di essere moderato. Comunque non fanno che
		discorsi. Hanno influenza, ma nessun potere.}}

{-- \emph{Da quanto ne so io, influenza è sinonimo di potere.}}

{-- \emph{Se dovessero sgarrare troppo, potremmo smascherarli
		pubblicamente.}}

{-- \emph{Solo per pochi anni ancora. Più aspettiamo e più invecchiano,
		e allora desterà ben scarso stupore scoprire chi sono.}}

{-- \emph{Lei sa quali sono stati i movimenti delle truppe russe. C'è
		sempre la possibilità che Demostene abbia ragione. E in questo
		caso\ldots{}}}

{-- \emph{Ci farà comodo avere Demostene all'opera. Va bene. Li terremo
		fra due guanciali di piume, per ora. Ma sorvegliateli. E io,
		naturalmente, dovrò lambiccarmi il cervello per tenere tranquilli i
		russi.}}

{~}

\begin{center}
	{* * *}
\end{center}

{~}

{A dispetto della sua apprensione, per Valentine era un divertimento
	essere Demostene. Quasi ogni video-giornale della nazione riportava ora
	la sua colonna, ed era soddisfacente vedere il denaro accumularsi nei
	conti a cui attingeva sotto falsa identità. Di tanto in tanto lei e
	Peter, a nome di Demostene, facevano a certi candidati e a certe
	organizzazioni delle donazioni accuratamente calcolate: abbastanza
	denaro da far notare la cosa, ma non abbastanza da far sospettare al
	candidato che si volesse comprare il suo voto. Lei riceveva una tale
	mole di corrispondenza che la Calnet doveva dirottargliela su una
	segreteria, la quale rispondeva a quella di routine. Le lettere più
	interessanti le arrivavano da uomini politici di statura nazionale e
	anche internazionale, talvolta ostili, talaltra amichevoli, ma sempre
	miranti a influenzare diplomaticamente le vedute di Demostene. Queste
	lei e Peter le leggevano insieme, ridacchiando al pensiero che gente
	come quella stesse, senza saperlo, scrivendo a due ragazzini.}

{A volte, però, se ne vergognava. Suo padre leggeva regolarmente
	Demostene; ignorava Locke o, se mai l'aveva letto, non ne parlava. A
	pranzo non di rado elargiva loro punti di vista che Demostene aveva
	espresso nel suo articolo di quel giorno. Peter ne rideva sotto i baffi:
	-- Vedi? Questo dimostra che l'uomo della strada ha bisogno che gli
	dicano quale opinione deve avere. -- Ma Valentine si sentiva umiliata
	per suo padre. Se mai venisse a sapere che ho scritto io gli articoli di
	cui ci parla, e che non credo a metà di quelle cose, la rabbia e la
	vergogna lo ucciderebbero.}

{A scuola rischiò di combinare un guaio quando la sua insegnante di
	storia incaricò ogni studente di scrivere un saggio critico sulle vedute
	di Demostene e di Locke espresse in due dei loro primi articoli.
	Valentine dimenticò la prudenza e fece un brillante lavoro di analisi.
	Come risultato, dovette sudare per dissuadere il preside dal far
	pubblicare il saggio da uno dei videogiornali della stessa California
	Network. Peter s'infuriò selvaggiamente. -- Sembrava uscito dalla penna
	di Demostene! Vuoi rovinare tutto? Piuttosto faccio fuori Demostene
	definitivamente. Tu non sai cos'è l'autocontrollo!}

{Se dava in escandescenze per cose da poco, Peter la spaventava ancor di
	più quando smetteva di parlarle. L'episodio accadde quando Demostene fu
	invitato a far parte del Comitato Presidenziale per l'Educazione al
	Futuro, un gruppo onorario che in realtà non faceva niente, ma lo faceva
	splendidamente. Valentine avrebbe supposto che Peter l'avrebbe presa
	come un'altra vittoria, ma così non fu. -- Rifiuta -- le disse.}

{-- Perché dovrei? -- si oppose lei. -- Non si tratta di un lavoro, e
	hanno perfino detto che rispettando il ben noto desiderio d'anonimato di
	Demostene potrò limitarmi a partecipare con interventi scritti. Questo
	darà un peso autorevole alla persona di Demostene, e\ldots{}}

{-- E ti godrai l'idea d'essere riuscita a ottenerlo prima di me.}

{-- Peter, non si tratta di te e di me, ma di Demostene e Locke. Noi li
	abbiamo costruiti. Non sono veri. E poi questo invito non significa che
	Demostene gli piaccia più di Locke, ma soltanto che ha una più vasta
	base di consenso popolare. Tu sapevi che sarebbe stato così. L'hai
	progettato per solleticare gli umori di tutti gli antisovietici e gli
	sciovinisti del mondo occidentale.}

{-- La cosa non doveva funzionare a questo modo. Era Locke quello
	destinato a diventare autorevole e rispettato.}

{-- Lo è. Il rispetto delle persone intelligenti non ha a che vedere con
	le manovre governative per accontentare le masse. Peter, non prendertela
	con me se ho fatto fin troppo bene quello che volevi.}

{Ma l'ira di lui durò molti giorni, e il suo mutismo costrinse Valentine
	a scrivere diversi articoli senza consultarlo. Probabilmente in
	quell'occasione Peter fu convinto che la colonna di Demostene avesse
	perduto mordente, ma anche se così fu nessuno parve notarlo. E la sua
	acrimonia dovette peggiorare quando vide che lei non veniva piangendo a
	chiedergli aiuto. Ma Valentine era stata Demostene ormai troppo a lungo
	per aver bisogno che le si dicesse cos'avrebbe pensato Demostene su
	questo o quell'argomento.}

{E mentre la sua corrispondenza con altri cittadini politicamente attivi
	s'intensificava, venne a conoscenza di fatti e retroscena di solito
	celati al grosso pubblico. Alcuni ufficiali delle forze armate di
	tendenze reazionarie le scrivevano accennando sovente a episodi e fatti
	tenuti sotto silenzio, e lei e Peter si divertivano a metterli insieme
	per ottenere un affascinante e inquietante quadro dell'attività segreta
	del Patto di Varsavia. I comunisti si stavano senza dubbio preparando
	alla guerra, una guerra che evidentemente prevedevano sanguinosa e di
	vasta portata. Demostene non sbagliava, strombazzando che il Patto di
	Varsavia stava violando ogni regola e tradiva la fiducia degli Alleati.}

{E il personaggio di Demostene pian piano cominciò ad acquistare una
	sorta di vita indipendente. A volte, dopo aver partecipato a dibattiti
	televisivi in cui era concesso inviare per video domande e risposte
	scritte, s'accorgeva d'aver \emph{pensato} come Demostene, e d'essersi
	trovata d'accordo con idee che avrebbero dovuto essere soltanto un
	copione fasullo. E a volte, leggendo articoli di Locke, si sentì
	irritata dalla sua evidente cecità per quello che stava realmente
	accadendo.}

{Forse non era possibile indossare l'abito senza diventare monaco almeno
	in parte. Questo le dava da riflettere, e un giorno in cui certe
	deduzioni finirono col preoccuparla scrisse un articolo usando quel
	concetto come una premessa, per dimostrare che i politicanti usi a
	tranquillizzare i russi per tener calme le acque finivano
	inevitabilmente per divenire loro succubi, o addirittura loro
	involontari strumenti. Questo fu il sottile e calunnioso pugnale che
	Demostene affondò nella schiena del Governo, e gli estremisti di destra
	ne furono elettrizzati. Ricevette moltissima corrispondenza, e consensi
	anche da parte di gente che finallora l'aveva osteggiata. L'episodio
	fece svanire la sua paura di diventare, realmente e fino a un certo
	punto, Demostene. \emph{È più intelligente di quel che io e Peter
		avessimo mai pensato}, \emph{} si disse.}

{~}

\begin{center}
	{* * *}
\end{center}

{~}

{All'uscita dalla scuola trovò ad attenderla Graff. Era dall'altra parte
	della strada, appoggiato alla portiera della sua auto, e poiché
	indossava abiti civili ed era ancora aumentato di peso Valentine non lo
	riconobbe subito. Ma quando l'uomo le fece cenno di avvicinarsi ella
	trasalì; quegli occhi penetranti non erano cambiati affatto.}

{-- Guardi che non scriverò un'altra lettera -- gli disse subito. -- Non
	avrei neppure dovuto scrivere la prima.}

{-- Allora suppongo che non le piaccia ricevere medaglie.}

{-- Non molto.}

{-- Venga a fare un giretto con me, Valentine.}

{-- Non salgo in macchina con sconosciuti dall'aria equivoca.}

{Lui le porse un foglio. Si trattava di una richiesta legale in piena
	regola, un'autorizzazione firmata dai suoi genitori.}

{-- Ammetto che lei non si può definire uno sconosciuto -- sospirò lei.
	-- Dove intende portarmi?}

{-- A vedere un giovane soldato che risiede provvisoriamente a
	Greensboro, di passaggio.}

{Lei salì in macchina. -- Ender ha solo dieci anni -- disse. -- Quando
	lei lo prelevò, disse che non sarebbe venuto in licenza sulla Terra
	prima dei dodici anni.}

{-- Ha superato certi esami più in fretta del previsto.}

{-- Dunque sta andando bene?}

{-- Potrà chiederlo a lui personalmente.}

{-- Perché io? Perché non tutta la famiglia?}

{Graff sospirò. -- Ender vede il mondo a suo modo. Abbiamo dovuto
	persuaderlo a incontrare almeno lei. Per Peter e i vostri genitori non
	prova interesse. La sua vita alla Scuola di Guerra è stata\ldots{}
	intensa.}

{-- Che significa? È diventato pazzo?}

{-- Al contrario. È la persona più sana di mente che io conosca. Lo è
	abbastanza da capire che i suoi genitori soffrirebbero, più che gioire,
	riaprendo pagine di affetto che dovettero sforzarsi di chiudere con
	fermezza anni fa. In quanto a Peter, non gli abbiamo proposto
	d'incontrarlo; così non è stato costretto a mandare all'inferno degli
	ufficiali superiori a cui deve rispetto.}

{L'auto seguì la Lake Brandt Road girando intorno al vasto specchio
	d'acqua, e poi se ne scostò lungo una strada secondaria che andava su e
	giù fra le colline verdeggianti. Infine risalirono verso una grande
	villa rivestita in assicelle di legno che sorgeva in cima a un'altura.
	Dalla facciata si godeva il panorama del Lago Brandt, mentre sul retro
	il pendio declinava fino a un laghetto privato largo poche centinaia di
	metri.}

{-- La villa apparteneva a un magnate di Hollywood che vi mandava in
	vacanza i suoi divi, in caso di esaurimento nervoso -- spiegò Graff. --
	La F.I. l'ha affittata una ventina d'anni fa. Ender ha insistito che la
	vostra conversazione avvenga lontano da orecchi elettronici. Io gliel'ho
	promesso. Anzi, per corroborare la sua fiducia ho consentito che
	facciate un giretto su una zattera che ha costruito lui stesso. Devo
	avvertirla, però: intendo farle delle domande quando avrà finito di
	parlare con lui. Non la costringerò a rispondere, ma spero che lo farà.}

{-- Non ho portato con me un costume da bagno.}

{-- Possiamo fornirgliene un'intera scelta.}

{-- Tutti con microspie all'ultima moda balneare?}

{-- Credo che sia possibile intenderci sul terreno della fiducia
	reciproca. Ad esempio, io so chi è Demostene.}

{Lei provò un brivido di sorpresa e di paura, ma non disse nulla.}

{-- L'ho saputo fin dal mio recente rientro sulla Terra. Al mondo ci
	sono soltanto sei persone, credo, al corrente di questo. Senza contare i
	russi\ldots{} Dio solo sa dove arrivino i loro servizi segreti. Ma
	Demostene non ha niente da temere da noi. Demostene può confidare nella
	nostra discrezione. Proprio come io confido che Demostene non dirà a
	Locke cos'ha fatto e detto oggi. Mutua fiducia. E mutuo scambio
	d'informazioni.}

{Valentine non seppe stabilire se la loro approvazione andasse a
	Demostene o a Valentine Wiggin. Nel primo caso non poteva fidarsi di
	quella gente, nel secondo forse sì. La loro proposta di mantenere
	all'oscuro Peter poteva suggerire che conoscevano le loro differenze
	intellettuali. Ma Valentine non aveva ancora smesso di chiedersi se lei
	stessa conosceva davvero quelle differenze.}

{-- Ha avuto il tempo di costruirsi una zattera? Da quanto tempo è qui?}

{-- Due mesi. Nei nostri progetti questa vacanza doveva durare appena
	pochi giorni, ma\ldots{} vede, sembra che lui non sia più interessato a
	proseguire gli studi.}

{-- Ah! Così io sono ancora la terapia.}

{-- Stavolta non dovrà darci una lettera da censurare. Siamo disposti ad
	accollarci il rischio. Abbiamo bisogno di suo fratello. Molto bisogno. È
	un momento cruciale per la razza umana.}

{Dall'ultima volta, Valentine era cresciuta abbastanza per sapere che
	quelle parole non erano vuota retorica. Ed era stata Demostene
	abbastanza da assimilare un certo tipo di reazioni di fronte a un
	pericolo. -- Va bene. -- Scese dall'auto. -- È qui in casa?}

{Graff interrogò un inserviente con un'occhiata. -- È giù allo scalo
	delle barche -- rispose.}

{-- Vediamo questi costumi da bagnò, allora.}

{Ender non alzò una mano a salutarla quando la vide scendere lungo il
	sentiero che serpeggiava giù verso il lago, né sorrise allorché lei
	avanzò sul moletto accanto allo scivolo per le barche. Ma Valentine
	seppe che era felice di vederla, perché per tutto il tempo lui tenne gli
	occhi fissi nei suoi.}

{-- Sei più alto di quello che ricordavo -- gli disse, stupidamente.}

{-- Anche tu -- rispose lui. -- Ricordo anche che da bambina eri bella.}

{-- La memoria ci gioca strani scherzi.}

{-- No. Il tuo volto è uguale. Solo che a quell'età non capivo cosa
	fosse la bellezza. Vieni. Andiamo a galleggiare un po' sul lago.}

{Lei esaminò la piccola zattera con aria piuttosto dubbiosa.}

{-- Basta non alzarsi in piedi sul bordo -- disse Ender. Camminando a
	quattro zampe si portò all'estremità anteriore del natante. -- È la
	prima cosa che faccio con le mie mani, da quando tu e io ci costruivano
	le capanne con i blocchi di edilplast. Rifugi a prova di Peter.}

{Lei rise. Non aveva dimenticato quanto s'erano divertiti studiando
	piccole costruzioni capaci di reggere anche nel caso che qualcuno ne
	demolisse le più ovvie strutture portanti. Peter, al contrario, era
	stato un demonio d'abilità nel costruire capanne dall'apparenza
	solidissima che franavano addosso al primo abbastanza incauto da
	penetrarvi. Peter era stato un punto focale della loro infanzia,
	qualcosa che li aveva uniti.}

{-- Peter è cambiato -- disse lei.}

{Ender scrollò le spalle. -- Non voglio parlare di lui.}

{-- D'accordo.}

{La fanciulla salì sulla zattera, con movimenti assai più incerti di
	quelli di Ender. Lui usò una pagaia per manovrare intorno al molo e poi
	prese a remare verso il centro del laghetto privato. Nel notare quanto
	fosse abbronzato Valentine lo disse. -- E ti sei anche irrobustito molto
	-- aggiunse.}

{-- Alla Scuola di Guerra si fa molto esercizio fisico, ma
	l'abbronzatura l'ho comprata qui. Passo le giornate in acqua. Quando
	nuoto è come essere di nuovo lassù, in gravità zero. Senza peso si può
	volare, e ne sento la mancanza. Inoltre, qui sul lago, tutto il
	territorio che mi vedo attorno è ricurvo all'insù.}

{-- Come sul fondo di una tazza.}

{-- Ho vissuto in una tazza per quattro anni.}

{-- E ora noi siamo due sconosciuti?}

{-- Lo siamo, Val?}

{-- No -- disse lei. Allungò una mano e gli toccò un polpaccio. Poi
	d'improvviso gli fece il solletico dietro il ginocchio, proprio dove lui
	era sempre stato più sensibile.}

{Ma quasi all'istante lui le bloccò il polso. Aveva una stretta forte,
	benché le sue mani fossero più piccole di quelle di lei, e per un attimo
	nelle pupille gli brillò una luce strana, pericolosa. Poi si rilassò. --
	Ah, già -- disse. -- Avevi l'abitudine di farmi il solletico. Sei sempre
	così dispettosa?}

{-- No, non più -- mormorò lei, ritraendo la mano.}

{-- Ti va di nuotare?}

{Per tutta risposta Val si calò giù dal bordo della zattera. L'acqua era
	limpida e pulita, senza alcun odore di clorina. Per un poco nuotò
	attorno, poi risalì sulla zattera e pigramente si distese sotto la calda
	luce del sole. Una vespa ronzò su di lei e atterrò a un palmo di
	distanza dalla sua testa. La fanciulla non si mosse. Sapeva che
	l'insetto era lì, e che solitamente questo l'avrebbe spaventata.
	\emph{Ma non oggi. Lasciamo che vada in zattera anche lei, e che si
		abbronzi al sole come sto facendo io.}}

{Poi la zattera ebbe un sussulto. Lei si volse e vide Ender rialzare con
	calma la mano da dove l'aveva abbattuta e gettare in acqua la vespa. --
	Questi sono insetti dannati -- disse il ragazzo. -- Ti pungono anche
	senza esser stati provocati. -- Le sorrise. -- Ed è così che ci
	insegnano a difenderci: strategia preventiva. Io sono diventato un asso
	nelle loro battaglie simulate. Il miglior soldato che abbiamo mai
	avuto.}

{-- Chi poteva aspettarsi di meno? Sei un Wiggin.}

{-- Già. Qualunque cosa questo significhi.}

{-- Significa che tu puoi spingere il mondo in una certa direzione, se
	spingi nel posto adatto e nel momento adatto -- disse Val, e gli rivelò
	ciò che Peter e lei stavano facendo.}

{-- Quanti anni ha Peter? Quattordici? E pensa già di conquistare il
	mondo?}

{-- Crede d'essere Alessandro il Grande. E perché non potrebbe esserlo?
	Perché anche \emph{tu} non potresti esserlo?}

{-- Non potremmo essere \emph{tutti e due} Alessandro.}

{-- Due facce della stessa medaglia. E io, il metallo che ne compone
	l'interno. -- Ma subito Val dovette chiedersi fino a che punto lei fosse
	in posizione centrale. Aveva condiviso tante esperienze con Peter in
	quei pochi anni che perfino mentre lo disprezzava si rendeva conto di
	capirlo. Ender invece fino a quel momento era stato soltanto un ricordo:
	un ragazzino fragile e delicato che aveva bisogno della sua protezione.
	\emph{Non questo giovinetto abbronzato e dallo sguardo freddo, che
		schiaccia le vespe con le mani. Forse io e Peter e lui siamo fatti della
		stessa pasta, lo siamo sempre stati, e abbiamo voluto crederci diversi
		per orgoglio e per invidia.}}

{-- Il guaio con le medaglie è che la luce del sole può illuminare
	soltanto una faccia. L'altra sta all'ombra.}

\emph{{E proprio adesso tu credi di essere tornato all'ombra. }}{--
	Vogliono che io ti incoraggi a proseguire gli studi.}

{-- Non sono studi, sono gare. Nient'altro che gare, dall'inizio alla
	fine, solo che loro cambiano le regole quando e come gli salta in
	ticchio di farlo. -- Mosse le mani a dita aperte. -- Hai mai provato a
	far ballare una marionetta appesa ai fili?}

{-- Puoi tirare anche tu gli stessi fili che ti legano.}

{-- Soltanto se loro rilassano le dita. Soltanto se pensano che così ti
	stanno ancora usando. No, è troppo duro, è un gioco che non voglio
	giocare più. Appena comincio a sentirmi tranquillo, appena m'illudo di
	riuscire a padroneggiare le cose, mi piantano un altro coltello fra le
	costole. Da quando sono qui ho perfino degli incubi\ldots{} sogno di
	essere in sala di battaglia, solo che invece di lasciarmi volare senza
	peso loro mi costringono a combattere nella gravità, e le cambiano
	continuamente direzione, così non riesco mai ad atterrare dove voglio,
	mai ad andare dove ho deciso di andare. E allora li supplico di
	lasciarmi uscire dalla porta, ma loro mi parlano solo con le luci del
	loro computer, mi risucchiano lì dentro. Mi trasformano in un
	ingranaggio di quella macchina insensata.}

{Val sentì l'ira della sua voce, e la sentì diretta anche contro di lei.
	-- Già. Si presume che io sia qui per questo. Per spingerti di nuovo
	nella loro macchina.}

{-- Io non volevo incontrarti.}

{-- Me l'hanno detto.}

{-- Avevo paura di scoprire che ti voglio ancora bene.}

{-- Questo era ciò che io speravo.}

{-- La mia paura, la tua speranza\ldots{} altri due fili, per loro.}

{-- Non è del tutto vero, Ender. Siamo troppo giovani, forse, ma non
	senza potere. Abbiamo giocato tanto secondo le loro regole che questa è
	diventata la nostra partita. -- Ebbe una risatina. -- Io faccio
	addirittura parte di una commissione presidenziale. Peter non è riuscito
	a mandarla giù.}

{-- Loro non mi permettono contatti con la videostampa. Qui non c'è
	neppure un computer, a parte un vecchio barattolo che si occupa degli
	impianti di sicurezza e degli elettrodomestici. Roba istallata un secolo
	fa, quando facevano computer che non s'inserivano sui satelliti. Mi
	hanno tolto la mia orda, mi hanno tolto il banco, e la sai una cosa? Non
	è che me ne importi molto.}

{-- Tu sai star bene in compagnia di te stesso.}

{-- Io sono soltanto in compagnia dei miei ricordi.}

{-- Forse è questo che siamo: i nostri ricordi.}

{-- No. I miei ricordi degli \emph{altri.} Degli sconosciuti. Degli
	Scorpioni.}

{Valentine rabbrividì, come all'improvviso passaggio di una brezza
	fredda. -- Io ho smesso di guardare i video sugli Scorpioni. Sono sempre
	gli stessi.}

{-- Io li studiavo per ore. Il modo in cui le loro navi si muovono nello
	spazio. E ti dirò una cosa strana, che ho capito veramente solo
	standomene qui al sole sul lago: tutte le battaglie in cui gli Scorpioni
	e gli uomini si scontrano faccia a faccia, sono roba della Prima
	Invasione. Mentre in ogni scena ripresa durante la Seconda Invasione,
	con i nostri soldati nell'uniforme della F.I., gli Scorpioni che vi
	compaiono sono già tutti morti. Non uno che si veda combattere o
	muoversi. E la battaglia di Mazer Rackham\ldots{} non è in circolazione
	una sola ripresa di quell'avvenimento.}

{-- Forse usò un'arma segreta.}

{-- No, no, non sto a preoccuparmi del \emph{come} li abbia uccisi. È un
	problema di censura ingiustificata: non vogliono dirmi niente degli
	Scorpioni, e nello stesso tempo pretendono che un giorno o l'altro io
	vada a combatterli. Io mi sono battuto già molte volte in vita mia,
	talvolta per gioco e talvolta\ldots{} non per gioco. E ogni volta che
	sono riuscito a vincere è stato perché potevo capire i processi mentali
	dell'avversario da quello che \emph{facevano.} Riuscivo a stabilire cosa
	pensavano che io avrei fatto, e come immaginavano che sarebbe andata la
	battaglia. E giocavo su questo. Oh, ero diventato un esperto. Ottenere
	un risultato basandosi su ciò che pensano gli altri.}

{-- La maledizione dei fratelli Wiggin -- scherzò lei, ma la spaventò il
	pensiero che Ender potesse leggere in lei come faceva con i suoi nemici.
	Peter la sapeva scrutare in fondo all'anima, o almeno era convinto di
	farlo, ma lui era una tale sentina di depravazione che Val non poteva
	provare imbarazzo neppure quando lo vedeva intuire anche i suoi pensieri
	peggiori. Ender, invece\ldots{} da lui non sopportava d'esser scrutata
	così a fondo. Si sarebbe sentita nuda sotto i suoi occhi. Avrebbe avuto
	vergogna. -- Credi che non riusciresti mai a battere gli Scorpioni senza
	saper nulla di loro?}

{-- La cosa ha anche altre sfaccettature. Stando qui, da solo e coi
	lussi dell'ozio, ho potuto anche riflettere su me stesso. E sul perché
	odio tanto me stesso.}

{-- No, Ender\ldots{} non devi.}

{-- Non dirmi che \emph{non devo.} Ci ho messo molto a capire che era
	così, e credimi, mi odiavo. Mi odio. E sono arrivato a intuire questo:
	nel momento in cui io capisco davvero il mio avversario, abbastanza
	profondamente da poterlo battere, in quel preciso momento io comincio ad
	amarlo. Penso che sia impossibile conoscere una persona, ciò che è e ciò
	in cui crede, senza amarla come lei ama se stessa. Ed è proprio allora,
	nell'istante in cui sento di amare il mio nemico, che io\ldots{}}

{-- Lo sconfiggi -- terminò lei, e d'un tratto non ebbe più paura della
	sua capacità di leggere in lei.}

{-- No, non hai capito. Io lo \emph{distruggo.} Gli precludo ogni
	possibilità di assalirmi ancora. Lo calpesto e continuo a calpestarlo
	finché non \emph{esiste} più.}

{-- Stai esagerando, naturalmente. -- Ma in lei tornò la paura, più
	gelida di prima. \emph{Peter si è ammorbidito, e tu\ldots{} hanno fatto
		di te un killer. Due facce della stessa medaglia, ma chi è una faccia, e
		chi l'altra?}}

{-- Io ho fatto davvero del male a qualcuno, Val. Non esagero.}

{-- Lo so, Ender. -- \emph{Come farai del male a me?}}

{-- Vedi cosa sono diventato, Val? -- mormorò lui. -- Anche tu mi temi.
	-- E le sfiorò una guancia, così gentilmente che lei avrebbe voluto
	piangere. Con la stessa morbidezza che la sua mano di bambino aveva
	avuto un tempo. Nella pelle di lei era rimasto il ricordo di quelle
	tenere dita innocenti che le toccavano il viso.}

{-- Non ti temo -- disse, e in quel momento seppe che era vero.}

{-- Dovresti.}

\emph{{Non dovrò mai temerti. }}{-- Smettila di remare coi piedi
	nell'acqua. Finirai per attirare gli squali, lo sai.}

{Lui sorrise. -- Se vedi un'ombra passare sotto la zattera stai
	tranquilla, è un agente di Graff. -- Poi si finse spaventato al pensiero
	e tirò fuori di colpo le gambe, schizzando l'acqua attorno. Valentine
	fremette alle gocce fredde che le caddero sulla schiena.}

{-- Ender, Peter otterrà quello che vuole. È abbastanza intelligente da
	dar tempo al tempo, ma si è già aperto la strada per arrivare al potere;
	se non nei prossimi anni, in quelli futuri. Non sono ancora certa se
	sarà un bene oppure un male. Peter può essere crudele, ma conosce già
	l'arte di tenere gli altri sotto il suo controllo. E ci sono segni
	chiari che una volta finita la guerra contro gli Scorpioni, o forse
	addirittura prima, il mondo precipiterà di nuovo nel caos. Se il Patto
	di Varsavia e altre nazioni tornassero alle mire espansionistiche che
	avevano prima delle Invasioni\ldots{}}

{-- Forse perfino Peter potrebbe essere un'alternativa migliore.}

{-- Hai scoperto in te l'istinto del distruttore, Ender. Be', anch'io.
	Peter non ha il monopolio di questo, qualunque cosa pensino quelli che
	lo hanno esaminato. E dentro di sé ha qualcosa del costruttore. Non
	conosce la pietà, ma apprezza le cose buone\ldots{} se funzionano. E
	quando rifletti che il potere finisce sempre in mano a chi lo brama,
	vedi che in giro ci sono molti individui più crudeli o più stupidi di
	lui.}

{-- Con una raccomandazione di questo genere, anch'io potrei dargli il
	mio voto.}

{-- Qualche volta mi sembra tutto una follia. Un ragazzo quattordicenne
	e la sua sorellina che complottano per conquistare il mondo. -- Cercò di
	ridere, con uno sforzo. -- Non siamo ragazzini qualunque, è chiaro.
	Nessuno dei tre.}

{-- Non hai mai desiderato di esserlo, qualche volta?}

{Lei cercò d'immaginare se stessa che si comportava e parlava come le
	sue compagne di scuola, i cui interessi erano accentrati su ben altri
	argomenti che la politica e il futuro del pianeta. -- Sarebbe una gran
	noia.}

{-- Io non la penso così. -- E si allungò sulla zattera con le mani
	unite dietro la testa, come a dirle che avrebbe potuto restare lì
	disteso per sempre.}

{Dunque era vero, si disse Val. Qualunque cosa gli avessero fatto, la
	Scuola di Guerra aveva spento tutte le ambizioni di Ender. Adesso non
	desiderava altro che godersi quella polla d'acqua fra le colline baciate
	dal sole.}

\emph{{No}}{, \emph{} rifletté poi, \emph{no. Lui crede di non
		desiderare altro che questo, ma dentro di sé ha ancora troppo di Peter.
		O troppo di me. Nessuno di noi tre può essere felice restando con le
		mani in mano troppo a lungo. E nessun essere umano può vivere pienamente
		senza altra compagnia che se stesso.} Così il suo tono tornò a farsi
	sicuro e stimolante:}

{-- Qual è il nome che tutti conoscono, da un capo all'altro del mondo?}

{-- Mazer Rackham.}

{-- E cos'accadrebbe se tu vincessi la prossima guerra così come ha
	fatto lui?}

{-- Mazer Rackham è stato un Jolly. Una carta inaspettata. Nessuno gli
	aveva mai dato credito. Ebbe la fortuna di trovarsi nel posto giusto al
	momento giusto.}

{-- Ma supponi che succeda a te. Supponi di sconfiggere gli Scorpioni, e
	che il tuo nome diventi famoso come quello di Mazer Rackham.}

{-- Lasciamo che a diventare famoso sia qualcun altro. Peter vuole
	essere famoso? Be', mandiamo lui a salvare il mondo.}

{-- Non sto parlando della fama, Ender. E neppure del potere. Parlo
	delle probabilità favorevoli, proprio come quella che Mazer Rackham
	seppe sfruttare quando si trovò nel luogo e nel momento in cui questa
	probabilità esisteva.}

{-- Se io sarò qui -- disse Ender, -- in quel luogo non ci sarò io. Ci
	sarà qualcun altro. Lasciamo che quella probabilità se la goda lui.}

{Il suo tono di pigra indifferenza la fece infuriare. -- Io sto parlando
	della \emph{mia} vita. Tu\ldots{} piccolo bastardo egocentrico! -- Se
	quell'insulto lo urtò, non ne diede alcun cenno. Restò disteso dov'era,
	rilassato e ad occhi chiusi. -- Quando eri piccolo e Peter ti torturava,
	avrei dovuto mettermi le mani in tasca e aspettare che Mamma e Papà
	venissero a salvarti? Loro non hanno mai capito quanto Peter fosse
	pericoloso. Io sapevo che avevi il monitor, ma non ho mai aspettato che
	loro o altri intervenissero. E sai cosa mi faceva Peter quando gli
	impedivo di farti del male?}

{-- Taci! -- sussurrò Ender.}

{E fu perché vide il suo respiro accelerarsi, fu perché s'accorse di
	averlo ferito, fu perché seppe che proprio come Peter aveva trovato il
	suo punto più sensibile e glielo aveva colpito, fu per questo che
	Valentine tacque, tremando.}

{-- Io non posso batterli -- disse sottovoce Ender. -- Certo, un giorno
	o l'altro potrei andare ad affrontarli come un secondo Mazer Rackham.
	Con tutti quanti che si affidano a me. E non riuscirei a batterli.}

{-- Se non puoi tu, Ender, allora non potrà farlo nessuno. Se sai di non
	poterli sconfiggere, allora meritano di spazzarci via perché sono più
	forti e migliori di noi. E non sarà colpa tua.}

{-- Questo è maledettamente sicuro.}

{-- E se non tu, chi altro?}

{-- Chiunque.}

{-- Nessuno, Ender. E adesso ti dirò una cosa: se tu ci provi e perdi
	non sarà colpa tua. Ma se non ci provi, e se loro ci distruggono, allora
	questo peso graverà su di te. Perché sarai stato tu ad assassinarci.}

{-- Io ho l'anima di un assassino, in un caso o nell'altro.}

{-- E cos'altro ti illudevi di essere? Gli esseri umani non hanno
	evoluto il loro cervello per ciondolare intorno a laghetti ameni.
	Uccidere è la prima cosa che abbiamo imparato. E abbiamo dovuto imparare
	a farlo bene o morire, altrimenti oggi sarebbero le tigri dai denti a
	sciabola a dominare la Terra.}

{-- Io non potrei mai battere Peter. Non importa quel che ho detto o
	fatto. Non ci sono mai riuscito.}

\emph{{E così torniamo a Peter. }}{-- Lui era parecchio più grande di
	te, e più forte.}

{-- Anche gli Scorpioni lo sono.}

{Lei riusciva a sentire il suo modo di pensare. O piuttosto, l'ostacolo
	che glielo bloccava. Ender sapeva di poter vincere tutto, ma in fondo al
	cuore era certo che sarebbe rimasto qualcuno capace di distruggerlo. E
	non era mai stato convinto d'aver vinto davvero, perché alle sue spalle
	era rimasto Peter, il campione imbattuto.}

{-- Vuoi sconfiggere Peter?}

{-- No -- rispose lui.}

{-- Sconfiggi gli Scorpioni, e poi torna a casa e guarda chi si ricorda
	ancora dell'esistenza di Peter Wiggin. Guarda i suoi occhi quando tutto
	il mondo ti amerà e ti onorerà. Soltanto in essi, e soltanto allora,
	potrai leggere la sua sconfitta. E la tua vittoria.}

{-- Tu non capisci -- disse lui.}

{-- Sì, che capisco.}

{-- Non è così. Io non voglio distruggere Peter.}

{-- E allora cos'è che vuoi?}

{-- Desidero che lui mi voglia bene.}

{A questo Val non poté rispondere. Da quel che ne sapeva lei, Peter non
	aveva mai voluto bene a nessuno.}

{Ender non disse nient'altro. Si limitò a restare sdraiato, senza
	muoversi e senza riaprire gli occhi.}

{Dopo un po' di tempo Valentine si accorse che era quasi il tramonto, e
	che sciami di zanzare si stavano alzando in volo nelle zone in ombra.
	Raccolse la pagaia e la affondò nell'acqua, cominciando lentamente a
	spingere la zattera verso riva. Ender non diede segno d'accorgersi di
	quel che stava facendo, ma dal suo respiro Val capì che non dormiva.
	Quando furono allo scivolo delle barche saltò sul molo e si volse a
	guardarlo. -- Io ti voglio bene, Ender. Te ne vorrò sempre, qualunque
	cosa tu decida di fare.}

{Lui non rispose, e Val si disse che non aveva creduto una parola di
	quell'ultima frase. Si avviò su per il sentiero che risaliva la collina,
	angosciata e furibonda contro quelli che l'avevano costretta ad
	incontrare Ender lì e in quei termini. Perché, alla fine, lei aveva
	fatto proprio ciò che loro volevano. Aveva ancora risucchiato nel loro
	ingranaggio suo fratello, e sapeva che stavolta lui non l'avrebbe
	perdonata facilmente.}

{~}

\begin{center}
	{* * *}
\end{center}

{~}

{Ender rientrò dalla porta posteriore, ancora bagnato dopo il suo ultimo
	tuffo nel lago. All'interno della villa non c'era una luce accesa, e
	nell'oscurità del soggiorno trovò Graff ad aspettarlo.}

{-- Possiamo andarcene da qui? -- chiese Ender.}

{-- Se è questo che vuoi -- annuì Graff.}

{-- Quando?}

{-- Appena sei pronto.}

{Ender si fece una doccia e si vestì. Gli era parso piacevole
	riabituarsi a maneggiare e indossare abiti civili, ma ancora non si
	sentiva a suo agio senza un'uniforme o una tuta da battaglia. \emph{Non
		indosserò mai più una tuta da battaglia}, \emph{} rifletté. \emph{Quelle
		erano le gare della Scuola di Guerra, una cosa con cui ho chiuso.} Dalla
	finestra entrava il coro dei grilli che frinivano nel prato; in distanza
	ci fu il crepitio della ghiaia sotto i pneumatici di un'auto che usciva
	lentamente dalla rimessa.}

{Cos'altro avrebbe potuto portare con sé? Aveva letto parecchi dei libri
	contenuti nella piccola biblioteca, ma appartenevano alla casa e
	dovevano esser lasciati lì. La sola cosa di sua proprietà era la
	zattera, e anche quella sarebbe rimasta lì.}

{Al pianterreno le luci erano accese, e Graff si alzò nel vederlo
	comparire. Anche lui s'era cambiato. Indossava di nuovo l'uniforme.}

{Sedettero sul divano posteriore della macchina, e l'autista guidò a
	velocità moderata per le oscure strade di campagna verso l'aeroporto.
	Dopo un po' Graff disse: -- Un tempo, quando la popolazione aumentava
	ancora, mantennero questa zona a boschi e fattorie. È una terra ben
	irrigata, con una quantità di sorgenti e fiumiciattoli e molta acqua nel
	sottosuolo. Gli alberi hanno affondato le radici fin nel cuore della
	terra, rendendola viva. Ma noi ne abitiamo solo la superficie, come gli
	insetti che scivolano sul pelo dell'acqua in riva al lago.}

{Ender non disse nulla.}

{-- Noi addestriamo i nostri ufficiali perché imparino a pensare in un
	certo modo, e questo richiede che molti elementi della vita normale
	scompaiano dalla loro mente, perciò li isoliamo. Voi. Vi teniamo
	appartati. E la cosa funziona. Ma è così facile, quando non incontri mai
	gente, quando non senti il profumo della terra, quando vivi fra pareti
	metalliche oltre le quali c'è il gelo dello spazio, è così facile
	dimenticare che vale la pena di combattere e morire per questa Terra.
	Perché il nostro pianeta e la sua gente meritano che si paghi qualunque
	prezzo per salvarli.}

\emph{{Così è per questo che mi avete portato qui}}{, \emph{} pensò
	Ender. \emph{Con tutta la vostra fretta di agire, è per questo che mi
		avete regalato tre mesi in riva a un lago: per farmi amare la Terra.
		Be', ha funzionato. Tutti i vostri trucchi hanno funzionato. Anche
		Valentine. Anche lei uno stratagemma, per ricordarmi che non vado a
		scuola solo per me stesso. Bene, me lo ha ricordato.}}

{-- Può darsi che io abbia strumentalizzato Valentine -- disse Graff, --
	e che tu mi odi per questo, Ender. Ma non dimenticare una cosa: lei ha
	ottenuto un risultato perché quel che c'è fra voi due è importante, è
	autentico, è una cosa che vale. Miliardi di legami simili uniscono
	miliardi di esseri umani. È per questo che combattiamo.}

{Ender si volse al finestrino e guardò le luci degli aeromobili che
	decollavano o atterravano sul campo d'aviazione.}

{Un elicottero li portò allo spazioporto della F.I. a Stumpy Point. La
	base aveva un altro nome, quello di un Egemone morto anni addietro, ma
	tutti continuavano a chiamarla Stumpy Point, dalla piccola e misera
	cittadina che era stata spazzata via dalle distese di cemento e acciaio
	e plastica sorte sulla riva del Pamlico Sound. C'erano ancora stormi di
	anatre selvatiche che nidificavano nelle paludi salmastre, dove i salici
	si piegavano quasi ad abbeverarsi. Cominciò a cadere una pioggia leggera
	e le immense piste si fecero lucide e scure; era difficile capire dove
	lasciassero il posto alle acque della baia.}

{Graff lo condusse attraverso un labirinto di controlli. L'autorità
	dell'ufficiale era contenuta in una pallina di plastica che si portava
	dietro: la lasciava cadere entro canaletti inclinati, le porte si
	aprivano, uomini in divisa si alzavano e salutavano, la macchina
	risputava fuori la pallina e Graff tirava via diritto. Ender notò che
	dapprima tutti guardavano Graff, ma quando furono penetrati abbastanza
	nelle strutture dello spazioporto la gente cominciò a prestare più
	attenzione a lui. All'esterno avevano badato solo all'uomo e alla sua
	autorità, ma più avanti, dove tutti avevano un'autorità, era il suo
	carico umano a destare maggiore interesse.}

{Soltanto quando s'accorse che Graff si stava allacciando la cintura di
	sicurezza, seduto accanto a lui nella cabina della navetta, Ender capì
	che anche l'ufficiale lasciava la Terra.}

{-- Fin dove? -- gli chiese. -- Fin dove viaggerà con me?}

{Graff ebbe un breve sorriso. -- Per tutta la strada, Ender.}

{-- L'hanno promossa direttore della Scuola Ufficiali?}

{-- No.}

{Così avevano rimosso Graff dal suo incarico alla Scuola di Guerra col
	solo scopo di accompagnare lui in quel trasferimento. \emph{Sono
		importante fino a questo punto}, \emph{} si meravigliò Ender. E
	insinuante come un sussurro di Peter un pensiero lo attraversò: che
	vantaggi posso trarne?}

{Con un brivido cercò di pensare a qualcos'altro. Peter poteva cullarsi
	nei suoi sogni di potere, ma lui non aveva simili fantasie. Eppure,
	ripensando ai suoi anni alla Scuola di Guerra, dovette dirsi che aveva
	sempre avuto del potere sugli altri. Un potere legato al fatto di
	eccellere, e non già alla sua capacità di dominare il prossimo. Dunque
	non aveva motivo di vergognarsene. Mai, salvo che con Bean, aveva usato
	quel potere per ferire qualcuno. E anche con Bean le cose s'erano volte
	al meglio, dopotutto. Bean era diventato suo amico, prendendo il posto
	di Alai, che a sua volta aveva sostituito Valentine. Valentine, che
	stava aiutando Peter nei suoi piani segreti. Valentine, che gli avrebbe
	voluto bene qualunque cosa fosse accaduta. E seguendo quei pensieri si
	lasciò trasportare di nuovo sulla Terra, di nuovo alle ultime quiete ore
	di sole al centro del piccolo lago, nell'abbraccio delle colline
	boscose. \emph{Ed è questa la Terra}, \emph{} pensò. \emph{Non un globo
		lontano sospeso nello spazio, ma gli alberi che succhiano la linfa dalle
		rive di un lago colmo di riflessi, una casa seminascosta dalla
		vegetazione in cima a un'altura, un pendio erboso su cui il sentiero si
		vede appena, i pesci che sfiorano un attimo la superficie dell'acqua, il
		guizzo del martin pescatore che vola a catturare un insetto fra le
		canne.} E la voce di una fanciulla che gli parlava attraverso il sipario
	degli anni trascorsi. La stessa voce che un tempo lo aveva rassicurato e
	consolato. La stessa voce a cui lui avrebbe impedito di spegnersi ad
	ogni costo, anche tornando a scuola, anche lasciando la Terra per altri
	quattro o altri quattromila anni. \emph{Anche se lei vuole più bene a
		Peter.}}

{I suoi occhi erano chiusi, e l'unico suono che gli usciva dalle labbra
	era stato il respiro; tuttavia Graff si sporse attraverso il passaggio
	centrale e gli poggiò una mano su un braccio. Ender trasalì sorpreso.
	Subito sentì la mano dell'uomo ritrarsi, ma per un attimo fu come
	folgorato dalla stupefacente intuizione che forse Graff provava un certo
	affetto per lui. Ma no, doveva essere un altro dei suoi gesti
	maledettamente calcolati. Graff stava fabbricando un comandante, pezzo
	dopo pezzo, a partire da un ragazzino. Senza dubbio nel suo manuale di
	istruzioni un paragrafo prevedeva: Comma-17/carezza affettuosa
	dell'insegnante sull'arto superiore destro del soggetto.}

{La navetta impiegò poche ore a raggiungere il satellite AIP. Attracco
	Inter-Planetario era una città di tremila abitanti, che respiravano
	l'ossigeno prodotto dalle stesse piante di cui si nutrivano, bevendo
	un'acqua già passata mille volte attraverso i loro corpi, e vivevano
	soltanto al servizio dei rimorchiatori che facevano il grosso dei
	trasporti merci nel sistema solare e delle navette che portavano
	passeggeri fra la Terra e la Luna. Era un mondo dove Ender poté sentirsi
	a casa per un poco, dato che i pavimenti s'incurvavano all'insù come
	alla Scuola di Guerra.}

{I loro rimorchiatori erano tutti nuovi fiammanti; la F.I. non faceva
	che togliere di circolazione i velivoli sorpassati per sostituirli con
	modelli più potenti e veloci. Quello su cui salirono aveva appena
	scaricato una gran quantità di lingotti d'acciaio fusi su un'astronave
	mineraria che raccoglieva minerale sulla Cintura degli Asteroidi.
	L'acciaio era stato scaricato in caduta libera sulla Luna, e ora il
	rimorchiatore s'era agganciato a quattordici chiatte. Ma Graff mise di
	nuovo la sua pallina in un lettore, e le chiatte furono rimandate in
	deposito sullo scalo. Sarebbe stato un viaggio più lungo stavolta, e per
	una destinazione che Graff aveva ordine di specificare soltanto dopo che
	il rimorchiatore avrebbe lasciato l'Attracco I.P.}

{-- Non è poi un gran segreto -- disse il comandante del rimorchiatore.
	-- Quando si parte per una destinazione «sconosciuta» è sempre per
	l'AIS. -- Per analogia con la sigla AIP, Ender si disse che questa
	doveva significare Attracco Inter-Stellare.}

{-- Non questa volta -- lo informò Graff.}

{-- Per dove, allora?}

{-- Comando F.I.}

{-- Non ho una qualifica di sicurezza abbastanza alta da sapere dove si
	trova, signore.}

{-- La sua astronave lo sa -- disse Graff. -- Lasci che il computer dia
	un'occhiata a questa, e seguirà una rotta già programmata. -- Porse al
	comandante la pallina di plastica.}

{-- E si suppone che durante il viaggio io tenga gli occhi chiusi, per
	ignorare ufficialmente dove stiamo andando?}

{-- Oh, no. Naturalmente no. Il Comando F.I. è sul planetoide Eros, vale
	a dire a circa tre mesi di viaggio da qui procedendo alla massima
	velocità possibile. Lei non dovrà fare risparmio sul carburante.}

{-- Eros? Ma credevo che gli Scorpioni l'avessero ridotto a una massa
	radioattiva di\ldots{} ah! E quando ho ricevuto la qualifica di
	sicurezza necessaria per sapere questo?}

{-- Non l'ha ricevuta. Presumo perciò che al nostro arrivo lei verrà
	assegnato in servizio permanente su Eros.}

{Il comandante strinse i denti. -- Ma che razza di figlio di puttana è
	lei? -- ringhiò. Ender pensò che le sue mani avrebbero afferrato Graff
	per il petto. -- Io sono un pilota! E voialtri non avete nessun diritto
	di sbattermi su un pezzo di roccia!}

{Graff non batté ciglio. -- Signore, sta cercando di convincere un
	ufficiale superiore a farle rapporto per insubordinazione? -- L'altro
	gli volse le spalle di scatto. Dopo qualche momento lui continuò: -- Non
	sono tenuto a offrirle la mia comprensione. Comunque, i miei ordini sono
	di requisire il mezzo di trasporto più veloce, e al momento questo è il
	suo. La consiglio di prenderla con filosofia. Del resto, la guerra
	potrebbe finire entro i prossimi quindici anni\ldots{}}

{-- Lo dica a mia moglie! È ausiliaria nella Sussistenza, a Orbit-Uno!}

{-- \ldots{} e al termine di questo periodo, ovviamente, la dislocazione
	dei nostri alti comandi non sarà più un segreto. Inoltre sarà bene che
	la informi sin d'ora che giunti a Eros il suo equipaggio non dovrà fare
	avvicinamento visuale, ma strumentale. Eros è stato oscurato, e la sua
	albedo è all'incirca quella di un buco nero. In quanto a sua moglie,
	sarà fatta salire a bordo di uno dei prossimi mezzi che seguiranno la
	nostra stessa rotta.}

{-- Grazie -- borbottò il comandante. -- Anche a suo nome.}

{Occorse circa un mese di viaggio prima che il comandante del
	rimorchiatore tornasse a rivolgere la parola a Graff.}

{Il computer di bordo aveva una biblioteca limitata, libri e film il cui
	scopo non era tanto di fornire istruzione quanto divertimento
	all'equipaggio. Così, per ingannare il tempo dopo la colazione e gli
	esercizi fisici mattutini, Ender e Graff presero l'abitudine di
	chiacchierare. Sulla Scuola Ufficiali. Sulla Terra. Sull'astronomia, la
	fisica, o altri argomenti che il ragazzo desiderava approfondire.}

{E ciò che lui soprattutto voleva erano notizie sugli Scorpioni.}

{-- Non ne sappiamo poi molto -- gli disse Graff. -- Non abbiamo mai
	potuto esaminarne uno vivo. Anche quando si riuscì a intrappolarne uno,
	disarmato e in apparenza sano, lui morì al momento della cattura.
	Perfino il \emph{lui} è incerto: sembra infatti probabile che la maggior
	parte degli Scorpioni combattenti siano femmine, ma con organi sessuali
	atrofizzati o mai sviluppati. Non possiamo dirlo con certezza. Ciò che
	ti sarebbe più utile è la loro psicologia, e nessuno ha mai avuto la
	possibilità di intervistarne uno.}

{-- Mi dica quello che sa, e forse riuscirò a ricavarne qualche dato
	utile.}

{Graff gliene parlò diffusamente. A detta degli studiosi, gli Scorpioni
	erano organismi che avrebbero potuto evolversi anche sulla Terra stessa,
	se nel periodo Cretaceo o nel Giurassico le cose fossero andate in modo
	diverso. A livello molecolare non presentavano sorprese; perfino il loro
	materiale genetico funzionava con gli stessi meccanismi. Ma non era un
	caso se agli occhi umani sembravano grossi insetti: benché i loro organi
	interni fossero più complessi e specializzati di qualunque altro
	insetto, ed avessero perso parte dell'esoscheletro per sviluppare
	un'autentica struttura ossea, la loro forma fisica riecheggiava ancora
	quella dei loro antenati, che probabilmente erano stato molto simili a
	formiche munite di pinze anteriori e coda aculeata. -- Ma non
	confonderti con queste ipotesi -- disse Graff. -- Hanno la stessa
	plausibilità di quelle che potrebbero fare loro su di noi, se
	deducessero che gli uomini discendono dagli scoiattoli.}

{-- Se è tutto qui quello su cui possiamo basarci, e pur sempre
	\emph{qualcosa} -- disse Ender.}

{-- Gli scoiattoli non costruirebbero mai astronavi -- osservò Graff. --
	Occorrerebbero troppi mutamenti sulla strada che corre fra il
	raccogliere noccioline e il raccogliere asteroidi o stabilire stazioni
	di ricerca sulle lune di Saturno.}

{Sembrava probabile che gli Scorpioni vedessero nello stesso spettro
	d'onde a cui erano sensibili gli occhi umani, poiché c'erano luci
	artificiali nelle loro astronavi e nelle istallazioni che costruivano al
	suolo. Le loro antenne dovevano essere vestigia sopravvissute
	all'evoluzione, e non sembravano possedere organi dell'udito né
	dell'odorato né recettori tattili o gustativi. -- Ovviamente non
	possiamo esserne sicuri. Ma alla dissezione non risulta nessun organo
	capace di emettere suoni. E la cosa più strana è che sulle loro
	astronavi non è stato mai trovato alcun apparato per la comunicazione.
	Niente radio o TV, niente che potesse trasmettere o ricevere qualsiasi
	tipo di segnale.}

{-- Comunicano da nave a nave. Ho visto i filmati, ed è chiaro che
	possono parlare fra loro.}

{-- Vero. Ma corpo a corpo, mente a mente. Questa è la cosa più
	importante che abbiamo appreso di loro: la comunicazione, comunque essa
	avvenga, è istantanea. O immensamente superiore alla velocità della
	luce. Allorché Mazer Rackham sconfisse la loro flotta d'invasione, tutti
	gli altri distaccamenti o avamposti chiusero bottega. All'istante. Non
	fu diramato nessun segnale di carattere fisico. Ogni loro attività
	cessò.}

{Ender ripensò ai filmati che mostravano Scorpioni in apparenza sani che
	giacevano dove la morte li aveva colti.}

{-- Fu allora che sapemmo, dinnanzi all'evidenza, che la comunicazione a
	velocità ultraluce era possibile. Questo accadde settant'anni fa. E una
	volta certi che la cosa poteva esser fatta, riuscimmo a realizzarla in
	pratica. Non \emph{io}, \emph{} intendo. Io non ero ancora nato.}

{Ender era stupefatto. -- Com'è possibile una cosa simile?}

{-- Non posso neppure cominciare a spiegarti la fisica filotica. È una
	scienza per metà ancora fuori dalla comprensione umana. Ciò che conta è
	che abbiamo costruito l'\emph{ansible.} Il termine ufficiale è
	Comunicatore Istantaneo di parallasse Filotico, ma qualcuno ha tirato
	fuori il nome ansible da un vecchio romanzo e gliel'ha appioppato. Non
	che siano molti a conoscere l'esistenza di questo apparecchio.}

{-- Questo significa che le astronavi possono comunicare fra loro anche
	dai lati opposti del sistema solare -- disse Ender.}

{-- Significa che possono farlo all'istante attraverso tutta la
	galassia. E gli Scorpioni ci riescono senza bisogno di apparecchiature.}

{-- Così hanno saputo della loro sconfitta nel momento stesso in cui è
	avvenuta -- rifletté Ender. -- Io pensavo\ldots{} tutti hanno sempre
	detto che sul loro mondo ne sono venuti a conoscenza soltanto
	venticinque anni fa.}

{-- Questo è servito a prevenire il panico -- annuì Graff. -- Ti sto
	dando informazioni che teoricamente neppure tu potrai portare fuori dal
	Comando della F.I. se mai dovessi partirne prima della fine della
	guerra.}

{-- Se lei mi conoscesse bene -- s'irritò Ender, -- saprebbe che sono
	capace di mantenere un segreto.}

{-- È il regolamento. Chiunque sia al di sotto dei venticinque anni è
	considerato un rischio per la sicurezza. Questo è ingiusto verso molti
	giovani meritevoli, ma aiuta a restringere il numero di coloro che
	potrebbero dare origine a una fuga di notizie.}

{-- Ma a che scopo tutta questa segretezza?}

{-- Perché\ldots{} ci siamo assunti un rischio terribile, Ender, e se la
	videostampa ne fosse a conoscenza ci sarebbe una mezza rivoluzione con
	conseguenze imprevedibili. Vedi, appena realizzato l'ansible lo montammo
	sulle nostre migliori astronavi ed esse partirono, con l'obiettivo di
	attaccare i sistemi solari abitati dagli Scorpioni.}

{-- Sappiamo dove si trovano?}

{-- Sì.}

{-- Dunque non stiamo aspettando la Terza Invasione.}

{-- La Terza Invasione \emph{siamo noi.}}

{-- Li stiamo attaccando! Nessuno ne ha mai fatto parola. Tutti sono
	convinti che le nostre flotte siano appostate fuori dei confini del
	sistema solare per\ldots{}}

{-- Non ce n'è una. Siamo praticamente senza difese, qui.}

{-- Che accadrebbe se mandassero una flotta ad attaccarci?}

{-- Allora siamo morti. Ma le nostre astronavi non hanno ancora
	avvistato una flotta simile, neppure un sospetto.}

{-- Forse hanno rinunciato e si sono decisi a lasciarci in pace.}

{-- Forse. Ma tu hai visto i filmati. Saresti disposto a scommettere
	l'esistenza della razza umana sulla possibilità che loro abbiano
	rinunciato ad aggredirci?}

{Ender cercò di fare un calcolo del tempo che poteva esser trascorso. --
	E le nostre navi hanno viaggiato per settant'anni\ldots{}}

{-- Alcune sì. Altre per trent'anni, e altre ancora per venti. Oggi
	costruiamo astronavi più veloci. Stiamo imparando a cavarcela un po'
	meglio nello spazio. Ma ogni nave che non sia ancora in cantiere sta
	viaggiando verso uno dei pianeti degli Scorpioni, o un loro avamposto.
	Ogni nave, con le stive piene di missili e di astrocaccia, è là fuori
	verso il suo bersaglio. E stanno decelerando. Perché sono quasi a
	destinazione. Le prime astronavi furono mandate contro gli obiettivi più
	lontani, e le successive verso altri pianeti più vicini. Il nostro
	calcolo del tempo è stato abbastanza buono. Tutte quante arriveranno sul
	loro bersaglio con uno scarto di pochi mesi l'una dall'altra.
	Sfortunatamente i nostri mezzi bellici meno progrediti stanno per
	attaccare proprio il loro mondo d'origine. Tuttavia sono equipaggiati
	piuttosto bene\ldots{} abbiamo alcune nuove armi che gli Scorpioni non
	hanno mai visto.}

{-- Quando arriveranno?}

{-- Entro i prossimi cinque anni, Ender. Tutto è già pronto al Comando
	F.I. L'ansible principale è là, in contatto con la nostra flotta
	d'invasione; le navi sono in pieno assetto di guerra. Tutto quello che
	ci manca, Ender, è un comandante in campo. Qualcuno che sappia cosa
	diavolo fare quando quelle astronavi dovranno entrare in azione.}

{-- E se nessuno fosse all'altezza delle vostre aspettative?}

{-- Faremmo del nostro meglio. Col miglior comandante che riusciremo a
	trovare.}

\emph{{Io}}{, \emph{} pensò Ender. \emph{Vogliono che io sia pronto in
		cinque anni.} -- Colonnello Graff, non c'è una sola possibilità che per
	allora io sia in grado di comandare una flotta.}

{Graff si strinse nelle spalle. -- Tu fai del tuo meglio. Se non sarai
	pronto, useremo i comandanti che abbiamo.}

{Questo confuse i pensieri di Ender.}

{Ma solo per un momento. -- Naturalmente, come avrai capito, finora non
	ne abbiamo neppure uno.}

{Ender sapeva che quello era un altro dei giochetti di Graff.
	\emph{Convincimi che tutto dipende da me, così lascerò che tu mi tenga
		alla frusta, così ci darò dentro fino a spezzarmi la schiena.}}

{Gioco o no, tuttavia, l'obiettivo era reale. E perciò lui avrebbe
	lavorato il più duramente possibile. Era a questo che Val aveva voluto
	spingerlo. \emph{Cinque anni. Soltanto cinque anni prima che la flotta
		arrivi là, e ancora non so niente di niente.} -- Avrò appena quindici
	anni -- mormorò.}

{-- Quasi sedici -- disse Graff. -- Tutto dipenderà dalle nozioni che
	avrai acquisito.}

{-- Sa una cosa, colonnello? Mi piacerebbe tornare a nuotare in quel
	laghetto.}

{-- Dopo che avremo vinto la guerra -- disse Graff. -- Oppure persa. In
	tal caso disporremo di qualche decina d'anni prima che arrivino a
	spazzarci via del tutto. Ma se la villa ci sarà ancora, ti prometto che
	potrai andare in zattera fino alla nausea.}

{-- Ma sarò sempre troppo giovane per avere una qualifica di sicurezza.}

{-- Ovvio, ma noi militari sappiamo come aggirare questi inconvenienti:
	vuol dire che le zattere che costruirai saranno top secret.}

{Entrambi risero, e Ender dovette ricordare a se stesso che Graff stava
	soltanto indossando l'abito dell'amico, e che tutte le sue azioni erano
	calcolate per trasformare lui in una macchina efficiente.
	\emph{Diventerò esattamente lo strumento che tu vuoi}, \emph{} disse
	dentro di sé, \emph{ma per lo meno non mi farete fesso. Andrò avanti
		perché questa è la mia scelta e non perché tu stai qui a manovrarmi,
		grosso bastardo d'un volpone.}}

{Il rimorchiatore si fermò nell'orbita di Eros prima che Ender potesse
	vedere il planetoide. Fu il comandante a mostrarglielo su uno schermo
	collegato a un visore a infrarossi. Gli stavano praticamente accanto --
	a circa quattrocento chilometri -- ma Eros, una montagna lunga
	ventiquattromila metri, non rifletteva che una minima frazione della
	luce solare e in parte sfuggiva anche al radar.}

{Il comandante attraccò a una delle tre piattaforme di sosta che
	orbitavano attorno a Eros. Quella era la distanza minima per il
	rimorchiatore, poiché sul planetoide c'erano impianti per la gravità
	artificiale e manovrare entro un campo di 0,5 G richiedeva agilità
	invece di potenza. L'uomo li salutò senza alcuna cordialità, ma questo
	non guastò il morale dei suoi due passeggeri: se al capitano seccava
	esser finito lì, Ender e Graff si sentivano come due galeotti all'uscita
	del penitenziario. Dopo che furono trasbordati sulla navetta che li
	avrebbe portati sulla superficie di Eros, risero di gusto ripensando al
	comandante e alla verbosità con cui Graff s'era impegnato solennemente a
	farlo raggiungere dalla moglie. L'uomo non aveva mostrato il minimo
	entusiasmo. E ancor meno entusiasta ne era stata la brunetta che
	lavorava sul rimorchiatore come ufficiale di rotta. Soltanto allora,
	girandosi a guardare fuori dal finestrino, Ender si rilassò abbastanza
	da dar voce a un'ultima domanda:}

{-- Perché siamo in guerra con gli Scorpioni?}

{-- Ho sentito ipotesi di ogni genere -- disse Graff. -- Perché hanno
	problemi di sovrappopolazione e devono colonizzare; perché non
	sopportano l'idea di dividere l'universo con altre specie intelligenti;
	perché non pensano che \emph{noi} siamo una forma di vita intelligente;
	perché hanno una religione fanatica e selvaggia; perché hanno ricevuto
	le nostre trasmissioni televisive e deciso che siamo dei pazzi
	criminali\ldots{} e chi più ne ha più ne metta.}

{-- Lei cosa crede?}

{-- Poco importa ciò che credo io.}

{-- Vorrei saperlo lo stesso.}

{-- Loro comunicano in modo assoluto, Ender, mente a mente. Ciò che uno
	pensa diventa il pensiero di un altro, ciò che uno ricorda diventa il
	ricordo di un altro. Perché avrebbero dovuto sviluppare un linguaggio? A
	cosa servirebbe loro leggere e scrivere, quando possono vedere e sapere
	tutto attraverso le menti degli altri? Lo stesso nostro concetto di
	comunicazione dev'essere estraneo a dei telepatici. Dunque non si
	tratterebbe di tradurre dal nostro linguaggio al loro, perché non
	posseggono neppure il concetto stesso di linguaggio. E altrettanto
	inutile sarebbe cercare di contattarli con i più diversi mezzi di
	segnalazione, poiché la cosa per loro non avrebbe significato. E magari
	loro hanno cercato di contattarci telepaticamente, e non hanno capito
	perché mai non abbiamo risposto.}

{-- Così la guerra è scoppiata perché non potevamo parlarci?}

{-- Se incontri qualcuno che non può farti capire in nessun modo chi è e
	cosa pensa, non sarai mai sicuro che non cercherà di ammazzarti.}

{-- Cosa succederebbe se li lasciassimo cuocere nel loro brodo?}

{-- Ender, non siamo stati noi ad andare a casa loro. Sono venuti qui.
	Se avessero intenzioni pacifiche ce lo avrebbero fatto capire evitando
	di invadere il nostro sistema.}

{-- Forse non hanno capito che siamo una specie intelligente.
	Forse\ldots{}}

{-- Ender, credimi, si è discusso per un secolo di quest'argomento.
	Nessuno conosce la risposta. E quando si torna al punto, la decisione da
	prendere può essere una sola: se una delle due razze dev'essere
	distrutta, meglio assicurarsi maledettamente bene che non sia la nostra.
	La stessa eredità genetica umana ci preclude altre scelte. La natura non
	lascia evolvere specie prive dell'istinto di sopravvivenza. L'individuo
	singolo può decidere di sacrificare la sua vita, ma la razza nel suo
	insieme non può mai scegliere il rischio dell'estinzione. Così, se ci
	riusciremo, stermineremo gli Scorpioni dal primo all'ultimo; nello
	stesso modo in cui loro, potendo, distruggerebbero noi.}

{-- In quanto a me -- disse Ender, -- voto a favore della
	sopravvivenza.}

{-- Lo so -- annuì Graff. -- È per questo che sei qui.}

\phantomsection\label{Orsonux20Scottux20Cardux20-ux20Ilux20Giocoux20Diux20Enderux20-ux20BY_SLY70A1_split_016.htm}{}
