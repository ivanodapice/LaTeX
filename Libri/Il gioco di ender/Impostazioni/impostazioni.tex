%----------DOCUMENT SETTINGS----------%

\documentclass[
a5paper,
twoside, 	% il documento sarà pubblicato con stampa su entrambi i lati della carta
onecolumn, 	% Una sola colonna di testo.
openany, 	% Un capitolo può iniziare sia su una pagina di fronte che su una di retro.
final 		% Per una copia pronta alla stampa.
]{memoir} 
% openright 	% Ogni capitolo inizierà su una pagina di fronte.
% openleft  	% Ogni capitolo inizierà su una pagina di retro.
% 13pt, 		% Dimensione del testo.
% article 		% La tipografia simula la classe dell'articolo.
% twocolumn		% Due colonne di testo.
% draft 		% In questo modo le linee strabordanti vengono contrassegnate da barre nere e si possono visualizzare alcuni segni di modifica. Possono verificarsi anche altri effetti, in particolare se si utilizzano alcuni pacchetti.
%showtrims 		% Stampa i segni agli angoli del foglio in modo da poter vedere dove la carta deve essere tagliata per ottenere il formato finale della pagina.

%----------DOCUMENT SETTINGS----------%

%----------------FONTS----------------%

\usepackage[utf8x]{inputenc} 			% Se la codifica è utf8.
\usepackage[T1]{fontenc}    			%
\usepackage{fontspec} 					%
\usepackage[italian]{babel} 			% Capitoli e altri testi in italiano.
\usepackage[
final,
protrusion=true,
expansion=true
]{microtype} 															% Meno badbox.
\usepackage{cjhebrew} 													% Necessario per scrivere testo in alfabeto ebraico.
																		% https://ctan.mirror.garr.it/mirrors/ctan/language/hebrew/cjhebrew/cjhebrew.pdf
\usepackage{garamondlibre} 												% Modifica del carattere.
																		% Controlla: https://tug.org/FontCatalogue/
%\renewcommand*\familydefault{\sfdefault} 								% se il font di base è sans serif.

%----------------FONTS----------------%

%---------------PACKAGES--------------%

\usepackage{amsmath,amssymb,mathtools} 	% Math.
\usepackage{ragged2e} 					% Permette \centering e hyphenation.
\usepackage{tikz} 						% Figures.
\usepackage{graphicx,calc} 				% Include figures.
\usepackage{svg}

%---------------PACKAGES--------------%

%--------------PAGE LAYOUT------------%

%\setlrmarginsandblock{0.1\paperwidth}{*}{1} 	% Margine destro e sinistro.
%\setulmarginsandblock{0.1\paperwidth}{*}{1}  	% Margine superiore e inferiore.
%\checkandfixthelayout[fixed]{} 				% Ho messo questo per sicurezza, per assicurare che il layout precedente sia fisso.

%\usepackage[paperwidth=148mm, paperheight=210mm,bindingoffset=0mm]{geometry}

\setlrmarginsandblock{10mm}{18mm}{*}  	% {left}{right}{ratio}
\setulmarginsandblock{22mm}{*}{*} 		% {top}{bottom}{ratio}
\setheaderspaces{*}{*}{1.618}
\checkandfixthelayout 

% Se vuoi verticalizzare il testo con più spazio:

\usepackage{setspace}

%\OnehalfSpacing								% Prima di \begin{document} influenzi l'intero documento
												% Tra \begin{document} e \maketitle influenza l'intero documento
												% Dopo \maketitle influisce su tutto il documento tranne che sul titolo
												
% O in alternativa
												
%\linespread{1.25}								% Prima di \begin{document}, influenza l'intero documento
												% Tra \begin{document} e \maketitle influenza solo il titolo
												% Dopo \maketitle non ha effetto
% O in alternativa

%\spacing{1.25}									% Prima di \begin{document} influenza l'intero documento
												% Tra \begin{document} e \maketitle influenza l'intero documento
												% Dopo \maketitle influisce su tutto il documento tranne che sul titolo

%--------------PAGE LAYOUT------------%

%-------------CHAPTER STYLE-----------%

\chapterstyle{thatcher}
% https://ctan.mirror.garr.it/mirrors/ctan/macros/latex/contrib/memoir/memman.pdf
% Pagina 389: Showcases
% default , section , hangnum , companion , article, demo

%-------------CHAPTER STYLE-----------%

%----------FLOATS AND CAPTIONS--------%

\makeatletter                  	% Non è necessario scrivere sempre [htpb].
\renewcommand\fps@figure{htbp} 	%
\renewcommand\fps@table{htbp}  	%
\makeatother                   	%
\changecaptionwidth          	% Cambia la larghezza della didascalia.
\captionwidth{1\textwidth}		%


%----------FLOATS AND CAPTIONS--------%

%---------------ABSTRACT--------------%

\renewcommand{\abstractnamefont}{\normalfont\small\bfseries} 	% Font del titolo dell'abstract.
\setlength{\absleftindent}{0.1\textwidth} 						% Larghezza dell'abstract.
\setlength{\absrightindent}{\absleftindent}

%---------------ABSTRACT--------------% 

%-----------TABLE OF CONTENTS---------%

\maxtocdepth{subsection} 					% Solo le parti, i capitoli e le sezioni dell'indice.
\settocdepth{subsection}
%\AtEndDocument{\addtocontents{toc}{\par}} 	% Aggiungere un paragrafo alla fine del TOC.

%-----------TABLE OF CONTENTS---------%

%-------------COLORED BOXES-----------%

\usepackage[noframe]{showframe}
\renewenvironment{shaded}
  	{%
		\def\FrameCommand{\fboxsep=\FrameSep \colorbox{shadecolor}}%
		\MakeFramed{\advance\hsize-\width \FrameRestore\FrameRestore}
	}%
{\endMakeFramed}
\usepackage{framed}                          % 
\definecolor{shadecolor}{gray}{0.75}         %  
\usepackage{lipsum,stackengine}              %
\setstackEOL{\\}

%-------------COLORED BOXES-----------%

%----------------LETTRINE-------------%

\begin{filecontents}{Zallman.sty}
	\NeedsTeXFormat{LaTeX2e}
	\ProvidesPackage{Zallman}[2007/11/24 v1.0 Zallman CFR]
	\input Zallman.fd
	\DeclareRobustCommand{\Zallmanfamily}{
		\fontencoding{U}%
		\fontseries{xl}%
		\fontshape{n}%
		\fontfamily{Zallman}%
		\selectfont}
	\DeclareTextFontCommand{\zall}{\Zallmanfamily}
	\endinput
	%% end Zallman.sty
\end{filecontents}

\usepackage{enumitem,Zallman,Acorn,GoudyIn,lettrine,textalpha,GoudyIn,needspace,textcase,xpatch}	% Pacchetto DropCap.
\renewcommand\LettrineFontHook{\color{red}\Zallmanfamily} 											% Pacchetto DropCap per usare lettrine.
\pretocmd{\lettrine}{\checklettrine}{}{}
\newcommand{\checklettrine}{%
	\ifnum\prevgraf<2 \vspace{\baselineskip}\fi
}
\SingleSpacing*
%----------------LETTRINE-------------%

%----------------DEDICA---------------%

\newenvironment{dedica}
{%\clearpage           		% nuova pagina.          
	\thispagestyle{empty}	% no header e footer.
	\vspace*{\stretch{1}}	% un po' di spazio in alto.
	\itshape             	% il testo è in corsivo.
	\raggedleft          	% sposta sul margine destro.
}
{\par 						% fine del paragrafo.
	\vspace{\stretch{3}} 	% lo spazio in basso è tre volte quello in alto.
	\clearpage           	% fine della pagina.
}

%----------------DEDICA---------------%

%--------------PAGESTYLE--------------%

\pagestyle{headings} 	% Scelta dello stile di pagina e dello stile del capitolo.
						% empty , plain , headings , ruled , Ruled , companion
						% https://ctan.mirror.garr.it/mirrors/ctan/macros/latex/contrib/memoir/memman.pdf
						% Pagina 156: 7. PAGINATION AND HEADERS
%\makeevenhead{ruled}{\leftmark}{}{}
%\makeoddhead{ruled}{}{}{\leftmark}
%\makeevenfoot{ruled}{}{\thepage}{}
%\makeoddfoot{ruled}{}{\thepage}{}
%\pagestyle{ruled}

%--------------PAGESTYLE--------------%

%---------------TINYTEXT--------------%
\newlength\myheight
\newlength\mydepth
\settototalheight\myheight{Xygp}
\settodepth\mydepth{Xygp}
\setlength\fboxsep{0pt}
\newcommand*\inlinegraphics[1]{%
	\settototalheight\myheight{Xygp}%
	\settodepth\mydepth{Xygp}%
	\raisebox{-\mydepth}{\includegraphics[height=\myheight]{#1}}%
}

\newcommand*{\img}[1]{%
	\raisebox{-.3\baselineskip}{% 		% Questi sono due metodi diversi per inserire delle foto
		\includegraphics[				%  vicino al testo mantenendo la stessa grandezza dei
		height=\baselineskip,			%  caratteri. 
		width=\baselineskip,        	% Per usare il primo metodo:
		keepaspectratio,				% \raisebox{-\mydepth}{\fbox{\includegraphics[height=\myheight]{imm.png}}}
		]{#1}%							% Secondo metodo:
	}%									% \img{imm.png}
}
%---------------TINYTEXT--------------%

%-----------=-----BLANK---------------%

\usepackage{afterpage}
	
\newcommand\blankpage{					% Per aggiungere una pagina bianca dopo quella di inserimento
	\null								%
	\thispagestyle{empty}				% Scrivere \afterpage{\blankpage} dove vuoi che ci sia una pagina bianca dopo
	\addtocounter{page}{-1}
	\newpage
}

%-----------=-----BLANK---------------%