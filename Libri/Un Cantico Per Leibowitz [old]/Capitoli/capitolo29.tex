	\chapter{\phantom{text}}

\lettrine{C}{redo} che siate già stato messo in guardia contro questi scoppi
d\textquotesingle ira --- chiese padre Lehy al penitente.

--- Sì, padre.

--- Vi rendete conto che l\textquotesingle intento era relativamente
omicida?

--- Non c\textquotesingle era nessuna intenzione di uccidere.

--- State cercando di scusarvi? --- domandò il confessore.

--- No, padre. L\textquotesingle intenzione era di fargli male. Mi
accuso di aver violato lo spirito del Quinto Comandamento con il
pensiero e con l\textquotesingle azione, e di aver peccato contro la
carità e la giustizia. E di aver portato disgrazia e scandalo sul mio
ufficio.

--- Vi rendete conto di aver infranto la promessa di non ricorrere mai
alla violenza?

--- Sì, padre. Me ne pento profondamente.

--- E l\textquotesingle unica circostanza attenuante è che avete visto
rosso e avete colpito. Perdete spesso il controllo in questo modo?

L\textquotesingle interrogatorio continuò; il superiore
dell\textquotesingle abbazia era in ginocchio, e il priore sedeva come
un giudice al di sopra del suo maestro.

--- Benissimo --- disse alla fine padre Lehy. --- Ora, per penitenza,
promettete di dire\ldots{}

Zerchi arrivò alla cappella con un\textquotesingle ora e mezzo di
ritardo, ma la signora Grales lo stava ancora aspettando. Era
inginocchiata in un banco vicino al confessionale, e sembrava
addormentata. Imbarazzato com\textquotesingle era,
l\textquotesingle abate aveva sperato che lei non ci fosse. Doveva
recitare la sua penitenza, prima di poterla ascoltare. Si inginocchiò
vicino all\textquotesingle altare e trascorse venti minuti recitando le
preghiere che padre Lehy gli aveva assegnato come penitenza per quel
giorno, ma quando si mosse verso il confessionale, la signora Grales era
ancora lì. Le parlò due volte prima che lei l\textquotesingle udisse; e
quando si alzò, incespicò. Si fermò per tastare il viso di Rachel,
esplorandone le palpebre e le labbra con le dita avvizzite.

--- C\textquotesingle è qualcosa che non va, figliola? --- chiese
l\textquotesingle abate.

La donna levò lo sguardo verso le alte finestre. I suoi occhi vagarono
sul soffitto a volta. --- Sì, padre --- sussurrò. --- Sento il Maligno
qui in giro, davvero. Il Maligno è vicino, molto vicino a noi, qui.
Sento il bisogno del perdono, padre\ldots{} e anche di
qualcos\textquotesingle altro.

--- Qualcos\textquotesingle altro, signora Grales?

Lei si avvicinò, per sussurrare, dietro la mano. --- Ho bisogno di
perdonare Lui, anche.

Il prete si ritrasse, leggermente. --- Chi? Non capisco!

Perdonare\ldots{} Colui che mi ha fatta come sono --- gemette. Ma poi un
lieve sorriso le schiuse la bocca. --- Io\ldots{} io non Lo avevo mai
perdonato, per questo.

--- Perdonare Dio\ldots{} Come potete voi\ldots{} Egli è giusto. Egli è
la Giustizia, Egli è l\textquotesingle Amore. Come potete dire\ldots{}

Gli occhi della donna erano supplichevoli. --- E una vecchia donna che
gira vendendo i pomodori non può perdonarlo un po\textquotesingle{} per
la Sua Giustizia? Prima che io chieda il Suo perdono per me?

Don Zerchi deglutì. Guardò l\textquotesingle ombra bicefala sul
pavimento. Alludeva a una terribile Giustizia\ldots{} la forma di
quell\textquotesingle ombra. Non poteva indursi a rimproverarla per aver
scelto quella parola, \emph{perdonare}. Nel suo semplice mondo, era
concepibile il perdonare la giustizia come perdonare
l\textquotesingle ingiustizia, era concepibile che
l\textquotesingle Uomo perdonasse Dio come Dio perdonava
l\textquotesingle Uomo. ``Così sia, allora, e abbi pazienza con lei, o
Signore, pensò, aggiustandosi la stola.''

La donna si genuflesse davanti all\textquotesingle altare, prima di
entrare nel confessionale, e il religioso notò che, quando si fece il
segno della croce, la sua mano toccò la fronte di Rachel, oltre alla
sua. Scostò la pesante cortina, entrò nel confessionale e sussurrò
attraverso la griglia.

--- Cosa volete, figliola?

--- Benedizioni, padre, perché io ho peccato\ldots{}

Parlò a scatti. Non la poteva vedere, attraverso la reticella che
copriva la griglia. C\textquotesingle era solo il lamento basso e
ritmico di una voce di Eva. Lo stesso, lo stesso, sempre lo stesso, e
neppure una donna con due teste poteva escogitare nuovi modi di
corteggiare il male, ma poteva seguire soltanto una ottusa imitazione
del Peccato Originale. Poiché provava ancora vergogna del suo
comportamento con la ragazza, gli agenti e Cors, gli era difficile
concentrarsi. Eppure, le mani gli tremavano mentre ascoltava. Il ritmo
delle parole diventava sordo e sommesso, attraverso la griglia, come il
ritmo di un martellare lontano. Chiodi infissi nelle palme trapassavano
il legno. Come un alter Christus, sentiva il peso di ogni fardello, per
un attimo, prima che passasse a Colui che li portava tutti. Era qualcosa
che riguardava il compagno di lei. Erano cose sordide e segrete, cose da
avvolgere in un giornale sporco e da seppellire durante la notte. Il
fatto che riuscisse a comprenderle solo in parte, sembrava peggiorarne
l\textquotesingle orrore.

--- Se state cercando di dirmi che siete colpevole di aborto ---
sussurrò --- devo avvertirvi che l\textquotesingle assoluzione è
riservata al vescovo e che io non posso\ldots{}

Si interruppe. C\textquotesingle era un ruggito lontano, e il debole
grugnito di missili lanciati dalla base.

--- Il Maligno! Il Maligno! --- gemette la vecchia.
L\textquotesingle abate si sentì accapponare la pelle del capo: un
brivido improvviso di irragionevole allarme.

--- Presto! Un atto di contrizione! --- mormorò. --- \emph{Dieci Pater,
	Ave e Gloria} per penitenza. Dovrete ripetere la confessione, più tardi,
ma adesso un atto di contrizione.

La udì mormorare, dall\textquotesingle altra parte della griglia. In
fretta sussurrò una assoluzione: --- T\emph{e absolvat Dominus Jesus
	Christus: ego autem eius auctoritate te absolvo ab omni vin-culo\ldots{}
	Denique, si absolvi potes, ex peccatis tuis ego te absolvo in Nomine
	Patris\ldots{}}

Prima che finisse, una luce splendette attraverso la spessa cortina del
confessionale. La luce divenne più fulgida e più fulgida, fino a che il
confessionale fu pieno del chiarore di mezzogiorno. La cortina cominciò
a fumare.

--- Aspettate! --- sibilò. --- Aspettate ché si spenga.

--- Aspettate aspettate aspettate che si spenga --- fece eco una strana
voce sommessa, oltre la griglia. Non era la voce della signora Grales.

--- Signora Grales? Signora Grales?

Lei rispose in un mormorio assonnato, con la lingua spessa. --- Non ho
mai voluto\ldots{} non ho mai voluto\ldots{} mai amore\ldots{}
Amore\ldots{} --- Il mormorio si smorzò. Non era la stessa voce che gli
aveva risposto un attimo prima.

--- E adesso, presto, correte!

Senza aspettare di vedere se lei l\textquotesingle aveva ascoltato, si
lanciò fuori del confessionale e lungo la corsia, verso
l\textquotesingle altare. La luce si era smorzata, ma bruciava ancora la
pelle con lo splendore meridiano. ``Quanti secondi rimanevano?'' La
chiesa era piena di fumo.

Entrò nel santuario, inciampò nel primo gradino, definì questo una
genuflessione, e andò all\textquotesingle altare. Con mani frenetiche,
tolse il ciborio pieno di Cristo dal tabernacolo, si genuflesse di nuovo
davanti alla Presenza, afferrò il Corpo di Dio e corse.

L\textquotesingle edificio gli crollò addosso.

Quando rinvenne, non c\textquotesingle era altro che polvere. Era
inchiodato al suolo fino alla cintura. Giaceva sul ventre; cercò di
muoversi. Un braccio era libero, ma l\textquotesingle altro era
prigioniero del peso che l\textquotesingle inchiodava. La mano libera
stringeva ancora il ciborio, che si era rovesciato nella caduta; il
coperchio si era staccato, spargendo intorno molte Ostie.

Pensò che l\textquotesingle esplosione l\textquotesingle avesse
scagliato fuori della chiesa. Giaceva sulla sabbia, e vedeva i resti di
un rosaio travolto dalla frana delle pietre. Una rosa era ancora
attaccata a un ramo\ldots{} una delle Armene color salmone. I petali
erano bruciacchiati. C\textquotesingle era un grande ruggito di motori,
nel cielo, e luci azzurre continuavano ad ammiccare, attraverso la
polvere.

Dapprima non sentì dolore. Cercò di girare il collo per dare
un\textquotesingle occhiata al mostro che lo teneva prigioniero, ma poi
cominciò a soffrire. Gli occhi gli si annebbiarono. Gridò,
sommessamente. Non si sarebbe più voltato. Cinque tonnellate di pietra
lo tenevano inchiodato. Tenevano fermo ciò che rimaneva di lui, al di
sotto della cintura.

Cominciò a raccogliere le Ostie. Mosse il braccio libero, goffamente.
Con cautela, le raccolse, a una a una, dalla sabbia. Il vento minacciava
di spargere intorno quelle minuscole scaglie di Cristo. ``Comunque,
Signore, io ho tentato, pensò. C\textquotesingle è qualcuno che ha
bisogno degli ultimi riti? Del Viatico? Dovranno trascinarsi fino a me,
in questo caso. Ma è rimasto qualcuno?''

Non poteva udire alcuna voce, al di sopra del terribile ruggito.

Un rivolo di sangue cominciò a scorrergli negli occhi. Lo asciugò con
l\textquotesingle avambraccio, per non macchiare le Ostie con le dita
insanguinate. Non è il vero sangue, Signore: è il mio, non il Tuo.
\emph{Dealba me.}

Rimise quasi tutte le Ostie nel ciborio, ma qualcuna era troppo lontana.
Si tese per prenderle, ma svenne di nuovo. --- Gesù Giuseppe Maria,
aiuto!

Debolmente udì una risposta, lontana e scarsamente udibile, sotto il
cielo ululante. Era la strana voce sommessa che aveva udito nel
confessionale, e anche questa volta echeggiava le sue parole:

--- Gesù Giuseppe Maria, aiuto!

--- Cosa? --- gridò.

Chiamò parecchie volte, ma non giunse altra risposta. La polvere
cominciava a cadere. Richiuse il coperchio del ciborio per impedire che
la polvere si mescolasse con le Ostie. Giacque immobile per qualche
tempo, con gli occhi chiusi.

``Il guaio, quando si è un prete, è che bisogna accettare i consigli che
si danno agli altri. La natura non ci impone niente che non ci abbia
messo in grado di sopportare. È quello che ottengo per aver detto a
quella ragazza ciò che disse lo Stoico, prima di dirle ciò che disse
Dio'' pensò.

Non provava dolore, solo un feroce prurito che veniva dalla sua metà
imprigionata. Cercò di grattarsi; le sue dita incontrarono soltanto la
pietra nuda. L\textquotesingle artigliò, per un momento, rabbrividì, poi
tolse la mano. Il prurito lo faceva impazzire. I nervi lesi
lampeggiavano folli richieste di una grattata. Si sentiva molto poco
dignitoso.

``Ebbene, dottor Cors, come sapete se il prurito non è un male più
fondamentale della sofferenza?''

Rise un poco, a quel pensiero. La risata provocò un improvviso
svenimento. Si fece strada a unghiate fuori
dall\textquotesingle oscurità, perché qualcuno gridava. Improvvisamente
si accorse che era lui a gridare. Zerchi ebbe paura. Il prurito si era
trasformato in dolore, ma le grida erano state di primitivo terrore, non
di sofferenza. Adesso soffriva persino a respirare. La sofferenza
persisteva, ma poteva sopportarla. Lo spavento era sorto da
quell\textquotesingle ultimo assaggio di oscurità simile
all\textquotesingle inchiostro. L\textquotesingle oscurità pareva
incombere su di lui, sorvegliarlo, aspettarlo avidamente\ldots{} un
grande appetito nero con una passione per le anime. Poteva sopportare la
sofferenza, ma non quella Spaventosa Oscurità. O in essa
c\textquotesingle era qualcosa che non avrebbe dovuto esservi, o
c\textquotesingle era ancora qualcosa, qui, che doveva essere fatto. Una
volta che si fosse arreso a quell\textquotesingle oscurità, non vi
sarebbe più stato nulla che avrebbe potuto fare o disfare.

Si vergognò della sua paura e tentò di pregare, ma le preghiere
sembravano diverse dalle preghiere\ldots{} simili a scuse, ma non
petizioni, come se l\textquotesingle ultima preghiera fosse già stata
detta, l\textquotesingle ultimo canto già cantato. La paura persisteva.
Perché? Cercò di ragionare. Hai già visto altra gente morire, Jeth.
Sembra facile. Si spengono, c\textquotesingle è un piccolo spasmo, ed è
finita. Quell\textquotesingle Oscurità
d\textquotesingle inchiostro\ldots{} abisso fra \emph{aham} e
Asti\ldots{} lo Stige più nero, abisso fra Dio e l\textquotesingle Uomo.
Ascolta, Jeth, credi davvero che vi sia qualcosa
sull\textquotesingle altra riva, non è vero? E allora perché tremi così?

Un versetto del \emph{Dies Irae} gli galleggiò nella mente, lo tormentò:

\emph{Quid sum miser tune dicturus?}

\emph{Quem patronum rogaturus}

\emph{Cum vix justus sit securus?}

``Che cosa dirò, io miserabile? Chi chiamerò come protettore, quando a
malapena l\textquotesingle uomo giusto sarà sicuro?'' \emph{Vix
	securus?} Perché \emph{a malapena sicuro}? Senza dubbio Egli non dannerà
il giusto. E allora perché tremi così?

"Davvero, dottor Cors, il male di cui persino voi avreste dovuto parlare
non era la sofferenza, ma l\textquotesingle irragionevole paura della
sofferenza. \emph{Metus doloris.} Mettetelo insieme al vostro
equivalente positivo, la ricerca per la sicurezza, mondana, per
l\textquotesingle Eden, e avrete la vostra \emph{radice del male},
dottor Cors. Minimizzare la sofferenza e massimizzare la sicurezza erano
i fini naturali e giusti della società e di Cesare. Ma poi ne
diventavano gli unici fini, e l\textquotesingle unico fondamento della
legge\ldots{} una perversione. Inevitabilmente, allora, nel cercare
soltanto quello, noi troviamo soltanto l\textquotesingle opposto:
massima sofferenza e minima sicurezza.

``Il guaio del mondo sono io. Provatelo su voi stesso, mio caro Cors. Tu
io Adamo l\textquotesingle Uomo noi. Non c\textquotesingle è male
mondano eccetto quello che è stato introdotto nel mondo
dell\textquotesingle Uomo\ldots{} io tu Adamo noi\ldots{} con un piccolo
aiuto da parte del padre delle menzogne. Biasima qualunque cosa, biasima
persino Dio, ma, oh, non biasimare me. Dottor Cors?
L\textquotesingle unico male del mondo, ormai, è che il mondo non esiste
più. Che dolore ha portato?''

Rise debolmente, ancora, e questo riportò l\textquotesingle inchiostro.

--- Me noi Adamo, ma Cristo, Uomo me: me noi Adamo, ma Cristo, Uomo me
--- disse a voce alta. --- Sapete una cosa, Pat?\ldots{} loro\ldots{}
insieme\ldots{} preferiscono esservi inchiodati, ma non da soli\ldots{}
quando sanguinano\ldots{} vogliono compagnia. Perché\ldots{} perché è
così. Perché è per questo che Satana vuole l\textquotesingle Uomo pieno
di Inferno. Voglio dire, è per questo che Satana vuole
l\textquotesingle Inferno pieno di Uomini. Perché Adamo\ldots{} Eppure
Cristo\ldots{} Ma io\ldots{} Ascoltate, Pat\ldots{}

Questa volta occorse più tempo per scacciare l\textquotesingle Oscurità,
ma era necessario che lo spiegasse a Pat, prima di sprofondarvi.

--- Ascoltate, Pat, perché\ldots{} perché sono stato a dirvi che la
bambina doveva\ldots{} è perché io. Voglio dire. Voglio dire che Gesù
non chiese mai a un uomo di fare una sola cosa che Gesù non fece. Ma è
lo stesso, perché io\ldots{} Perché non posso lasciar perdere, Pat?

Batté le palpebre, più volte. Pat svanì. Il mondo si congelò di nuovo e
l\textquotesingle oscurità scomparve. In qualche modo aveva scoperto di
che cosa aveva paura. C\textquotesingle era qualcosa che doveva compiere
prima che quella Oscurità si chiudesse sopra di lui. ``Mio Dio, lasciami
vivere abbastanza per compierlo.'' Aveva paura di morire prima di aver
accettato tanta sofferenza quanta si era abbattuta sulla bimba che non
la poteva capire, la bimba che aveva tentato di salvare per
un\textquotesingle ulteriore sofferenza\ldots{} no, non per quella, ma a
dispetto di quella. Aveva comandato la madre in nome di Cristo. Non
aveva sbagliato. Ma ora aveva paura di scivolare in
quell\textquotesingle oscurità prima di aver sopportato quanto Dio
poteva aiutarlo a sopportare.

\emph{Quern patronum rogaturus,}

\emph{Cum vix Justus sit securus?}

``Sia per la bambina e per sua madre, allora. Ciò che io ho imposto, io
devo accettare. Fas est.''

La decisione sembrò diminuire il dolore. Giacque, quietamente, per
qualche tempo, poi, cautamente, guardò di nuovo dietro di sé, al mucchio
di pietre. Erano più di cinque tonnellate. C\textquotesingle erano
diciotto secoli lì. L\textquotesingle esplosione aveva aperto le cripte,
perché notò alcune ossa bloccate fra le rocce. Tese la mano libera,
incontrò qualcosa di liscio, e finalmente riuscì a liberarlo. Lo lasciò
cadere sulla sabbia, accanto al ciborio. Mancava la mandibola, ma il
cranio era intatto, a eccezione di un foro sulla fronte, da cui spuntava
una scheggia di legno secco e semi putrefatto. Sembrava
l\textquotesingle avanzo d\textquotesingle una freccia. Il cranio pareva
molto antico.

--- Fratello --- sussurrò, perché nessuno, tranne un monaco
dell\textquotesingle Ordine, poteva essere stato sepolto in quelle
cripte. Che cosa hai fatto per loro, Osso? Hai insegnato loro a leggere
e a scrivere? Li hai aiutati a ricostruire, hai dato loro Cristo, li hai
aiutati a restaurare una civiltà? Hai ricordato di avvertirli che non
avrebbe mai potuto essere un Eden? Naturalmente lo hai fatto. Sii
benedetto, Osso, pensò, e tracciò un segno della croce sulla sua fronte
con il pollice. Per tutte le tue fatiche, ti hanno ripagato con una
freccia fra gli occhi. Perché qui c\textquotesingle è ben più di cinque
tonnellate e diciotto secoli di pietre. Immagino che vi siano circa due
milioni di anni, là\ldots{} sin dal primo \emph{Homo inspiratus}.

Udì di nuovo la voce\ldots{} la sommessa voce-eco che gli aveva risposto
poco prima. Questa volta venne in una specie di cantilena infantile:
\emph{``la-la-la, la-la-la\ldots''}.

Anche se pareva la stessa voce che aveva udito nel confessionale, senza
dubbio non poteva essere la signora Grales. La signora Grales aveva
perdonato Dio ed era corsa a casa, se era uscita dalla cappella in
tempo\ldots{} e ti prego di perdonare il rovesciamento, Signore. Ma non
era sicuro neppure di aver rovesciato la frase. Ascolta, Vecchio Osso,
avrei dovuto parlare così a Cors? Ascoltate, mio caro Cors, perché non
perdonate a Dio di permettere la sofferenza? Se Egli non la permettesse,
il coraggio, la nobiltà, l\textquotesingle abnegazione umana sarebbero
cose prive di significato. Inoltre, voi sareste senza lavoro, Cors.

Forse è questo che abbiamo dimenticato di dire, Osso. Bombe e collere,
quando il mondo è amareggiato, perché rimane privo di un Eden ricordato
a metà. L\textquotesingle amarezza era essenzialmente contro Dio.
Ascolta, Uomo, devi rinunciare all\textquotesingle amarezza\ldots{}
``concedere perdono a Dio'', direbbe lei\ldots{} prima di qualunque
altra cosa\ldots{} prima di amare.

Ma le bombe e le collere. Quelle non perdonano.

Dormì, un poco. Era un sonno naturale e non l\textquotesingle orribile
nulla dell\textquotesingle Oscurità che afferrava la mente. Cadde una
pioggia che cancellò la polvere. Quando si svegliò, non era solo. Levò
la guancia dal fango e le guardò, bruscamente. Erano tre, posate sul
mucchio di macerie e lo guardavano con funerea solennità. Si mosse.
Distesero le ali nere e sibilarono, nervosamente. Gettò contro di loro
un pezzetto di pietra. Due si levarono e volarono in cerchio, ma la
terza rimase dov\textquotesingle era, zampettando e sbirciandolo
gravemente. Un uccello scuro e brutto, ma non simile
all\textquotesingle Altra Oscurità. Questo desiderava soltanto il corpo.

--- Il pranzo non è ancora pronto, fratello uccello --- disse irritato.
--- Dovrai aspettare.

Non avrebbe dovuto pensare a molti pasti, osservò, prima che
l\textquotesingle uccello diventasse il pasto per qualcun altro. Le sue
penne erano bruciacchiate dalla vampata, e teneva un occhio chiuso.
L\textquotesingle uccello era fradicio di pioggia, e
l\textquotesingle abate pensò che anche la pioggia era piena di morte.

--- \emph{la-la-la, la-la-la\ldots{}} aspettate aspettate aspettate che
si spenga\ldots{}

La voce ritornò. Zerchi aveva temuto che fosse una allucinazione. Ma
anche l\textquotesingle uccello l\textquotesingle udiva. Continuava a
sbirciare qualcosa al di fuori della portata dello sguardo di Zerchi.
Alla fine emise un sibilo rauco e prese il volo.

Aiuto! --- gridò, debolmente.

--- \emph{aiuto} --- pappagallò la strana voce.

E la donna con due teste comparve, girando attorno a un mucchio di
macerie. Si fermò e guardò Zerchi.

--- Grazie a Dio! Signora Grales! Guardate se potete trovare padre
Lehy\ldots{}

--- grazie a Dio signora Grales guardate se potete\ldots{} Batté le
palpebre per rimuovere il sangue dagli occhi e l\textquotesingle osservò
attentamente.

--- Rachel --- mormorò.

--- \emph{rachel} --- rispose la creatura.

Si inginocchiò davanti a lui e si appoggiò sui calcagni.
L\textquotesingle osservò con freschi occhi verdi e sorrise
innocentemente. Gli occhi erano desti, carichi di stupore, di
curiosità\ldots{} e forse di qualcosa d\textquotesingle altro\ldots{} ma
a quanto pareva non sembrava capire che lui soffriva.
C\textquotesingle era qualcosa, in quegli occhi, che gli impedì di
notare qualunque altra cosa per parecchi secondi. Ma poi notò che la
testa della signora Grales dormiva sonoramente
sull\textquotesingle altra spalla, mentre Rachel sorrideva. Era un
sorriso giovane e timido, che sperava amicizia.

Ritentò.

--- Sentite, è rimasto vivo qualcun altro? Andate\ldots{}

Melodiosa e solenne venne la sua risposta: --- sentite è rimasto vivo
qualcun altro\ldots{} --- Lei assaporava le parole. Le enunciava
distintamente. Sorrideva su di esse. Era più che
un\textquotesingle imitazione riflessa, decise. Stava cercando di
comunicare qualcosa. Attraverso quella ripetizione, stava cercando di
portargli l\textquotesingle idea: \emph{Io sono un po\textquotesingle{}
	simile a te.}

Ma lei era appena nata.

``E tu sei anche diversa, in un certo senso'' notò Zerchi, con una
sfumatura di timore. Ricordava che la signora Grales aveva
l\textquotesingle artrite alle ginocchia, ma il corpo che le era
appartenuto adesso stava inginocchiato là, appoggiato sui calcagni,
nella sciolta posa della gioventù. E poi, la pelle rugosa della vecchia
sembrava meno grinzosa di prima, e sembrava splendere un poco, come se
il vecchio tessuto coriaceo fosse di nuovo vivificato. Improvvisamente
notò il braccio di lei.

--- Sei ferita!

--- \emph{sei ferita.}

Zerchi indicò il braccio di lei. Invece di guardare dove lui. indicava,
imitò il suo gesto, guardandogli il dito e tendendo il proprio per
toccarlo\ldots{} servendosi del braccio ferito. C\textquotesingle era
pochissimo sangue, ma c\textquotesingle erano almeno una dozzina di
tagli, uno dei quali sembrava profondo. Zerchi le tirò il dito, per
avvicinare il braccio. Ne trasse cinque schegge di vetro rotto. Forse
lei aveva spinto il braccio attraverso una finestra o, più
probabilmente, si era trovata sulla traiettoria d\textquotesingle una
finestra che esplodeva, quando c\textquotesingle era stato lo scoppio.

Solo una volta, quando tolse una scheggia di vetro lunga un pollice
apparve una traccia di sangue. Quando tolse gli altri frammenti,
lasciarono minuscoli segni azzurri, senza emorragia:
Quell\textquotesingle effetto gli ricordò una dimostrazione di ipnosi
cui aveva assistito una volta, di qualcosa che aveva rifiutato come
un\textquotesingle impostura. Quando guardò di nuovo il volto di lei, il
suo timore crebbe. Continuava a sorridergli, come se la rimozione delle
schegge di vetro non le avesse causato alcun fastidio.

Lanciò un\textquotesingle occhiata al viso della signora Grales. Era
diventata grigia, l\textquotesingle impersonale maschera del coma. Le
labbra erano esangui. In qualche modo, fu certo che stava morendo.
Poteva immaginarla mentre avvizziva e alla fine cadeva come una crosta o
un cordone ombelicale. Chi era, dunque, Rachel? E che cosa?

C\textquotesingle era ancora un po\textquotesingle{} di umidità sulle
pietre bagnate dalla pioggia. Inumidì un polpastrello e le fece cenno di
avvicinarsi. Qualunque cosa fosse, probabilmente aveva ricevuto una dose
di radiazioni troppo forte per vivere a lungo. Cominciò a tracciarle una
croce sulla fronte con un dito umido.

--- \emph{Nisi baptizata es et nisi baptizari nonquis, te baptizo..}

Non riuscì ad andare oltre. Lei si scostò in fretta da lui. Il suo
sorriso gelò e svanì. No! sembrava gridare la sua espressione. Si
allontanò da lui. Si asciugò dalla fronte la traccia di umidità, chiuse
gli occhi, abbandonò le mani in grembo. Una espressione di completa
passività apparve sul suo viso. Con il capo piegato in quel modo faceva
pensare a una preghiera. Gradualmente uscì dalla passività; il sorriso
rinacque. Quando apri gli occhi e lo guardò di nuovo, lo fece con lo
stesso aperto calore di prima. Ma si guardava attorno, come se cercasse
qualcosa.

Il suo sguardo cadde sul ciborio. Prima che Zerchi potesse fermarla, lo
raccolse. --- No! --- tossì lui, con voce rauca, e cercò di afferrarlo.
Lei era troppo svelta, e lo sforzo gli costò uno svenimento. Quando
ritornò alla coscienza e alzò di nuovo il capo, riuscì a vedere solo
immagini confuse. Lei era ancora inginocchiata, lì davanti. Finalmente
riuscì a capire che reggeva la coppa d\textquotesingle oro nella mano
sinistra, e nella destra, delicatamente, fra pollice e indice,
un\textquotesingle Ostia. La stava offrendo a \emph{lui}, oppure
l\textquotesingle immaginava soltanto, come poco prima aveva immaginato
di parlare a frate Pat?

Aspettò che le immagini confuse si schiarissero. Questa volta non si
schiarirono, non completamente.

--- \emph{Domine, non sum dignus\ldots{}} --- sussurrò --- \emph{sed
	tantum dic verbo\ldots{}}

Ricevette l\textquotesingle Ostia dalla mano di lei. Lei richiuse il
coperchio del ciborio e lo ripose in un punto più protetto, sotto una
pietra sporgente. Non aveva usato i gesti convenzionali, ma la reverenza
con cui l\textquotesingle aveva maneggiato lo convinse
d\textquotesingle una cosa: lei sentiva la Presenza sotto i veli. Colei
che non sapeva ancora usare le parole o comprenderle, aveva fatto ciò
che aveva fatto come per \emph{istruzione diretta}, in risposta al suo
tentativo di battesimo.

Cerco di rimettere a fuoco gli occhi per dare un\textquotesingle altra
occhiata al viso di quell\textquotesingle essere, che per mezzo dei soli
gesti gli aveva detto: ``Io non ho bisogno del vostro primo Sacramento,
Uomo, ma io sono degna di impartirti questo Sacramento di Vita''. Ora
sapeva chi ella era, e singhiozzò debolmente quando non riuscì a
rimettere a fuoco gli occhi su quei freschi, verdi occhi imperturbati di
una nata libera.

--- \emph{Magnificat anima mea Dominum} --- sussurrò. ---
L\textquotesingle anima mia magnifica il Signore e il mio spirito si è
rallegrato in Dio mio Salvatore, perché Egli ha posato lo sguardo sulla
umiltà della Sua serva\ldots{}

Voleva insegnarle quelle parole, come suo ultimo atto, perché era certo
che ella condivideva qualcosa con la Fanciulla che per prima le aveva
pronunciate.

\emph{Magnificat anima mea Dominum et exultavit spiritus meus in Deo,
	salutari meo, quia respexit hurnilitatem\ldots{}}

Rimase senza fiato prima di avere finito. Lo sguardo gli si annebbiò;
non riusciva più a distinguere la figura di lei. Ma dita fresche gli
toccarono la fronte, e la udì dire una parola: --- Vivi.

Poi lei scomparve. Poté udire la sua voce allontanarsi fra le nuove
rovine: \emph{``la-la-la, la-la-la\ldots''}.

L\textquotesingle immagine di quei freschi occhi verdi rimase con lui
quanto la vita. Non chiese perché Dio avesse scelto di far crescere una
creatura di originale innocenza dalla spalla della signora Grales, o
perché Dio le avesse dato i doni preternaturali
dell\textquotesingle Eden\ldots{} quei doni che l\textquotesingle Uomo
aveva cercato di strappare al Cielo con la forza bruta, fin da quando li
aveva perduti. Aveva veduto l\textquotesingle innocenza originale, in
quei giorni, e una promessa di resurrezione. Quell\textquotesingle unico
sguardo era stato un grande dono, e pianse di gratitudine. Poi giacque
con il viso nella terra umida e attese.

Non venne null\textquotesingle altro\ldots{} nulla che egli vedesse, o
sentisse, o udisse.
