	\chapter{\phantom{title}}

\lettrine{C}{on} grande sbalordimento di frate Francis, l\textquotesingle abate Arkos
non aveva più fatto obiezioni contro l\textquotesingle interesse del
monaco verso le reliquie. Poiché i Domenicani avevano accettato di
esaminare la faccenda, l\textquotesingle abate si era tranquillizzato; e
poiché la causa per la canonizzazione aveva fatto qualche progresso a
Nuova Roma, a volte sembrava dimenticare completamente che qualcosa di
speciale era accaduto durante la vigilia vocazionale di Francis Gerard,
AOL, già dello Utah, attualmente adibito alla copisteria.
L\textquotesingle incidente era ormai vecchio di undici anni. Gli
assurdi mormorii dei novizi sull\textquotesingle identità del pellegrino
si erano spenti da molto tempo. Il più giovane dei nuovi arrivati non
aveva mai sentito parlare della faccenda.

La faccenda era costata a frate Francis sette vigilie quaresimali nel
deserto in mezzo ai lupi, in ogni caso, e quindi non osava mai
considerare sicuro quell\textquotesingle argomento. Ogni volta che ne
parlava, la notte seguente sognava i lupi e Arkos; nel sogno, Arkos
continuava a gettare della carne ai lupi, e quella carne era Francis.

Tuttavia, il monaco aveva scoperto che poteva continuare il suo progetto
senza essere molestato, tranne che da frate Jeris, il quale continuava a
punzecchiarlo. Francis cominciò la vera e propria alluminazione della
cartapecora. Gli svolazzi intricati e la tremenda delicatezza
dell\textquotesingle inserzione delle foglie d\textquotesingle oro ne
avrebbero fatto, naturalmente, un lavoro che avrebbe richiesto molti
anni, tenendo conto della limitatezza del suo tempo libero: ma in un
buio mare di secoli, in cui nulla sembrava scorrere, una vita intera era
soltanto una breve marea, anche per l\textquotesingle uomo che la
viveva. C\textquotesingle era il tedio dei giorni e delle stagioni che
si ripeteva; poi c\textquotesingle erano sofferenze e dolori, finalmente
l\textquotesingle Estrema Unzione, e un attimo di tenebre alla
fine\ldots{} o piuttosto al principio. Perché allora la piccola anima
tremante che aveva sopportato il tedio, bene o male, si sarebbe trovata
in un luogo di luce, assorbita nel bagliore ardente di occhi
infinitamente compassionevoli, mentre si presentava davanti al Giusto. E
poi il Re avrebbe detto ``Vieni'', o il Re avrebbe detto ``Vai'' e il
tedio di anni interi era esistito solo per quei momento. Sarebbe stato
difficile credere diversamente, in un\textquotesingle epoca come quella
in cui viveva Francis.

Frate Sarl fini la quinta pagina del suo restauro matematico, crollò
sulla scrivania, e morì poche ore dopo. Non importava. Le sue note erano
intatte. Fra un secolo o due, sarebbe venuto qualcuno che le avrebbe
giudicate interessanti, e forse avrebbe completato il suo lavoro. Nel
frattempo, le preghiere salivano al cielo per l\textquotesingle anima di
Sarl.

Poi c\textquotesingle era frate Fingo e la sua scultura in legno. Un
anno o due prima era stato rimandato nella carpenteria, e aveva ottenuto
il permesso di scolpire e di levigare, di tanto in tanto, la sua
immagine semi finita del Martire. Come Francis, Fingo poteva dedicare
solo un\textquotesingle ora ogni tanto al compito che si era scelto; la
scultura progrediva con una velocità che era quasi impercettibile, a
meno che non la si guardasse soltanto a intervalli di parecchi mesi.
Francis la vedeva troppo di frequente per notare i progressi. Scoprì di
essersi lasciato affascinare dalla esuberanza di Fingo, anche quando
comprendeva che Fingo aveva adottato quelle maniere affabili per
compensare la propria bruttezza, e gli piaceva trascorrere i minuti di
ozio, quando gli capitava di averne, osservando Fingo al lavoro.

Il laboratorio di carpenteria era saturo dell\textquotesingle odore del
pino, del cedro, della segatura, del sudore umano. Non era facile
procurarsi il legno, all\textquotesingle abbazia. Salvo gli alberi di
fico e un paio di pioppi vicino al pozzo, la regione era priva
d\textquotesingle alberi. Occorreva un viaggio di tre giorni per
raggiungere la più vicina rivendita di arbusti atrofici che passavano
per legname, e spesso i monaci che andavano a procurare il legno si
assentavano dall\textquotesingle abbazia per una settimana, prima di
ritornare con qualche asinello carico di rami che servivano a fare
pioli, raggi per ruote o al massimo una gamba di sedia. Qualche volta
trascinavano dietro di sé un tronco o due per sostituire una trave
rotta. Ma con un rifornimento di legname così scarso, i carpentieri
erano necessariamente anche scultori e incisori in legno.

Qualche volta, mentre osservava Fingo al lavoro, Francis sedeva su una
panca in un angolo del laboratorio e disegnava, cercando di immaginare i
particolari della scultura che erano ancora soltanto rozzamente incisi
nel legno. C\textquotesingle erano i lineamenti del viso, vaghi e ancora
mascherati da schegge e da segni dello scalpello. Con i suoi disegni,
frate Francis cercava di anticipare quei lineamenti prima che
emergessero dalla grana del legno. Fingo lanciava occhiate ai suoi
disegni e rideva. Ma, via via che il lavoro progrediva, Francis non
poteva respingere l\textquotesingle impressione che il viso della
scultura sorridesse d\textquotesingle un sorriso familiare. Schizzò
anche quello, e l\textquotesingle impressione di familiarità aumentò.
Eppure, non riusciva a individuare quel viso, o a ricordare chi gli
avesse sorriso in quel modo ironico.

--- Non è male, in verità. Non è affatto male --- diceva Fingo dei suoi
disegni.

Il copista alzava le spalle. --- Non riesco a liberarmi
dall\textquotesingle impressione di averlo già visto.

--- Non da queste parti, fratello. Non ai miei tempi.

Francis si ammalò durante l\textquotesingle Avvento, e passarono
parecchi mesi prima che ritornasse a fare visita alla carpenteria.

--- La faccia è quasi finita, Francisco --- disse lo scultore. --- Ti
piace, adesso?

--- Ma io lo conosco! --- boccheggiò Francis, fissando gli occhi
grinzosi, gai e tristi, l\textquotesingle accenno d\textquotesingle un
sorriso ironico agli angoli della bocca\ldots{} qualcosa che era quasi
troppo familiare.

--- Davvero? E chi è, allora? --- domandò Fingo.

--- È\ldots{} bene, non sono sicuro. Mi pare di conoscerlo. Ma\ldots{}

Fingo rise. --- Stai riconoscendo i tuoi stessi disegni --- disse, come
spiegazione.

Francis non ne era altrettanto sicuro. Eppure, non riusciva a
riconoscere quel viso.

\emph{Hmmmm-hmmm!} sembrava dire quel sorriso ironico. Tuttavia,
l\textquotesingle abate giudicò irritante quel sorriso. Sebbene
permettesse che l\textquotesingle opera venisse completata, dichiarò che
non avrebbe mai consentito che la si usasse per lo scopo cui era stata
destinata in origine\ldots{} come immagine da porsi nella chiesa se la
canonizzazione del Beato fosse stata compiuta. Molti anni dopo, quando
la statua fu completata, Arkos la fece collocare nel corridoio della
foresteria, ma più tardi la trasferì nel suo studio, quando
l\textquotesingle immagine ebbe scandalizzato un visitatore proveniente
da Nuova Roma.

Lentamente, faticosamente, frate Francis stava facendo della cartapecora
un fulgore di bellezza. La voce del suo progetto si sparse oltre la
cerchia della copisteria, e spesso i monaci si raccoglievano attorno
alla sua tavola per osservare il lavoro e per mormorare la loro
ammirazione.

--- E ispirato --- sussurrava qualcuno. --- È una prova sicura. Può
darsi che abbia veramente incontrato il Beato, là fuori\ldots{}

--- Non capisco perché non dedichi il tuo tempo a qualcosa di utile ---
brontolava frate Jeris, il cui spirito sarcastico era stato esaurito da
molti anni di pazienti risposte da parte di frate Francis. Lo scettico
aveva usato il proprio tempo libero per fabbricare paralumi per le
lampade della chiesa, guadagnandosi così l\textquotesingle attenzione
dell\textquotesingle abate, che ben presto lo incaricò di occuparsi dei
perenni. E, come i libri dei conti cominciarono ben presto a
testimoniare, la promozione di frate Jeris era giustificata.

Frate Horner, il vecchio maestro amanuense, si ammalò. Dopo qualche
settimana, fu chiaro che il monaco benvoluto da tutti era sul letto di
morte. La messa funebre fu cantata nei primi tempi
dell\textquotesingle Avvento. I resti del vecchio maestro, che aveva
vissuto santamente, furono resi alla terra da cui avevano avuto origine.
Mentre la comunità esprimeva con la preghiera il suo dolore, Arkos
nominò tranquillamente frate Jeris maestro della copisteria.

Il giorno dopo essere stato insignito di quell\textquotesingle incarico,
frate Jeris informò frate Francis che considerava giusto che mettesse in
disparte i lavori da bambino e cominciasse a fare un lavoro da uomo.
Obbediente, il monaco avvolse nella pergamena il suo prezioso progetto,
lo protesse con pesanti tavole, lo ripose in uno scaffale, e cominciò a
fare paralumi, durante il suo tempo libero. Non mormorò proteste, ma si
accontentò di pensare che un giorno o l\textquotesingle altro
l\textquotesingle anima del caro fratello Jeris sarebbe partita per la
stessa strada dell\textquotesingle anima di frate Horner, per iniziare
quella vita di cui il mondo era soltanto un anticipo\ldots{}
l\textquotesingle avrebbe cominciata in età piuttosto giovanile, a
giudicare dal modo in cui si irritava e si comportava; e poi, a Dio
piacendo, Francis avrebbe potuto ottenere il permesso di completare il
suo prediletto documento.

Tuttavia la Provvidenza si incaricò di affrettare i tempi, senza
chiamare l\textquotesingle anima di frate Jeris al suo Creatore. Durante
l\textquotesingle estate che seguì la sua nomina, un protonotario
apostolico e il suo seguito di chierici vennero da Nuova Roma, con una
carovana di asini, fino all\textquotesingle abbazia. Il protonotario si
presentò come monsignor Manfredo Aguerra, postulatore per il beato
Leibowitz nella causa di canonizzazione. Con lui c\textquotesingle erano
parecchi Domenicani. Era venuto per assistere alla riapertura del
rifugio e all\textquotesingle esplorazione dell\textquotesingle Ambiente
Sigillato. Inoltre, era venuto per indagare su ogni prova che
l\textquotesingle abbazia potesse produrre e che potesse avere qualche
importanza nella causa: compresi, con grande sbigottimento
dell\textquotesingle abate, i rapporti su una presunta apparizione del
Beato che, a quanto affermavano i viaggiatori, si era presentato a un
certo Francis Gerard dello Utah, AOL.

L\textquotesingle avvocato dei Santi fu accolto con calore dai monaci,
fu ospitato nelle stanze riservate ai prelati in visita, fu
prodigalmente servito da sei giovani novizi che avevano ricevuto
l\textquotesingle ordine di obbedire a ogni suo capriccio, benché
risultasse chiaro ben presto che monsignor Aguerra era un uomo di pochi
capricci, con grande delusione dei dispensieri. Furono aperte bottiglie
dei vini migliori; Aguerra li assaggiò educatamente, ma preferì bere
latte. Il frate cacciatore procurò quaglie grassottelle e galli
selvatici per la mensa dell\textquotesingle ospite, ma dopo essersi
informato sulle abitudini alimentari dei galli selvatici (``Mangiano
grano, fratello?'' ``No, mangiano serpenti, monsignore'') monsignor
Aguerra preferì la pappa d\textquotesingle avena che mangiavano i monaci
in refettorio. Se si fosse informato circa la provenienza degli anonimi
pezzetti di carne che galleggiavano negli stufati, avrebbe preferito,
tuttavia, i galli selvatici che erano veramente succulenti. Manfredo
Aguerra insistette perché la vita nell\textquotesingle abbazia
continuasse come al solito. Nonostante questo,
l\textquotesingle avvocato veniva intrattenuto ogni sera da concerti di
violino e da una troupe di pagliacci fino a che cominciò a credere che
la solita vita nell\textquotesingle abbazia fosse straordinariamente
vivace, in confronto a quella delle altre comunità monastiche.

Il terzo giorno dopo l\textquotesingle arrivo di Aguerra,
l\textquotesingle abate mandò a chiamare frate Francis. I rapporti fra
il monaco e il suo superiore, sebbene non fossero stretti, erano
ufficialmente amichevoli dal tempo in cui l\textquotesingle abate gli
aveva permesso di professare i voti, e frate Francis non tremava neppure
mentre bussava alla porta dello studio e chiedeva: --- Mi avete mandato
a chiamare, Reverendo Padre?

--- Sì --- disse Arkos, poi chiese, tranquillamente. --- Dimmi, hai
pensato spesso alla morte?

--- Di frequente, Monsignor Abate.

--- Preghi san Giuseppe perché la tua morte non sia infelice?

--- Uhm\ldots{} spesso, Reverendo Padre.

--- Allora immagino che non ti importi se morirai
all\textquotesingle improvviso? Se qualcuno userà le tue budella per
farne corde d\textquotesingle un violino? Se verrai dato in pasto ai
porci? Se le tue ossa saranno sepolte in terra non consacrata? Eh?

--- Nnnn-no, \emph{Magister meus}.

--- Pensavo il contrario, quindi stai attento a quel che dirai a
monsignor Aguerra.

--- Io\ldots?

--- Tu. --- Arkos si soffregò il mento e sembrò perdersi in una
melanconica meditazione. ---. Lo immagino con molta chiarezza. La causa
di Leibowitz viene accantonata. Un povero fratello viene colpito da un
mattone. E giace là, implorando fra i gemiti
l\textquotesingle assoluzione. In mezzo a noi, pensa. E noi siamo lì, lo
guardiamo con molta pietà\ldots{} lo guardiamo mentre esala il suo
ultimo respiro, senza neppure un\textquotesingle ultima benedizione.
Dannato. Non benedetto. Proprio sotto il nostro naso. Che peccato, eh?

--- \emph{Monsignore?} --- squittì Francis.

--- Oh, non rimproverare me. Sarò troppo occupato a impedire ai tuoi
confratelli di sfogare l\textquotesingle impulso di finirti a calci.

--- Quando?

--- Mai, speriamo.. Perché \emph{tu sarai prudente}, non è vero\ldots{}
quando parlerai con monsignore? Altrimenti potrebbe darsi che ti
lasciassi uccidere a calci.

--- Sì, ma\ldots{}

--- Il postulatore ti vuole vedere immediatamente. Ti prego di reprimere
la tua immaginazione e di badare bene a ciò che dirai. Ti prego di non
cercare di pensare.

--- Bene, penso che ci riuscirò.

--- Fuori, figliolo, fuori.

Francis era spaventato quando bussò alla porta di Aguerra, ma comprese
subito che la sua paura era infondata. Il protonotario era un uomo
anziano, dolce e diplomatico che sembrava molto interessato alla vita
del piccolo monaco. Dopo parecchi minuti di cordiali preliminari,
abbordò l\textquotesingle argomento cruciale: --- Ora, circa il tuo
incontro con la persona che poteva essere il Beato Fondatore del\ldots{}

--- Oh, ma io non ho mai detto che fosse il nostro beato Leibo\ldots{}

--- Naturalmente non lo hai mai detto, figlia mio. Naturalmente. Ora, io
ho qui una versione dell\textquotesingle avvenimento, raccolta da fonti
non sicure, naturalmente\ldots{} e vorrei che tu la leggessi, e la
confermassi o la correggessi. --- Si interruppe per prendere dal baule
un rotolo che porse a frate Francis. --- Questa versione è basata sui
racconti dei viaggiatori --- aggiunse. --- Soltanto tu puoi descrivere
ciò che è avvenuto\ldots{} quindi io voglio che tu la controlli con
estremo scrupolo.

--- Certamente, monsignore. Ma ciò che è accaduto è veramente molto
semplice\ldots{}

--- Leggi, leggi! Poi ne parleremo, eh?

La grossezza del rotolo era sufficiente a spiegare che la versione
elaborata sulla base delle dicerie non era ``molto semplice''. Frate
Francis la lesse con crescente apprensione.
L\textquotesingle apprensione assunse presto le proporzioni
dell\textquotesingle orrore.

--- Sei pallido, figliolo --- disse il postulatore. ---
C\textquotesingle è qualcosa che ti turba?

--- Monsignore, \emph{questo\ldots{}} non è andata affatto così!

--- No? Ma, almeno indirettamente, tu devi essere stato
l\textquotesingle autore di questa versione. Come potrebbe essere
altrimenti? Non eri tu il solo testimone?

Frate Francis chiuse gli occhi e si soffregò la fronte. Aveva detto ai
suoi compagni di noviziato la semplice verità. Gli altri novizi avevano
sussurrato fra loro. Avevano raccontato la storia ai viaggiatori. I
viaggiatori l\textquotesingle avevano riferita ad altri viaggiatori.
Fino a che\ldots{} \emph{questo}! Non c\textquotesingle era di che
stupirsi se l\textquotesingle abate Arkos si era intromesso nella
discussione. Se almeno non avesse mai parlato del pellegrino!

--- Mi disse solo poche parole. Lo vidi quella volta soltanto. Mi
insegui con un bastone, mi chiese la strada per
l\textquotesingle abbazia, e fece dei segni sulla pietra, dove poi io
trovai la cripta. Poi non lo rividi mai più.

--- Niente aureola?

--- No, monsignore.

--- Niente cori angelici?

--- \emph{No!}

--- E il tappeto di rose che spuntò dove posava i piedi?

--- No, no, niente di tutto questo, monsignore! --- boccheggiò il
monaco.

--- Non scrisse il suo nome sulla pietra?

--- Così come Dio è il mio giudice, monsignore, si limitò a tracciare
quei due segni. Non sapevo che cosa significassero.

--- Ah, bene sospirò il postulatore. --- Le storie dei viaggiatori sono
sempre esagerate. Ma mi domando come sono cominciate. Adesso raccontami
cosa accadde, in realtà.

Frate Francis glielo raccontò, brevemente. Aguerra sembrò rattristato.
Dopo un silenzio meditabondo, prese il grosso rotolo, gli diede un
colpetto di commiato e lo lasciò cadere nel cesto dei rifiuti.

--- E così finisce il miracolo numero sette! --- brontolò.

Francis si affrettò a scusarsi.

L\textquotesingle avvocato l\textquotesingle interruppe con un gesto.
--- Non pensarci più. Abbiamo già prove sufficienti. Vi sono parecchie
guarigioni spontanee, parecchi casi di guarigioni istantanee da
malattie, dovute all\textquotesingle intercessione del Beato. Sono
semplici, chiare, ben documentate. Le canonizzazioni sono fondate
proprio su casi come questi. Naturalmente, non hanno la poesia di
\emph{questa} storia, ma sono quasi contento che sia infondata\ldots{}
contento per te. L\textquotesingle avvocato del diavolo ti avrebbe messo
in croce, lo sai.

--- Non ho mai detto niente di\ldots{}

--- Capisco, capisco! Tutto è cominciato a causa del rifugio.
L\textquotesingle abbiamo riaperto oggi, fra l\textquotesingle altro.

Francis si illuminò. --- Avete\ldots{} avete trovato qualche altra
reliquia di san Leibowitz?

--- \emph{Beato} Leibowitz, prego! --- corresse il monsignore.

--- No, non ancora. Abbiamo aperto la camera interna. È occorso molto
tempo per dissigillarla. A quanto pare la donna\ldots{} era una donna,
fra parentesi.,. di cui trovasti i resti fu ammessa nella stanza
esterna, ma quella interna era già piena. Probabilmente questo le
avrebbe garantito una certa protezione, se un muro non fosse crollato,
provocando una frana. Le povere anime che erano
nell\textquotesingle interno furono intrappolate dalle pietre che
bloccarono l\textquotesingle ingresso. Sa il cielo perché la porta non
fu progettata per aprirsi verso l\textquotesingle interno.

--- E la donna nell\textquotesingle anticamera era Emily Leibowitz?

Aguerra sorrise. --- Possiamo provarlo? Non lo so ancora. Io credo che
lo fosse, sì, lo credo, ma forse la mia speranza eccede la ragione.
Vedremo cosa potremo scoprire, ancora, vedremo. L\textquotesingle altra
parte ha un testimonio presente. Non posso balzare alle conclusioni.

Nonostante la sua delusione per la versione data da Francis sul suo
incontro con il pellegrino, Aguerra si mantenne amichevole. Trascorse
dieci giorni nella zona archeologica prima di ritornare a Nuova Roma e
lasciò due dei suoi assistenti perché sovraintendessero ai futuri scavi.
Il giorno della partenza, andò a visitare frate Francis nella
copisteria. Mi dicono che stavi lavorando su un documento per
commemorare le reliquie da te ritrovate --- disse il postulatore, --- A
giudicare dalle descrizioni che ne ho udito, credo che mi piacerebbe
molto vederlo.

Il monaco protestò che era in realtà una cosa da nulla, ma andò
immediatamente a prenderlo, con tanta impazienza che le mani gli
tremarono mentre svolgeva la cartapecora. Rilevò con gioia che frate
Jeris stava osservando con un cipiglio preoccupato.

Il monsignore guardò la cartapecora per molti secondi. --- Bella! ---
esplose finalmente. --- Che splendidi colori! È superba, superba.
Finiscila\ldots{} Fratello, finiscila!

Frate Francis levò lo sguardo verso frate Jeris con un sorriso
interrogativo.

Il maestro della copisteria gli voltò in fretta le spalle. La nuca gli
diventò rossa. Il giorno seguente, Francis tirò fuori i colori e le
foglie d\textquotesingle oro e riprese il suo lavoro sul diagramma
alluminato.
