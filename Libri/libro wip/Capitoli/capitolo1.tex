\mainmatter

\chapter{\phantom{text}}

"Addio piccolina, non fare i miei stessi errori ok?"

Queste furono le parole che Sofia Ferrari non ebbe mai avuto modo di ricordare, di un padre alla fine del mondo e all'inizio di un altro, e che sarebbe stato ricordato come il primo uomo ad aver riacceso il fuoco della civiltà. Un fuoco questa volta non salvifico, ma vile, creando ombre non di uomini, ma demoni.

\begin{center}
	{\Large 9:34}
\end{center}

"Argh! È già tardi per prendere l'iperscafo, non volevo usare la porta quantica ma sarà il caso" 

pensò tra sé e sé Sofia, mentre ebbe appena finito di indossare la tuta antiquanto della Linesero (Compagnia di vestiti a basso costo, anche se nell'anno 4078 nessuno notava più certe cose).

Le porte quantiche erano installate in quasi ogni camera del mondo a quel punto. Erano sottili infissi che, una volta aperti, davano spazio a un incavo nel muro fatto di metallo. Era come trovarsi in un ascensore, solo senza mai dover sentire la sensazione di discesa o salita (cosa che a quei tempi ormai nessuno quasi più conosceva, a parte i planetary Bungee Jumpers e pochi altri).

Funzionavano inserendo delle coordinate prima di entrare, e poi una volta dentro c'era una corda, collegata al funzionamento di una lampadina. Quando si spegneva la luce, e ci si ritrovava in un buio che faceva pensare se le cose intorno fossero ancora reali o meno, la si riaccendeva e si usciva nel punto specificato, da un'altra porta uguale.

Ogni tanto poteva capitare di ritrovarsi nello stesso punto di partenza, questa era una delle più rare sfortune che potesse succedere. Una persona aveva usato la porta nel nostro stesso momento e occupava il punto richiesto.

Successe 3 volte quel giorno a Sofia di aprire la porta quantica e di ritrovare il suo salotto,
