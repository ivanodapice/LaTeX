{\pagestyle{empty} % Blocca le pagine dal venire considerate nella numerazione
	
	{
		\centering
		
		~
		
		\vspace{24pt}
		{\scshape\Huge Un cantico\\per Leibowitz \par}
	}
	
	\cleardoublepage
	
	\newlength\drop
	\makeatletter
	\newcommand*\titleM{\begingroup% Misericords, T&H p 153
		\setlength\drop{0.08\textheight}
		\centering
		\vspace*{\drop}
		{\Huge\bfseries Un cantico per Leibowitz}\\[\baselineskip]
		{\scshape Walter M. Miller}\\[\baselineskip]
		\vfill
		{\large\scshape \phantom{the author}}\par
		\vfill
		{\large\scshape DiracEdizioni \img{Immagini/DiracEdizioniLogo.png}}\par
		\endgroup}
	\makeatother
	
	\titleM
	
	\clearpage
	
	\null\vfill
	
	\begin{flushleft}
		\begin{justify}
			{
				\footnotesize \textit{Un cantico per Leibowitz}
				
				\bigskip
				
				COPYRIGHT © 1959 Walter M. Miller
				
				\bigskip
				
				Tutti i diritti riservati.\\
				Nessuna parte di questa pubblicazione può essere riprodotta,\\
				memorizzata o trasmessa in qualsiasi forma o con qualsiasi mezzo,\\
				elettronico, meccanico, di fotocopiatura, registrazione, scansione o altro\\
				senza il permesso scritto dell'editore. È illegale copiare questo libro, pubblicarlo\\ 
				su un sito web o distribuirlo con qualsiasi altro mezzo senza autorizzazione.
				
				\bigskip
				
				Questo romanzo è interamente un'opera di fantasia.\\
				I nomi, i personaggi e gli episodi in esso rappresentati\\
				sono frutto dell'immaginazione dell'autore.\\
				Qualsiasi somiglianza con persone reali, vive o morte, eventi o località\\
				è del tutto casuale.
				
				\bigskip
				
				\textonesuperior Edizione, 2024
				
				\bigskip
				
				\begin{tabular}{rl}
					ISBN--10:& 0-4908-2330-0\\ 
					ISBN--13:& 978-2-0827-3855-2\\ 
				\end{tabular}	
				
				\bigskip
				
				Pubblicato da DiracEdizioni \img{Immagini/DiracEdizioniLogo.png}
			}
		\end{justify}
	\end{flushleft}
	
	\let\cleardoublepage\clearpage

}

\chapter*{\phantom{text}}
Sognato un mondo dove anni prima successe l’evento più catastrofico di sempre. Con una pallina di carta e la quantistica si è riuscito a far esplodere un intero cantiere navale, bloccando nel tempo le persone in vicinanza nei ristoranti sul mare. Chi l’ha causato non è un terrorista o una persona malvagia, ma un uomo con una bimba piccola e un’altra appena nata che ha preso troppe pillole quella sera (il grande sonno e altre) e che decise di mettersi a guidare la moto. Dopo l’accaduto nascose la bimba più piccola (che era come se) su una nave ancora intatta e successivamente venne accoltellato da persone normali per strada fino a che il corpo quasi non ebbe più forma

\frontmatter
