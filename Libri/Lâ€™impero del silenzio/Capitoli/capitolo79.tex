\backmatter

\chapter{Personaggi}

\section{CASATO MARLOWE DI DELOS}

\begin{center}
	\emph{La Spada, Nostro Oratore!}
\end{center}
\leavevmode\\
\leavevmode\\
Elevato dall'inferiore livello di patrizi verso la metà dell'VIII
millennio per opera del duca Tiberius Ormund, il Casato Marlowe ha
governato la prefettura di Meidua, su Delos, per trentuno generazioni.
In precedenza i Marlowe erano patrizi militari, ufficiali delle Legioni
orionidi con una linea di discendenza che risaliva ad Avalon e al regno
di Windsor l'Esiliato. Julian Marlowe era lo strategos della 117°
Legione di Orione e si dimostrò di fondamentale importanza nella difesa
della rivendicazione del duca Tiberius al trono di Delos nella Seconda
Guerra di Auriga. Per i suoi servizi venne elevato a Casato palatino e
gli venne concesso il dominio di Meidua, a quel tempo un villaggio di
pescatori privo di importanza su un brandello di terreno roccioso. Sotto
il regno della trisnonna di lord Hadrian, l'arcontessa lady Sabine, le
ricchezze del Casato Marlowe crebbero fino a che il suo tesoro divenne
superiore a quello del Casato Kephalos. I Marlowe sono imparentati alla
lontana con i pari dell'Impero sia per i legami matrimoniali con il
Casato Kephalos, sia -- più alla lontana -- tramite gli antichi legami
avaloniani, e sono identificabili per la carnagione alabastrina, i
capelli neri e gli occhi viola. Il loro sigillo è un diavolo rampante
carminio su campo nero.

Il nonno di lord Hadrian, l'arconte lord Timon, venne assassinato
nell'\foreignlanguage{italian}{isd} 15861 da una concubina, un'omuncola
che gli era stato donata dalla Manifatture Stellari Yuen, un fabbricante
di astronavi alleato ai Casati esuli di Delos, indignati perché lord
Timon aveva usato i suoi poteri come esecutore della viceregina per
assicurarsi un decreto di monopolio a favore del Casato Marlowe per
l'estrazione dell'uranio nel sistema di Delos, distruggendo i loro
profitti. Con la viceregina assente perché a Forum il giovane Alistair
si trovò proiettato nel ruolo di esecutore del sistema e in una guerra
contro i Casati esuli quando dichiararono una \emph{poine} contro i
Casati Marlowe e Kephalos. Gli altri Casati però sottovalutarono lord
Alistair e vennero annientati nella battaglia di Linon,
nell'\foreignlanguage{italian}{isd} 15863. Il Casato Orin, capo della
ribellione, fu completamente distrutto e l'autorità nel sistema venne
concentrata in modo più rigido nelle mani dei Casati planetari. Per
questo, e per aver difeso il suo titolo, la viceregina Kephalos diede in
sposa ad Alistair la sua figlia più giovane, Liliana. Il regno di lord
Alistair fu duro ma prospero, caratterizzato da uno Stato di diritto.

\textit{Segue una lista dei membri e servitori del Casato Marlowe menzionati nel
resoconto di lord Hadrian:}
\leavevmode\\
\leavevmode\\
\phantom{123}\textbf{Lord Alistair Diomedes Friedrich Marlowe}, arconte della
prefettura di Meidua e signore del Riposo del Diavolo, ex lord Esecutore
del Sistema di Delos, il Macellaio di Linon.
\leavevmode\\
\leavevmode\\
\phantom{123}Sua moglie, \textbf{lady Liliana Kephalos-Marlowe}, una celebrata
librettista d'opera e regista, nonché nota donnaiola, figlia più giovane
della viceregina-duchessa. Spesso lontana dal Riposo del Diavolo perché
preferisce il palazzo d'estate della sua famiglia, ad Haspida.
\leavevmode\\
\leavevmode\\
\phantom{123}Sua madre, Sua Grazia \textbf{Elmira Gwendolyn Kephalos}, viceregina
della provincia di Auriga, duchessa di Delos, arcontessa della
prefettura di Artemia.
\leavevmode\\
\leavevmode\\
\phantom{123}Le sue sorelle, \textbf{Amalia}, \textbf{Ciaran}, \textbf{Rhea},
\textbf{Alienor}, \textbf{Elena} e \textbf{Talia}.
\leavevmode\\
\leavevmode\\
\phantom{123}Il suo servitore, \textbf{Mikal}, un plebeo appartenente al personale
domestico di Haspida.
\leavevmode\\
\leavevmode\\
\textit{I loro figli:}
\leavevmode\\
\leavevmode\\
\phantom{123}\textbf{Hadrian Anaxander Marlowe}, il Semimortale, Divoratore di Soli,
Distruttore di Stelle, Uccisore di Pallidi, Immortale. Responsabile
della morte dell'intera specie dei Cielcin.
\leavevmode\\
\leavevmode\\
\phantom{123}\textbf{Crispin Marlowe}, erede presunto del Riposo del Diavolo, un
giovane insensibile e violento
\leavevmode\\
\leavevmode\\
\textit{I suoi genitori: }\{\textbf{lord Timon Marlowe}\} e \{\textbf{lady
	Fuchsia Bellgrove-Marlowe}\}, entrambi deceduti.
\leavevmode\\
\leavevmode\\
\phantom{123}Suo fratello, \{\textbf{Lucian Marlowe}\}, morto nello schianto di una
navetta.
\leavevmode\\
\leavevmode\\
\phantom{123}Il suo antenato \{\textbf{lord Julian Marlowe}\}, ricompensato con il
dominio di Meidua per i servizi resi al Casato Ormund ora estinto. Ha
cominciato la costruzione del Riposo del Diavolo.
\leavevmode\\
\leavevmode\\
\textit{I suoi dipendenti:}
\leavevmode\\
\leavevmode\\
\phantom{123}\textbf{Sir Felix Martyn}, castellano e comandante delle guardie del
Casato. Incaricato anche dell'istruzione nell'uso delle armi degli eredi
Marlowe.
\leavevmode\\
\leavevmode\\
\textit{I suoi cavalieri:}
\leavevmode\\
\leavevmode\\
\phantom{123}\textbf{Sir Roban Milosh}, littore incaricato della protezione personale
della famiglia Marlowe.
\leavevmode\\
\leavevmode\\
\phantom{123}\textbf{Dama Uma Sylvia}, littore incaricata della protezione personale
della famiglia Marlowe
\leavevmode\\
\leavevmode\\
\phantom{123}\textbf{Sir Ardian Tullo}, capitano del corpo dei piloti dei Marlowe.
\leavevmode\\
\leavevmode\\
\phantom{123}\textbf{Kyra}, pilota di navetta.
\leavevmode\\
\leavevmode\\
\phantom{123}\textbf{Tor Alcuin}, primo consigliere scoliasta dell'arconte.
\leavevmode\\
\leavevmode\\
\phantom{123}\textbf{Tor Gibson}, tutore dei giovani Marlowe e funzionario minore nel
consiglio dell'arconte. Diplomato a Nov Acor, su Syracuse.
\leavevmode\\
\leavevmode\\
\phantom{123}\textbf{Tor Alma}, medico personale del Casato Marlowe. Autorizzata
dall'Alto Collegio a lavorare con le vasche di riproduzione per dare
alla luce i bambini palatini.
\leavevmode\\
\leavevmode\\
\phantom{123}\textbf{Eusebia}, priora della Sacra Cappellania della prefettura di
Meidua e principale consigliera dell'arconte.
\leavevmode\\
\leavevmode\\
\textit{I suoi accoliti:}
\leavevmode\\
\leavevmode\\
\phantom{123}\textbf{Severn}, un giovane cantore che funge da segretario della
priora.
\leavevmode\\
\leavevmode\\
\phantom{123}\textbf{Abiatha}, vecchio cantore che in genere gestisce i servizi della
Cappellania al Riposo del Diavolo.
\leavevmode\\
\leavevmode\\
\phantom{123}\textbf{Helene}, ciambellano del Riposo del Diavolo e capo dei domestici
e dei servitori, inclusi personale di servizio, cuochi, camerieri e
stallieri.
\leavevmode\\
\leavevmode\\
\leavevmode\\
\section{SUL PIANETA DELOS}
\leavevmode\\
\leavevmode\\
Delos è uno dei più antichi esempi di terraformazione coronata dal
successo nella storia dell'Impero. Colonizzato formalmente nel IV
millennio \foreignlanguage{italian}{isd}, il progetto ha richiesto
parecchi secoli per essere completato. In origine non c'erano biosfera,
ciclo dell'ossigeno o oceani, ma la forza di gravità e le magnetosfere
erano perfette. Son state portate delle comete per introdurre l'acqua, e
molti degli animali utilizzati sono stati di origine terrestre, con il
risultato di ottenere un mondo il più simile possibile alla Vecchia
Terra. A causa delle condizioni ambientali clementi e della sua
posizione strategica sulle rotte commerciali imperiali con le provincie
del {Sagittario}, questo mondo è sempre stato prospero. È fiorito prima
sotto il Casato Ormund, poi di nuovo sotto il Casato Kephalos, quando i
duchi Ormund hanno perso il favore imperiale durante la ribellione dei
Chablon nel XIII millennio e sono stati privati del potere. Tale
prosperità è stata aumentata dalla scoperta degli incredibili depositi
di uranio del sistema, che sono da tempo oggetto di lotte interne fra i
Casati palatini del sistema.

Centro nevralgico culturale e commerciale, Delos ha attratto il
commercio interstellare delle corporazioni nipponesi e mandari, oltre a
quelle eudoriane, durantine e ai liberi mercanti.

\textit{Segue un elenco dei personaggi che appaiono nell'episodio di Delos del
resoconto di lord Hadrian:}
\leavevmode\\
\leavevmode\\
\phantom{123}\textbf{Adaeze Feng}, direttrice del Consorzio Wong-Hopper per la
provincia di Auriga.
\leavevmode\\
\leavevmode\\
\textit{I suoi dipendenti:}
\leavevmode\\
\leavevmode\\
\phantom{123}\textbf{Xun Gong Sun}, ministro di rango minore che si occupa del
commercio di materie prime.
\leavevmode\\
\leavevmode\\
\phantom{123}\textbf{Tor Terence}, scoliasta e consigliere presso l'ufficio personale
della direttrice.
\leavevmode\\
\leavevmode\\
\phantom{123}\textbf{Lena Balem}, capofazione della Gilda Mineraria di Delos,
capitolo di Meidua; incaricata di sovrintendere a tutta l'estrazione
dell'uranio nel sistema.
\leavevmode\\
\leavevmode\\
\phantom{123}\textbf{Jem} e \textbf{Zeb}, due criminali.
\leavevmode\\
\leavevmode\\
\phantom{123}\textbf{Demetri Arello}, un libero mercante jaddiano, capitano
dell'astronave \emph{Eurynasir}.
\leavevmode\\
\leavevmode\\
\phantom{123}Sua moglie, \textbf{Juno Arello}, comproprietaria della nave.
\leavevmode\\
\leavevmode\\
\textit{Il loro equipaggio:}
\leavevmode\\
\leavevmode\\
\phantom{123}\textbf{Bassem}, jaddiano e timoniere della nave.
\leavevmode\\
\leavevmode\\
\phantom{123}\textbf{Sarric Jugo}, espatriato tavrosiano e medico di bordo.
\leavevmode\\
\leavevmode\\
\phantom{123}\textbf{Saltus}, un omuncolo dall'umorismo poco appropriato.
\leavevmode\\
\leavevmode\\
\phantom{123}\textbf{Emar} e \textbf{Imani}, gemelle.
\leavevmode\\
\leavevmode\\
\leavevmode\\
\section{CASATO MATARO DI EMESH}

\begin{center}
	\emph{Nato dalla Conquista}
\end{center}
\leavevmode\\
\leavevmode\\
Un Casato relativamente giovane, i Mataro di Emesh sono una delle
numerose famiglie di nuovi ricchi che hanno acquistato preminenza negli
anni della conquista imperiale in Norma e in particolare nel Velo di
Marinus\textbf{.} Solo plutocrati fino a qualche generazione fa, i
Mataro hanno accumulato soldi comprando e vendendo schiavi, in
particolare ordini all'ingrosso di omuncoli da usare come manovalanza
agricola e industriale nelle nuove colonie. Estremamente ricco, Armand
Mataro -- allora solo un patrizio -- ha equipaggiato tre intere Legioni
e le ha usate per conquistare Emesh senza un mandato imperiale. Simili
guerre private sono state tipiche delle conquiste normanne, e quanti
hanno avuto successo hanno ottenuto la concessione dello stato di
palatino. I Mataro non sono ancora allineati con nessuna delle grandi e
antiche costellazioni dei Casati palatini perché il loro sangue e il
loro status sono di livello relativamente basso.

Emesh è un dominio privo di importanza al limite estremo dello spazio
umano civilizzato, costantemente nell'ombra della minaccia costituita
dai Cielcin. Relativamente poveri, i residenti hanno cercato di
sfruttare al meglio i coloni umandh del pianeta, anche se quei nativi
sono biologicamente tanto alieni che la loro utilità come schiavi è
quantomeno discutibile. Il pianeta comunque sopravvive grazie
all'esportazione di pesce e del petrolio estratto dal fondo marino. Il
sigillo del Casato Mataro è una sfinge d'oro dormiente su campo verde, a
volte bordato di bianco.

\textit{Segue una lista dei membri e servitori del Casato Mataro menzionati nel
resoconto di lord Hadrian:}
\leavevmode\\
\leavevmode\\
\phantom{123}\textbf{Lord Balian Mataro}, terzo conte di Emesh, arconte della
prefettura di Borosevo e signore del castello di Borosevo.
\leavevmode\\
\leavevmode\\
\phantom{123}Suo marito, \textbf{lord Luthor Astin-Shin-Mataro}, un Mandari, ministro
delle Finanze della contea ed ex funzionario del Consorzio Wong-Hopper
per Marinus.
\leavevmode\\
\leavevmode\\
\textit{I loro figli:}
\leavevmode\\
\leavevmode\\
\phantom{123}\textbf{Dorian Mataro}, erede presunto della Contea di Emesh, un giovane
nell'anno del suo Efebeia.
\leavevmode\\
\leavevmode\\
\phantom{123}La sua amica \textbf{Melandra}, una patrizia di una fedele famiglia di
Borosevo.
\leavevmode\\
\leavevmode\\
\phantom{123}\textbf{Anaïs Mataro}, giovane e astuta esponente dell'alta società.
\leavevmode\\
\leavevmode\\
\phantom{123}Il suo antenato \{\textbf{lord Armand Mataro}\}, primo conte di Emesh,
conquistatore della Compagnia Normanna Unita. Ha sovrinteso alla
costruzione della base militare di Borosevo e agli inizi della città
stessa.
\leavevmode\\
\leavevmode\\
\textit{I suoi dipendenti:}
\leavevmode\\
\leavevmode\\
\phantom{123}\textbf{Liada Ogir}, Alta Cancelliera, capo dei logoteti di palazzo e
del servizio civile.
\leavevmode\\
\leavevmode\\
\phantom{123}\textbf{Ligeia Vas}, grande priora della Sacra Cappellania di Emesh.
\leavevmode\\
\leavevmode\\
\phantom{123}Il suo accolito, \textbf{Gilliam Vas}, cantore e figlio bastardo della
grande priora, un intus mutato geneticamente.
\leavevmode\\
\leavevmode\\
\phantom{123}\textbf{K.F. Agari}, capo inquisitrice della Sacra Cappellania di
Borosevo.
\leavevmode\\
\leavevmode\\
\phantom{123}Fratelli \textbf{Rhom} e \textbf{Udan}, cathar dell'Inquisizione di
Borosevo.
\leavevmode\\
\leavevmode\\
\phantom{123}\textbf{Sir Preston Rau}, anziano istruttore di scherma incaricato di
addestrare i figli del conte.
\leavevmode\\
\leavevmode\\
\phantom{123}\textbf{Dama Camilla}, littore incaricata della protezione del conte e
della sua famiglia.
\leavevmode\\
\leavevmode\\
\phantom{123}\textbf{Tor Vladimir}, capo scoliasta e consigliere del conte.
\leavevmode\\
\leavevmode\\
\phantom{123}\textbf{Malo}, un concubino di palazzo al servizio del conte.
\leavevmode\\
\leavevmode\\
\textit{I suoi vassalli:}
\leavevmode\\
\leavevmode\\
\phantom{123}\textbf{Lord Perun Veisi}, arconte della prefettura di Tolbaran e
signore del castello di Fonteprofonda.
\leavevmode\\
\leavevmode\\
\phantom{123}Sua moglie, \textbf{lady Lidia Redgrave-Veisi}, una palatina di un altro
pianeta.
\leavevmode\\
\leavevmode\\
\phantom{123}Suo zio, \textbf{sir Elomas Redgrave}, famoso viaggiatore, avventuriero
e duellante. Si è ritirato a fare vita di campagna ed è uno xenologo
dilettante.
\leavevmode\\
\leavevmode\\
\phantom{123}Una sua collaboratrice, \textbf{tor Ada}, scoliasta e direttrice degli
scavi archeologici di Calagah.
\leavevmode\\
\leavevmode\\
\phantom{123}I suoi assistenti, \textbf{Maros} e \textbf{Bel}, archeologi laici
dell'Università di Tolbaran.
\leavevmode\\
\leavevmode\\
\phantom{123}I suoi servitori \textbf{Orso} e \textbf{Damara}, che ha portato con sé
dai suoi viaggi.
\leavevmode\\
\leavevmode\\
\textit{I loro figli:}
\leavevmode\\
\leavevmode\\
\phantom{123}\textbf{Alexander Veisi}, erede presunto della prefettura.
\leavevmode\\
\leavevmode\\
\phantom{123}\textbf{Karthik Veisi}, un giovane, scudiero del prozio {Elomas}.
\leavevmode\\
\leavevmode\\
\textit{Sua madre:} \textbf{lady Kamala Veisi}.
\leavevmode\\
\leavevmode\\
\phantom{123}\textbf{Etan Vriell}, un centurione al suo servizio.
\leavevmode\\
\leavevmode\\
\phantom{123}\textbf{Lord Tivan Melluan}, arconte ed esule della prefettura di Binah.
\leavevmode\\
\leavevmode\\
\phantom{123}\textbf{Lord Ivanis Kvar}, arconte ed esule della prefettura di Armand.
\leavevmode\\
\leavevmode\\
\leavevmode\\
\section{SUL PIANETA EMESH}
\leavevmode\\
\leavevmode\\
Emesh è stato colonizzato dalla Compagnia Normanna Unita, una piccola
democrazia diretta, all'inizio del XVI millennio. Il pianeta è
prevalentemente oceanico, con diverse acquacolture native, e in quanto
tale molti dei suoi insediamenti sono stati costruiti su atolli o isole
che sono quel che rimane del passato vulcanico da tempo morto del
pianeta. Ha un solo continente, Anshar, una piccola sporgenza vulcanica
che ha un diametro di appena millecinquecento miglia, sul quale la
\foreignlanguage{italian}{cnu} aveva costruito la sua capitale,
Tolbaran. Quando ha invaso il pianeta, il Casato Mataro ha costruito una
base militare avanzata sull'atollo di Borosevo, che nei secoli è
cresciuta fino a trasformarsi in un vasto labirinto di canali e di
piattaforme. Adesso Tolbaran è stata relegata a un ruolo secondario
nella vita civica e gran parte del territorio di Anshar è dedicato
all'agricoltura in quanto il terreno vulcanico si è dimostrato
favorevole a raccolti agricoli e alla viticoltura.

Il pianeta ha due lune, Armand e Binah. Quella più grande, Binah, è
stata aperta di recente alla terraformazione e le colture di alghe hanno
attecchito in modo estremamente aggressivo. Armand è troppo piccola per
trattenere un'atmosfera ma ospita energiche attività minerarie,
soprattutto nel campo dei metalli pesanti.

\textit{Segue un elenco dei personaggi che appaiono nell'episodio di Emesh del
resoconto di lord Hadrian:}
\leavevmode\\
\leavevmode\\
\phantom{123}\textbf{Valka Onderra Vhad Edda}, una xenologa e, di fatto, dignitaria
in visita da Edda, nella Demarchia di Tavros. Ricercatrice che studia
gli Umandh e le rovine sul continente meridionale. Fa parte della
spedizione amatoriale di sir Elomas Redgrave.
\leavevmode\\
\leavevmode\\
\phantom{123}\textbf{Cat}, una ragazza povera che si guadagna da vivere per strada.
\leavevmode\\
\leavevmode\\
\phantom{123}\textbf{Sepha}, donna anziana che gestisce una clinica per i poveri e i
senzatetto della città, spesso gratuitamente.
\leavevmode\\
\leavevmode\\
\phantom{123}La sua assistente, \textbf{Maris}, una giovane donna.
\leavevmode\\
\leavevmode\\
\phantom{123}\textbf{Gila}, capo di un cantiere di recupero di navi mercantili
abbandonate.
\leavevmode\\
\leavevmode\\
\phantom{123}I suoi dipendenti \textbf{Skag} e \textbf{Bor}, meccanici di astronave.
\leavevmode\\
\leavevmode\\
\phantom{123}\textbf{Rells}, capo di una banda di strada che minaccia la città e
terrorizza gli altri senzatetto e ribelli.
\leavevmode\\
\leavevmode\\
\phantom{123}I suoi scagnozzi \textbf{Joi}, \textbf{Kaller} e \textbf{Tur}, comuni
criminali.
\leavevmode\\
\leavevmode\\
\phantom{123}\textbf{Gin}, prefetto-ispettore della divisione di Risposta al Crimine,
principale nemico delle bande di strada.
\leavevmode\\
\leavevmode\\
\phantom{123}\textbf{Ko}, \textbf{Ren}, \textbf{Yoh}, tutti prefetti urbani.
\leavevmode\\
\leavevmode\\
\phantom{123}\textbf{Niles Engin}, vilicus della riserva per alieni degli Umandh a
Ulakiel, membro di alto livello della Gilda dei Pescatori.
\leavevmode\\
\leavevmode\\
\phantom{123}\textbf{Quintus}, un douleter alle dipendenze del rifugio di Ulakiel e
della Gilda dei Pescatori.
\leavevmode\\
\leavevmode\\
\phantom{123}\textbf{Il Corvo}, un viaggiatore proveniente da una terra antica.
\leavevmode\\
\leavevmode\\
\textit{Nel colosseo:}
\leavevmode\\
\leavevmode\\
\phantom{123}\textbf{William di Danu}, detto \textbf{Switch}, un combattente
inesperto. Precedentemente un catamita sotto contratto e mozzo a bordo
di una nave mandari.
\leavevmode\\
\leavevmode\\
\phantom{123}Il suo precedente padrone, \textbf{Set}, proprietario di nave, pederasta
e mercante.
\leavevmode\\
\leavevmode\\
\phantom{123}\textbf{Pallino}, veterano guercio di quarantacinque anni che ha
prestato servizio attivo nelle Legioni imperiali. Uno dei capitani di
fatto della squadra mirmidone.
\leavevmode\\
\leavevmode\\
\phantom{123}La sua amante \textbf{Elara}, una donna locale portata al sarcasmo.
\leavevmode\\
\leavevmode\\
\phantom{123}\textbf{Ghen} e \textbf{Siran}, prigionieri costretti a combattere.
\leavevmode\\
\leavevmode\\
\phantom{123}\textbf{Banks}, un brizzolato veterano. Uno dei capitani di fatto dei
mirmidoni.
\leavevmode\\
\leavevmode\\
\phantom{123}\textbf{Kiri}, una donna di mezza età che combatte volontariamente nella
speranza di guadagnare abbastanza da mandare suo figlio a scuola.
\leavevmode\\
\leavevmode\\
\phantom{123}Suo figlio \textbf{Dar}, un giovane prossimo a essere abbastanza grande
da sostenere gli esami per il servizio civile.
\leavevmode\\
\leavevmode\\
\phantom{123}Fra gli altri, \textbf{Keddwen}, \textbf{Erdro}, \textbf{Alis} e
\textbf{Light}.
\leavevmode\\
\leavevmode\\
\phantom{123}\textbf{Kogan}, ex mercenario della Compagnia del Cavallo Bianco,
veterano della battaglia di Wodan. Di recente si è unito ai mirmidoni
del Colosso.
\leavevmode\\
\leavevmode\\
\phantom{123}\textbf{Jaffa} e \textbf{Amarei}, gladiatori professionisti.
\leavevmode\\
\leavevmode\\
\phantom{123}\textbf{Chand}, una schiava e medico dei mirmidoni. Ex patriota othriana
e ausiliario imperiale.
\leavevmode\\
\leavevmode\\
\phantom{123}\textbf{Stromos}, capo delle carceri dell'ipogeo del colosseo. Si
considera un artista troppo grande per il posto che occupa.
\leavevmode\\
\leavevmode\\
\phantom{123}\textbf{Lento} e \textbf{Più Lento}, guardie di prigione incaricate
della sorveglianza di prigionieri speciali.
\leavevmode\\
\leavevmode\\
\phantom{123}Il loro prigioniero, \textbf{Makisomn}, un Cielcin regalato al conte
dalla Compagnia del Cavallo Bianco e destinato a una pubblica
esecuzione.
\leavevmode\\
\leavevmode\\
\leavevmode\\
\section{IL RESTO DEL MONDO}

\textit{Segue una lista di personaggi del resoconto di lord Hadrian che non sono
collegati né a Delos né a Emesh:}
\leavevmode\\
\leavevmode\\
Sua radiosità imperiale l'\textbf{imperatore William XXIII del Casato
	Avent}, figlio primogenito della Terra. Guardiano del Sistema Solare, re
di Avalon, signore sovrano del regno di Windsor l'Esule, principe
imperatore delle Braccia di Orione, del Sagittario, di Perseo e di
Centaurus, primarca di Orione, conquistatore di Norma, grande strategos
delle Legioni del Sole, supremo signore delle città di Forum, Stella del
Nord delle costellazioni del Sangue Palatino, difensore dei figli degli
uomini e servo dei servi della Terra.
\leavevmode\\
\leavevmode\\
\phantom{123}Il suo strategos, \textbf{sir Titus Hauptmann}, duca di Andernach e
primo strategos delle Legioni di Centaurus.
\leavevmode\\
\leavevmode\\
\phantom{123}La sua tribuno, \textbf{dama Raine Smythe}, una tribuno della 437°
Legione di Centaurus, capitano della \foreignlanguage{italian}{isv}
\emph{Incrollabile}.
\leavevmode\\
\leavevmode\\
\phantom{123}Il suo primo ufficiale, \textbf{sir William Crossflane}, comandante,
primo grado, anziano palatino.
\leavevmode\\
\leavevmode\\
\phantom{123}Il suo tenente, \textbf{Bassander Lin}, un promettente giovane ufficiale
a bordo dell'\emph{Incrollabile}. Funge di frequente da littore per la
tribuno nelle questioni politiche.
\leavevmode\\
\leavevmode\\
\phantom{123}\textbf{Sir Alexei Karelin}, lord presidente e capitano della Compagnia
del Cavallo Bianco, un gruppo di foederati composto principalmente da ex
legionari imperiali e annesso alla 437° Legione.
\leavevmode\\
\leavevmode\\
\phantom{123}\textbf{Varric Cousland}, lord presidente, fondatore e capitano dei
Draghi di Cousland, una compagnia di foederati che opera dalle Proprietà
Normanne di Monmara. Annesso alla 437° Legione.
\leavevmode\\
\leavevmode\\
\phantom{123}\textbf{Edouard Albe}, un agente operativo dell'intelligence della
Legione.
\leavevmode\\
\leavevmode\\
\phantom{123}Sua altezza reale \textbf{Aldia Ahmad Rodrigo-Phillipe di Otranto}, Alto
Principe di Jadd. principe di Laran, primo-fra-i-pari dei principi dei
principati dei popoli jaddiani. Signore delle Lune Circostanti, custode
del Pianeta di Fuoco, capo del \emph{Dham-Eali.}
\leavevmode\\
\leavevmode\\
\phantom{123}\textbf{Lady Kalima Aliarada Udiri di Sayyiph}, satrapo di Ubar, fedele
al principe di Thessaloniki. Inviata ufficiale sul fronte di guerra per
conto dell'Alto Principe.
\leavevmode\\
\leavevmode\\
\phantom{123}Il suo littore, \textbf{sir Olorin Milta}, un maeskolos maestro di spada
della Scuola di Fuoco.
\leavevmode\\
\leavevmode\\
\phantom{123}La sua tenente, \textbf{Jinan Azhar}, promettente giovane ufficiale.
\leavevmode\\
\leavevmode\\
\phantom{123}\textbf{Utsebimn Aranata Otiolo}, \emph{aeta} principe-condottiero e
capitano del Suo \emph{scianda}, maestro-custode del Suo popolo,
servitore dei suoi schiavi.
\leavevmode\\
\leavevmode\\
\textit{Le sue proprietà:}
\leavevmode\\
\leavevmode\\
\phantom{123}\textbf{Casantora Tanaran Iakato}, \emph{baetan,} prete-storico \emph{}
della \emph{scianda}.
\leavevmode\\
\leavevmode\\
\phantom{123}\textbf{Itana Uvanari Ayatomn}, \emph{ichakta}, capitano dell'astronave 
\phantom{123}\emph{Yad Ga Higatte} e orgoglioso ufficiale militare.
\leavevmode\\
\leavevmode\\
\phantom{123}Sua proprietà e soldato-schiavo, \textbf{Svatarom}.

\newpage\blankpage