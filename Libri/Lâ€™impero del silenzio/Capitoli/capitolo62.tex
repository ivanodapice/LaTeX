\chapter{La gabbia dorata}

La bottiglia di vino vuota rimbalzò contro le piastrelle del pavimento e
rotolò sotto il tavolo. La lasciai andare. Avevo una mezza idea di
chiamare Switch sul mio terminale perché non avrei dovuto stare da solo
in un momento del genere, e tuttavia sapevo che non avrei tollerato la
compagnia di nessuno. Fuori non era neppure ancora buio: quella giornata
-- quel giorno interminabile -- rifiutava di finire. La luce arancione
del sole cadeva cupa attraverso le strette finestre, mettendo in risalto
le mie stanze spartane. Esaurite le mie scorte di vino, recuperai il mio
diario dal tavolino adiacente il mio corto divano e attivai un comando
del tablet per polarizzare la luce che entrava dalle finestre,
offuscando il mondo.

Procedetti ad appuntire la matita, spezzando due volte la punta fra le
dita malferme mentre il volto di Gilliam continuava ad affiorare. `Non
vi fidate' aveva detto. `Non vi fidate...' Pensai che forse gli dèi
della Cappellania erano reali e mi odiavano, che forse dovevo espiare.
\emph{Lord consorte di Emesh...} digrignai i denti. \emph{All'uccello
	non importa che la sua gabbia sia dorata.} Era un grande onore, anche se
era una sorta di prigione. \emph{Una sorta di veleno.} Sarei stato un
lord nel titolo oltre che per sangue, e uno più grande di mio padre.

Recuperai il bicchiere vuoto e lo sollevai in un brindisi drammatico,
poi dovetti reprimere l'impulso di fracassarlo contro la parete e scelsi
invece di sbatterlo di nuovo sul tavolo, accigliandomi. Accasciato
contro il bracciolo della poltrona sollevai una mano senza rivolgermi a
nessuno in particolare -- forse alle videocamere nel mio appartamento --
e feci un gesto volgare. Nella mia ubriachezza, dimenticai che non erano
le videocamere del Riposo del Diavolo.

La porta si aprì. L'avevo bloccata perché volevo essere solo. Con un
inizio di panico, aspettandomi di vedere un cathar o un sicario della
Cappellania, sollevai il bisturi che usavo per affilare le matite e lo
puntai come un bruciatore al plasma. Nella fretta sbattei con il
ginocchio contro il bracciolo del divano e barcollai, ricadendo a sedere
e gettando a terra il bicchiere. «Dannazione...» Fui scosso da un
singhiozzo, per fortuna silenzioso. Incapace di guardare Valka abbassai
invece lo sguardo sul bicchiere rotto e sul vino che formava una piccola
pozza sul legno.

«Se stai cercando di annegarti, su Emesh ci sono specchi d'acqua più
grandi» disse, con finta freddezza.

La fissai con rabbia dal mio posto sul pavimento, poi guardai il bisturi
che avevo in mano e lo gettai da un lato con disprezzo, guardandolo
cadere rumorosamente sotto il piccolo tavolo vicino al cucinotto. «Non
era acqua.»

«Credo che sia questo il problema.» Riempì d'acqua un bicchiere al
lavandino e me lo porse, poi mi aiutò a rimettermi a sedere sul divano.
«Bevi.» La sua mano toccò di sfuggita la mia. Anche attraverso i fumi
del bere lo percepii, e ne ricordo il calore. C'era interessamento in
quel gesto, e una tenerezza che quel giorno, fra tutti i giorni, non
meritavo.

Bevvi, poi appoggiai la testa contro lo schienale del divano mentre lei
andava a recuperare il bisturi. Si girò, mi colse a fissarla e inarcò un
sopracciglio. Nei vecchi archivi avevo visto statue di Venere meno
belle, anche se qui c'era senza dubbio Pallade. Quel pensiero mi strappò
una risata nasale che collassò nel silenzio, e dopo che fu passato quasi
un minuto, durante il quale la xenologa si sedette su una stretta
poltrona vicina alla finestra, riuscii a borbottare: «Mi dispiace.» Poi
lo ripetei con maggior vigore. «Mi dispiace.» Serrai gli occhi e mi
strinsi l'arco del naso. «C'è altro vino... da qualche parte. Sir Elomas
ne ha mandato una cassetta. Vuoi ringraziarlo da parte mia?»

«Mi sembra che tu ne abbia bevuto abbastanza.» Mi guardò in un modo che
mi fece sentire sminuito, parte del divano su cui ero semidisteso.
Quegli occhi dorati notarono la bottiglia che era rotolata sul
pavimento, le scaglie di legno della matita, i miei vestiti spiegazzati
e il groviglio di coperte che giaceva sparso per terra fra il divano e
la porta aperta della mia camera da letto. Incrociò le braccia. «Volevo
inculcarti un po' di buonsenso, ma credo che il danno sia fatto.»
Continuò a osservarmi.

«Scusami...» Avevo la lingua inspessita, lenta a seguire la poca presa
che il cervello aveva su di essa. «Valka, volevo limitarmi a ferirlo, ma
lui mi ha colpito.» Sollevai il braccio bendato perché lo esaminasse.
«Quel bastardo mi ha colpito, è stato più rapido di quanto mi
aspettassi.» Rabbrividii e chiusi gli occhi. Non potevo guardarla, non
nel mio stato attuale, e forse se avessi tenuto gli occhi chiusi se ne
sarebbe andata. Non avrebbe dovuto neppure entrare. «Sono stato
stupido... io...»

«\emph{Sei} un idiota» dichiarò, ma con un sorriso appena accennato.
«Però non sei un bugiardo.»

«Cosa?» Sollevai lo sguardo appannato dal diario, che si era aperto su
un'immagine del Riposo del Diavolo, con il castello in nero, privo di
dettagli come se fosse stato visto dalle strade di Meidua... da quella
stessa strada in cui per poco non ero morto, una vita prima. Erano
davvero passati solo tre anni? O trentacinque?

Valka socchiuse gli occhi fino a ridurli a due fessure che scintillavano
nella fioca luce color terra d'ombra del pomeriggio, ma il suo sorriso
non si incrinò. «Non avevi mai ucciso nessuno, prima d'ora.»

«Non sono... io... no.» Avrei voluto piangere, ma più di ogni altra cosa
volevo stare bene, o almeno dare l'impressione di stare bene davanti a
quella strana e squisita straniera. Lei era un colosso, alta e calma e
remota come le stelle.

«Non è facile, vero?» commentò, dopo un pregnante momento.

Stava agendo in modo sottile, ma al momento io ero troppo ubriaco per le
sottigliezze. Era già una lotta mantenere le mie parole coerenti,
impedire che scorressero come fondotinta sotto la pioggia. «Cosa?»

«Uccidere.»

Inclinai la testa verso di lei con le dita intorpidite che faticavano a
chiudere il diario. «Già.» La sua voce non era stata quella di
un'accademica ma era stata intrisa di dolorosa esperienza. Non le feci
pressioni mentre si tormentava l'interno di una guancia con l'attenzione
fissa sul mio volto, persa nei suoi pensieri. Con la mente appannata,
raccolsi quel poco cervello che mi rimaneva e domandai: «Cosa c'è?»

Lei scosse il capo e si spinse una ciocca di capelli dietro l'orecchio
con un gesto deciso. «Adesso cosa farai?»

Scrollai le spalle e mi mossi come per afferrare una bottiglia dal
tavolino, solo per ricordare troppo tardi che era vuota, e sul
pavimento. Borbottai qualcosa riguardo al matrimonio, poi le raccontai
ogni cosa. Di Gilliam, di Anaïs, di mio padre e della Cappellania. «Il
conte mi vuole mandare su Binah per tenermi lontano da Vas.» Feci una
pausa e mi schiarii la gola. «Io però gli ho chiesto di permettermi di
venire con te, lontano da Vas.» Feci una pausa e mi schiarii la gola.

«Cosa?» Valka sollevò la testa di scatto. «Perché?»

«Non voglio... non intendo autoinvitarmi, è solo che... credo che tu
abbia ragione. Non penso che l'Impero vada bene per me.» Soffocai un
altro singhiozzo e sbattei la testa all'indietro contro il divano. Una,
due volte.

Se mai c'è stato un momento, un punto in cui Valka si è fatta cordiale
nei miei confronti, è stato allora. Potevo quasi avvertirlo, come si può
avvertire il rumore del ghiaccio che si spezza quando si versa acqua in
un bicchiere. Il suo sorriso freddo e fragile si ammorbidì, e invece di
rispondere lei si alzò per sfilarmi di mano la coppa vuota e tornare a
riempirla. Venni lasciato in silenzio per quasi un minuto, intento a
guardare il dondolarsi e il saettare di minuscole navi sul distante mare
di ceramica. Nella luce serale le acque verdi apparivano del colore del
fango e tutto il mondo era dozzinale come un brutto dipinto. La
bruttura, pensai. Era questo che intendeva Gibson riguardo alla bruttura
del mondo. O forse il mondo era perfetto, ed ero io a essere orribile.

Valka tornò indietro e questa volta sedette sul basso tavolino davanti
al divano, mettendomi in mano l'acqua. «Perché Calagah?»

«Per gli alieni, la storia» mentii. \emph{Per te. Per fuggire.} «Se sei
disposta ad accettarmi. Non voglio essere un... un fardello.»

«Il conte lo permetterà?» Valka appariva nervosa, e quell'espressione
non le si addiceva.

«Non posso rimanere qui, non dopo...» Agitai una mano in un gesto
incoerente. «Dopo oggi.» Annuendo, mi accasciai di lato sul bracciolo
del divano. «Mi dispiace.»

«Per cosa?»

«Eri venuta per urlarmi contro, la cosa giusta da fare» borbottai. «E
adesso ti stai mostrando gentile.»

Arricciò le labbra e il suo sguardo si spostò rapido fuori dalla
finestra, in direzione della città dipinta e putrida. «Ti urlerò contro
più tardi» ribatté. Poi: «Tuo padre ti ha diseredato?»

Scossi il capo e mi pentii di quel gesto quando la stanza cominciò a
ruotare, spingendomi a chiudere gli occhi. «Mi ha disconosciuto. E lo ha
fatto anche mia nonna.» Mi sfuggì una risata, folle, sottile e
incrinata. «In realtà non sono più Hadrian \emph{Marlowe}, sono Hadrian
nessuno.» Mi sfilai dal dito l'anello con il sigillo... quel dannato
arnese che mi aveva messo in tutto quel pasticcio... e con uno sforzo
minimo lo scaraventai dall'altra parte della stanza. Che i droni delle
pulizie se lo tenessero, che lo mettessero nel bidone dei rifiuti e lo
portassero agli inceneritori. In ogni caso era senza valore, diseredato
com'ero le mie proprietà dovevano essere state annullate nel momento
stesso in cui era successo, e la terra che avevo su Delos era stata
assegnata per default alla prefettura di Meidua e al Casato Marlowe.

Il Casato Marlowe.

Avevo creduto di essere io il Casato Marlowe, ma ne ero soltanto
un'estensione, un'appendice. Avevo sempre pensato che il mio Casato
fosse composto di individui, ma perfino su questo strano mondo avevo
dovuto fare affidamento sul mio nome e sull'anello che lo simboleggiava
per significare qualcosa. Ci crediamo padroni di simboli del genere, ma
i veri padroni sono loro. Diavoli. Sfingi. Soli. Mi ero aggrappato a
quel dannato anello come a un talismano, sperando che mi proteggesse, e
invece mi aveva condannato, aveva reso catastrofico il mio comportamento
idiota. «La Spada, Nostro Oratore!» sibilai, ripetendo le parole della
mia famiglia e trasformandole in un'imprecazione. «Vorrei che non fosse
così.»

Le luci si affievolirono in una delle fin troppo familiari fluttuazioni
di energia del castello.

Senza preamboli o preavviso, Valka mi colpì con forza in piena faccia.
Fu più il suono che la forza del colpo a cogliermi di sorpresa. Stupito,
mi premetti una mano sulla guancia. «Smettila» ingiunse lei, aggrottando
le sopracciglia nel protendersi verso di me. «Va bene che tu abbia
ripensamenti, ma arrivano troppo tardi. Non puoi biasimare altri che te
stesso, lo capisci? Questo non sta succedendo \emph{a} te, ma \emph{per
	causa} tua.»

`Lo stolto crede che le iniquità del mondo siano colpa degli altri
uomini.' La voce di Gibson, arida come le pagine di un vecchio
manoscritto, non era mai stata più nitida. `Chi è veramente saggio cerca
di cambiare sé stesso, il che è un compito più difficile e meno
grandioso.' Che bisogno avevo quindi del Casato Marlowe? Di
quell'inutile anello?

La faccia mi doleva dove lei mi aveva colpito, ma era un dolore remoto.
Non obiettai perché aveva ragione. «I suoi occhi, Valka, gli occhi di
Gilliam. Io... li ho visti... vuoti. Un momento era lì e il
successivo...» Avevo visto altri morti, prima di allora, avevo visto i
loro occhi simili a soli lontani e altrettanto freddi, ma non avevo mai
osservato neppure una volta il momento del passaggio. Perfino Cat, che
mi era morta fra le braccia, era passata oltre con gli occhi chiusi. «È
stato orribile. Orribile...» Mi parve di vedere il suo scintillante
occhio azzurro che mi fissava, velato dalla morte. L'occhio di un
avvoltoio.

Valka emise un verso di conforto a cui seguì una lunga pausa di
gradevole silenzio. Oscillai sull'orlo del sonno, quel morbido e
temporaneo oblio, ma non vi sprofondai. «Posso venire con te?» ripetei
infine. «A Calagah? Non voglio...» Artigliai l'aria in direzione
dell'anello. «Non voglio niente di tutto questo.»

Arricciò le labbra. «E cosa mi dici di Anaïs?»

Sbuffai. «Lei che c'entra?»

«Siete fidanzati.» Accennò alla stanza che mi circondava e, per
estensione, al castello al di là di essa. «Dovrai tornare qui.»

«Lo so.» Ripiegai le gambe fin quasi al mento e soffiai per togliermi i
capelli dagli occhi. «Ma non sono obbligato a restare qui.»

Valka sorrise di nuovo -- quel caldo sorriso che mi aveva elargito in
precedenza e non quello freddo che sfoggiava abitualmente -- e mi
appoggiò la mano tatuata contro la guancia. Pronunciò parole che non
riuscii a sentire, o forse non parlò affatto... quel momento è una calda
caligine, una chiazza lasciata nella mia memoria mentre l'oscurità e le
prove di quella giornata mi trascinavano via con loro.

