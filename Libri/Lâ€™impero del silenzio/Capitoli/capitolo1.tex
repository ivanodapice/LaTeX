\mainmatter

\chapter{Hadrian}

{~}

{~}

{~}

{-- \emph{Io ho guardato con i suoi occhi, ho ascoltato con i suoi
		orecchi, e le dico che è l'unico. O almeno, il migliore che possiamo
		avere.}}

{-- \emph{Questo lo aveva detto anche del fratello.}}

{-- \emph{I test hanno rivelato che il fratello è inadatto. Per altre
		ragioni. Niente a che vedere con le sue capacità.}}

{-- \emph{Lo stesso per sua sorella. E su di lui ci sono dei dubbi. È
		troppo malleabile. Si adegua troppo volentieri alla volontà degli
		altri.}}

{-- \emph{Non se questi altri sono suoi nemici.}}

{-- \emph{E allora cosa dovremmo fare? Circondarlo di nemici giorno e
		notte?}}

{-- \emph{Se sarà necessario.}}

{-- \emph{Credevo d'averle sentito dire che questo bambino le piace.}}

{-- \emph{A confronto di ciò che gli potrebbero fare gli Scorpioni, io
		gli sembrerei uno zietto affettuoso.}}

{-- \emph{E va bene. Dobbiamo salvare il mondo, dopotutto. Lo prenda.}}

{~}

{* * *}

{~}

{La donna del monitor sorrise con molta simpatia, gli scarruffò i
	capelli e disse: -- Credo proprio che tu non ne possa più di avere
	quell'orribile monitor, Andrew. Be', ho buone notizie per te. Oggi è
	l'ultimo giorno che lo porti. Adesso te lo leveremo, e non sentirai male
	neppure un poco.}

{Ender annuì. Che non gli avrebbero fatto male, naturalmente, era una
	bugia. Ma visto che gli adulti dicevano sempre così quando \emph{faceva}
	male, lui poteva basarsi su quella frase per un'accurata previsione di
	quel che lo aspettava. A volte le bugie risultavano più affidabili della
	stessa verità.}

{-- Bene, Andrew, se vuoi venire qui, intanto puoi sederti sul lettino
	per le visite. Il dottore verrà a occuparsi di te fra un minuto.}

{Il monitor tolto. Ender cercò d'immaginare la sua nuca priva del
	minuscolo apparecchio. \emph{A letto potrò girarmi sulla schiena senza
		sentirmi pigiare qui. Non lo sentirò più formicolare freddo quando
		faccio il bagno.}}

\emph{{E Peter non mi odierà più. Appena torno a casa gli faccio vedere
		che mi hanno levato il monitor, così saprà anche che non ce l'ho fatta.
		E che sarò un bambino qualsiasi, adesso, come lui. Non sarà più così
		crudele, allora. Dimenticherà che io ho tenuto il monitor per un anno
		più di lui. E saremo\ldots{}}}

{Non amici, probabilmente. No, Peter era troppo pericoloso. Peter andava
	in collera troppo facilmente. Fratelli, comunque. Non nemici, non amici,
	ma fratelli\ldots{} capaci di vivere nella stessa casa. \emph{Non mi
		odierà, mi lascerà in pace. E quando avrà voglia di giocare a Scorpioni
		e Astronauti, forse sarò io a non volere, forse me ne andrò a leggere un
		libro.}}

{Ma anche mentre si diceva questo, Ender sapeva che Peter non avrebbe
	smesso di prendersela con lui. C'era qualcosa nei suoi occhi quando
	Peter era in vena di pazzia, e ogni volta che lui vedeva quello sguardo,
	quel lampo nelle pupille, poteva star certo che Peter avrebbe fatto di
	tutto salvo che lasciarlo in pace. Voglio esercitarmi al piano, Ender.
	Vieni a girare le pagine per me. Oh, il bambino col monitor ha troppo da
	fare per aiutare suo fratello? Si crede molto intelligente, vero? Vuoi
	ammazzare un po' di Scorpioni, Astronauta? No, no, io non ho
	\emph{bisogno} del tuo aiuto. Posso benissimo fare da solo, razza di
	bastardo, piccolo stupido \emph{Terzo}!}

{-- Non ci vorrà molto, Andrew -- disse il dottore.}

{Ender annuì.}

{-- È progettato per essere rimosso. Senza infezioni e senza danni. Ma
	proverai un po' di prurito, e qualcuno a volte dice d'avere la
	sensazione che gli manchi \emph{qualcosa.} Capiterà anche a te di
	guardarti intorno come in cerca di questo qualcosa, senza trovarlo, e
	senza neanche sapere cosa stai cercando. Perciò te lo dico io: quello
	che ti scoprirai a cercare è il monitor, e non ci sarà più. In pochi
	giorni questa sensazione sparirà.}

{Il dottore stava girando un oggetto dietro la testa di Ender. A un
	tratto un ago rovente di dolore lo attraversò dalla nuca all'inguine. I
	muscoli della schiena gli si contrassero di colpo e s'inarcò
	all'indietro, con violenza, sbattendo la testa sul lettuccio. Si accorse
	che le sue gambe scalciavano a vuoto, e aveva le mani strette l'una
	all'altra così forte da fargli male.}

{-- Deedee! -- gridò il dottore. -- Ho bisogno di te! -- L'infermiera
	sopraggiunse di corsa, ansando. -- Cerca di fargli rilassare questi
	muscoli. Qui, tira verso di me, adesso. Che stai aspettando?}

{Altre mani s'impadronirono di lui, ma Ender non poteva vedere niente.
	Si torse di lato e cadde giù dal lettino delle visite. -- Lo blocchi! --
	strillò l'infermiera.}

{-- Basta che tu lo tenga saldamente e\ldots{}}

{-- Lo tenga lei, dottore, è troppo forte per me\ldots{}}

{-- Non tutta la fiala! Vuoi rischiare di fermargli il cuore?}

{Ender sentì la puntura di un ago giusto sopra il colletto della
	camicia, dietro la nuca. Bruciava, ma dovunque quel bruciore si
	espandeva i suoi muscoli si rilassavano gradualmente. Adesso riusciva ad
	aprire la bocca per gemere, spaventato e dolorante.}

{-- Va meglio, Andrew? -- lo interrogò l'infermiera.}

{Ender non ricordava neppure come si facesse a parlare. I due lo
	rimisero sul lettino. Gli controllarono le pulsazioni e fecero altre
	cose, che lui non fu assolutamente in grado di capire. Il dottore stava
	tremando; quando parlò la sua voce era rauca. -- Lasciano questa roba
	addosso ai ragazzini per tre anni, e poi cosa si aspettano? Avremmo
	potuto rovinarlo, ti rendi conto? Avremmo potuto alterare il suo
	cervello irreversibilmente.}

{-- Quanto dura l'effetto del tranquillante? -- chiese l'infermiera.}

{-- Tienilo qui per almeno un'ora. Sorveglialo. Se fra quindici minuti
	non riesce ancora a parlare, chiamami. Potremmo averlo rovinato per
	sempre. Certa gente si comporta peggio degli Scorpioni, maledizione!}

{~}

{* * *}

{~}

{Rientrò nella classe di miss Pumphrey appena quindici minuti prima che
	suonasse l'ultima campanella. Era ancora un po' instabile sulle gambe.}

{-- Ti senti bene, Andrew? -- domandò miss Pumphrey.}

{Lui annuì.}

{-- Hai avuto la febbre?}

{Lui scosse il capo.}

{-- Mi sembri pallido.}

{-- Sto benissimo.}

{-- Meglio che ti sieda, Andrew.}

{Lui si diresse al suo posto, ma si fermò. \emph{E adesso cosa sto
		cercando? Non riesco a ricordare cosa sto cercando.}}

{-- Il tuo banco è dall'altra parte -- disse miss Pumphrey.}

{Lui sedette, ma la cosa di cui sentiva il bisogno era un'altra,
	qualcosa che gli sembrava d'aver perso. \emph{La cercherò più tardi.}}

{-- Il tuo monitor -- sussurrò la bambina dietro di lui}

{Ender scosse le spalle.}

{-- Il suo monitor! -- la sentì sussurrare agli altri.}

{Ender alzò una mano a tastarsi la nuca. Le sue dita incontrarono un
	cerotto. Gliel'avevano tolto. Adesso era come tutti gli altri.}

{-- Ti senti giù, eh, Andy? -- chiese un bambino della fila accanto, un
	posto più indietro. \emph{Non riesco a ricordare come si chiama. Peter.
		No, quello è qualcun altro.}}

{-- Silenzio laggiù, signor Stilson -- disse miss Pumphrey. Stilson
	ridacchiò sottovoce.}

{Miss Pumphrey stava parlando delle moltiplicazioni. Ender cominciò a
	scribacchiare sullo schermo del banco, disegnò i contorni orografici di
	alcune isole montuose e poi ordinò al banco di svilupparglieli in tre
	dimensioni da ogni angolo visivo. La maestra, naturalmente, si sarebbe
	accorta che non stava attento, ma questo non lo preoccupava. Sapeva
	sempre quali risposte dare, anche quando lei era convinta che fosse
	distratto.}

{Nell'angolo in basso del banco una parola apparve e cominciò a
	scivolare lungo il bordo dello schermo. All'inizio era capovolta, ma
	Ender ne conosceva il significato già molto prima che ruotando sul lato
	superiore del banco si raddrizzasse.}

{~}

{TERZO}

{~}

{Ender sorrise. Era stato lui a scoprire il modo di mandare messaggi e
	farli muovere: anche se quel suo nemico anonimo lo stava insultando, il
	metodo scelto per farlo lo inorgogliva. Non era colpa \emph{sua} se era
	un Terzo. L'idea l'avevano avuta quelli del Governo, i soli che potevano
	autorizzare una cosa simile\ldots{} altrimenti come avrebbe potuto un
	Terzo come lui essere iscritto a scuola? E adesso il monitor non c'era
	più. L'esperimento etichettato «Andrew Wiggin» non aveva funzionato,
	dopotutto. Se avessero potuto farlo, era certo che avrebbero volentieri
	ritirato anche il permesso speciale in base al quale lui era stato messo
	al mondo. Esperimento fallito: cancellare e gettare via.}

{La campanella suonò. Gli alunni cominciarono a spegnere i banchi, e
	alcuni batterono in fretta gli ultimi appunti. Altri stavano trasferendo
	i dati della lezione al computer di casa loro. Due o tre si misero in
	fila davanti a una stampante per farsi riprodurre qualche illustrazione
	che li aveva interessati. Ender poggiò le mani sulla piccola tastiera
	del banco, adatta alle dita di un bambino, e si chiese cosa si provasse
	ad avere mani larghe come quelle degli adulti. Dovevano sentirsele
	massicce e goffe, con quei ruvidi palmi carnosi. Naturalmente essi
	avevano tastiere più grandi\ldots{} ma come avrebbero potuto i loro
	pesanti polpastrelli tracciare una linea così fine e precisa che poteva
	farla spiraleggiare settantanove volte dal centro del banco verso i
	lati, senza che si sovrapponesse mai. Questo almeno gli teneva occupate
	le mani, intanto che la voce della maestra gli ronzava negli orecchi
	noiose spiegazioni di aritmetica. Aritmetica! Valentine gli aveva
	insegnato quella roba quando lui aveva appena tre anni.}

{-- Ti senti meglio, Andrew?}

{-- Sì, signora.}

{-- Perderai l'autobus.}

{Ender annuì e si alzò. Gli altri ragazzini erano usciti. Lo avrebbero
	aspettato però, quelli più perfidi. Nella sua nuca non c'era più un
	monitor a udire quel che udiva lui, e a vedere ciò che vedeva. Potevano
	dirgli tutto quello che s'erano tenuto in bocca finallora. Avrebbero
	potuto anche picchiarlo: non ci sarebbero stati altri occhi a
	osservarli, e dunque nessuno sarebbe comparso a difendere Ender. Il
	monitor aveva comportato anche dei vantaggi, e adesso li aveva perduti.}

{Ad attenderlo fu Stilson, naturalmente. Non era più robusto di altri
	ragazzini, ma superava Ender di tutta la testa. E con lui c'erano i suoi
	amici, cinque o sei. Come sempre.}

{-- Ehi tu, Terzo.}

\emph{{Non rispondere. Non hai niente da dirgli.}}

{-- Ehi, Terzo! Stiamo parlando con te, Terzo. Ehi, amico degli
	Scorpioni, è con te che parliamo.}

\emph{{Non riesco neanche a pensare a qualcosa da dire. E dire qualsiasi
		cosa sarebbe peggio. Così starò zitto.}}

{-- Ehi, Terzo, Terzetto, stronzetto\ldots{} fai finta d'essere sordo,
	eh? Pensavi di essere meglio di noi, eh? Ma adesso l'hai perduto
	l'occhio spione, Terzino stronzone, e sulla testa ti ci han messo un
	tampone!}

{-- Volete lasciarmi passare, o no? -- chiese Ender.}

{-- Vogliamo lasciarlo passare, o no? Dobbiamo lasciarlo passare? --
	tutti risero. -- Sicuro che ti lasciamo passare. Prima lasciamo passare
	i tuoi denti, però. E poi la testa. E poi lasciamo passare anche il tuo
	culo, a calci.}

{I ragazzini cominciarono a girare in cerchio, stringendosi attorno a
	lui. -- L'occhio-spia te l'hanno rotto, Terzotto! L'occhio-spia ha fatto
	fagotto, Terzotto!}

{Stilson gli appoggiò una mano in mezzo al petto e lo spinse; qualcuno,
	dietro di lui, lo proiettò di nuovo verso Stilson.}

{-- Vuoi giocare all'altalena, Terzo? -- gridò un altro.}

{-- Vuoi giocare alla palla da tennis, Terzo?}

{Uno spintone lo gettò indietro. -- Sei una palla da pingpong, Terzo?}

{Ender capì che la cosa sarebbe finita male. Ma finisse come finisse,
	decise, lui non sarebbe stato il solo a piangere. E appena Stilson fece
	per spingerlo ancora, lui lo afferrò per il petto. L'altro si liberò con
	uno strattone.}

{-- Ah, vuoi sfidarmi, eh? Vuoi batterti con me, Terzocchio?}

{I ragazzini alle spalle di Ender lo afferrarono per le braccia e lo
	tennero fermo.}

{Ender non aveva nessuna voglia di ridere, ma rivolse loro un sogghigno
	misurato. -- Ci vogliono cinque di voi per picchiare un Terzo solo?}

{-- Noi siamo \emph{normali}, \emph{} non \emph{Terzi}, \emph{} faccia
	di merda. Tu non hai la forza di una scoreggia!}

{Ma gli tolsero le mani di dosso. E nello stesso istante in cui lo
	lasciavano Ender colpì Stilson allo sterno con un pugno in cui mise
	tutta la sua forza. L'avversario cadde lungo disteso. Questo lo colse di
	sorpresa: non s'era aspettato di mettere a terra Stilson con un sol
	pugno. Non si rese conto che l'altro doveva aver preso la sfida alla
	leggera, e non era stato preparato a un colpo così disperato.}

{Nel vedere l'immobilità di Stilson gli altri sbarrarono gli occhi e si
	azzittirono, come chiedendosi se fosse vivo o morto. Ender stava invece
	pensando a come rintuzzare la prevedibile vendetta del ragazzo.
	L'indomani Stilson avrebbe fatto polpette di lui. \emph{Devo vincere
		adesso, e una volta per tutte, altrimenti mi dovrò battere con lui di
		continuo e ogni giorno sarà peggio.}}

{Benché avesse appena sei anni Ender conosceva le regole non scritte
	della lotta. Era proibito infierire sull'avversario che giaceva a terra
	inerme; soltanto un animale l'avrebbe fatto.}

{Così si accostò a Stilson e lo colpì con un violento calcio nelle
	costole. Lui emise un grugnito e cercò di rotolare via. Ender gli girò
	attorno e gli sferrò una pedata al basso ventre. Dalla bocca di Stilson
	non uscì un lamento, ma si piegò in due e i suoi occhi si empirono di
	lacrime.}

{Ender rivolse agli altri uno sguardo freddo. -- Forse vi sta venendo
	l'idea di buttarvi su di me. Probabilmente mi potete picchiare a sangue.
	E allora guardate quello che faccio alle carogne. Se ci provate, saprete
	che d'ora in poi aspetterò di trovarvi da soli, e saprete che vi
	succederà questo. -- Con un altro calcio colpì Stilson in piena faccia.
	Il sangue gli uscì dal naso e ruscellò sul pavimento. -- Solo che con
	voi non sarà così -- disse. -- Sarà molto peggio.}

{Volse loro le spalle e si allontanò. Nessuno provò a seguirlo. Uscito
	da scuola s'avviò nel corridoio sotterraneo verso la fermata del bus, e
	fece in tempo a sentire uno di loro che diceva: -- Gesù! Guardalo, gli
	ha spaccato la faccia. -- Ender appoggiò la fronte alla parete, e pianse
	fino all'arrivo dell'autobus. \emph{Sono uguale a Peter. Mi avete levato
		il monitor, e adesso sono proprio come Peter.}}