\chapter{Il limitare del mondo}

Il libero mercante non era affatto come mi ero aspettato, ma del resto
non avevo saputo bene cosa aspettarmi. Demetri Arello era uno Jaddiano
esile come uno stocco, con la pelle del colore del bronzo oliato, e nel
sorridere mostrava denti tanto bianchi da farmi comprendere che dovevano
essere impianti di ceramica. «È strano che un nobile sia tanto disperato
da abbassarsi al mio livello» commentò con una risatina autoironica
mentre si appoggiava all'indietro sul sedile, arruffandosi i capelli
luminosi come una stella. Erano ancora più vividi dei suoi denti, di un
bianco intenso e lucente.

«Al tuo livello?» ripetei, versandomi del vino dalla caraffa di vetro.
Fuori dalla porta ad arco la giornata era calda e afosa, e potevo
sentire i rumori di costruzione che provenivano da quel complesso di
appartamenti finito a metà che stava sorgendo vicino ai moli. «Cosa
intendi dire?»

Arello sorrise. «La mia nave è veloce, ma non è un incrociatore di
lusso.» Mi esaminò con occhi scrutatori, mordicchiandosi un labbro con
aria concentrata. «Potresti non essere molto a tuo agio.» Si accarezzò
il mento liscio con una mano adorna di anelli senza che il suo sorriso
mostrasse di incrinarsi.

«Non cerco comodità» risposi. «Solo un passaggio fino a Teukros.»

Mi interruppe, includendo nella sua occhiata anche Kyra, che sedeva in
mezzo a noi. «È chiaro che le comodità non sono il tuo scopo, altrimenti
perché porteresti con te questa ragazza?» Sorrise ancora. Sorrideva
sempre. Kyra non rispose, ma potevo sentire l'urgenza che emanava da lei
a ondate. Voleva che concludessimo in fretta quella conversazione.

«Se cercassi le comodità, messere, sarei rimasto a casa.»

«Infatti. Anche se a quanto ho capito, non hai una casa in cui stare,
giusto?» Svuotò il suo bicchiere e fece una smorfia. «Come fate voi
imperiali a bere questo piscio di cavallo? Mi chiedo...» Scosse il capo.
«Nella mia patria lapiderebbero l'uomo che osi vendere una cosa del
genere.

«La tua nave è veloce?» intervenne Kyra, che a quanto pareva aveva
superato il limite della pazienza.

«Abbastanza per la dama che l'ha noleggiata.» Nonostante le sue
critiche, sollevò la caraffa e tornò a riempirsi il bicchiere con il
denso liquido fra il rosso e il nero, bevendo un altro sorso indagatore.
«Se non altro è forte.» Posò il bicchiere e si appoggiò all'indietro e
di traverso, tendendo le pieghe morbide della veste verde e arancione
che portava sul petto altrimenti nudo e glabro. «Senti, se vuoi arrivare
a Teukros, allora la \emph{Eurynasir} ti porterà su Teukros, La nostra
rotta passa da Obatala e da Siena. È un viaggio di tredici anni.»

«Da Obatala...» Mi interruppi, accigliandomi. «Non è un viaggio
diretto?» Guardai verso Kyra, che era lì solo per accertarsi che
arrivassi a bordo della nave del mercante. Si era tolta la tuta da
pilota e indossava semplici abiti civili, leggings aderenti e una larga
tunica con una stampa di salamandre albane e il nome di un qualche
gladiatore del Colosso. Le si addiceva.

Le sopracciglia candide di Arello si contrassero. «Diretto? Fino a
Teukros? Un viaggio fottutamente lungo, amico mio. Non intendo
imbarcarmi in un lungo viaggio su per la spirale contando solo sui
proventi di un incarico come corriere. Ho un equipaggio da nutrire e da
pagare, e se devo andare così lontano puoi scommettere il tuo pallido
culo palatino che ci fermeremo per commerciare. La guerra ha fatto
salire ogni sorta di richieste... un uomo potrebbe fare di sé stesso un
re.»

Kyra si protese verso di me. «Non mi piace perdere tempo in questo modo,
mio signore» sussurrò. «Presto si accorgeranno della mia assenza.»
Contro le mie obiezioni, aveva insistito per accompagnarmi nella taverna
per l'incontro con il capitano jaddiano {nonostante} le cinque ore del
viaggio di ritorno ad Haspida. Adesso era mattino avanzato e a giudicare
dall'orologio su una parete mancava solo un'ora al momento previsto per
la mia partenza dal palazzo d'estate alla volta del \emph{Lavoratore
	Lontano} e di Vesperad. Avevo abbandonato il mio terminale nel palazzo
perché non volevo che mi potessero rintracciare con il suo segnale.

«È probabile che si siano già accorti della tua assenza» dissi in tono
serio, ancora riluttante a guardarla negli occhi. Che fosse o meno un
tenente subalterno, ci si sarebbe aspettati che fosse presente ai
controlli dei sistemi. Potevo solo sperare che mia madre avesse
l'astuzia necessaria per guadagnare tempo. Forse avrebbe detto che nella
fuga avevo stordito anche Kyra e che l'avevo lasciata in qualche posto
fuori mano dove non era stata trovata. Avrebbe escogitato qualcosa.

«Questa è una conversazione privata?» chiese Demetri, con quel suo
accento cantilenante che era quasi felino nella sua pronuncia pigramente
strascicata. «Oppure può partecipare chiunque, eh? Questa inerzia non mi
piace più di quanto piaccia a voi, ma devo essere certo che abbiamo
un'intesa.» E si posò una mano sul cuore, come un vassallo che giurasse
fedeltà al suo signore.

Socchiusi gli occhi con espressione estremamente seria. «Quale intesa?
Sei stato pagato, giusto?»

«Sì, sì.» Demetri Arello annuì vigorosamente. «Cinquemila hurasam in
anticipo e la promessa di novemila marchi dalla tua banca di Teukros
quando arriverai.» Agitò una mano come per accantonare quei pensieri
quasi fossero stati altrettante mosche. «Tutto questo va benissimo,
ma... come posso dire? Tu sei un nobile, e i nobili sono... come mi
posso esprimere? Sono complicati» continuò, fissando in modo espressivo
prima me e poi Kyra. Lo fissai a mia volta, mentre entrambi aspettavamo
che lui sbattesse le palpebre. Silenzio. Ho sempre trovato che è lo
strumento più efficace in qualsiasi conversazione, quindi attesi che
fosse il mercante a riprendere la parola. Il fracasso dei lavori di
costruzione si arrestò per un momento e sentii un uomo lanciare un
richiamo in un gergo di strada. «Non sei un qualche tipo di criminale,
vero?»

Sconcertato, inarcai un sopracciglio. «Cosa? No.» Cosa gli aveva detto
mia madre?

«È solo che non voglio che i miei uomini vengano messi in pericolo»
spiegò Demetri, tenendo lo sguardo fisso su di me nel riempire ancora il
bicchiere, incapace di mascherare il sogghigno di disappunto al suono
del liquido che si riversava dentro di esso. «Abbiamo già abbastanza
guai per conto nostro senza trovarci immischiati nella politica di
Delos.»

Lanciai un'occhiata in tralice a un poster sbiadito di una
rappresentazione operistica che raffigurava una donna nuda dalla pelle
nera che brandiva una spada di altamateria e teneva un piede piantato
sulla faccia di un legionario imperiale.
\foreignlanguage{italian}{tiada, la principessa di thrax} c'era scritto.
«Mi stai portando lontano dalla politica di Delos.» Quando Demetri parve
sul punto di discutere passai a esprimermi in jaddiano e dissi: «Senti,
tu vieni dai principati, vero?»

Lui sbatté le palpebre e la sorpresa gli si dipinse sul volto affilato.
«Sì... \emph{sì}» ripeté nella sua lingua, studiandomi con occhi velati.

«Cosa ne pensi della Cappellania della Terra?» L'espressione gioviale di
Demetri si incrinò e lui fece una smorfia, come se avesse bevuto altro
di quel vino aspro. Soddisfatto continuai, sempre in jaddiano: «È quello
che pensavo. Ebbene, si dà il caso che io la pensi nello stesso modo,
\emph{mi sadji}, ma mi volevano mandare in seminario. Tu mi stai
aiutando a fuggire.» Sorrisi di nuovo, un sorriso involontariamente in
tralice come lo era in tutti i membri della mia famiglia. In quel
momento mi resi conto del mio accento, della raffinatezza dell'antica
élite imperiale, dell'essere progenie dei più antichi Casati dei mondi
interni. Era una voce che veniva associata al cattivo nel genere di
opere raffigurate sui poster alquanto di cattivo gusto che coprivano le
pareti della taverna.

Demetri serrò la mascella appuntita, poi si protese in avanti e sibilò,
questa volta nel galstani imperiale: «La stai ficcando su per il culo
alla Cappellania, è così?» Lanciò un'occhiata al di sopra della mia
spalla, in direzione dell'altro lato della sala comune dove un paio di
drogati di jubala erano attaccati al loro narghilè. Quei fumatori erano
le sole altre persone presenti in quella squallida taverna, vuoi perché
erano i primi clienti della giornata o gli ultimi della notte
precedente. Continuò a tenerli d'occhio. «Ho sentito tutto questo dai
nostri amici del Consorzio, ma volevo solo accertarmi che non ci fosse
sotto niente di più. Niente di... incasinato.»

Il corpo di Crispin che giaceva privo di sensi sul pavimento sbocciò
come un fiore nella mia mente, e a giudicare dai tratti affilati di
Kyra, anche lei stava pensando la stessa cosa. «No, no, niente del
genere.»

Se si accorse che stavo mentendo, Demetri non parve curarsene. Buttò giù
il vino in un solo sorso e sussultò per il suo sapore orribile, poi
socchiuse gli occhi e a voce più bassa domandò: «Tu chi sei?»

«Te l'ho detto» risposi. «Mi chiamo Hadrian.»

Lui agitò un lungo dito nella mia direzione e notai un tatuaggio
leggermente riflettente che gli brillava sul dorso della mano. «No, no,
no, posso non essere originario del tuo Impero ma non sono un cane
bastardo da prendere a calci e a cui mentire. Tu sei Hadrian
\emph{qualcosa.}» Puntò il dito in direzione di Kyra. «Questa piccola
femmina non è un'amica. È una serva, giusto? Una guardia del corpo,
magari?» Quando esitai il suo sorriso da imbonitore si accentuò e lui si
appoggiò allo schienale con una risata sommessa, giocherellando con il
medaglione d'oro triangolare che aveva al collo. «Quale Casato? Feng non
me lo ha voluto dire.» Non vedendo nessuna utilità a negare, mi tolsi
l'anello dal pollice e glielo mostrai. Poteva essere un estraneo, ma
alla sua vista si accigliò. «Avrei dovuto rifiutare la richiesta di
quella cagna.»

«Se parti in fretta non ci saranno problemi» scattò Kyra, con la
mascella serrata. Credo che la parola `cagna' riferita a mia madre --
che in segreto era la sua padrona -- avesse acceso qualcosa dentro di
lei.

«Marlowe...» Arello la stava ignorando mentre rigirava il bicchiere da
vino sul tavolo fino a farlo tintinnare. «Marlowe... non sei stato
aggredito? Qualche tempo fa? Attaccato mentre uscivi da un bordello,
giusto?»

Questo mi seccò in modo particolare per via della presenza di Kyra.
Calai con forza una mano sul tavolo mentre l'ira della notte precedente
tornava a divampare. «Non era un bordello!»

Arello rise ancora, un suono simile a legno lucido che attirò
l'attenzione dei due fumatori di jubala vicino all'arcata che portava
alla balconata esterna. «Allora si trattava di te!»

Mi accigliai. Quello era il più vecchio dei trucchi. «È successo al
Colosso.»

«Comunque sia.» Demetri agitò una mano e tornò a riempirmi il bicchiere.
«La tua adorabile guardia del corpo ha ragione, \emph{domi}. Dovremmo
andare e subito.» Sollevò il suo bicchiere vuoto a mimare un brindisi.
«Mia nonna però diceva di non sprecare mai il vino, anche un piscio di
capra come questo. Alla tua salute, \emph{mi sadji. Buon atanta}.»

«\emph{I tuo}» replicai, e buttai giù quella roba.

\begin{figure}
	\centering
	\def\svgwidth{\columnwidth}
	\scalebox{0.2}{\input{divisore.pdf_tex}}
\end{figure}

Karch si trovava ai confini del globo, quanto più lontano dalla civiltà
era possibile arrivare su un pianeta come Delos. Se avessero condiviso
il romanticismo degli antichi, i nostri cartografi avrebbero potuto
disegnare draghi e serpenti marini nelle acque che la circondavano.
Mentre Meidua era alta, con le sue torri orgogliose che si stendevano
come dita supplici verso i cieli grigi, Karch era bassa e tozza, un
vasto groviglio di edifici a due o tre piani lungo le alture rocciose
che dominavano la baia. Sulle sue acque fra il blu e il grigio
galleggiava come un ammasso di rifiuti una confusione di ponti di barche
e di galleggianti ancorati a moli di cemento come ossa in un pesce.
Moltissime navi -- a vela, a vapore e astronavi -- si radunavano là.

E la gente, per la Terra e l'imperatore, la gente. Una calca terribile,
come lo erano il suo peso, la puzza e il rumore che produceva. Per una
volta ero un uomo alto che sovrastava di quasi tutta la testa e le
spalle il più alto dei plebei di quella folla, quindi cominciai a tenere
una posa accasciata, con la sacca appesa a una spalla e la camicia
sbottonata fino allo sterno a causa dell'insolita calura. Vestiti con
abiti normali, ma con la pistola che pendeva dalla cintura-scudo, i due
legionari di mia madre trasportavano in mezzo a loro il mio baule e mi
seguivano a rispettosa distanza, mentre Kyra procedeva spedita al mio
fianco, muovendosi con una determinazione che portava la calca ad
aprirsi di sua iniziativa. I pontoni ondeggiavano sotto i nostri piedi,
muovendosi su e giù con il moto gentile delle onde.

Demetri ci aveva preceduti, e così quando ci avvicinammo alla losanga
bassa e scura che era la sua nave accoccolata sulle onde lui venne fuori
per accoglierci. Aveva aperto la veste verde e arancione, la cui seta
gli si agitava intorno. Quando agitò una mano risposi al gesto e
accelerai il passo, aggirando due massicci marinai che stavano
scaricando il loro piccolo mercantile. Li vidi a stento perché la mia
attenzione era tutta sullo scafo di un nero opaco della nave premuta
contro la superficie della baia.

La nave jaddiana mi ricordò un catamarano, con due pattini gonfi che
formavano ciascuno dei suoi fianchi e si estendevano di un poco a prua e
a poppa rispetto alla sua lunghezza di quasi quaranta metri. Il corpo
dell'astronave si innalzava dall'acqua in mezzo a loro: una cupola di
alluvetro simile a un occhio velato sbirciava in mezzo a quei pattini e
nella parte posteriore una sottile torre di comunicazione si levava fra
le pesanti pinne aeree che servivano da timoni quando la nave era
nell'acqua. Ogni pollice di spazio era scuro come l'universo e lo scafo
era un composto di adamant e di ceramiche a impatto elevato, con pezzi
di alluvetro e di titanio esposti e visibili qua e là. Detto così può
apparire affascinante, e se io fossi stato un qualche tecnico di
fattoria di una zona arretrata con a stento duemila hurasam a suo nome
forse lo sarebbe stata. Ma ai miei occhi -- quelli del figlio di un
arconte -- appariva... preoccupante.

Sottilissime fratture venavano la ceramica in alcuni punti e in altri
era calafatata o saldata. Un murale raffigurante due mani unite a coppa
si scrostava vicino alla prua, con le dita che racchiudevano le fluide
lettere jaddiane che componevano una singola parola:
\foreignlanguage{italian}{eurynasir}\emph{.}

La salsedine dell'oceano di Delos ricopriva le parti inferiori dello
scafo e le volute di fumo che salivano dai propulsori di elevazione
posteriori mi fecero pensare a una qualche antica locomotiva a legna. Se
quella nave aveva generatori di campo di soppressione, non riuscivo a
scorgerli.

«Una bella nave, capitano» commentai, abbassando la mano. «Spero che non
ti abbiamo fatto aspettare.» Era trascorsa forse mezz'ora da quando ci
eravamo separati in quella squallida taverna con la sua aria che odorava
di jubala. Là sul pontone, invece, l'aria puzzava dell'ozono dei
propulsori a fusione e del carburante diesel dei motori fuoribordo.

Demetri Arello sorrise, con i denti candidi che brillavano nella luce
del sole mentre si legava una fusciacca verde intorno alla vita sottile.
«Sei arrivato giusto in tempo. Spicciati.» Avvistò i due soldati in
abiti civili che trasportavano il mio baule e il suo sorriso si incrinò
mentre aggiungeva: «Se mai avessi avuto dei dubbi sulla tua identità,
questo li avrebbe dissolti.» Guardò i due soldati posare il baule.
«Possiamo portarlo dentro.» Si morse di nuovo il labbro e mi guardò come
se fossi stato un campione su un vetrino di microscopio. Tamburellò con
le dita contro le gambe.

«Un momento.» Mi girai verso Kyra. «Hai fatto tutto il possibile,
tenente. Prendi con te gli altri e vai. Con un po' di fortuna la tua
assenza non sarà notata.»

Lei scosse il capo, agganciando un pollice nella cintura della tunica.
«È troppo tardi per questo.»

D'un tratto la punta del mio stivale assorbì tutta la mia attenzione e
fu a essa che mi rivolsi invece che alla donna che avevo davanti. «Mi
dispiace.» Volevo che lei dicesse qualcosa, qualsiasi cosa. Che mi
dicesse che era tutto a posto. Pensai alla minaccia di Crispin nei suoi
confronti e aggiunsi: «Mia madre ti proteggerà, lo giuro. Chiedile di
assegnarti una posizione presso mia nonna, in un qualsiasi posto lontano
dal castello.» \emph{E da mio fratello}.

«Starò bene» replicò, sulla difensiva, e si girò per andarsene. Non
potevo biasimarla per la sua fretta, ma la trattenni per un polso.

«Kyra, aspetta.» Lei si girò parzialmente per guardarmi con gli occhi
duri fissi con espressione rovente sulla mia mano che la tratteneva.
Adesso mi chiedo se pensasse che l'avrei baciata di nuovo, ma non feci
niente del genere. Sapevo che quello era un momento importante, che il
suo era l'ultimo volto familiare che avrei mai visto, l'ultimo pezzo
umano della mia vita prima della fine dell'infanzia, e volevo dire
qualcosa che avrebbe ricordato. Invece la lasciai andare e mi premetti
il pugno contro il petto in un saluto, riuscendo solo a ripetere quello
che avevo già detto. «Mi dispiace.»

Desiderai che mi rispondesse ma non lo fece. Si limitò ad annuire, si
girò e scomparve, passando fra i due legionari che imitarono il mio
saluto e si mescolarono alla folla. Nella mia memoria, sono là a
guardare tre soldati in abiti civili scomparire fra la calca su un molo
oscillante, ma quello è un sogno. Non passò neppure un secondo prima che
Demetri mi afferrasse per una spalla con dita insistenti. «Spicciati,
ragazzo. Sprechiamo tempo.»

«Sì» risposi con voce fievole, allungando il collo e tastandomi le
tasche per controllarne il contenuto: il mio coltello, la mia
identificazione statica, alcuni hurasam, la lettera che Gibson aveva
scritto per me e la carta universale che avevo ottenuto da Lena Balem e
dalla Gilda Mineraria. Ventimila marchi erano una cosa preziosa. Una
volta lasciato il pianeta, lontano da mio padre e dai suoi occhi
indiscreti, sarebbero stati sufficienti a cominciare quasi qualsiasi
genere di vita. Nonostante la lettera di Gibson, potevo andare \emph{da
	qualsiasi parte}. Quei ventimila erano abbastanza per pagarmi un
passaggio su una nave. Su parecchie navi. Fra quei marchi e il mio
sangue potevo comprare a credito una nave e diventare un mercante o un
mercenario. Immaginai di andare su Judecca, come Simeon il Rosso, di
spezzare il pane con gli aviari irchtani, di vedere l'universo. Non
riuscii a trattenere un sorriso.

\emph{Teukros, per prima cosa.}

Chinandomi, aiutai Demetri con il mio baule, percorrendo la rampa di
imbarco ed entrando nella fresca e sterile penombra del portello stagno,
lasciandomi alle spalle per l'ultima volta il sole argenteo e il cielo
di casa.

