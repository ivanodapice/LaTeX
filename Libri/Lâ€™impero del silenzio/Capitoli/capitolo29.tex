\chapter{Meno ali per volare}

Con le tasche insolitamente pesanti dopo una trattativa con un
prestatore su pegno meno che scrupoloso, Cat e io ci fermammo a comprare
panini con fette di carne creata nelle vasche in una bottega la cui
proprietaria non ci scacciò a vista. Ci presentammo alla porta
posteriore, come ci aveva detto di fare la volta precedente e mangiammo
su un angolo di strada in vista del canale, guardando passare sulle sue
acque stagnanti piccole imbarcazioni a remi o a motore.

«Parlami ancora del tuo castello, Had» chiese Cat, quando ebbe finito
metà del suo cibo.

Naturalmente le avevo detto chi e che cosa ero. All'inizio non mi aveva
creduto, aveva pensato che fossi soltanto un altro marinaio bugiardo
abbandonato su un mondo sconosciuto, ma quando le avevo mostrato il mio
anello mi aveva creduto con l'innocenza di qualcuno il cui mondo
personale era limitato ai confini di una città, per il quale la vastità
dell'Impero e della galassia era solo una fiaba.

Finii di mangiare e mi pulii la bocca dal sugo con il dorso della mano.
«In realtà non c'è molto da dire. Conosci la storia.» Le avevo spiegato
qualcosa, solo una parte, del perché ero fuggito.

«Non la storia!» Mi pungolò una guancia, girandomi la testa da un lato.
«Com'era?»

Sorridendo, posai il mio panino sul suo piccolo vassoio di carta e la
guardai intensamente con la testa inclinata da un lato. Cercai di
parlare, poi distolsi di colpo lo sguardo e le battei una mano su un
ginocchio. Non sono mai stato granché come poeta, quindi trovai con
fatica le parole. «Era freddo, e tutto era pulito. Non era il genere di
posto che dava la sensazione che ci fossero persone che ci vivevano.
Tutto e tutti avevano un loro posto.» Scossi il capo. «È difficile da
spiegare.»

«Ma che aspetto aveva? C'erano delle torri?»

Scoppiai a ridere e accentuai la stretta sul suo ginocchio. Lei scivolò
un po' più vicina e posò la mano sulla mia. «Sì, c'erano delle torri,
tutte di granito e vetro nero che intercettava la luce del sole...» La
osservai mentre condividevo con lei la storia di come Julian Marlowe
aveva innalzato quel vecchio castello dopo aver aiutato il duca Ormund
ad assicurarsi il potere su Delos. Mentre intessevo le mie storie sui
giorni più antichi dell'Impero, notai la luce nei suoi occhi color miele
e sentii creparsi lo strato di ghiaccio che avevo dentro, appena un
poco. Stavo ricordando una notte di alcune settimane prima, quando
l'avevo baciata sotto un ponte che portava nel Distretto Bianco e lei
aveva ricambiato il bacio. «E quando le nuvole erano giuste le potevi
sentire come un tetto sopra la testa e vedere il sole riflettersi sul
mare grigio.»

«Vuoi tornare indietro?»

«Cosa?» Mi immobilizzai e ritrassi la mano, abbassando lo sguardo sul
panino che avevo sulle ginocchia e poi sul davanti liso della camicia
sotto cui era appeso l'anello. «No. Dio, no.»

«Ma lo fai apparire così bello.»

Non riuscivo a ricordare quando avessi serrato i pugni, ma mi ci volle
uno sforzo per rilassarli. «Lo era, però...»

«Non puoi... non ti può piacere \emph{questo}.» Cat agitò una mano in
direzione della strada. «Questa è merda, Had, tu non appartieni a questo
posto.» Una nota di fragilità nel suo tono mi indusse a sollevare lo
sguardo, e il modo in cui si passò con fare impacciato una mano sul
ginocchio del vestito logoro, là dove l'avevo toccata, mi spezzò il
cuore. «Tu dovresti essere lassù, sai?» Cat accennò al cielo con un
pollice. «Da come parli, il tuo posto è in un castello.»

«No!» ribattei, ricordando il sogno che avevo fatto della statua di mio
padre nella cripta, con i suoi occhi rossi come soli morenti. Non facevo
più quel sogno da anni, da quando mi ero svegliato su questo strano
mondo puzzolente.

«Comunque il tuo posto non è qui» dichiarò in tono piatto, poi mi
appoggiò la testa su una spalla, con i suoi capelli che mi
{solleticavano} la guancia. La cinsi con un braccio, come se con quel
gesto avessi potuto dimostrare che si sbagliava, ma non riuscii a
pensare a niente da dire. Aveva ragione. «Se potessi andare da qualsiasi
parte... ovunque nel grande Buio... dove andresti?»

Per poco non dissi `Teukros'. `Nov Senber.' Quella risposta però mi morì
in modo brusco e tagliente sulla lingua mentre mangiucchiavo le patate
arrosto sul fondo del mio vassoio del pranzo. «Hai mai sentito parlare
di Simeon il Rosso?» Cat scosse il capo e io mi costrinsi a non
esplodere con un `Cosa?', dicendo invece: «Era il mio più grande eroe.
Simeon era uno scoliasta. Non un visir di corte ma l'ufficiale
scientifico di una nave e parte dei Corpi di Spedizione. Questo
succedeva... oh, migliaia di anni fa, quando l'Impero era giovane. La
sua è stata una delle prime navi veloci che hanno raggiunto Centaurus,
esplorando mondi che potessero diventare future colonie. Perlopiù, hanno
trovato liberi proprietari, barbari che preferivano la frontiera alla
civiltà, come i Normanni odierni, e li hanno lasciati perdere, oppure
hanno commerciato con loro o li hanno sottomessi in nome
dell'imperatore, spostandosi sempre più all'esterno nel Buio. Là hanno
trovato uno strano, nuovo mondo. Un posto aspro e gelido dominato da
uccelli giganteschi e da una razza di xenobiti volanti.»

«Com'erano?» chiese Cat. «Gli xenobiti, intendo.»

«Anche loro erano come uccelli, un po' più piccoli di noi ma con grandi
ali al posto delle braccia e un corto becco.»

«E avevano gli artigli?» Appoggiò il suo peso contro di me.

«Oh, sì» confermai. «E anche artigli alle estremità delle ali, che
usavano per brandire sciabole lunghe quanto io sono alto!» Sollevai una
mano per indicare le dimensioni di quelle armi. «Tor Simeon guidò alcune
missioni sulla superficie del pianeta, commerciando con i nativi, che si
facevano chiamare Irchtani. Imparò a parlare con loro e a comprendere il
linguaggio dei segni che utilizzavano, e tutto andava bene fra la sua
gente e la loro.

«Però secondo l'equipaggio la nave si era spinta troppo lontano. Gli
uomini erano stufi e sentivano la mancanza di donne umane e di casa.
Così mentre Simeon era a terra con la sua scorta e la sua squadra
scientifica l'equipaggio si è ammutinato e ha ucciso il capitano e gli
altri ufficiali con l'intenzione di portare la nave nelle libere
proprietà e di vivere come pirati. Però hanno commesso un errore.»

«Si sono dimenticati di Simeon?»

«Oh, non si sono dimenticati di lui, lo volevano con loro. Vedi, lui
sapeva parlare l'irchtani e poteva persuadere il loro capi a vendere i
loro nemici agli ammutinati come schiavi. Nella galassia nessuno a parte
loro aveva mai visto prima gli Irchtani... immagina che prezzo avrebbero
ottenuto per uno di loro! Da un ateneo per le ricerche, oppure nel
Colosso, o vendendoli a qualche lord di sangue antico. Volevano
arricchirsi, capisci, e pensavano che Simeon li avrebbe aiutati. No,
l'errore che hanno commesso è stato quello di pensare di poterlo
comprare. Era impossibile perché Simeon era uno scoliasta e aveva
rinunciato a ogni ricchezza quando aveva indossato la veste verde del
suo ordine. Cosa ancora peggiore, aveva stretto amicizia con i principi
degli Irchtani e si unì a loro per combattere contro gli ammutinati
quando vennero a cercarlo con le loro navette. Simeon non era mai stato
un soldato, ma guidò lo stesso gli xenobiti contro i traditori umani e
contribuì a organizzare la loro ritirata fino al tempio di Athten Var,
il luogo più sacro del popolo irchtani.»

«Com'era?»

Agitai una mano. «Dicevano che i loro dèi lo avevano costruito tutto di
pietra nera sulla montagna più alta del mondo e che poteva essere
difeso. È stato là che hanno organizzato la loro resistenza. Simeon e
gli Irchtani hanno vinto, gli ammutinati sono stati gettati giù dalla
cima della montagna e Simeon ha ripreso possesso della sua nave. Come
dono, gli Irchtani gli hanno dato un grande mantello come quello
indossato dai loro principi. `È come la tua veste!' disse il Principe
degli Uccelli, ma non lo era. Gli Irchtani non vedono i colori come
fanno gli uomini, e il mantello che avevano approntato era rosso, non
verde. Da allora Simeon è stato chiamato `il Rosso' per via del mantello
degli Irchtani, e lui ha battezzato il loro pianeta Giudecca per il
tradimento che vi aveva subìto. L'Impero gli ha dato il titolo di
capitano e un nuovo equipaggio, che si è spinto più lontano nel Braccio
di Centaurus, portando su molti altri mondi la luce del Sole Imperiale.»

«E cosa è successo agli xenobiti quando è arrivato l'Impero?» chiese
Cat. «Adesso sono scomparsi?»

«Sono ancora là» replicai. «Non ricordo più quale Casato governi
Giudecca adesso. Calbren, forse? O Brannigan? E ci sono forze ausiliarie
irchtani fra le Legioni imperiali che combattono contro i Cielcin. Una
volta ho sentito dire che l'imperatore stava prendendo in considerazione
la possibilità di dare la cittadinanza a quelli che prestavano servizio
per un periodo di vent'anni.»

«Davvero?»

«Davvero!» confermai, stringendole un braccio. Mentre parlavo aveva
finito di mangiare e rubò qualcuna delle mie patate nonostante le mie
deboli proteste. Rendendomi conto che non avevo risposto alla sua
domanda originale, la schivai chiedendo: «E cosa mi dici di te? Dove
andresti?»

Cat scrollò le spalle. «Non andrò mai da nessuna parte. Da nessuna se
non qui.»

«Neppure io» commentai, pizzicando la mia camicia lisa.

«Tu però lo \emph{potresti} fare» obiettò Cat, raddrizzandosi per
guardarmi con quei suoi occhi luminosi. «Potresti mostrare quel tuo
anello a un qualsiasi vecchio mercante che dovrebbe fare qualsiasi cosa
gli dicessi. E potresti portarmi con te!» Sorrise con dolcezza,
mostrando quei denti storti. Sul mio volto magro dovette apparire una
qualche espressione, perché la sua si fece avvilita. «So che non lo puoi
fare.»

Rimanemmo in silenzio, e dopo un momento finii il resto del mio panino,
passando a lei le patate rimaste. Per un po' restammo lì seduti a
guardare un gruppo di persone ben vestite della nostra stessa età che
passavano in fretta, ridendo e scherzando, indifferenti e senza
preoccupazioni. «Andrei su Luin» affermò Cat, dopo parecchio tempo. «Mia
madre diceva sempre che là nei boschi ci sono le fate e alberi d'argento
più alti di quello.» Indicò il Castello Borosevo, con le sue torri di
arenaria e le mura che erano una massa artistica sulla cima delle
terrazze di cemento dello ziggurat. «Diceva che le fate guidavano le
persone attraverso la foresta e fino a polle magiche che nessuno aveva
mai visto prima.» Mi sorrise e io ricambiai il sorriso, anche se dentro
di me ero consapevole che le phasma vigrandi di Luin in realtà
attiravano gli insetti più vicino agli alberi carnivori che rendevano
tanto bello il loro mondo. Non c'erano fate.

«Io voglio vedere degli xenobiti» replicai in un sussurro, appoggiandomi
sui gomiti per guardare quel cielo sgargiante fra il verde e
l'arancione. D'un tratto ricordai quel marinaio, il Corvo, e questo mi
indusse ad aggiungere: «Vorrei avere una mia nave in modo da poter
viaggiare. In quel modo mio padre e la Cappellania non mi troverebbero
mai e potrei andare su Jadd, su Durannos, su Lothriad e anche nelle
Proprietà. Voglio vedere \emph{tutto}.» Passò un bel po' di tempo prima
che mi accorgessi che stavo ancora parlando. Dovevo aver farfugliato per
almeno cinque minuti, forse anche dieci. Era bello sognare di nuovo ad
alta voce, mettere a nudo i desideri del mio cuore.

Non mi ero reso conto che quello che avevo detto era vero finché non
avevo parlato. Volevo una nave, la libertà delle stelle. Se non potevo
essere uno scoliasta e apprendere i segreti scritti e conservati, avrei
cercato le verità dove vivevano. Se solo avessi potuto trovare i soldi!
Fra Cat e me avevamo meno di un kaspum d'argento, che non era
sufficiente neppure per pagarmi un paio di scarpe per camminare e
tantomeno ali per volare. Cat non arrestò il mio farfugliare, non mi
interruppe mai, il che era decisamente strano. Mi stava osservando con
attenzione come se fosse stata incerta su cosa dire o su come dirlo. Poi
una piccola ruga le si formò sulla fronte e si morse un labbro.

«Cosa c'è?» chiesi, pungolandole una spalla. «Non è da te farti così
rigida.»

«Vuoi davvero andartene, vero?» replicò, con voce sottile come la carta.

Sbattei le palpebre, guardando intorno a me il canale afoso e i rifiuti
sparsi lungo la strada. «Ecco... sì.» Agitai una mano. «Devo solo
trovare il modo di lasciare in qualche maniera questo mondo. Qualcuno
che ci porti via in segreto, tu e io. Devo solo trovare il denaro.»

Cat scosse il capo. «Non posso andarmene. Sono vincolata al pianeta.» Si
era fatta molto silenziosa, cosa che mi indusse a pormi degli
interrogativi.

«Non intendo lasciarti qui!» mormorai, urtandola con la spalla. «Credi
davvero che ti abbandonerei? Troveremo una soluzione, e allora vedremo
Luin e incontreremo gli Irchtani. Ti comprerò perfino un vestito che non
sia tutto strappato.»

«Non devi lasciare Emesh per vedere gli xenobiti.» Abbassò lo sguardo.
«Puoi rimanere qui.»

«Di cosa stai parlando?» domandai, raddrizzandomi sulla persona.

Fece una smorfia. «Non sai un bel niente su di noi, vero?» A quel punto
si alzò in piedi, lasciando i resti del suo pasto sul bordo del
marciapiede. «Abbiamo xenobiti su Emesh.»

