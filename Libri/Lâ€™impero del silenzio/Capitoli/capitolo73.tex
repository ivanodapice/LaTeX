\chapter{Diecimila occhi}

Nelle settimane che seguirono fui testimone di dozzine di interrogatori
separati, ma con mio sollievo furono tutti soltanto questo:
interrogatori. Uvatari veniva tenuto in una cella privata, isolato come
lo era stato Makisomn in un posto appena al di sopra del livello del
mare, alla base della bastiglia della Cappellania. Anche gli altri erano
isolati allo stesso modo, tenuti lontani uno dall'altro per far crescere
la verità dalle menzogne in giardini separati. Quelli la cui versione
differiva dalla maggioranza furono tenuti d'occhio, le loro parole
annotate e confrontate con quelle degli altri e del torturato Uvanari.
Io presenziai a ogni sessione. Lord Mataro rifiutava di vedermi, e così
pure lady Kalima e la tribuno-cavaliere Smythe.

Da tutto questo concludemmo sorprendentemente poco. Tanaran, che
sembrava una sorta di impiegato o l'equivalente di un logoteta di rango
minore, parlò per tutto il tempo con decisa sincerità, e dalle sue
parole e da quelle di quanti confermarono la sua versione scoprimmo che
avevo avuto ragione nel fidarmi delle affermazioni di Uvatari. I Cielcin
non stavano attaccando Emesh, o quantomeno non lo avevano attaccato. I
gruppi di analisti forensi orbitali posti sotto la direzione congiunta
del Casato Mataro e della 437° Legione convalidarono questa rivelazione
con soddisfazione della zelante Agari. La loro non era una forza
d'invasione, la si poteva a stento considerare militare.

A nessuno pareva importare quanto io soffrissi per tutto questo.

Sedevo da solo nell'appartamento assegnatomi, lo stesso che avevo
occupato prima del mio volo a Calagah. Il conte aveva {incaricato} un
paio di opliti di montare la guardia nel corridoio, e sospettavo che lo
avesse fatto tanto per tenermi d'occhio quanto per proteggermi, quindi
non protestai per la cosa e non cercai neppure di lasciare le mie camere
tranne quando la Cappellania mi mandava a chiamare. Anaïs venne a
trovarmi e mi lasciò messaggi sulla console di comunicazione della
stanza. A volte portava con sé suo fratello nella speranza di
persuadermi a fare qualche gioco o a indulgere in futili distrazioni.
Feci tutto il possibile per mantenere le distanze e con mia sorpresa lei
parve recepire il messaggio. Soprattutto, parve capire. Forse ero troppo
crudele. Lei e Dorian erano benintenzionati.

Così non rimasi sorpreso quando mi bussarono alla porta, il quindicesimo
giorno dall'inizio degli interrogatori. Chiunque fosse, aveva avuto
l'autorizzazione delle guardie, quindi aprii senza esitare. «Cosa c'è?
Cosa vu...» Ammutolii.

Sulla porta c'era Valka Onderra. Si era tagliata i capelli, rimuovendo
la coda che portava di preferenza e accorciando di molto i lati e la
nuca pur lasciando ciocche scure che le pendevano intorno e sulla
faccia, oscurando un occhio. Il rosso presente in quei capelli
risplendeva mentre se ne stava ferma là, tamburellando con una mano sul
tablet agganciato su un fianco per controllare gli Umandh. D'un tratto
consapevole dell'oscurità della mia camera agitai una mano davanti al
sensore, alzando il livello di illuminazione e mettendo in rilievo i
suoi toni cupi e i dipinti a olio dai toni ancora più cupi che
raffiguravano lo spazio.

«Valka!» esclamai, cercando di mostrarmi vivace. «Non sapevo che fossi
tornata.»

Sorrise, un piccolo sorriso triste -- doveva aver saputo quello che
stava succedendo -- ma non parve irritata. «Sono arrivata solo questo
pomeriggio.» Quasi mi aspettavo che mi colpisse, ma non lo fece. «La
Forza Difensiva ha chiuso Calagah per questa stagione e non ci ha
permesso fino a ieri di lasciare Fonteprofonda.»

«Ed Elomas?»

«Sta cercando di trattare con l'\foreignlanguage{italian}{fd} perché gli
permetta di tornare a sgomberare l'accampamento. Credo abbia lasciato là
il suo vino.»

«Non il vino!» Feci del mio meglio per fingermi inorridito, ma non
riuscii a metterci il cuore. Traendomi di lato agitai una mano. «Vuoi...
vuoi entrare?»

Lei si soffermò sulla soglia, studiandomi, con una mano che continuava a
tastare il tablet per le comunicazioni. Ero lieto di vedere che le
avevano permesso di continuare a portarne uno a Borosevo nonostante gli
eventi di quel giorno nel magazzino per l'imballaggio del pesce. Una
parte di me -- una piccola e stupida -- aveva bisogno di reclamare un
po' del merito per quella minuscola vittoria e mi illuminai nel vederlo.
Adesso so che la cosa non aveva niente a che vedere con me. Alla luce di
tutto il resto, volevo solo credere di essere stato di aiuto.

«Hai un aspetto orribile» disse infine.

Ne ero certo, perché neppure dieci minuti prima avevo visto la mia
faccia nello specchio del bagno: severa e angolosa, aveva acquisito un
che di cereo e di infossato. Gli occhi viola erano velati e segnati, i
capelli nerissimi che un tempo erano pettinati con cura mi ricadevano in
un groviglio ricciuto fin sotto il mento e probabilmente avevo un
cattivo odore. Il tempo in cui avevo vissuto come un senzatetto lungo i
canali sottostante lo ziggurat del castello aveva infranto il mio
bisogno automatico di una doccia, e dopo quello che avevo visto... ecco,
mi dimenticavo di mangiare, e ancor più di fare qualsiasi altra cosa.

Valka però entrò senza altri commenti e mi permise di chiudere la porta.
Consapevole delle videocamere, mi accasciai su una poltrona vicina alla
piastra olografica e al tavolino. Lei esaminò la stanza, notando il suo
noncurante disordine, le attrezzature da disegno sparse sul tavolino, la
giacca spiegazzata buttata su una sedia, i bicchieri di vino vuoti e le
tazze piene a metà d'acqua su scaffali, piani e tavoli. Si diresse alla
finestra e con un gesto secco tirò indietro le tende con un motivo
cachemire in modo da lasciar entrare la luce del sole. «Dovresti dare
una ripulita» osservò, soffermandosi nella cornice della finestra, una
scura sagoma femminile sullo sfondo del vetro e della città sottostante.
Sospirai.

«Lo faranno i servitori» ribattei, ma cominciai a prelevare le matite
dal tavolo e dall'ampia superficie aperta del mio diario, la cui pagina
esposta recava l'immagine di una mano aliena con sei dita dotate di
artigli, ciascun dito con un'articolazione in più e ogni articolazione
troppo lunga per essere umana. Solo tre delle sei dita avevano
l'artiglio e uno era stato accorciato all'altezza della nocca. La pelle
all'altezza del polso era scorticata in una fascia larga due pollici che
esponeva il muscolo. A una prima occhiata sembrava un'illustrazione di
qualche antico e fantasioso testo di anatomia e sarebbe potuta passare
per questo se non fosse stato per il dolore, che era facile da ritrarre
e ancora più facile da provare.

«Quando è stata l'ultima volta che sei uscito?» chiese Valka, poi
ripeté: «Hai un aspetto orribile.»

«Non ce n'è stato il tempo!» Quelle parole mi uscirono in tono più duro
di quanto fosse stata mia intenzione, fragili e vitree. «Sono stato...
non voglio parlare di dove sono stato.»

Valka si sedette sul bracciolo del divano, di fronte a me, illuminata
lateralmente dalla finestra. Il suo volto adorabile risplendeva in
nitido rilievo, con gli occhi dorati che scintillavano anche se uno di
essi era in ombra. La musicalità della sua voce si incrinò. «Ho... ho
saputo. Me lo ha detto Anaïs.»

«L'hai vista?» Mi presi la testa fra le mani e abbassai lo sguardo sullo
spesso tappeto dal disegno a mandala nei toni del bianco e del nero,
realizzato a imitazione della moda tavrosiana popolare nel Velo e sui
mondi coloniali. Cercai di non pensare ad Anaïs, a quello che era.

«È preoccupata per te. Dice che la Cappellania ti sta costringendo a
tradurre mentre loro \emph{interrogano} i Cielcin.» Annuii e mi guardai
intorno in cerca di un bicchiere abbandonato che potesse ancora
contenere un po' di vino, ma non se ne vedevano. «Lei però dice anche
che sei tu il motivo per cui li hanno presi vivi. Che sei un eroe.»

L'asciutto divertimento che Valka si costrinse a imprimere alle sue
parole fu più di quanto la mia mente esausta e sconvolta dalle urla
potesse tollerare. «Lei non sa di cosa parla. Pensa che sia tutto una
grande avventura, ma non lo è.» La mia voce salì di tono a ogni singola
affermazione e le mie dita si incurvarono come artigli. Alla fine stavo
quasi urlando, con gli occhi fiammeggianti.

Lei imprecò nella sua lingua natale, poi sibilò: «Imperiali!» Era la sua
imprecazione preferita. In qualche modo il suo disprezzo mi fu di
conforto, la consapevolezza che qualcun altro nell'universo condivideva
la mia furia attuale mi fu di aiuto, come una luce che si protendesse
verso un'altra luce attraverso l'oscura materia della civiltà. «Non
possono usare un autotrad come fa il resto dell'universo civilizzato?»
Si spinse i capelli dietro le orecchie, una cosa che avrei voluto fare
io per lei.

«Un cosa? Oh.» Un'intelligenza meccanica, uno di quei congegni schiavi
che i Tavrosiani consideravano comuni e che la Cappellania vedeva come
peccaminosi. «Sai che non possono.»

I suoi strani occhi fiammeggiarono. Dopo un momento di amichevole
silenzio, mi chiese: «Come sono?»

«I Cielcin?» Distolsi lo sguardo, volgendolo fuori dalla finestra, sopra
i tetti rossi e quelli di metallo, fino alla vasta distesa del campo di
atterraggio, cercando di nuovo di trovare quel posto nel labirinto dove
un altro, più giovane Hadrian era stato scaricato nudo a morire. Non ero
mai tornato, non avevo mai ripagato il debito che avevo con la vecchia
dell'ospedale, non mi ero mai vendicato dei portuali che mi avevano
derubato. E non avevo mai neppure appurato cosa ne fosse stato di
Demetri e del suo equipaggio. Come aveva fatto tutto a svilupparsi in
questo modo? Per la Terra, Dio e l'imperatore, avrei dovuto essere su
Teukros, a Nov Senber, ormai avrei dovuto essere tor Hadrian; avrei
dovuto esserlo adesso, vestito tutto di verde. Però c'erano altri poteri
che muovevano il nostro mondo, poteri più grandi dell'uomo e che come il
tempo e la marea non aspettavano nessuno. Perfino gli imperatori, come
luce stellare, si piegano alle forze più oscure della legge naturale.
Attesi per lo spazio di parecchi respiri prima di continuare: «Loro...
loro sono come noi, ma non quanto ho creduto inizialmente.»

A quel punto le raccontai tutto: la discesa nelle gallerie, la
situazione di stallo nel sepolcro della Quiete, le dissi di Uvanari e di
Tanaran, omettendo solo il mio spiacevole ruolo nell'interrogatorio del
Cielcin nelle gallerie. È facile affermare soltanto che non volevo
rovinare l'opinione in via di miglioramento che Valka aveva sul mio
conto, ma la verità è che non pensavo di poter condividere in quel
momento quel particolare episodio. Il solo pensiero mi faceva stare
male.

«Però siamo al sicuro.» Mi sfregai gli occhi con le nocche e mi premetti
contro i cuscini della poltrona. «Quantomeno, credo che lo siamo. Non
pare che i Cielcin fossero una forza d'invasione e perfino la
Cappellania ha cominciato a crederci.» Sfoggiai il più pallido dei miei
sorrisi.

Valka si protese leggermente in avanti con espressione intensa. «E non
hanno cercato di...» Si passò una mano di taglio sulla gola.

«Di uccidermi? No, no, al momento hanno bisogno di me.» Mi alzai e
recuperai qualche bicchiere sparso dai piani fra la mia poltrona e
l'area del cucinotto. Riempiendone uno d'acqua dall'unità filtrante lo
svuotai e tornai a riempirlo. «Ne vuoi? Potrei aver finito il vino.»
Valka rifiutò sollevando una mano e io mi appoggiai pesantemente contro
il piano di granito. «Sto cercando di convincere la tribuno-cavaliere a
usare i prigionieri come leva per aprire trattative di pace con i
Cielcin, ma nessuno mi vuole ascoltare e il conte non presenzia neppure
più alle riunioni del consiglio. Credo voglia solo che finisca tutto.»
\emph{E credo che i Cielcin adorino la Quiete}. Avrei voluto dirlo
apertamente, ma un plumbeo sfinimento e quella nauseante, formicolante
sensazione di essere osservato che era svanita a Calagah mi facevano
accapponare la pelle. Adocchiai la videocamera più in vista della
camera, una minuscola lente nera inserita nel pannello di controllo
della luce e del climatizzatore della stanza. Uno dei diecimila occhi
che non dubitavo essere collegati in rete e che inoltravano le
informazioni alla postazione di monitoraggio del servizio di sicurezza
del Casato, dove venivano quantomeno catalogate se non esaminate da una
polizia voyeuristica.

«Puoi biasimarlo?» chiese Valka, con le dita adorne di anelli che
tracciavano le linee nere sul dorso esposto dell'avambraccio sinistro.

«Per gli dèi, no!» Feci ruotare la tazza come una trottola, con un
acciottolio di vetro pesante, poi le chiusi intorno la mano per impedire
che la rotazione diventasse incontrollata. «Però dobbiamo essere
migliori, non \emph{anaryoch}.»

Lei sbuffò, ma le linee marcate del suo volto tradivano una risata. «La
tua pronuncia... è orribile, sai.»

«Ne sono certo.» Feci ruotare di nuovo il vetro. Quello che stavo
dicendo probabilmente era abbastanza sicuro, ma il vasto numero di altre
cose che volevo dire ed ero costretto a tacere mi stava rendendo
nervoso. «Vorrei poterti dire di più, Valka, davvero.» Non si sapeva mai
chi fosse in ascolto. Ufficialmente, le registrazioni servivano solo
alla sicurezza del castello, ma chi pensava che la Cappellania non vi
potesse accedere in un modo o nell'altro era uno stolto.

Valka chinò il capo, un po' troppo abbattuta. «Lo capisco. Questioni
imperiali.» Per chi la conosceva, quell'improvviso arrendersi da cane
bastonato era come uno schiaffo. Per poco non lasciai {cadere} la tazza,
ma all'ultimo istante riuscii ad afferrarla, ancora sorpreso. La
mascella di Valka si irrigidì per un momento, i muscoli delle tempie si
flessero e i suoi occhi si chiusero. Poi le luci tremolarono fino a
spegnersi, lasciandoci al buio: un altro dei frequenti cali di tensione
del castello. Quasi mi disperai per il timore che il climatizzatore si
rompesse e ci lasciasse a sudare nell'oscurità.

Lei intanto sollevò lo sguardo con le sopracciglia alzate, di nuovo sé
stessa. «Mi dimentico della presenza di quegli stupidi arnesi.»

«Cosa?»

«Le videocamere. Passeranno alcuni minuti prima che le rimettano
online.»

\emph{Le videocamere.} «Ah.» Impiegai un momento a elaborare fino in
fondo le sue parole, poi realizzai, ricordai e sgranai gli occhi.
Diecimila occhi si erano chiusi, eravamo soli. «Come hai...» Il mio
sguardo si spostò sul pannello di controllo della stanza: la minuscola
luce rossa accanto alla videocamera era scomparsa. «Sei sicura?»

«Assolutamente.» Si batté un colpetto sulla tempia. «Di cosa vogliamo
palare? Omicidio? Tradimento?» Sentii il sangue che mi defluiva dal
volto e mi mossi barcollando. Valka rise, indicandomi. «Vedessi la tua
faccia!» Si asciugò un occhio con un dito e si batté un altro colpetto
sulla tempia. «Questa me la tengo da parte, Marlowe.»

Io non stavo ridendo e risposi a denti stretti: «Sei pazza?»

«Quello che si comporta da pazzo sei tu» ribatté con il suo sorriso più
tagliente. «Nessuno mi ha sentita.»

Mi rimisi a sedere lentamente, certo che da un momento all'altro le
guardie avrebbero fatto esplodere verso l'interno la porta di legno, con
i fucili fiammeggianti. «Ma come hai fatto?» chiesi, quando non successe
niente. «Gli sbalzi di corrente... sei tu a causarli?»

Si batté la tempia per la terza volta. «Una stringa neurale.» Aveva già
usato quel termine in passato, durante il nostro primo incontro.

«Non so cosa significa.»

Lei sibilò qualcosa di incomprensibile nella sua lingua natale, poi
aggiunse: «Continuo a dimenticarlo. Qui.» Mi volse le spalle e si tastò
con le lunghe dita la base della nuca, allargando i capelli rossi.
«Tasta in questo punto.»

«Cosa? Io...»

«Fallo e basta.» Obbedii, premendo le dita alla base del suo cranio,
dove c'era qualcosa, un grumo non più grosso di un pisello. Fra i
capelli potevo a stento scorgere il bagliore di qualcosa di bianco come
la porcellana sullo sfondo della pelle pallida. Non sapevo bene cosa
fosse e neppure cosa dire, quindi rimasi lì in piedi, confuso e in
silenzio, con la mano ancora fra i suoi capelli. «Su Tavros ce lo
applicano da bambini, se i nostri genitori hanno il credito sociale
necessario. Adesso puoi togliermi la mano dai capelli.»

La ritrassi come se avessi preso la scossa. «Sì, scusami» balbettai,
voltandole le spalle. «Ma... cos'è?» Mentre ancora formulavo la domanda,
frammenti del mio bagaglio culturale ereditario mi sussurrarono
all'orecchio. \emph{Demoniaco. Strega. Maga}. Era una macchina, su
questo non c'erano dubbi. La perversione del corpo mediante le macchine.
Aveva commesso uno dei Dodici Abomini, uno degli arcipeccati per cui la
Cappellania avrebbe giustiziato \emph{chiunque} senza processo o
ripensamenti. Mi ritrassi interiormente, sentendo un compulsivo bisogno
religioso di purificarmi, ma sapevo una cosa: Valka non stava scherzando
sul fatto che le videocamere erano fuori uso. Non sapevo come avesse
fatto, ma non mi avrebbe rivelato una cosa del genere senza l'assoluta
certezza di essere al sicuro. Ho paura, dissi a me stesso. Ho paura di
Valka.

La paura però è un veleno, e qualsiasi cosa fosse, ero meglio di così.
«Puoi semplicemente... farlo?» Agitai una mano.

Il suo comportamento divertito crollò come un'onda. «In questo non c'è
niente di \emph{semplice}.» Fece una smorfia. «È costoso.» Accavallò le
gambe mentre l'increspatura del sorriso svaniva dal suo volto. «Però ha
i suoi vantaggi.»

Con lo stesso fascino che uno studente che conduce una vita protetta
prova per il macabro, mi protesi in avanti. «Per esempio?»

Quegli occhi dorati fiammeggiarono. «Non ora. Cosa stavi dicendo?»

«Mmm?» Ero solo un'ombra del mio io migliore, appiattito e disperso ad
angoli distorti attraverso l'universo delle cose più vere, deformato
dalla gravità relativistica degli eventi. «Oh! La Quiete!» Quello che
avevo scoperto era stato più o meno cancellato completamente dal mio
cervello, lavato via dalla diabolica manifestazione di potere di quella
donna, ma a quel punto le dissi tutto. Ripetei la confessione di Uvanari
secondo cui la spedizione dei Cielcin era venuta su Emesh non per
combattere ma per pregare nelle camere di Calagah. Era per questo che
avevano diretto la loro nave danneggiata verso le rovine -- fra tutti i
posti del globo -- in modo da poter morire nelle sale dei loro dèi
morti.

«Perché non sono tornati indietro quando la loro nave esplorativa è
stata attaccata?» chiese Valka, assestandosi il gilè color terracotta.

«Credo che stessero cercando qualcosa» spiegai. `Loro non sono qui.'
Valka inarcò le sopracciglia e io ripetei le parole dell'\emph{ichakta}
nella sua lingua natale. «\emph{Rakasuryu ti-saem gi}.» È questo che ha
detto. Potevano essere alla ricerca di qualsiasi cosa. I Cielcin hanno
una cultura basata sul saccheggio, è per questo che combattono contro di
noi, per le risorse. Prima che arrivassimo, dovevano agire da parassiti
delle loro stesse flotte. Ha senso che portino via qualsiasi tecnologia
della Quiete che riescano a trovare.»

Lei scosse il capo. «Quale tecnologia della Quiete? Hai visto Calagah,
Hadrian, è solo pietra.»

«Forse su Emesh! Ma questo non può essere vero per tutte le rovine di
questa parte della galassia.» Non accennai alla mia visione. Non
intendevo parlarne, non di nuovo.

«Sono stata su Sadal Suud, su Rubicon. Non c'è niente. Non sappiamo
neppure quanto fossero evoluti perché non hanno lasciato nulla.»

La passione lottò contro l'ombra appiattita della depressione e mi
sedetti più eretto, allontanandomi i capelli lunghi dalla faccia.
«Niente che comprendiamo, ma i Cielcin... Valka, loro si sono evoluti
circondati da tutto questo. Devono sapere qualcosa che noi ignoriamo.»

Rimase in silenzio per un momento, con il bel viso rivolto verso il
basso. Infine annuì. «D'accordo, cosa...» Bussarono alla porta. Mi
immobilizzai. Anche Valka si paralizzò, poi sgranò lentamente gli occhi.
«Aspetti qualcuno?»

Tutto quello che riuscii a fare fu scuotere la testa, e dovetti
deglutire un paio di volte prima di ritrovare la voce. «Sei certa di
aver bloccato le videocamere con il tuo... quello che è?» Parve ferita,
ma non persi tempo con un ego ammaccato. «Farai meglio...» Mi toccai la
nuca a indicare la macchina demoniaca annidata alla base del suo cranio.
«Parleremo più tardi.»

«Sarà meglio.»

Come per Valka, sapevo che chiunque ci fosse alla porta era stato
autorizzato dai due opliti di Mataro che se ne stavano nel corridoio
come armature in mostra, quindi aprii senza timore. «Sir Olorin!» Feci
fatica a mascherare la sorpresa nella mia voce. «A cosa devo
quest'onore?»

«Nessun onore!» replicò con un gesto gioviale, agitando verso di me una
bottiglia scura. «È un piacere! Mi hanno detto che ti avrei trovato qui,
che dovevo solo guardare dove c'erano delle guardie.» Diresse un sorriso
di un'ampiezza assurda verso i due opliti mascherati dal casco che se ne
stavano sull'attenti ai lati della mia porta, con un appuntito disegno a
croce sui piani convessi del volto che indubbiamente nascondeva una
profusione di apparecchiature sensorie. Il maeskolos era ancora vestito
di nero come sua abitudine, ma il tessuto era seta e non cuoio. Aveva
anche accantonato l'ampia mandayas carminia e la sua camicia era aperta
fino allo sterno, esponendo un petto bronzeo e un medaglione dal taglio
squadrato che mi ricordò quello di Demetri, solo che su questo era
impresso un singolo cerchio ininterrotto. Mi soffermai per un momento a
interrogarmi sulla cosa, facendo senza dubbio la figura di qualcuno che
era stanco o stupido. O di uno stanco stupido.

«Posso entrare?» chiese infine lui.

Sbattei le palpebre. «Io...» Mi trassi da un lato, con i piedi che per
poco non inciampavano uno sull'altro. «Sì, sì, certamente. Prego.»
Oltrepassò la soglia e io richiusi la porta dopo aver rivolto alle
guardie un sorriso non ricambiato. Ricordando le parole del maestro di
spada, feci un altro tentativo. «Allora, a cosa devo questo... piacere?»

Sir Olorin Milta ruotò agilmente sui tacchi con gli stivali al ginocchio
che scricchiolavano sul legno massiccio e l'impugnatura delle tre spade
che oscillavano liberamente, facendole urtare fra loro come un
segnavento. «Mi è stato dato di capire che le ultime settimane sono
state... alquanto... difficoltose. Ho pensato che ti servisse qualcosa
da bere.»

I miei riflessi di politico, innati e alquanto affinati da quando ero a
Borosevo, attivarono la mia reazione combatti-o-fuggi, e una domanda
prese a risuonarmi come un campanello d'allarme nelle orecchie: Cosa
vuole? Cosa vuole? Però sorrisi. «Anche se non fossero tempi difficili
farei fatica a rifiutare. Lascia però che ti presenti la mia...» La mia
cosa? Amica? Collega? Musa? Valka non si era mossa dal divano, ignara o
incurante della posizione altolocata del maestro di spada, e questo mi
sorprese... potevo anche essere un compassato palatino imperiale, ma i
maeskoloi erano materiale da leggenda perfino nel lontano, stregato
mondo di Valka. «La mia... lei è Valka Onderra. La \emph{dottoressa}
Valka Onderra del clan Onderra di Vhad Edda.»

«Una xenologa» aggiunse Valka, con un accento d'un tratto più marcato,
mentre si alzava e porgeva la mano all'effemminato spadaccino jaddiano.
Nei suoi occhi c'era un'espressione non dissimile da quella di una buona
forchetta che si trovasse di fronte a un taglio di carne particolarmente
buono. Lo jaddiano accettò la mano che gli offriva e -- con mio privato
sollievo -- non la baciò.

Olorin sfoggiò quindi di nuovo quel suo sorriso tutto denti, tanto
bianchi da farmi pensare che fossero sintetici. «Sono sir Olorin Milta.
Deliziato di conoscerti. Io...» Si girò a guardarmi. «Ho scelto un
brutto momento, Marlowe?»

«Ah...» Esitai, sul punto di rispondere con un `temo di sì'.

Spostai lo sguardo su Valka, che stava ancora studiando lo spadaccino e
mi batté sul tempo. «Per nulla. Dovrei andare.»

«Devi proprio?» Stranamente, lo sguardo di Olorin si spostò rapido su di
me, poi parve rimuginare su un qualche pensiero personale prima di
tornare a concentrare in modo deliberato la sua attenzione su Valka.
«Per favore. La mia è una visita di cortesia, e ho portato lo zvanya.»
Sollevò la bottiglia di brandy di un pallido colore arancione perché lei
la esaminasse. «Qualsiasi amica del nostro amico traduttore è una mia
amica.»

\emph{Amico?} Adocchiai con cautela il maestro di spada, grato che la
sua attenzione fosse rivolta altrove. \emph{Quando era successo?} Di
certo non aveva protestato contro la cattura dei Cielcin con lo stesso
vigore del tenente Lin, ma non lo avrei definito un amico.

«Cos'è?» chiese Valka, chinandosi a esaminare l'etichetta.

«Zvanya!» ripeté Olorin, sfregandosi le labbra sottili e il lieve velo
di barba che cresceva sui duri piani del suo volto; era chiaro che gli
jaddiani non usavano nella pubertà il laser sui pori. «Non lo hai mai
assaggiato?» Valka rispose in senso negativo, cosa che indusse Olorin ad
aprire immediatamente la bottiglia. «\emph{Tavmasie}! Allora devi
rimanere! Per favore!» Dietro sua richiesta trovai tre bicchieri da
whisky puliti e attesi che lui vi versasse tre dosi di quel liquido
rosato.

Annusai il contenuto del bicchiere. «Dèi, se è forte.»

«Lo è quanto deve essere! \emph{Buon atanta}!» disse Olorin con gravità,
poi buttò giù il contenuto del suo bicchiere in un sorso.

L'una dopo l'altro, Valka e io lo imitammo e il sapore di cannella
grezza mi sopraffece, moderato un poco da quello di vino forte e da una
vaghissima traccia di arancio. Tutte quelle note di sapore, tuttavia,
erano annegate dal morso dell'alcol, limpido e violento come il fuoco.
Mi lacrimarono gli occhi, Valka tossì e Olorin rise. «\emph{Ehpa}!»

«Salute» risposi. «Tuttavia, maeskolos, perdonami, ma... non vorrei
sembrarti ingrato, però... ecco, mi sorprende alquanto vederti qui.»
Valka si era diretta in tutta fretta nel cucinotto, dove stava
riempiendo d'acqua il suo bicchiere. Pareva che il liquore jaddiano non
facesse per lei.

Con mosse feline, il maestro di spada sedette sulla poltrona che avevo
occupato un momento prima, scrutando con i suoi grandi occhi scuri il
mio diario aperto con aria di distaccato interesse. «Volevo parlare con
te da quando eravamo a Calagah.»

Valka si schiarì la gola. «Sei certo di non volere che me ne vada?» Con
mia sorpresa accennò con fare interrogativo in direzione della bottiglia
di zvanya, suscitando un gesto cortese da parte di Olorin. «Se voi
ragazzi dovete parlare sarò felice di andare altrove. Non voglio
disturbare.»

«No!» esclamai, troppo in fretta, suscitando un sorriso da parte dello
jaddiano. «Non disturbi, e comunque non stavo facendo niente.» \emph{E
	non abbiamo finito di parlare, Valka, ricordi?}

«Hai disegnato tu questo, lord Marlowe?» chiese il maestro di spada,
cambiando bruscamente argomento nel sollevare il mio diario dal tavolo.

Mi girai e guardai da sopra la sua spalla l'immagine del polso scuoiato
del Cielcin e delle sue dita spezzate. Olorin indugiò per un momento su
quell'immagine, sfiorando a stento gli artigli {anneriti} con il
carboncino. «Sì, lo sto facendo.» Scossi la testa per cercare di
schiarirmi la mente. «Non è finito.»

Tornò indietro di una pagina, rivelando lo schizzo di un mamelucco
jaddiano appoggiato a un falcione. Non ero più certo di averne visto uno
che portasse un'arma del genere, ma quella lancia con una lama al plasma
troppo lunga dava all'omuncolo dalle gambe sottili un'aria ancor più da
marionetta, con ginocchia nodose e tutta gomiti. Olorin continuò a
sfogliare all'indietro, oltrepassando in fretta una serie di ritratti --
di lui stesso, di lady Kalima, di sir Elomas, di Balian Mataro, di
Switch e di Pallino. Di mia madre. «Sono opere notevoli. Davvero...»

«Grazie» risposi, venendo avanti, in bilico fra la cortesia e il
desiderio che mi restituisse il libro. Mi fermai appena fuori portata
della poltrona, come se la mia vicinanza potesse indurre il maestro di
spada a smettere di sfogliare. Da qualche parte, verso il centro del
diario c'era un disegno di Valka che non volevo condividere. Non era
niente di volgare, ma non era destinato ad altri occhi che ai miei.

Olorin chiuse il diario con un colpo risonante e lo posò di nuovo sul
tavolo. «Un altro drink?» Valka, che nel frattempo aveva svuotato un
secondo bicchiere, gli porse la bottiglia e lui riempì il suo bicchiere
e il mio prima di restituirgliela. Una volta che quel secondo giro di
liquore fu scomparso, disse: «Volevo essere io a informarti perché non
volevo che la Cappellania ti prendesse di sorpresa.» Abbassò la voce a
un sussurro e chiuse le lunghe dita intorno al bicchiere di nuovo vuoto.
«Vengo ora da una riunione con la mia signora, il conte e la vostra
tribuno-cavaliere. Ritengono che non abbiamo fatto molti passi avanti
negli ultimi giorni e vogliono tentare... qualcos'altro.»

«Qualcos'altro? Non hanno tentato abbastanza?» Poi fui assalito da
un'altra domanda e lanciai una rapida occhiata a Valka, che rimaneva
impassibile vicino all'alto piano che separava il salotto dal cucinotto.
«Solo i nobili e la tribuno? Niente grande priora Vas?»

«No, la \emph{bruhir} della Cappellania non era presente.» Olorin passò
una lunga gamba sul bracciolo della poltrona e fece crocchiare le nocche
una alla volta con calma abilità. No, non si trattava di questo... era
un gesto che lo calmava, ogni articolazione che schioccava faceva
defluire la tensione come un aforisma di uno scoliasta. Stava chiamando
a raccolta il coraggio, preparandosi a dire ciò per cui era ovviamene
venuto nel mio umile alloggio. «Vogliono che parli con il capitano. Da
solo.»

«Cosa?» Per poco non lasciai cadere il bicchiere e dovetti abbassarmi
per afferrarlo. «Dici sul serio?» chiesi, quando mi raddrizzai. Era
quello che avevo voluto dall'inizio. Dèi. Quei dannati burocrati avevano
fatto tutto a rovescio.

Il maeskolos inclinò la testa e gli arruffati capelli neri gli ricaddero
sugli occhi. «Volevo essere io a dirtelo.» Un tenue sorriso divampò come
un lampo sul volto olivastro e scomparve. «Da amico. Riceverai la
convocazione domani.» Sempre là in piedi come uno stupido con il
bicchiere vuoto in mano, gli segnalai che avevo capito. Di colpo mi
ritrovai incapace di parlare. «Ovviamente monitoreranno la
conversazione, ma non sarà fatto del male alla creatura.»

Lo sguardo di Valka mi pendeva addosso come una catena e dovetti
resistere all'impulso di girarmi verso di lei. Volevo urlare,
accasciarmi sul pavimento di legno e colpirlo con le mani fino a
staccarmi la carne dalle ossa e a ridurre anche quella in schegge. Non
era finita. Non mi ero aspettato che lo fosse, ma quel desiderio c'era,
rovente e remoto dentro di me. Quello di cui avevo bisogno era di essere
di nuovo a casa, al sicuro nel mio letto sotto le costellazioni dipinte,
in una torre sul mare, e di passeggiare con Gibson sulle alte mura,
nella pace e nel silenzio.

Non avevo parole per Olorin, che intanto continuò: «Quello che hai fatto
in quelle gallerie. Tu hai fatto... di \emph{meglio.} Ci hai fatti agire
meglio.» Era la mia parola, che mi veniva rinfacciata dal passato. La
mia parola e la mia maledizione: \emph{meglio}. «Avevi ragione. In
passato non avevo mai sentito di Cielcin che si fossero arresi. Tu ce
l'hai fatta.» Fece scorrere lo sguardo sul disordine delle mie camere,
percependo forse l'apatia e la distimia che sottintendeva. Non c'era
nessun giudizio sul suo volto, nessuna compassione, come nel caso di
Valka. Non c'era niente, assolutamente niente. Questo non avrebbe dovuto
essere un conforto, e tuttavia lo era. «Questo non è... non intendo
giocare al prefetto buono e a quello cattivo.»

«Cosa?» Il maeskolos si mostrò confuso mentre Valka dovette soffocare un
sogghigno. «So quanto questo debba essere duro per te, ed è stato per
questo che ho voluto informarti di persona.»

«No! Dannazione!» esclamai quasi gridando e sorprendendo perfino me
stesso. «Non capisci. Come potresti? Tu non sei stato là dentro ogni
giorno, non hai dovuto sentirlo urlare, startene lì a ripetere tutto
quello che diceva lui e che diceva lei. Non. Hai. Dovuto. Essere. Là.»
Potevo sentire su di me i diecimila occhi della sorveglianza del
castello. «Io sì.»

«C'è un detto» interloquì la voce vivace di Valka, risplendendo
attraverso la nuvola che mi si era addensata intorno. «Da noi si dice
che la galassia è curva, e che se ti muovi abbastanza in fretta, vai
abbastanza lontano, ti ritrovi esattamente da dove eri partito.»

Qualcosa nel modo in cui lo disse -- o forse solo nel fatto che era
Valka a dirlo -- mi riportò con i piedi per terra. Le spalle mi si
accasciarono, poi si raddrizzarono. «Benissimo» affermai, con voce
simile a pietra scolpita dal vento. «Lo farò.»

