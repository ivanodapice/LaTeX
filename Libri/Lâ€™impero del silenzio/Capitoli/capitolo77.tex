\chapter{Una cosa rara}

«Dobbiamo cominciare immediatamente con un secondo prigioniero!»
dichiarò l'inquisitrice Agari, in ginocchio davanti ai suoi superiori.
«Marlowe ha identificato il loro capo politico.»

Due seggi erano stati disposti alla base della piattaforma del conte,
entrambi ad alto schienale e modellati interamente in legno dorato, e
ciascuno era affiancato da un littore. Lady Kalima sedeva sulla
sinistra, accompagnata da sir Olorin, e la tribuno-cavaliere Smythe
sedeva sulla destra con accanto il suo anziano primo ufficiale. Sir
William Crossflane. Smythe si protese in avanti. «Se lo facessimo,
inquisitrice, comprometteremmo i Cielcin rimasti. Siamo stati fortunati
a non perderne nessuno quando è scoppiata quella rissa. Possiamo far
passare la morte del loro capo per uno sfortunato incidente, ma se li
preleviamo uno alla volta cominceranno a notarlo.» Nel parlare guardò
verso di me. Mi chiesi se sapesse che avevo già compromesso quella parte
del copione nel tentativo di entrare nelle grazie del nemico.

Dal suo posto sull'alto seggio -- un insieme massiccio di coralli locali
fatti crescere in uno schema oro e verde che si diramava come un albero
-- il conte agitò una mano inanellata. «Inoltre, inquisitrice, non puoi
dirmi che credi davvero a quest'assurdità relativa a \emph{Vorgossos}.»
Agari aprì la bocca per ribattere, ma il conte continuò: «Se mai è
esistito, Vorgossos è morto da tempo. È solo una cosa nelle vecchie
storie.»

«Lo è anche la Terra» interloquì lady Kalima, con un'espressione
accuratamente altera e lo sguardo che evitava studiatamente quello di
Ligeia Vas, che aleggiava come uno spettro nell'ombra del trono di
Mataro. «Eppure è là fuori, da qualche parte.»

Con mio assoluto stupore Vas non abboccò all'esca. Invece la
tribuno-cavaliere mi sorprese. «Forse,» osservò in toni misurati «è meno
che caritatevole paragonare il Mondo Natale a un pianeta pirata da
fiaba.» Quell'affermazione mi colpì come alquanto conciliante da parte
dell'ufficiale abitualmente energico e mi agitai a disagio dietro
l'inquisitrice in ginocchio, con le mani intrecciate. Il fianco e il
braccio sinistro mi facevano male ed erano caldi dove i tamponi
correttivi lavoravano per guarire le mie ustioni da piombo. Attraverso
la manica armeggiai con quello appena al di sotto del gomito. La sala
del trono, con le sue alte finestre e le strisce di pannelli di vetro
inserite nella cupola era già abbastanza calda senza che altro calore
filtrasse dentro di me da sotto i vestiti. Almeno non avevo riportato
fratture. Il pensiero del supporto correttivo che avevo dovuto portare
dopo quell'aggressione di un migliaio di anni prima mi fece tremare e mi
serrai il braccio più forte che potevo, facendo affidamento sul dolore
perché mi distraesse.

Agari era ancora in ginocchio con la testa bassa. «Con il dovuto
rispetto, Vostra eccellenza, Vostra reverenza, signore, non riesco a
capire come lo xenobita morto avesse potuto sapere di Vorgossos se non è
reale.»

«L'avrà appreso da qualche prigioniero, ragazza» dichiarò Ligeia Vas, il
cui sguardo gelido era però fisso su lord Mataro, rivolgendo quelle
parole direttamente a lui. «Deve essere stato così.» Il conte annuì e
armeggiò con i drappeggi delle vesti di seta verdi e gialle, ma non
disse nulla. Era sempre stato così privo di spina dorsale? Oppure era
solo la questione dei Cielcin che lo aveva privato della sua forza? Lord
Luthor sedeva accanto al marito su un seggio più piccolo non dissimile
da quello della dama jaddiana e dell'ufficiale della Legione, e i suoi
occhi sottili si socchiusero ulteriormente mentre studiava tanto
l'inquisitrice quanto me. Quantomeno, i loro figli erano assenti. Non
avrei potuto affrontare Dorian e Anaïs, non con tutto quello che
succedeva.

«Un prigioniero che parlava la lingua dei Pallidi?» Il conte non pareva
convinto.

«Vostra reverenza, eccellenza» intervenni nel venire avanti, tenendo il
braccio ustionato ripiegato contro il petto, con la mano agganciata alla
spalla opposta. «Il Cielcin lo ha detto \emph{mentre stava morendo},
mentre cercava di salvare il suo equipaggio. Non avrebbe legato le sue
speranze a un'oscura favola umana che aveva sentito da qualche parte.
Non era stupido.»

Lord Luthor si schiarì la gola. «Ma al riguardo abbiamo soltanto la tua
parola, lord Marlowe. Il sistema di sorveglianza della stanza è rimasto
inattivo per la maggior parte della conversazione.»

Agari infine si alzò, tenendo china la testa rasata. «E come è successo?
Siamo più vicini a trovare una soluzione al problema?»

«Il personale di manutenzione del Casato insiste che i danni causati
dalle tempeste sono stati più gravi di quanto pensassero in origine»
ribatté lord Luthor, lanciando un'occhiata al marito e lasciando capire
dal suo tono che non credeva a una sola parola. Serrai la mascella e
guardai in direzione delle file di logoteti e di ministri minori seduti
sui due lati della sala, aspettandomi di vedere Valka in mezzo a loro.
Lei però non c'era. Mi morsi un labbro e pregai che la verità sul
problema rimanesse nascosta.

«Possiamo tornare al problema in questione?» chiese la
tribuno-cavaliere, tamburellando con le dita sul bracciolo del seggio.
Quando la sua domanda fu accolta dal silenzio Raine Smythe si ricompose,
appoggiando il mento squadrato su un pugno. «Non sono convinta che
avviare questa procedura da farsa eudoriana partendo dalla cima sia la
nostra linea d'azione migliore.»

Questo lanciò all'offensiva la grande priora. La vecchia megera mosse
tre passi rumorosi per scendere dalla piattaforma con la treccia avvolta
intorno alle spalle che oscillava sulle vesti nere e argento e sollevò
un dito in un gesto di accusa o per un incantesimo. «E quale sarebbe,
soldato? Vorresti trattare con questi demoni?»

Raine non si alzò e invece chiuse gli occhi, con la voce che le si
induriva nel replicare: «Sì.»

«Osi pronunciare parole blasfeme in presenza di Sua eccellenza?» gridò
la grande priora, sovrastando qualsiasi parola potesse seguire, e nel
protendere un braccio verso il conte aggiunse: «In sua presenza?» Poi si
volse di scatto verso Mataro, il suo signore e -- pensai -- il suo
schiavo, continuando. «Sire, tutto questo si è protratto fin troppo a
lungo. I prigionieri hanno confessato che Emesh non è sotto minaccia.
Dovrebbero essere uccisi e dati al popolo, in modo che tutti sappiano
che il Sangue della Terra è forte.»

«Nessuno ne dubita!» esclamò sir William Crossflane, aggredendo a sua
volta verbalmente la sacerdotessa.

«Preferirei parlare con quegli uomini piuttosto che ucciderli» dichiarò
Raine Smythe. Quella frase era intrisa del peso di un lungo uso, come se
avesse l'abitudine di usarla durante le riunioni dello staff.

Ligeia si aggrappò a un dettaglio tecnico con tutta la ferocia della
zelota che era. «Questi non sono uomini, tribuno! Li hai visti! Sono
bestie, demoni del Buio modellati in una parodia di forma umana. Devono
essere epurati dai nostri cieli!»

«Parlando come qualcuno che sta facendo l'epurazione, grande priora,»
ribatté Raine Smythe, con voce sorprendentemente delicata per una donna
tanto indelicata «puoi essere certa che sono consapevole dei miei
doveri.»

«Se mi è permesso...» La voce che interruppe la discussione era molto
fievole, resa più sottile dal suo sommesso accento cantilenante. Sir
Olorin ruotò su un tacco dietro il seggio della sua signora. «Cos'hanno
fruttato finora i metodi della Cappellania, se non la perdita di un
prigioniero e un paio di nomi?»

Con mio stupore, la grande priora si mostrò impaurita al punto da
indietreggiare di un passo su per i gradini della piattaforma, verso
lord Mataro e il trono di corallo. La solidarietà mi spinse ad avanzare
di un passo, oltrepassando Agari e meravigliandomi di come Olorin avesse
intimorito la priora della Cappellania con una domanda così semplice.
«Cosa fare, allora?» ribatté Ligeia, con una frazione della sua
precedente irruenza. «Cercare la conciliazione? La resa?»

«Nessuno sta parlando di arrendersi, signora» interloquì sir William
Crossflane dal suo posto accanto alla tribuno.

Lady Kalima si alzò, ruotando su sé stessa in un gesto che rispecchiava
quello di Olorin, mentre diceva: «Dovremmo considerare la proposta
originale: usare i prigionieri Cielcin per trattare la pace con il loro
capo.»

«Un piccolo clan? Fra quanti?» intervenne la cancelliera Ogir, da un
banco lungo la parete, attirandosi lo sguardo di quanti erano intorno al
trono. «Non sappiamo neppure dove sono.»

«Vorgossos!» esclamai in tono vivace, stringendomi il braccio mentre mi
giravo verso la cancelliera. «Avete sentito tutti quel nome, ma prima
che tornasse la corrente l'\emph{ichakta} ha indicato un rapporto fra i
Cielcin e gli Extrasolari.»

Ligeia ritrovò la voce, forse imbaldanzita dal suo odio nei miei
confronti. «Esseri demoniaci! Alleati con i Pallidi? Traditori e
apostati.»

«Ma uomini» ribattei. Spostai la mia attenzione sulla tribuno-cavaliere
Smythe. «E se non fosse un mito? Se ci fosse qualche base commerciale
degli Extrasolari là fuori?» Conoscevamo tutti le storie, avevamo visto
le opere olografiche. «E se loro conoscessero un modo per contattare i
Cielcin? Potremmo prendere i prigionieri, trovare questo posto...»

«Gli Extrasolari non collaborerebbero mai con noi. Se cercassi quel
posto si darebbero alla fuga, o peggio. Hai mai visto un Glorificato
Visitatore, ragazzo? Le navi più grandi che si siano mai viste. Un gran
brutto avversario. Tuttavia...» Raine appoggiò pesantemente il mento
sulla mano, con i muscoli del collo e della mascella che si contraevano
come se i suoi denti stessero cercando di tagliare il cuoio.

Approfittai di quel momentaneo silenzio. «E se non si trattasse di te?
Se non fosse la Legione?»

Crossflane parlò prima che il suo superiore riuscisse a ribattere. «Cosa
diavolo vuoi dire?»

Mi guardai intorno nella stanza, cercando senza però vederle le uniformi
bianche e marrone che mi servivano. Il comandante Alexei doveva aver
lasciato Emesh con la sua flotta. Deluso, mi girai e risposi: «I
foederati! Potremmo usare i mercenari foederati.»

Luthor Shin-Mataro si protese in avanti sul suo piccolo trono. «Non puoi
dire sul serio. E chi guiderebbe questi tuoi romantici soldati? Tu?»

Poteva davvero essere tanto facile? Potei vedere l'ira silenziosa sul
volto di lord Balian mentre i suoi piani matrimoniali andavano all'aria.
«Perché no?» risposi.

«Non hai \emph{nessuna} esperienza» sogghignò lord Luthor. «E la tua
sbadataggine ci è già costata la perdita di un prigioniero.»

«Dovevo difendermi» scattai, già preparato ad affrontare un'accusa del
genere. «Avete visto cos'ha fatto Uvanari al cathar che era rimasto con
me. Cos'ha fatto al mio braccio!» Spinsi indietro la manica, mettendo a
nudo l'adesivo nero dei correttivi medici che premevano caldi sulla mia
pelle. «In tutta onestà non puoi considerarmi responsabile per essermi
difeso.» Scossi il capo. «Miei signori, signore. Quanto passerà prima
che gli altri Cielcin e questo \emph{aeta} vengano a cercare qui i
compagni? Non è meglio batterli sul tempo? Portare lo scontro o il ramo
d'olivo o qualsiasi altra cosa il più lontano possibile da qui? Lord
Mataro, non voglio vedere il tuo mondo attaccato dai Pallidi... tu lo
sai. Lasciami andare. Ho stabilito un qualche rapporto con il capo dei
Cielcin superstiti e giuro che collaborerà con me.»

Raine Smythe distolse lo sguardo da me per lanciare un'occhiata agli
Jaddiani mentre un'espressione aggrottata increspava il suo ordinario
volto plebeo. Pensosa, tornò a girarsi verso di me e riprese a
tamburellare con le dita sul bracciolo. «C'è un grosso divario fra il
lasciarti andare e il metterti al comando di un esercito di foederati,
Marlowe.»

Olorin si girò per sussurrare qualcosa a lady Kalima, che annuì,
battendo un colpetto sul braccio del suo protettore e consigliere. Nel
notarlo mi schiarii la gola. «Non un esercito, signora, ma una nave, una
sola. L'\emph{Incrollabile} è un trasporto per truppe, giusto? Dovete
avere qualche nave pirata catturata in passato.»

«Una nave pirata?» sogghignò Ogir. «Quanti anni hai, ragazzo?»

Passò un momento mentre mi ricomponevo, riluttante ad abboccare all'esca
della cancelliera. «Ho combattuto contro i Cielcin, cancelliera, e ho
parlato con loro, il che è più di quanto abbia fatto tu.» Distolsi la
mia attenzione dalla brizzolata cancelliera per concentrarla sui grandi
signori dell'Impero e di Jadd. «Non foederati, allora, ma... ma che
sembrino foederati. Ripeto, dovete avere un vecchio relitto nelle vostre
stive. Cos'avete da perdere?» Rivolsi quell'ultima domanda al conte, che
naturalmente aveva \emph{me} da perdere. Avevo fatto affidamento proprio
su questo momento, pregando che la Cappellania uscisse compromessa dal
suo fallimento con Uvanari. La via d'uscita -- la via per sottrarmi al
complotto matrimoniale di Mataro -- era appena dall'altro lato di questa
discussione.

«Che \emph{sembrino} foederati?» ripeté Raine, intrecciando le dita e
protendendosi in avanti sul seggio. «Una missione sotto copertura?»

«I mercenari dell'imperatore» dissi, annuendo. «C'è un precedente, Kasia
Soulier.»

«Kasia Soulier?» sbuffò Ogir. «Sei un cliché letterario ambulante?»

«Sì» scattai. «Chiedilo a chiunque mi conosca.» Il fatto che Kasia
Soulier fosse una persona reale -- un ammiraglio corsaro nella Guerra di
Fondazione e un eroe della mia infanzia -- non parve importare alla
cancelliera o alla folla, perché risero e ciangottarono come uccelli.
Feci scorrere lo sguardo lungo la sala, cercando almeno un alleato fra
quei banchi senza trovarne. Mi ero ingraziato così poche persone a
corte?

«Il piano di lord Marlowe è così privo di merito?» domandò sir Olorin,
sorridendo. «Ricordo un tempo in cui quella di tentare di stabilire un
contatto con i Pallidi era una considerazione condivisa da tutti noi.»
Con l'angolazione della testa abbracciò il trono e tutti coloro che
erano seduti o in piedi sulla piattaforma. Il maeskolos aveva un modo di
fare che era indefinibile. Forse era qualcosa nella cadenza melodica
della sua voce, nel suo atteggiamento noncurante, o di qualche fatto
ignoto della sua storia che imponeva attenzione con la stessa facilità
di un grido. Sotto questo punto di vista era come mio padre, come se nel
suo sangue ci fosse stata una qualche nobiltà che perfino i miei geni
attenuati riconoscevano come superiore, una qualità della persona che si
estendeva fino ai nucleotidi. Quando parlava, la gente lo ascoltava.

«Il ragazzo è necessario qui!» gridò Balian Mataro, infrangendo infine
un qualche muro interiore di controllo o di agitazione. Si protese in
avanti sul trono di corallo, un dio in tutto tranne che nel sangue.
«Deve sposare mia figlia!» allungò alla cieca una mano verso Luthor che
la strinse, offrendo un silenzioso supporto. Potevo vedere il bianco dei
suoi occhi furenti.

Raine però stava annuendo, con le dita che scandivano un loro ritmo sul
bracciolo. «È finalizzato? Il contratto di matrimonio.» Si guardò alle
spalle, inarcando le sopracciglia. Per le ossa della Terra... l'avevo in
pugno.

«Ecco, io...» balbettò Balian.

«No» rispose lord Luthor, stringendo la mano del marito. «La ragazza non
ha ancora avuto il suo Efebeia.»

La tribuno-cavaliere si alzò e si assestò la giacca dell'uniforme, con
gli stivali che echeggiavano sul pavimento di marmo quando si avvicinò
al punto in cui mi trovavo, nel centro della sala accanto ad Agari. Si
fermò a due passi da me, e per la prima volta mi resi conto di quanto
fosse piccola, tanto che quasi non mi arrivava alle spalle. Alzò il
volto per scrutarmi, con le sottili cicatrici bianche che scintillavano
sullo sfondo della pelle bruna, e mi fissò con quegli occhi castani duri
e astuti. «Sai, credo che tu abbia ragione, ragazzo.» Niente `mio
signore' o `signoria' o traccia di deferenza nel suo tono. «Gli altri
Cielcin sono inconsapevoli di cosa è successo al loro capo?»

Annuii. «Sì, signora, mi hanno specificatamente ordinato di non
dirglielo.»

Arricciò le labbra e si massaggiò la mascella. «Bene, continua così. Di'
loro...»

«Quando è arrivato era ferito» la interruppi, cosa che chiaramente la
disturbò. «Dirò loro che non siamo riusciti a salvarlo.» Sentii le mani
che mi si serravano lungo i fianchi, come se stringessero il coltello
che aveva ucciso Uvanari.

La tribuno mi studiò per un lungo momento, riuscendo in qualche modo a
osservarmi dall'alto in basso anche se doveva sollevare lo sguardo.
«Sono d'accordo con il nostro emissario jaddiano. Ritengo ci sia ben
poco da guadagnare interrogando i prigionieri rimasti e propongo di
riesaminare la mozione di mandare un gruppo alla ricerca di questo
principe Cielcin... come lo hai chiamato?»

«Aranata.»

«Di questo Aranata» concluse, un po' goffamente.

La cancelliera Ogir si alzò dal suo posto all'estremità del banco.
«Tribuno-cavaliere, tutto questo è quanto mai irregolare. Sei stata tu a
insistere inizialmente per l'Inquisizione.»

«L'ho fatto» convenne Raine Smythe, piantandosi i pugni sui fianchi come
a sfidare chiunque a essere in disaccordo con lei. «Da allora ho però
avuto il tempo di rivedere la mia posizione, cancelliera. Sai cosa
significa `rivedere'? Significa che ho cambiato idea.» Questo intimidì
Ogir per un momento e \emph{} lei guardò verso la piattaforma in cerca
di supporto, ma non ne ricevette neppure da Vas. «Le indagini della
Cappellania ci hanno fornito qualche utile informazione, soprattutto
grazie all'interferenza di lord Marlowe, con le loro procedure operative
standard. Non sono diventata la tribuno della 743° sprecando vantaggi
strategici e diplomatici, e non intendo cominciare a farlo oggi.» Per un
momento guardò di nuovo verso di me. «In base all'articolo 119 dei
Grandi Atti Costitutivi e in virtù della mia carica, recluto al mio
servizio il qui presente Hadrian Marlowe, insieme a qualsiasi esperto
che lui voglia raccomandare. Mi dovresti fornire una lista completa
entro la fine della settimana.» L'espressione di sofferenza sul volto di
Balian Mataro fu l'unica reazione nella sala. Raine però non aveva
finito. «Richiedo anche che i Cielcin prigionieri siano consegnati sulla
\foreignlanguage{italian}{isv} \emph{Incrollabile} nel nome della Sua
Imperiale Radiosità l'imperatore William XXIII del Casato Avent.»

Questo strappò dei sussulti ai logoteti che stavano doverosamente
prendendo annotazioni e ascoltando dai loro banchi e mise in azione
Ligeia. La vecchia priora barcollò scendendo i gradini della piattaforma
in un fluire di vesti nere, sollevando un dito. «Il Sinodo non lo
permetterà mai!»

«Il tuo Sinodo non è qui, la Legione imperiale sì» ribatté con fermezza
Raine Smythe. «Faccio questa richiesta nel nome del nostro imperatore e
nella luce della Fede. Davvero, Vostra reverenza, una manciata di
prigionieri vale più in uno sforzo bellico che nella propaganda.» Con le
mani sempre sui fianchi si fissò gli stivali per un momento, quasi
dondolandosi all'indietro sui tacchi. Era uno strano gesto, che però
tenne tutti ad aspettare con il fiato sospeso le sue parole, certi che
stesse per continuare. Presi mentalmente nota di utilizzarlo anch'io.
Per la Terra, quella donna aveva una presenza che incuteva obbedienza.
«Se stiamo cercando un mondo extrasolare per stabilire un contatto,
questa cosa dei foederati ha senso, ma concordo con la cancelliera e --
credo -- con i nostri amici jaddiani sul fatto che sarebbe un errore
lasciare questa impresa nelle mani di lord Marlowe che, come si è fatto
notare, non ha esperienza in questioni del genere. Mi piacerebbe mettere
al comando uno dei miei ufficiali.»

«Chi?» domandai, incapace di trattenermi.

«Bassander Lin» rispose Raine, poi procedette a delineare il suo piano.
Mi fissai i piedi con rabbia, pensando all'alto ufficiale alquanto
severo che non mi aveva voluto a Calagah. Forse non ero uscito da tutto
questo vincitore quanto avevo immaginato.

\begin{figure}
	\centering
	\def\svgwidth{\columnwidth}
	\scalebox{0.2}{\input{divisore.pdf_tex}}
\end{figure}

Quando fu tutto finito, Raine mi fermò nel corridoio, all'ombra di un
murale di vetro colorato che raffigurava una battaglia fra il Casato
Mataro e i colonizzatori normanni di tanto tempo prima. Ombre vivide
come gemme si proiettavano sulle piastrelle a spina di pesce e sulle
colonne smaltate che salivano fino ai soffitti a volta. «Non so bene
come tu ci sia riuscito,» disse, serrandomi le dita intorno ai bicipiti
«ma so che hai fatto togliere la corrente.» Non risposi e guardai con
sorpresa quella dura donna sfregiata. «Puoi smetterla con la recita,
Marlowe. Adesso non ti possono toccare... sei mio.»

Una vecchia abitudine mi spinse a guardare in alto per cercare di
individuare le videocamere fra le decorazioni a spirale in stile rococò
e le decorazioni barocche dell'architettura palatina. Non riuscii a
trovarle e mi resi anche conto che quello non era il gesto, o
l'abitudine, di un uomo innocente. Mi limitai a sorridere. «Non approvo
la tortura» dissi, a titolo di risposta.

«Neppure io» ammise la tribuno-cavaliere, ottenendo un cenno di assenso
dal suo littore, l'anziano sir William Crossflane. «La guerra però esige
dure risposte da tutti noi, vero? E non approvo i soldati che fanno
giochetti come il tuo. Ti prendo con me, ma non perché lo volevi. Lo
faccio perché penso di aver bisogno di te e perché il tenente Lin può
tenerti saldamente sotto controllo.» Aprii la bocca per replicare, anche
se non sapevo cosa dire, ma la tribuno non aveva finito. «Per quel che
vale, credo che tu abbia ragione. Hai fatto una cosa rara... stupida, ma
rara. Puoi non averne visto molto di sicuro su questa roccia, ma la
guerra si protrae da decisamente troppo tempo...» Si interruppe
scuotendo il capo, e pensai che fosse turbata da visioni del passato.
«C'è qualcuno che vuoi portare con te? Qualcuno che posso arruolare?»

Mi presi un momento di pausa, esitando perché la scelta ovvia era una
che si trovava al di là della portata della tribuno, per quanto fosse
lunga. «Mi piacerebbe parlare con la xenolga tavrosiana, Onderra.»

«È una cittadina straniera» mi ricordò Smythe. «Non possiamo
reclutarla.»

«Io posso» replicai, anche se non ero certo del come facessi a sapere
che era possibile. «Se vogliamo apparire come foederati non dovremmo
puzzare tutti di Legione, giusto?»

Questo le fece affiorare sulle labbra un accenno di sorriso mentre
annuiva. «Qualcun altro?»

Nonostante tutto, sorrisi anch'io. «Oh, sì.»

