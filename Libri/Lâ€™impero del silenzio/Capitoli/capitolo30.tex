\chapter{Gli Umandh}

Borosevo aveva le sue fattorie urbane, grandi torri di vetro che
filtravano la luce di quel sole gonfio e la trasformavano in qualcosa di
più delicato per la coltivazione delle forme di vita terrestri, ma non
erano sufficienti per nutrire una popolazione di cinque milioni di
persone, e un uomo non vive solo di verdure. Il resto del cibo locale
veniva dall'oceano, portato a terra dagli allevamenti ittici ai moli che
si trovavano sul lato meridionale del quartiere dei mercanti. La maggior
parte dei senzatetto della città -- gli storpi, gli orfani, gli uomini
distrutti -- se ne teneva alla larga, ma non noi.

«Davvero non mi credi?» domandò Cat, grattandosi i capelli che erano
irregolari dove li aveva accorciati. Prima che la necrosi la facesse
dimagrire, prima della pestilenza che si abbatté sulla città durante il
mio secondo anno di permanenza la stroncasse, era vivace e focosa. Era
\emph{felice,} davvero felice, contenta di vivere per strada, di
raccattare e di rubare, felice della sua ignominia come io lo ero della
mia libertà. Era questo che ci univa. «Giuro che sono reali.»

Seguendola, mi massaggiai il mento permanentemente glabro perché i
follicoli erano stati bruciati al mio tredicesimo compleanno secondo
l'usanza di Delos, e scossi il capo. «È solo che non li ho mai visti.»
Emesh si era mai trovato nell'elenco dei pianeti su cui Gibson mi
interrogava? Cercai di ricordarlo.

«Io non ho visto questo pianeta da cui dici di venire,» mi fece notare
Cat, sorridendo in un modo che le nascondeva i denti «ma sono sicura che
è là fuori.»

«Questo è diverso» ribattei in tono di rimprovero, seguendola in un
canale sotterraneo aperto che ci avrebbe portati sotto un bazar al
chiuso per turisti provenienti dall'esterno e fino ai magazzini e alle
fabbriche di inscatolamento dove i pescherecci depositavano i loro
carichi. I pescherecci, con i loro equipaggi che pareva non fossero
umani.

«I coloni sono reali, Had.» Mi strinse la mano. «È per questo che gli
altri non vengono qui.» Si riferiva agli altri poveri della città. Si
diceva che i magazzini del pesce fossero poco sorvegliati, ma tutti
erano terrorizzati dai nativi del pianeta, gli Umandh. Da quanto avevo
sentito, quelle creature erano tozze, mostruose e del tutto diverse da
un essere umano nella forma, con tre gambe, la carne simile a pietra o
corallo e la bocca piena di filamenti simili a braccia robuste.

Invece di risponderle accennai alla rozza calcina sopra la nostra testa.
«Il condotto si riempirà completamente con il cambiare della marea?» Lei
sorrise, questa volta rivelando i denti. «Un poco, sì.»

«Un poco?» ripetei, in un tono che trasformava quelle parole in una
domanda, mentre l'antico sorriso dei Marlowe mi incurvava le labbra.

«Muovi il culo e non lo scoprirai.» Cat rise e mi spinse nel condotto
coperto d'acqua.

Camminammo per circa cinque minuti, immersi fino al ginocchio nell'acqua
di mare che saliva in piccole onde fino a lambirci le cosce. Un pesce ci
oltrepassò rapido, spaventato dal nostro passaggio. «Perché gli altri
non vengono qui? Se rubare pesce è facile come dici...»

«Hanno paura dei coloni.»

«Perché?» insistetti, sinceramente perplesso. Avevo visto parecchi
filmati olografici sui coloni in passato: le razze suddite dell'Impero,
gli alieni primitivi che erano stati catturati e sottomessi. Avevamo
trovato creature intelligenti su quarantotto mondi, alcune geniali,
altre ottuse e altre ancora strane. Per quarantotto volte le avevamo
rese schiave, perché nessuna era più progredita dell'era del bronzo. I
Cavaraad su Sadal Suud, gli Irchtani di Giudecca, gli Arci-costruttori
di Ozymandias. E ce n'erano altri, molti altri, alcuni protetti, altri
estinti, ridotti in polvere dalle necessità dell'espansione umana. Solo
i Cielcin erano diversi, solo loro erano abbastanza forti da opporre
resistenza.

Gli antichi erano soliti lamentarsi che le stelle del cielo erano troppo
numerose per supporre che noi fossimo la sola forma di vita, i soli
eredi dell'universo. Trovavano strano che nessun'altra razza gridasse
nel Buio, che le loro onde radio e il loro rumore non risuonassero
attraverso l'oscurità infinita, ma quando le nostre lunghe navi avevano
attraversato gli oceani della notte e piantato bandiere su rive lontane,
avevamo scoperto che la verità era semplice: noi eravamo i primi. La
Cappellania aveva preso a cuore quel fatto, dichiarando a gran voce e
spesso che le stelle erano \emph{nostre}, che appartenevano ai figli
della Terra, e aveva strutturato la sua religione su quel fatto
essenziale nella stessa misura in cui l'aveva basata sulla paura del
potere corruttivo della tecnologia, sulla sua contaminazione della forma
umana. Avevamo il diritto di conquista, sosteneva, come avevano
affermato gli antichi spagnoli quando le loro tristi navi avevano
toccato terra.

Cat non aveva risposto alla mia domanda, camminando davanti a me in
silenzio e in preda a una tensione improvvisa da cui trapelava un
nervosismo che la tradiva nel tremito delle piccole mani ossute, nella
linea tesa delle spalle sotto il vestito lacero che copriva il suo corpo
magro. «Perché hanno paura dei coloni?» ripetei a me stesso.

Lei si girò a guardarmi da sopra la spalla con la fronte aggrottata come
se fossi stato l'uomo più stupido che avesse mai incontrato. «Sono
demoni, Had. Perché non ne hai paura?»

Non seppi cosa rispondere. Provavo solo quello scintillio fatto di
curiosità e di eccitazione che da ragazzo mi aveva spinto a consultare
libri e ologrammi. «Credi che possiamo farcela?»

«A derubare i coloni?» Scrollò le spalle, soffermandosi a una
biforcazione del condotto per orientarsi. Vagamente, potevo sentire il
passo di parecchie migliaia di piedi nel bazar sopra la nostra testa, un
tuono ovattato e costante che copriva il sussurro delle voci umane,
filtrando attraverso la pietra come da un altro mondo. «Non è difficile
perché non sorvegliano il pesce.» Imboccò il ramo di sinistra.

La seguii sollevando le ginocchia per muovermi con maggiore precisione
nell'acqua, dietro di lei. La mia maggiore statura era un vantaggio che
mi permetteva di procedere con più facilità anche se minacciavo di
strisciare con la testa contro il soffitto. «Allora qual è il problema?»
Tenni lo sguardo sull'oscillare dei suoi fianchi magri nella penombra,
sul modo in cui il vestito umido le aderiva addosso.

Lei si girò a guardarmi con gli occhi che lampeggiavano nell'oscurità.
«Non mi stai ascoltando. Loro sono...» Scosse il capo. «Sono sbagliati.»

\begin{figure}
	\centering
	\def\svgwidth{\columnwidth}
	\scalebox{0.2}{\input{divisore.pdf_tex}}
\end{figure}

La prima cosa che notai fu il ronzio. In un primo momento pensai che si
trattasse di mosche, perché quel tormento onnipresente era una cosa
comune in quella dannata città. Quel suono però era più profondo di
quello di qualsiasi insetto o della più profonda voce umana. Tutta
l'aria ne risuonava come se fossimo stati rinchiusi sotto la pelle di un
grande tamburo, facendomi rizzare i peli delle braccia per la tensione.
Cat si ritrasse di fronte a quel suono, indietreggiando verso il punto
in cui ci eravamo issati sul molo all'esterno del magazzino dalle pareti
di latta. Per un momento pensai che il suono provenisse dai razzi che
partivano sopra le nostre teste, ma esso non aveva una direzione
apparente anche se era chiaramente vicino. Due grandi pescherecci
dondolavano su e giù con la risacca, in attesa all'estremità del molo,
con lo scafo dipinto con una vernice bianca da tempo scrostata e
chiazzata di ruggine e di salsedine. Attraversai di corsa il molo e
trascinai Cat dietro di me nell'ombra di una catasta di casse
refrigerate con i lati d'acciaio viscidi per la condensa. Premetti la
fronte contro il metallo, grato per quella frescura e la fugace
sensazione di pulizia che il contatto con quell'acqua fresca portava
alla mia pelle coperta di salsedine. Sollevando lo sguardo trovai quello
che stavo cercando, una scala antincendio di metallo marrone fissata con
bulloni alla parete, anch'essa di metallo, del magazzino. Mi mordicchiai
un labbro nel valutarne l'altezza e la distanza.

Il ronzio si era fatto ancora più forte. Le unghie scheggiate di Cat mi
affondarono nella carne del braccio, inducendomi ad abbassare lo sguardo
su di lei. Con quanta chiarezza ricordo quel volto, le linee lisce degli
zigomi e della fronte sotto la pelle bruna chiazzata dal sole e dalla
salsedine. Gli occhi erano grandi, vividi e spaventati, il piccolo naso
e il sorriso con i denti storti -- assente in quel momento -- erano
pervasi di paura. Le strinsi la mano. «Ce la caveremo bene. Entriamo e
usciamo.» Lei non rispose. «Puoi aspettare qui.»

«Da sola?» Sgranò gli occhi color ambra. «E se uno di loro venisse
fuori?»

«Questa è stata una \emph{tua} idea» sibilai, allungando il collo per
sbirciare al di sopra delle casse refrigerate e verso le navi. Un paio
di douleter umani scese la passerella, con le uniformi cachi che
aderivano ai corpi grassi e le teste calve che brillavano lucide. Mi
abbassai maggiormente, osservandoli. Uno dei due stringeva in una mano
una frusta arrotolata e nell'altra un bastone storditore come quelli che
a volte usavano i prefetti.

Cat distolse lo sguardo, fissandosi i piedi nudi e la sabbia grigia di
cui erano incrostati. «Lo so. Io...» La vidi serrare la mascella,
pervasa di determinazione. Mi lasciò andare il braccio e io la baciai
sulla fronte prima di balzare sulla cassa alle mie spalle, chiudendo le
dita intorno al sigillo di gomma che manteneva l'aria fredda
all'interno. Mi girai, tornando a valutare la distanza dalla scala
ripiegata che pendeva sopra di me. «Hadrian, aspetta... aiutami a
salire.»

«Abbasserò la scala» risposi, lottando per tenere bassa la voce.
Girandomi, spiccai un salto nel vuoto fino a chiudere le dita intorno a
un piolo alla base della scala. I miei pochi mesi a Borosevo avevano
consumato tutta la morbidezza e il peso in eccesso, e il semplice
muovermi in quella gravità più accentuata mi aveva reso più forte.
Comunque fui fortunato a non essere visto e ancora più fortunato per il
fatto che quel profondo e orribile ronzio contribuì a mascherare il
clangore e lo stridio della scala che si abbassava. Segnalai a Cat di
spicciarsi a salire e ben presto fummo sul tetto, sferzati dal vento di
mare. Per un secondo fu come essere di nuovo a casa, circondato dal
mare. L'odore di salsedine era lo stesso, anche se il cielo fra il rosa
e la terra d'ombra era sbagliato, come pure il rovente sole arancione.
Le ombre turbolente e increspate dell'aria intessevano arabeschi sulla
ghiaia bianca che copriva la sommità del tetto. Ci dirigemmo in fretta
verso la porta, che aprimmo lentamente per accedere a una scala buia che
scendemmo fino a una passerella traballante come la scala antincendio.

Le canzoni e le opere, gli ologrammi, i poemi e le opere epiche
affermano tutti che il momento della rivelazione è uno shock, un
culmine, un istante di schiacciante realizzazione che altera il mondo, e
non si sbagliano. Chiedete a chiunque si sia trovato con me a Gododdin,
che abbia visto il sole assassinato sprofondare nel fuoco, e vi
confermerà la verità di quelle storie. Tuttavia, noi facciamo in fretta
a non fare caso alle tacite rivelazioni, a quei momenti che non derivano
dal caos del mondo ma da un cupo seme nel profondo del proprio stomaco.

Dall'alto di quella passerella Cat e io guardammo le casse aperte piene
di piccoli pesci argentati sotto sale e di altri più grossi sotto
ghiaccio. Guardammo i douleter in uniforme con frusta e bastone in mano,
e i lavoranti a loro affidati. Non so bene cosa mi fossi aspettato dai
coloni, dagli indigeni che avevano posseduto Emesh prima che diventasse
un territorio dell'uomo, ma di certo non questo.

Gli Umandh erano come colonne che oscillassero sotto il soffio di un
vento inesistente, come torri ambulanti alte quanto un uomo o anche di
più, ciascuna in equilibrio su tre gambe arcuate che emergevano in modo
radiale da quella che suppongo si potrebbe definire la cintola. Laddove
le torri vere erano dotate di merlatura, di una corona di pietra
consumata dal tempo, quelle creature coralline con la carne simile a
roccia fra il bianco e il rosa avevano ciglia carnose larghe quanto il
braccio di un uomo e lunghe quasi il triplo. Senza che mi venisse detto,
compresi che era quella la fonte dell'incessante ronzio, e nonostante il
calore soffocante del magazzino mi sentii assalire da un brivido.
«Stanno cantando» sussurrai.

Cat mi lanciò un'occhiata in tralice, ma io non mi soffermai a guardarla
perché avevo occhi solo per quelle cose inumane sotto di noi. Avevo
trascorso anni, innumerevoli ore, a studiare i Cielcin, il loro
linguaggio, le loro usanze, la loro storia, e di colpo quegli
implacabili nemici dell'uomo mi apparivano estremamente umani. I Cielcin
avevano due occhi, due braccia, due gambe, due sessi, per quanto l'uno
fosse legato all'altro. Avevano un linguaggio parlato, indossavano
vestiti e armatura, mangiavano a tavola, parlavano di onore e di
famiglia. Avevano sangue che scorreva nelle vene in maniera molto simile
al nostro.

Gli Umandh erano diversi, come se l'Evoluzione Colta in Flagrante, nella
sua capricciosità avesse modellato i nativi di Emesh come una critica
alla nostra somiglianza con i pallidi Cielcin. Due di quelle creature
sollevarono una delle casse con le ciglia, avvolgendole intorno alle
grosse maniglie. E il loro tronco vibrò, alterando il tono del canto
ronzante. Per la prima volta notai lo spesso collare stretto intorno al
torso, con il metallo che sfregava contro la carne perlacea, tingendola
di un rossore infiammato. Quella vista mi fece pensare ad alberi intorno
ai quali fosse stato legato un cavo, con il risultato che con il passare
del tempo e la crescita della pianta, il metallo affondava sempre più in
profondità nel legno.

Uno degli umani fece crepitare la frusta nell'aria. «Più in fretta,
razza di cani!» gridò, con una voce rude che mi ricordò Gila e i
lavoranti degli hangar che avevano derubato il mio corpo incosciente e
saccheggiato la nave di Demetri. Le enormi creature barcollarono, con il
costante ronzio che si fletteva per lo sforzo come la corda pizzicata di
un'arpa.

«Porteranno il pesce a una chiatta per trasferirlo in città» sussurrò
Cat, arroventandomi il collo con il suo respiro nel protendersi verso di
me. «Una di quelle barche. Dobbiamo fare in fretta.»

Fu il mio turno di afferrarla per un braccio. «Aspetta che siano di
nuovo all'esterno.» Lei trasse un profondo respiro, combattuta fra la
fame e la paura. «E dammi il sacco.»

Mi fissò con occhi roventi. «Posso farcela, Had.»

«Lo so, ma lascia che ci pensi io.» Intanto continuavo a osservare la
scena con espressione sempre più aggrottata. «Perché li usano come
schiavi? Verrebbe logico pensare che ci siano modi più facili.»

«Loro possono vivere sott'acqua» disse Cat. «Camminano sul fondale
marino.»

«Pastori?» Mi accigliai di nuovo. «Per i pesci?» Il ronzio mi scosse in
tutto il corpo mentre sfilavo il sacchetto di plastica dalle dita di Cat
e lo liberavo dall'acqua del condotto. Non ci sarebbe servito molto
pesce, un sacchetto ci sarebbe bastato per una settimana, anche più a
lungo se fossimo riusciti a trovare qualcosa con cui metterlo in
salamoia. Mentre guardavo, uno dei douleter usò il bastone per colpire
il tronco stretto di uno degli Umandh. La creatura emise un grido simile
al barrito di un elefante o al canto di una balena. Come un singhiozzo
umano soffocato. Non lo so descrivere, non c'è parola umana che possa
definire quell'agonia aliena. La creatura barcollò, crollando sulle tre
ginocchia, con il tronco che si accasciava come un fiore che appassisce
alla prima ondata di calura estiva, e l'enorme cesto di vimini che
trasportava si rovesciò, spargendo il pesce sul pavimento del magazzino.

«Cosa cazzo hai che non va, Diciassette?» chiese il douleter con
un'imprecazione. Mentre parlava, un uomo con una massiccia console mosse
un paio di manopole, alterando il tono e la frequenza del grande ronzio
che pervadeva l'aria. Realizzai che stava traducendo, e una parte di me
-- antica e quasi dimenticata -- sorrise interiormente, scordandosi
dell'orrore del momento. Non c'era niente che desiderassi di più dello
smontare quella console, del sedermi a parlare con quell'uomo e con le
creature con cui era stato addestrato a parlare. Allora quello era un
linguaggio? O qualcosa di totalmente diverso? Volevo saperlo. Dovevo
saperlo. Poi lo stomaco mi si contrasse, gemendo per la fame che era
diventata una parte di me, della mia vita.

Il timbro del suono cambiò ancora mentre l'Umandh atterrato lavorava per
raccogliere il pesce caduto, con i tentacoli che afferravano e si
spostavano sul rozzo pavimento di cemento. Era una cosa strana, perché
la maggior parte del canto alieno rimase immutato e la differenza
risultò qualcosa di fine, di minuscolo -- un contrappunto in una
sinfonia le cui note erano così indistinte che non le potevo separare
una dall'altra.

«Si sta scusando, Quintus» disse l'uomo alla console.

«Sarà fottutamente \emph{meglio} che lo faccia» ringhiò il primo uomo, e
colpì di nuovo l'Umandh che emise un altro gemito, anche se il ronzio
non cessò mai. «Guarda questo pesce sparso per tutto il pavimento, che
la Terra ti bruci» sibilò a denti stretti. «Razza di fottuta... bestia!»
Fra la seconda parola e la terza scaraventò a terra con un calcio la
creatura allampanata, e anche se essa non aveva una faccia visibile mi
tornò in mente quel giorno -- che sembrava risalire a diecimila anni
prima -- in cui avevo visto quel gladiatore schiacciare nel Colosso la
faccia di uno schiavo mutilato. Eccola di nuovo: la faccia della nostra
specie, escoriata, rossa ed esposta. Il douleter, Quintus, pungolò nelle
costole l'Umandh caduto. «Alzati!» Esso non lo fece.

Adesso mi dico che avrei potuto impedirlo, che sarei potuto intervenire
lasciandomi cadere dalla passerella su quel supervisore infuriato. È
difficile ricordare quei brevi anni di impotenza, dopo tutto il potere
che ho esercitato nella guerra. Gli ingranaggi dell'Impero riducono in
polvere la loro pula umana e sono solo le creature più rare a resistere.
A crescere. A elevarsi. Intoniamo canti, intessiamo storie di sir
Anthony Damrsh -- nato nella condizione di servo -- o di Lucas Skye,
storie che di notte avevo condiviso con Cat centinaia di volte. Ci piace
immaginare che sia facile elevarci, essere un eroe, ma non lo è. Quello
non era il mio momento, non ero un eroe. Non lo sono adesso e non lo ero
allora.

Ero soltanto un ladro.

«Alzati!» comandò il douleter, mentre il suo assistente muoveva le
manopole per aggiungere quel tono al canto. «Alzati, dannazione a te!»
La carne simile a pietra non era davvero di pietra e si crepò sotto il
suo stivale, lasciando colare qualcosa di giallo e di denso che riempì
immediatamente l'aria di un fetore degno della fossa più profonda
dell'inferno. Quando l'Umandh non si rialzò, l'uomo calò il bastone come
la spada di un littore una, due, tre volte. I gemiti della creatura
cessarono ed essa collassò singhiozzando mentre sanguinava sul
pavimento.

«Quintus, basta!» L'altro sovrintendente lasciò ricadere la console,
appesa alla cinghia che gli cingeva il collo e si affrettò a fermare il
compagno. «Lascia in pace questa bestia!» Dentro di me qualcosa si
rilassò e si distese, perché era apparsa l'altra faccia della razza
umana, la Pietà. L'alto uomo afferrò Quintus per una spalla e lo tirò
indietro prima che potesse colpire ancora. «Il capo ti toglierà il tuo
bonus se uccidi questo \emph{colonus}.»

Le viscere tornarono a contrarmisi. Quella non era affatto l'altra
faccia dell'umanità, era solo la vecchia avidità. «Dobbiamo spicciarci»
mi sussurrò Cat.

«Non ancora» replicai, appoggiandole una mano sulla gamba mentre se ne
stava accoccolata accanto a me vicino alla ringhiera. «Presto.» Serrai i
denti, guardando mentre il secondo douleter -- con l'assistenza di un
altro dei coloni -- aiutava la creatura ferita a rimettersi sui suoi tre
piedi allargati.

Il ronzio cambiò, salì di tono, ululando e pulsando come il battito di
un cuore. Il secondo douleter controllò lo schermo della console.
«Vogliono portare Diciassette dal medico, Quint» disse.

«Fottuti...» Lo schiavista scosse il capo. «Benissimo! Fatelo! Engin mi
pelerà vivo se ne perdiamo un altro.» E si massaggiò il braccio con cui
reggeva il bastone come se gli facesse male, come se esso, e non lui,
fosse stato responsabile dell'accaduto.

Poi se ne andarono, varcando la porta e uscendo nella luce del
pomeriggio. «Torno subito.» Battei un colpetto sulla gamba di Cat e
abbassai la scala più vicina, che toccò terra accanto a una delle casse
aperte. Agendo in fretta, ficcai nel sacchetto due grossi pesci --
tonni, credo -- insieme a creature simili a serpenti di cui non
conoscevo il nome. Continuai così, mettendo altri pesci nel sacco fino
ad averne a sufficienza per tre giorni, per quattro. Abbastanza per
nutrire i piccoli orfani di cui Cat si prendeva cura quando non
riuscivano a ottenere un'elemosina di pane dai cantori.

Sorridendo, risalii la scala.


