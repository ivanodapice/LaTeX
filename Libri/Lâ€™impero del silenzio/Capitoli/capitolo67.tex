\chapter{Tempo perduto}

La nostra esposizione agli oceani dello spazio ha trasformato i nostri
vasti mondi in piccole isole. I nostri potenziamenti genetici hanno
alterato la nostra percezione del tempo. Come hai appreso, lettore, in
passato ho vissuto sulle strade e lungo i canali di Borosevo senza
curarmi del costo in termini di anni. Sopraffatto da tutto quel rumore,
quel colore, quella vivacità e l'andirivieni di una città e della sua
popolazione morente, ho pensato che il tempo da me perduto non valesse
niente, considerati i secoli di vita che il mio sangue mi garantiva.
Quanto è stato facile credere che sarei potuto rimanere là, con Cat, nel
nostro fatiscente condomino, finché il tempo stesso non mi avesse
abbattuto come gli edifici.

L'uomo che
spera nel futuro ne ritarda l'arrivo e quello che lo teme lo convoca
alla sua porta.

Una volta Agostino ha detto che se ci sono cose come il futuro e il
passato non esistono come tali ma sono solo il presente del loro tempo.
Il passato, afferma, esiste soltanto nella memoria e il futuro
unicamente nelle aspettative, e nessuna delle due cose è reale. Il
passato e il futuro -- la nostra vita e i nostri sogni -- sono storie.
Alla fine siamo tutti delle storie, solo questo, ed è nella natura delle
storie che il presente e il passato siano presenti nel futuro e che esso
sia presente nel passato. Di conseguenza, tutto il tempo è sempre il
presente nella mente e nella storia della mente, e forse in quelle forze
che la modellano. Il poeta ha scritto che tutto il tempo è
irrecuperabile, che quello che sarebbe potuto essere è soltanto
{un'astrazione}: parole nello spazio quantico, irrealizzate, che a loro
volta definiscono gli eventi tramite la loro esclusione \emph{da} essi.

E se? Cosa sarebbe potuto succedere?

Mi vedevo nelle sale polverose di quell'ateneo su Teukros, e nelle
camere a volta del seminario su Vesperad. Altri Hadrian calpestavano la
polvere di altri mondi, distrutti, irreali. I passi echeggiavano nella
memoria lungo passaggi che non avevo imboccato, porte che non avevo
aperto.

Non erano niente al confronto dei pensieri su dove era possibile che
stessi andando.

Il futuro può anche venire soltanto a suo tempo, ma gli scoliasti
insegnano che ci sono molti futuri e che è soltanto l'infrangersi delle
onde del tempo e delle possibilità contro l'interminabile \emph{adesso}
che crea il mondo. Non è \emph{il} futuro a essere presente nel Tempo
Sempre Fuggevole, ma \emph{i} futuri. La libertà di pensiero e di azione
sono importanti e garantite perché il futuro non lo è. Non ci sono
profezie, solo probabilità. Non c'è il fato, solo il caso. Il presente
non è quando siamo, ma cosa facciamo.

\begin{figure}
	\centering
	\def\svgwidth{\columnwidth}
	\scalebox{0.2}{\input{divisore.pdf_tex}}
\end{figure}

Meditavo su questi e su altri fatti materiali, seduto mezzo ubriaco
sulla spiaggia che si affacciava su un mare di un pallido colore
turchese. La notte su Emesh non era mai veramente buia perché il
qualsiasi momento o Binah o Armand erano visibili nel cielo,
risplendendo di un colore verde o rosa nella penombra. Quella notte
erano presenti entrambe: la massiccia e boschiva Binah bassa
sull'orizzonte e la piccola Armand che risplendeva come una gemma alta
nel firmamento, dove surclassava le stelle stesse. Il vento fischiava
lungo la fenditura alle mie spalle, gemendo in mezzo alle affusolate
colonne di Calagah nell'uscire nel mondo.

\emph{L'Impero. La Cappellania. Anaïs. Gilliam. Ligeia Vas...}
giocherellavo con piccoli frammenti di pietra, grigi sullo sfondo della
nera sabbia vulcanica, allineandoli. \emph{Gli Jaddiani. Sir Olorin. Sir
	Elomas. I lord Balian e Luthor.} Erano tutti pezzi su una scacchiera.
Distrussi la fila di pietre. \emph{I Cielcin. La guerra.}

`Tutto quello che dici deve suonare come se venisse dritto da un
melodramma eudoriano?' Le parole di Gibson emersero dalla storia antica,
attinte da un tempo più semplice. Soffocando una risata inclinai la
bottiglia di vino prima di incastrarla nella sabbia. Ero entrato in un
melodramma, giusto? O meglio, me ne ero aspettato uno, creandolo così io
stesso. Scossi con rabbia la testa per allontanare i ricordi e cercai di
ultimare il mio disegno di lady Kalima, pur disperando di riuscire mai a
catturare adeguatamente il disprezzo nei sui occhi di \emph{eali}. Il
carbone si spezzò e io imprecai, lasciando cadere il diario sulla sabbia
accanto a me mentre mi appoggiavo con la schiena contro la roccia.

«Stai bene?»

Sussultai con un grido e per poco non rovesciai il vino. «Devi smetterla
di prendere le persone di sorpresa!» esclamai. Più sommessamente
imprecai contro me stesso, chiusi il mio diario e raccolsi la matita
spezzata.

Valka era in piedi su un'altura sopra di me, in equilibrio con ciascun
piede su un diverso supporto di basalto, appollaiata come un assassino
mandari pronto a uccidere. Prima dell'incidente nella galleria avevamo
parlato molte volte così, ma da allora ero rimasto seduto da solo in
numerose notti. «Lo fai suonare come se fare cose del genere fosse un
mio hobby.» Aveva le mani affondate nelle tasche della giacca di pelle
rossa che aveva preso a indossare da quando eravamo venuti al Sud, le
cui falde le si agitavano intorno alle ginocchia sotto il soffio del
vento notturno.

Una smorfia spettrale mi contorse il volto nel ricordare il suo momento
di tempismo sbagliato, una settimana prima nelle gallerie. Da allora non
avevamo parlato molto e la verità era che l'avevo evitata. «Ecco, allora
non... cominciare.» La mia smorfia si accentuò. \emph{Ottimo lavoro,
	Marlowe. Sei davvero coerente.} Cercai di salvare la mia dignità. «Come
facevi a sapere che ero qui? Da quanto sei ferma lassù?»

«Non da molto, e tu vieni qui fuori quasi ogni notte.» Saltò giù dalla
scarpata, sollevando piccoli sbuffi di sabbia nell'atterrare, e mi
sbirciò da sotto un velo di capelli scuri. Alla luce delle lune i
riflessi rossi ardevano come rame brunito, risplendevano come i contorni
in fiamme di un pezzo di pergamena. Credo che si preoccupò per me nel
vedere la bottiglia, mezza vuota com'era, perché quando riprese a
parlare lo fece con il tono che qualcuno poteva impiegare nel parlare di
un membro della famiglia colpito dalla peste. «Abbiamo... parlato qui
fuori in passato. Parecchie volte.» Era vero. Da quando ero arrivato a
Calagah lei e io -- spesso avendo a traino Elomas oppure Ada o lo
scudiero Karthik -- avevamo vagato lungo la spiaggia entro un miglio
circa dalla fenditura.

«Questo lo so.» Feci una smorfia e abbassai lo sguardo sul mio diario,
tirandomelo in grembo sporco di sabbia com'era. «È solo... che stavo
cercando di starmene da solo, ecco tutto.» Cercai di ripulire il diario
con una manica. Quando lei non si mosse o parlò, limitandosi a incombere
al limitare del mio campo visivo, lasciai cadere il diario in grembo ed
esplosi, esasperato: «È solo che... ho un sacco di cose a cui pensare.
Ti dispiace?» Il mare di notte era di quel colore scuro come il vino
descritto dal vecchio, cieco Omero, con riflessi di un bianco simile
alla neve quanto i capelli di Valka erano del pulito colore carminio
delle fiamme. Lei non si mosse, non se ne andò. Avrebbe potuto essere di
pietra, una delle guglie di basalto, se non fosse stato per la pressione
di quegli occhi dorati su di me. La pazienza è una grande insegnante e
il silenzio è ancora migliore... estorcono le cose dall'animo degli
uomini senza che ci sia la necessità dei coltelli. Quando il rumore
delle onde gentili che lambivano la riva si protrasse troppo a lungo
come unico suono, sbottai: «Riguardo a quello che ho visto nella grotta,
io...»

«Non siamo obbligati a parlarne» mi interruppe lei. «Abbiamo detto
entrambi cose orribili.»

«Entrambi...» Serrai i denti perché quella era una bugia e se non altro
questa volta ero innocente. L'occhio azzurro da avvoltoio di Gilliam mi
sbirciava però da sopra la sua spalla e mi fermai. «Come preferisci.»
Cercando una via di uscita cominciai a dire: «Riguardo ad Anaïs, io...»

«Hadrian, non mi importa.» Mi sedette accanto, e in qualche modo quel
semplice gesto ammorbidì l'impatto devastante delle sue parole. «Non so
di cosa ti vergogni tanto. È previsto che sposi quella ragazza. È una
cosa buona che tu l'abbia baciata, migliore di quello in cui può sperare
la maggior parte di voi palatini \emph{inmane}.»

Mi irrigidii per l'insulto. `Inumani.' Quel termine mi stordì, mi fece
sentire come immaginavo si sarebbe sentito un nonno nel venir definito
un bambino ignorante. «Migliore di...?» `Non so di cosa ti vergogni
tanto.' Come potevo spiegarlo? Distolsi il volto e allungai la mano
verso la bottiglia che avevo accanto, desiderando di poter svanire al
suo interno come una sorta di genio e dimenticarmi di tutto il mondo.

«Ecco, come i tuoi genitori.» Mi ero dimenticato di averle parlato di
loro. Ritrasse le ginocchia contro il mento con i talloni che scavavano
solchi nella sabbia nera. «Freddi. Sai cosa intendo. Questo è un bene, è
meglio. Lei è una brava ragazza.» Sentir definire così la figlia di un
conte era una novità che mi fece sorridere. «Potrebbe capitarti di molto
peggio, sai? Le piaci.» Mi assestò un pugno sul braccio in modo
stranamente giocoso. «Ed è anche splendida.»

Mi sfuggì di bocca un verso incoerente. «Non voglio sposarla» dissi.
Afferrai una manciata delle piccole pietre con cui stavo giocando e le
lanciai verso il mare. Caddero con un tonfo sul fango della battigia.
Era stato piacevole dirlo ad alta voce. «Non voglio rimanere bloccato su
questo pianeta. Ho ucciso un uomo, Valka, e presto cercheranno di
uccidermi. Mi riferisco alla Cappellania, alla grande priora. Questo
posto... tu sei il solo motivo per cui io...» Mi interruppi,
imbarazzato.

Non c'erano parole. Tacqui e mi limitai a \emph{guardare} il mare, il
gioco della luce rosea della luna sulle acque scure, le stelle che
ammiccavano nel cielo, le onde spinte dal vento e sottoposte alla
trazione di Binah e di Armand. Quella bellezza accese qualcosa dentro di
me, quanto bastava per escludere per un momento il caos urlante. Quanto
era fragile quella quiete. Il lambire delle onde, lo stridio lontano di
un qualche uccello notturno. Distanti e remote ma in qualche modo a
portata di mano, le luci delle navi in orbita e dei satelliti
tracciavano una silenziosa processione sullo sfondo delle stelle non
fisse. «Non avrei dovuto essere qui, Valka. Non doveva andare così.»
Liberai la bottiglia dalla sabbia e tolsi il tappo.

Valka me la tolse di mano prima che potessi bere e trangugiò lei stessa
un lungo sorso. «Sai, io volevo diventare una pilota.»

«Cosa?» recuperai la bottiglia. «Dici sul serio?»

«Assolutamente. Volevo comprare una nave e commerciare avanti e indietro
per il Fuoco Fatuo, magari trasportare passeggeri.»

«Cosa è successo?»

«Mio padre è morto» rispose, con lo sguardo fisso su un punto del mare o
del cielo che non avrei saputo indicare. Chinai la testa e mormorai
qualche parola di scusa. «È tutto a posto. Non lo sapevi.» Non pareva
per niente scossa, anche se si strinse maggiormente le ginocchia contro
il corpo.

«Come è morto?»

Valka si girò a guardarmi. «È stato ucciso, facendo proprio questo.»
Bevve un altro sorso, poi guardò in tralice la bottiglia con occhi
velati. «Questo vino non è buono come l'ultimo.»

«Elomas tiene la roba buona per sé.» Mentre parlavo cominciai ad
appuntire la matita rotta con il bisturi che tenevo nel mio kit da
disegno. Per un momento Valka parve allarmata, come se temesse che mi
potessi ferire, ma avevo le mani salde. «Scusami. Non aspettavo
compagnia.» Le mani mi si immobilizzarono in grembo, stringendo ancora
gli strumenti. «Anche tuo padre era uno xenologo?»

«Perché appuntisci la matita con un coltello?»

«Prego?» Mi girai a guardarla, confuso, e lei ripeté la domanda,
agitando una mano in direzione dei miei utensili. «Oh.» Sollevai la
matita perché la esaminasse, ammirandone la punta nera, così ben fatta.
«Crea una punta migliore.» Una volta Crispin mi aveva fatto quella
stessa domanda.

Sentii che mi fissava, per nulla divertita. «È assurdo. Fanno
temperamatite, lo sai?» Crispin non aveva detto la stessa cosa?

Potei soltanto scrollare le spalle, agitando il bisturi. «In realtà non
si tratta di questo. È... gli strumenti che usiamo ci aiutano a
modellare i nostri pensieri.»

«Cosa vuoi dire?»

«Quando non sto bene disegno.» Aprii il libro e ne sfogliai un paio di
pagine, lontane dai ritratti di Valka che avevo fatto, ciascuno scuro e
minuziosamente dettagliato, marcato da ombre profonde. «A volte me ne
sto seduto e fisso la pagina per un'eternità ma non vedo niente. Quando
succede, cerco di capire cosa è andato storto, perché non riesco a
farlo.» Riposi il bisturi nel kit che avevo accanto. «Mi prendo del
tempo per appuntire di nuovo la matita anche se non ne ha bisogno. È un
bene esercitarsi nei movimenti, mette a fuoco la mente e l'aiuta -- mi
aiuta -- a lavorare meglio.» Durante tutti quei discorsi sconclusionati
lei non aveva emesso un solo suono e non mi aveva interrotto per
deridermi, quindi aggiunsi: «Naturalmente, ci sono volte in cui l'arte
affiora senza sforzo.» Le sorrisi, tenendo una mano appoggiata sul
diario per paura che si aprisse -- come in qualche commedia -- a
rivelare il suo ritratto.

Lei urtò con i denti l'orlo della bottiglia e annuì. D'un tratto
consapevole di quanto quel gesto fosse sciocco, posò la bottiglia sulla
sabbia in mezzo a noi senza distogliere lo sguardo dalle acque scure del
mare. «Lo era. Uno xenologo. Mio padre, intendo. È entrato in conflitto
con la vostra Inquisizione mentre era a uno scavo su Ozymandias.»

«Non è la mia Inquisizione.»

Per parecchio tempo rimanemmo entrambi in silenzio, con il mormorio del
mare e i vaghi richiami degli uccelli che erano i soli suoni. Il vento
gemeva lungo la fenditura alle nostre spalle, solo e solitario. «Mi odi,
Valka?» chiesi infine.

«Lo fai già abbastanza da te.» Mi elargì un piccolo sorriso che trapassò
la coltre che mi avvolgeva, come inchiostro attraverso un tessuto. «Non
hai bisogno del mio aiuto.»

Una sorta di follia si impadronì di me e salì ribollendo da un qualche
punto nel profondo della gola. Scoppiai in una risata sommessa che però
fu troncata da un singhiozzo che mi costrinse a serrare la mascella e a
trattenere il respiro per impedire che la cosa peggiorasse. «Non so cosa
dire.»

«Non sei chi credevo che fossi» affermò, parole intrise di luce quanto i
suoi capelli nel chiarore lunare. Ci guardammo a vicenda e il suo
sorriso si accentuò.

Sentii il mio sorriso che si intensificava in risposta al suo e la
sommessa risata di poco prima che minacciava di riaffiorare. «Chi
credevi che fossi?» Non aveva bisogno di rispondere perché lo sapevo.

Valka mi guardò a lungo con quegli occhi dorati che nel buio brillavano
di luce propria. «Sono certa che lo puoi immaginare.»

Potevo farlo. Aveva pensato che io fossi Crispin, che fossi un
macellaio, un delinquente, che mi piacesse la violenza del nostro mondo
e che fossi un lupo fra i lupi, ma anche se il nostro Impero era una
landa selvaggia popolata da lupi io non mi ritenevo uno di essi.
Qualcosa di quel pensiero dovette affiorarmi sul volto perché lei
aggiunse: «Però non lo sei. Porti il tuo mantello imperiale come se ti
desse fastidio.»

«Lo fa.»

«Perché?» chiese. «Perché sei infelice? Questo posto... vogliono darti
tutto. Hai tutto. Sei un palatino e loro vogliono fare di te il
{consorte} di una ragazza che governerà un pianeta. Sai quanto questo
sia folle? Anaïs Mataro che governa un pianeta? O chiunque altro, se è
per questo.»

Ridemmo entrambi, lei dell'Impero, io di Anaïs, poi distolsi lo sguardo
e giocherellai di nuovo con i frammenti di pietra che avevo sulla sabbia
accanto a me. «Cosa ti fa pensare che qui abbia qualcosa di tanto
prezioso?»

«I tuoi privilegi, intendi? Preferiresti essere uno dei vostri
popolani?»

«Lo sono stato per anni» risposi in tono aspro, fissandola con occhi
roventi. «Ho vissuto nelle caditoie, Valka. Ho perso la mia... la mia
amica per colpa della dannata necrosi grigia. Per poco non sono morto
nel Colosso più volte di quante ne possa contare. Ho vissuto cose che
non puoi immaginare, quindi non mi fare prediche sui privilegi. So cosa
sono. Non l'ho scelto io, ma non pensare che non abbia sofferto per
questo, e rimanere qui non sarebbe un privilegio.» Non potevo continuare
a guardarla, non in quel momento, non con quello che stavo per dire. «Ma
Gilliam... lui è stato colpa mia. Espierò per questo e imploro il tuo
perdono. Ho agito in modo sbagliato, ma se pensi che essere costretto a
sposare Anaïs Mataro non sia per me una prigione solo perché è splendida
-- una parola tua, non mia -- allora non sai cosa sia una prigione.»

Con mia eterna sorpresa, lei non disse nulla e coprì il suo silenzio
bevendo un sorso di vino.

Un'espressione molto simile alla sofferenza le tese i muscoli sotto la
pelle pallida. «Anch'io volevo essere un pilota» aggiunsi goffamente,
dopo un momento. «Switch e io e alcuni degli altri -- i mirmidoni,
intendo -- volevamo comprare una nave e magari avviare un'attività
mercantile, o forse lavorare come mercenari, spostandoci di Colosso in
Colosso.» Raccolsi uno dei sassi e lo scagliai verso il mare, ma non lo
raggiunse. «Sarei stato come Simeon il Rosso, avrei viaggiato fra le
stelle, incontrato xenobiti, salvato principesse... non so.»

«Hai una visione molto romantica dell'universo» dichiarò. Lo intendeva
come un insulto ma rifiutai di interpretarlo come tale.

«Vorrei averla,» ribattei «e mi dispiace che l'universo non condivida
quest'aspirazione.»

Potevo avvertire quegli occhi innaturali puntati su un lato del mio
cranio ma non guardai verso di lei. «Sei sempre così drammatico?»

«Chiedilo a chiunque mi conosca.»

Valka sbuffò e mi passò la bottiglia di vino quasi vuota. «Per quel che
vale, dispiace anche a me.»

\begin{figure}
	\centering
	\def\svgwidth{\columnwidth}
	\scalebox{0.2}{\input{divisore.pdf_tex}}
\end{figure}

Un altro problema legato alla visione del tempo piena di buonsenso di
Agostino è che suppone una sorta di rapporto causale fra passato e
presente, fra presente e futuro. Forse è vero in senso fisico, ma nella
narrativa? No. Le storie non sono assoggettate al Tempo Sempre
Fuggevole, lo trascendono. Sono eterne. Nell'inglese classico, la parola
\emph{present} significa tanto `adesso' quanto `regalo'. Non capirò mai
come gli antichi siano sopravvissuti a una simile confusione, ma c'è
bellezza in simili stravaganze. Nel suo scorrere ogni momento è
prezioso, e così è separato da quelli che lo seguono e che lo precedono.

La verità? La verità è che non riesco a ricordare se abbiamo condiviso
quella bottiglia durante la nostra ultima notte a Calagah o se c'è stata
qualche altra conversazione in quella ultima notte che non riesco a
ricordare. Non importa. Nella mia memoria ci alzammo dalla spiaggia e ci
avviammo verso la fenditura proprio mentre il cielo si trasformava in
fiamme e tuoni.

Un grande bagliore riempì il cielo, rosso e bianco, incanalando ombre
profonde sul paesaggio roccioso, poi seguirono bagliori più piccoli,
azzurri come la luce del giorno. Come paralizzato, rimasi a fissare la
luce che sbiadiva mentre il fuoco descriveva scie sulla volta celeste.
\emph{Qualcosa} proiettò le sue fiamme contro le nuvole, trasformando la
notte in una parodia di tramonto i cui colori erano tutti sbagliati
anche per il rosso sole gigante di Emesh. In quella luce c'erano del
rosa e dell'azzurro -- i colori del plasma -- che cadevano come fulmini
attraverso il cielo.

Non avevo il tempo di ricordare i princìpi elementari della fisica. In
realtà ero così sconvolto e meravigliato da aver dimenticato ogni
raziocinio.

Il suono seguì presto la luce e il suo impatto mi scaraventò a terra.
Era come essere uno di quei profeti dell'antica mitologia, costretto in
ginocchio -- addirittura prono -- dalla voce possente di Dio. Era come
un tuono, ma più di un tuono, come se qualcuno {avesse} infranto il
cielo. Mi premetti le mani sulle orecchie e mi sentii gemere, ma non lo
udii fisicamente perché non era possibile udire nulla sotto quel fragore
spaventoso. La luce stava sbiadendo, e il suono insieme a essa,
lasciandomi nelle orecchie un fischio coperto da un ovattato rombo
sotterraneo, come se un altro pianeta stesse strisciando contro la
superficie del mondo.

«Alzati!» gridò qualcuno e qualcosa mi tirò per un braccio e una spalla,
aiutandomi a sollevarmi. Valka, era Valka. I tecnici si stavano
riversando fuori dalle loro case di plastica, alcuni in preda al panico,
altri che si muovevano a vuoto e fissavano il cielo senza capire.

«Una meteora?» gridò qualcuno, lì vicino.

«Impossibile» dichiarò un'altra voce.

«Una nave!» stridette un terzo. «Una delle nostre?»

«I Cielcin.» Persi il filo di chi stesse parlando, dei membri senza
faccia di quel coro greco che gridava che un esercito era alle porte.
«Sono i Cielcin!» La paura è una cosa strana, irrazionale, ma
incredibile nel modo in cui arriva alla verità più in fretta della
ragione.

Le orecchie mi fischiavano ancora e mi dolevano gli occhi a causa della
sfera di fuoco. In alto il cielo era solcato da strisce di luce, le
stelle erano perse in quella confusione, punti più sottili che
tremolavano nella cupola celeste, bianchi sullo sfondo del Buio. Visto
da terra era splendido, davvero splendido, una torre di fiamma e fumo di
un rosso intenso che precipitava sul mondo come una spada. Senza aver
bisogno che me lo dicessero compresi che il nostro coro aveva ragione,
che una nave stava precipitando, abbattuta da altre ancora più lontane,
e che quei più piccoli punti di luce nel cielo erano quelle dei
propulsori di velivoli più leggeri, piccole navi che non trasportavano
più di due uomini ciascuna e che venivano usate per proteggere e
bloccare lo spazio aereo di Emesh.

E compresi, con l'amara certezza della paura.

I Cielcin erano arrivati.

A quel punto trovai con facilità le parole. «Bel!» gridai, rivolto al
tecnico più vicino. «Corri a chiamare Elomas e digli di contattare
Fonteprofonda... ci serviranno dei velivoli. E soldati.» Una cosa tanto
piccola, e tuttavia quando ricordo quel giorno lo ricordo come un
momento orgoglioso e vitale, quello in cui avrei potuto cedere e
crollare a terra ma ho tenuto duro e ho agito.

Il tecnico, un uomo effemminato con gli zigomi alti e la carnagione
pallida di un extraplanetario, balbettò e si mostrò confuso. «Cosa?»

Il mondo tremò sotto i nostri piedi, punteggiato dallo schianto di un
tuono empio, stentoreo come soli che muoiono. Valka barcollò contro di
me ma l'afferrai e sostenni entrambi sul terreno sobbalzante.
«Dannazione...» Guardai verso est, dove si era abbattuta la colonna di
fuoco, una ferita nel cielo che toccava l'orizzonte. Scie di luce
arancione continuavano a precipitare, lingue di fiamma che seguivano la
traiettoria dei resti incendiati della nave appena caduta dal cielo per
andare a schiantarsi sul terreno roccioso. «Bel, vai! Trova il vecchio!»
Mi rivolsi a Valka. «Dobbiamo radunare tutti e farli scendere sulla
spiaggia, lontano da qui.»

«Lontano?»

«Il campo è visibile da miglia di distanza... è un bersaglio!» A
occidente il cielo era soffocato dal fumo, illuminato dal basso dalle
fiamme che ardevano ancora. Ricordando lo schianto, rimasi turbato dalle
scariche di luce azzurra che fendevano quel caos. Jet di assetto? Sì,
doveva trattarsi di questo. Per gli dèi, si erano diretti da questa
parte, ma era ovvio che lo avessero fatto. Avevano preso di mira il solo
continente presente su tutto Emesh e stavano scommettendo sul riuscire
poi ad andare via di qui a piedi. Cercai di ricordare i dettagli ma il
caos li aveva disintegrati tutti. «Poteva essere una delle nostre navi,
abbattuta.»

Valka era ferma da un lato e non si era mossa per dieci secondi
abbondanti. «Abbattuta forse, ma non era umana.» Non ho idea di come
facesse a determinarlo in mezzo al fumo e al rumore. «Credo che tu abbia
ragione, dobbiamo portare tutti via di qui.»


