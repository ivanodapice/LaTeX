\chapter{Verità senza bellezza}

Nell'arena del colosseo una squadra di quattro douleter lavorava con
aste stordenti per tenere sotto controllo l'azndarch mentre altri tre
lavoravano per sgombrare i resti degli schiavi che erano stati mandati a
combattere contro quella bestia extraplanetaria. Ne guardai uno lavorare
di pala per gettare segatura sulle macchie di sangue, dato che come
molte altre persone trovavo difficile guardare quel predatore alieno.
Nelle mie cellule c'era qualcosa, un ricordo radicato nel profondo di
come la vita avrebbe dovuto essere, che risaliva ai giorni in cui la
curva della Terra racchiudeva il nostro universo collettivo. E
l'azhdarch era semplicemente `sbagliato'. Sotto molti aspetti ricordava
uno pterosauro, quell'antico incrocio fra un pipistrello e una
lucertola, con le ali di cuoio. Sotto altri faceva pensare a un drago
uscito dalla fantasia, con la lunga coda coperta di aculei e gli artigli
ricurvi. Il collo, però, era lungo quasi il doppio di quello di un uomo,
ed era aperto dalle vestigia della testa fino al torace, pieno di denti
ricurvi e ringhianti che si aprivano e si chiudevano come le foglie a
mandibola di una pianta carnivora.

Vidi una coraggiosa douleter -- una giovane donna dai capelli rossi --
colpire con l'asta stordente e raggiungere la creatura su un fianco.
Essa emise un ululato gorgogliante e scattò di lato, trascinando gli
altri tre douleter mediante le catene che stringevano in mano. La folla
sussultò e applaudì mentre la bestia schizzava saliva dalla gola aperta.
Anche dalla sicurezza del palco nobiliare al di sopra della cortina
dello scudo riuscii a vedere il sangue al suo interno.

«I prossimi sono i Diavoli» disse Crispin, assestandomi un pugno sul
braccio. «Sei pronto?»

«Come non lo sarò mai.» Tornai al libro che avevo in grembo e mi sfregai
il lato della mano destra. Il carbone che vi si era raccolto aveva
cominciato a sbavare l'immagine, annebbiando il profilo del volto del
tenente Kyra che avevo disegnato. Crispin e io eravamo stati mandati a
presenziare al giorno di apertura della stagione del Colosso al posto di
nostro padre, che era ad Artemia con nostra nonna per discutere di
questioni di Stato.

Da bambino avevo odiato l'arena, la sua furia, il sangue e le
circostanze. La violenza mi offendeva, le grida e le urla mi
percuotevano le orecchie, come pure le trombe che squillavano dall'alto,
amplificate dal massiccio sistema sonoro del colosseo. L'odore di corpi
non lavati, misto a quello di carni artificiali alla griglia e al puzzo
metallico del sangue aggrediva il naso come le urla facevano con
l'udito.

Eppure la cosa che mi feriva di più era l'offesa contro la vita,
quell'insensibile spendere vite umane. I combattenti erano schiavi, lo
sapevo, e forse questo giustificava la violenza agli occhi di molti, ma
avevo appena visto tre uomini fatti a pezzi da un mostro volante
xenobita e bambini che strillavano di delizia e di terrore sulle
gradinate. Uomini con il torso nudo dipinto di rosso e oro o di rosso e
nero si assestavano pacche a vicenda e si rovesciavano addosso birra
economica, gridando e ridendo di fronte a quello spettacolo. La vista
del sangue mi nauseava più di quanto avessero fatto le notizie del
massacro a Cai Shen, perché qui c'era qualcosa di immediato e di
concreto. E le persone ne godevano.

Mi chiedo spesso cosa avrebbero pensato di noi gli antichi, della nostra
violenza. Ho sentito dire che quelle generazioni che avevano ucciso la
Vecchia Terra avevano deriso questo tipo di violenza nella loro vita di
tutti i giorni. Era ironico che quelle stesse persone che avevano
scatenato una guerra nucleare sul Mondo Natale, che avevano presieduto a
campi profughi e corrotto l'ecosfera si fossero tirate indietro di
fronte a uno sport sanguinario. Ci avrebbero definiti dei barbari,
quegli uomini dei tempi antichi.

Scurii con la matita la linea lungo il contorno del volto di Kyra. Basta
filosofeggiare. Adesso Crispin stava applaudendo. «Potrò scendere in
campo dopo il primo incontro.»

«Cos'hai detto?» Non sollevai lo sguardo dal disegno che avevo in grembo
e accentuai un ricciolo di capelli, con la mente su altre cose. Su Cai
Shen, su mio padre. Sullo stesso Crispin. Nella mente continuavo a
sentire le parole che mio padre mi aveva rivolto sotto la Cupola delle
Incisioni Radiose. `Chi ha mai detto che avresti governato dopo di me?'
Era destinato a Crispin tutto quanto. Sarei stato scartato, mandato via,
sposato a qualche barone o baronessa come un ornamento o costretto a
entrare nella Legione.

«Potrò scendere in campo.» Crispin sorrise, mostrandosi sinceramente
eccitato alla prospettiva. «Nostro padre ha detto che oggi potrò
combattere.»

«Oh.» Lo guardai per un momento. «Lo sapevo, continui a ripeterlo.»
Premetti la matita contro la pagina con tanta forza che il carbone si
spezzò, deturpando il naso sottile di Kyra. Imprecai interiormente,
rendendomi conto che ero risentito nei confronti di mio fratello. Già lo
odiavo, ma l'odio è qualcosa di puro, come un fuoco nel ventre, mentre
il risentimento mi divorava come un cancro. Non volevo quello che era
suo, ma ero risentito che avesse preso qualcosa che avevo implicitamente
compreso essere mio. Come ho detto, non desideravo conquistare il trono
di mio padre, volevo solo che Crispin lo perdesse.

«Nostro padre dice che posso combattere solo contro gli schiavi
mirmidoni, ma mi piacerebbe affrontare Marcoh, sai? Roban, potrei
abbattere Marcoh?» Crispin si alzò in piedi. Per l'occasione aveva
indossato l'armatura di titanio e ceramica accentuati da un rosso cupo.
Cuoio nero era stato teso sulla corazza modellata a imitare i muscoli, e
su di esso era goffrato il diavolo dei Marlowe, che beveva la luce come
fosse stato sangue. Sulla spalla sinistra portava un corto mantello di
ricco velluto, carminio dove intercettava la luce, nero dove non lo
faceva.

Il cavaliere dal volto rotondo si passò una mano sui riccioli compatti.
«Sono certo che potresti, giovane signore.»

«Preferisce quelle grosse spade... com'è che si chiamano?» Crispin bevve
un sorso da un bicchiere pieno di una qualche bevanda energetica azzurra
e schioccò le dita in direzione di Roban.

«Un montante» dissi. Tracciai il nome di Kyra in piccole lettere precise
nell'angolo inferiore sinistro del disegno, al di sopra della data,
usando una seconda matita prelevata dal kit di pelle che tenevo fra i
piedi.

«Proprio quella!» esclamò Crispin, con una sommessa risata gorgogliante,
prendendo un altro paio di olive dalla ciotola di porcellana posata sul
piccolo tavolo fra noi due. «Sono così lente.»

Roban non si mosse dal suo posto vicino alla porta. «Il giovane signore
ha proprio ragione.»

«Una spada corta e una mancina sono molto meglio» dichiarò Crispin,
piantando un piede sul tavolo e rovesciando la ciotola delle olive che
andò in pezzi sul pavimento, con le olive che rotolavano sulle
piastrelle. Lui le ignorò, come ignorò i servi che si precipitarono a
raccogliere le olive cadute e i pezzi di porcellana. «Sai chi hanno
scelto come avversario contro cui combattere?»

Sir Roban scrollò le spalle. «Alcuni degli schiavi, suppongo.»

«Più di uno?» I denti di Crispin brillarono nella fioca luce
sovrastante. Dava le spalle all'arena del colosseo e aveva il volto in
ombra.

«Forse, sire.» Roban fece rimbalzare l'elmo sotto il braccio destro.
«Non mi hanno informato, hanno lasciato tutta l'organizzazione al
vilicus del colosseo. Mi hanno solo detto che sarebbe venuto a cercarci
quando fossero stati pronti per te.»

Crispin si rimise a sedere e afferrò il drink con la mano guantata nel
protendersi in avanti sulla ringhiera. Nell'arena sottostante i Diavoli
di Meidua emersero da un ascensore sul lato destro in mezzo al tumulto
della folla e a squilli di trombe. Indossavano i colori avorio e
scarlatto dei legionari imperiali, il volto era una vuota maschera
corazzata del colore delle ossa e il loro nome era scritto in rosso
sulla schiena al di sopra del numero identificativo. Tom Marcoh era nel
centro, con il numero nove che gli spiccava enorme sulla schiena. Era un
uomo largo di spalle, con alcune strisce sulla ceramica della parte
superiore del braccio che lo contrassegnavano come un centurione, anche
se non era niente del genere. Era un attore, un soldato giocattolo che
non valeva niente al confronto dei veri soldati che avevo conosciuto.

«Era l'estate del `987!»

La voce dell'annunciatore riempì il colosseo, rimbalzando contro le
masse plaudenti con le loro bandiere, i segni dipinti a mano e le grida
di `Diavoli! Diavoli! Diavoli!' che quasi soffocavano i suoni
amplificati artificialmente. Fu sufficiente ad attirare perfino la mia
attenzione perché conoscevo quella data e sapevo cosa stavamo per
vedere.

«Gli ultimi uomini del 617° erano bloccati su Bellos, dove la loro nave
si era schiantata e i loro fratelli e sorelle erano stati uccisi. Non
stava arrivando nessuno a salvarli!»

A quelle parole l'ascensore all'estremità opposta della piattaforma
venne sollevato, riversando nell'arena trenta uomini e donne, tutti
schiavi. Erano tutti criminali, perché era pratica comune su Delos --
come nella maggior parte dell'Impero -- che i criminali venissero
costretti a una vita del genere. Venivano loro tagliate le narici per
contrassegnare i loro crimini e i loro reati venivano marchiati loro
sulla fronte. A qualcuno mancava una mano, ad altri un occhio o entrambe
le orecchie, tutti avevano la testa rasata e il corpo dipinto di bianco
per farli apparire più simili ai Cielcin. Anche se era impossibile dirlo
per via della tuta che indossavano, sapevo che i maschi erano stati
tutti castrati e che alle schiave erano stati rimossi i seni per
renderle più simili ai Cielcin -- che non erano né uomini né donne -- e
per demoralizzarli. Erano destinati a morire lì quel giorno, destinati a
rivivere la narrazione in cui i superstiti del 617° della Legione
Centaurina avevano respinto un'orda di Cielcin che aveva devastato la
colonia di Bellos.

I sette uomini dei Diavoli di Meidua erano dotati di scudo corporeo e di
armatura, gli schiavi non ne avevano. I Diavoli avevano fucili e lance
al plasma, settati a una bassa potenza in modo da provocare solo ustioni
superficiali, mentre gli schiavi erano armati di rozze lame d'acciaio e
di randelli, perché i Cielcin rifiutavano le armi da fuoco. Non era uno
scontro equo, ma del resto non era destinato a esserlo.

Quello era il Colosso, il grande evento sportivo dell'Impero, ed era una
cosa sanguinosa. Prima si eccitava il pubblico con l'azndarch, poi
questa mischia, una occasione per gli eroi conquistatori di far
ribollire il loro sangue, quindi seguivano gli incontri su scala più
ridotta, campione contro campione. E la prima di quelle battaglie
sarebbe stata quella di Crispin, il giovane e audace figlio del lord.
Risplendente nella sua bella armatura, valoroso con la sua lucente spada
e i capelli alla moda. In quel momento tutto l'insieme mi apparve
perverso. Forse gli antichi avevano ragione. Forse ne {aveva} Valka.
Forse siamo barbari. Volevo andarmene, volevo le mie stanze al Riposo
del Diavolo.

«Guardate i nostri nobili eroi circondati dai bestiali Pallidi!»
continuò l'annunciatore. Muniti delle aste stordenti, i douleter in
uniforme rossa pungolarono gli schiavi in modo da costringerli a formare
un cerchio intorno ai sette Diavoli di Meidua. «Guardate i nostri nobili
eroi, la sola cosa che si interponga fra la povera gente di Bellos e il
suo fato come cibo per i mostruosi Cielcin! Guardateli mentre resistono
eroicamente!»

Suonò un gong, riempiendo l'arena delle sue vibrazioni sonore, dolenti e
stranamente serene. Il suono di quel gong echeggia ancora dentro di me,
gettando un'ombra sul mio futuro. La folla urlò, estatica. Io volsi le
spalle, girando un'altra pagina del mio libro e rifacendo la punta alla
matita rotta con il piccolo bisturi che tenevo nel mio kit.

Le prime urla, quando il fuoco del plasma si riversò sugli schiavi che
attaccavano mi ustionò. Loro non potevano fare altro se non combattere,
perché i douleter disposti lungo il perimetro avevano lance
incandescenti che li avrebbero abbattuti in un istante, costringendoli a
combattere un altro giorno o uccidendoli dove erano caduti.

Rabbrividii.

Accanto a me, Crispin era fuori dalla sedia e brandiva la spada
sguainata, con la lama di ceramica che scintillava, affilata come un
rasoio sopra la testa degli spettatori che si trovavano sotto il nostro
palco. Gridava parole incoerenti insieme alla folla, spronato dalla
violenza. In quel momento pensai a sir Felix, alla rotonda per la
scherma e alla strigliata che avrebbe dato a Crispin per aver estratto
la spada così inutilmente. Io non avevo addosso nessuna arma a parte un
lungo coltello, e avevo detto a Roban che non mi serviva altra difesa se
non la sua presenza. Questo lo aveva rallegrato, ma io mi sentivo
sciocco e terribilmente piccolo, poco elegante rispetto a mio fratello,
armato e corazzato com'era.

Uno dei Diavoli di Meidua calò il piede sulla faccia di uno schiavo,
spezzandogli il naso. Sangue rosso gli corse lungo la guancia e il
mento, trascinando al suo passaggio scaglie di pittura bianca. Il
Diavolo calò nuovamente il piede e la folla sussultò per poi applaudire.
Lo stivale si sollevò e si stampò di nuovo sulla faccia dello schiavo,
che non si mosse. Era morto -- morto già da prima. Quella era
artisticità gratuita e priva di significato e non faceva per me. Era per
creature come mio fratello, come i servi e i plebei che ululavano sulle
gradinate con in mano il loro kebab o prelibatezze di zucchero filato,
le loro bevande dolci o la birra dozzinale.

«Giovane signore Crispin.» Un'aspra voce maschile risuonò dal fondo del
palco. Mi girai a guardare oltre il lato della sedia e rimasi sorpreso
di constatare che a parlare era stata una donna, tozza, con penetranti
occhi castani e i capelli che erano un groviglio di riccioli color
sabbia tagliato appena sopra le orecchie. Il suo brutto volto bruciato
dal vento era contorto in un sorriso grottesco. Allungai la mano ad
afferrare Crispin per il suo assurdo mantello corto.

Lui rovesciò la bevanda azzurra sul pavimento nel girarsi, sogghignando
alla vista di quella brutta donna. «È ora?»

«Sì, giovane signore.» Crispin se la fece praticamente addosso per
l'eccitazione, abbandonando il drink e dirigendosi quasi di corsa verso
la grassa douleter che era ferma appena oltre la porta aperta del palco.
I suoni della folla arrivarono attraverso quell'apertura, più nitidi e
netti, senza l'effetto ovattante degli scudi del palco che contribuivano
a bloccare il rumore.

«Vieni, signore Hadrian?» Sir Roban avanzò di un passo.

«No, Roban.» Volsi le spalle e sfregai una macchia di carbone sul taglio
della mano, riuscendo soltanto a sporcarmi il pollice sinistro.
Concentrai la mia attenzione sulla pagina che avevo davanti e non sul
lavoro sanguinoso in corso nell'arena. La verità era che sarei stato
felice di andare con Roban se il littore fosse stato diretto da
qualsiasi altra parte e non al padiglione per i gladiatori all'ingresso
dell'arena.

«Allora devo restare qui?»

«No, no. Crispin avrà più bisogno di te di quanto ne abbia io. Il palco
si può chiudere a chiave.»

«Sì, signore.»

Ero solo nel palco del lord e occupai il seggio destinato a mio padre,
mentre in basso sette uomini abbigliati come legionari imperiali
massacravano trenta prigionieri-schiavi con bruciatori al plasma e lance
a energia. L'odore di carne bruciata e di stoffa strinata cominciò a
levarsi dall'arena, mescolandosi a quello del kebab e dei popcorn che
arrivava dalle gradinate. Era un aroma inquietante e disgustoso.
Sfogliai le pagine del mio libro di schizzi, immagini di persone e di
posti intorno al castello.

Mi era piaciuto disegnare fin da quando ero bambino, ma nel crescere mi
ero reso conto che in quel processo c'era qualcosa di singolare. Una
fotografia può catturare i fatti dell'aspetto di un oggetto, i colori e
i dettagli resi alla perfezione con una risoluzione superiore a quella
che qualsiasi occhio umano possa riconoscere. Nello stesso modo, una
registrazione o un'iniezione mnemonica a rna potrebbe trasmettere un
argomento con perfetta chiarezza, ma così come un'attenta lettura
permette al lettore di assorbire e di sintetizzare la verità di quello
che legge, disegnare permette all'artista di catturare l'anima di una
cosa.

L'artista vede le cose non nei termini di quello che sono o che
potrebbero essere, ma in quelli di ciò che \emph{devono} essere, di
quello che il nostro mondo deve \emph{diventare}. È per questo che agli
occhi di un osservatore umano un ritratto sconfiggerà sempre una
fotografia. Questo è il motivo per cui ci rivolgiamo alla religione
anche quando la scienza ha da obiettare e perché il più infimo degli
scoliasti può dare dei punti a una macchina. La fotografia cattura la
Creazione così com'è, cattura i fatti, e ora che sono vecchio i fatti mi
annoiano. È la \emph{verità} che mi interessa, e la verità risiede nel
carbone; o nel vermiglio, grazie alle cui proprietà sto scrivendo questo
resoconto. Non è nei dati o nella luce laser. La verità non risiede
nella ripetizione mnemonica ma nelle piccole e sottili imperfezioni,
negli errori che definiscono tanto l'arte quanto l'umanità.

Come ha scritto il poeta, la bellezza è verità. Verità, bellezza.

Si sbagliava. \emph{Non} sono la stessa cosa.

Non c'era bellezza in quell'arena, ma c'era verità. Là, mentre uomini
urlavano e morivano nell'arena, giustiziati per divertire settantamila
spettatori, io la vidi. O meglio, la sentii; la sentii dietro alle urla
e agli applausi e alle risate del pubblico adorante, mentre Crispin
avanzava nell'arena in mezzo al fumo e douleter e servitori trascinavano
verso l'ascensore i corpi degli schiavi morti. Era un silenzio profondo
ed echeggiante, non la quiete nelle orecchie, ma nella mente. La folla −
che adesso era un singolo essere -- stava urlando per soffocare lo
stentoreo silenzio nella sua anima.

Lo sentii, ma non compresi cosa fosse. Cosa significasse.

Abbottonando la giacca mi girai, mi diressi alla porta, lasciando il
palco. Avevo bisogno di aria. Scoprii subito che non sopportavo più di
rimanere a guardare la scena. Quello non era il mio mondo, non era
qualcosa che desiderassi ereditare insieme al resto. Mentre lasciavo il
palco i popolani applaudirono Crispin.

Poteva tenersi quegli applausi.