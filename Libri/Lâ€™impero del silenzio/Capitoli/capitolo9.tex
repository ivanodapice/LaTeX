\chapter{Pane e giochi circensi}

Trascorse una settimana prima che l'apparecchio correttivo venisse
rimosso dalle mie costole, e anche se avevo ancora la mano intrappolata
nel suo congegno c'era un ordine a cui dovevo obbedire, quindi trascinai
la mia fragile persona fuori dal letto. Con una sola mano mi risultò
troppo difficile mettermi le scarpe, e siccome non volevo ordinare a una
cameriera di farlo per me uscii senza, prendendo l'elevatore che dalle
mie camere portava al pianterreno della Grande Rocca. Ero stato accudito
a sufficienza, e comunque volevo che mio padre vedesse le mie
condizioni. Scesi con una serie di tramvie sotterranee e risalii negli
uffici del campidoglio della prefettura, un edificio triangolare vicino
al barbacane del castello, incoronato da una cupola centrale e da tre
torri squadrate modellate come le torri campanarie di antiche
cattedrali. A piedi scalzi, attirando sguardi e sussurri da parte dei
logoteti in uniforme grigia dei diversi ministeri, attraversai il grande
sigillo dei Marlowe sul pavimento della rotonda e mi diressi verso la
più alta delle tre torri d'angolo.

L'ufficio di mio padre era sulla cima, e vi si accedeva attraverso una
porta di metallo circolare, sorvegliata da sir Roban e da una decuria di
opliti muniti di lancia, uomini e donne senza volto in un'armatura di
ceramica di un nero liquido coperta da un mantello rosso. Il
cavaliere-littore sorrise. «Giovane signore! È bello rivederti in
piedi.» Venne avanti. «Sei qui per vedere tuo padre?»

Annuii, intorpidito, esausto e consapevole dello scomodo peso del
dispositivo medico correttivo fissato nella mia carne. Nascosi quella
cosa ingombrante dietro la schiena e repressi un sussulto quando
l'esoscheletro mi urtò contro il fianco. «Mi è dato di capire che devo
ringraziare te per la mia vita, signore.»

Roban accantonò la cosa con la consueta malagrazia. «Stavo solo facendo
il mio lavoro, sire.»

Non essendo tipo da lasciarmi respingere, posai la mano sana sulla
spalla corazzata di Roban -- tanto per sostenermi quanto per
ringraziarlo -- e gracchiai: «Comunque sia, signore, mi hai salvato la
vita.» Chinai la testa. «Grazie.»

«Tuo padre ti stava aspettando, mio signore» si limitò a replicare
Roban, non sapendo forse come accettare una tale gratitudine da parte di
uno dei suoi padroni palatini. O forse la vista dell'apparecchio
correttivo sulla mia mano gli diede fastidio. Lui era un patrizio di
nobile nascita, anche se non elevata quanto la nostra. I suoi
potenziamenti genetici erano più rozzi, cose da mercato secondario
applicate dopo la sua ascesa al cavalierato. Era un mutante, come molti
della sua casta. «È dentro.»

Ritrassi la mano con un sospiro. «Benissimo.» Raddrizzai la schiena.
«\emph{Morituri te salutant}, eh?» Feci scorrere lo sguardo sulla
decuria di opliti muniti di scintillanti lance a energia. `Noi che
stiamo per morire ti salutiamo.'

«Sire?»

«È latino, Roban» dissi mentre avanzavo zoppicando, con i miei spessi
calli che sfregavano contro il mosaico del pavimento dell'anticamera.
Non gli fornii la traduzione. «Potresti aprire la porta? Io... ecco...»
Sollevai la mano ferita, notando di nuovo i piccoli punti di sangue
secco incrostati dove i sottilissimi aghi mi avevano trapassato la
carne.

«Sì, giovane signore.» Si protese in avanti e le piastre articolate
della sua armatura si aprirono per mettere a nudo il palmo che lui
premette su un emisfero trasparente nel centro della porta. Il congegno
scannerizzò le vene all'interno della mano del cavaliere e i pesanti
catenacci si aprirono rumorosamente, poi la porta scivolò di lato dentro
una fessura nella parete e Roban gridò verso l'interno: «Hadrian è qui
per vederti, mio signore.»

«Fallo entrare» rispose dall'interno la voce bassa di mio padre. Era
strano che non si fosse rivolto a me, anche se ero a portata di udito,
ma del resto non sollevò neppure lo sguardo o lo distolse dalle
costellazioni di diagnostiche olografiche che lo {avviluppavano} dove
sedeva alla scrivania simile a un monolite. Passai dal mosaico ai
tappeti tavrosiani spessi due pollici. La poltrona ad alto schienale di
mio padre era incorniciata da un'enorme finestra rotonda che si
affacciava su Meidua e sull'arco del porto marittimo. Verso sud il cielo
era solcato dalla scia di condensazione dei razzi che trasportavano i
carichi in orbita e al di là di essa. Le due pareti laterali erano
rivestite di scaffali di libri, ma laddove quelli di Gibson erano
strapieni, descritti dal caos di un lungo utilizzo e di un'attenzione
amorevole, quelli di mio padre erano ordinati e -- sospettai -- privi di
polvere solo grazie all'esercito di servitori che ripuliva il palazzo.

Mi fermai quasi nel centro esatto della stanza quadrata, al limitare dei
raggi di luce inclinati, contraendo le dita dei piedi sul tappeto. Come
un penitente davanti all'altare di un dio geloso, attesi a testa china.

Finalmente si accorse di me e posò lo stilo di tungsteno sul piano di
vetro nero, cancellando gli ologrammi con un gesto della mano. Con la
luce alle sue spalle, era seduto nell'ombra. Dopo quelli che parvero
eoni disse soltanto: «Siediti.»

Esitai per un momento, non più di un paio di secondi, ma mio padre si
limitò a guardarmi immobile, senza parlare, e alla fine cedetti,
lasciandomi cadere su una bassa poltrona dallo schienale rotondo di
fronte alla sua grande e vecchia seduta di pelle rossa e ottone. Il
silenzio pervase un singolo, acido momento fra di noi, che estese il
tempo nelle sue dita insensibili. Attesi. La pazienza è un tratto comune
fra pari dell'Impero -- e in realtà fra i nobili di qualsiasi casta --
ma mentre io avevo a disposizione tutto il giorno per la mia
convalescenza, mio padre aveva senza dubbio riservato solo una piccola
quantità di tempo al nostro appuntamento. Io potevo concedermi il lusso
di essere paziente, lui no.

«Perché eri in città?» chiese infine.

«Cosa?» Sentii le spalle che mi si irrigidivano, come in previsione di
un colpo. «Niente `Come ti senti, Hadrian? Stai bene?'»

«È ovvio che stai bene, ragazzo.» I capelli grigi sulle sue tempie e
sulla frangia scintillavano argentei nell'alta luce solare. «È per
questo che ho rimandato questa conversazione. Era inutile parlare finché
non fossi stato bene.»

«Avresti potuto venire a trovarmi.»

Mio padre sbuffò. «Non hai risposto alla mia domanda.»

«Stavo tornando a casa a piedi.» A metà della mia risposta lo sguardo di
lord Alistair si spostò dalla mia faccia al fine globo smaltato del
pianeta sull'angolo di estrema destra della sua scrivania. Era il genere
di oggetto per ricchi collezionisti, che si illuminava in tempo reale
con la rotazione del mondo e le giornate di ventisei ore di Delos, con
il sistema climatico e i banchi di nubi realizzati con squisiti
ologrammi.

«Tornavi a casa a piedi.» In qualche modo ogni parola era un manifesto,
una condanna. Nella mia condizione alterata dai medicinali faticai a non
ritrarmi davanti a lui. Non alzò la voce, una cosa che non faceva quasi
mai e che lo rendeva così terrificante. «Lo sai quanto ci sei costato
con questa tua piccola disavventura?»

«Disavventura?» La voce mi si incrinò mente scivolavo in avanti sulla
poltrona, con la veste che mi si apriva sul petto mentre ripetevo:
«Disavventura? Sono stato aggredito, signore.»

Il signore del Riposo del Diavolo tamburellò con le dita sulla
scrivania, smuovendo alcuni fogli di pergamena formale, che supposi
essere contratti, se non ordini di fedeltà. I documenti veramente
importanti venivano ancora stilati a mano. «I colpevoli sono stati presi
e consegnati alla Cappellania per essere puniti.»

Esposi al suo esame la mano sinistra, dove l'anello con il sigillo era
di nuovo sul pollice. «L'avevo immaginato. Avete tagliato il dito a quel
povero bastardo?»

Mio padre sorrise. «Il momento in cui un popolano può fare del male a
uno di noi è il momento in cui il nostro Casato cessa di essere temuto.
Non è come nei tempi antichi, figlio. Noi governiamo \emph{di persona},
non tramite organismi politici, non nel nome o con il consenso dei
governati. Il nostro potere è \emph{nostro}, lo capisci? E quel potere è
\emph{nostro} solo finché riusciamo a conservarlo.»

«Obbedienza mediante la paura del dolore» ribattei, ricordando le parole
di Gibson.

«Quello è il solo modo in cui un uomo può controllare un cane, ragazzo.»
Si appoggiò all'indietro con uno scricchiolio sulla pelle rossa.
Prendendo esempio da lui, spinsi lo sguardo oltre la sua figura per
osservare il distante traffico a terra della città sottostante la nostra
acropoli e le vele bianche nel porto.

Un senso di malessere mi si formò alla bocca dello stomaco mentre
chiedevo: «Che cosa avete fatto?» Mi protesi, aspettandomi di vedere un
isolato cittadino che bruciava o un cratere di vetro annerito dove c'era
stato il quartiere suburbano colpevole.

«Abbiamo imposto il coprifuoco al distretto e sparato a chiunque lo
violasse.»

«Non puoi farlo!» protestai, scuotendo il capo. «Peggiorerà soltanto le
cose.»

«Non hai ancora risposto alla mia domanda.» Lord Alistair riportò per un
breve momento lo sguardo sul mio volto. Prima che potessi mettere
insieme una domanda chiarificatrice, aggiunse: «Perché hai lasciato il
Colosso?»

«Non mi andava a genio lo spargimento di sangue» ringhiai, sovrastando
in parte la sua voce e levando gli occhi al cielo.

«Non ti andava a genio?» Sua signoria sogghignò visibilmente, con le
labbra che si incurvavano a esporre i denti inferiori. «Non ti andava a
genio? E ti chiedi perché ci sia un interrogativo su chi mi debba
succedere, se tu o Crispin?» Arrestò la rotazione del globo con un colpo
della mano.

«Perché non voglio starmene seduto a guardare durante il Colosso?»

«Perché la gente \emph{ha visto} che non sei rimasto seduto a guardare
durante il Colosso. Quel palco non è opaco, bambino idiota.» Stavo per
ribattere, ma mio padre sollevò una mano per impormi di tacere. «Mentre
eri impegnato a cercare di andartene, tuo fratello stava affrontando
senza cedere terreno i migliori dei miei gladiatori, resisteva per il
suo popolo.» Calò con violenza una mano sul piano della scrivania. «E
hai idea di cosa comunichi alla gente il tuo essertene andato? Sai quale
messaggio trasmette? Insieme al tuo essere stato ferito da
\emph{popolani}?» Scosse il capo con le labbra incurvate, come se
quell'ultima parola gli avesse lasciato in bocca un cattivo sapore.
«Dopo questo non avranno paura di te!»

«E hanno paura di Crispin?»

«E hanno paura di Crispin» ripeté lord Alistair, tamburellando di nuovo
con le dita sul piano della scrivania. «Avresti dovuto esserci tu in
quel colosseo, ragazzo. Felix mi dice che sei il combattente migliore.»
Cercai di contenere la mia sorpresa di fronte a quell'affermazione. Era
vero, ma non mi sarei mai aspettato di sentirlo dire da lord Alistair.
Con il tono cupo di un magistrato domandò: «Capisci che cosa hai fatto?»

«Mi sono fatto pestare.»

«Dimentica il pestaggio.» Mio padre agitò una mano e si appoggiò allo
schienale come un inquisitore della Cappellania che stesse per emettere
un giudizio.

\emph{Dimentica il pestaggio}. Abbassai lo sguardo sulla mano
imprigionata nel supporto correttivo. \emph{Se solo fosse stato tanto
	facile.} «Al momento ho qualche difficoltà al riguardo» risposi,
inarcando le sopracciglia.

«Taci.» Mio padre si protese in avanti così di scatto da strapparmi un
sussulto che mi scatenò uno spasmo di dolore lungo il torso. «La gente
ama quei giochi. Il Colosso. Adesso tu li hai disprezzati pubblicamente.
Era sul notiziario di Meidua, lo sai? È stato trasmesso per quattro o
cinque ore prima che tor Alcuin e io potessimo bloccarlo.» I suoi occhi
freddi si ridussero a mere fessure e fece ruotare il sedile verso la
finestra.

Mi stava trattando come un bambino, e forse lo meritavo. Quello che
diceva aveva senso e avrei dovuto rendermene conto... lo avrei fatto se
non fosse stato per il dolore al braccio. Mi parve comunque poco saggio
parlare, quindi rimasi in silenzio, guardando mio padre contemplare il
suo dominio. «Pane e giochi circensi, ragazzo.»

«Prego?» riconoscevo quella citazione, che era molto antica. Risaliva a
Giovenale, un uomo morto da così tanto che credevo soltanto gli
scoliasti si ricordassero di lui o della sua specie. D'altronde ero uno
sciocco che appena pochi minuti prima aveva parlato in latino agli
uomini di mio padre, e lui aveva dentro di sé profondità segrete.

«La legge imperiale proibisce ai servi vincolati a un pianeta di
manovrare qualcosa di più complesso di un veicolo di terra, a parte
qualsiasi cosa sia necessaria alle diverse Gilde. Sono fatti per
occuparsi di bestiame e motori a combustione, e sai il perché?»

«Perché si potrebbero ribellare?»

«Perché potrebbero cominciare a pensare di avere il diritto di farlo.»

«Prego?» Quell'affermazione mi offese, era un colpo contro la comune
decenza di cui non credevo avrei potuto risentire.

Mio padre non si girò, non si prese la briga di rilevare il senso di
shock o l'indignazione che trasparivano dal mio tono. «Guarda gli
Eudoriani, le Proprietà Normanne, gli Extrasolari. Sai che cos'hanno in
comune?» Prima che potessi rispondere, mio padre calò di nuovo la mano
sul telaio della finestra rotonda. «Non hanno capi. Nessun ordine.
L'Impero è ordine. Noi.» A quel punto si girò e si premette sul petto
esile una mano adorna di anelli. «È lo stesso con i principi jaddiani,
con i Lothriani: ordine. Senza quello, la civiltà su scala galattica
sarebbe impossibile. Si frantumerebbe.»

«Gli Eudoriani se la cavano benissimo» obiettai, pensando ai carovanieri
nomadi con la loro rete di stazioni su asteroidi sparsa per tutto lo
spazio umano. «E anche le Proprietà.»

«Ma per favore.» Lord Alistair sogghignò. «Quegli endogamici non
riescono a tenere insieme un singolo pianeta, tantomeno un migliaio.» E
con un cenno impaziente della mano cancellò miliardi di vite umane dalla
nostra conversazione come si potrebbe allontanare una mosca. «Sai che
alcuni dei mondi delle Proprietà hanno \emph{nazioni}? Stati come quelli
che esistevano prima dell'Esodo? Alcune di quelle piccole colonie non
sono neppure in grado di costruire astronavi! E combattono fra loro
nella stessa misura in cui combattono contro chiunque altro.»

Scrollai le spalle. «E noi no?»

«Le regole della \emph{poine} hanno i loro ammiratori nell'Impero, te lo
concedo, ma la Cappellania regola le nostre azioni e minimizza i danni
collaterali.»

«Vuoi dire che minaccia i nobili dissidenti con armi biologiche. Ma
questo cosa c'entra con i giochi circensi?»

L'arconte di Meidua protese in fuori il mento. «Noi non siamo come
quelle altre nazioni, figlio, qui non c'è un congresso, un corpo
politico. Quando emetto un decreto sono \emph{io} a crearlo, di persona,
senza sostituti o ripieghi. I vecchi sistemi della democrazia e del
parlamento permettevano soltanto ai codardi di nascondersi. Il nostro
potere non dipende dal consenso del popolo ma dalla sua fiducia in noi.»

«So tutto questo» replicai, agitandomi sulla poltrona. Dilatai le
narici. Non avevo perdonato quell'uomo per avermi abbandonato alle mie
ferite. Era mio padre, nel nome della Terra. Mio padre. E mi stava
facendo la predica perché ero stato brutalizzato. Tuttavia, aveva
ragione, io non ero \emph{soltanto} un ragazzo, ero suo figlio, e avevo
la responsabilità di portare su di me il peso del mio Casato. In quella
responsabilità c'era un potere, e anche un obbligo. Era per questo che
un lord era meglio di un parlamento, perché un lord non aveva scusanti.
Se abusava del suo potere, come temevo potesse fare Crispin, non avrebbe
governato a lungo. Se era freddo nell'esercitare il suo potere, come
sapeva esserlo mio padre, non avrebbe governato con facilità.

«No, non lo sai» scattò lui, spingendosi una ciocca di capelli ricciuti
dietro l'orecchio. «Dobbiamo relazionarci con i popolani, ragazzo,
mostrare che siamo persone e non un qualche astratto \emph{concetto}
politico. Questo è ciò che capiscono, ed è per questo che ho mandato te
e Crispin al Colosso mentre trattavo con Elmira. Sono il patriarca del
popolo di Meidua e voi due siete stati mandati entrambi a rappresentare
me e il nostro Casato. Di persona. Crispin ha svolto il suo ruolo in
modo ammirevole: adesso il popolo lo ama perché lo ha visto come parte
del suo mondo. Ha combattuto nel \emph{suo} Colosso, mentre tu... tu gli
hai voltato le spalle.»

Mentre parlava, mio padre tirò fuori dalla manica un piccolo cristallo
di stoccaggio lungo circa quattro pollici e largo mezzo, rigirandolo fra
le dita come un cieco che cercava di capire cosa fosse, come se stesse
soppesando un hurasam d'oro per determinare se era contraffatto.
«Sarebbe già stato abbastanza grave che te ne fossi semplicemente
andato, ma sei anche riuscito a farti ferire. Il nostro potere è
intensificato dal fatto che la popolazione comprende che siamo al di
sopra di essa, e tu hai danneggiato quella comprensione.»

«Sanguinando?» Non riuscii a evitare che l'incredulità mi trasparisse
dalla voce.

«Sì.» Mio padre sbatté il cristallo sul tavolo e si rimise a sedere.
Potevo vedere con chiarezza il suo stemma sul cristallo, il diavolo
rosso che scintillava contro il suo sfondo nero lavorato a sbalzo sul
cristallo bluastro.

«Credevo dovessimo dimostrare che siamo uomini e non fantasmi.»

«Dobbiamo mostrare di non essere astrazioni,» mi corresse «che siamo
poteri tangibili, non che siamo umani.»

Nello scomparso Egitto ci si aspettava che i faraoni si comportassero
come dèi: placidi e imparziali, al di sopra della bramosia e delle
preoccupazioni della vita mortale. Quando un faraone non riusciva a
essere all'altezza di quelle aspettative rivelava la sua mortalità ai
suoi sudditi e così facendo invitava rappresaglie da parte di coloro che
avevano servito e adorato la sua divinità. Noi non eravamo molto
diversi, nessun lord lo era. Con le nostre modifiche genetiche
personalizzate e la durata prolungata della vita eravamo come dèi in
modi che quei faraoni da tempo scomparsi non avrebbero mai potuto
sognare. Anche se era un mero arconte, un governatore {provinciale}, il
dominio di mio padre era più grande delle dimensioni della storica
Europa e la remota parentela della nostra famiglia con quella imperiale
serviva solo a elevarci maggiormente. La madre di mia madre governava
non solo i palatinati planetari -- un ducato -- ma anche una provincia
interstellare. Era la viceregina di tutte le stelle della provincia di
Auriga − circa quattrocento in tutto − e faceva rapporto direttamente al
Trono Solare e all'imperatore, che era un suo lontano cugino. Tramite
mia madre, anche io ero un cugino di Sua radiosità e in linea per
l'ascesa al trono, anche se a parecchie migliaia di posizioni di
distanza da esso. Mio padre aveva a sua volta sangue reale, ma più
remoto, dato che erano passate parecchie generazioni dall'ultima volta
che qualcuno del Casato Marlowe si era sposato all'interno della
costellazione del Casato Imperiale. Nonostante il nostro sangue antico e
la nostra considerevole ricchezza -- fonte di invidia per molti Casati
più recenti e più potenti -- solo un miliardo scarso di persone doveva
fedeltà a mio padre o era di sua proprietà: servi vincolati ai pianeti,
artigiani e schiavi.

«Intendo mandarti al college Lorica, su Vesperad, perché entri in
seminario.»

«No!» A quel punto scattai in piedi, rovesciando a terra la piccola
poltrona delicata con un tonfo sordo. «No, non puoi farlo!»

Lord Alistair Marlowe mi guardò mostrando solo una vaga sorpresa,
sinceramente perplesso. «Credevo che questo ti avrebbe fatto piacere. La
tua facilità con le lingue sarebbe molto utile là. La Cappellania ha
sempre bisogno di nuovi ambasciatori.»

«Di nuovi missionari, vuoi dire.» Riuscii a stento a contenere un
sogghigno. Sapevo cos'era la Cappellania e la disprezzavo. Non era una
vera religione, come quella degli adoratori che mantenevano ancora in
vita gli antichi dèi, era solo il pugno nel guanto di velluto imperiale,
unto con olio consacrato, un cinico atteggiamento di fede, con preghiere
memorabili ma vuote che grondavano una immeritata tradizione. Era uno
strumento di terrore e di sacra soggezione, il più grande circo sotto il
sole. \emph{Obbedienza tramite la fede}. Una certa dose di sofferenza
sarà sempre parte dell'universo umano, ma io chiamo simili terrori con
il loro nome e non li amo.

«Parole, parole» mormorò distrattamente l'arconte.

Il sottinteso più grande della sua decisione si riversò su di me. «Mi
stai diseredando?»

Lui si incupì in volto e contrasse le sopracciglia, gettando un'ombra
sugli occhi viola. «Non ti ho mai dichiarato mio erede.»

«Ma sono il figlio maggiore!» protestai. Ero incapace di chinarmi per
raddrizzare la poltrona caduta perché perfino il semplice atto di stare
in piedi mi scatenava uno spasmo orribile nel torace, e immaginai
fratture simili a ragnatele che modificavano le mie ossa risaldate,
ancora fragili per il trattamento di cellule staminali. Sapevo che la
mia era una debole obiezione e che l'ordine di nascita significava poco
nell'Impero, meno del decreto di un lord. «Crispin...» Non riuscii a
proferire le parole. «Crispin...»

Mio padre le trovò per me. «Rimarrà qui al mio fianco e a suo tempo
prenderà il mio posto, a patto che continui a dimostrare quello che
vale.»

Per poco non mi strozzai per quel qualcosa che avevo in gola e che non
voleva identificarsi come una risata o un singhiozzo. «Dimostrare quello
che vale? Percuotendo un'altra serva? Oppure uccidendo un altro eunuco
nel colosseo? Nella sua testa non ci sono due neuroni collegati!» Adesso
ero addossato al bordo della scrivania, fissando dall'alto mio padre
seduto al suo posto. Lui si alzò come la scarica di un tuono e mi sferrò
un potente schiaffo in piena faccia. Stordito, già incerto sulle gambe,
barcollai e caddi in ginocchio, poi cercai di rimettermi in piedi. Nella
fretta e nella confusione usai la mano ferita e anche se quell'arnese
infernale mi impedì di flettere le dita, la semplice pressione
esercitata su tutti gli aghi che avevo nella carne scatenò un'ondata di
dolore su per il braccio. Ululai, e quasi mi aspettai che sir Roban
venisse a vedere qual era la causa di quel trambusto, ma non arrivò
nessuno.

«Crispin è tuo fratello. Non ti permetto di parlare di lui in quel
modo.»

Invece di rispondere mi rimisi in piedi, facendo appello a ogni ultimo
brandello della mia dignità. «Non voglio diventare un prete, padre.»

«Ti rivolgerai a me chiamandomi `sire', o `mio signore'» ribatté mio
padre, aggirando la scrivania come una pantera che bracca una preda. Non
avendo alternative, eseguii il mio inchino più profondo, quello
destinato a un lord planetario. Una vendetta meschina, dato che lui non
lo era.

«Voglio diventare uno scoliasta» dichiarai, raddrizzandomi.

Il secondo colpo mi raggiunse all'altra guancia, ma me lo aspettavo e mi
girai con l'impatto, e questa volta riuscii a rimanere in piedi. «È
davvero quello che vuoi? Essere una macchina calcolatrice per qualche
barone di un mondo di confine?»

«Voglio entrare nei Corpi Esplorativi, viaggiare fra le stelle come
Simeon il Rosso» replicai, usando il braccio sano per sorreggermi contro
la sua scrivania.

«Simeon il...» ripeté mio padre, poi si interruppe e rise. E perché no?
Adesso quello sembra anche a me un sogno infantile. Cambiò tattica,
tornando a usare la logica. «La Cappellania esercita un vero potere,
Hadrian. Potresti diventare un inquisitore, magari perfino uno del
Sinodo.» Con la mascella serrata, tanto che le labbra si muovevano a
stento, aggiunse: «Abbiamo bisogno di qualcuno nella Cappellania,
ragazzo. Qualcuno che sia dalla nostra parte.»

Una sorta di vuoto risucchiante mi si formò alla bocca dello stomaco.
Per il Buio, pensai. «Hai pianificato tutto questo?» Scossi il capo.
«Non lo farò.»

Mio padre era più alto di me di più di tutta la testa, ed era ad appena
un pollice da me, guardandomi dall'alto in basso con gli occhi
socchiusi. «Lo farai.» Mi mise in mano il cristallo. «Partirai per
Vesperad alla fine di Boedromion.» Si riferiva al mese locale che
segnava l'inizio dell'autunno.

«Ma è fra soli tre mesi!» protestai, temendo un altro schiaffo.

«Questo incidente ha accelerato i nostri piani. Dobbiamo allontanarti
dall'attenzione pubblica prima che tu mi possa causare ulteriore
imbarazzo.»

«Imbarazzo!» Avrei potuto urlare. «Padre, io...»

«Basta così!» Per la prima volta dall'inizio della conversazione alzò il
tono di voce, dilatando le narici. «La cosa è decisa!» Ripiegò le labbra
in un'espressione di disprezzo nel guardare il congegno correttivo che
avevo sulla mano. «Vattene di qui, prima di causarti ulteriori danni.»

Riuscii a stento a non urlare, a non ululargli in faccia, a non prendere
la poltroncina che avevo rovesciato per fracassarla sulla scultura greca
del suo volto. Trassi un profondo respiro -- quanto più me lo
permettevano le costole danneggiate -- poi mi ersi in tutta la mia
insignificante statura e girai sui tacchi.

