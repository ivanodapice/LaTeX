\chapter{Come un tuono lontano}

Le strette finestre dell'alloggio di Gibson erano aperte e si
affacciavano dall'alto di dodici piani su un cortile interno dove i
servitori si prendevano cura della vegetazione del giardino roccioso. La
luce bianca del sole scendeva da un cielo di un colore biancastro,
mettendo in evidenza il disordine nello studio di Gibson. Le pareti
erano occupate da scaffali di libri talmente pieni che le carte cadevano
sul pavimento come neve in mezzo a pile di altri libri. Alcuni scaffali
contenevano rastrelliere per conservare i cristalli e le bobine di
microfilm, che però erano numericamente inferiori nella misura di cento
a uno rispetto ai libri.

Lo scoliasta stava leggendo.

Le ingiunzioni tecnologiche presentate contro il loro ordine per eresie
di molto tempo prima proibivano agli scoliasti libero accesso anche alle
limitate tecnologie permesse alle Casate imperiali dalla Sacra
Cappellania della Terra. Erano permesse loro solo le attività della
mente e quindi i libri -- che sono per i pensieri ciò che l'ambra è per
la mosca prigioniera al suo interno -- erano il loro più grande tesoro.
E così Gibson, un vecchio curvo sulla sua poltrona appiattita, viveva
assorbendo la luce del sole. Per me era un mago uscito da antiche
storie, come l'ombra di Merlino proiettata attraverso il tempo. Era
tutto quel sapere a incurvargli le spalle, non il passare degli anni, e
lui non era soltanto un tutore ma il rappresentante di un antico ordine
di preti-filosofi che risaliva alla fondazione dell'Impero e ancora più
indietro, ai signori delle macchine Mericanii, morti da sedicimila anni.
Gli scoliasti consigliavano gli {imperatori}, navigavano nei posti
oscuri al di là della luce dei soli e fino a strani pianeti. Prestavano
servizio in squadre che portavano nel mondo nuove invenzioni e nuovo
sapere, e possedevano poteri di memoria e di cognizione al di là di ciò
che era meramente umano.

Volevo essere uno di loro, essere come Simeon il Rosso, volevo una
risposta a tutti i miei interrogativi e il controllo di cose segrete e
arcane. Per questo motivo avevo implorato Gibson di insegnarmi la lingua
dei Cielcin. Le stelle sono innumerevoli, ma a quei tempi credevo che
Gibson le conoscesse tutte per nome, sentivo che se lo avessi seguito
nella vita di uno scoliasta avrei potuto apprendere i segreti nascosti
sotto quelle stelle e viaggiare al di là di esse, perfino oltre i
confini delle terre di mio padre.

Duro di udito com'era, Gibson non mi sentì entrare, quindi sussultò
quando parlai da dietro la sua spalla.

«Hadrian! Per le ossa della Terra, ragazzo! Da quanto tempo sei lì in
piedi?»

Consapevole che il mio posto era quello di uno studente davanti al suo
insegnante, eseguii il mezzo inchino che il mio maestro di danza mi
aveva insegnato. «Solo da un momento, maestro. Volevi vedermi?»

«Cosa? Oh, sì, certo...» Il vecchio notò la porta chiusa alle mie spalle
e ripiegò il mento contro il petto. Sapevo cosa significava quel gesto,
era la radicata paranoia di un veterano del palazzo, l'impulso di
controllare se ci fossero droni con videocamera e microfoni. Non ce ne
sarebbero dovuti essere nell'alloggio di uno scoliasta, ma non si poteva
mai essere sicuri. Privacy e segretezza erano i veri tesori della
nobiltà. Quanto erano rare e preziose! Gibson fissò un occhio grigio
sulla maniglia della porta e cambiò lingua, passando da quella standard
galattica a quella gutturale di Lothrian, consapevole che nessuno dei
servitori del palazzo era in grado di comprenderla. «Queste sono cose
che non dovrebbero essere dette. Ci sono degli ordini, capisci? È
proibito parlarne.»

Questo catturò la mia attenzione e mi sedetti su un basso sgabello,
prendendomi solo il tempo necessario a rimuoverne una pila di libri.
Esprimendomi in lothriano come il mio tutore, dissi: «Qui dentro è un
disastro.»

«Non c'è nessuna correlazione fra l'ordine nello spazio di lavoro di una
persona e quello nella sua mente.» Lo scoliasta appiattì con una mano i
capelli scompigliati, ma la cosa non fu di aiuto.

«La pulizia non è vicina alla santità?» Faticavo a esprimermi in quella
strana lingua. I Lothriani non avevano pronomi personali, non
riconoscevano nessuna identità. Avevo sentito che la loro gente non
aveva neppure un nome.

Il vecchio sbuffò. «Oggi siamo sfacciati, eh?» Tossì piano, grattandosi
una basetta cespugliosa. «Benissimo. Questa notizia non può aspettare. È
stata ricevuta la scorsa notte, altrimenti sarebbe stata condivisa
prima.» Trasse un profondo respiro, poi proseguì in toni misurati: «Un
entourage del Consorzio Wong-Hopper dovrebbe arrivare qui entro la
settimana.»

«Entro la settimana?» Ero così stupito che per un momento mi dimenticai
del lothriano, aggiungendo: «Com'è che io non l'ho saputo?»

Lo scoliasta mi lanciò una seria occhiata da sopra il naso ricurvo,
replicando in lothriano: «L'onda \foreignlanguage{italian}{qet} è
arrivata solo alcuni mesi fa e il Consorzio ha deviato dalle sue usuali
rotte commerciali per fare questo viaggio.» Poi aggiunse senza
preamboli, e senza attutire l'impatto delle sue parole: «Cai Shen è
stato colpito. I Cielcin lo hanno distrutto.»

«Cosa?» Quella parola mi sfuggì di bocca in galstani e mi affrettai a
fare marcia indietro, ripetendola in lothriano: «\emph{Iuge}?»

Gibson continuò a guardami, fissando intensamente il mio volto come se
fossi stato un'ameba sulla piastra di Petri di qualche magio. «La flotta
del Consorzio ha ricevuto il telegramma dal sistema di Cai Shen appena
prima che il pianeta cadesse.»

È strano, vero, come i più grandi disastri della storia spesso sembrino
vuoti e astratti, come un tuono lontano. Un antico re ha scritto che una
singola morte è una tragedia, ma un genocidio può essere compreso solo
attraverso le statistiche. Non avevo mai visto Cai Shen, non avevo mai
lasciato il mio mondo natale, Delos. Quel posto era soltanto un nome per
me. Le spalle di Gibson portavano il peso di milioni di vite, ma sulle
mie spalle non ce n'era nessuna. Forse mi considererete mostruoso, ma
nessuna mia preghiera o azione potrebbe riportare indietro quelle
persone o spegnere i fuochi che ardono sul loro mondo, così come non
potrei risanare ogni uomo e donna mutilati dalla Cappellania. Qualsiasi
potere avessi in quanto figlio di mio padre si estendeva solo fin dove
mi era permesso di arrivare, quindi incassai la notizia senza elogi
funebri e il mio shock iniziale lasciò il posto a un'intontita
accettazione. Poi qualcosa di più profondo, di freddo e pragmatico, si
impadronì di me. «Sono venuti in cerca di una nuova fonte di uranio»
dissi. Sembravo mio padre.

L'accenno di sorriso dello scoliasta mi indicò che avevo ragione ancora
prima che lui lo ammettesse. «Molto bravo.»

«Di che altro si potrebbe trattare?»

Gibson cambiò rumorosamente posizione sulla poltrona, gemendo per
qualche protesta dell'età. «Con Cai Shen distrutto, il Casato Marlowe
diventa il più grande fornitore di uranio dotato di licenza presente in
questo settore.»

Deglutii a vuoto e mi protesi in avanti, appoggiando il mento sulle mani
intrecciate. «Allora vogliono fare un accordo? Per le miniere?» Prima
però che Gibson potesse formulare una risposta fui assalito da un
interrogativo più cupo, uno che non sapevo formulare in lothriano, per
cui sussurrai: «Perché non mi hanno informato di questo?» Quando Gibson
non rispose ricordai un suo precedente commento e mormorai: «Ordini.»

«\emph{Da}.» Annuì, cercando di riportarmi a parlare in lothriano.

«Specificatamente?» Mi ritrassi in modo brusco. «Lui ha ordinato
specificatamente di non dirlo a \emph{me}?»

«Ci è stata data istruzione di non condividere la notizia con chiunque
non avesse l'autorizzazione dei corpi di propaganda o senza il permesso
dell'arconte.»

Mi alzai in piedi e dimenticando la mia posizione parlai in galstani.
«Ma io sono il suo \emph{erede}, Gibson. Lui non avrebbe dovuto...» Mi
accorsi che lo scoliasta mi fissava con occhi roventi e tornai al
lothriano. «Questo genere di cosa non dovrebbe essermi tenuto nascosto.»

«Non so cosa dirti, ragazzo mio, davvero.» Gibson passò con disinvoltura
a esprimersi in jaddiano, lanciando un'occhiata fuori dalla finestra
quando un addetto alla manutenzione salì su un'impalcatura oltrepassando
il vetro colorato nell'ombra di un contrafforte. Se avessi piegato il
collo avrei potuto quasi vedere al di là del muro di cinta la vasta
distesa dell'Oceano Apollano che si {stendeva} a est fino alla curva
dell'orizzonte. «Continua a comportarti come se non ne sapessi niente ma
preparati. Sai come sono questi incontri.»

Accigliandomi, mi mordicchiai l'interno di una guancia, poi imitai il
suo cambiamento di linguaggio, chiedendo: «Ma, e i Cielcin? Sono certi
che sia stata una scorreria?»

«Ho visto io stesso le riprese dell'attacco. Il Consorzio ha trasmesso
tramite onda l'ultimo pacchetto di notizie giunto da Cai Shen insieme
all'annuncio della sua visita. Tuo padre ha fatto stare svegli tutta la
notte Alcuin e me per riesaminarlo con i logoteti. Sono stati i Cielcin,
non ci sono dubbi.»

Per parecchio tempo rimanemmo seduti senza che nessuno dei due si
muovesse. «Cai Shen non è nel Velo» osservai infine, riferendomi alla
frontiera al di là del Braccio del Centauro della galassia che
costituiva il grosso del fronte bellico contro i Cielcin. Mi fissai le
mani. «Stanno diventando più audaci.»

«Le ultime informazioni dicono che la guerra non sta migliorando, sai.»
Gibson distolse nuovamente da me gli occhi acquosi e guardò fuori dalla
finestra, oltre i merli dall'aspetto deliberatamente antico e i bastioni
puramente simbolici che racchiudevano la casa della mia famiglia. Il
servitore era ancora là fuori, a lucidare i vetri a mano.

Di nuovo scese il silenzio e anche questa volta fui io a infrangerlo.
«Credi che verranno qui?»

«Su Delos? Nello Sperone?» Gibson mi adocchiò in modo significativo,
aggrottando le sopracciglia cespugliose. «Siamo a quasi ventimila anni
luce dal fronte. Direi che per il momento siamo al sicuro.»

«Perché mio padre insiste nell'avere segreti con me?» chiesi,
continuando a parlare jaddiano. «Come si aspetta che possa governare
questa prefettura dopo di lui se non mi vuole coinvolgere?» Gibson non
rispose, e siccome è nella natura tipica dei giovani essere sordi ai
silenzi io non colsi il suo significato né vidi la risposta presente in
esso. Preso dalla gravità della situazione, che non potevo più scuotermi
di dosso, insistetti: «Crispin lo sa? Del Consorzio?»

Gibson mi rivolse un lungo sguardo compassionevole, poi annuì.