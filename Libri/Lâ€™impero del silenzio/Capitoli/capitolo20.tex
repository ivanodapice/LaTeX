\chapter{Fuori dalle mappe}

«Siamo pronti al decollo?» chiese una rauca voce femminile quando
Demetri e io finimmo di sistemare il baule in uno spazio fra casse di
legno e fusti d'acciaio, nella stiva dal soffitto basso. L'aria a bordo
della \emph{Eurynasir} era gelida, come lo è a bordo di molte astronavi,
e le luci erano regolate a una bassa intensità e su un tono dorato,
proiettando fiochi bagliori sulle pareti nere e sullo sfregiato
pavimento di metallo. Il posto odorava di polvere da sparo usata, di
grasso per motori e di metallo bruciato. E di ruggine. Non era un odore
pulito o che ispirasse sicurezza. Quella nave era in circolazione da
molto tempo; almeno da decenni e forse più a lungo.

Girandomi vidi avvicinarsi una donna in una tuta di un grigio opaco.
Aveva la stessa pelle bronzea di Demetri e gli stessi capelli luminosi,
anche se i suoi erano lunghi il doppio e le scendevano ondulati fin
quasi ai gomiti. Erano così simili che avrebbero potuto essere cugini o
fratelli, se non fosse stato per il modo in cui il volto di Demetri si
illuminò quando si girò verso di lei per prenderla fra le braccia con un
basso verso che gli echeggiava nella gola mentre premeva le labbra sulle
sue. «Juno! Vieni a conoscere il nostro nuovo amico!» Protese un braccio
verso di me. «Bassem ha già preparato i motori? Voglio tagliare
immediatamente la corda.» La donna, Juno, lo seguì e mi porse la mano.
La guardai per un momento, e in quella pausa di confusione Demetri
aggiunse: «Lui è Hadrian Marlowe, mia signora.»

«Mia signora?» Mi inchinai, temporaneamente dimentico della mia
confusone nel bisogno di rispettare un impulso quasi genetico alle buone
maniere. La mano della donna rimase protesa ancora per un po' -- non
sapevo perché -- ma dopo qualche secondo lei la lasciò ricadere.

Entrambi risero, poi la donna replicò: «Non sono una dama, e Demetri sta
solo cercando di essere affascinante. È fatto così.» Si portò la mano al
petto. «Qui non c'è sangue nobile.» Avrebbe potuto dirmi che era una
principessa di Jadd e avrei potuto crederle. Su Jadd l'ossessione
eugenetica per la bellezza era stata elevata a imperativo morale e
perfino le loro classi medie si riproducevano in modo da abbellire e
quindi glorificare il loro popolo. Né i suoi capelli né quelli di
Demetri potevano essere naturali, si doveva trattare di una modifica
acquistata sul mercato secondario, ed era il primo segno che stavo
lasciando il giardino ben curato della vita imperiale.

«Tranne che per lui» dichiarò Demetri, portandosi una mano alla bocca e
facendo scorrere un pollice sul labbro inferiori. «Il ragazzo è di
sangue reale, figlio di un qualche arconte.»

La donna si illuminò, con gli occhi che risplendevano ambrati nella luce
gialla che scendeva dall'alto. «Davvero? Non avevo mai incontrato un
palatino imperiale prima d'ora.»

A disagio, distolsi lo sguardo. «Non sono più un palatino, madama.»

«Chiamami Juno, per favore» disse, venendo avanti per darmi un'occhiata
con gli occhi socchiusi nella penombra. Quando ci eravamo incontrati in
quella taverna sui moli avevo pensato che Demetri fosse alto, ma nessuno
dei due jaddiani era alto quanto me. Non sapevo come categorizzarli
secondo gli standard imperiali perché i capelli denunciavano un qualche
cambiamento nel loro sangue per cui non li potevo definire plebei. Erano
patrizi, quindi? Potenziati come sir Roban e gli altri cavalieri di mio
padre?

Una serie di note crescenti risuonarono attraverso gli altoparlanti di
bordo, un suono lento in ascesa come i rintocchi di un orologio che
scandisse l'ora, poi dagli altoparlanti scaturì una profonda voce
maschile con un accento marcato quanto quello di Demetri. «Il passeggero
è a bordo, capitano?»

«Sì, Bassem!» confermò Demetri, avviandosi verso la murata arrotondata e
fuori dalla stiva gelida. «Non sprecare tempo a chiedere il permesso di
decollo... vai e basta. Prendi quota dal mare e vai su, conosci la
strada. Noi arriviamo subito.» Sulla soglia si girò con entrambe le mani
che stringevano il telaio di metallo, come un attore che si pavoneggia
sulla scena. «È una cosa a cui vorrai assistere.»

`Su.'

Quella parola mi risuonò dentro, vibrando, nonostante tutte le cose che
potevano andare storte. Sorrisi e seguii il mercante oltre la murata e
su per una rumorosa scala di metallo, oltre la sigillata porta di vetro
dell'infermeria di bordo. Un paio di donne pallide con la faccia sporca
sbirciò dall'ombra di una cabina e qualcuno gridò una domanda al
capitano in quella che mi parve una delle lingue della Demarchia di
Tavros, ma Demetri non rispose.

«Quanti membri ha il tuo equipaggio, messere?» chiesi.

«Chiamami Demetri» mi corresse mentre mi guidava in una sala comune dal
soffitto basso. «O `capitano', se lo preferisci.» Un tavolo di metallo
ellissoidale dominava il basso spazio, completo di panche rozzamente
saldate al pavimento. L'ambiente era del tutto spoglio, perché gli
oggetti casuali che indicano l'abitazione umana erano stati portati via
e imballati. «Siamo solo in sei, senza contare te.» Indicò Juno, che mi
seguiva da vicino. «Hai conosciuto la mia adorabile moglie. Poi ci sono
Bassem, le gemelle, il dottor Sarric e il vecchio Saltus.» Si fermò di
colpo, accigliandosi. «Credo che allora siamo in sette. Scusami.»

Un'astronave. Quella era una vera astronave, non una delle navette
suborbitali a cui ero abituato e che erano a stento adatte a grattare la
volta del cielo. Il cuore mi premeva contro la gola. Una vera astronave,
e io ero a bordo. Avevo sognato questo momento fin da quando ero
bambino, da quando avevo appreso che Delos non era tutto il mondo ma un
pianeta isola in esso. L'\emph{Eurynasir} sussultò sotto di noi e colsi
il suono fangoso dell'acqua che si agitava. Barcollai, spinto contro la
parete concava del corridoio, e per poco non caddi attraverso un
portello aperto che conduceva al livello sottostante.

«Ehi!» Un piccolo volto avvizzito con la pelle del colore della cenere
fece capolino da un portello che da un lato si apriva nel pavimento.
All'inizio pensai che si trattasse di un bambino, anche se nessun
bambino aveva un aspetto tanto rugoso. Perfino Gibson, la cui vita si
era ritirata a tal punto nella vecchiaia da perdersi in essa, appariva
giovane al confronto di quel piccolo goblin. Doveva di certo trattarsi
di un omuncolo: un replicante dai geni modificati su richiesta, come il
piccolo araldo che mio padre teneva presso di sé o la uri dalla pelle
azzurra di mia madre. La creatura scarna parlò di nuovo, con una voce
acuta in modo impossibile: «Stiamo partendo, Demetri?»

«Sì, Saltus.» Lo Jaddiano si girò. «Meglio allacciare le cinture. Stiamo
per accelerare.»

L'ometto si tirò su fino a emergere dal portello, arricciando la faccia
già rugosa. Non poteva misurare più di quattro piedi in tutta la sua
altezza ed era strutturato in un modo che mi ricordava gli oranghi che
avevo visto nello zoo della nonna, con le mani che quasi strisciavano al
suolo e avevano il dorso coperto di peli grigi quanto i suoi fitti
capelli. Le gambe erano corte e incurvate. Saltus sorrise, passandosi
una di quelle mani enormi e `innaturali' sul cuoio capelluto glabro,
fino ad afferrare il codino che gli cresceva alla base del cranio. Nel
parlare rigirò fra le mani quella coda di capelli. «È questo il
passeggero?»

«Certo che lo è, \emph{haqiph}» replicò Juno, in tono sprezzante.
Sembrava disgustata quanto me dalla creatura, anche se il suo disgusto
aveva un che di stanco.

L'omuncolo, Saltus, mi scrutò dal basso in alto, rigirando i capelli in
un'orribile parodia di un comportamento da bambina. «Non mi avevate
detto che era come me.»

Sussultai al punto che per poco non schizzai fuori dalla mia pelle e
sentii i capelli che mi si rizzavano sul collo. «Cosa vuoi dire?»
Dovetti impormi di rilassare i pugni stretti. Non avremmo potuto essere
meno simili se quel piccolo mostro fosse stato un Cielcin. Gli omuncoli
non erano umani, non in senso vero, rappresentavano una falla nelle
leggi tecnologiche della Cappellania -- i loro decreti religiosi -- e
come con tutte le falle, avidità e crudeltà umana si riversavano come
vino in quello spazio. Gli omuncoli venivano creati per svolgere compiti
che gli uomini normali, anche i servi vincolati a un pianeta, trovavano
disgustosi. Che uno di loro si paragonasse a me...

«Siamo entrambi omuncoli» continuò allegramente, e mi porse la mano come
aveva fatto Juno. Non la presi, perché non comprendevo il gesto. «Siamo
entrambi figli delle vasche.»

Un riflesso aristocratico mi spinse a scagliarmi in avanti. «Non sono un
omuncolo!»

Non riuscii a evitare che il disgusto mi trasparisse dalla voce.

«Zitto, Salt» intervenne Demetri. «Niente liti. Quest'uomo paga meglio
di te.»

«Ha anche un odore migliore» interloquì Juno con un sogghigno. Accanto e
intorno a noi la \emph{Eurynasir} cominciò a stridere, con i motori che
passavano da un basso ringhio a un suono acuto e costante, come se acque
profonde stessero scorrendo nelle vene del mondo. «Faresti meglio a
metterti comodo, Sal» aggiunse la donna, maliziosamente, incrociando le
braccia.

L'omuncolo borbottò qualcosa mentre lei mi incitava a seguire il marito
fino alla fine del corridoio, giù per una breve rampa di scale e nella
cupola di vetro posta a prua che avevo visto dall'esterno. Il ponte di
comando -- perché di certo si trattava di questo -- era disposto su un
dito d'acciaio proteso verso l'esterno situato al centro della cupola,
in modo che la bolla di vetro si allargasse in fuori e intorno a noi su
tutti i lati, permettendoci di vedere in pari misura il cielo e il mare.
Saltus era svanito di nuovo giù per il portello nel pavimento e un uomo
enorme, largo di spalle e con la pelle e i capelli neri quanto la
bandiera della mia famiglia, sedeva ai comandi su una poltrona troppo
piccola per lui. Al mio ingresso la nave attraversò una delle rare onde
e sobbalzò, scaraventandomi contro l'ingresso ovale imbottito e da lì
contro una serie di strumenti che lampeggiavano in silenzio.

«Siete in ritardo» disse l'uomo, con la voce profonda che tuonava al di
sotto del suono della musica trash che usciva dalla sua console di
comando. «Abbiamo quasi raggiunto la velocità di decollo.»

Demetri occupò la sedia accanto alla sua, affibbiando le cinture mentre
l'altro azionava una serie di interruttori rossi posti sopra la sua
testa, procedendo da sinistra a destra. «Qualche problema con il
controllo del traffico aereo?»

«In una schifosa città retrograda come questa?» Il timoniere sbuffò.
«Non che abbia sentito.» Si girò per sorridere a Demetri. «D'altronde ho
chiuso le comunicazioni. Ero stanco di tutto quel chiacchierare.» Si
interruppe per un momento, materializzando nell'aria davanti a sé una
fila di scintillanti ologrammi fra il bianco e l'azzurro. Con la
disinvoltura derivante da lunga pratica, infilò le dita in un reticolo
luminoso e parlò nell'aria, davanti a sé, con voce che si diffuse nel
sistema sonoro di bordo a beneficio dell'omuncolo e degli altri che
avevamo oltrepassato. «Se voialtri bastardi non avete allacciato le
cinture, adesso è il momento di farlo.»

La nave si inarcò di nuovo, sollevandosi dalle onde per un paio di
secondi abbondanti. Quando ricadde scivolai di lato su uno dei bassi
divani antiaccelerazione lì in attesa. Juno cercò di afferrarmi; in
qualche modo era riuscita a rimanere in piedi. Dopo le cose procedettero
lisce per un po', abbastanza a lungo perché mi girassi e mi assicurassi
al sedile.

«Chiudo la cupola!» avvertì Demetri, protendendosi sul grande pannello
del copilota per attivare una piccola leva. Le sue dita danzarono
sull'arco della console come fosse stata un pianoforte, con la musica
che risuonava e sferragliava ancora mentre fuori enormi pezzi di scafo a
forma di petalo di fiore si chiudevano sulla cupola, lasciandoci al
buio. Ero così affascinato da quella precisione meccanica e da quella
velocità che non notai di aver contemplato per l'ultima volta la
superficie del mio mondo che svaniva come fa la luce attraverso
l'apertura di una videocamera: divisa in cunei, poi in scaglie, poi
ridotta a oscurità. Delos era scomparso, e l'interno della cupola era
invece occupato da un modello olografico delle traiettorie di volo,
dalla telemetria di bordo e dal suono pulsante della stridente musica
metal di Bassem.

Ed eccola là, la lieve sensazione che lo stomaco mi precipitasse
attraverso cieche profondità, l'infuriare dei motori gemelli a fusione,
lontano a poppa. Stavamo volando, sollevandoci lungo la curva di una
catena invisibile stesa attraverso l'aria e l'oscurità, diretti verso
un'oscurità ancora più fitta. Cosa non avrei dato in quel momento per
avere una finestra! «La sentirai davvero fra un secondo, ragazzo» mi
gridò Demetri al di sopra della musica e dell'ululato dei motori a
fusione.

Non si sbagliava. L'accelerazione mi piombò addosso come uno spaventoso
stivale che mi schiacciava contro il sedile. Il mio posto era rivolto
verso il centro della nave, ad angolo retto con l'asse di propulsione,
quindi venni premuto di lato contro il poggiatesta. Sentii la pelle che
mi pendeva dalle ossa, il suo peso fu come uncini nella mia carne: Delos
non voleva lasciarmi andare. La vista mi si offuscò e sputai, con gli
occhi che mi si spegnevano come una candela. Gemetti, ma quel suono si
perse nel clamore di una chitarra artificiale e nel ringhiare di lingue
scurrili che provenivano dalla console di bordo.

Poi fu tutto finito, svanito nel nulla. Perfino la musica cessò.

«Ehi!» protestò il timoniere, assestando un pugno al braccio del
capitano. «Io stavo ascoltando!»

«E noi ci dobbiamo accertare che nessuno se la prenda per questo decollo
non autorizzato, e io non mi posso concentrare con quello \emph{skubus}
che tu chiami musica nelle orecchie, Bassem» scattò Demetri, usando il
termine jaddiano per `merda'. Le sue dita danzarono su una serie di
comandi del pannello di controllo e lui si chinò in avanti per studiare
il display. «Non vedo segnalazioni di sorta. E tu?»

Bassem scosse il capo. «Ancora niente. Però ci vorrà qualche ora prima
di poter entrare in curvatura.» Mi lanciò un'occhiata con aria
riflessiva. «Ti ho sentito dire a Sat qualcosa sul fatto che il ragazzo
sarebbe di sangue reale.» Deglutì a fatica mentre rimetteva a posto la
leva attivata da Demetri prima di slacciare le cinture e di girarsi in
parte per vedermi meglio. «Non pensi che la
\foreignlanguage{italian}{fod} ci piomberà addosso?»

«La Forza Difensiva? Non se i codici che quella dama ci ha passato
risulteranno validi» ribatté Demetri. «Tienili di riserva nel caso in
cui qualcuno ci segnali. Non voglio attirare l'attenzione prima del
tempo.»

Non sentii quasi niente di tutto questo perché ero distratto; non da
quello che stava succedendo ma da ciò che era scomparso. La forza di
gravità. Galleggiavo sotto le cinghie e rilassai le braccia per
guardarle galleggiare come bastoni nell'aria, davanti a me.

Mi sfuggì una sottile risata e mi nascosi il volto fra le mani. Juno si
era assicurata al divano accanto al mio e si girò al di là del
poggiatesta per guardarmi. «Perché ridi?» Aggrottò la fronte e lanciò
un'occhiata a Demetri in cerca di supporto. Lui non la vide perché stava
inserendo i comandi necessari per riaprire la schermatura della cupola.
Per la prima volta mi accorsi che il ruggito lamentoso dei motori era
scomparso e che il silenzio era punteggiato da un profondo clangore
metallico a mano a mano che il reattore a fusione si quietava e il
dissipatore di radiazioni si apriva. In {precedenza} non lo avevo udito,
ma adesso che ogni altro suono era svanito nella notte infinita, quel
rumore risuonò nitido nelle mie orecchie come le campane della
Cappellania, e laddove pochi minuti prima quell'iride aveva escluso la
luce del mio mondo natale, adesso esso si aprì su un'oscurità totale e
assoluta.

Mi passai le mani sulla faccia, sperando di cancellarne le emozioni. La
voce di Gibson mi risuonò all'orecchio, tanto vicina che quasi lo
percepii accanto alla spalla. `La gioia è un vento, Hadrian.
\phantomsection\label{fileintero-22.xhtml__idTextAnchor001}{}Ti solleva
solo per farti schiantare di nuovo sulle rocce.' Mi aggrappai alla parte
iniziale di quell'affermazione e borbottai:
«\phantomsection\label{fileintero-22.xhtml__idTextAnchor002}{}La gioia è
un vento.»

«Prego?»

La sensazione di assenza di peso svanì all'improvviso quando il campo di
soppressione entrò in funzione. Prima di allora non ne avevo mai
sperimentato uno senza l'influenza della forza di gravità di Delos --
avevo sperimentato l'effetto del campo di Royse solo come mezzo per
contrastare l'inerzia nei voli ad accelerazione elevata -- quindi non
ero preparato al senso di nausea che mi provocò. Le braccia mi ricaddero
lungo i fianchi e il mio peso collassò sul sedile. Mi sentivo come se mi
avessero drappeggiato addosso una rigida coperta fredda. Le poche cose
sospese nell'aria... una penna luminosa, una lattina vuota, una scatola
di carte da gioco... caddero tutte, ma io continuai a sentirmi privo di
peso. Un campo di soppressione non era una forza di gravità vera e
propria e neppure una vera gravità artificiale, ci bloccava soltanto sul
ponte come farfalle fissate con uno spillo sotto un vetro.

Sentendomi impallidire, borbottai: «Credo di essere sul punto di
sentirmi male.» Juno tirò fuori all'istante un sacchetto di carta che mi
tenni davanti alla faccia, respirando con cautela.

«Cosa diavolo era quella storia del vento?» domandò il grosso timoniere,
girandosi e slacciando le cinghie.

Spinsi lo sguardo alle sue spalle, verso l'indicibile bellezza del
cosmo, eterna, intoccabile e pulita. «Qualcosa che era solito dire il
mio tutore» replicai, e quando i tre mercanti continuarono a fissarmi,
aggiunsi: «Era uno scoliasta.»

Bassem parve sorpreso, ma Demetri e Juno annuirono entrambi. «Questo
spiegherebbe la nostra destinazione» commentò poi il capitano.

«Quale sarebbe?» chiese il timoniere.

Demetri puntò un dito verso il mio petto. «Il ragazzo andrà sotto
ghiaccio finché non atterreremo su Teukros.»

«Questo lo so» scattò Bassem, alzandosi e stiracchiando la schiena con
un gemito. Quale che fosse il suo sangue, era di certo più alto di me --
alto quasi quanto mio padre -- e lo sapeva, lo si poteva notare dal modo
in cui guardava dall'alto in basso il suo capitano e me. «Ma questo cosa
c'entra con gli \emph{hudr}?» Quell'espressione mi lasciò sconcertato
perché prima di allora non avevo mai sentito chiamare così uno
scoliasta: `verde'.

«Ci si aspetta che lo sbarchiamo a Nov Senber» spiegò Demetri, passando
allo jaddiano, poi tornò a esprimersi nel galstani imperiale mentre mi
indicava con un gesto della mano. «Il nostro qui presente amico si unirà
agli \emph{hudr}.»

Bassem mi guardò con aria accigliata e rughe profonde che incorniciavano
le labbra grigie. «Perché?» Il disgusto nella sua voce era tanto denso
da essere quasi solido e mi colpì come uno schiaffo.

Non risposi subito. In qualche modo, non riuscivo a vedere davvero quel
grosso uomo e guardavo piuttosto al di là della sua figura, in direzione
del disco del pianeta visibile attraverso la cupola trasparente. Delos.
I suoi mari grigi si stendevano ampi e selvaggi sotto di noi, e la
monocromia era più pronunciata a causa del misero contrasto fornito da
qualche sparsa nuvola bianca. Solo la terra vi dava sollievo, fra il
verde e il nero o color ocra, o ancora di un marrone chiaro, o di un
rosso iroso, o di una bruciante terra di Siena. Immaginai il globo di
mio padre, quello che fluttuava sopra la sua scrivania nell'edificio del
campidoglio della prefettura. Da quell'alta orbita era facile immaginare
di vedere solo quel globo e non il mondo effettivo, con la sensazione
che da un momento all'altro mio padre mi avrebbe schiaffeggiato di nuovo
e non sarei ricaduto sul sedile antiaccelerazione ma contro la sua
scrivania, con la sedia di legno padouk che si rompeva dietro di me.

Alla fine scrollai le spalle. «Sempre meglio della Cappellania.»

«Vuoi che ti svuotino il cervello?» chiese Bassem, con il disgusto che
contraeva per un momento il suo volto cordiale. «Vuoi che ti riempiano
la testa di ingranaggi?»

«Non è così che funziona!» Mi alzai a mia volta, trapassando il grosso
marinaio con uno sguardo tagliente. «Non sono demoniaci, solo che
studiano per secoli, addestrano la mente a lavorare meglio, con maggiore
efficienza.»

«Trasformandosi in fottuti stronzi senz'anima, ecco come.» Bassem guardò
il capitano con occhi roventi. «Questo non mi piace, capo.»

Demetri scrollò le spalle e si torse le mani come Pilato sulla bacinella
d'acqua. «Non ti deve piacere, Bassem. Abbiamo in ballo novemila marchi
e tutto quello che dobbiamo fare è scaricarlo a destinazione.»

