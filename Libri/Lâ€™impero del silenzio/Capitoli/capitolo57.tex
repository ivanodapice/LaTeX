\chapter{Il secondo}

«Un'ora, Vostra signoria» disse il pilota, che proveniva dall'ufficio
dei Prefetti urbani come le due guardie che mi accompagnavano nella mia
missione per accertarsi che non cercassi di fuggire dalla città. Saremmo
potuti andare a piedi, ma i miei custodi parevano pensare che questo mi
avrebbe solo offerto maggiori opportunità di tentare qualcosa, per cui
avevo permesso che mi trasportassero così attraverso la piazza che
separava il castello di Borosevo dal colosseo. Lasciammo il pilota nel
velivolo mentre gli altri due mi scortavano oltre i douleter e le
guardie dell'arena, fin nell'ipogeo.

I soffitti a volta di cemento incombevano bassi, poster e stampe
economiche erano fermate con il nastro adesivo sopra la testata di
alluminio dei letti su cui erano sparse in disordine le cose
appartenenti ai combattenti. Il letto che era stato di Ertro era già
stato rioccupato e il mio era del tutto scomparso; smontato senza dubbio
dai soldati del conte dopo che ero stato... scoperto. Un membro della
mia scorta si schiarì la gola per ricordarmi che stavo sprecando tempo,
ma la verità era che temevo i momenti successivi. Avevo dichiarato
guerra a un uomo con una sfida formale a una monomachia, avevo sconvolto
la corte del conte, rivelato il mio sangue nobile, insultato e aggredito
la Cappellania, ed era possibile che nel fare tutto questo mi fossi
anche alienato le simpatie di Valka.

E tuttavia... tuttavia ero sul punto di sistemare una cosa e questo mi
terrorizzava più di tutto il resto messo insieme. Come qualcuno ha detto
una volta, andare in guerra è facile, è fare la pace che è difficile.

Switch era seduto a uno dei piccoli tavoli dell'area in fondo al
dormitorio, vicino ai vecchi dispensari di cibo e ai servizi igienici.
Era solo e stava girando distrattamente le pagine di un romanzo
illustrato. Le mie guardie erano rimaste indietro, quindi il mio
approccio fu più silenzioso di quanto avrebbe potuto essere. Alcuni
degli altri mirmidoni presenti nella stanza mi notarono per primi e su
di loro scese un silenzio che infine avvertì il giovane della mia
presenza.

Non so se sia possibile per un volto illuminarsi e incupirsi nello
stesso momento, ma se lo è il suo lo fece. Sgranò gli occhi, ma la
sorpresa si trasformò all'istante in sospetto e lui si alzò quasi del
tutto, con le labbra serrate in una sottile linea bianca per poi
rimettersi subito a sedere e chiudere il libro con un tonfo sommesso.
«Cosa ci fai qui?»

La notte precedente avevo provato una dozzina di versioni di questa
conversazione, ma nessuna era suonata nel modo giusto. Ero convinto di
aver protetto i miei amici nascondendo il mio nome legittimo, ma paura e
orgoglio mi avevano spinto in una posizione che non avevo voluto né
inteso occupare e qualsiasi altra cosa avessi potuto dire ce n'era una
che andava \emph{} detta. «Mi dispiace, Switch.» Non mi inchinai o
inginocchiai, non chinai neppure il capo.

Switch mi guardò e annuì. «Pallino ha detto che eri rinchiuso nel
castello.» Guardò in modo espressivo verso le mie guardie, aggiungendo:
«Che non ti lasciavano uscire.»

«Sì.» Da sopra la spalla lanciai un'occhiata ai due impassibili prefetti
nella loro uniforme cachi, e d'un tratto fui consapevole di aver tratto
un profondo respiro. Il mio segreto era venuto alla luce, barattato con
il livido che avevo segnato sulla mascella di Gilliam e con le mie
circostanze disperate. Adesso non c'era più bisogno del mascheramento
che avevo adottato durante quei primi giorni su Emesh. Avrei affrontato
qualsiasi punizione mio padre, il mio Casato e la Cappellania avessero
scelto di infliggermi, ma lo avrei fatto dopo quel duello, e prima avevo
bisogno di Switch, di qualcuno disposto a supportare legalmente la mia
sfida.

Avrei potuto chiederlo a Pallino, ma se stavo per morire non volevo
farlo senza prima sistemare le cose con Switch. In tutto {l'universo}
avevo così pochi amici che non potevo permettermi di fare a meno di uno
di essi, quindi gli raccontai tutto. Parlai in fretta, consapevole dei
limiti di tempo che l'ufficio dei Prefetti aveva {imposto} a quella mia
uscita, e non fu difficile, perché c'erano molte cose che non volevo
ricordare o raccontare.

Gli dissi di Gibson e della scuola della Cappellania su Vesperad, di
come fossi fuggito dopo aver pestato quasi a morte mio fratello. Gli
raccontai di come mia madre mi avesse salvato e di come fosse stato per
amor suo e per la sua sicurezza che non avevo dichiarato il mio sangue e
il mio status alle autorità e non lo avevo confidato neppure a lui. Non
mi vergogno di ammettere che piansi nel parlare di questo, per il
rinnovato timore della sicurezza di mia madre alla luce di quello che
avevo fatto.

Poi parlai del periodo in cui ero vissuto lungo i canali anche se non
accennai a Cat, ai pestaggi subiti o a quella notte senza luna nel
vicolo. Alcune cose dovrebbero rimanere non dette e non ricordate. Gli
spiegai di come mi fossi ritrovato nell'arena e lui lo comprese,
accettando anche lo stupido motivo che mi aveva spinto a penetrare nelle
prigioni. Rise, e quando ebbi finito la narrazione conclusi: «Ho bisogno
del tuo aiuto, Switch.»

Lui sbatté le palpebre, sconcertato ma non offeso. «Per cosa?»

«Ho colpito quel prete intus» dissi semplicemente. «È per questo che
nulla di tutto questo è più un segreto. Mi avrebbe fatto giustiziare se
non fossi...»

«Un palatino?» Switch pronunciò quella parola come se stesse sputando
veleno succhiato da una ferita.

«Un palatino» confermai, distogliendo lo sguardo. «So che ti sto
chiedendo molto e che non ho il diritto di farlo, ma fra due giorni
dovrò duellare con lui, e speravo che tu...»

«Sì» mi interruppe lui, alzandosi in piedi.

Per un momento dovetti apparire come un grande idiota. Congelato a bocca
aperta a metà di una parola, con le mani aperte in un gesto di supplica.
Quando ritrovai me stesso, balbettai. «Lo farai? Mi starai al fianco?»

«Come tuo secondo?» Serrò la mascella e annuì. «Certo che lo farò, Had.
Non sarei arrivato fin qui senza di te.» Non ho mai dimenticato
quell'\,`Had' al posto di Hadrian, il mio nome di mirmidone e non quello
di palatino. «Ma non vuol dire che non sia ancora dannatamente infuriato
con te.»

Mi sentii digrignare i denti e riuscii a mascherare la cosa annuendo.
«Switch, qualsiasi cosa...» Scossi il capo e squadrai le spalle, senza
allontanare i capelli che mi erano ricaduti sulla faccia. «Quegli altri
palatini... non ero io.»

Switch abbassò lo sguardo sul romanzo a fumetti, chiudendo gli occhi.
«Lo so, Had, lo so. È solo che tu non hai idea di come sia stato. Voi...
voialtri...» Scosse il capo. «Non riuscite a vederci, non ci vedete mai,
siamo solo parte dell'arredamento. Ci trattate come omuncoli ma non lo
siamo. Siamo umani, come voi.» Durante tutto quel discorso non mi guardò
e si limitò a incassare il mento e a sollevare le spalle come per
prepararsi a piegare un colpo.

«Io non sono quegli uomini» ribadii. La sua argomentazione funzionava
nei due sensi, ma questo non voleva dire che avesse torto. «Non credo
che esista una cosa come un genere di persone, ci sono soltanto
persone.» Non replicò e continuò a fissare il tavolo che aveva davanti
con le mani in grembo. Quello non era il momento per un discorso del
genere, quindi lasciai perdere. La sedia strisciò contro il cemento
smaltato del pavimento quando la tirai indietro, la girai e sedetti a
cavalcioni, dal lato opposto del tavolo rispetto a Switch. Gli altri
mirmidoni avevano da tempo smesso di stare con il fiato sospeso,
tornando alle loro attività. Lui continuò a non guardarmi finché non
ripresi a parlare. «Grazie, Switch.»

Lui annuì, senza guardarmi davvero, e alla fine chiese: «Perché io?»

«Eh?»

«Perché non Pallino? È un combattente migliore.»

«Non ho bisogno di un combattente ma di un amico» risposi. «E tu sei il
migliore amico che ho.»

Un sorriso malizioso gli apparve sul volto. «E questo cosa dice di te?»
Reagii con un gesto volgare del pollice e il suo sorriso si accentuò.

«Se la cosa ti mettesse in pericolo non te lo chiederei. Non dovrai
combattere, ma la legge richiede che abbia un secondo, e se le cose
dovessero andare male...»

«Non andranno male.»

«\emph{Se} dovessero andare male,» insistetti «non voglio che questo
rimanga non detto.» Mi interruppi e fu il mio turno di distogliere lo
sguardo. «Mi dispiace davvero, lo sai.»

Switch accantonò quelle scuse con un gesto. «Che cosa ha fatto quel
prete per farti infuriare tanto?»

«Cosa?» Per un breve momento ero riuscito a dimenticare per quale motivo
ero lì nell'ipogeo.

«Perché gli hai dato un pugno?»

In tutta la confusione seguita a quel giorno nel distretto dei magazzini
in cui avevo preso a pugni Gilliam, lui era la prima persona che non
contestasse la mia azione o mi rimproverasse per quello che avevo fatto.
Come ho detto, era il genere di amico migliore. Fu il mio turno di
sorridere. «Ha insultato una signora.»

Switch batté le mani e si sfregò i palmi, annuendo. «Classico, è un
classico.» Comunque non mi criticò e si comportò come se avessi detto la
cosa più ragionevole del mondo.

Per tutto il tempo, sentii nella mia mente la voce di Gibson che
mormorava: `Melodrammatico. Dannatamente melodrammatico.' Sorrisi,
arrossendo, e anche Switch sorrise. Premetti la fronte contro lo
schienale della sedia e ben presto ci ritrovammo a ridere. Per un
momento Gilliam venne bandito, da quel demone che era, e anche Valka, e
il conte, e Vesperad. E mio padre. Sarebbero stati ancora tutti lì per
un altro momento, che fosse una delle tre notti che avrebbero preceduto
il duello o quando sarebbero giunte notizie tramite telegrafo quantico
da Delos o dalla Cappellania. In quell'istante tutto quello che contava
era che avevo riavuto il mio amico e che eravamo sempre stati amici.

«Lei non vorrebbe che lo uccidessi e non lo voglio neppure io. Non più.»
Adesso che il calore del momento si era raffreddato con un collasso
entropico, non provavo più per niente quel desiderio di spargere il
sangue di Gilliam che mi aveva assalito il giorno precedente, ma ormai
avevo sferrato quel colpo e sigillato la mia sorte. In base alla legge
imperiale non potevo ritirare la sfida, in quanto era solo giusto che
una simile impulsività venisse punita dalle sue conseguenze. «Avrò
bisogno del tuo aiuto, Switch. Sono passati mesi dall'ultima volta che
ho usato una spada e credo di essere fuori esercizio.»

«Stai dicendo di aver bisogno di qualcuno che ti sbatacchi un po' in
giro?» Un'espressione decisamente gongolante apparve sul volto del mio
amico e io avvertii un nodo allo stomaco pur rispondendo con uno dei
miei sorrisi in tralice.

Da dove era spuntato questo giovane mirmidone che avevo davanti? Era
come se qualcuno avesse fatto sparire il mio amico Switch e lo avesse
sostituito con un essere fatato, come nelle storie che mia madre
condivideva con me. Quanto era cresciuto in quegli anni! Ero stato io a
fargli questo? No, era sulle sue gambe che si reggeva, io lo avevo solo
tirato in piedi.

«Had?» Switch mi stava osservando con le sopracciglia aggrottate. «Stai
bene?»

Stavo fissando, non lui ma il libro sul tavolo, con la sua immagine
piena di ombre cupe raffigurante una giovane coppia minacciata dalle
caricature distorte dei Cielcin le cui ombre ricadevano intorno alla
coppia umana, con la donna che tremava e l'uomo pallido di terrore. In
primo piano c'era una singola rosa, la sola macchia di colore sullo
sfondo chiaroscuro di una copertina da incubo, rossa come sangue
arterioso. E accanto a essa c'era una mano dalle dita stranamente
distorte e spezzate, che si protendeva come per afferrarla. Anche se ho
dimenticato il titolo del libro, non ho mai dimenticato quella mano, o
quella rosa.

«Cosa?» Guardai Switch dritto negli occhi. «Sì... credo di sì.»

Quello non era il momento di andare in pezzi, di riversare tutte le mie
trepidazioni sul tavolo in mezzo a noi. Potevo sentire su di me lo
sguardo della mia scorta, duro e privo di comprensione, e il peso di
tutto il cemento circostante mi gravava sulla testa come la Spada Bianca
di un cathar. «Giovedì potrei morire.»

«Ma non lo farai.» Switch non suonò conciliante e neppure amichevole,
pronunciò quelle parole come se stesse enunciando un dato di fatto. «Ho
visto quel prete. È un mutante, tutto deforme. Però è come dici: non
vuoi ucciderlo, quindi rifilagli una bella ferita e falla finita.»

Il primo sangue. Avrei potuto ridere. Il primo sangue, e avrei potuto
chiudere lì il duello. Nessuno sarebbe morto, la legge sarebbe stata
soddisfatta e avrei dichiarato di esserlo anch'io. I modi di Switch si
fecero d'un tratto allegri. «Potrebbero anche accattivarti le simpatie
di questa tua signora, per la galanteria e tutto il resto. L'ho visto
succedere.»

«Non lei» ribattei, appoggiando il mento sulle braccia.

«Troppo orgogliosa?»

«Troppo...» Non riuscii a trovare la parola giusta. «È una Tavrosiana.»

Switch sollevò di scatto le sopracciglia. «Una Demarchica? Davvero?»

Sollevai una mano in segno di conferma e mi alzai in piedi. «Se verrai
al palazzo vicino alle porte principali ti incontrerò al barbicane
riservato al pubblico. Pallino conosce la strada.» Avrei voluto
aggiungere che Pallino e io avevamo ripreso a parlare della nave, volevo
approfittare di quell'opportunità per parlare senza le onnipresenti
videocamere di palazzo, ma non era destino che fosse così e in ogni caso
sarebbe stato sbagliato affrontare l'argomento che ci aveva separati
tanto presto dopo la nostra riconciliazione. Battei le nocche sul tavolo
in un cortese applauso. «Grazie, Switch.»

Non sapendo che altro dire infilai le mani nelle tasche dei pantaloni e
mi avviai verso la porta e il palazzo. E il futuro. E il duello.

«Mi stavo chiedendo... qual è il tuo nome?» mi gridò dietro Switch. «Il
tuo vero nome?»

Sorrisi, il più classico sorriso in tralice mai apparso sul mio volto
affilato. «Oh, \emph{è} Hadrian» risposi. «Hadrian, del Casato Marlowe
di Delos.»

Anche se per me era significativo assumere di nuovo il mio vero nome,
Switch accettò la cosa solo con una scrollata. «Sembra appropriato.»

Reagii con una vuota risata e mi girai per metà verso le guardie.

«È appropriato, sai?»

«Sì, immagino lo sia» replicò Switch.

«Allora ci vediamo domani?»

«Domani.» Abbassò la testa. «Arrivederci, Hadrian.»


