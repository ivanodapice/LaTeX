\chapter{Mostri}

Mentre scendevo in fretta la rampa, con una cintura-scudo che mi
appesantiva la vita, stentavo a credere che avesse funzionato. Vriell
era andata con Elomas e con gli altri -- perfino Valka non aveva
protestato per quella ritirata a Fonteprofonda -- lasciandomi affidato
al suo aiutante e comandante in seconda, un optio che non conoscevo
dalla pelle scura come il carbone che mi portò insieme alla sua decuria
di soldati presente sul nostro velivolo attraverso il panorama primitivo
coperto di erba e di muschio, fino al posto dove la nave dei Cielcin
giaceva fumante come un dito caduto da un cadavere in fiamme, fracassata
all'estremità di un grande solco nel terreno scavato dall'impatto e
dalla strisciata con cui si era fermata. Mentre ci avvicinavamo
rallentai il passo, sopraffatto dalla realizzazione che prima di allora
non avevo mai visto una nave così grande e terribile. Intatta doveva
aver misurato mezzo chilometro da un'estremità all'altra e adesso era un
rottame fumante di metallo annerito dal calore e di qualcosa che,
incredibilmente, sembrava pietra. Non era affatto un castello di
ghiaccio... quale ghiaccio sarebbe sopravvissuto al contatto con
l'atmosfera di {Emesh}?

Le fiamme ardevano ancora e alcune figure spiccavano illuminate dai
pochi velivoli jaddiani che ancora giravano nell'aria, neri sullo sfondo
della notte, con i repellenti vibranti che risplendevano azzurri sopra
di noi. Una strana mescolanza di mamelucchi jaddiani in blu e arancione
e di legionari imperiali dall'armatura avorio e dal tabarro rosso si
aggirava sulla scena, supportata da qualche membro del personale di
Mataro, in verde e oro. Dritto davanti a noi c'era un uomo in aderenti
abiti di pelle la cui ampia veste e voluminosa manica sinistra vuota si
agitavano al vento.

«Sir Olorin!» chiamai, oltrepassando l'optio con una mano sollevata in
un gesto di saluto.

Il maeskolos, che pareva avere il comando, distolse l'attenzione dal
legionario che aveva accanto per guardare verso di me. Un sopracciglio
appuntito si sollevò per poi abbassarsi con fare sospettoso. «Lord
Marlowe, giusto?»

«Sì, \emph{domi}» risposi, con le mani intrecciate davanti a me.

Il legionario accanto a sir Olorin si colpì la corazza con il pugno in
una parodia di saluto mentre la visiera e l'elmo dalla pronunciata
protezione per il collo collassavano come una scultura di carta a
rivelarne la faccia. «Cosa ci fai tu qui?»

«Tenente Lin!» Inclinai il capo in un saluto meno solenne. «Mi fa
piacere vederti.»

Bassander Lin ricambiò il gesto e si passò una mano fra i ridicoli
capelli. I lati apparivano rasati di fresco, ma la sommità era un
groviglio del colore del fumo che cercò di domare senza successo,
apparendo molto più infastidito che non al banchetto del conte. «Perché
sei qui?»

«Posso essere di aiuto» affermai, ripetendo quello che avevo detto alla
centurione non più di una decina di minuti prima. «Parlo la lingua dei
Cielcin. Non in modo scorrevole, bada, ma quanto basta. Ho pensato che
se ci sono dei superstiti vi avrei potuto aiutare a negoziare.»

«Negoziare?» mi fece eco Bassander, incupendosi in volto. «Con gli
xenobiti? Sei pazzo?»

Olorin sorrise con calore, simile alla caricatura di un diavolo cordiale
con i suoi lineamenti che erano tutti esagerati. L'effetto avrebbe
potuto riuscire sconcertante in un uomo meno carismatico ma conferiva al
maestro di spada un certo fascino composto. «Non sappiamo se una
qualsiasi di quelle bestie sia sopravvissuta. Dobbiamo dare la priorità
al contenimento della minaccia, a mio parere.» Si batté un colpetto sul
naso.

«Se possiamo evitare spargimenti di sangue, credo che sia per noi un
obbligo provarci» tentai ancora, consapevole di quanto fosse delicata la
mia posizione.

«\emph{Ben jidaan}!» esclamò Olorin. «Credi di poter ragionare con
loro?»

Lanciai un'occhiata a Bassander e deglutii a fatica. «Non lo potrò
sapere a meno di provarci. Conosco le parole.»

Il maeskolos incrociò le braccia con il mento abbassato come un pugile.
«Allora va benissimo.»

«Invece no» interloquì Bassander, aggredendomi verbalmente. «Sei forse
il loro comandante? Se dovessimo trovare un superstite cosa farai? Gli
dirai: `Getta le armi e arrenditi a queste persone'?» Rise con
disprezzo. «Mi dispiace, lord Marlowe, ma cosa ti induce a credere di
poter essere di un qualsiasi aiuto qui? Non sei un soldato.»

«No, in realtà no, ma...» Non sapendo dove altro voltarmi guardai verso
Olorin in cerca di aiuto.

«Allora cos'hai intenzione di fare? Convincerli a sottomettersi? Nel tuo
stesso interesse, torna dai tuoi nobili. Questo è un lavoro da soldati.»
Il tenente \emph{sorrise} nel dirlo e scosse il capo, guardando per un
momento verso il cielo mentre uno degli strani velivoli organici degli
jaddiani ci sorvolava. Si rivolse quindi all'optio -- ora proprio dietro
di me -- e a due dei suoi mamelucchi. «Soldati, riportate lord Marlowe
alle navette e trattenetelo là per la sua stessa sicurezza. Questo non è
posto per un palatino.»

«Non sono un bambino!» Avendo apparentemente ricevuto l'ordine di
obbedire all'ufficiale della Legione, il primo mamelucco mi posò con
gentilezza le mani sul braccio e sulla spalla, facendomi girare. La
trama di carbonio delle dita asciutte e troppo sottili era dura, ma mi
liberai. «Posso parlare con loro! Per favore.» La creatura tornò a
chiudermi le mani sul braccio, questa volta con forza maggiore.

«Parlare con loro?» ripeté Olorin. «Cosa ti fa pensare di poterli
indurre a parlare?» Una ruga profonda gli increspò il volto angoloso
come se quella domanda lo turbasse profondamente.

«Posso parlare con loro» ribadii, e cercai di liberare il braccio con
una torsione, ma il mamelucco mantenne la presa. Sollevai allora il
braccio di scatto, raggiungendolo con il gomito nel punto morbido sotto
il mento e spingendogli con violenza la testa all'indietro. La creatura
si piegò come un punching ball giocattolo per bambini e piegò le
ginocchia fino a essere quasi parallela al terreno. Non emise nessun
suono, non grugnì neppure, poi si raddrizzò sempre in silenzio. Mi
spostai lateralmente di un passo e quasi balzai all'indietro quando esso
si riprese. «Datemi una possibilità.»

Olorin sollevò una mano per fermare il suo servitore. «Questa non è una
risposta.» Il suo cipiglio si accentuò, scavando ombre lungo entrambi i
lati della sua bocca a punta.

Guardai implorante verso Bassander. «Al momento stiamo discutendo di un
problema inesistente» affermò lui. «Non possiamo fare a meno di
personale per sorvegliare lord Marlowe ma dobbiamo metterlo al sicuro.
Rinchiudetelo in una navetta, se necessario.»

I mamelucchi si erano immobilizzati entrambi. Quello che avevo colpito
era fermo là e fletteva le dita come in cerca di una gola. Il cappuccio
gli era ricaduto all'indietro nello scontro, rivelando una cuffia di
nanocarbonio nero che gli copriva il cuoio capelluto, il collo e le
orecchie, ancorando la maschera della tuta. Le lenti nere dei suoi occhi
mi osservavano, morte come gli occhi di una bambola. Rabbrividii e le
labbra di Bassander si contrassero in un'espressione di empatia. I
mamelucchi erano cose fredde, irreali, come se da dietro quella maschera
fosse potuta uscire una colonia di ragni al posto di un uomo.

Mi raddrizzai e cercai di chiamare a raccolta tutta la dignità
possibile. «Possiamo fare di \emph{meglio}.» Mentre parlavo tenni lo
sguardo sul mamelucco senza cappuccio, pronto a reagire se avesse
tentato... qualcosa. «Le guerre non si vincono con i soldati, signore, a
meno di essere disposti a uccidere ogni singolo nemico nella galassia.
Le guerre si \emph{combattono} con i soldati, ma si vincono con le
parole.» Avrei rimpianto quell'affermazione -- la sua ingenuità -- e
mentre la scrivo qui il mio cuore si annerisce per l'ironia e l'amara
consapevolezza che mi sbagliavo. «Un giorno dovremo cominciare a parlare
con loro.»

Tanto Olorin quanto Bassander avevano un'espressione su misura, un po'
confusa. Per la Terra e l'imperatore, erano così... limitati. «Questo si
avvicina pericolosamente all'eresia, signore» affermò infine il tenente
imperiale.

«\emph{È} eresia» ringhiai. «Farai rapporto su di me?» Mi girai di
scatto verso Olorin. «E tu?» In quel mio momento di distrazione il
mamelucco guadagnò mezzo passo verso di me e io sollevai le mani,
{tenendomi} pronto. «Ordina al tuo omuncolo di ritirarsi, dannazione!
Basta così!»

Olorin sibilò qualcosa in jaddiano e il soldato si immobilizzò. Con il
volto ancora contratto in un'espressione determinata fissai lo sguardo
sul maestro di spada straniero. La sua caricatura della faccia di un
demone riluceva, improntata a un ironico divertimento inaspettato come
un fulmine. «Meglio? Dobbiamo fare meglio?»

Prima che potessi rispondere un ufficiale jaddiano nella familiare
armatura d'ottone e veste a strisce sopraggiunse di corsa. Avrebbe
potuto essere un mamelucco a giudicare dallo stile dell'uniforme e dalla
snellezza del suo corpo, ma le curve dei fianchi e del seno tradivano il
fatto che non si trattava di un androgino ma di una donna umana.
«\emph{Dom Olorin, domi}» salutò in jaddiano, poi cominciò a parlare
tanto in fretta che non riuscii a seguire quello che diceva.

Olorin sollevò una mano. «Per favore, togliti il casco quando parli con
me, Jinan» ordinò.

In risposta a quel gesto la maschera cromata e il cappuccio si aprirono
e si ritrassero, mettendo a nudo la sua testa e rivelando un'umana
affannata dal volto ovale, con riccioli scuri e la carnagione olivastra
propria di Jadd. Lanciò un'occhiata a Bassander, mi rivolse un sorriso
incerto, poi continuò il suo discorso con un accento tanto marcato che
ebbi difficoltà a seguirlo. Intanto continuò a guardarmi, e a un certo
punto chiese chiaramente cosa ci facessi lì. Olorin accantonò la
domanda. La donna aveva un nastro di seta azzurra avvolto nei capelli
che erano intrecciati e le cingevano la testa come una corona. Anche se
a quel tempo non la conoscevo, il suo nome era Jinan Azhar ed era la
comandante in seconda di sir Olorin. Attesi accanto a Bassander,
guardandola parlare e ammirando la sua alta statura.

Il mio capitano.

È strano pensare che quanti ci sono cari un tempo erano sconosciuti,
solo un'altra persona in un mare di facce, ed è ancora più strano
pensare che li avremmo incontrati di nuovo con piacere, pur consapevoli
di tutto il dolore che quell'incontro avrebbe portato.

«Cosa c'è?» domandò Bassander, nel momento in cui l'ufficiale jaddiano
ebbe finito di parlare. «Cosa succede?» Agganciò una mano
all'impugnatura della spada di ceramica che portava al fianco sinistro.
«Si tratta dei Cielcin? Un rapporto per te dalle tue... creature?»
Guardò verso i due mamelucchi fermi poco lontano.

«Grazie, tenente Azhar» disse Olorin, poi assestò i drappeggi della
mandayas carminia e sollevò le pieghe del suo colletto a ventaglio per
meglio coprire il collo sottile. «I mamelucchi non fanno rapporti»
precisò. «Non a me. Sono \emph{haqiph}.» Tradussi quel termine con
`ripugnanti' non riuscendo a capire, dato che pronunciò quella parola
senza malizia. «Le mie squadre hanno aperto un varco fin dentro la nave
e pare che non ci sia nessuno xenobita a bordo.» Nel dirlo mi guardò in
modo significativo.

«Nessun Cielcin?» chiese l'optio, parlando per la prima volta. «Su una
nave di quelle dimensioni?»

«Ci deve essere spazio per centinaia di loro» sussurrò Bassander.
«Questo non ha senso. Perché avrebbero dovuto lanciare una nave vuota
contro il pianeta?»

Io stavo scuotendo il capo. «È stata abbattuta, giusto?» Quando nessuno
rispose cambiai tattica. «C'erano dei Cielcin ancora nelle capsule
criogeniche?»

«Morti» rispose la tenente Azhar. «Tutti morti.»

«Hanno abbandonato la nave all'ingresso nell'atmosfera» affermai con
assoluta sicurezza. «Ho visto i bagliori azzurri e ho pensato che
fossero correzioni di rotta, ma dovevano essere capsule di salvataggio.»

«Non esistono capsule di salvataggio!» esclamò Bassander, con la voce
che saliva di tono per l'esasperazione. «La maggior parte delle volte
non c'è dove fuggire.»

«Allora erano navette» ritorsi in tono secco.

Olorin scosse il capo. «Le avremmo individuate di certo.»

«Ne sei tanto sicuro? La prima incursione dei Cielcin nel sistema è
stata annientata nell'eliopausa e questa è arrivata fin qui senza che
nessuno lo notasse. Non avete forse una flotta parcheggiata accanto
all'\emph{Incrollabile}?» Sir Olorin spostò il peso del corpo da un
piede all'altro, sbilanciato. Feci schioccare le dita e puntai l'indice
verso di lui. «Allora è possibile?» Non sapevo praticamente niente di
astronavigazione o di tattiche militari orbitali, ma era solo logico che
fosse così.

Bassander si grattò vigorosamente i capelli arruffati. «Lampi azzurri,
hai detto?» Glielo confermai e lui si mordicchiò l'interno della
guancia, poi imprecò furiosamente. Olorin e io ci scambiammo un'occhiata
quando lui si girò di scatto. Impiegò un momento a ricomporsi e quando
tornò a voltarsi aveva di nuovo la normale espressione cordiale, anche
se sfumata di una sorta di sofferenza. «Dove possono essere andati?»

Sorrisi perché sapevo che stavo per averla vinta. «C'è un solo posto
dove possono essere andati.»

\begin{figure}
	\centering
	\def\svgwidth{\columnwidth}
	\scalebox{0.2}{\input{divisore.pdf_tex}}
\end{figure}

Ci volle quasi un'ora per trovare la navetta dei Cielcin, una cosa scura
simile a uno scarabeo e grande quanto un autobus cittadino. Non emetteva
segnali di calore e non era individuabile con il radar, segno di un
qualche tipo di campo di schermatura. Lasciai che la spiegazione di
Bassander mi scivolasse sopra: non era per quello che ero là.

Il tunnel che si apriva nel terreno era uno dei condotti di ventilazione
che perforavano la distesa rocciosa, in parte sfondato dal passaggio di
quasi un milione di anni, e scendeva tanto di piedi e braccia fino a
unirsi al sottostante complesso labirintico di Calagah. A quanto pareva,
la Quiete aveva avuto bisogno di respirare, oppure quei condotti avevano
avuto qualche altra funzione. Erano pochi, alcuni tanto profondi che una
caduta dentro uno di essi avrebbe ucciso un uomo, altri nei quali era
abbastanza sicuro saltare. Questo apparteneva alla prima categoria,
quindi perdemmo tempo per calare nelle gallerie quattro decurie di
soldati, assicurati uno all'altro come in una cordata. Avrei voluto
protestare contro una tattica del genere, facendo notare gli spazi
ristretti e quanto fosse difficile muoversi negli angusti tunnel della
Quiete, ma Olorin aveva visto in precedenza quel labirinto e non ne
volle sapere.

Fui fra gli ultimi a scendere, disarmato e senza corazza tranne che per
lo scudo. Con uno scatto deliberato attivai la sua cortina di energia e
sentii il formicolio della statica che mi faceva rizzare i capelli. La
luce si piegò per un momento attraverso il campo Royse quando esso si
attivò, distorcendo ai miei occhi la sala vagamente trapezoidale.

«Da che parte, Marlowe?» chiese il tenente Lin, omettendo il mio titolo.

Nel sottosuolo l'aria era immobile, in contrasto con i venti della
superficie. Eravamo scesi in una delle camere di intersezione {rotonde}
ed eravamo al centro di cinque passaggi divergenti. Non sapevo bene dove
fossimo, ciascuna galleria era molto simile alle altre e gli interstizi
presentavano solo differenze molto sottili. Alla fine scrollai le
spalle. «Siamo in alto, molto al di sopra delle camere principali.
Suggerisco di dirigerci verso il basso.» Indicai due dei passaggi,
ciascuno dei quali si inclinava visibilmente verso il basso nel buio.

«Quale dei due?» chiese ancora Bassander. Nella penombra la sua armatura
color avorio pareva quasi risplendere e la doppia striscia rossa del suo
grado aveva il colore del sangue secco nel corrergli lungo il braccio.

«Abbiamo uomini a sufficienza per entrambi? Non potevano esserci più di
una dozzina di Cielcin su quella navetta.»

Sir Olorin annuì. «Noi prenderemo quello di destra, tenente.»

«Molto bene.» Bassander trasmise le istruzioni con una serie di secchi
gesti sopra la testa e immediatamente i suoi soldati si separarono dai
mamelucchi jaddiani e dai loro pochi ufficiali per formare una colonna e
imboccare il corridoio. Ogni uomo impugnava un distruttore a fase la cui
fenditura verticale scintillava rossa, a indicare che erano regolati per
uccidere. Si avviarono il più silenziosamente possibile nell'imboccatura
della galleria mentre Bassander aggiungeva: «Prenderò Marlowe con me.»

Olorin scosse il capo. «Inutile. Terrò d'occhio io l'\emph{amralino},
amico mio.» Serrai la mascella, soffocando il desiderio di protestare.
Dopotutto, mi ero ficcato io in quel pasticcio, quindi lasciai che mi
portassero con loro.

I mamelucchi si muovevano più silenziosamente dei legionari e lo
facevano senza le luci della tuta che gli imperiali usavano per vederci.
Lo stesso Olorin non pareva disturbato dall'oscurità, quindi scesi
incespicando in quelle profondità seguendo la galleria in una spirale
che descriveva un arco verso il basso e verso sinistra. Quando inciampai
per la terza volta qualcuno mi prese per l'avambraccio e mi sospinse
lungo il condotto. All'inizio pensai che fosse uno dei soldati omuncoli,
ma il rumore del respiro nelle mie orecchie era leggermente affannoso,
quindi era umano, di un vero umano. Decisi che si trattava di uno dei
tenenti, forse la donna che avevo visto in precedenza, la tenente Azhar.
Mi {sentivo} esposto in compagnia di tutti quei soldati dotati di
corazza, infiammato come un nervo. Per la Terra e l'imperatore, in che
cosa mi ero ficcato?

L'aria si aprì intorno a noi, segno che eravamo entrati in una camera, e
il lieve rumore di passi svanì invece di restituirci la sua eco. Ci
fermammo, o almeno lo fece la mia guida. «Luce» sussurrai. «Ci serve una
luce.»

Una voce femminile mi risuonò all'orecchio, soffocata, vicina e con un
forte accento. «Noi ci vediamo.»

«Loro possono vederci \emph{meglio.}»

«Se sono qui» replicò Olorin, a bassa voce. «Taci.»

Qualcosa di pesante sbatté contro la pietra alle mie spalle e mi girai
di scatto giusto in tempo per vedere un bagliore di luce dorata
nell'oscurità, riflesso dallo scudo della mia guida. Uno dei mamelucchi
era stato atterrato e qualcosa di massiccio era accoccolato su di lui,
scuro contro il nero scintillante del pavimento di pietra.
Quell'immagine mi aderì al cervello, più nitida del suono del collo del
mamelucco che si spezzava.

Le fiamme degli storditori illuminarono la bestia come fulmini, ma la
sua armatura ne respinse le scariche senza difficoltà mentre si girava
di scatto, nove piedi di muscoli pallidi in policarbonato flangiato, con
la faccia simile al teschio bianco della morte, occhi che sembravano
orbite marce e denti simili a file di coltelli di vetro. Era sangue
quello sulla sua bocca? Indietreggiai incespicando e la tenente Azhar si
gettò fra me e la creatura. Alcuni lampi giunsero dalla mia destra e nel
girarmi vidi altri due Cielcin scattare attraverso l'oscurità, facendosi
più vicini a ogni bagliore. Mi serviva una lama, un'arma. Qualsiasi
cosa. Il mio regno per un cavallo, pensai e il mio cervello frastornato
rise di quell'assurdità. \emph{Mantieni il controllo. Mantieni il
	controllo.}

Indietreggiai e intanto sentii Olorin gridare in jaddiano. «Storditeli!
Storditeli!»

Un raggio stordente raggiunse in piena faccia la creatura con la bocca
insanguinata che barcollò, rallentando il passo, ma non cadde. Nella
postluminescenza fosforescente di tanti raggi stordenti i capelli del
Cielcin parvero risplendere, raccolti in una spessa treccia, bianca come
quella di Ligeia Vas. Per un momento fu tutto quello che potei vedere,
poi un secondo raggio lo colpì in faccia seguito da un terzo ed esso
crollò su un ginocchio gemendo di dolore. Cercò di rialzarsi.

Qualcuno urlò. Era una voce umana, quella di uno dei tenenti. Non smise
di urlare e la sua voce rimbalzò contro le pareti e le colonne inclinate
di quello spazio echeggiante fino a riempire l'universo, alta e acuta e
incessante. Se stava pronunciando delle parole, non le conoscevo.

«Luce!» gridai, facendo leva su tutta la forza e l'aria che mi
rimanevano e sul mio passabile jaddiano. «\emph{Fos}! \emph{Fos}!
\emph{} Luce!»

La tenente Azhar fece eco al mio grido e raggi bianchi emanarono dalle
corazze e dai guanti di tutti i soldati jaddiani, dei mamelucchi e del
tenente. La luce divampò perfino sul corpo dei tre mamelucchi morti.
Quattro Cielcin -- perché ce n'erano ben quattro -- ulularono e
barcollarono, accecati, indietreggiando con le mani dalle dita troppo
lunghe e troppo numerose sollevate davanti alla faccia e sibilando come
un oceano che si tramuta in vapore. Per un momento riuscii a stento a
trattenermi dal sorridere. Aveva funzionato. Si erano fermati.

Ebbi così un orribile momento di tempo per contemplare la forma che si
contorceva del tenente urlante, i cui arti si agitavano mentre ghermiva
il proprio petto e il terreno circostante senza che le sue dita
trovassero un appiglio. Le sue strida erano amplificate dagli
altoparlanti della tuta e riempivano l'aria, scuotendo le colonne stesse
che sostenevano la volta di quel luogo alieno. Il sangue gli fiottava
dal collo attraverso un buco aperto a morsi passando il tessuto di
nanocarbonio della sua tuta e chiazzando quasi di nero la veste a
strisce arancione e blu che indossava. Poi le sue strida cessarono e un
umido ronzio riempì l'aria, misto ai gemiti e ai sibili dei Cielcin che
lottavano contro la loro cecità e il fuoco degli storditori.

Qualcosa emerse dal buco nel collo del tenente, una cosa umida che si
contorceva, un infuriato serpente bianco con una punta da trapano che
ruotava ancora nell'aria dove ci sarebbe dovuta essere la bocca.
Qualcuno imprecò in jaddiano mentre la cosa si tirava fuori e si levava
come il fumo di un sacrificio nell'oscurità rischiarata dalle tute.
Lunga quanto il mio avambraccio e larga mezza volta di più, fluttuò
nell'aria e volò verso il bersaglio più vicino... solo per cadere a
terra divisa in due pezzi fumanti.

Sir Olorin impugnava con noncuranza la spada di altamateria in una mano
guantata, con il braccio sinistro ancora infilato nella mandayas. La
lama di altamateria leggermente luminescente si increspava nell'aria,
azzurra come un cristallo, come acqua di mare, come la luce della luna.
Mai fissi, gli strani atomi della lama scorrevano uno sull'altro e il
suo filo era sottile come l'idrogeno.

Due dei Cielcin erano a terra, storditi. Un terzo era accoccolato sul
corpo di un altro mamelucco e continuava a sbattergli la testa contro il
terreno, indifferente alle scariche degli storditori. Il quarto si
scagliò contro Olorin. Il maeskolos si spostò, fluido come acqua, come
il metallo della sua lama, descrivendo un arco aggraziato intorno
all'alieno e ruotando il polso per vibrare un colpo preciso con la
spada. Fece uno sforzo minimo, lo stesso con cui avrebbe potuto aggirare
un cliente importuno in un negozio, e non rimosse neppure il braccio
sinistro dalla sua posizione noncurante all'interno della veste. Giuro
che scrollò le spalle mentre la corsa della creatura la portava
attraverso la lama. Essa si immobilizzò per un momento con
un'espressione sconvolta e quasi confusa sulla faccia vagamente
familiare, poi il suo torso si divise scivolando in diagonale e finì sul
terreno. Le gambe lo seguirono un momento più tardi.

«Non li uccidere tutti!» gridai, muovendo passi esitanti verso il punto
in cui quattro silenziosi mamelucchi lottavano per bloccare il Cielcin
che i raggi stordenti non erano riusciti ad abbattere. «Dobbiamo
parlare!»

«Parlare?» ripeté Olorin con uno scatto del polso in reazione al quale
le molecole azzurrine della lama svanirono lasciandosi dietro un vago
odore di ozono dove avevano ionizzato l'aria circostante.

Lo ignorai e descrissi un ampio cerchio intorno al Cielcin che lottava
ancora con i mamelucchi. «\emph{Iukatta}!» gridai. Era una parola di cui
ero assolutamente sicuro, quindi la pronunciai con autorità. `Smettila!'

Lo shock di sentir parlare la sua lingua ridusse il Cielcin
all'immobilità quanto un raggio stordente avrebbe fatto con un essere
umano. Sbatté le palpebre e girò la testa protetta dal casco per
guardare nella mia direzione, inclinandola da un lato in un gesto
stranamente umano. «Perché siete qui?» domandai, sempre nella sua
lingua, poi ripetei la domanda alzando la voce. «\emph{Tuka'ta detu
	ti-saem gi ne}?»

Nessuna risposta.

«Dove sono gli altri? In quanti siete?» Mi fermai appena fuori dalla
portata di un eventuale scatto in avanti dello xenobita, confidando
nella capacità dei soldati di trattenerlo anche se era più alto di loro
sebbene fosse in ginocchio. I mamelucchi gli tenevano le braccia
bloccate e piegate all'indietro, pronti a spezzargli le spalle se avesse
opposto resistenza. Quando ancora rifiutò di rispondere, continuai: «I
miei amici ti uccideranno, lo capisci?» Niente. La visiera del casco era
grigia e a specchio, assolutamente priva di espressione o di dettagli.
«Rispondi alle mie domante e ti giuro che sarai trattato equamente.»

«Equamente?» L'inumano emise un forte verso stridulo. Sapevo che mi
stava guardando. «Equamente?»

«Perché siete qui?» ripetei. «Perché venire qui? In questo posto?» Mi
girai di qua e di là, ad abbracciare lo spazio cavernoso che ci
circondava. «\emph{Detu ne}?» `Perché?'

Il Cielcin ringhiò attraverso il casco e cercò di spingersi in avanti
solo per emettere un gemito quando i mamelucchi gli torsero le braccia
sottili. «Non è per voi!» Quelle parole erano tanto diverse da qualsiasi
risposta mi fossi aspettato che rimasi stordito, con le mani immobili
nell'atto di formare un gesto, come una marionetta comandata da un
burattinaio distratto. Per puro caso, fu proprio la cosa giusta da fare,
perché il Cielcin continuò: «Questo è un luogo sacro.»

«Voi adorate la... coloro che hanno costruito tutto questo?» Le immagini
che avevo scorto nella mia visione tornarono ad affiorare: i Cielcin in
mezzo alle stelle, le loro schiere lucenti sovrastate dall'enorme nave e
la luce che distruggeva il sole.

«Non è per voi» ripeté il Cielcin.

«Cosa sta dicendo?» chiese la tenente Azhar.

La liquidai con un cenno, con l'attenzione concentrata interamente sulla
creatura prigioniera davanti a me.

«Cosa sta dicendo?» domandò Olorin, rendendosi conto che la mia
precedente millanteria non era affatto tale.

Rimasi concentrato sul Cielcin e, in preda a un'ispirazione, dissi:
«Loro vogliono farti del male.» Avanzai di un passo e mi abbassai
accanto al cadavere di un mamelucco per il tempo necessario a rimuovere
il suo disintegratore a fase dalle dita scheletriche. Controllai che
fosse regolato per stordire, ricordando quando il prefetto-ispettore Gin
aveva minacciato la banda di Rells fuori da quel negozio d'angolo di
Borosevo. Ricordai anche la negoziante che avevo accoltellato, il
portuale a cui avevo spezzato il braccio e Crispin che giaceva a terra
sanguinante, e Gilliam, morto ai miei piedi. «Io te ne farò.» Non ero
certo di poterlo fare. I disintegratori neurologici a fase sono arnesi
sgradevoli. Settati a un alto livello possono carbonizzare ogni cellula
nervosa del corpo. A un livello basso possono provocare una perdita di
conoscenza... o dolore.

Non fu difficile capire come funzionava l'antiquato sigillo che chiudeva
la più primitiva armatura del Cielcin e nel rimuovere il casco riflettei
che la Cappellania non si sbagliava, che i Cielcin ci erano decisamente
inferiori sotto molti aspetti. A elevarli erano la loro tenacia e pura
ostinazione. Il sigillo era del tipo che avevo visto nei drammi storici
sugli inizi dei voli spaziali e il casco era un arnese voluminoso fatto
di materiali comuni. Niente nanocarbonio o ceramica. Il rivestimento
dell'armatura era di metallo, goffo, pesante e con eccessive funzioni di
progettazione.

«Marlowe...» mi interruppe Olorin. Non riuscii a decifrare il suo tono
perché non mi rimaneva attenzione da rivolgere a lui.

Faccia a faccia, il Cielcin appariva rinsecchito. Non aveva capelli e la
corona di corna sulla sua testa era stata limata fino a ridurla a
protuberanze arrotondate. Le quattro fessure delle sue narici si
dilatarono. «Non ti temo, \emph{yukajji-do.}»

`Io sì' avrei voluto rispondere, e serrai il pugno per evitare che il
disintegratore tremasse mentre glielo appoggiavo contro la fronte. «Dove
sono gli altri?» domandai.

«Non ci sono altri.»

Feci fuoco.

Il Cielcin barcollò all'indietro, mettendo a nudo i denti trasparenti e
affilati come vetro nelle gengive nere quando le labbra gli si
ritrassero in una smorfia. Non era stordito, era a stento intontito e
quegli occhi completamente neri mi guardavano fissi. C'era disprezzo
nelle loro profondità? Sfida?

Non ero in grado di decifrarli. Adesso la mia mano \emph{stava}
tremando. La creatura lo vide... lo videro tutti. «In quanti siete qui?»
Non attesi una risposta e premetti di nuovo il grilletto, con la mano
che sobbalzava quando il Cielcin si ritrasse di scatto, con le braccia
che si tendevano dolorosamente nella stretta decisa dei mamelucchi.

«\emph{Ubimnde}!» ansimò, con il respiro alquanto affaticato.

«In undici?» ripetei, poi lo dissi di nuovo in jaddiano a beneficio dei
presenti nella stanza. «Dove?» Una parte di me era convinta che avrei
potuto continuare, insistere, ma quella parte non aveva informato la mia
mano, che agitò il disintegratore. Sparai una terza volta, colpendo il
Cielcin alla faccia. Quando si accasciò gemendo ripetei la mia domanda.
«\emph{Saem ne}?»

Avevo sentito storie di persone che morivano durante un interrogatorio,
di soldati che combinavano un pasticcio per via di quanto erano poco
abili se paragonati ai cathar e avevo sempre pensato che quelle storie
non fossero credibili e tuttavia adesso ero lì, lieto che Valka non mi
potesse vedere, e anche se avvertivo la vergogna nel suo sguardo pregai
che non venisse mai a sapere di questo momento. Avvertivo il suo
disprezzo per la violenza, per me, e abbassai l'arma. Cercai di dirmi
che quello che stavo facendo non era \emph{davvero} tortura, che la
creatura si sarebbe ripresa, non sarebbe stata come gli storpi che
affollavano i vomitoria del Colosso con la ciotola delle elemosine in
mano.

Non era così.

Le menzogne che ci diciamo per proteggerci da noi stessi.

Abbassai l'arma.

«Dove sono?»

