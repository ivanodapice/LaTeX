\chapter{Perdere le stelle}

Non riesco a ricordare quasi niente degli incontri a cui assistetti quel
giorno nel Colosso. A parte Anaïs e Dorian, non rammento neppure quasi
nessuno dei palatini e dei patrizi appartenenti all'alta società emeshi
che ebbi modo di incontrare: ogni faccia perfetta scolpita nei toni del
tek, del bronzo e dell'avorio si fonde con la successiva. Per me sono
senza nome e senza volto come lo erano per loro i mirmidoni e gli
schiavi che si trovavano nell'arena sottostante. E furono proprio quei
combattenti ad attirare la mia attenzione, non i frivoli rapporti
sociali. Alis e Light erano nuovi e più o meno non ancora messi alla
prova, altri quattro li conoscevo solo di vista -- li avevo visti
mangiare alla mensa non più di quindici giorni prima -- e poi c'era
Erdro, che aveva combattuto con me in quel primo giorno nel colosseo. Mi
piaceva abbastanza. Era il genere di mirmidone che aspirava alla
posizione di gladiatore, faceva della sua forma fisica una scienza e
mangiava con un cucchiaio dosatore.

Niente di tutto questo ebbe importanza. La prima freccia rimbalzò contro
la sua corazza, strappando un sussulto alla folla, seguito da un
applauso quando lui continuò la sua carica contro Jaffa, il capitano dei
gladiatori dalla pelle nera come l'inchiostro. Il gladiatore però si
limitò a caricare di nuovo l'antiquata balestra ed Erdro morì. La folla
applaudì anche per questo, stesso suono e stessa inflessione. Due
mirmidoni si abbatterono su Jaffa e lo colpirono finché la sua tuta non
lo bloccò e un paio di servitori dovettero trascinarlo, paralizzato,
fuori dal campo. Avrebbe riportato dei lividi, mentre avrebbe dovuto
essere morto.

Mentre sedevo lì in mezzo a tutti quegli ori, alle sete e ai velluti,
fui assalito da una sensazione familiare: il desiderio di andarmene. Le
scie di condensazione dei razzi striavano il cielo a sud, sopra la terra
piatta dell'isola artificiale che si stendeva appena al di là della rete
di canali di Borosevo. Le linee aggraziate dell'astronave uhran si
delinearono sullo sfondo nero dei miei pensieri mentre sedevo
impassibile come la pietra sotto al ronzio dell'aria condizionata e al
sottofondo di una musica sommessa mentre schiavi a torso nudo servivano
calici di vini ghiacciati. In basso, schiavi umandh stavano trascinando
il corpo di Erdro fuori dal campo, e senza farmi notare posai per lui
sulla ringhiera un calice pieno di vino. Nessuno lo smosse.

Però non potevo andarmene. Anaïs e Dorian erano lì vicino e mi
presentavano a figli di arconti e figlie di magnati delle Gilde, per cui
non potevo andare via senza rischiare di recare un'offesa spaventosa,
cosa che nel clima attuale non mi potevo permettere. Anaïs, in
particolare, non era mai lontana e mi riempiva di bicchieri di vino
nella speranza di strapparmi qualche storia sull'arena... e siccome ero
giovane e un po' ubriaco, e servito da una donna di non poco fascino,
confesso che mi vantai. Soprattutto, mentii. Dove ho imparato a
combattere? Da un maeskolos jaddiano con cui avevo viaggiato per anni.
Perché avevo combattuto come mirmidone sacrificabile prima di ottenere
la mia nomina a corte? Quella era una cosa complicata perché avevo perso
la mia lettera di presentazione e mi ci era voluto del tempo per
rintracciare le navi commerciali di mio padre e farne mandare una nuova
copia al castello di Borosevo. Intanto, un uomo doveva pur guadagnarsi
di che vivere. Come l'avevo persa? Ecco, come di certo sapevano,
Borosevo aveva un suo sporco sottobosco criminale.

Raccontai una versione dell'aggressione subita a Meidua in cui non
figuravano le motociclette e la trasposi fra i canali del Distretto
Inferiore. Questo entusiasmò il mio pubblico ancor più di quanto
avessero fatto le storie sui combattimenti nell'arena, e siccome è una
qualità peculiare del pericolo eccitare coloro i quali non lo hanno mai
sperimentato, Anaïs non fu la sola ad appendersi al mio braccio fino
alla fine della storia. Incatenato dalle convenzioni sociali e dalla
cortesia, sapevo di aver perso le stelle. Tanto valeva che gli hangar in
cui erano custodite l'astronave uhran e la più goffa nave anduniana
fossero stati vuoti. Adesso la corte mi aveva nelle sue grinfie, avevo
fatto la mia scelta, barattato un futuro con un altro.

Vedendo Erdro morire, non potei evitare di sentire di aver commesso un
errore.

«È vero, Dorian?» chiese una grassa ragazza patrizia con il volto
rotondo, sfoggiando un piccolo broncio. «Lo è?»

«Sai che non te lo posso dire, Melandra!» replicò Dorian, tirando un po'
più vicina la ragazza sul morbido divano su cui erano entrambi
semisdraiati. «I miei padri mi manderebbero al patibolo se parlassi del
trionfo!»

Il trionfo. Il Cielcin. Stavano parlando del Cielcin, Makisomn. Dorian
lanciò un'occhiata alla sorella, poi fissò il tendone sopra di noi.
Riconobbi l'atto istintivo di cercare le videocamere, ma subito dopo lui
si guardò intorno e ci strizzò l'occhio.

«Dici sul serio?» chiese Melandra, facendosi ancora più vicina al
giovane nobile. Era la sua amante? «Come ne hanno preso uno?»

Come succedeva spesso, Anaïs rispose al posto di suo fratello. «Gilliam
Vas l'ha ottenuto dai foederati che sono assegnati alla Legione in
visita.»

«Sul serio?» intervenne il figlio di un capofazione della Gilda
Industriale di Binah, con il suo marcato accento lunare.

«Quel gargoyle?» Melandra fece una smorfia e io risi sotto voce, perché
`gargoyle' era il termine esatto per descrivere l'intus\emph{.}
«Suppongo che abbia senso. Quel mutante ha lui stesso sangue demoniaco.»
Non era il primo commento del genere che avevo sentito a corte sul conto
del cantore. Devi capire, lettore, che gli inti \emph{spaventano} la
nobiltà. Sono quello che noi tutti saremmo se non fosse per la grazia
della Terra e dell'imperatore, e ci ricordano che i palatini non hanno
il controllo del loro destino genetico, a meno di voler rischiare
mutazioni del genere. Gilliam Vas ci ricordava che eravamo legati
all'imperatore e a me faceva tornare in mente le parole pronunciate
tanto tempo prima da Saltus: `Siamo entrambi omuncoli.' A quell'epoca
avevo rifiutato quell'affermazione, ma c'era stata un po' di verità
nelle parole di quella creatura e negli occhi spaiati del cantore. E
come per molte verità, apprenderla non era facile.

Dorian le assestò un colpo scherzoso con la mano. «Attenta... è di un
prete che stai parlando.»

«È una bestia, Dorian!»

Non avevo conosciuto bene Erdro, e avevo visto morire molti dei miei
compagni mirmidoni da quando avevo cominciato a combattere nel Colosso,
ma questo non rendeva più facile accantonare la sua morte. Sentendo di
non avere nessun contributo positivo da dare nella conversazione sul
prete mutato, portai il mio calice di vino finito a metà dove avevo
lasciato sulla ringhiera quello per l'anima di Erdro. Uno dei servi unti
e seminudi si avvicinò per rimuoverlo ma l'allontanai con un gesto della
mano che mi venne fin troppo spontaneo. Mi appoggiai alla ringhiera
accanto al bicchiere, guardando una compagnia di mimi eudoriani eseguire
una scena tratta da \emph{Cyrus il folle} di Bastien. Credo fosse presa
dalla seconda parte, quando il principe sopravvive al rientro
nell'atmosfera nascondendosi dietro le gonne della madre. Recitarono la
farsa in stile classico, fra immagini olografiche e giochi pirotecnici.
Mia madre avrebbe approvato, nonostante il sangue eudoriano degli attori
(non amava quel popolo itinerante). Le maschere erano dipinte a colori
vivaci, visibili anche dalla nostra altezza, e apparivano splendide
sugli enormi schermi in funzione a beneficio della folla.

«Lo conoscevi?»

Allungai la mano verso il coltello a serramanico che non avevo con me,
ma si trattava solo di Anaïs, che sussultò e per poco non lasciò cadere
il suo vino.

«Mi dispiace» disse, portandosi una mano al seno. «Non volevo
spaventarti. Non te ne è finito addosso, vero?»

«Cosa?» Faticai a seguire il senso delle sue parole. Si riferiva al
vino, che aveva schizzato in parte le piastrelle ai miei piedi, rosso
come l'inchiostro con cui sto scrivendo questo resoconto. «No, Vostra
signoria. Perdonami, sussulto con una certa facilità.»

A quel punto rise e la mano che stringeva al petto si rilassò mentre
l'abbassava. «Ha senso. L'arena...»

Mi trassi indietro, appoggiandomi di nuovo alla ringhiera del palco nel
guardare l'esibizione degli Eudoriani. «Io...» Pensai a quando avevo
vissuto nelle strade, a quando mi nascondevo dai prefetti, che fossi
innocente o meno, agli altri criminali, alle costole rotte, al piangere
nella notte. «L'arena... sì.»

«Lo conoscevi?» ripeté, accennando con la testa in direzione del campo
di combattimento e del punto dove Jaffa aveva {abbattuto} Erdro con
l'antiquata balestra. Uno degli schiavi umandh era impegnato a ripulire
i mattoni dal sangue mentre la compagnia metteva in scena la commedia di
Bastien dall'altro lato dell'arena. Quando annuii passivamente,
aggiunse: «Cosa si prova? Non riesco a immaginarlo.»

Anche se ero alquanto ubriaco ebbi il buonsenso di tenere a bada la
lingua. Un sorriso addolorato mi affiorò sul volto e mantenni la mia
concentrazione sullo schiavo Umandh che, con il tronco chino, usava i
tentacoli per lavare il terreno dove Erdro aveva sanguinato a causa
della ferita inferta dalla freccia. La sua domanda, l'insensibile
distacco di quella premessa, si congelarono dentro di me. Rigirai quelle
parole nella mia mente come se fossero state un dono particolarmente
dubbio, esaminandone l'intento al di sotto di quel volgare disprezzo.
Decisi che non aveva voluto insultarmi anche se io mi ero offeso. «Non
lo conoscevo molto bene. È una cosa a cui ti abitui... laggiù.»
Abbracciai con un gesto l'arena, i cui mattoni erano tempestati dalla
sommità delle colonne di cemento retrattili. «Suppongo che non fossimo
amici.»

Mentre parlavo pensai a quei mirmidoni che consideravo amici, e anche se
non ero un uomo religioso ringraziai Dio che non fossero stati Switch o
Pallino ad affrontare Jaffa, quel giorno.

«Ha combattuto coraggiosamente.»

«Sì» convenni. Il coraggio non c'entrava niente. Erdro combatteva perché
aveva bisogno di soldi. Guardai oltre il bordo, lungo spogli metri di
muro di pietra e fino al suolo del colosseo. Appena al di là della
portata del mio braccio scintillava e si increspava un campo Royse,
quasi invisibile nell'aria ma con abbastanza forza latente da fermare un
autobus scagliato da un cannone a rotaia. E tuttavia mi sentivo esposto,
ricordando come non fossi riuscito a rimanere nel palco di mio padre,
nel colosseo di Meidua, e quanto fosse stato visibile quel fallimento.

Anaïs si appoggiò alla ringhiera accanto a me, rendendomi consapevole
dell'odore fumoso del suo profumo. Potevo sentire su di me lo sguardo di
quegli occhi verdi, ma scoprii di non riuscire a distogliere il mio
dall'arena. Tutto appariva così diverso da quel palco. Rividi
mentalmente la morte di Erdro, vidi Jaffa caricare la balestra, solo per
diventare Crispin quando lanciò la quadrella. {Anaïs} parlò,
riscuotendomi dalla mia visione. «Almeno tu non devi più rischiare la
vita, giusto? Oppure ti manca? Potremmo farti combattere come
gladiatore. Dorian lo adorerebbe! Ha sempre desiderato essere amico di
un gladiatore...»

«No!» esclamai, a voce troppo alta. Di colpo non era più Crispin a
impugnare la balestra ma ero io, ed era Switch che ci si aspettava che
uccidessi. O Siran. O Ghen. O Pallino. «Per la Terra nera, no.»

Lei si ritrasse un poco, sorpresa dalla mia veemenza. Avrei voluto
gettarmi oltre la balaustra, fracassarmi sui mattoni. Perché ero andato
a cercare Makisomn nelle prigioni del colosseo? Ero intrappolato proprio
come lo ero stato, prigioniero dei capricci del conte più di quanto
fossi stato prigioniero della povertà.

Il resto di quello che Anaïs mi disse mi scivolò addosso e non lo
ricordo. Quando si allontanò -- chiamata dal fratello o da uno dei suoi
amici patrizi per questo o quel capriccio passeggero -- rimasi di nuovo
solo e guardai Cyrus il folle sopravvivere al fuoco e alla morte per
pura fortuna e semplice stupidità. Tutti ridevano. Feci cadere il calice
di Erdro dalla ringhiera e lo guardai oltrepassare lentamente il campo
di Royse fino a infrangersi sul suolo dell'arena.


