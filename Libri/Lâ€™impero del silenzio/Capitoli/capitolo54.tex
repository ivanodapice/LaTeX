\chapter{Luce a gas}

Dopo un'altra delle mie visite serali a Valka, mi avviai da solo lungo
un colonnato che si affacciava su uno dei cortili interni del castello.
In basso, tre cortigiani dalle vesti a colori vivaci sedevano
all'interno di un gazebo in ferro battuto e giocavano a un qualche gioco
olografico nella morbida luce del crepuscolo. Era stata una giornata
umida, non piovosa ma intrisa di vapore -- cosa fin troppo normale a
Borosevo -- e gli abiti mi aderivano addosso come alghe al corpo di un
annegato. Un logoteta mi fermò e chiacchierammo per un po'. La mia
esibizione a cena con la grande priora mi aveva fruttato un piccolo e
breve momento di celebrità, perché pareva che Ligeia non fosse amata,
come accadeva spesso con le alte sfere della Cappellania. Quell'uomo
però era anche un fan del Colosso, e si ricordava di me dai tempi in cui
vi combattevo. Quando ci separammo, se ne andò con una fotografia di
entrambi scattata con il suo terminale e io con un sottile sorriso
divertito sulle labbra.

Ho spesso constatato che un posto nuovo è come un paio di scarpe nuove e
adesso, dopo aver vissuto per alcuni mesi al castello di Borosevo
sentivo che il posto mi si stava adattando fino a diventare
confortevole. Gli archi arrotondati e le mura di pallida arenaria sotto
i tetti di tegole rosse erano caldi e invitanti, segnati a tratti da
viticci rampicanti e decorati con statue di ferro raffiguranti ninfe che
danzavano e legionari. Perfino gli Umandh erano diventati meno strani,
con il ronzare delle squadre di lavoro che incrociavo. Soprattutto, i
servitori sorridevano o chinavano la testa al mio passaggio. Potevo
anche essere un prigioniero, ma ero {libero} di girare per la mia
prigione, e se dover stare in compagnia di Anaïs e di Dorian era noioso,
i miei studi serali con Valka erano tutt'altra cosa. Avevamo condiviso
una bottiglia di rosso di Kandarene, com'era nostra abitudine, e ci
eravamo fatti una risata a spese della priora, anche se ero stato
attento a non offendere troppo apertamente la rappresentante su Emesh
eletta dal Sinodo della Cappellania.

Valka non era frenata da una censura del genere ed era scoppiata a
ridere. `L'espressione che c'era su quella vecchia faccia rovinata!'
aveva esclamato. `Sembrava che ti avrebbe sparato volentieri, se avesse
potuto.'

`Avrebbe potuto' avevo replicato in tono serio, incapace di evitare di
sorridere a mia volta. La risata di Valka era un suono profondo e
musicale che le scaturiva dal petto, da qualche parte vicino al cuore.
Non era la risata di una volgare cortigiana, addestrata a una precisione
effemminata e contegnosa. No, lei rideva come una nube di tempesta sul
mare e si era abbattuta sul mio cupo riserbo come un'onda contro la diga
marina del Riposo del Diavolo.

Ero quindi intento a pensare a lei -- alla sua diversità di straniera e
a come mi facesse sentire un po' sciocco -- mentre attraversavo i
giardini del castello diretto alla mia camera e al mio letto. Pensai che
magari avrei disegnato per qualche ora prima di dormire. Proprio quello
che mi serviva.

Valka era ancora un mistero. Nonostante il suo stato diplomatico non era
un ambasciatore di qualche tipo. Quello che non ricordai in quel momento
fu che siccome nella Demarchia di Tavros tutti hanno potere di voto,
qualsiasi appartenente ai clan tavrosiani potrebbe essere considerato un
rappresentante di quel popolo strano e lontano. Questo è uno dei molti
motivi per cui gli ospiti tavrosiani sono spesso malvisti nelle corti
dell'Impero. Complicano le cose.

Le mie stanze si trovavano su uno dei livelli mediani della sala del
Sole di Vetro, uno degli edifici più nuovi e meno opulenti del complesso
del palazzo, alquanto distante dagli appartamenti diplomatici dell'ala
settentrionale. Nel cortile sottostante l'edificio c'erano luci a gas e
oltrepassai un lampionaio con il suo stoppino. Avevo creduto che quelle
lampade fossero un simbolo di opulenza, dato che bisognava pagare dei
servitori perché le accudissero, ma in seguito mi ero reso conto che
erano solo una precauzione di {sicurezza} perché con la griglia di
illuminazione del palazzo prona a interruzioni, le luci a gas
garantivano di notte una buona visibilità sulle mura e sotto i colonnati
del castello.

L'uomo agitò una mano verso di me con un sorriso e si tolse il cappello.
Un uccello cantò in una delle onnipresenti aree di verde del castello e
di lì a poco un ornithon levò in risposta un verso di sfida. Mi fermai
per un momento a contemplare le stelle mobili. Pallino e io ci eravamo
incontrati qualche altra volta, e in ciascuna occasione avevamo parlato
dei nostri piani per acquistare un'astronave. Il contratto di Switch con
il colosseo era quasi scaduto, ma Pallino aveva rinnovato il suo più di
recente, e questo significava altri nove mesi locali di attesa prima che
fosse libero dal suo obbligo. Questo era più di un anno standard, il che
significava che era nell'interesse di Switch rinnovare a sua volta il
contratto, in modo da guadagnare un altro anno di paga e garantire un
leggero sovrapporsi dei due contratti. Era un tempo più lungo di quanto
mi sarebbe piaciuto.

Comunque lasciai correre. Per il momento ero comodo, nutrito e al
sicuro, e coltivavo l'incerta, privata speranza che se il mio piano
avesse funzionato e mi fossi lasciato Emesh alle spalle, Valka sarebbe
venuta con me. Non era impossibile. Io volevo viaggiare fra le stelle,
incontrare xenobiti, e lei era un'esperta delle loro culture. C'era un
futuro in questo? Adesso era molto meno fredda nei miei confronti di
quanto non lo fosse stata in quel nostro primo incontro, ma anche se
concepivo questi sogni non mi illudevo... lei aveva una sua strada da
seguire. Perdonerai però forse a un giovane i suoi sogni. Avevo sofferto
molto e avevo il diritto di sognare.

«Ragazzo!»

Mi bloccai ad appena pochi passi dalla porta di accesso al corridoio e
dalla solitudine. Conoscevo quella voce, quell'aristocratico accento
bleso e strascicato. Prima che potessi girarmi del tutto una mano mi
afferrò per una spalla e nel voltarmi mi trovai faccia a faccia con
Gilliam Vas. I suoi occhi spaiati riflettevano la luce a gas, che faceva
scintillare quello azzurro. I capelli biondi gli pendevano flosci
intorno al volto deforme e i denti lampeggiavano fra le labbra ritratte
in un ringhio mentre mi spingeva con la schiena contro il palo del
lampione più vicino.

«Non so chi credi di essere.» Il suo alito aveva un odore dolce, come di
menta o dello stimolante verrox, o di entrambi. Distolsi la faccia e
cercai di non tossire. «Il conte può anche trovare divertenti le tue
pagliacciate plebee -- anche se solo la Terra sa il perché -- ma io non
tollererò simili offese, non da te e non a spese della dignità della mia
famiglia. Ti prego di ricordare chi siamo.»

Con deliberata lentezza posai una mano sulla sua, preparandomi ad
abbassarmi all'altezza della vita e a spezzargli il polso, se
necessario.

«Vostra reverenza, se la dignità della tua famiglia è appesa a un filo
così sottile che io possa minacciarla... ecco, forse dovresti rivedere
quanto essa valga effettivamente.» La mia voce suonò un po' affannosa
per la sorpresa.

Il prete espirò dal naso come un toro sul punto di caricare e mi scrollò
contro il lampione al punto che sbattei la testa e lanciai un grido.
«Osi farti beffe di me? Di me?» Le sue dita ineguali si torsero nella
stoffa della mia tunica come se mi stessero stringendo per la gola. «La
mia famiglia è più antica di quanto tu possa comprendere, piccolo
plutocrate borioso.» Era più forte di quanto avessi immaginato. «Se il
conte non ti volesse per chissà quale ragione, e se non mi fossi tanto
inferiore, ti ucciderei per aver insultato in questo modo la mia
famiglia.»

Con la sua mano ancora stretta intorno alla spalla mi ersi in tutta la
mia statura, guardandolo dall'alto in basso, sorpreso dal fatto che
proprio lui -- che avrebbe dovuto essere acutamente in sintonia con il
sangue di quanti lo circondavano -- non riuscisse a intuire le mie
ascendenze. D'altronde io ero basso per essere un palatino, e qualsiasi
altro elemento insolito del mio aspetto avrebbe potuto essere spiegato
dal mero fatto che ero uno straniero. A palazzo nessuno aveva intuito il
mio segreto, quindi perché avrebbe dovuto farlo lui?

«Forse tu e i tuoi non dovreste agire in modo tanto bellicoso. La priora
ha interrotto con la sua retorica una conversazione personale,
reverenza, e io sto cercando di andare a letto.»

Dallo spasmo che percorse i muscoli della mascella di Gilliam compresi
che quella era una cosa sbagliata da dire. La sua mano si spostò e la
sua faccia si protese ancora di più verso la mia, anche se adesso doveva
guardare verso l'alto per fissarmi. Così da vicino, sotto la luce del
lampione, potevo vedere lo spesso strato di trucco che portava per
mascherare la carnagione butterata. Per quanto ben applicato, la
sudorazione lo aveva fatto sciogliere un poco e rilucere lungo
l'attaccatura dei capelli. «Ti lascerò andare quando avrò finito, signor
nessuno» affermò. Poi, incongruamente, abbandonò la presa e indietreggiò
barcollando di un passo irregolare. Adesso che era a distanza di un
braccio, ruotò su sé stesso e mi assestò un manrovescio in piena faccia
con l'altra mano, strappandomi un sussulto di dolore. Quell'uomo aveva
vissuto su Emesh per tutta la vita, e la forte gravità del pianeta aveva
dato a quel braccio una forza amplificata dal grosso anello che portava
al dito medio.

Mi massaggiai la faccia in silenzio con una mano, fissandolo con occhi
roventi. Quando ritrassi la mano, risultò appena macchiata di sangue. Se
fossi stato me stesso -- Hadrian Marlowe, figlio di un arconte
possidente terriero -- avrei potuto sfidarlo a duello per quell'insulto,
ma non ero un Marlowe, non più, non quel giorno, quindi Hadrian Gibson
chinò la testa con le labbra serrate in una dura linea sottile. «Eri uno
spaziale,» ringhiò il prete, agitando una mano nel guardarmi con rabbia
«quindi sei poco meglio di un barbaro. Ti spiegherò come stanno le
cose.» L'occhio azzurro scintillò intenso alla luce del fuoco, ma quello
scuro parve assorbire quella stessa luminosità come un'orbita vuota, un
buco sulla sua faccia cerea. Sollevò di nuovo quella mano deforme come
per colpirmi. Sussultai, e avrei potuto imprecare contro me stesso per
averlo fatto. Il cantore rise, un suono come di strumenti di valore
inestimabile che venissero fracassati. «Così va bene.» Puntò un dito
contro la mia faccia, con la massa nera dell'anello che riluceva sotto
la luce a gas.

Spostò quindi le mani dietro la schiena e si protese in avanti, con il
mento che sporgeva verso di me come a offrirmi un bersaglio. Per un
momento credo di essermi pietrificato, paralizzato dalle convenzioni
sociali. Non potevo colpirlo e rimanere lì come Hadrian Gibson. «Ah, sei
capace di imparare» commentò, senza smettere di sorridere. «Vedi,
\emph{mirmidone}, questo non è il Colosso. A corte ci aspettiamo
civiltà, e per quanto Balian possa pensare che tu sia ben addestrato, un
mezzo selvaggio è un mezzo selvaggio. Il sangue rivela ogni cosa.»

«Senza dubbio Vostra reverenza è un esperto in fatto di \emph{mezzi}
selvaggi.» Badai a enunciare con cura quel `mezzi', accennando con la
testa per fare riferimento alla sua condizione. «È gentile da parte tua
abbassarti al mio livello. E con quale dedizione! Dimmi, sei nato così
oppure hai pagato un segaossa per meglio relazionarti a quelli come me?»
Non per la prima volta, mi chiesi il perché del fuoco che divorava
quell'uomo, del suo odio. Era la sua deformità che lo spingeva ad agire
con tanta cattiveria nel suo ruolo di palatino e di prete? Oppure era
solo il fatto che vedeva in me qualcuno che gli era inferiore? Era ciò
che i plebei vedevano in tutti i palatini?

\emph{È questo che Switch ha visto in me?}

Gilliam si ritrasse e mi colpì ancora con la mano che portava l'anello.
Questa volta non sussultai di dolore, non gridai. I capelli mi si
allargarono a ventaglio intorno alla faccia ma tenni lo sguardo fisso su
di lui. Aveva il respiro affannoso, il sangue che gli ribolliva. «Sei
fortunato di piacere al conte, altrimenti ordinerei ai miei cathar di
aprirti quel naso grazioso che hai per la tua insolenza.»

«Cosa vuoi da me, prete?» domandai. «Io non voglio niente da te.»

Mi guardò come se la sua sedia gli avesse chiesto perché l'aveva
allontanata dal tavolo. «Perché sei qui?»

Sbattei le palpebre e lo fissai, confuso. «Prego?»

«Te lo chiedo di nuovo. Perché sei qui? Quali menzogne hai detto al
conte? Come lo hai subornato? Tu e quella strega demarchica?»

Anche se l'aria era calda come sempre lo era a Borosevo, credo che il
sudore mi si sia di colpo trasformato in ghiaccio, imprigionandomi: d'un
tratto non potei muovermi per il freddo. Quindi si trattava di questo.
Finalmente era venuta fuori una risposta. Gilliam era convinto che
stessi complottando qualcosa con Valka, che fossi una spia e un
sabotatore.

Lo guardai con espressione accigliata. «Di cosa stai parlando?»

«Hai fraternizzato con quel demone Pallido e ti sei arruffianato quella
sgualdrina dello spazio profondo...» Barcollò verso di me con i denti
snudati. «Insulti la Sacra Cappellania della Terra, sei blasfemo alla
tavola di Sua eccellenza e credi che siamo tutti troppo stupidi per
accorgercene? So cosa sei, eretico.» Sollevò di nuovo una mano, ma
questa volta ero pronto e non battei ciglio quando mi colpì, non girai
neppure la testa. Gli occhi gli sporgevano dalle orbite per la furia, ma
in essi c'era una scintilla che credo fosse di paura. «Perché stai
sorridendo?»

«Se questo fosse vero, credi sul serio che il conte mi lascerebbe
libero?» Sputai saliva rossa sulle piastrelle. «Sii realistico,
reverenza.» Si mosse per colpire ancora, ma ormai ero più che pronto e
mi spostai, per cui la sua mano non incontrò nulla neppure nel
successivo manrovescio. Per quanto fosse forte, io ero veloce, avevo
sempre dovuto esserlo, non solo da bambino quando Crispin era il mio
solo avversario, ma anche sui canali e i tetti della città, come pure
nel colosseo. Continuai a indietreggiare verso le scale e il mio letto.
Forse le guardie sarebbero arrivate presto. Non avrebbero fermato il
cantore né messo le mani addosso a Gilliam, ma un pubblico alterava la
natura delle esibizioni, e anche se tutto quello che avevamo detto e
fatto era stato registrato dai diecimila occhi del castello, la semplice
presenza di altri occhi poteva modificare il suo comportamento. Il
lampionaio ci stava guardando con sgomento e dalla mia posizione più
alta potevo vedere un paio di servitori nell'ombra del colonnato, ma
erano tutti remoti come stelle. «Tu non sei un mio nemico, prete. Non ho
mai chiesto di trovarmi qui.»

Gilliam ritrovò il controllo, spingendo indietro i capelli chiari mentre
si raddrizzava quanto più gli era possibile. Sputò e tracciò un gesto
protettivo. «Non osare insultare ancora mia madre, ragazzo, o ti farò
gettare nudo in quella tua arena.»

Mi guardai bene dal rispondere.

