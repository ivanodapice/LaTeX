\chapter{Barbari}

«Sei impazzito?» domandò Valka, senza preamboli o saluti, facendo
irruzione nel giardino chiuso dove mi stavo addestrando per il duello.
«Questa non è un'opera, Gibson... o come altro ti chiami. Chi ti ha
chiesto di difendere il mio... il mio...» Era così agitata che le parole
le vennero meno e imprecò in panthai.

«Il tuo onore, signora?» suggerì Switch, finendo la frase per lei. Dopo
la visita al colosseo ero stato confinato nell'area della sala del Sole
di Vetro e Switch era stato accompagnato da me come avevo richiesto. Mi
consolava il fatto che anche Gilliam fosse confinato nello stesso modo e
cercavo di non pensare alla catena di tessere di domino che avevo messo
in movimento.

La mascella di Valka si contrasse, formulando parole silenziose. «Questa
non è la Vecchia Terra, dannazione! Non ho mai chiesto il tuo aiuto!»
Dilatò le narici e ripiegò su quella che un giorno avrei appreso essere
la sua imprecazione preferita: «\emph{Imperiali}.» Switch incurvò le
labbra in una smorfia amichevole e io scossi il capo, grato di riavere
accanto il mio amico, sia pure temporaneamente.

Seppellendo la mia esasperazione -- `l'ira è cecità' -- appoggiai la
spada da esercitazione contro una betulla bianca e mi girai verso Valka.
La verità era che mi sorprendeva il fatto che non fosse venuta da me
prima. Che ci avesse messo una notte e mezza giornata. «Mi dispiace» mi
scusai, con l'aria di un uomo rassegnato a lottare con una vipera. «So
che non approvi la violenza.»

«Il problema non è la violenza!» Si spinse indietro i capelli. «Avrei
dato a quel piccolo troll un pugno nei denti, se avessi potuto fare a
modo mio, ma...» Si interruppe, fece come per mordersi un'unghia poi si
bloccò e serrò la mano a pugno. «Tu non sei \emph{responsabile} per me,
dannazione!»

Mi sentii sgranare gli occhi. «Certo che no» risposi. Non era stata
quella la mia intenzione, non proprio. Ripensai all'istante prima che
colpissi Gilliam. L'aveva chiamata ripetutamente strega, sgualdrina. La
faccia mi si tinse di un colore simile a quello dei capelli di Switch,
che intanto si schiarì leggermente la gola. «Perdonami, dottoressa
Onderra.» Eseguii un inchino poco profondo. «Questo è il mio amico
Switch.»

Il mirmidone chinò il capo. «Buon pomeriggio, mia signora.»

«Lei è una dottoressa, Switch» mormorai, prevenendo la classica reazione
acida di Valka.

«È per lei che hai preso a pugni il prete?»

Mi serrai l'arco del naso in preda alla frustrazione. «Questo non è
d'aiuto.» Se non altro, il mirmidone ebbe la buona grazia di apparire
mortificato e passò gli istanti successivi a esaminarsi le scarpe.

La dottoressa incrociò le braccia, comprimendosi un poco il torace. «Voi
imperiali... razza di retrogradi e sciovinisti \emph{kaunchau rhobsa
	mehar di}...» Passò a esprimersi in un dialetto tavrosiano di cui capivo
una parola su dodici.

«Non lo siamo! Una volta mia madre ha impegnato un duello per una donna»
protestai, senza riflettere. «Ecco, due donne. Due donne e un cavallo.
Neanche questo è d'aiuto.» Compresi che era la cosa sbagliata da dire
nel momento in cui le parole mi uscirono di bocca, perché mi ritrovai a
pensare all'omuncolo dalla pelle azzurra che mia madre teneva nel suo
harem.

Valka si limitò a guardarmi. «Tua madre... era una qualche grande
\emph{lady}, vero? Lord... lord...»

«Marlowe.» Mi inchinai di nuovo. «Hadrian Marlowe.» Quando mi raddrizzai
protesi un poco il mento in fuori e solo un momento più tardi mi resi
conto che era un errore, perché esprimeva più altezzosità aristocratica
di quanta l'egalitaria Valka ne potesse tollerare. Mi sentii un
terribile idiota, e l'anello che avevo al pollice mi parve più
un'affettazione che non l'aver assunto di nuovo il mio vero io. Avevo la
sensazione che non mi appartenesse affatto, come se lo avessi preso in
prestito o rubato... e suppongo che lo avessi fatto. Switch rimase in
silenzio, senza guardare l'anello sulla mia mano. «Il conte mi aveva
ordinato di tenere celato il mio nome. Vedi, mi stavo nascondendo, e...»
Era di nuovo la conversazione che avevo avuto con Switch, solo peggiore,
molto peggiore, perché si trattava di Valka.

«Ora non più» ribatté lei. Non era un rimprovero o una condanna, solo la
nuda asserzione di un dato di fatto. La fissai, acutamente consapevole
del vuoto della mia espressione, e contro ogni probabilità lei arrossì,
distogliendo lo sguardo. «Mi dispiace.» Strane emozioni le passarono sul
volto, la sua ira si dissolse e si mescolò a qualcosa... di più morbido?
Se pure ci avessi provato, non avrei saputo etichettarlo.

D'un tratto scoprii di non poterla guardare in faccia e giocherellai
invece con un pezzo di gomma logora sul bordo dell'impugnatura della
spada da addestramento. «Hai ragione, sono io quello che si dovrebbe
scusare. Suppongo che prendere a pugni quel bastardo non mi abbia
portato molto di buono...» Feci una pausa e mi arrischiai a guardarla.
Si stava ancora studiando il dorso della mano tatuata. «Suppongo di
averlo fatto per le ragioni sbagliate.»

Era la mia immaginazione, o per un momento si era immobilizzata? Ma quel
momento era passato e lei era di nuovo la \emph{dottoressa Onderra}.
«Grazie» disse infine. La cortesia represse per un momento la sua
indignazione e lei si rivolse al mio amico mirmidone. «Ti chiami
Switch?»

Lui annuì. «Sì, signora. Ecco, il mio nome è William -- come
l'imperatore -- solo che ci sono troppi William e Switch era il nome che
usavo per lavorare prima di comprarmi la libertà dalla casa di piacere.
Mi va bene così.»

«Switch sia, allora. Sei uno degli... amici di lord Marlowe? Di quando
era al colosseo?»

Per un momento la natura sincera di Switch ebbe la meglio sulla sua
cautela di plebeo. «Had e io siamo amici da un po', certo.» Si grattò la
testa. «Stavamo cercando entrambi si pagarci un passaggio fuori dal
pianeta. Di andare dove fosse capitato, capisci?»

Rivolgendomi a Valka, chiesi: «Quindi sei qui... per quale motivo? Per
dirmi di ritirare la sfida? Non posso farlo.»

«Perché no?» scattò Valka. «Credevo che voi palatini poteste fare quello
che volete.»

Per quanto cercassi di resistere, scoppiai a riderle in faccia. «Quello
che vogliamo? Mi dispiace, ma ti è forse sfuggita la parte in cui ti ho
detto che stavo vivendo qui sotto falso nome?» Accennai a Switch, che
indossava la tenuta da combattimento di maglia sintetica comune ai
mirmidoni fuori servizio. «Credi che abbia rischiato la vita nel
colosseo per amore di quel gioco? Mio padre mi ha \emph{venduto},
dottoressa. Mi ha venduto alla Cappellania. Quindi non startene lì a
fare supposizioni.»

Valka contrasse le labbra. «Non lo sapevo.» La sua voce -- quella sua
splendida voce -- si abbassò fino a essere quasi inudibile,
rinforzandosi solo quando inclinò la testa per rispondere. «Ma questo
cosa c'entra con Gilliam?»

«\emph{Non posso} ritirare la sfida, né per te né per chiunque altro.»
Rigirai l'anello che avevo al pollice. «È un vincolo legale. Non puoi
tirarti indietro da una sfida formale. Sono impegnato.» Distolsi lo
sguardo, poi riportai di scatto l'attenzione su di lei nell'aggiungere:
«E quel figlio di puttana mi aveva fatto stordire.»

Anche a cinque passi di distanza la potei sentire digrignare i denti.
«Questa deve essere l'usanza più stupida di cui abbia mai sentito
parlare.»

«Non lo è!» interloquì Switch, avanzando di un passo e pulendosi le mani
sui calzoni. «Se sai di doverti impegnare a sostenere un duello, è meno
probabile che tu...» Mi lanciò un'occhiata e le parole gli vennero meno.
«Ecco... è meno probabile che tu inizi qualcosa. Se non sei ubriaco...»

Lo sorpresi a guardarmi. «Non ero ubriaco, Switch!»

«Stavo solo controllando» ribatté con un sorriso.

Un sorriso ironico e forse un po' triste incurvò le labbra di Valka.
«Comunque non avresti dovuto farlo. E se vincerai ti inimicherai quella
sacerdotessa. Cosa diavolo stavi pensando?»

«Non mi andava che ti chiamasse strega, d'accordo?» Mi massaggiai la
nuca e le volsi le spalle. «Era questo che volevi che dicessi?» Non
aggiunsi quello che stavo pensando, che le società senza l'uso del
duello lo sostituivano con l'omicidio, e che il potere dato a Gilliam
dalla sua posizione avrebbe potuto permettergli di farla franca con ogni
sorta di atto ignobile. Nonostante tutta la sua apparente barbarie, la
nostra stupida usanza forniva un canale tramite il quale un problema
poteva essere affrontato in modo legittimo.

Lei non rispose. Accanto a me Switch si agitò, a disagio, e io mossi
qualche passo per creare un po' di distanza fra noi. Una parte di me
desiderava che il mio amico mirmidone se ne andasse, che ricordasse di
colpo un impegno urgente altrove. Era una cosa ingiusta, dopo tutto
quello che avevamo passato insieme e quello che io avevo fatto passare a
entrambi. Mi stavo mostrando ingrato, ma la verità era che detestavo
dover affrontare quella conversazione davanti a chiunque. «Sì, \emph{mio
	signore}» disse infine lei.

Mi aveva lasciato confuso fin da quando l'avevo conosciuta. La sua
stranezza di straniera, quegli occhi dorati, la pelle come pergamena
nuova, la ferrea determinazione e l'evidente intelligenza. Perfino le
sue sottili crudeltà. Qualsiasi cosa riveli di me l'ammetterlo, lei mi
rivolgeva un canto in un linguaggio chimico che era al di sotto e al di
là della poesia. Dipendeva forse proprio dal fatto che mi sfidava? C'era
acciaio in lei, qualcosa di più dell'acciaio. Era di adamant, come le
astronavi. Di altamateria. `Mio signore.' Quelle parole mi vibrarono
nelle orecchie e mio malgrado accasciai le spalle. «Hadrian» la
corressi.

«Cosa?» non mi aveva sentito.

«Chiamami Hadrian.»

L'aria le sfuggì con forza dalle labbra. «\emph{Imperiali}.» Si rivolse
a Switch. «È meglio che il tuo amico non si faccia uccidere.» Girò sui
tacchi con precisione e se ne andò, avendo apparentemente detto tutto
quello che doveva. «Se dovesse farlo, lo ucciderò.»

Switch e io rimanemmo a fissarci per almeno trenta secondi, comunicando
un inarticolato sfinimento. «Cosa diavolo voleva dire?» commentai
infine.

Lui inarcò le spesse sopracciglia rosse. «Che non devi morire, è ovvio.»

«Grazie, Switch.»

Ripiombammo nel nostro silenzio pieno di disagio senza che nessuno dei
due si muovesse. Dopo un momento lui accennò con il mento e sillabò in
silenzio: «\emph{Seguila}.»

\begin{figure}
	\centering
	\def\svgwidth{\columnwidth}
	\scalebox{0.2}{\input{divisore.pdf_tex}}
\end{figure}

«Aspetta!» Raggiunsi Valka sotto un ombroso colonnato il cui marmo rosa
appariva scurito dal sole rovente. Mi sentivo sporco e piccolo davanti a
lei, vestito com'ero degli abiti da esercitazione e della mia vergogna.
«Aspetta, dottoressa Onderra.» Lei si girò con una mano sul fianco.
Contrariamente a me, avrebbe potuto essere stata scolpita nel
ghiaccio... ma era un piccolo sorriso quello sulle sue labbra? Rideva di
me? Non c'era via di uscita. «Mi dispiace, hai ragione. Ho colpito
Gilliam per quello che ha detto su di te, non sono riuscito a
trattenermi.» Immagini di Crispin privo di sensi sul pavimento di marmo
mi passarono davanti agli occhi e per un momento lo vidi giacere su quel
marmo liscio, lì in mezzo a noi. Da qualche parte un uccello urlò fra
gli alberi al di là del colonnato, stridendo contro la calura
pomeridiana. Distolsi la mente dall'immagine di mio fratello e serrai il
pugno che aveva colpito Gilliam. «Colpa mia» aggiunsi, con voce fievole.

`L'ira è cecità,' affermano gli scoliasti `la calma è vista'. Essi
rigettano l'ira come rigettano tutte le emozioni estreme, fango nella
pozza limpida della mente. Forse è un bene che io non sia mai arrivato
su Teukros e a Nov Senber. La paura. La paura giace alle radici di
tutto, un drago nel senso classico del termine, che genera mostri, è la
morte della ragione. Ma perché avevo paura? Cosa c'era in Valka che
prendeva sentimenti familiari e li rendeva strani come le stelle nel
cielo di Emesh?

«Hai ragione» replicò, con la voce squillante che suonava cupa e
sommessa nell'aria sottostante i pilastri del colonnato. «Tutto questo è
colpa tua.» Non aggiunse nient'altro ma neppure se ne andò. Cercai di
concentrarmi su questo, di placare il terrore galoppante del mio sangue.
Terrore di cosa? Che mi odiasse? Mi odiava? Non mi avrebbe più rivolto
la parola? Forse era davvero una strega, per la Terra e l'imperatore, e
io ero in suo potere.

Mi schiarii la gola. «Ho accennato al fatto che un giorno sarei
diventato un diplomatico...» Quanto erano andati storti quei progetti.
Prendere a pugni in faccia Gilliam era la cosa meno diplomatica che si
potesse immaginare. «Nella diplomazia devi essere disposto a perdonare
alla gente le sue... le sue divergenze di vedute. Devi almeno cercare di
capirla... per un po'.» Stavo farfugliando, ne ero consapevole, ma
continuai come un uomo che sta annegando e che potrebbe sperare di
trovare la riva o un pezzo di legno a cui aggrapparsi. «Mi dispiace di
aver agito in tua difesa... non spettava a me farlo... ma è una cosa che
non posso annullare.» Lei continuò a tacere, limitandosi a tamburellare
con le dita sul tablet per le comunicazioni che le pendeva da un fianco
come un'arma nella sua fondina. «Solo che... lui non avrebbe dovuto
formulare quelle accuse.» Fui assalito da un nuovo pensiero. «Non
sospetteranno di te, vero?»

Valka scosse il capo. «Mi avrebbero già sbattuta nelle segrete della
Cappellania nonostante il mio lasciapassare diplomatico.» Allargò le
braccia. «Non sarei libera se mi ritenessero responsabile di
quell'insurrezione, ed è per questo che non avresti dovuto interferire.
Gli Umandh, quegli stolti disperati, hanno agito di loro iniziativa.»
Rilassando il suo atteggiamento aggressivo, si appoggiò a una colonna
per chinarsi a tirare su uno stivale che le era scivolato lungo il
polpaccio. «Davvero, il mio supposto ruolo in tutto questo sarebbe già
stato dimenticato se tu non avessi preso a pugni quel mutante.»

«Qualcuno doveva farlo.»

«No!» sbottò, raddrizzandosi. «Non \emph{qualcuno}, io avrei dovuto
farlo. Lui era un mio problema.» Tirò la canotta per assestarla
guardandomi con durezza. «Non avevi nessun diritto di immischiarti.»

«Avevo ogni diritto! Considerati i nostri rapporti e gli insulti nei
tuoi e nei miei confronti... e non ti ho vista intervenire per
difenderti. Volevi farlo?»

«No!» scattò. «Perché litigare non risolve niente.»

«Chi te lo ha detto?» domandai, sinceramente sconcertato. «Se lotti per
risolvere un problema e vinci, quel problema è risolto, Valka.» Non
sapevo quello che stavo dicendo, ma se lo avessi saputo mi avrebbe
risparmiato una quantità di dolore quando giunse la guerra... o quando
io entrai in guerra.

«E hai creato sette nuovi problemi da risolvere.»

«Settantasette nuovi problemi» convenni. «Ma si continua a lottare,
perché se puoi scegliere quando farlo hai un minimo di controllo. Se
invece seppellisci tutto, lasci che ti marcisca dentro...» Scossi il
capo. «Gilliam non ha fatto altro che minacciarmi da quando sono
arrivato qui.»

Lei sbuffò, riuscendo a stento a contenere il suo disprezzo. «E questo
ti dà il diritto di cercare di \emph{assassinarlo}? Questo è anche
peggio, \emph{mio signore}.»

Mi morsi la lingua prima di reagire con un `Tu non puoi capire', perché
dal fuoco nei suoi occhi ero consapevole che sarebbe stato un errore
letale. Invece feci una pausa e mi controllai prima di ribattere: «È un
duello formale, non un assassinio.»

Sbuffò di nuovo. «\emph{Okthireakham anaryoch kha}.»

«Forse siamo barbari, forse da dove vieni le cose sono diverse... non lo
so. Quello che so è che se permetti a una persona come Gilliam di agire
impunemente lui calpesterà tutti quelli che troverà sulla sua strada,
moltissimi dei quali non potranno mai sperare di poterlo sfidare. Io
sono un palatino e posso farlo.»

«A proposito, cos'è questa storia? Chi diavolo sei?» mi interruppe.

«Te l'ho detto, mi chiamo Marlowe, Hadrian Marlowe. Mio padre è lord
Alistair Marlowe di Delos. Io... voleva che entrassi nella Cappellania,
ma io avevo... altre idee. Non ti ho mentito più di quanto dovessi fare
perché il conte lo pretendeva da me. Tutti quei discorsi sugli Umandh,
sui Cielcin... su Calagah. Quello sono \emph{io}.» Nel parlare mi resi
conto appieno di cosa questo significasse. Per gli antichi dèi... la
Cappellania lo avrebbe scoperto. Una volta che mi avessero rilasciato
dai miei quasi arresti domiciliari, se fossi sopravvissuto sarebbero
venuti a prendermi? Se la sarebbero presa con mia madre? Fornii a Valka
una versione ridotta della storia... di come mi fossi ritrovato
abbandonato su Emesh, fossi stato derubato e lasciato in assoluta
povertà lungo i canali. «Non ho avuto altra scelta se non il colosseo.
Dovevo pur mangiare.»

«Saresti potuto venire al castello in qualsiasi momento. Non ti hanno
certo punito.»

«Non ancora» sibilai. «Perché il conte ha tenuto segreta la mia presenza
a mio padre e alla Cappellania, non so perché.»

Valka rise con disprezzo. «Non lo sai?»

Avevo alcune teorie, ma non ero dell'umore di condividerle. «Qui sono un
prigioniero, Valka. Perché è tanto difficile da spiegare? Non sono
libero di andarmene più di quanto lo siano gli Umandh. Perché credi
abbia lavorato così duramente per rimanere nel colosseo? Non volevo...
\emph{niente} di questo. Non ho chiesto io di essere qui, non ho chiesto
che Gilliam se la prendesse con me, che tu...» Mi interruppi prima di
dire qualcosa di veramente sciocco e distolsi lo sguardo. «Tu confondi
le cose.» Un velivolo passò davanti al castello, incorniciato dagli
archi del colonnato. Valka non parlò, non si mosse. «È dannatamente
certo che preferirei che le cose fossero diverse.» Dopo un momento mi
arrischiai a darle un'occhiata.

Si stava mordicchiando pensosamente il labbro inferiore. «Sai cos'hai
fatto, vero, Hadrian?»

«Prego?» Sollevai bruscamente lo sguardo smettendo di fissarmi le mani.
Era la prima volta che usava il mio nome.

«Hai fatto succedere questo» continuò, serrando la mascella nel
pronunciare le parole successive. «Lo hai fatto riguardo a me. Qualcuno
morirà perché dovevi dimostrare... cosa? Che sei un uomo? Eri un
combattente, per gli dèi! Nessuno ne dubita.» Tacque per un momento, con
lo sguardo fisso su qualcosa al di là del mondo mortale. «Non voglio
avere sulla coscienza la morte di nessuno. Non voglio che qualcuno muoia
per causa mia.»

Mossi un passo in avanti e mi protesi verso la sua mano, ma ebbi paura
di toccarla, avendo bisogno di farlo ma sapendo che non avrei dovuto.
«Hai ragione» replicai. «Hai ragione, ma qualsiasi cosa succeda, non
sarà per causa tua. È stata una mia scelta e mi dispiace di averti
trascinata in tutto questo.» Ritrassi la mano, sentendomi di colpo molto
sciocco. «Non deve morire nessuno.»

«Ma hai detto...»

«Dovremo combattere se i nostri secondi non potranno risolvere a parole
i nostri contrasti, cosa che non potranno fare, ma il primo sangue sarà
sufficiente. Infliggerò il primo colpo e la chiuderò lì. Lo giuro.»

Arricciò le labbra. «E cosa mi dici del risolvere i problemi? Che ne è
stato del...» Il suo tono cambiò, imitando con spaventosa precisione le
mie parole di poco prima... «Gilliam che agisce impunemente, calpestando
tutti quelli che trova sulla sua strada?»

«Questo non è leale» protestai. «Vuoi che lo affronti oppure no? Non
puoi avere entrambe le cose.»

Fu il suo turno di distogliere lo sguardo, incrociando le braccia senza
dire niente.

«Non mi posso scusare più di quanto abbia già fatto» affermai con
sincerità. «Non posso tornare sui miei passi e neppure fuggire, ma
\emph{cercherò} di sistemare le cose il più possibile.» Le parole mi si
spensero lentamente e il mio tono si fece più sommesso, perdendo forza.
«Spero... spero che mi perdonerai.» Con voce ancora più sommessa
aggiunsi: «Io non voglio uccidere nessuno, dottoressa Onderra.»

«Valka» disse infine. «Chiamami Valka.»


