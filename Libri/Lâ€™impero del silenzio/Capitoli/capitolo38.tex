\chapter{Sangue come cera}

Non persi mai un singolo combattimento nel Colosso, non dovetti mai
inginocchiarmi davanti ai gladiatori professionisti o agli altri
mirmidoni per aspettare il giudizio della folla. Non in cinque
combattimenti, e neppure in dieci. Dopo sette mesi e uno scontro
particolarmente ingegnoso in cui usai la sabbia sul pavimento del
colosseo per mandare in corto l'alimentatore dello scudo di un
gladiatore, mi ero guadagnato una certa reputazione, e non ero stato
costretto a uccidere nessuno. Ai gladiatori non era permesso di morire,
e nelle rare occasioni in cui avevo affrontato uno degli altri mirmidoni
lo avevo disarmato. Il popolo amava quella forma di galanteria. La
maggior parte degli altri mirmidoni mancava dell'addestramento formale
che io avevo ricevuto, e i duelli erano quelli in cui mi sentivo
maggiormente a mio agio.

Il rischio di morire subentrava solo nelle azioni di gruppo come quel
primo combattimento che ho descritto. Per consacrare ogni giornata di
scontri alla Terra e all'imperatore, quelli di noi che non combattevano
nelle arene più piccole o in tornei di duelli avrebbero versato il loro
sangue nelle mischie di apertura. Chiamatela una tradizione. Vi presi
parte spesso, a volte cavandomela per un pelo e altre con un trionfo
spettacolare. Una volta ne uscimmo senza perdere un solo uomo, e in
un'altra occasione rimanemmo solo io e Switch. Kiri aveva lasciato il
Colosso poco dopo il mio arrivo e Banks era morto poco tempo dopo quella
prima vittoria, ucciso in un duello con il gladiatore capitano Jaffa,
quando la sua lancia lo aveva raggiunto in una giuntura dell'armatura.

Anche gli Umandh venivano costretti a combattere. Una volta vidi un
quartetto di xenobiti ronzanti affrontare un paio di pantere importate
da fuori il pianeta. Una di esse cadde rapidamente, perché i grossi
felini erano pungolati dalla fame e da iniezioni di ormoni che servivano
a farli infuriare. Gli altri -- avendo appreso che quelle cose aliene
erano predatori -- cedettero al panico e cercarono di difendersi,
sferzando i grandi felini con i tentacoli. Riuscirono nell'intento, ma
non prima che un altro di loro venisse ferito in modo grave, versando il
suo pestilenziale sangue verde sui mattoni. Personalmente, non avevo mai
affrontato una di quelle creature, a cui non era permesso combattere
contro i mirmidoni umani. Anche senza armatura avevano possibilità di
vincere e non sarebbe stato accettabile che un figlio della Terra
cadesse per mano -- o per quello che passava per mani -- di un barbaro
xenobita.

\begin{figure}
	\centering
	\def\svgwidth{\columnwidth}
	\scalebox{0.2}{\input{divisore.pdf_tex}}
\end{figure}

La mensa dei mirmidoni nei dormitori del colosseo puzzava di sudore e di
carne generata nelle vasche, e aveva l'odore di casa. Dopo quasi un anno
nel Colosso, dopo cinquantasette degli scontri di gruppo previsti dal
mio contratto e quasi altrettanti duelli non letali, quel posto
ammuffito con il suo basso soffitto e le lampade tremolanti era una casa
più di quanto lo fosse mai stato il Riposo del Diavolo. Venivo sempre
accolto con gesti amichevoli da Elara e dagli altri che mi conoscevano,
e dai sussurri delle reclute inesperte. Di recente il trasporto per
truppe della Legione, l'\emph{Incrollabile,} era entrato in porto
danneggiato, scaricando i pochi soldati foederati a contratto che
volevano abbandonare la guerra. Per loro la vita dei mirmidoni nel
Colosso di uno strano nuovo mondo era una vacanza, un paradiso dopo i
rigori del vero combattimento.

«A Wodan per poco non abbiamo spazzato via i Pallidi» stava dicendo uno
di essi quando gli passai accanto. «Il primo strategos Hauptmann ha
guidato di persona la sortita.»

«Davvero?»

Il foederatus annuì da sopra la sua bottiglia di bevanda energetica.
«Certo. Come credi che abbiamo fatto a prendere in ostaggio così tanti
di quei demoni?»

Mi fermai ad ascoltare. Ostaggi cielcin. Quel pensiero ridestò una parte
di memoria morta da tempo. Parole sparse del loro linguaggio mi
risuonarono nelle orecchie. Faticando per {sopravvivere} nelle strade,
come avevo fatto, avevo dimenticato la guerra, che mi era parsa sempre
così remota, così distante. I mostri mi erano sembrati dipinti lungo i
contorni della mappa, ma adesso stavano strisciando più vicino,
avanzando dal Buio.

«Ostaggi?» chiese Switch, seduto al tavolo con il foederatus, poi mi
vide e agitò una mano. «Had! Devi sentire questo!» Mi incitò ad
avvicinarmi e anche se avevo già mangiato mi sedetti sulla panca accanto
a lui. «Lui è Kogan, un mercenario.»

«Ero un mercenario» precisò Kogan, parlando con un accento marcato che
non riconobbi, senza dubbio quello di una minoranza etnica dell'interno
di un qualche pianeta di cui non avevo mai sentito parlare, e mi porse
la mano. Finalmente avevo imparato il senso di quel gesto dei popolani,
e la strinsi. «Kogan» disse.

«Had» mi presentai, con un'occhiata in tralice a Switch. «Sei stato in
guerra?»

«Nella battaglia di Wodan, quarant'anni fa. Ho appena mollato il mio
contratto con le Legioni e lasciato la mia compagnia.» Si grattò la
barba. Ai miei occhi Kogan appariva molto più chiaro di pelle degli
Emeshi, anche se aveva una cicatrice da plasma in alto su una guancia
che la trasformava in uno strato di tessuto cicatriziale a bolle. Il
grosso collo brulicava di tatuaggi in parte nascosti dai peli fitti
quasi quanto la sua barba. «Abbiamo preso una delle loro navi-mondo, o
quel che ne rimane. Quei demoni l'hanno messa fuori uso in orbita prima
che il loro capo fuggisse a velocità di curvatura.» Sollevò la bottiglia
di plastica della sua bevanda. «Un punto per la Terra.» Poi mi osservò
con aria riflessiva. «Allora sei tu quell'Had di cui continuo a sentir
parlare?»

Questo aveva cessato di sorprendermi anche se non cessava mai di darmi
fastidio. Risposi come avrei potuto fare alla corte di mio padre. «Non
so cosa hai sentito, ma che io sappia qui sono il solo che si chiami
Had.»

«Ho sentito che sei davvero notevole come duellante e ti ho anche visto,
nel tuo duello con quella gladiatrice dai capelli rossi. Com'è che si
chiama?»

«Amarei» rispose Switch, assestandosi inconsciamente la criniera rossa.

«Quella.» Kogan svuotò la bottiglia. «Ho anche sentito dire che hai
ricevuto un addestramento di palazzo, che sei una sorta di nobile.»

Lo studiai, socchiudendo gli occhi mio malgrado. «L'ho sentito dire
spesso.» Impaziente di distogliere l'attenzione da me chiesi: «Avete
catturato dei Cielcin a Wodan?»

«Solo un paio di centinaia. Hauptmann li ha dati allo
\foreignlanguage{italian}{uil}» replicò Kogan, inclinando la testa con
fare da cospiratore nel riferirsi all'ufficio di Intelligence della
Legione, poi si protese maggiormente verso di me. «Stavo giusto dicendo
al ragazzo che prima che lasciassi la mia compagnia il comandante Alexei
-- il mio vecchio capo -- ne ha trattenuti un paio per divertircisi.»

«Divertircisi?» ripetei, accigliandomi. «Non ho mai sentito di nessuno
che cercasse di tenere schiavi i Cielcin.»

«Suppongo che quella balla sul fatto che saresti un nobilotto sia
davvero una balla, allora.» Kelgan sorrise. «Stando a quello che ho
sentito io, i palatini commerciano in Cielcin fin dall'inizio della
guerra.»

Una mano mi si sollevò a premere sull'anello che portavo sul petto,
sotto la stoffa della tunica, e feci una pausa lunga un respiro per
evitare di dire qualcosa di stupido. Non avevo mai sentito parlare di
una cosa del genere, ma questo non faceva necessariamente di Kogan un
bugiardo. Piuttosto che dissentire o dire qualcosa che poteva rivelare
qualsiasi tipo di verità sul mio conto, gli chiesi: «Con quale compagnia
eri? I Draghi di Cousland?» Avevo sentito fare il nome di quella
compagnia in connessione con l'\emph{Incrollabile}, con le navi
imbarcate nell'enorme stiva del trasporto. Switch stava assorbendo le
notizie sulla nave parcheggiata in orbita come acqua per qualcuno che
non beve da una settimana.

Kogan sputò sul pavimento, attirandosi più di un'occhiataccia dai
mirmidoni del tavolo accanto. «I fottuti Draghi di Cousland? Ero con il
Cavallo Bianco, sotto sir Alexei Karelin. Ti sembro uno dei
mangia-pillole di Arno Cousland? No.» Calò una mano sul tavolo. «Ho
fatto diciassette anni di servizio attivo con la Compagnia del Cavallo
Bianco. Quasi centoventi anni standard.» Si riferiva al tempo trascorso
dentro e fuori da una capsula criogenica. «Ho servito per cinque
contratti la Legione in sette grandi battaglie mentre quei cani di
Cousland facevano i passacarte e marciavano nelle eleganti parate di
Hauptmann.»

Mi alzai lentamente, in modo che il mio gesto non venisse {recepito}
come una minaccia e mi inchinai leggermente. «Non volevo offenderti,
messere.»

«Offendermi?» Kogan scosse il capo, d'un tratto cordiale. «No, voi tonti
vincolati a un pianeta non potete offendermi. Stavo solo correggendo il
tuo errore.»

\begin{figure}
	\centering
	\def\svgwidth{\columnwidth}
	\scalebox{0.2}{\input{divisore.pdf_tex}}
\end{figure}

Alcuni giorni più tardi lasciai l'addestramento e uscii in fretta nel
corridoio, grato per il condizionatore che faceva del suo meglio per
mantenere un po' di frescura e ancor più grato per l'improvvisa
solitudine. Kogan elargiva alla nostra squadra lunghe dissertazioni
sulle sue imprese nella battaglia di Wodan, raccontando come la sua
compagnia di foederati aveva assistito la 437° e 438° Legione Centaurina
-- sotto il comando diretto del duca Titus Hauptmann -- nella
distruzione di una delle navi-mondo fortificate dei Cielcin. Avrebbe
potuto essere una bella storia se chi la raccontava non fosse stato un
narratore bellicoso e divagante.

Pensai malinconicamente a un bagno nell'area comune a disposizione dei
mirmidoni liberi. All'ora di cena era probabile che fosse quasi vuota e
non avevo combattimenti in programma per la settimana successiva. Mentre
camminavo la mia mente prese a divagare e ricordai il mio precedente
incontro, quello menzionato da Kogan contro la gladiatrice Amarei. Era
stato il mio ventisettesimo duello -- la mia ventisettesima vittoria --
da quando mi ero fatto assumere al Colosso di Borosevo dalla dottoressa
Chand. A dire la verità, per poco non era stata una sconfitta perché lei
era la migliore combattente che avessi mai visto, e avevo vinto solo
perché avevo cominciato a manipolare l'armatura invece di combattere
come se fosse stato un vero duello. Amarei portava una delle tute
aderenti da combattimento usate da tutti i gladiatori, che non aveva
altro modo di registrare i colpi incassati se non irrigidendosi, per cui
ripetuti colpi alle braccia avevano rallentato il tempo di risposta
programmato della tuta. Forse era stata una cosa subdola, ma non era
stata lei ad avere linee rosse sanguinanti sull'interno del braccio e
sul petto. Non era stata lei a versare il suo sangue nell'arena.

Scesi una rampa di gradini di metallo e sbucai in un corridoio curvo,
oltrepassando file di dormitori, con i nomi illuminati su pannelli a
parete al di sopra della serratura a impronta del palmo. Seguendo il
corridoio arrivai dove incrociava una galleria che {saliva} verso la
strada e il campo di atterraggio del complesso, oltrepassandola per
raggiungere i bagni che erano vicino alle celle di detenzione, dove
c'era il blocco dei mirmidoni che erano dei detenuti. Svoltai l'angolo
con passo spedito e per poco non gettai a terra un uomo alto dalla veste
nera.

Non nera. Ancora più scura.

Lui farfugliò, barcollando all'indietro contro una guardia che portava
una strana uniforme marrone con le spalline color crema. «Attento a dove
vai, schiavo!» Poi si raddrizzò e notò il mio aspetto e il mio
abbigliamento semplice mentre si premeva un panno profumato sulla faccia
nel tunnel maleodorante.

Cauto, mi inchinai profondamente, allungando davanti a me la gamba
destra. «Chiedo perdono, Vostra reverenza, ma non sono uno schiavo.»

Il cantore abbassò il fazzoletto, rivelando un naso adunco arricciato
per il disgusto. «No, suppongo di no, signore.» La sua voce aveva un
accento bleso e strascicato da aristocratico, una liquida altezzosità
che mi indusse a serrare i pugni. Era alto quasi quanto me, tanto che a
una prima occhiata avevo pensato fosse un palatino, anche se un po'
basso per quella casta. Uno studio più prolungato mi rivelò però che era
un patrizio perché i leggeri segni chirurgici che contrassegnavano
quella casta inferiore, potenziata artificialmente, lo avevano tradito.

No, non era neppure un patrizio. Serrai la mascella e sentii la pelle
che mi si accapponava.

In quel prete c'era qualcosa di \emph{sbagliato}. Qualcosa che non
quadrava. Nella scarsa luce potevo vedere che un occhio era di un
azzurro penetrante mentre l'altro era nero come la pece. Aveva una massa
di folti capelli biondi oliati e pettinati all'indietro da un volto
squadrato dalla mascella pesante, con il naso piegato e le ampie spalle
incurvate. Il sangue nobile che mi scorreva come fuoco nelle vene
affondò nel suo come cera, e una cinquantina di minuscole imperfezioni
risultarono evidenti nella faccia, nella postura e nel portamento, ancor
più che nei servi e nei plebei che avevo conosciuto. «Togliti di mezzo»
disse.

Obbediente, mi trassi da un lato e mi addossai alla parete, concentrando
la mia attenzione sul quartetto di guardie, le cui uniformi mi erano del
tutto sconosciute: giacca marrone scuro stretta in vita da una cintura e
alti stivali neri. Ciascuno portava sul braccio destro una mostrina con
lo stemma di un cavallo bianco rampante sullo sfondo di tutto quel
marrone. Ricordai le parole di Kogan: la Compagnia del Cavallo Bianco.
Liberi mercenari. Foederati. I quattro spingevano in mezzo a loro su un
carrello una capsula criogenica verticale che fluttuava a parecchi
pollici dal pavimento ed era vuota, quiescente, con le luci fioche. Il
gruppo proveniva dalla sezione della prigione. Addossato alla parete
lanciai un'occhiata nella direzione da cui erano giunti e mi morsi un
labbro.

Presi una decisione e mi schiarii la gola. «Chiedo scusa, messeri. Non è
che per caso appartenete alla Compagnia del Cavallo Bianco?»

La scorta del cantore si girò e rallentò un poco il passo. L'uomo in
nero continuò a camminare per un breve tratto, poi si fermò quando la
più anziana delle quattro guardie rispose.

«Agli ordini di Alexei Karelin?»

«Vattene, signor nessuno» ingiunse il cantore, socchiudendo gli occhi in
quella larga faccia tutt'altro che attraente nel fissarmi con ira.
«Subito.»

«\emph{Sir} Alexei Karelin» mi corresse un soldato più giovane, nel
quale l'orgoglio ebbe la meglio sull'ordine del suo padrone.

«Perdonami.» Mi inchinai, non nello stesso modo formale di poco prima,
ma sfruttando quel momento per esaminare la capsula che fluttuava nel
suo campo di soppressione. Era decisamente troppo grande per un uomo,
una losanga sospesa grossa abbastanza da contenere una mucca. Se Kogan
aveva detto la verità, sapevo cosa c'era stato dentro. Non una mucca, ma
neppure un essere umano. «Perdonami, non mi ero reso conto che fosse un
cavaliere.» Feci una pausa e mi umettai le labbra. Quello di sopra era
ancora spaccato per il pugno con cui Amarei lo aveva lacerato la
settimana prima. «Assoldate combattenti?» Era una domanda oziosa, che
non mi aspettavo desse risultati di sorta.

Il cantore tirò di nuovo fuori il fazzoletto dalla manica e se lo
premette sulla faccia con un'espressione improvvisamente dura in quegli
occhi spaiati mentre avanzava verso di me. Ah, che espressione di
disprezzo aristocratico! L'avevo vista spesso in mio padre, ma no...
questa somigliava di più a quella negli occhi di Crispin, incontrollata
e febbrile «Sei sordo, ragazzo?» Mi afferrò per il {davanti} della
camicia e mi sbatté contro la parete. Avevo incassato di peggio, di
molto peggio, quindi cercai di non sorridere dei suoi sforzi. Che
pensasse pure di avere il controllo della situazione. «Ti ho detto di
andartene.»

Ignorando ostentatamente il prete che mi teneva per la camicia mi
rivolsi alle guardie. «Parlo otto lingue, cinque delle quali molto bene,
e ho quasi un anno di esperienza di combattimento nel Colosso.» Il
pensiero mi era letteralmente venuto alla mente mentre lo esprimevo,
eppure era lì: un modo per lasciare Emesh, e presto. Switch sarebbe
potuto venire con me, come pure Pallino e gli altri, se lo avessero
voluto. I foederati si mossero a disagio, adocchiando il prete furente,
ma mi augurai che anteponessero gli affari al timore. Erano i soldati
che dovevo ascoltare, non l'uomo della Cappellania.

Avrebbe potuto funzionare, ma il prete tornò a sbattermi contro la
parete. La mia testa colpì la pietra e sussultai, perdendo la
focalizzazione per un secondo mentre lui indietreggiava, pulendosi le
mani sul davanti delle vesti nere di stoffa sintetica. Evitai ancora di
reagire perché quell'uomo era un sacerdote della Sacra Cappellania
Terrestre, unto con la cenere del Mondo Natale. Quale che fosse il mio
sangue, colpirlo avrebbe significato la morte. Rivolse un cenno alle
guardie. «Storditelo.»

«Reverenza?» chiese una di esse, spostando lo sguardo dal prete al suo
superiore.

«Storditelo!» gridò il prete. «E lasciatelo qui!»

Non ricordo neppure l'arma che usciva dalla fondina o di aver sbattuto
contro il muro.

\begin{figure}
	\centering
	\def\svgwidth{\columnwidth}
	\scalebox{0.2}{\input{divisore.pdf_tex}}
\end{figure}

Qualcosa mi colpì la faccia, costringendomi ad aprire gli occhi e a
lasciar passare la dannata luce. Tutto quello che riuscivo a vedere era
luminosità e per un folle istante pensai di svegliarmi in quella dannata
clinica vicino al campo di atterraggio dell'astroporto, che avrei visto
di nuovo la vecchia dalla faccia rossa e la sua esile assistente, che il
mio tempo su Emesh era ricominciato daccapo. «Perché diavolo sei
svenuto, \emph{momak}?» Era la voce di una donna anziana, con un accento
marcato. Mi schiaffeggiò di nuovo e mi puntò una luce negli occhi, una
semplice torcia a stilo che serviva a controllarmi le pupille per
verificare se c'era un trauma alla testa. Era la dottoressa Chand, e
dietro di lei c'era Switch, con la preoccupazione dipinta a chiare
lettere sulla faccia, le braccia incrociate e il mento basso.

«Sta bene, dottoressa?»

«Il prete» dissi, mentre tutto il mondo prendeva a girare quando cercai
di sedermi. Chand mi serrò le mani sulle braccia per sostenermi, con le
unghie che mi affondavano nella carne. «Ahi! Lasciami andare!»

«Solo se la smetti di cercare di muoverti.» Abbandonò la presa sul
braccio con un cipiglio che rese indistinto il suo tatuaggio, controllò
lo scanner posato per terra accanto a lei, poi me lo premette sulla
pelle. Esso creò un formicolio e mi trasmise una breve pulsazione per
tutto il corpo. «La conduttività è elevata. Se non avessi troppo
buonsenso per supporlo, ragazzo, direi che sei stato stordito.»

«Lo sono stato!» insistetti, concedendomi di riposare con la ruvida
pietra alle spalle. «Dal prete. Lui dov'è?»

«Quale prete?» chiesero contemporaneamente Switch e Chand.

Lo descrissi, massaggiandomi la faccia con le mani. Mi doleva il fianco
e la porzione sopra le costole sembrava indolente e inspessita. I
vestiti mi aderivano addosso. Mi passai una mano sui capelli arruffati,
che erano ricresciuti nei mesi passati da quando era cominciato a
decorrere il mio contratto nel Colosso. Quando fui assalito da un
accesso di tosse Chand mi porse una bottiglia di quella bevanda fra il
blu e il verde che servivano sempre al colosseo e la trangugiai pur
facendo una smorfia per il suo sapore chimico misto a dolcificanti
economici. Quando finii di raccontare l'episodio, Switch si era fatto
del colore del latte -- o sarebbe stato così se non fosse stato per le
onnipresenti lentiggini -- e il suo volto dai lineamenti fini appariva
svuotato. Si morse l'interno di una guancia, con le braccia ancora
conserte sul petto allargato da mesi di combattimenti e di addestramento
che lo avevano rinforzato. Sapevo che stava avendo il mio stesso
pensiero e lo espressi ad alta voce. «Se quello che dice Kogan è vero,
credo che ci sia un Cielcin nel blocco delle prigioni.»

«Ma perché?» Chand richiuse il kit medico e fece schioccare le dita in
direzione di Switch.

Lui non si mosse, si limitò a stare lì fermo ad annuire con gli occhi
chiari dilatati, mentre si tormentava l'unghia di un pollice. «Spero che
ti sbagli.»

Prima che potessi rispondere, Chand trafisse il giovane mirmidone con
uno sguardo rovente. «A cosa servono tutti quei muscoli se non vuoi
aiutare una vecchia ad alzarsi da terra?» Rabbonito, Switch la aiutò ad
alzarsi, e lei giurò di aver sentito le sue ossa che scricchiolavano.
Trasse un respiro e si appoggiò al braccio di Switch. «Direi di lasciar
perdere, ragazzi. Non è il caso di pestare i piedi alla Cappellania.»

Ancora seduto con la schiena contro la parete di pietra guardai lungo il
corridoio nella direzione che quel cantore senza nome e le guardie del
Cavallo Bianco dovevano aver preso per lasciare il complesso e tornare
sulla strada e lungo il Canale Rosso. «Qualche idea su chi fosse?»

«Hai detto che era gobbo?» chiese Switch.

«Eh?» Chand sollevò lo sguardo su di lui e gli batté distrattamente un
colpetto sul braccio. Era così minuta che accanto a lei il ragazzo
sembrava quasi un gigante. «Aveva l'aria di un patrizio, con una faccia
che ti viene voglia di camminarci sopra?»

Involontariamente pensai a Severn, l'assistente della vecchia priora
Eusebia. Quel cantore dal volto affilato aveva avuto la stessa aria di
crudele dignità e non ebbi difficoltà a immaginare che anche Eusebia, da
giovane, dovesse aver sogghignato in quello stesso modo. Forse lo
imparavano al seminario, su Vesperad. «Capelli biondi, occhi di due
colori diversi.» Mi indicai gli occhi, agitando le dita.

Chand esalò un respiro sibilante fra i denti. «È il bastardo della
grande priora. È sempre con il gruppo del conte.»

«Il... bastardo della grande priora.» Mi accigliai. Non aveva senso. I
palatini non avevano figli bastardi -- non spesso, in ogni caso -- e la
grande priora era di certo una palatina. Semplicemente, era una cosa che
non si faceva. L'Alto Collegio vagliava la richiesta di un figlio
avanzata da ogni coppia palatina e si assicurava che le stravaganti
alterazioni nella genetica dei genitori non provocassero un nato morto o
un mostro come quello. Era così che l'Impero manteneva il controllo sui
suoi nobili, controllandone il destino genetico attraverso la loro
prole, garantendo che qualsiasi palatino alla ricerca di un erede
dovesse inginocchiarsi e umiliarsi davanti al trono. Pensai alla miriade
di difetti del prete: le spalle spaiate, il naso storto, la fronte
rigonfia e gli occhi anch'essi spaiati. Mutazioni causate dal fiorire
trascurato della sua eredità genetica palatina, da un eccesso di
cromosomi che gli inacidiva il sangue. Avevamo un termine per questo. «È
un intus?»

«Non farti sentire da lui a chiamarlo così» ammonì Chand, in tono
tagliente. «Gilliam Vas siede nel consiglio del conte e ti farebbe
mettere alla gogna prima che possa scusarti.»

«No, non potrebbe» scattai. «Non si va alla gogna per diffamazione. La
pena prescritta è di non più di venti colpi di frusta.» In realtà erano
non più di quindici... dovetti consultare una copia dell'Indice. C'era
stato un tempo in cui conoscevo la punizione formale per ogni peccato,
crimine e disobbedienza prevista dal canone della Cappellania perché mio
padre aveva insistito al riguardo, ma è stato molto tempo fa. Ho
dimenticato molte delle cose che sapevo un tempo. «E non è una
diffamazione se è vero.» L'indolenzimento causato dalla scarica dello
storditore stava svanendo e diventando un dolore più simile all'odio,
denso e ulceroso. Gemendo, cercai di alzarmi ma ci rinunciai scuotendo
la testa. Dovetti spingere indietro Switch e la dottoressa. «Restate qui
un minuto» mormorai, chiudendo gli occhi. Un momento più tardi domandai:
«È vero?»

«Riguardo al cantore Vas?» Chand sputò e grugnì qualcosa in durantino
che non riuscii a sentire. «Credo di sì. Voglio dire... lo hai visto. È
chiaro che qualcosa ha scompaginato i suoi nucleotidi.»

Noi umani abbiamo sempre attribuito una virtù morale alla bellezza, e
noi più di tutti. Adesso mi chiedo se una simile beffa e un tale abuso
avessero modellato Gilliam nello spirito fino a farlo corrispondere alle
sue mutazioni o se la sua petulante crudeltà fosse una cosa innata.
Adesso posso \emph{quasi} compatirlo, ma a quel tempo il dolore causato
dallo storditore e il mio orgoglio ferito lo rendevano impossibile.

Non stavo ascoltando perché la mia attenzione era tornata a concentrarsi
sul corridoio che portava alla sezione della prigione. Il sudore causato
dalla scarica dello storditore mi faceva desiderare quel bagno, ma era
probabile che adesso il posto fosse affollato dagli altri mirmidoni, e
io ero ancora un palatino quanto bastava per desiderare la solitudine
per cose del genere. La mia antica curiosità mi stringeva nei suoi
artigli e dovette trasparirmi in parte dal volto, perché Switch disse:
«Had, non farlo.»

«Cosa non devo fare?» Sollevai lo sguardo su di lui e su Chand, cercando
di assumere un tono innocente.

«Lascia perdere» insistette Switch, liberandosi da Chand per ergersi su
di me. «Kogan racconta un sacco di cazzate.»

Non gli risposi e rimasi seduto in silenzio con le gambe protese
attraverso il corridoio. Mi parve di sentire in modo vago un rumore di
sandali che strisciavano sul pavimento rozzo, ma quando sollevai lo
sguardo non c'era nessuno. Mi venne spontaneo pensare a Gibson, a cui
non pensavo più da mesi. La storia di Kogan. Il cantore. Le compagnie di
foederati. Il trasporto per truppe della Legione in orbita intorno a
Emesh. Voci, verità o totale finzione: ogni elemento era un pezzo di
vetro colorato, una tessera di mosaico. Gibson mi avrebbe detto di fare
un passo indietro e di cercare di scorgere l'immagine completa. Non ero
uno scoliasta, ma intravidi lo stesso cosa stava succedendo.

«C'è un Cielcin nella segreta del colosseo.»

