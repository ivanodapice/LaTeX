\chapter{Il commiato}

Finalmente giunse il giorno precedente la mia partenza, che sorse
argenteo e soleggiato, dipingendo del suo fulgore il paesaggio fra il
verde e il nero. In alto il cielo era del colore di un mare turbolento e
agitato dalla tempesta, ma c'era il sole ed era una delle giornate più
belle che avessi mai visto. Mi pareva sbagliato... la pioggia e le
tempeste che avevo incontrato nel lasciare Meidua mi parevano molto più
appropriate per una fine del genere. Ufficialmente mi sarei imbarcato su
una navetta il mattino successivo per essere portato a bordo della cocca
mercantile \emph{Lavoratore Lontano}, la cui abituale rotta circolare
nell'interno dell'Impero avrebbe finito per portarmi al college Lorica e
all'esilio su Vesperad. Questo ufficialmente. Sapevo da fonte autorevole
che sarei scomparso in un momento della notte e sarei stato invece
trasportato nella città insulare di Karch, nel cuore dell'Oceano
Apollano, a est del Riposo del Diavolo, dove avrei incontrato il
misterioso contatto di mia madre.

Tentando di apparire disinvolto aspettai vicino al campo di atterraggio
l'avvicinamento della navetta di mio padre, il mezzo orbitale che mi
avrebbe dovuto portare al mio appuntamento con la \emph{Lavoratore
	Lontano} e con il mio destino. Emissari del Riposo del Diavolo stavano
venendo ad accompagnarmi al mio esilio, e le buone maniere richiedevano
che fossi lì ad accoglierli. Crispin era accanto a me... non avrei
saputo dire se perché si annoiava o se era un sincero interesse. Negli
ultimi minuti si era mostrato incredibilmente silenzioso e questo mi
aveva dato modo di riordinare i miei pensieri sconvolti. Stavo pensando
alle meditazioni degli scoliasti, all'apatia, e cercavo di scolpire
l'immagine più nitida possibile di questo momento, di assimilarne ogni
dettaglio. `La focalizzazione offusca' era solito affermare Gibson. `La
focalizzazione acceca. Devi assimilare tutto di una cosa vedendone la
totalità, non focalizzandoti sui dettagli. Questo è importante tanto per
un governante quanto per un pittore.'

Un nodo mi si formò nella gola mentre me ne stavo accanto a mio
fratello, osservando la navetta in avvicinamento. All'inizio apparve
come una forma minuscola, quasi un uccello, una chiazza al limitare del
mio campo visivo che cadeva dal cielo come una lancia. La forma di
uccello crebbe, divenne un drago, e portò con sé un urlo di furia
metallica che strideva nel cielo... dapprima un rombo sordo come quello
di un tuono, poi un rumore come di parecchie centinaia di spade che
venissero affilate nel firmamento. Il velivolo zigzagò avanti e indietro
nel cielo per liberarsi di enormi quantità di velocità a ogni curva,
proprio come avevamo fatto al nostro arrivo.

«Vorrei che mi permettessero di venire su con te» disse Crispin. «Non
sono mai stato in orbita.»

Non gli risposi e mi schermai gli occhi in tempo per vedere i retrorazzi
accendersi per un momento, eliminando altra velocità mentre essa si
affrettava all'avvicinamento finale. Intorno a noi e sul campo di
atterraggio tecnici con la livrea dei Kephalos si muovevano frettolosi
per gli ultimi preparativi, e trenta legionari imperiali nella loro
lucida armatura bianca, senza volto o occhi per via della visiera
abbassata e del casco sigillato, si schierarono in posizione di riposo
con il fucile in mano, spalla a spalla con i dieci opliti dei Marlowe
che avevamo portato con noi da Meidua.

La navetta si girò per assumere il vettore di approccio e si inclinò
verso l'alto quando i jet di assetto l'aiutarono a rallentare sempre di
più, insieme al campo di soppressione di bordo. Sembrava più una lama di
coltello che un velivolo, del tutto diversa dall'avvoltoio che aveva
portato lì me e Crispin. Venti metri di adamant nero, con lo scafo
fissato su un telaio di titanio capace di resistere agli impatti delle
micrometeore anche senza i proiettori di scudi montati a prua e a poppa,
piccoli dischi concavi che scintillavano come mercurio.

Il lato inferiore della navetta era ancora rovente per l'attrito dovuto
al rientro nell'atmosfera quando essa si posò sul campo di atterraggio,
avvolta in lingue di fumo come un qualche drago malvagio. I tecnici
accorsero per raffreddare l'adamant nero con spray chimici e l'intero
velivolo sibilò come un nido di vipere mentre la passerella si
abbassava. Per un singolo, terribile istante mi sentii certo che la
sagoma larga di spalle di mio padre avrebbe disceso quella rampa con
passo pesante e che il piano accuratamente disposto da mia madre e il
sacrificio di Gibson -- nonché il denaro che avevo nascosto -- non
sarebbero serviti a nulla perché sarebbe stato tutto perduto.

Invece si trattò soltanto di tor Alcuin, con la sua testa rasata, la
pelle scura e le vesti voluminose che gli si agitavano sulle spalle come
bandiere. Sir Roban, altrettanto scuro, seguì sulla sua scia, non in
armatura ma in una semplice tenuta nera semiformale e con la spada di
altamateria che gli pendeva dalla cintura-scudo. Il mio gruppo di
commiato. Un terzetto di funzionari minori li seguì giù per la rampa,
poi fu la volta di un ufficiale di volo anziano e di... Kyra. La tenente
appariva fuori luogo insieme a loro, più giovane degli altri di almeno
un decennio, e mi meravigliai per le sfortunate probabilità che avevano
garantito che fra tutti gli ufficiali piloti di mio padre fosse stata
scelta proprio lei per pilotare la navetta. Era quasi abbastanza da
indurmi a credere che ci fosse davvero un dio e mi odiasse.

Il giovane scoliasta e i funzionari si inchinarono profondamente, Roban
salutò portando il pugno al petto e gli altri ufficiali lo imitarono.
«Per me è un onore accompagnarti nella tua partenza da Delos, lord
Hadrian» dichiarò Alcuin, in toni servili.

Inclinai il capo e scoccai una rapida occhiata a Kyra, che si teneva in
fondo al gruppo, per poi distogliere in fretta lo sguardo pregando di
non essere arrossito. «Mi sentirei poi onorato, consigliere, se Gibson
potesse unirsi a noi» ribattei, incoraggiato dall'assenza di mio padre e
dal piano di mia madre. Se mi ero aspettato una reazione dallo
scoliasta, lui mi deluse perché il suo volto rimase impassibile, con gli
occhi piatti e lisci come agata. Gli altri tradirono tutti un certo
turbamento per il modo in cui strisciarono i piedi con disagio. Nel
profondo, sotto la mia superficie, un carbone ardente si trasformò in
fiamma, una furia diretta contro quell'uomo -- quella macchina
calcolatrice -- che non provava niente, assolutamente niente, per la
brutalità inflitta al suo collega e fratello d'armi da parte dell'uomo
che servivano entrambi. Alcuin doveva conoscere la storia e sapere che
il trattamento subìto da Gibson era una colossale ingiustizia.

Non gli importava... non gli poteva importare, perché quella di
interessarsi a qualcuno era per lui un concetto alieno quanto lo erano
gli xenobiti Cielcin nelle loro navi-mondo labirintiche. Alieno quanto
una qualsiasi delle razze di coloni schiavizzate sui loro stessi mondi.
Addirittura alieno quanto gli dèi oscuri che sussurrano nel silenzio
della notte. «Il tradimento di Gibson è stato una cosa sfortunata» disse
soltanto.

«Hadrian,» interloquì Roban, venendo avanti e porgendomi la mano «mi fa
piacere rivederti prima che tu vada.»

Gli strinsi la mano ma la mia attenzione quasi non si spostò dal volto
di Alcuin. «Anche a me fa piacere rivederti, Roban.» Mi interruppi,
distogliendo infine lo sguardo dallo scoliasta per guardare il volto
rude del cavaliere-littore, con il naso largo e gli occhi profondamente
infossati sotto la fronte pesante. Sentendomi d'un tratto imbarazzato,
dissi: «Avrei dovuto ringraziarti in modo... più appropriato per avermi
salvato. E per tutto.» Spinto da un ricordo improvviso allungai il collo
per parlare al di sopra dei tre funzionari in uniforme. «Grazie anche a
te, tenente.»

Lei si inchinò appena, e Roban mi assestò una pacca su una spalla. «Un
ultimo viaggio insieme, quindi. Hai fatto tutti i bagagli?»

«Certamente!» replicai, con un sorriso che temo non mi si estese agli
occhi.

«So che questo non è il futuro che avevi immaginato, giovane signore,»
interloquì Alcuin, con voce simile a uno strisciare di foglie secche «ma
il tuo ruolo nella Cappellania servirà per la maggiore gloria del tuo
Casato. Un Marlowe nella Cappellania permetterà...»

Con mia sorpresa, fu Crispin a interromperlo. «Lo sa, non ha bisogno di
questo discorsetto.»

Alcuin tacque, irrigidendosi, e chinò il capo. «Capisco quella
necessità, Alcuin» affermai, ansioso di riportare una certa misura di
calma, e cancellai dal mio volto ogni espressione, fissando lo scoliasta
-- il consigliere principale di mio padre -- con una faccia vuota quanto
la sua. La mia serenità era però una cosa superficiale, uno strato di
ghiaccio su acque turbolente, mentre quella di Alcuin era un blocco di
ghiaccio. Potevo sentire nelle orecchie le urla di dolore di Gibson
mentre la frusta lo aggrediva e mi sentii scivolare sempre più lontano
da quel ricongiungimento sul campo di atterraggio, avvertendo
l'immediato bisogno di essere solo.

«Certamente, giovane signore.» Tor Alcuin si inchinò fin sotto le
ginocchia, con le mani infilate nelle ampie maniche, davanti a sé.
«Perdonami.»

«Non c'è niente da perdonare, consigliere» ribattei con freddezza. Avevo
un altro mistero da chiarire, quindi mi girai per parlare con Kyra,
sentendo il rossore che mi saliva al volto. «Tenente, mi sorprende
vederti qui.» Quante erano le probabilità? Avrei voluto chiederlo,
scherzare, cercare di salvare una brutta situazione e di cancellare il
mio precedente errore.

Lei distolse in fretta lo sguardo e abbassò la testa in modo che la
corta tesa del cappello da ufficiale di volo le nascondesse gli occhi.
«Mi hanno detto che ero stata espressamente richiesta.»

Sentii il sangue che mi defluiva dal volto. «Da chi?»

Lei sollevò di scatto lo sguardo, nel quale non c'era traccia della
paura che mi ero aspettato di scorgere, solo qualcosa di affilato e di
duro. «Da te.»

Sta mentendo, pensai, sorridendole. Lo sapevamo entrambi, lo potevo
leggere sul suo volto, nel modo in cui catturava il mio sguardo come non
aveva fatto in precedenza. Ho spesso constatato che questa è una cosa
che fanno i bugiardi: scrutano con attenzione la loro vittima alla
ricerca del momento in cui la credulità getta radici. Conscio di quanti
ci osservavano, replicai. «Oh, ma certo, me ne ero dimenticato! Mi
piacerebbe scambiare due parole in privato, quando avremo un momento.»
Interiormente mi accigliai. Stava succedendo qualcosa, ma non riuscivo a
intuire di cosa si trattasse. Avevo saputo dell'arrivo di quella
delegazione, ma anche così non mi entusiasmava il pensiero di fuggire
dal palazzo di Haspida -- comunque mia madre intendesse organizzare la
cosa -- sotto il naso di Roban e di tor Alcuin.

«Dov'è lady Kephalos-Marlowe?» chiese Alcuin, avanzando di un passo con
fare affettato.

Crispin si spostò per raggiungere il consigliere e si girò a indicare le
cupole del palazzo che ci sovrastava. «Venite da quella parte, tutti
quanti. Per favore.»

