\chapter{La legge degli uccelli e dei pesci}

Sedevo immerso in un cupo silenzio fissando il mare con occhi roventi.
Erano passate due settimane dal mio incontro con mio padre e in quel
tempo avevo fatto l'impossibile per evitarlo, quindi sedevo nell'ombra
protettrice di uno sperone di roccia sulla distesa di sassi che passava
per una spiaggia sotto la nostra acropoli. Là, lontano dalle videocamere
e dagli occhi attenti delle guardie, un ragazzo poteva tenere il broncio
come la natura voleva che facesse. La mia mano era del tutto guarita, e
mentre me ne stavo appoggiato contro la facciata dell'altura disegnavo
su una pagina del mio diario, realizzando il profilo di un peschereccio
d'alto bordo che si dirigeva placidamente in porto, dominando con la sua
mole le piccole giunche da pesca con le loro vele rosse o bianche che
punteggiavano il mare dalla riva all'orizzonte.

Un gabbiano scese in picchiata attraverso l'aria salmastra e trafisse la
superficie dell'oceano per riemergerne grondante e con un pesce nel
becco. Lo guardai allontanarsi, poi guardai la nave che stavo disegnando
allontanarsi sempre di più fino ad aggirare il capo e il faro, diretta
verso la città e l'estuario del fiume.

L'angolo del cristallo datomi da mio padre mi premeva contro, pungente,
attraverso la tasca, e mi ricordava acutamente l'ologramma codificato
che conteneva: una registrazione di mio padre, verificata da terabyte di
codice di autenticazione, che declamava le mie qualifiche a beneficio
dei supervisori della scuola della Cappellania, su Vesperad. Avevo
guardato quella registrazione una cinquantina di volte nelle ultime due
settimane, e ogni volta la mia scorta privata dei vini della casa era
diminuita, ogni volta sul mio diario era apparsa un'altra pagina.

Frustrato, chiusi il libro sulla matita e appoggiai la testa
all'indietro contro la pietra. La mano mi faceva ancora male dove era
stata fratturata anche se sapevo che la cosa sarebbe scomparsa nel
tempo. La massaggiai con la sinistra, notando l'assortimento di piccole
cicatrici simili a teste di spillo che mi punteggiavano la pelle pallida
dalla punta delle dita a metà dell'avambraccio. Splendevano sotto la
luce argentea del sole di Delos mentre flettevo le dita scricchiolanti,
snudando un poco i denti per il disagio. Tor Alma, il medico di
famiglia, giurava che le ossa funzionavano di nuovo a dovere, ma a mia
volta io giuravo che erano cresciute in modo strano ed erano fastidiose
come denti nuovi.

«È qui che vieni quando nessuno riesce a trovarti?»

Non avevo bisogno di girarmi per sapere di chi si trattasse. «Pare di
no.»

Appoggiandosi al bastone di frassino, tor Gibson entrò ondeggiando nel
mio campo visivo proveniente da destra, dopo aver disceso -- cosa
incredibile -- una rampa di parecchie centinaia di gradini mascherati
abilmente dalle pieghe casuali dell'altura rocciosa. L'orlo della sua
bella veste verde strisciava sulla sabbia, ma anche se lo notò non parve
esserne seccato. «Eri in ritardo per le lezioni.»

«Impossibile. Sono le dieci del mattino.» Chiusi gli occhi e tornai ad
appoggiare la testa sulla pietra, ma continuai a sentirlo incombere su
di me e alla fine socchiusi un occhio, riuscendo \emph{quasi} a
sorprendere un'espressione di sconcerto sul suo duro volto rugoso.

«Lo erano tre ore fa» replicò con un cenno del capo. «Adesso è quasi
mezzogiorno.»

Mi alzai così in fretta che un osservatore avrebbe potuto pensare che mi
fossi scottato o che fossi stato punto da uno degli anemoni comuni lungo
la costa. «Io...» Annaspai in cerca di una scusante, ma non ne avevo.

Il vecchio sollevò una mano. «Non ti preoccupare, in realtà non avevi
bisogno di un'altra lezione di base di retorica.»

Feci una smorfia. «Forse no.»

Con squisita lentezza Gibson si adagiò sull'ultimo gradino della scala
che risaliva l'altura verso il castello. Mi affrettai, per andare ad
aiutarlo ma lui mi allontanò con un gesto. «Questa è la seconda volta in
altrettante settimane che salti una lezione, Hadrian, e non è da te.» Mi
limitai a grugnire per tutta risposta, e Gibson esalò un grande respiro.
«Capisco. Dopotutto, forse quel corso di base di retorica ti
servirebbe.»

Accigliandomi, gli volsi le spalle e mi diressi dove finivano le pietre
e la sabbia si perdeva nell'acqua argentea come vetro. Senza una luna
che esercitasse su di essa la sua trazione, il mare era sempre placido,
con appena qualche mulinello in direzione della piccola spiaggia. «Non
riesco ancora a crederci. La dannata \emph{Cappellania}, Gibson.» Ne
avevamo già discusso, due volte.

«Potresti diventare importante, sai.»

«Non voglio essere importante, dannazione.» Sferrai un calcio a una
pietra con il lato interno del piede, mandandola a saltellare sull'acqua
dove un altro gabbiano si stava tuffando. «Ho detto a mio padre che
volevo diventare uno scoliasta. Te l'ho detto, vero?» Dalla mia voce
traspariva un assoluto senso di sconfitta, temperato soltanto dalla
derisione dell'autocritica, come se solo il buffone dell'imperatore
avrebbe potuto fare una cosa del genere.

Gibson rimase a lungo in silenzio... tanto a lungo, in effetti, che
quasi ripetei la frase. Alla fine replicò con voce tremula: «Hai le
capacità per riuscire. Sei abbastanza acuto, ne ho parlato io stesso con
tuo padre in un paio di occasioni, ma ha rifiutato la cosa a priori.»

«Ma potrei farcela?» insistetti, sorvolando su quella notizia
addizionale. «A diventare uno scoliasta, intendo.»

Gibson scrollò entrambe le spalle. «Con il tempo potrebbero insegnarti a
pensare in modo adeguato, Hadrian, ma non dovresti sfidare tuo padre in
questo.»

«Questo è un mio fardello da portare» ribattei, sfoggiando la mia
migliore espressione da patrizio. «È questo che stai dicendo?»

Lo scoliasta passò di colpo a esprimersi in inglese classico. «Se la
sopravvivenza richiede di prendere le armi, allora le devi prendere»
replicò.

Inarcai un sopracciglio e chiesi nella mia lingua natale: «Shakespeare?»

«Serling.» Lui guardò verso il cielo e le nuvole simili a ragnatele
spinte dal vento sotto la luce bianca del sole. «Anche se credo che la
citazione sarebbe più adatta se tuo padre ti stesse mandando a far parte
delle Legioni.»

«C'è l'Inquisizione» gli ricordai, accigliandomi. «Loro sono peggio.»

Gibson dondolò il capo in un gesto affermativo, mantenendo il mento
saldamente appoggiato al bastone. «Questo è vero.» Si grattò una basetta
leonina con la riflessione dipinta sul volto avvizzito. «Non vedo per te
una via di uscita da questa situazione, ragazzo mio. Se tuo padre si è
preso il disturbo di registrare la lettera su quel cristallo, puoi
scommettere che l'ha già trasmessa su Vesperad e che l'accordo è fatto e
sigillato.»

La mia testa si scosse senza che il cervello le ordinasse di farlo. «Non
lo posso accettare.»

Gibson notò la tensione nervosa sul mio volto e mi puntò contro il petto
un dito nodoso. «Questa è la strada verso la follia, Hadrian.»

Sollevai la testa di scatto. «Prego?»

«La paura è la morte della ragione.» Quelle parole erano una sorta di
riflesso, la risposta automatica della sua mente all'emozione in
questione, tanto in lui che negli altri.

Sbattei le palpebre, smisi di cercare una pietra da lanciare e
dichiarai: «Non ho paura.»

«Di entrare nella Cappellania? Certo che ne hai.» Mi guardò dritto in
faccia, con un volto non diverso da quello di una statua, nel quale gli
unici segni erano quelli lasciati dal tempo e non dalle espressioni.
Avrebbe potuto essere forgiato nel bronzo. «Vuoi diventare uno
scoliasta? Domina quella tua paura o non sarai migliore del resto di
loro.» Agitò una mano in direzione del castello come ad abbracciare
tutta l'umanità laica. «Imita l'azione della pietra, non lasciare che il
futuro ti turbi. Se dovrai farlo, lo affronterai con le stesse armi
della ragione che oggi ti proteggono dal presente.» A quel tempo non
riuscii a riconoscere la citazione: Marco Aurelio, un altro imperatore
romano.

Sorridendo, ribattei con un aforisma del \emph{Libro della Mente}:
«L'uomo spaventato divora sé stesso.»

Qualsiasi altro tipo di uomo avrebbe sorriso, ma la bocca di Gibson si
incurvò appena mentre annuiva in segno di approvazione. «Sai queste
cose, ma non le hai imparate.» Fra noi cadde un altro silenzio e io
tornai a guardare gli uccelli intenti a cacciare. Erano veri gabbiani
che discendevano da una scorta di sperma portata su Delos, con i suoi
oceani, innumerevoli secoli prima. {Erano} veri gabbiani terrestri
bianchi e grigi, come quelli che avevano volato lungo le coste
dell'Antica Terra ai tempi di Sargon e anche prima. «La Cappellania non
è una brutta alternativa. Saresti al di sopra dei nobili dell'Impero.
\emph{Vedresti} l'Impero, il Commonwealth, forse perfino la Demarchia di
Tavros. Hai l'opportunità di usare bene il tuo addestramento.»

«Si supponeva che venissi addestrato alla diplomazia e non... non...»
Non riuscii a trovare la parola giusta.

«Alla teologia.»

«Alla propaganda.» Sogghignai. «È soltanto questo. Tengono tutti in riga
con la paura, Gibson, perfino mio padre. Sai che ha ammesso che è per
questo che mi sta mandando da loro? Dice di aver bisogno di qualcuno che
`sia dalla sua parte', come se stesse progettando qualcosa di illegale.»
Serrai di nuovo i denti. «È questo tutto ciò che sono? Uno strumento?
Cerca di procurarsi con l'imbroglio un vero titolo?

Mentre aspettavo una risposta guardai lo scoliasta che sedeva come
l'icona del Tempo che Fugge, nel sancta sanctorum della nostra
Cappellania, consumandosi lì appoggiato al suo bastone. A replicare fu
però lo scoliasta e non l'uomo sotto di esso. «In senso tecnico, tutti i
Casati palatini hanno figli proprio per quella ragione. Si tratta di
strategia.»

«Pezzi degli scacchi.» Sputai sulla spiaggia. «Non voglio essere una
pedina, Gibson. Non voglio stare al gioco.» Avevo sempre odiato quella
metafora.

«Devi giocare, Hadrian. Non hai scelta. Nessuno di noi ne ha.»

«Non sono \emph{suo}.» Pronunciai quelle parole come avrebbe potuto fare
un serpente, fissando il mio insegnante con occhi roventi e il veleno
che mi gocciolava dalla lingua.

Lo scoliasta socchiuse gli occhi velati. «Non ho mai detto che lo fossi.
Siamo tutti pedine, ragazzo mio. Tu, io, Crispin. Perfino tuo padre e la
viceregina-duchessa. È così che funziona l'universo. Però ricorda!» La
sua voce salì di tono e lui agitò il bastone contro la pietra bianca
segnata dagli elementi. «Non importa chi cerchi di muoverti, che si
tratti di tuo padre o di qualsiasi altro uomo di potere, tu avrai sempre
una scelta perché la tua anima è nelle tue mani. Sempre.»

Era strano sentire Gibson -- o un qualsiasi scoliasta -- parlare di
anima. Non sapendo cosa dire riportai lo sguardo sugli uccelli e la loro
caccia, spostandomi lungo la spiaggia per tornare alla pietra su cui ero
stato seduto e recuperando il mio diario con un sussulto di dolore
quando le dita indolenzite si chiusero intorno alla copertina di cuoio
nero. «Che scelta è quella?» Non ripresi a guardarlo, riportando la mia
attenzione sui gabbiani e la loro caccia.

Gibson non rispose, e ne conoscevo il motivo. Perfino lì, lontano dal
castello con le sue orecchie tese e gli occhi curiosi, non poteva
parlare di tradimento. L'istinto all'obbedienza era troppo radicato
dentro di lui. Ma che forma di obbedienza era? Me lo chiesi, e lo faccio
ancora. «Cosa stai guardando?» domandò invece.

«I pesci.»

«Non li puoi vedere.»

«No, finché gli uccelli non li prendono» risposi, indicando, anche se
supponevo che il vecchio non li potesse vedere comunque. Adesso mi rendo
conto che non sapevo quanto fosse anziano il caro Gibson. La sua pelle
era come vecchia pergamena, tesa al massimo, e i suoi occhi... sapete
quanto deve essere vecchio un uomo di nobile nascita perché i suoi occhi
comincino a perdere la vista? Ho conosciuto uomini che avevano più di
cinquecento anni e la cui vista era ancora affilata come un coltello per
scuoiare. A volte penso che il mio caro tutore fosse l'uomo più vecchio
che abbia mai conosciuto, a parte me stesso.

Sempre socratico, lui mi chiese: «E, di grazia, cosa c'è in quei pesci
che cattura la tua attenzione in un momento come questo?»

«Il fato» sussurrai.

«Cosa?» domandò Gibson, la domanda istintiva di qualcuno che sta
diventando sordo.

Fui grato che non mi avesse sentito, perché potevo immaginare la
strigliata che avrei ricevuto per aver osato fare riferimento a una cosa
tanto mistica e dozzinale come il fato. Mi girai di scatto a
fronteggiarlo, scrollai le spalle e riformulai il mio pensiero. «Non
hanno voce in capitolo sull'essere divorati. Di nuovo, sono pedine. La
biologia è destino.»

Gibson inarcò un sopracciglio cespuglioso e sbuffò: «Tutto quello che
dici deve suonare come se fosse appena uscito da un melodramma
eudoriano?»

«Cosa c'è che non va nel melodramma?» Mi illuminai in volto, sollevato
per quel breve intermezzo di umorismo.

«Niente, se sei un attore.»

«Tutto il mondo è un palcoscenico.» Allargai le mani e tentai di
sorridere, assolutamente certo che questo, almeno, era Shakespeare.
Provai anche a scoppiare in una debole risata, ma mi bloccai quasi con
la stessa rapidità con cui avevo cominciato mentre Gibson chiudeva gli
occhi per il tempo di due respiri, un gesto che per lunga esperienza
sapevo essere il suo modo di esercitare la capacità psichica di
reprimere a sua volta il bisogno di ridere. `La mente deve essere come
la sabbia di un giardino, rastrellata e in ordine' aveva scritto lo
scoliasta Imore nel III millennio. «È solo che al momento mi sento come
uno di quei dannati pesci.»

Il vecchio compresse le labbra. «Non so che cosa dirti.»

«Non voglio andare su Vesperad, Gibson.»

«Perché?» Non era un'obiezione ma una domanda penetrante. \emph{Che
	Socrate fosse condannato al Buio Esterno per l'eternità...}

Aprii la bocca, la richiusi, sollevai lo sguardo verso il castello e
tornai ad aprire bocca, dicendo: «Perché... perché è tutto un mucchio di
cazzate. Il Culto della Terra, l'icona, niente di questo è reale. La
Terra non tornerà a noi di nuovo verde e pulita se ci pentiamo per i
peccati dei nostri antenati.» Scossi il capo e sputai le parole
successive come fossero bile. «Pane e giochi circensi.» Mi sentii sporco
anche solo a dirlo, a fare mio quel pezzo della tradizione di mio padre.
Da ragazzo ce l'avevo con la religione, mentre avrei dovuto avercela
solo con la Cappellania.

A quel punto la bocca di Gibson ebbe una contrazione, formando
l'infinitesimale impressione di un sorriso. Era trionfo quello che
vedevo? Poi l'espressione scomparve. «Sai, sono cose che dovresti
proprio tenere per te» commentò.

«Credi che non lo sappia?» Indicai la massa tenebrosa del Riposo del
Diavolo che ci sovrastava. «Non l'ho detto a \emph{lui}! Per la Terra e
l'imperatore! Pensi che sia un idiota?»

«Penso,» rispose Gibson, soppesando le parole con estrema cura «che tu
sia il figlio di un arconte e manchi quindi della cautela propria di una
persona comune.»

Scoppiai in una breve risata, fredda e priva di umorismo. «Cautela? Per
gli dèi dell'inferno, Gibson, non ho dimostrato abbastanza cautela? Sono
anni che mi aggiro in punta di piedi intorno a mio padre e a Crispin. E
a Eusebia e a Severn e agli altri cantori. Devo fare qualcosa...» Un
folle sorriso mi affiorò sul volto quando mi resi conto di cosa fosse
quel qualcosa.

«Quell'espressione non mi piace affatto.» Lo scoliasta quasi si accigliò
di fronte al mio sfoggio di emozione.

Il piano prese forma dietro i miei occhi, formandosi un pezzo dopo
l'altro. «Non andrò.» Pronunciai quelle parole come una preghiera,
piccola, sicura e potente. «Non andrò su Vesperad.»

«Devi farlo.»

«No.» Indicai Gibson, brandendo il diario nella sua direzione. «Hai
detto che ho una scelta.» Abbassai lo sguardo sull'uomo accasciato sui
gradini e un sorriso ferino mi illuminò gli occhi. «Potresti stilare una
lettera di presentazione per il primate dell'ateneo di... diciamo di
Teukros.» Gibson sollevò stancamente lo sguardo su di me, con gli occhi
grigi e velati che contenevano un'espressione pericolosamente vicina a
un sentimento coerente e duraturo. Compresse le labbra e grugnì per lo
sforzo di alzarsi. Momentaneamente dimentico della mia proposta che non
aveva avuto risposta, venni avanti per aiutarlo. Anche con la schiena
incurvata da un numero imprecisato di secoli, era più alto di me, un
chiaro segno che la sua era una linea di discendenza antica quanto gli
imperi. Nella rinnovata quiete dissi: «Potresti farlo? Per me?»

Sapevamo entrambi che quello che stavo chiedendo era un atto di
tradimento nei confronti del suo signore e di secoli di servizio a
Meidua. Gibson conosceva mio padre da tutta la vita, forse si erano
soffermati su quella stessa piccola spiaggia, con Gibson che elargiva
consigli a un giovane e gelido Alistair su come far fronte alle
difficoltà di governo. Dopotutto, mio padre aveva avuto a stento
cinquant'anni quando un omuncolo -- un dono da parte di uno dei suoi
concorrenti Mandari -- aveva ucciso mio nonno e imposto il titolo di
arconte sulle sue spalle. Il vecchio lord Timon era morto nel suo letto,
strangolato da quella persona artificiale mentre era impegnato a fare
l'amore, e mio padre ci aveva messo quasi un secolo -- e la battaglia di
Linon -- per indurre i nobili di Delos a dimenticare quell'imbarazzo.
Una parte di me si chiese se Gibson avesse consigliato l'attacco, se
l'aria fosse stata fatta uscire dal castello del Casato Orin su suo
suggerimento e la loro linea di discendenza fosse stata annientata per
un suo consiglio.

Con voce rotta e parole che erano solo frammenti di loro stesse,
replicò: «Posso farlo.»

Circondai con le braccia quell'uomo che mi era più caro di mio padre,
cercando di reprimere nel mio petto il calore della gioia. «Grazie!
Grazie, Gibson»

Vivendo come facevo in un mondo di servitori, di padrini e di politica,
la vera amicizia mi era sconosciuta. I rapporti che avevo con i miei
genitori non potevano essere descritti come amorevoli, e così pure
quello con Crispin, caratterizzato dalla mia avversione nei suoi
confronti. I legami che avevo con gli altri membri della corte di mio
padre -- sir Felix, sir Roban, tor Alma, tor Alcuin, la priora Eusebia e
tutti gli altri -- erano solo l'attaccamento di uno studente verso
l'insegnante o del padrone verso il servitore. Perfino i miei nascenti
sentimenti per Kyra -- anche se non sapevo o apprezzavo cosa
significassero -- erano filtrati dalla membrana sterilizzante imposta
alla mia vita dalla mia posizione. Solo Gibson era riuscito ad aprirsi
un varco. Come ho detto, era la cosa più vicina a un padre che avessi
mai avuto.

E questo ci condannava entrambi.

