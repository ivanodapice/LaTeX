\chapter{Non oso incontrarti nei sogni}

«Perché passare il sito al setaccio se l'intero posto è stato mappato e
catalogato?» chiesi, mentre salivo i gradini neri al seguito di Valka e
di sir Elomas. Eravamo a Calagah da alcune settimane, e ogni giorno
scendevo lungo la fenditura con Valka, tor Ada o Elomas, per poi
percorrere le anguste gallerie per ore. Io ero soltanto un ospite, un
dilettante, quindi perlopiù mi accodavo a Valka e alla scoliasta o
assistevo i tecnici nello spostamento delle apparecchiature.

Sir Elomas si fermò in cima alla scala, agitando il thermos del tè per
attivare l'elemento riscaldante in esso inserito. «Perché non siamo
certi che lo sia, ragazzo!» Sorrise con un bagliore dei denti candidi
mentre svitava il tappo del thermos. Sopra di noi le irregolari colonne
angolose si stendevano e incurvavano come spine, e lui apparve così
ordinario, così banale e fuori posto sullo sfondo di quell'oscurità
aliena mentre beveva il suo tè. «Chiunque abbia costruito questo dannato
posto...» Scosse il capo. «Ci sono intere camere che abbiamo scoperto
con i rivelatori ai neutrini, sigillate dietro metri di solida roccia,
non sepolte ma costruite in quel modo, come se qualcuno avesse tagliato
attraverso la roccia e avesse insinuato uno spazio al suo interno. È per
questo che ho gente che trascina dappertutto i gravimetri.»

Mi dolevano ancora i muscoli per aver aiutato a trasportarli, il giorno
precedente. Sapevo che stavano sondando le rovine, ma... «Camere
sigillate?»

«Del tutto separate. Costruite sigillate» interloquì Valka. Tirò fuori
una delle sue sfere luminose e la lasciò fluttuare dietro come una luce
fatata.

«C'era qualcosa in quelle camere?» chiesi, stuzzicato nella mia
curiosità. «Oppure erano...»

«Vuote, come tutto il resto» rispose lei. «Ancora più vuote, perché
nelle gallerie principali c'erano alcuni oggetti degli Umandh, quando i
Normanni sono arrivati. Abbiamo praticato un foro in alcune di esse,
quanto bastava per inserire una sonda.»

A parte la luce della sfera di Valka, il buio della galleria era
assoluto. Armeggiai nella giacca alla ricerca della lampada portatile,
seguendo le forme spettrali dei miei compagni oltre una curva. «Perché
costruire camere separate?»

«Credo che il `come' sia un domanda più interessante, non trovi?»
ribatté Valka, guardandosi indietro da sopra la spalla.

«Giurerei che le gallerie si muovono, dottoressa» borbottò Elomas,
piantandosi le mani sui fianchi con tutto il thermos del tè. «Mi sono
già perso.»

Valka si guardò di nuovo indietro. «Non siamo ancora neppure alla
cupola, signore.»

Elomas rise, un suono troppo forte per quel passaggio angusto. «Lo so,
lo so, ma sai cosa intendo.»

«Sì» mormorò lei, guidandoci nella sala a cupola che avevo visto nel
corso della mia prima visita in quel posto. «Quaggiù è facile perdere
l'orientamento.» Uno dei gravimetri era posizionato sul suo treppiede
nel centro del pavimento, con il pendolo che oscillava in modo costante
avanti e indietro e ammiccanti indicatori rossi e verdi.

Proseguimmo solo per un po', seguendo la luce di Valka sempre più
addentro alle gallerie, oltre un paio di tecnici che stavano applicando
alle pareti nuove strisce di adesivo luminoso. L'aria ci gravava
intorno, gelida, e qua e là attraversammo alcune pozzanghere dove il
pavimento si inclinava oppure era crepato e infossato dal tempo. Niente
pareva muoversi, e a parte i nostri rumori, il solo suono era il debole
gocciolare dell'acqua prodotto dalla condensa che cadeva dal soffitto
per tamburellare sul pavimento, mentre la sfera luminosa e la mia
lampada proiettavano goffe ombre tremanti sulle pareti coperte di quegli
anaglifi circolari così simili a quelli della cupola e a quelli dei
tuguri degli Umandh, a Ulakiel. Mi sorpresi a immaginare le cieche
creature dotate di tentacoli che tastavano quei segni lasciati dagli
antichi xenobiti di Valka. Cos'aveva pensato di questo posto quella
strana mente comunitaria? Degli esseri che l'avevano costruito?

«Non riesco ancora a credere che tutto questo sia reale» sussurrai. Ci
eravamo fermati in un'altra camera con diramazioni, questa volta un
ambiente basso e rettangolare, con una foresta di colonne che sporgeva
dal pavimento senza uno schema. Alcune non arrivavano appieno al
soffitto, altre si facevano affusolate prima di toccare il pavimento,
inutili come dita spezzate.

«Sei qui da settimane» sussurrò Valka.

«Non ci si abitua mai davvero» commentò sir Elomas nello stesso momento,
e rabbrividì. «Sono solo contento che la Cappellania abbia deciso di non
frantumare questo posto dall'orbita. Sai, l'Inquisizione ha alcune armi
davvero sgradevoli.»

Passai una mano su una delle colonne, avvertendo il tenue merletto delle
linee che le segnavano, tanto incise quanto in rilievo, e prive di uno
schema. «Se lo avessero fatto avrebbero così dichiarato che c'era
qualcosa da nascondere. Si può dire quello che si vuole dei preti, ma
non sono stupidi.» Valka sbuffò, anche se non saprei dire se per
derisione o in segno di assenso. Mi spostai verso l'estremità più
lontana della sala, tenendo la lampada puntata davanti a me per evitare
il groviglio di colonne nere. «Non credo che a loro importi quello che
pensano quanti di noi sono bloccati su questo miserabile pezzo di
roccia.»

«Vacci piano» mi rimproverò l'affabile vecchio cavaliere. «Io vivo su
questa miserabile roccia!» Si interruppe per un momento e sentii il
rumore del thermos del tè che veniva aperto. «Però hai ragione. Quello
che gli interessa sono i commerci su per il pozzo gravitazionale. Che il
vecchio Elomas e la sua strega straniera scavino pure in giro, a patto
che tengano la testa bassa e non trovino niente.» La sua voce suonava
lontana, soffocata. «E non mi fraintendere... io bado a tenere la testa
bassa. Mi piace che il mio sangue resti dentro di me.»

Un arco circolare si apriva sull'oscurità nella parete che avevo
davanti. Mi girai, parlando da sopra la spalla. «Perché fai questo,
signore?»

«Perché sponsorizzo gli scavi?» chiese Elomas. Riuscivo a stento a
distinguere l'alone dei suoi capelli bianchi nella luce della sfera di
Valka. «Vedi le stesse rovine che vedo io? Credevo fossi uno studioso,
lord Marlowe. Guarda questo posto! Inoltre...» proseguì, allargando le
braccia come per avviluppare la foresta di pietra nera che ci
circondava. «Questo luogo è un mistero, l'unico che valga la pena di
risolvere, almeno su questo mondo. Sai, quando avevo la tua età -- è
stato prima dell'invasione dei Cielcin, bada bene -- viaggiavo su e giù
per il sistema di Perseo, dov'ero nato. Là fuori, lungo la frontiera del
confine galattico. Ho visto tutto quello che potevo, dozzine di mondi,
ma adesso sono vecchio e preferisco quest'avventura nel cortile stesso
di casa mia, grazie. Anche se la dannata galassia sta andando in pezzi.»

Era un genere di cosa a cui non era facile rispondere e nascosi il mio
silenzio vagando un po' più lontano da loro. Ero quasi in linea con la
porta rotonda quando domandai: «Non credi sul serio che la guerra
arriverà qui, vero?»

Il silenzio che scese sulla camera avrebbe potuto soffocare un uomo.
Stavamo ripensando tutti al comunicato radio della notte precedente:
Cielcin nell'eliopausa. Nel sistema. «Questa non è la prima volta che i
Pallidi arrivano nel sistema» osservò però Valka, prendendomi di
sorpresa. «Le vostre Forze Difensive hanno catturato un esploratore
parecchi anni fa, subito dopo il mio arrivo.»

«Non lo sapevo» esclamò Elomas. «Dove lo hai sentito, ragazza?»

Potei quasi percepire la scrollata di spalle di Valka nell'aria nera
come l'inchiostro. «Al palazzo, un paio di stagioni di alta marea fa.
Elomas, vuoi venire a vedere questo?» Il suo tono era sbrigativo, quasi
disinteressato, quindi non mi affrettai a seguire il vecchio e rimasi
fermo come un bambino sperduto in quel posto di pietra aliena. Può
sembrare strano a dirsi -- dopotutto ero in quel posto da settimane e mi
ero soffermato molte volte in quella stessa stanza -- ma ancora non
riuscivo ad assimilare il tutto. Percepivo quegli architetti ignoti come
un peso opprimente che non gravava sulla mia mente ma sui miei geni. La
pressione della mia mortalità potenziata mi gravava addosso come un
giogo quando contemplavo da quanto tempo quelle rovine si trovavano lì,
quasi mille volte l'arco vitale della civiltà umana. Com'erano stati
questi antichi costruttori e dèi? Erano stati più potenti di noi? Un
grande potere che dominava le stelle nella sua focosa giovinezza? Oppure
erano più deboli? A quanto pareva, avevano colonizzato meno mondi degli
uomini e non ne avevano terraformato nessuno. Forse erano soltanto i
primi e non erano affatto grandi.

Una brezza mi alitò sommessa lungo la nuca, arruffando i capelli e
inducendomi a girarmi. La porta spalancata era aperta alle mie spalle e
la galleria al di là di essa era rischiarata soltanto dal bagliore
verdastro delle strisce di nastro sulle pareti. Accigliandomi, lasciai
Valka ed Elomas al loro esame e oltrepassai l'arco per addentrarmi nella
galleria rotonda. Il pavimento si stendeva davanti a me in quella sorta
di tubo e sollevai qualche schizzo nel passare sulla poca acqua di mare
che non si era asciugata. Non ero mai stato da quella parte prima di
allora, quindi avanzai lentamente, proiettando il raggio della mia
lampada in alto e intorno a me, incidendo ombre profonde sugli anaglifi
che là rivestivano le pareti in cerchi di diverse dimensioni, tangenti
uno all'altro come gruppi di bolle di sapone. Formavano una sorta di
cappio estendendosi sopra la mia testa e sotto i miei piedi, per cui mi
trovavo in un passaggio tubolare coperto di cerchi incisi in modo poco
profondo, alcuni incassati e alti convessi, dipingendo la superficie
della pietra come puntinismo.

Ricordando i protocolli di sicurezza che Elomas mi aveva inculcato
durante la nostra permanenza a Fonteprofonda, gridai qualcosa agli altri
per spiegare dov'ero e attesi che rispondessero. Soddisfatto, proseguii,
estraendo dalla tasca del mio caban una piccola sfera luminosa. Tenendo
fra i denti la lampada feci scattare il sigillo e agitai la sfera per
attivarne la fonte di luce e il minuscolo campo repellente Royse, poi la
lanciai con delicatezza lungo il corridoio e la guardai percorrere metri
umidi di galleria prima di rallentare nell'aria e di fermarsi. Badando a
evitare l'acqua gelida sul fondo del tunnel, seguii la parete facendo
scorrere le dita sulla pietra nera e sentendo così le sporgenze e le
rientranze che la Quiete vi aveva intagliato in un tempo distante in
maniera incalcolabile.

Dopo aver camminato lentamente per alcuni minuti arrivai al punto in cui
la sfera aveva rallentato fin quasi a fermarsi e si librava nell'aria.
La afferrai e la lanciai più avanti, stando attento a eventuali insidie
mentre si muoveva. Ripetei quel procedimento per un paio di minuti,
addentrandomi maggiormente nel passaggio e dopo averlo fatto per sei
volte decisi di tornare indietro.

E mi paralizzai.

Una fenditura si apriva nella parete che avevo appena oltrepassato,
larga abbastanza perché un uomo vi si potesse infilare di traverso. Per
un lungo momento rimasi del tutto immobile, certo di non aver notato la
spaccatura quando ero passato inizialmente di lì. L'intero tunnel era
largo poco più di due metri e non potevo non averla vista, lo giuro su
tutta l'arte che ho in me. Uno stivale levò schizzi nell'acqua mentre
barcollavo verso di essa dopo essermi girato per afferrare la sfera
luminosa. Con una luce in ciascuna mano sbirciai nella fenditura e vidi
che non era affatto tale.

Era un passaggio, con le pareti lisce e lucide, senza traccia di glifi
che le segnassero, che riflettevano la luce in increspature di un
candore assoluto sullo sfondo della pietra simile a inchiostro. Quella
non era una frattura da stress, ma una caratteristica intenzionale delle
rovine. Come aveva fatto a sfuggirmi? Puntai all'interno la lampada
manuale, intravedendo una camera e quelli che sembravano gradini. Era
una cosa strana. In tutto il tempo che ero stato a Calagah non avevo
visto altri gradini se non quelli esterni di accesso alle rovine. Mi
insinuai nella fessura e mi guardai intorno. Il misero raggio di luce
della lampada non riusciva a raggiungere il soffitto sovrastante. Avendo
studiato gli ologrammi di Valka sapevo che in quel punto non eravamo a
più di cento piedi nel sottosuolo, ma l'oscurità sovrastante pareva il
nudo Buio dello spazio, sempre aperto, senza stelle e spalancato sopra
di me. Curioso, e avendo bisogno di una migliore illuminazione lanciai
verso l'alto la sfera luminosa pur sapendo che non avrei potuto
recuperarla. Essa fluttuò sempre più su, spargendo la sua luce fra il
bianco e il dorato sulla stanza trapezoidale in cui ero appena entrato.
Ancora più turbato, rimasi a fissare ciò che essa rischiarava.

\emph{C'erano} dei gradini, ma erano solo quelli di una piattaforma --
tre in tutto -- posta di fronte a... al murale sulla parete opposta. Lo
intravidi a stento, perché un momento più tardi il congegno si oscurò e
si fracassò sul pavimento a meno di quattro metri da dove mi trovavo.
Morto. Questo non era possibile, le sfere luminose erano fatte per
risplendere per giorni in quel buio abissale.

«Valka!» chiamai. «Sir Elomas! Avete visto questo?» Mi {interruppi},
imbarazzato, certo che non mi potessero sentire. «Certo che lo hanno
visto, Marlowe» farfugliai, guardandomi indietro da sopra la spalla.
«Lavorano qui.» Stavo parlando con me stesso, il che non era mai un buon
segno. Fra un momento, solo un momento, sarei tornato indietro. La mia
lampada era ancora accesa e la puntai contro l'immagine incisa nella
parete opposta, allargandone il raggio quanto più potevo. Anche così,
però, non riuscii a cogliere la totalità di quel singolo glifo alto
cinquanta piedi.

Era un cerchio, come gli altri, ma ne differiva perché nei suoi confini
non era incisa nessuna suddivisione, nessuna forma o arco geometrico.

Il cerchio era semplice e liscio tranne che nel punto più basso del suo
arco, dove era interrotto da un singolo raggio che si allargava fino a
diventare un cuneo nell'avvicinarsi al pavimento. Mentre avanzavo verso
di esso per un momento mi parve di sentire un rumore di passi e pensai
che Valka ed Elomas fossero venuti a cercarmi, ma quando mi girai non
vidi nessuno. La luce riflessa dalla mia lampada sembrava risplendere
sulla pietra nera come se scaturisse dalle sue profondità e vidi la mia
immagine riflessa, sottile, una forma spettrale. Salii i tre gradini
della piattaforma e protesi una mano per accarezzare quel singolo raggio
intagliato nel muro. La roccia all'interno del cuneo era stata scavata
di due pollici rispetto alla superficie liscia della parete, che lì
risultava venata e ruvida sotto le mie dita.

L'incisione era ancora così nitida che mi parve quasi di poter vedere
quell'antico tagliapietre con il suo cesello, e in quel momento mi colpì
il fatto che l'intera camera era asciutta... non drenata, come lo era
stato il corridoio tubolare ma \emph{asciutta}, come se l'acqua del mare
non vi fosse mai entrata. Esalai un respiro che si trasformò in
caligine, bianco come fumo di un tempio. Come avevamo mai potuto pensare
di essere soli nell'universo? Di esserne i principi? Quale antica
arroganza aveva alimentato quella superstizione e modellato la
Cappellania a sua immagine?

Nello stesso modo in cui quell'incisione mi faceva apparire piccolo al
suo confronto, così le sue implicazioni -- quelle di tutto quello strano
complesso di pietra -- lo facevano con tutti noi. Di nuovo il mio
respiro si congelò nell'aria e sentii un gelo improvviso che mi
penetrava nelle ossa. Decisi di aver indugiato in quel luogo fin troppo
a lungo e stavo per andare a cercare i miei compagni quando vidi, o
credetti di vedere con la coda dell'occhio, qualcosa che si muoveva. Non
c'era niente, solo la mia immagine riflessa. Poi il freddo mi aggredì,
intenso e penetrante come lo era stato in quel primo pomeriggio, sui
gradini esterni, quasi che qualcuno mi avesse trapassato il braccio con
un punteruolo di ghiaccio, crocifiggendomi. Per un momento fui incapace
di pensare, persi perfino l'istinto di ritrarre la mano.

La mia immagine riflessa \emph{si mosse} nella parete. Guardò dritto
verso di me, e i suoi occhi non erano viola come i miei ma di un verde
perfetto e stupefacente. Anche se rimasi immobile, essa protese l'altra
mano verso di me e sentii il freddo che mi investiva come un'onda,
scaturendo dalle mie estremità per arrivare al mio nucleo più profondo.
Il dolore mi divampò dentro, non di un candore incandescente ma azzurro,
e così intenso che dimenticai di gridare, così breve che non ebbi
bisogno di farlo, anche se sapevo che mi avrebbe arrestato il cuore.

Quegli occhi verdi mi fissarono ed ebbi la sensazione di avvertire una
mano che prendeva la mia, premuta contro la parete. Cercai di urlare ma
la mascella rifiutò di funzionare, le ginocchia mi cedettero ma non
caddi. Quegli occhi, quei terribili occhi verdi mi fissavano dalla mia
faccia... ma era la mia faccia? Non riuscivo a vedere niente se non
quegli occhi che riempivano l'universo, \emph{diventavano} l'universo, e
dietro di essi, attraverso di essi, contemplai innumerevoli soli che si
sparsero come braci e si spensero, tutti tranne uno. Precipitai verso di
esso e in una città le cui guglie e torri campanarie ricordavano il
castello della mia patria, ma dove tutti gli edifici erano strani.
Sentii un forte pianto, come di un neonato, mentre sostavo sotto le
volte di una maestosa cappella. In mezzo alle statue c'era una culla, ma
quando mi avvicinai vidi che conteneva soltanto aria. L'immagine si
sgretolò e precipitai all'indietro verso la fitta nebbia. Quando si
dissipò scorsi una grande nave tempestata di statue di uomini, dèi e
demoni, che si estendeva attraverso i cieli e sovrastava le stelle non
fisse.

E vidi i Cielcin schierati in ranghi e file in mezzo al nero dello
spazio mentre marciavano nella notte. Come scintillavano le loro lance!
E il loro canto era come il bagliore di un fulmine crudele. Dove
passavano le stelle cadevano e i pianeti esplodevano come fumo. E ne
contemplai uno più grande degli altri. La sua corona era d'argento, come
lo erano gli intarsi della sua armatura nera, i suoi occhi erano
terribili e i mondi bruciavano al suo passaggio. La grande nave con le
statue eclissò le schiere dei Pallidi e penetrò nella stella più vicina
come il calare di un coltello.

Luce.

Ero accecato, anche se in quella luminosità percepii una presenza, forme
che si muovevano invisibili, senza proiettare nessuna ombra. Cercai di
gridare ma le parole non mi uscivano perché le avevo dimenticate. Non
provavo niente, non sentivo niente. Non sapevo niente.

Se non tre parole.

`Questo deve succedere.'

Caddi all'indietro dalla piattaforma come una torre abbattuta e scivolai
sul pavimento liscio come se mi avessero scagliato via. Il corpo mi
doleva per il ricordo del freddo anche se la sensazione in sé era del
tutto svanita. Gemendo e tremando come una foglia morta mentre il sangue
caldo mi martellava nelle vene mi sollevai a sedere. In preda al terrore
avevo gettato lontano la lampada a mano e strisciai a recuperarla come
una bestia spaventata usata come esca nell'arena prima che l'azhdarch o
il leone l'aggredisse.

«Valka!» gridai, dimenticandomi di Elomas per lo stupore e il dolore.
«Valka!»

\begin{figure}
	\centering
	\def\svgwidth{\columnwidth}
	\scalebox{0.2}{\input{divisore.pdf_tex}}
\end{figure}

Dovevo dirglielo, dirle quello che avevo visto.

Lei non era nella stanza con la foresta di colonne e non c'era neppure
Elomas, e anche se li chiamai entrambi nessuno rispose. Non li trovai
neppure nella stanza dove il gravimetro poggiava sul suo treppiede sotto
la cupola scolpita. Imboccai parecchi passaggi laterali, seguendo il
nastro luminescente e la luce delle sfere fluttuanti finché non mi
sentii certo che avrei già dovuto trovarli, chiamandoli per tutto il
tempo. Decisi allora di tornare in superficie e seguii i passaggi
inclinati fino a riemergere alla luce del giorno. Dopo il tempo
trascorso nel sottosuolo il sole mi abbagliò mentre sostavo sotto la sua
debole luce meridionale, ricordando il freddo.

«Dove diavolo sei stato?» Girandomi, vidi Valka attraversare in fretta
la base sabbiosa della fenditura con i capelli rossi che si agitavano al
vento. Nell'avvicinarsi accelerò il passo per poi fermarsi di colpo a
pochi piedi di distanza. Cercai di parlare, ma lei mi assestò un pugno a
una spalla. «Credevo che fossi caduto in un pozzo e ti fossi rotto il
collo! Elomas è andato a contattare Fonteprofonda per richiedere una
squadra di ricerca.»

La guardai interdetto. «Di cosa stai parlando?»

Valka non parve sentirmi. «Sai cosa ci avrebbe fatto il conte se ti
avessimo lasciato morire là dentro?» Nell'agitazione si passò le dita
fra i capelli.

Indietreggiando cautamente di un passo, sollevai le mani in un gesto
inteso a calmarla. «Valka, di cosa stai parlando? Non posso essere
rimasto assente per più di una ventina di minuti.»

Lei rimase a bocca aperta e io mi ritrovai a fissarla in silenzio quando
ribatté: «Hadrian, sei scomparso per \emph{sei ore}.»

Aprii la bocca e la richiusi, con le parole di diniego che mi morivano
sulle labbra quando notai il cielo. Il sole stava tramontando ed era già
svanito dietro la sommità della fenditura. Mi uscì di bocca qualche
parola frammentaria e scossi il capo. «Non è... c'era una stanza, su un
lato del corridoio rotondo. Io...» Cosa dovevo dire? La mia immagine
riflessa si mosse di nuovo nella mia memoria, spostandosi nella pietra
nera senza di me.

\emph{\phantomsection\label{fileintero-67.xhtml__idTextAnchor004}{}Questo
	deve succedere.}

Non avevo mai visto Valka accigliarsi in quel modo. «Hadrian, non c'è
una stanza sul lato di quella galleria» disse, confusa.

«Cosa?»

«È la strada che porta al complesso orientale e non ci sono
diramazioni.»

Scossi il capo. «No, ce n'è una.» E le descrissi l'enorme glifo al di
sopra della piattaforma e la stretta porta. Valka si incupì in volto,
confermando i miei peggiori timori, e per il momento non le parlai del
resto, della voce, delle mie visioni, il cui ricordo mi faceva ancora
tremare. Una dura linea sottile si formò fra le sue sopracciglia e io
interruppi la mia narrazione per dire: «Dovete essermi passati davanti e
aver proseguito.»

Le sue narici si dilatarono. «Aver proseguito...» La voce le si spense e
mi volse parzialmente le spalle.

Abbassai lo sguardo sui miei piedi, prendendomi un momento per
rivalutare la situazione. Aveva avuto paura per me? Si era preoccupata?
Aveva agito sulla spinta della paura e lo stava ancora facendo? Non
volevo mentirle. Sapevo cosa avevo visto: quella porta misteriosa, la
volta troppo alta della camera. Avevo visto il mio riflesso muoversi,
avevo visto i suoi occhi, verdi come la morte e alieni. Avevo visto...
altre cose.

Non poteva essere reale, dovevo aver sognato. Tornammo in quella
galleria, con l'acqua che ci si raccoglieva intorno alle caviglie e la
percorremmo tutta quattro volte senza trovare niente. «Non ha senso»
protestai, scuotendo il capo. «Era qui, proprio qui.» Premetti le mani
contro la parete di pietra nera, affondando le dita negli anaglifi della
Quiete. Poi, con voce più debole, ripetei: «Era proprio qui.»
Guardandomi intorno la sorpresi a fissarmi con la luce che si rifletteva
nei suoi occhi dorati come in quelli di un gatto. Aveva un'espressione
confusa, come se stesse cercando di accigliarsi e di mordersi l'interno
della guancia nello stesso tempo. «Non sto mentendo.»

«Non credo che tu lo stia facendo, ma...»

«C'era una piattaforma, proprio al di là di questo.» Indicai il muro.
«Era una camera enorme, delle dimensioni del sanctum della Cappellania.
Forse...» Cominciavo a essere disperato. «Forse Elomas ha ragione e le
pareti si muovono.» Lei scosse il capo e accennò a girarsi per
andarsene. «Dico sul serio, Valka! Davvero! C'è questo... murale. È uno
dei glifi, solo che deve essere alto cinquanta piedi. L'ho toccato e...
ricordi quando siamo arrivati qui? Quando ho toccato il gradino che mi
ha congelato la mano?» Si incupì in volto ma non disse niente. «È
successo di nuovo. Ho toccato il glifo e...» Le dissi tutto. Che la
Terra e l'imperatore mi proteggano, le raccontai ogni cosa.

All'inizio non parlò e quella fu la parte peggiore. Non rise né mi colpì
e neppure incrociò le braccia. Se ne rimase lì ferma come una delle
statue della mia visione, impassibile e immobile. Nel tunnel regnava un
silenzio spaventoso rotto soltanto dal gocciolio dell'acqua e dal
fievole suono del nostro respiro. Era una montagna di silenzio, un
oceano che con la sua massa nascondeva tutto quello che avevo visto e
appreso.

Lei dilatò le narici e distolse lo sguardo con le labbra serrate. «Sei
incredibile.»

«Prego?»

«Dopo tutto quello che hai fatto con Gilliam, dopo che ti ho {permesso}
di venire qui a Calagah hai l'\emph{audacia} di mentirmi in faccia. Se
ti sei perso nelle gallerie, dillo e basta. Non devi darti delle arie
per me. Non ne sono colpita.» Si interruppe prima che la sua voce
potesse salire di tono fino a diventare un urlo. «Visioni? Visioni!
Marlowe, voi gente idiota potete definirmi una strega, ma quella è una
\emph{vostra} superstizione. Io sono una scienziata, credo in cose
verificabili e misurabili, non nei fantasmi. Qui abbiamo a che fare con
una civiltà estinta, non con... quale che sia il fottuto gioco che stai
imbastendo.»

Ricacciai indietro un rimprovero e implorai: «Perché dovrei mentire?
Soprattutto dopo Gilliam?»

«Perché \emph{sei} un selvaggio ignorante di un Paese retrogrado che
crede ancora nelle favole» scattò. «Perché sei annoiato. Condurre una
vita rude sul campo con noi gente da poco non è abbastanza per te, mio
signore? Ti mancano i vini pregiati e le uri profumate?»

«Quello \emph{non} sono io» ribattei, quasi ringhiando, e nonostante
tutta la sua veemenza Valka indietreggiò di un passo. Non le feci notare
che non stavamo certo conducendo una vita rude con i servi di sir Elomas
al nostro seguito, e neppure che era stata lei, e non io, a ricorrere ai
servizi dei servitori sessuali del palazzo. Mi sentivo annientato ma non
sarei stato meschino. «Quello non sono io, e tu lo sai. Se non mi credi
va bene così. Mi dispiace di avertelo detto, ma la Terra stessa mi è
testimone che non sto mentendo.»

Occhi viola incontrarono e catturarono quelli dorati, ma solo quelli
dorati sbatterono le palpebre.

«Di' quello che preferisci» ribatté in tono di scherno. E se ne andò
borbottando: «Barbaro.»

