\chapter{Streghe e demoni}

Per quanto incredibile, la mia piccola bravata con Ligeia Vas durante il
banchetto pareva avermi fatto entrare alquanto nelle grazie di lord
Mataro, anche se mi aveva condannato a camminare sulle uova quando non
ero in compagnia di Sua signoria o dei suoi figli. Credo che lord Mataro
fosse uno di quei nobili che mal sopportano il gioco della supervisione
da parte della Cappellania. Quanti lord palatini, quanti pianeti tiene
nel palmo della mano quell'istituzione teocratica, tremanti per la paura
di rappresaglie, dei coltelli dei cathar? Per il timore di invocare
l'ira atomica dell'Inquisizione, la potenza di quelle armi capaci di
bruciare pianeti di cui la Cappellania non permette l'uso neppure
all'imperatore? E tuttavia, come aveva detto Valka, la Cappellania era
composta da uomini, e gli uomini possono essere superati in astuzia,
ingannati, derisi a tavola.

Il vilicus Engin dalla pelle grigia e i capifazione presenti alla cena
si prostrarono in inchini quando il conte, preceduto da tre littori
armati e muniti di scudo, lasciò la sala delle conferenze per
ricongiungersi a quelli di noi del suo seguito lasciati ad aspettare nel
corridoio. La dottoressa Onderra e io ci alzammo, interrompendo la
nostra conversazione con un giovane scoliasta riguardo alle forme di
vita native di Emesh che tutti chiamavano insetti. «In realtà non sono
affatto insetti» aveva precisato. Un quintetto di logoteti in squallide
uniformi marroni si diresse in fretta verso il proprio signore, con i
tablet olografici attivati per preparare registrazioni audio e tenendosi
pronti a prendere appunti con scintillanti penne luminose.

Mi affiancai a Valka, incastrato fra due file di guardie in armatura
verde. Più avanti Ligeia e Gilliam Vas seguivano il conte come un'ombra
e la prima gli stava dicendo qualcosa con un'arida voce monocorde.
Quante volte avevo seguito così mio padre, da ragazzo? In quante
centinaia di occasioni? L'anello con il sigillo, appeso alla sua nuova
catena, era freddo e umido contro il mio petto, nascosto dalla camicia
color crema e da una veste di seta argento alla moda che mi era stata
data per l'occasione. Mentre il seguito del conte passava dal caos
all'essere una fila ordinata, un logoteta mi cedette il passo
mormorando: «Traduttore di Corte, signore.» Non seppi stabilire se
quello era il mio titolo o se quell'uomo era soltanto confuso.
Imbarazzato, mi misi ad armeggiare per sistemare l'ampia fusciacca che
stringeva la veste leggera intorno alla mia vita snella, desiderando che
mi permettessero di tenere una cintura-scudo, un qualche tipo di arma,
ma no... soltanto ai palatini era permesso di girare armati alla
presenza di altri palatini e io dovevo recitare il mio ruolo. La mia
bravata con Ligeia poteva avermi procurato una certa dose di amicizia da
parte del conte, ma mi aveva messo in una posizione precaria. Tor Gibson
diceva sempre che ero melodrammatico. `Un giorno quella tua bocca ti
farà uccidere.' Ecco, di certo mi aveva fatto esiliare, e adesso...

«Sai, credo di essermi sbagliata sul tuo conto, Gibson» affermò Valka,
tenendo fermo l'apparecchio di comunicazione con gli Umandh che le
oscillava pesantemente alla cintura. Proiettavamo un'ombra notevole sul
mosaico del pavimento del corridoio principale della Gilda dei
Pescatori, le cui tessere erano disposte ad arte per formare immagini di
uomini e di Umandh che tiravano fuori dal mare le reti per nutrire una
città affamata. Mi ci volle un momento per ricordare che il nome Gibson
era riferito a me, quasi che una parte del mio io si fosse aspettata di
sentir giungere da un angolo sporco la voce del vecchio scoliasta dagli
occhi vividi ma sbiaditi dal tempo. Non successe, e quando mi resi conto
che ci avevo messo troppo a rispondere decisi di tenere la bocca chiusa
e di aspettare che la dottoressa continuasse, sfruttando il mio silenzio
per indurla a parlare ancora. Dopo un po' lei si schiarì la gola. «Non
sei un barbaro.»

«Non sono \emph{soltanto} un barbaro, vuoi dire.» Passammo sotto l'ombra
di una piastra olografica a parete, il cui pannello spettrale proponeva
le riprese di una mischia nel Colosso risalente a meno di due giorni
prima. Il volume era abbassato, ma le parole del commentatore venivano
riportate in sottotitoli in galstani.

Valka sbuffò e sollevò lo sguardo sullo schermo proprio quando
l'immagine passava a inquadrare la presentatrice, un'avvenente donna
nativa dalla carnagione nera ma con ondulati capelli di un biondo quasi
bianco. Usando il panthai, la sua lingua natale, commentò: «Qui sono
tutti barbari, ma tu sei a posto.» E mi assestò una gomitata mentre ci
soffermavamo a guardare gli ologrammi.

«Dirlo è davvero dolce da parte tua» ribattei in imperiale e in un
sussurro tutt'altro che sommesso. Il mio sorriso rendeva chiaro che la
stavo stuzzicando, e lei distolse lo sguardo con frustrazione.

Fingendosi offesa, inclinò il mento verso l'alto e, sempre in panthai,
aggiunse: «Però sei strano per essere un imperiale, sai.»

«Non siamo tutti uguali» risposi, prima di realizzare quello che stavo
dicendo. «C'è un oceano di differenza fra il conte e me.»

«Siete entrambi uomini palatini.»

«È tutto quello che vedi?» domandai, scoccandole un'occhiata. «La mia
classe? Il mio sesso?»

«È quello che sei.»

Qualcosa mi si contorse dentro nel sentire quelle parole. Non era forse
una variante della stessa argomentazione usata dalla Cappellania per
accantonare i Cielcin definendoli demoni? O da mio padre per accantonare
i suoi plebei? Da me per fare lo stesso con Gilliam? Questi ultimi due
paragoni a quel tempo mi sfuggivano, ma i semi erano stati piantati e
crescevano in modo che un giorno potessi capire. Per il momento sapevo
solo che Valka aveva anche lei i paraocchi, perché non mi vedeva.

«In noi c'è più di questo» obiettai. «Più di dove o come siamo nati.
Siamo... di più» conclusi goffamente.

Questo fece affiorare un cipiglio sul suo volto strano ma adorabile.
«Cosa intendi dire?»

Scrollai le spalle. «Dovresti giudicare le persone dalle loro azioni, da
quello che fanno e non da chi sono.»

Nel sentire questo lei si contorse un poco e si grattò il braccio
tatuato. A quel tempo non fui in grado di apprezzare appieno il
significato di quel tatuaggio, ma il fitto intaglio frattale che la
decorava dalla scapola e dal seno fino alla base delle dita aggraziate
conteneva nei suoi vortici e nelle linee geometriche la storia
codificata della sua linea di discendenza. Era una rappresentazione
culturale e visiva dei geni delle sue stesse ossa. Indossava come una
manica la sua storia e quella del suo clan, scritta in ideogrammi che
non avrei mai potuto comprendere. Tutto questo si muoveva dietro le
linee fini del suo volto in una muta contraddizione alle mie parole.
«Sarebbe carino se applicassi quella generosità a qualcuno che non è uno
dei vostri nobili.»

«Lo faccio! Ci provo.» Lei stava ancora parlando in panthai, quindi
replicai nella stessa lingua, anche se indubbiamente lo feci con
esitante fatica. «Da dove credi che vengano i patrizi? Sono plebei che
sono stati ricompensati per le loro azioni.» Dovetti ricordare a me
stesso che stavo recitando il ruolo di un patrizio.

«Mentre tu sei stato ricompensato dall'essere nato nel posto giusto.»

Serrai la mascella. «Chi potrebbe punirmi per via di chi sono i miei
genitori? Hanno lavorato per quello che avevano, hanno edificato su ciò
che i loro genitori gli avevano dato, proprio come chiunque altro,
palatino o plebeo. Non ho rubato niente.» Mi fermai, timoroso di
rivelare la mia identità palatina.

«Basta così» commentò spensieratamente, seguendo il frusciare delle
vesti verdi dello scoliasta che ci precedeva. «Non avrei mai creduto di
vedere il giorno in cui una priora imperiale sarebbe stata umiliata al
banchetto di un nobile.» Interpretai il cambio di argomento come l'aver
segnato un punto a mio favore, ma non ne gongolai.

Impacciato dalla mia scarsa familiarità con la sua lingua, scossi il
capo. «Da dove provieni è davvero così diverso?»

«Sì.» Oltrepassammo un campo statico e ci ritrovammo nell'aria
soffocante. Lo stridio dei velivoli salutava il nuovo giorno e il loro
frastuono riverberava fra gli edifici di Borosevo. «Il Fuoco Fatuo è
abbastanza lontano da potersi permettere di esserlo.»

Abbassai lo sguardo sugli stivali che sporgevano sotto le vesti di seta.
«Un giorno mi piacerebbe vederlo.»

Lei si fermò per un momento, andando quasi a sbattere contro il logoteta
che la seguiva, e mi guardò in modo strano, come se avessi detto che
volevo distruggere la sua Demarchia e non visitarla; o come se fosse
stato questo che si era spettata di sentirmi dire. Dopo che si fu
spostata incespicando per sgombrare il passo al logoteta dovemmo
affrettarci per ricongiungerci alla fila.

\begin{figure}
	\centering
	\def\svgwidth{\columnwidth}
	\scalebox{0.2}{\input{divisore.pdf_tex}}
\end{figure}

Forse l'icona della Cappellania era reale, forse quegli spiriti sentono
le nostre preghiere, o forse no. Mi sono sempre considerato un
agnostico, ma vedi... per un popolano, un servo che non ha mai visto
l'imperatore, ai suoi occhi lui e quegli dèi sono la stessa cosa. Le
leggi di Sua radiosità hanno sempre effetto sui provinciali, anche
quando non c'è un imperatore, ed è un errore credere che dobbiamo
conoscere una cosa per esserne influenzati, credere che le cose debbano
sempre essere reali. L'universo esiste, e noi siamo in esso, e in virtù
delle strane forze che ci muovono attraverso il tempo -- quali che
siano, che si tratti di Dio o di altro -- la nostra visita alle
pescherie ci portò nello stesso magazzino in cui avevo visto per la
prima volta gli Umandh, una vita prima.

Non era cambiato, come se le pareti di metallo e le passerelle
traballanti fossero state un reperto di un museo e il loro artistico
pessimo stato fosse stato mantenuto con cura. Nell'entrare guardai in
alto, quasi aspettandomi di vedere Cat accoccolata nervosamente sulla
passerella, in alto. Una fitta di dolore mi percorse e mi sorpresi a
formare con discrezione con la mano il segno del sole lungo il fianco:
un gesto dannatamente superstizioso. \emph{Riposa bene, e trova pace
	sulla Terra.} Rischiai di scoppiare improvvisamente a ridere e ricacciai
indietro quell'impulso, immaginando cosa avrebbe detto Cat nel vedermi
vestito in quel modo, con fini sete e stivali alti.

Il mio sgomento dovette trasparirmi dal volto perché sorpresi Valka a
osservarmi, il che mi fece solo sentire peggio. Avevo davvero
dimenticato Cat tanto presto? No. No, di certo la vita doveva andare
avanti e io non ero un asceta, non avrei dovuto essere solo.

«Stai bene, Gibson?»

`Hadrian' avrei voluto dire, come avevo fatto nella riserva di Ulakiel.
`Chiamami Hadrian.'

«Sì, io...» Cosa potevo dire? Che una volta, tanto tempo prima, avevo
rubato in quel posto? «Scusami, stavo solo pensando.»

«I lavoranti vengono tutti dal complesso sulla terra ferma, come Vostra
eccellenza sa» affermò Engin, tutto cordialità e cortese deferenza. La
sua uniforme cachi era stirata di fresco, decorata da medaglie e dalle
mostrine del servizio civile sul colletto. In mano torceva un cappello
dotato di tesa, che avrebbe potuto essere formale se non fosse stato
schiacciato dalle due dita tozze. Guardò nervosamente verso una squadra
della sua gente che aveva radunato un gruppo di ronzanti Umandh lungo un
muro del magazzino che sapeva di muffa.

Gilliam si premette il fazzoletto sulla faccia «Quante di quelle bestie
avete ancora?»

«A Ulakiel?» Engin si accigliò e guardò verso la sua assistente, una
donna ancora più magra di me.

Lei aveva l'aspetto della gente delle tribù del Nord di Emesh, cosa
indicata dalla spessa treccia che le scendeva fino alla base del collo.
Il suo accento vibrante confermò la mia supposizione quando disse:
«Settecentoquarantatré, Vostra reverenza.»

«E in tutto?» Il conte si accigliò mentre si spostava per ispezionare
gli schiavi riuniti. Balian Mataro non era un uomo minuto -- in effetti
era fra i palatini più grossi che avessi mai visto -- ma appariva
piccolo davanti agli alieni, mentre ondeggiavano sulle tre gambe con i
loro tentacoli che si agitavano e le ciglia più sottili che si muovevano
più in alto.

«Circa otto milioni, Vostra eccellenza» replicò la donna del Nord. La
luce si riflesse sulle mostrine del suo colletto, quella di sinistra con
il simbolo della ruota dentellata che indicava il suo grado e la destra
in quella di una mano d'argento aperta su uno sfondo di smalto nero,
sigillo del servizio imperiale. Quindi non era un'assistente, ma la
sovrintendente imperiale degli uffici della Gilda dei Pescatori. Essi
potevano anche essere al servizio del conte, ma tutti i loro registri
andavano dritti all'ufficio imperiale di Forum. Rimuginai su quella
fedeltà divisa, e sulla situazione degli Umandh. Cos'era che Engin aveva
detto quando Valka e io eravamo stati alla loro riserva? Che avevano
venduto una popolazione fuori dal pianeta a scopo riproduttivo?
Immaginai quelle bestie disseminate in tutto l'Impero, ornamenti
indicanti la superiorità umana come gli omuncoli che a volte i ricchi
ordinavano come moglie, modellandone il corpo perché si adeguasse ai
loro desideri. Una cosa vuota, infantile, crudele. Da quel profeta che
ero, immaginai i Cielcin fare la stessa fine. L'uomo è un lupo nei
confronti degli altri uomini e un drago in quelli degli inumani.

«Quel numero è aumentato in modo significativo dal mio ultimo rapporto.»

Ligeia Vas si spostò in silenzio attraverso il magazzino, rimanendo fra
suo figlio e il lord di cui era al servizio. «A quanto mi è dato di
capire, permetti ancora a queste bestie i loro riti.» Non saprei dire
come mai non stesse sudando con indosso la casula di broccato, e
tuttavia appariva del tutto a suo agio mentre studiava il vilicus e la
sua assistente con quei suoi scintillanti occhi da strega.

«Dovrebbero essere portati alla luce della Cappellania» esclamò uno dei
ministri più giovani, un laico che non conoscevo.

Repressi uno sbuffo, riluttante a evidenziare il palese errore logico
inerente alla devozione di quell'uomo, ma per fortuna non dovetti farlo
perché Valka lo trafisse con occhi roventi e parlò senza rivolgersi a
nessuno in particolare nel gruppo del conte. «Perché uno xenobita
dovrebbe mai acconsentire a essere assorbito dalla vostra fede?»

Quel `vostra' \emph{} non passò inosservato al funzionario dal volto
acido, e non sfuggì neppure all'attenzione della priora e del suo
cantore. Ligeia e Gilliam si trattennero per un momento dal replicare
mentre lo stupido ministro farfugliava: «Cosa... cosa vuoi dire,
dottoressa?»

Valka dilatò le narici e parve pronta a colpirlo. Gilliam sogghignò
visibilmente dietro il fazzoletto. «Avresti più fortuna nell'indurre i
ratti a adorare i gatti.»

La grande priora sollevò una mano ossuta e si girò per rivolgersi tanto
al conte quanto al vilicus. «Credevo che il nostro ordine fosse chiaro,
lord Mataro, quando abbiamo concesso al tuo Casato le tecnologie di
terraformazione che avete richiesto.»

Questo era successo oltre mille anni prima, e Balian lo sapeva, ma il
peso di quegli anni gli gravava addosso anche se non li aveva vissuti.
Le spalle gli si accasciarono, come schiacciate da un giogo. «Sì,
certo.»

«La cultura nativa deve essere cancellata. Se necessario togliete i
figli ai genitori, ma non abbiamo bisogno di ribellioni e non possiamo
tollerare altri dèi se non la Terra e suo figlio.» Si riferiva
all'imperatore.

Senza essere notato lanciai un'occhiata in tralice a Valka, che era
ferma con le mani strette dietro la schiena e il mento inclinato verso
l'alto come a invitare un pugile a colpirla. Pensai a quello che aveva
condiviso quel giorno a Ulakiel, a quello che mi aveva mostrato nei suoi
ologrammi. A quel semplice fatto segreto, terrificante e terribile: non
eravamo i primi. Ligeia lo sapeva? E Gilliam? Anche se la Quiete era un
segreto noto solo a pochi nella Cappellania -- quelli incaricati di
proteggere il segreto -- di certo la madre e il suo bastardo dovevano
saperlo. Dopotutto erano i membri della Cappellania più alti di grado su
Emesh. Non c'era da meravigliarsi che quella donna stesse esercitando
tanta pressione, quello che mi meravigliava era perché non avessero
ridotto dall'orbita il sito a vetro fuso, perché nessuno di quegli
antichi siti fosse stato cancellato dalla Cappellania nel corso degli
anni.

«Lascia in pace la dottoressa, Ligeia.» Balian Mataro posò una mano sul
braccio della priora. «Quella ragazza è una straniera e non è abituata
alle nostre usanze.»

«È un'infedele» interloquì Gilliam, guardando Valka con occhi maligni.

«E tu sei un piccolo goblin fanatico» ribatté il conte, forse ancora
iroso e scosso per il commento sulle apparecchiature per la
terraformazione.

Mio malgrado sorrisi, e abbassai lo sguardo sui miei piedi per
nascondere la mia espressione.

«Balian, per favore.» La grande priora si fece avanti. «Una misura di
decoro.»

«Sono il lord di questo pianeta, grande priora. Stai attenta a come ti
rivolgi a me.»

Dalla parete opposta, il canto degli Umandh cambiò tono, gorgheggiando
con uno strano ritmo costante. Dovevano essere una cinquantina, tutti
ondeggianti come polipi di corallo in una forte corrente, e il rumore
che producevano era incredibile, al punto di scuotere i vetri da quattro
soldi delle finestre. «Per favore, qualcuno vuole farli tacere?» Gilliam
schioccò le dita in direzione dei coloni e usò il fazzoletto per
tamponare il sudore che gli si formava sulla fronte.

Al comando del cantore, uno dei douleter colpì con il bastone la
creatura più vicina per farla tacere, contando probabilmente che il
messaggio si estendesse agli altri, connessi com'erano. Essa lanciò un
barrito di dolore, infrangendo la sua porzione della grande {melodia}
che condivideva con i suoi confratelli, e io provai una terribile fitta
di dejà-vu nel ricordare l'ultima volta che mi ero trovato in quel
posto. Invece di girarsi mitemente per aiutare il compagno caduto,
questa volta però gli Umandh protesero i tentacoli quanto più era
possibile, e il ronzio si trasformò in un arido rantolo, come di aria
che fuoriesce da una trachea lacerata.

Accanto a me Valka trattenne il respiro e si affrettò a sganciare il
tablet dalla cintura, fissandolo con aria confusa mentre batteva sullo
schermo con un indice. Non rendendosi conto che quel suono era strano,
il conte si rivolse a Engin. «Fai tacere quei bastardi, vilicus.»

Engin fendette l'aria con una mano e diede un ordine ai douleter, che
presero ad armeggiare con i tablet. «Riportateli subito sotto controllo
e sulla nave.»

Fra gli adoratori che dimorano sulle montagne di Meidua, si dice che
l'orgoglio è il più grande dei peccati. Non sono sempre stato d'accordo
con quella supposizione, o con il mio amico Edouard, che per primo
l'aveva condivisa con me, ma lì lo fui. Il primo degli Umandh si lanciò
ribollendo al di là della linea invisibile del loro schieramento,
staccandosi dal branco come un derviscio jaddiano, ruotando su sé stesso
e contorcendosi sulle tre gambe in una strana carica vorticante mentre
il verso che emetteva saliva di tono fino a diventare un acuto stridio.
Si scagliò contro il nostro gruppo come una bestia frenetica e sorprese
una delle guardie del conte, avvolgendola nei tentacoli e cadendole
addosso. Gli Emeshi avevano fatto affidamento sulle armi e su mille anni
di oppressione per intimorire quelle creature.

Orgoglio.

La diga si infranse e il branco degli Umandh si riversò su di noi con
strida che sembravano metallo lacerato in qualche profondo inferno
marino. Gilliam barcollò all'indietro, poi si girò con sorprendente
velocità per portare via sua madre. Il resto delle guardie serrò i
ranghi per formare un cordone fra quell'orda d'un tratto animata e il
loro conte. Io mi girai a guardare verso Valka proprio mentre le altre
guardie, quelle che erano rimaste doverosamente schierate all'esterno,
irrompevano nel magazzino. L'uomo aggredito dall'Umandh, la cui armatura
portava una striscia bianca a indicare che era un littore, stava
lottando, ma lo xenobita lo {bloccava} con le sue innumerevoli braccia
la cui stretta si andava accentuando fino a immobilizzarlo. Sentii le
ossa spezzarsi sotto l'armatura, o almeno così mi parve. Una delle altre
guardie aprì il fuoco con un'arma al plasma contro la creatura,
aprendole un cuneo fumante al posto di un fianco.

L'Umandh ululò come un elefante che si sgonfia ma continuò a lottare e a
stringere finché la quinta scarica di plasma non lo abbatté.

«Portate fuori Sua eccellenza!» gridò dama Camilla, la cui voce era
amplificata dai microfoni inseriti nella corazza, poi si girò di scatto
verso di me e Valka, esclamando: «Voi due venite con me!»

Valka era curva in avanti, leggermente distaccata nell'armeggiare con il
tablet. Non era in preda al panico ma era a stento sudata nonostante il
caldo soffocante del magazzino. Per poco non inciampai in una cassa di
pesce rovesciata per raggiungere il capannello dei soldati. «Datemi
un'arma!» Non sapevo quello che stavo dicendo, ma quando loro esitarono
ringhiai: «Non intendo uccidere il vostro dannato signore, datemi
qualcosa!» Feci schioccare le dita e protesi il palmo. Qualcosa di
enorme, coperto di squame, mi colpì, e la sua consistenza ruvida lacerò
le mie belle vesti di seta, graffiandomi la pelle sulle costole. Sbattei
contro il terreno, con la testa che risuonava come una campana e
ringhiai, cercando di fare presa con le dita su carne dura come il
corallo, come la pietra, mentre l'odore dolciastro di pesce crudo mi
riempiva le narici, seguito da filamenti sottili che mi privarono
dell'aria, insieme a un'altra appendice che prese a gravarmi sulla gola.

La vista mi si offuscò e, in preda al panico, azzannai tanto i filamenti
quanto la mia stessa lingua. Il sapore metallico del sangue mi riempì la
bocca insieme all'icore sulfureo che scorreva nelle vene degli Umandh,
ma continuai a soffocare perché non riuscivo a rimuovere la cosa che
avevo fra i denti.

Non mi potevo muovere.

La pressione era eccessiva, avevo ogni arto che sembrava sul punto di
spezzarsi e immaginai colonne di vetro andare in schegge sotto quel
peso. Di colpo diventai cieco, mi sentii debole ed ebbi la sensazione
che il mondo stesse scivolando via. Vorrei essere morto là e aver
risparmiato il mio fetore all'universo, un altro mostro strangolato
nella culla, spento prima che potesse essere inflitto all'universo. Il
sangue che mi scorreva sempre più lento nelle vene portò con sé un suono
di piedi in marcia, un precipitare di astronavi e soli che si
consumavano, poi il mondo svanì nell'oscurità, nel vero Buio di cui
cantano i cantori. Facce bianche sbocciarono come fiori in
quell'oscurità, solo per spegnersi e ridursi in polvere. Vidi mio padre
e Crispin, Cat e Valka e mia madre. E Gibson, con il naso tagliato, la
schiena eretta, gli occhi lucenti.

Scosse il capo. `Torna indietro' disse, poi si ritrasse nell'ombra,
lasciando soltanto occhi verdi che si trasformarono in vetro. In luce
stellare. In oscurità e nient'altro.

Nient'altro.

\begin{figure}
	\centering
	\def\svgwidth{\columnwidth}
	\scalebox{0.2}{\input{divisore.pdf_tex}}
\end{figure}

Luce.

C'era luce. Luce e aria mi si riversarono dentro e la sensazione di
vetro che si rompeva che avevo nelle ossa si trasformò in un acuto
indolenzimento. Valka aveva rimosso il tentacolo che avevo morso. «Stai
bene?»

\emph{Salve, dottoressa Onderra, che strano incontrarti in un posto come
	questo}. Il cervello privato dell'ossigeno mi fece ridere a quel
pensiero e due rughe si formarono fra le sue sopracciglia, poi lei
sussultò quando mi sollevai di scatto a sedere. «Sì» risposi, poi
sgranai gli occhi. «Giù!» L'afferrai per le spalle, quasi svenendo di
nuovo nel trascinarla a terra per poi rotolarle sopra mentre un altro
Umandh passava oltre con i tentacoli che sferzavano l'aria. Valka rimase
paralizzata sotto di me, con gli occhi dilatati. Non volevo muovermi ma
mi rialzai in piedi con un gemito, barcollando, mentre la larga fascia
che portavo per tenere chiusa la veste si srotolava e l'indumento mi si
aggrovigliava intorno. Con un ringhio me ne liberai, rimanendo in
camicia e calzoni color crema, con la camicia che mi aderiva al fianco
dove il sangue era fuoriuscito caldo e appiccicoso. Issai Valka in
piedi. «Stai bene?»

Non le sfuggì il fatto che avevo deliberatamente imitato il suo tono e
riuscì a esibire un accenno di sorriso che per poco non mi sfuggì perché
cercò di farlo passare per una compressione delle labbra serrate fino a
far sbiancare la pelle. «Sì.»

«Vieni.» L'afferrai per il polso e mi spostai verso la scala da cui un
tempo ero sceso per rubare il pesce. «Sali lì! Presto!» Dov'era il
conte? Non riuscivo a vederlo in tutta quella confusione, nel
{groviglio} di schiavi umandh e di umani e attraverso la caligine di
fumo di plasma e di princìpi di incendio. Valka indugiava ancora sul
primo piolo della scala. «Sali, dannazione!»

Lei sgranò gli occhi e obbedì. Uno degli Umandh dovette sentire la
nostra fuga perché si scagliò dritto contro di me, nove piedi di pelle
di pietra e di tentacoli oscillanti. Stupidamente, mi tuffai da un lato
e rotolai nel colpire il terreno sul lato opposto di una fila di
contenitori di pesce aperto per poi andare a sbattere contro la
successiva. L'Umandh si abbatté fra le casse, facendole rotolare nella
sua cecità mentre io mi rialzavo e afferravo un paio di carpe congelate.

Intorpidito, confuso, le scagliai contro la creatura barcollante mentre
indietreggiavo e cercavo un'arma di qualche tipo. Magari uno dei
douleter aveva lasciato in giro un bastone storditore o qualche
lavorante dei moli poco giudizioso aveva abbandonato un piede di porco
usato per aprire le casse refrigerate. \emph{O magari c'è una Legione
	imperiale in agguato, sepolta sotto tutto quel pesce}.

Mi guardai intorno in cerca dei soldati, ma erano troppo impegnati a
completare il loro massacro vicino alle porte per aiutarmi con il mio
avversario vagante. Erano così pochi. La maggior parte dei rinforzi che
erano affluiti per aumentare il numero degli uomini presenti dall'inizio
era scomparsa, ritirandosi con gli svaniti palatini. Mi parve di vedere
il corpo grigio di Engin prono e sanguinante sul pavimento di cemento,
ma non persi tempo con lui.

L'Umandh era in piedi e sibilava dai tentacoli che gli scaturivano dalla
bocca in cima al tronco. Da quella prospettiva mi resi conto che i
tentacoli non erano braccia... erano lingue. Indietreggiai e per poco
non inciampai di nuovo nella carcassa di un congrid lungo sette metri.
Qualcosa di metallico cadde rumorosamente contro le file di casse.
Quando vidi di cosa si trattava per poco non scoppiai a ridere: quel
machete era uno dei tanti usati per sventrare le massicce anguille
congrid e gli squali terrestri pescati dalla Gilda, un'arma lasciata
indubbiamente in giro da uno schiavo o da un douleter poco coscienziosi,
proprio come avevo sperato. Avrei potuto baciare chiunque l'avesse
lasciato là. Lo afferrai, rotolai e mi girai a fronteggiare il mio
avversario. La lama fendette due filamenti dell'Umandh, ne intaccò un
terzo e ne raggiunse un quarto. La creatura muggì e mi bloccò il passo
cercando di raggiungermi alle gambe con una delle sue. Eseguii una
torsione e afferrai nel pugno uno dei tentacoli, quindi calai la lama e
allo stesso tempo sbattei uno stivale contro una delle ossute ginocchia
del colono.

Sentii l'osso che si rompeva e il grido di guerra dell'Umandh divenne un
gorgoglio di dolore mentre cadeva. Appoggiai la punta del machete contro
l'esoscheletro ossuto della creatura e sollevai l'altra mano per colpire
il pomo e trapassare la pelle spessa. Esso gemette, emise un suono
simile al richiamo delle balene d'ottone nelle acque del mio pianeta e
giacque inerte. La mia mano rimase là sospesa, sollevata come la spada
di un boia, con la mia ombra simile a quella di un cathar proiettata sul
corpo del colpevole. Mentre esitavo guardai in alto e vidi Valka che mi
osservava dalla passerella, proprio dal punto in cui Cat lo aveva fatto
un tempo e dove avevo guardato i douleter percuotere quell'Umandh senza
nome.

Ritrassi il machete e sollevai la lama in un silenzioso saluto alla
dottoressa.

Poi uno dei soldati del conte sparò alla creatura che giaceva ai miei
piedi.

\begin{figure}
	\centering
	\def\svgwidth{\columnwidth}
	\scalebox{0.2}{\input{divisore.pdf_tex}}
\end{figure}

«È stata la strega straniera!» stava gridando Gilliam, quando finalmente
sbucai alla luce del sole. Cosa irritante, era illeso e riparava con un
braccio la madre. La grande priora, che rivestiva il ruolo della strega
meglio di come avrebbe mai potuto fare Valka, se ne stava annidata
contro il figlio nonostante la calura. «Può usare quel suo terminale per
parlare con quelle... quelle creature!» infuriò. «E solo la Madre sa di
che altro è capace!»

Il conte si asciugò la fronte sudata con una manica ricamata. «Usiamo
quei tablet fin dalla colonizzazione e non hanno niente a che fare con
la nostra emissaria da Tavros.»

«Allora lei l'ha contaminato con qualche perversione straniera. Qualche
congegno di Tavros!»

La dama Camilla venne avanti e interruppe con disinvoltura il flusso di
insinuazioni dell'intus. Salutò e si inchinò al suo signore, dicendo:
«Tutti morti, eccellenza.»

Lord Balian Mataro si accasciò contro la cassa. «Molto bene. Fateli a
pezzi e gettateli in mare, Camilla.» La congedò con un cenno della mano
appesantita dagli anelli e dagli ordini

Lei però non si mosse, e immaginai quegli occhi duri come le gemme
mentre fissavano il conte da dietro l'elmo. «E i nostri morti?»

«Quanti sono?»

«Tre» replicò la cavaliere-littore, come se fosse stato soltanto un
numero. «Engin e due dei nostri.»

Lanciai un'occhiata a Valka e inarcai le sopracciglia come a chiedere:
`Hai avuto qualcosa a che fare con tutto questo?' Lei scosse il capo,
più stanca che indignata, e intanto il conte si rivolse a una dei
douleter superstiti. «Cosa diavolo è successo?»

La douleter, una donna dai capelli fra il giallo e il bianco rimase
rigida sull'attenti nel rispondere, con lo sguardo fisso su un punto al
di là della spalla massiccia del conte. «Non... non lo so, mio signore.
Vostra eccellenza. Non li ho mai visti così. A volte li ho visti
infuriarsi, caricare, certo, ma questo...»

«Erano frenetici» intervenne Ligeia Vas, liberandosi dall'abbraccio
protettivo del figlio. «Infuriati.» Sopra di noi il rumore dei repulsori
Royse riempiva l'aria e parecchi velivoli -- veicoli da sbarco militari
che scortavano uno schifo affusolato e cromato -- entrarono nel nostro
campo visivo, scendendo in cerchio dal cielo punteggiato di nuvole come
samare a primavera. «Quelle \emph{bestie} dovrebbero essere spazzate
via.»

«Vi hanno sentiti dire che lui era il conte» interloquì Valka,
protendendo il mento in avanti con un gesto che la fece apparire più
alta di un cubito. «Il ronzio è cambiato in quel momento. Forse
capiscono la nostra lingua meglio di quanto crediamo, signore.» Alcuni
dei membri superstiti del seguito sussultarono per l'eccessiva
familiarità dei suoi toni ma lei li ignorò.

Gilliam avanzò zoppicando. «Ti rivolgerai al conte in modo corretto o
non lo farai affatto.» Si girò con fare implorante verso l'uomo in
questione. «Ti ripeto che la strega è responsabile dell'accaduto. Chi
meglio di lei conosce quelle creature? Eh? Chi ha maggiori ragioni per
volerti morto? Ti avevo avvertito dall'inizio, eccellenza. È un'agente
straniera.»

Sentii un muscolo che mi pulsava lungo la mascella. «Non puoi provare
che la dottoressa sia responsabile, cantore» ribatté stancamente il
conte, facendosi impassibile in volto mentre i primi velivoli si
adagiavano sul mare in lontananza e si spostavano verso la terraferma
sollevando grandi nuvole di caligine che formò scintillanti arcobaleni
sotto il sole. «Almeno piantala con le tue accuse finché non saremo
tornati al casello di Borosevo.» Sollevò lo sguardo verso il castello,
che si ergeva a una media distanza come una fiaba, sovrastando i bassi
edifici e il puzzo di alghe della città.

«Quale altra spiegazione ci potrebbe essere?» Gilliam avanzò zoppicando
e tamponandosi la fronte sudata con il fazzoletto estratto dalla manica
per poi premerselo contro il petto. «È una straniera, Vostra signoria,
una strega demarchica.» La bocca mi si contrasse di fronte a quella
parola, all'etichetta che la Cappellania usava per chiunque flirtasse
con le odiate macchine pensanti senza il suo divino consenso. Con quanta
facilità scivolava dalle labbra di Gilliam!

«Potrebbe essere una semplice ribellione» sbottai, entrando nella
conversazione con un abbozzo di inchino. «Come ha detto la dottoressa,
sanno a chi appartiene lo stivale che preme loro sul collo... per così
dire.» Spostai lo sguardo dal conte alla priora e chinai il capo.

La mascella distorta di Gilliam si contrasse e quegli occhi spaiati
saettarono dalla mia faccia a quella del suo signore. Qualsiasi altra
cosa fosse, non era uno stupido. Puntò un dito tozzo contro il mio
petto. «Vedi? Tiene questo popolano sotto la sua influenza. Ricorda le
mie parole, c'è la strega dietro a tutto questo!»

Valka dilatò le narici e avanzò di un passo, con il corpo snello
contratto come per colpire. «Chiamami di nuovo `strega', prete, e ti
farò desiderare di esserlo.» Giuro che Gilliam tremò di fronte alla luce
di quegli occhi dorati. Mi piace immaginare che lui abbia mosso un passo
indietro. Lottai per ricacciare indietro un sorriso. Senza essere notato
e momentaneamente dimenticato di fronte alla minaccia di Valka, mi
portai più vicino al cantore. Come osava?

«Sire, mi minaccia!» L'intus tentò di fare del suo meglio per
raddrizzare la schiena storta e non rispose a Valka, rivolgendosi invece
al suo signore. «Questa... donna straniera.» Non disse più `strega'.
«Chi altri ha le conoscenze per comunicare con gli Umandh? L'ho vista!
L'ho vista armeggiare con quel congegno appena prima dell'attacco! E
adesso mi minaccia, sire! Questa prostituta straniera con il suo...»

Non finì mai la frase e finì invece steso a terra. Non sapevo {neppure}
che lo avrei colpito, la cosa successe da sola e mi ersi su di lui, che
giaceva aggrovigliato nelle vesti nere, agitando il pugno ammaccato,
solo nel centro di un abissale silenzio. Non provavo ira, o disprezzo, e
neppure odio. Mi sentivo... pulito. Giustificato. Virtuoso. Mi
massaggiai le nocche, momentaneamente dimentico del fianco ferito, poi
trassi un profondo respiro ed esalai il fiato attraverso il naso. «Hai
detto più che abbastanza.»

Il sangue colava dal naso rotto dell'intus. Uscendo dal suo stato di
intontimento Gilliam usò il fazzoletto per arginarlo mentre mi puntava
contro la mano libera. «Barbaro!» stridette, rimettendosi in piedi con
rapidità sorprendente. «Mi hai colpito!»

«E lo rifarei.» Avanzai, ancora notevolmente calmo. Avevo appena colpito
un prete della Cappellania Terrestre, e a rigor di logica avrei dovuto
fondermi per la paura dei cathar e dei loro coltelli, ma mi limitai a
succhiarmi le labbra, guardando in basso. «Chiedi scusa alla signora.»

«M Gibson!» Valka fece per avanzare, ma una delle guardie superstiti la
trattenne, timorosa che la mia violenza eretica fosse contagiosa. «Cosa
diavolo stai facendo?»

Il prete snudò i denti gialli. «Pagherai per questo, popolano. Guardie!
Guardie, avete visto cos'ha fatto questo barbaro. Prendetelo!» Sotto la
visiera a specchio le guardie si guardarono a vicenda per poi guardare
il loro signore. Balian Mataro, che era così di recente sopravvissuto a
un incontro ravvicinato con la morte, si limitò a fissarmi, esausto e
logorato. Per gli dèi, in quel momento dimostrava ognuno dei suoi anni,
con gli occhi neri opachi e distanti. Gilliam stava ancora gridando.
«Cosa state aspettando? Prendetelo! Storditelo!»

Gli storditori uscirono dalla fondina con la loro luce azzurra
scintillante quando due opliti decisero di obbedire. Valka sbiancò in
volto. «Aspettate» dissi, sollevando le mani. «Quest'uomo è stato
offensivo, Vostra signoria. Lo hai sentito tu stesso.» Le parole chiave
erano nella frase `è stato offensivo', termini che qualsiasi nobile
dell'Impero avrebbe riconosciuto. Le scelsi con precisione, sperando di
catturare lo sportivo che sapevo essere nell'anima del conte. Avevo
poche carte da giocare, ma del resto mi ero scavato la fossa da solo e
stava a me uscirne. «Esigo soddisfazione.» C'era una via di uscita,
pericolosa come la lama a monofilamento di una falce, sottile come un
cordino, ed era una catena che mi pendeva dal collo.

«Una monomachia?» Il conte inarcò un sopracciglio ma c'era un bagliore
nei suoi occhi. «Di certo la dottoressa ne avrebbe più diritto di te.»

Guardai a lungo Valka, cercando di decifrare i sentimenti incisi nelle
linee del volto finemente cesellato. Rabbia? Paura? Scosse il capo e
potei praticamente sentirla borbottare di nuovo la parola `barbaro'. «Ha
detto che ero sotto la sua influenza, pensa che sia uno stupido.» Quello
era uno scudo di carta, e forse ero stupido, avendo permesso che mi
provocasse come aveva fatto. Adesso non avevo altro ricorso, altra via
di uscita se non quella di andare avanti. \emph{Salve, padre}. Ebbene,
si trattava di questo o di morire. Prima che chiunque potesse sfidarmi
aggiunsi: «Quest'uomo mi ha aggredito due volte, mio signore.» Questo
fece inarcare più di un sopracciglio, ma io proseguii, imperterrito.
Delle persone erano morte, molti inumani erano stati massacrati e quel
pedante voleva lanciare accuse. «Prima ha ordinato ai foederati della
sua scorta di attaccarmi nel colosseo e poi mi ha assalito nel tuo
palazzo, dopo l'ultimo banchetto che hai dato.»

«Eccellenza, questo popolano non ha nessun diritto!» protestò Gilliam.

«Allora non lo neghi?» Snudai i denti.

«Non hai diritto a un duello» intervenne Ligeia Vas. «Mio figlio è un
palatino, non hai il diritto di sfidarlo.»

Guardai Balian Mataro dritto in faccia, aspettandomi che scuotesse la
testa. Lo sapeva, sapeva che non avevo altra scelta e che il suo gioco
delicato era finito, come pure il mio. Con dita prive di sensibilità
spezzai la catena che avevo al collo e mi infilai l'anello nel pollice,
sollevandolo. «Ho tutti i diritti» dichiarai, fissando Gilliam con occhi
roventi. Lanciai poi un'occhiata di sfuggita a Valka e lessi la sorpresa
su quei lineamenti delicati ma duri. «Te l'avevo detto, reverenza. Tu
non sai tutto.»


