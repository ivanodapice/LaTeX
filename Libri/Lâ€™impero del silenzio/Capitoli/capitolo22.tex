\chapter{Marlowe da solo}

La prima cosa che notai fu la puzza. Dovunque fossi, il fetore di pesce
marcio e di fogna era sopraffacente. Poi registrai il calore, umido e
opprimente, che mi aderiva addosso come una tela umida. E la luce. C'era
luce, un universo di luce intensa quasi quanto quella del sole di
Gododdin. Forse fu quella luce, scagliata all'indietro nel tempo, ad
accecarmi nella mia infanzia, a farmi tornare indietro. Non ci vedevo.

«È vivo.» Il suono era sbagliato, remoto, come se la voce mi arrivasse
dal fondo di un lungo tubo di gomma o fosse spinta dalla risacca di un
qualche mare agitato dalla luna. «Qualcuno prenda dell'acqua!» Ebbi a
stento il tempo di sentire un rumore di piedi nudi contro la pietra, poi
qualcuno mi tirò su e mi costrinse a bere dell'acqua da una ciotola di
argilla. L'universo bianco sbiadì un poco, tingendosi di grigio e di
rosso fino a farsi un insieme di chiazze indistinte. Tossii, sentii
l'acqua che mi si rovesciava sul petto, poi mi piegai su me stesso con
le spalle che sussultavano mentre espellevo a forza qualcosa di colloso
e acido dalla gola e dai polmoni. Quel qualcuno mi sorresse le spalle,
impedendomi di cadere. «Nel nome della Terra, ragazza, prendi un fottuto
straccio!» gridò la voce. «Sta vomitando di nuovo altra di quella
schifezza.»

Riuscii a stento a respirare, a placare l'improvviso pulsare dei
capillari nel mio cranio. Gemendo, mi lasciai spingere ancora contro le
lenzuola. Ero in un letto. Dèi, se mi sentivo pesante! Mi sembrava di
avere gli arti fatti di pietra. «Dove?» gracchiai, con voce che era poco
più di un sussurro stridulo. «Dove?»

Una mano ruvida mi si posò sugli occhi, tastandomi la fronte. «Sei
salvo. Adesso sei al sicuro. Ti abbiamo raccolto sulla strada.»

«Sulla strada?» Non aveva senso. Poi però mi assalì un pensiero ancora
più pressante. «Non ci vedo.»

«Cecità criogenica» rispose la voce, che era quella di una vecchia.
Sentii un'altra persona entrare nella stanza, seguita da un rumore di
acqua. Qualcuno aveva trovato lo straccio per i pavimenti richiesto da
chi aveva parlato. «I ragazzi ti hanno trovato in un vicolo vicino allo
spazioporto. È una cosa terribile, ma la si vede di continuo, in casi
come il tuo.» Volevo chiederle cosa intendesse parlando di `casi come il
mio', ma mi sentivo la lingua spessa e gonfia e non ci provai neppure.
«Una cosa terribile,» ripeté la voce roca «ma almeno non ti hanno
venduto come carne, eh? Abbandonato è meglio.» Mi urtò una spalla.
«Abbandonato è qualcosa a cui possiamo rimediare.»

Passò un intero minuto prima che ritrovassi la parola, e in quel tempo
cominciai a distinguere una chiazza color ruggine sopra di me e alla mia
destra. Poteva essere la forma di una vecchia. «Teukros?» ansimai,
tossendo nell'aria sopra di me e sentendo gocce di saliva che mi
ricadevano sul petto nudo. «Stavo... andando su Teukros.»

«Teukros?» La voce stanca si fece sottile come la carta e la chiazza
color ruggine si chinò maggiormente, tanto vicina che potei avvertire
l'odore alcalino dello stimolante verrox nel suo alito. «Che siamo
benedetti, no. Questo è Emesh, nel Velo.»

«No.» Mi sentii scuotere la testa, ma mi pareva che succedesse a qualcun
altro. «No, no, no...» serrai gli occhi, costringendoli a funzionare
meglio, come se lo sforzo potesse obbligare i muscoli delicati a
contrarsi e a farsi di nuovo acuti.

La mano della sconosciuta mi si posò di nuovo sulla spalla. «Andrà tutto
bene, ragazzo. Starai bene. Vedrai.» Mi diede altra acqua, tiepida e
oleosa. La bevvi con avidità, versandomene parecchia sul petto, ma non
importava. Sentii delle mani sulla faccia, sul braccio, poi credo di
essermi assopito. Quello che dicono della sospensione criogenica è vero:
non si sogna. Provavo... cosa? Dislocamento? Sconcerto? Sì, e ancora sì,
ma era qualcosa di più di questo. Avvertivo un incredibile senso di
discontinuità, come immaginavo avrebbe potuto sentirsi un neonato se
avesse avuto le facoltà e il linguaggio per esprimere pensieri
complessi. Non avevo nessuna sensazione di quello che c'era stato prima,
come accade a un dormiente al risveglio. Non avevo sensazioni relative a
ieri, e quindi mi sentivo vacuo e vuoto. Distante, come se stessi
cominciando a sognare solo ora.

Quasi a conferma di quella supposizione, quando aprii gli occhi vidi tor
Gibson che mi scrutava, con il volto rugoso che si faceva accigliato ed
era l'unico punto chiaro nel mio mondo indistinto. Le sue labbra si
mossero ma non riuscii a sentirlo e quando infine sbattei le palpebre
scomparve, lasciandomi immerso in un posto dai colori indefiniti.

Se non altro avevo ritrovato la parola. «Dove sono?»

«Sei sordo?» chiese la vecchia, facendo schioccare le dita vicino al mio
orecchio per dimostrare la sua tesi. «Ti ho detto che sei su Emesh,
giusto?»

Grugnendo, aggiunsi: «Nello specifico.»

Ci fu uno scricchiolio di articolazioni legnose. «Nella mia clinica. I
ragazzi ti hanno trovato praticamente morto in un vicolo. Direi che è
stato un orrore se non recuperassimo a martedì alterni persone
abbandonate come te in un canale di scolo. Le navi si liberano di
continuo dei loro passeggeri, li tirano fuori dalle capsule e li
scaricano dove pensano che nessuno andrà a guardare.» Sospirò, con un
rantolo lungo la gola. «E con la guerra in corso arriva ogni sorta di
gente... le strade si fanno ogni giorno sempre più affollate. Navi vuote
vengono trovate fracassate sulle rotte commerciali... Sei fortunato a
essere qui.» Sopra di me, la forma di un ventilatore a soffitto si mise
a fuoco, circondato da una sporca stanza di mattoni rossi. La mia
salvatrice era in piedi accanto a me, una bestia curva con il naso
incurvato e verruche che punteggiavano la faccia rossa. Credo che vide i
miei occhi mettersi di nuovo a fuoco perché sorrise, non senza
gentilezza. «Hai un nome, ragazzo?»

«Hadrian» risposi, più che altro di riflesso.

Lei emise un fischio. «Un nome raffinato, per qualcuno che è stato
trovato nudo in un canale di scolo.» Mi fissò di sbieco, con l'occhio
destro cieco e ricoperto da una qualche escrescenza rossa che spuntava
vicino al naso. «Sei un nobile di qualche tipo?» I capelli bianchi le
ricadevano flosci sulle spalle curve, arrivando fin quasi all'ombelico
anche se aveva cercato invano di legarli sulla nuca. Sembrava il
personaggio della strega nel teatro delle maschere eudoriano, tanto che
quasi mi aspettavo di vederle fra le braccia un gatto nero.

«No» risposi, troppo in fretta. «Non lo sono.» Poi notai una ragazzina
alle spalle della strega. Non poteva avere più di quindici anni, era
flessuosa e bella, ma esile. Così esile. E quelle lentiggini... non
c'erano dubbi, questi erano plebei, forse perfino servi. Dovunque mi
trovassi, dovunque fosse esattamente questo `Emesh', mi ero ritrovato
nei suoi bassifondi. \emph{} «Cosa è successo a Demetri? Alla
\emph{Eurynasir}?»

«Era la tua nave?» La vecchia tirò in avanti una sedia dalle gambe
sottili da un angolo della piccola corsia torrida e sporca, e mi si
sedette accanto. In fondo alla stanza qualcuno gemette, e nel girare la
testa vidi parecchi altri letti come il mio, oltre una dozzina. La
maggior parte era vuota, ma i tre all'estremità in fondo erano occupati
da altrettanti uomini. La vecchia schioccò le dita. «Maris, vai a vedere
se il letto di quei poveri bastardi ha bisogno di essere pulito.» Quando
la ragazzina non si mosse, sembrando incollata sul posto, la donna
schioccò di nuovo le dita. «Dannazione, ragazza! Me la caverò benissimo
con il qui presente Sua signoria, grazie!» Agitò una mano in un gesto di
congedo e la giovane donna si allontanò in tutta fretta. Un momento dopo
che se ne fu andata la vecchia grugnì e ripiegò in grembo le mani dalla
pelle sottile. Non mi piaceva come aveva detto quel `Sua signoria', con
una nota di derisione nella voce. «Questa non sarà una cosa facile da
sentire per te, ragazzo, ma le navi si liberano delle persone, di
continuo. Il capitano riceve un prezzo migliore per la sua nave, cambia
il piano di volo e decide che il tuo magro posteriore non vale il
carburante o il debito temporale.»

Mentre parlava cominciai a scuotere il capo. «No, non questo capitano.»
Non quadrava, non poteva quadrare. Demetri aveva novemila marchi che lo
aspettavano su Teukros, per non parlare della mia carta universale. E
poi c'era mia madre con cui vedersela. Oh, l'Impero era vasto e la
galassia ancora più vasta, ma non si pestavano i piedi alla progenie di
una viceregina imperiale. Doveva esserci una ragione, una qualche
spiegazione. In qualche modo il tutto doveva avere un senso. Cercai la
calma dell'apatia, desiderando vedere le cose come uno scoliasta, ma le
mie finzioni erano solo questo: finzioni. Serrai i pugni sulle lenzuola
e chiusi gli occhi. «Non era ancora stato pagato. Deve essere successo
qualcosa.»

«Se lo dici tu, ragazzo. Se lo dici tu.» Spinse lo sguardo fuori dalla
finestra priva di vetri, verso un mondo che non potevo vedere. Non mi
credeva. «Quando starai bene potrai andare allo spazioporto e dare tu
stesso una bella occhiata in giro, ma scoprirai che il tuo capitano se
n'è andato da tempo. Scommetto che scaricarti è l'ultima cosa che hanno
fatto prima di decollare.»

Questo mi mise a tacere per parecchio tempo e mi agitai a disagio sulle
lenzuola. Qualcosa mi strisciò dolorosamente contro la mano, e
nell'abbassare lo sguardo vidi una benda bianca intorno al pollice, dove
c'era stato l'anello. «Quanto tempo?»

«Prima che ti rimetta? Fino a domani?»

Scossi la testa, che mi fece male. «Per quanto tempo sono... stato
congelato? In che anno siamo?»

«È l'anno 447 del Dominio del Casato Mataro.»

«No.» Cercai di sollevare una mano ma non ci riuscii. «Non va bene. Che
anno standard è?»

L'espressione della donna si fece acida. «Ti sembro un viaggiatore
spaziale? A cosa mi serve la data stellare imperiale?»

Un altro pensiero si fece largo con violenza nel mio universo. «E le mie
cose?»

«Non avevi addosso neppure i pantaloni quando ti hanno trovato. Hai cose
più grandi di cui preoccuparti che non quello che è successo ai tuoi
effetti personali. Troveremo qualcosa per te nel retro: i poveri
bastardi che muoiono si lasciano dietro abbastanza roba da intasare
l'intero fottuto Impero.»

«Ma il mio denaro?» insistetti, sollevandomi a sedere tanto in fretta da
farmi girare la testa. «Ti devo pagare!»

La donna sorrise, mostrando i denti storti il cui smalto aveva le
chiazze verde menta di chi abusava ripetutamente del verrox. «Sei un
nobile, vero? È scritto dappertutto su quella tua bella pelle bianca.»
Nel parlare passò un dito sul mio braccio e io lo ritrassi di scatto.
«Devi avere un conto. Con Roths, o i Mandari o qualcun altro. Non lo
so.»

E se ne andò, continuando a borbottare, issandosi in piedi e
raggiungendo la sua serva nel corridoio.

Un conto. Non so descrivere facilmente la paura che quel {pensiero} mi
generò dentro. Un conto, la mia famiglia. La vecchia aveva detto che
quel pianeta, Emesh, era nel Velo, il che significava che si doveva
trattare del Velo di Marinus, dove il Braccio di Norma cominciava ad
allungarsi intorno al nucleo galattico, allontanandosi dal cuore
dell'Impero che era nel vecchio Sperone di Orione. Era la cresta
dell'ondata di espansione spaziale che aveva portato la nostra potente
civiltà in contatto con i Cielcin. Solo gli dèi sapevano quanto fossi
lontano da casa, quanto fossi sperduto e quanto tempo avessi perso.
Chiusi gli occhi, spremendone fuori alcune lacrime mentre mi assaliva
un'altra terribile realizzazione, una cosa peggiore della mia situazione
o del fatto che ero sperduto e solo su un mondo di cui non avevo mai
sentito parlare, peggiore anche della perdita della carta universale
conquistata con tanta fatica.

Avevo perso la lettera di Gibson.

Quella lettera di presentazione da lui stilata per gli scoliasti di Nov
Senber, senza la quale non sarei mai stato ammesso nell'ateneo. Mi
avrebbero mandato via sulla porta. Cercai di dire a me stesso che uno
scoliasta non avrebbe pianto, ma io non ero uno scoliasta e non lo sarei
mai stato. Calai un pugno contro il materasso una, due volte. Mi colpii
una coscia, mentre suoni inarticolati mi sfuggivano fra i denti serrati,
suoni angosciati e maledetti. Forse era soltanto un sogno. Doveva
esserlo. Doveva essere un sogno, un qualche incubo spaventoso. Forse in
stato di congelamento criogenico si sogna, e lo stavo facendo. Forse fra
un minuto mi sarei svegliato vedendo il sorriso irrefrenabile di
Demetri. Su Teukros.

Non lo feci mai.

Ero consumato dal pensiero della mia famiglia, di ordini inviati fra le
stelle tramite onda \foreignlanguage{italian}{qet}. Potevo immaginare
mio padre che richiedeva ai prefetti di trattenermi finché non fosse
stato possibile recuperarmi, e mi chiesi dove mi avrebbe fatto
consegnare. Su Vesperad? Di nuovo su Delos? O semplicemente mi avrebbe
fatto buttare fuori da un portello stagno? Mi ero sottratto ai miei
doveri, avevo abbandonato il mio ruolo di figlio. Secondo i Grandi Atti
Costitutivi e tutte le leggi dell'Impero io ero una cosa sua a cui dare
ordini. `Hadrian, elencami le Otto Forme dell'Obbedienza.' Non lo avrei
fatto. Da qualche parte in quella città senza nome le campane di un
sacrario della Cappellania cominciarono a rintoccare. Mi chiesi se mi
fossi assopito di nuovo, scivolando verso quello stato simile alla morte
in cui ero rimasto per un numero ignoto di anni fino a quel giorno. Per
un folle istante pensai che fossero le campane del sacrario della
Cappellania di Meidua, credetti di essere di nuovo a casa e che mio
padre potesse arrivare a grandi passi lungo la corsia, in quell'aria che
puzzava di pesce marcio e di muffa stantia.

E compresi. Capii che non potevo permettere a nessuno di esaminare il
mio sangue perché nel momento in cui la mia impronta genomica fosse
stata inserita nel sistema di questo mondo sarei stato segnalato e non
c'era nulla nell'universo che avrei potuto fare per impedire che
quell'informazione arrivasse fino a Meidua. Nell'istante in cui i miei
geni fossero saltati fuori in un qualche censimento o avessi cercato di
prelevare fondi da uno dei miei conti fuori dal sistema, la cosa si
sarebbe saputa al Riposo del Diavolo, e nei mondi dell'Impero
l'estradizione funzionava in modo tale per cui chiunque governava su
questa roccia intrisa di sudore che la vecchia dottoressa aveva chiamato
Emesh non avrebbe avuto altra scelta se non quella di prendermi e di
spedirmi in un viaggio di ritorno su Delos.

Dovevo scomparire.

\begin{figure}
	\centering
	\def\svgwidth{\columnwidth}
	\scalebox{0.2}{\input{divisore.pdf_tex}}
\end{figure}

La notte scese con una velocità stupefacente e ben presto l'indistinta
luce fra il rosso e l'oro che filtrava dalle finestre aperte fu
sostituita dalla tremula luce gialla delle lampade della corsia. Gli
uomini nei letti in fondo gemevano incessantemente, e quello era il solo
suono a parte il rombo delle auto di terra, all'esterno. Dormii un sonno
irrequieto, con tutto il corpo che si sentiva come se qualcuno mi avesse
colpito con un batticarne, e quando mi svegliai fui accolto da
quell'odore orribile e dalla vista della brutta vecchia e della sua
esile assistente che camminavano su e giù per la corsia di quel posto
desolato. Nel corso di uno di quei momenti di veglia mi resi conto di
non aver visto nessun segno di vere apparecchiature mediche: niente
flebo o monitor o scanner. Per fortuna, non vedevo neppure tutori
correttivi. Mi sentivo come se fossi andato alla deriva, fuori dal mondo
che conoscevo e in un qualche universo più squallido, come nelle
fantasticherie delle opere olografiche di mia madre, luoghi dove le
macchine da stampa erano qualcosa di magico e risanare un uomo
significava praticargli un salasso. Quasi mi aspettavo che quelle luci
tremolanti fossero lampade a gas.

«È davvero un lord, signora?» La ragazza si girò per guardarmi da sotto
quei capelli color lino, e la sua voce suonò sommessa e affannata.
Chiusi gli occhi fino a ridurli a mere fessure, fingendo di dormire come
riesce a farlo solo chi è veramente stanco.

Sentii un tintinnio metallico, poi il rumore di quelle mascelle ossute
che trituravano qualcosa, senza dubbio le foglie di verrox della
vecchia. «Sì, Maris, credo proprio di sì.»

«È molto alto» aggiunse la ragazza, con voce ancora più sommessa. «Pensi
che sia un principe?»

La vecchia scosse il capo, con i capelli flosci che le si agitavano
intorno alla faccia. «I principi hanno i capelli rossi come il fuoco, lo
dicono tutti. Scopriremo chi è quando arriverà il denaro, ragazza.
Intanto lascia in pace quel poveretto.»

Qualcosa di oleoso rigirò le dita nel mio ventre e girai la testa da un
lato perché non volevo continuare a guardare le due donne che avevano
contribuito a salvarmi la vita. Forse avrei vomitato ancora se avessi
avuto di nuovo qualcosa in corpo. La zuppa di pesce che mi avevano dato
era stata poco più che un po' di brodo e mi era rimasta nello stomaco.
Io però non potevo restare.

Non le potevo pagare.

Nel silenzio quasi assoluto mi parve di sentire il gocciolare dell'acqua
nel mausoleo dei miei antenati, il passo di marcia dei soldati nella
livrea dei Marlowe. Non potevo tornare indietro. Avevo picchiato mio
fratello quasi a morte ed ero fuggito nella notte. Già solo per questo
mio padre... non mi piace pensare a quello che mio padre avrebbe potuto
farmi, ma si trattava di più di questo. Se avessi temuto così tanto la
furia di lord Alistair non me ne sarei mai andato da casa. No, era per
mia madre che avevo paura. Cosa le sarebbe successo se mio padre avesse
scoperto la parte che aveva avuto? Sperai che mia nonna la proteggesse.

\begin{figure}
	\centering
	\def\svgwidth{\columnwidth}
	\scalebox{0.2}{\input{divisore.pdf_tex}}
\end{figure}

Com'è inevitabile, scese la notte, una notte vera, tanto che perfino le
strade si fecero silenziose e io -- che avevo dormito quasi tutto il
giorno e per un numero ignoto di anni prima di allora -- scivolai nudo
fuori dalle coltri. Mi sentivo i muscoli molli, deboli, pesanti come
piombo, e una volta in piedi mi accasciai contro la testiera di
alluminio. Ero grato di essere solo nella mia nudità, ricordando come
quel mutante, Saltus, mi avesse deriso. Dov'erano andati? Cosa era
successo mentre dormivo, congelato? Cosa era cambiato? La vecchia che
gestiva quella clinica -- non ho mai saputo il suo nome -- giurava che
era tipico delle navi dei liberi mercanti scaricare i passeggeri come
immondizia, ma non riuscivo a credere che fosse successo questo.

Temendo che quella povera ragazza scegliesse quel momento per entrare
nella corsia, tirai fuori dal letto il lenzuolo appiccicoso e me lo
avvolsi intorno come una toga, tenendolo chiuso con le mani e sentendomi
grato per gli spessi calli che avevo ai piedi a causa di anni di
addestramento. Il pollice fasciato mi doleva, la testa mi girava e
barcollai contro una parete dall'intonaco scrostato. Dovevo trovare dei
vestiti perché non potevo uscire in città con indosso solo la mia pelle.
Appoggiandomi a un angolo del pianerottolo a metà di una stretta rampa
di scale mi soffermai a riflettere. La donna aveva detto che mi
avrebbero trovato dei vestiti `nel retro'. Il retro... un magazzino? Un
ripostiglio? Di certo in giro ci doveva essere qualcosa.

E in effetti c'era, dietro un'ammaccata porta di metallo verde del primo
piano, oltre una credenza e un paio di fontanelle. La stanza odorava di
funghi e di marcio, come se fosse finita sott'acqua più di una volta e
non fosse mai stata areata in modo adeguato, o ripulita. Non volevo
sprecare tempo perché temevo che Maris o la vecchia trovassero il mio
letto vuoto e venissero a cercarmi. Scovai una camicia, un maglione
grigio con una stella nera dipinta sul petto, e dopo parecchi tentativi
anche un paio di pantaloni che mi calzavano abbastanza bene intorno alla
vita. Erano larghi, di un marrone chiaro disseminato di macchie e di
tasche spaiate. Non c'era traccia di scarpe, calzini o biancheria
intima, ma quasi quindici anni passati a duellare a piedi nudi o a
correre senza scarpe lungo le mura del Riposo del Diavolo o di Haspida,
avevano reso la pianta dei miei piedi dura come sostanza cornea. Quelle
piante indurite sbatterono contro il pavimento sporco mentre lasciavo la
clinica, muovendomi verso la porta a due battenti. Una lampada a
soffitto tremolava e strappava riflessi alle piastrelle bianche e nere
del pavimento. Un ratto mi tagliò la strada, strappandomi un sussulto.
Lo guardai allontanarsi furtivo nella notte, così simile a me.

Serrai il pugno intorno al pollice fasciato, causandomi una fitta di
dolore che mi saettò su per il braccio, fino ai denti, e serrai la
mascella con un grugnito. Qualsiasi cosa fosse successa, quei bastardi
si erano presi il mio anello con tutti i dati che conteneva, ogni prova
che ero chi dicevo di essere, dei miei titoli e delle mie tenute.
C'erano persone per le quali la sola vista di un anello palatino era
sufficiente per aprire porte e ungere ingranaggi, e quell'anello avrebbe
potuto aiutarmi senza inserirmi nella dannata mappa genetica dell'alta
società, tenendomi fuori dai registri dello Stato e della Cappellania.

La porta stridette quando la spalancai, premendomi contro l'umida aria
notturna che mi investì come un muro, come un'onda. Avevo pensato che la
clinica fosse calda e umida, ma mi ero sbagliato. Respirare quell'aria
pesante era come riempirsi i polmoni di acqua, e sentii i vestiti rubati
che cominciavano ad appiccicarmisi al corpo.

Alle mie spalle qualcosa cadde rumorosamente per terra con un fracasso
di vetro, metallo e legno. Girandomi vidi Maris che mi fissava dal fondo
del corridoio, con i resti del pasto di qualcuno -- il mio? -- riversati
fra i cocci sulle piastrelle a scacchi. Sembrava pronta a urlare. Si
morse un labbro, con le mani che sussultavano davanti a lei, e mi resi
conto che aveva capito che stavo fuggendo, che stavo rubando l'aiuto che
mi avevano dato, lasciandole senza niente. Pensai di nuovo a quel ratto
e mi misi a correre. A quel punto lei cominciò a urlare, ma le sue
parole si persero in un'improvvisa folata di aria notturna.

Corsi per interi isolati, sollevando schizzi oleosi nelle pozzanghere
che si erano formate nell'asfalto deformato, oltrepassando le vetrine
illuminate con il neon dei negozi e sgusciando sotto la sporgenza dei
piani superiori di bassi edifici che apparivano color ruggine nella luce
arancione dei lampioni. La pioggia cadeva morbida e calda sul mio volto
chino e anche se sapevo che era sbagliato continuai a correre con il
respiro affannoso e la testa pulsante a mano a mano che il mio sangue,
per lungo tempo immobile, ritrovava la sua familiarità con le necessità
della vita. Finalmente mi fermai e mi accasciai contro un bidone dei
rifiuti fuori da una panetteria, una figura solitaria in abiti rubati
accoccolata nel buio della notte senza un posto dove andare o dove
nascondersi.

E mi resi conto che quella che avevo sul volto non era pioggia. Erano
lacrime.


