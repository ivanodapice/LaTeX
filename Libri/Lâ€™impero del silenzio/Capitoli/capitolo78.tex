\chapter{Qualità}

Alcuni schiavi umandh stavano caricando le navette quando scesi dalla
macchina di terra sul campo di atterraggio privato al di là della
periferia di Borosevo. Il sole rosso guardava la scena dall'orizzonte
orientale, presiedendo a una giornata già rovente, che sarebbe stata per
me l'ultima su Emesh. Posai per un momento la valigia accanto a me e
tirai fuori gli occhiali rosso scuro da una tasca del cappotto color
carbone che il personale del castello aveva scovato per me. I miei
ridicoli cappotti, così li chiamava Gibson, e non si era sbagliato:
avrei dovuto aspettare a indossarlo perché sentivo che stavo cominciando
a sudare, un effetto reso molto peggiore dai correttivi medici ancora
assicurati al mio braccio.

Il conducente si affrettò a lasciare il suo posto per trascinare il
resto del mio scarso bagaglio fuori dal retro dell'auto, poi mi si
accodò quando presi la valigia e mi avviai in fretta sulla pista verso
il punto in cui era in attesa un gruppo di figure, alcune per unirsi a
me e altre per vedermi partire. Valka mi venne incontro e io mi
affrettai a dire al conducente di affidare il bagaglio agli Umandh e ai
loro douleter. Mentre Valka si avvicinava adocchiai nervosamente quelle
creature, ricordando il modo pesante, roteante, con cui ci avevano
attaccati quel pomeriggio, nel magazzino. Da allora non mi ero più
trovato così vicino a loro, e sentii la schiena che mi si irrigidiva.

«Sei certo che questa sia una buona idea?» domandò Valka, a voce molto
bassa in modo che non si sentisse sul terreno aperto della pista. Sopra
di noi uno stormo di ornithon solcava l'aria con le ali bianche che
percuotevano il cielo.

«No,» risposi, incapace di impedire all'umorismo e al sollievo di
trapelarmi dalla voce «ma sono felice che tu venga con me, dottoressa.»
L'istinto mi gridava di prenderle la mano, ma l'errore commesso con Kyra
aveva ancora il sapore del ferro sulla mia lingua e ricordavo il bacio
di Anaïs, quindi mi limitai a un sorriso.

Valka si fece più vicina, inducendo quel mio istinto a urlare in preda
al panico, ma mantenni la mia posizione pur trattenendo il respiro.
«Ecco, non è che abbia avuto alternative, dopo averti aiutato» ribatté
con un sorriso, in un sussurro che solo io potei sentire, poi si trasse
indietro con il sorriso che persisteva, irriducibile. «Anche se con te
potrei apprendere più di quanto potrei fare qui.»

«I Cielcin avranno le risposte» dichiarai, sicuro di avere ragione. «È
questo che vuoi, giusto? E una galassia pacificata in cui cercarli.»

Continuò a sorridere nell'indietreggiare. «Spero che tu abbia ragione.»
Da sopra la spalla lanciò un'occhiata verso gli altri, che stavano
salendo la rampa della navetta diretta all'\emph{Incrollabile} e al
relitto di nave, quale che fosse, che Raine Smythe aveva selezionato per
noi. «Il maestro di spada è qui. Il tuo amico.»

«Olorin?» allungai il collo per guardare oltre la sua persona. «Perché?»

«Ha detto che voleva salutarti.»

Annuii, rigido. «D'accordo, allora suppongo di dovergli parlare.»

«Io sarò sulla navetta.»

Olorin Milta era fermo su un lato della navetta nera dalla forma simile
a quella di uno scarabeo, avvolta nelle nubi di vapore gelato esalate
dai tubi di rifornimento. Quando mi avvicinai venne avanti con un
braccio infilato nella mezza veste cerimoniale di seta pesante, che si
agitò alle sue spalle quando sollevò una mano. «Lord Marlowe!»

«Sir Olorin!» Mi inchinai. «Che onore inaspettato.»

Con mio stupore, il maestro di spada \emph{ricambiò} l'inchino,
protendendo il braccio destro in un'imitazione del più alto stile di
corte. «Volevo vederti prima che partissi perché ho qualcosa per te.»

E con mio stupore ancora più grande staccò una delle tre spade che
portava sul fianco destro, porgendomela con il pomo in avanti.

Senza riflettere, protesi le dita d'un tratto intorpidite per
accettarla, mentre rimanevo senza parole.

Era un genere di spada che un imperatore avrebbe potuto brandire senza
vergognarsene, nella quale erano infuse tutta l'arte e la maestria
artigianale dei fabbricanti di spade di Jadd, e dal momento che lo scopo
di quel talento artigianale è di rendere bello ciò che non lo sarebbe,
senza di essi nessuna tecnica artigianale avrebbe potuto elevarla tanto.
Nell'arco di millecinquecento anni, non ho mai visto niente di più
bello, neppure i grandi mosaici e affreschi dei palazzi della Presenza
Imperiale sono paragonabili a essa, perché sono cose vistose ed
elaborate, mentre la sua fattura era semplice e pulita. L'impugnatura
era avvolta in cuoio rosso tanto scuro da apparire quasi nero, il pomo e
le finiture erano placcati in argento, con un singolo cappio inserito
nella guardia vicino alla bocca da cui sarebbe scaturita la lama.

«Non posso accettarla... vale più di me.»

Il sorriso di Olorin lasciò il posto a un'espressione seria e solenne.
«Ti sottovaluti, e la devi prendere. Sei il comandante di questa strana
spedizione, giusto?»

Scossi il capo. «Solo di nome. Bassander Lin ha il comando...» Mi stavo
mostrando ingrato. «Grazie, \emph{domi}.»

Lui abbozzò un gesto evasivo con la mano sinistra che pendeva nella
piega della mandayas. «Prova il suo bilanciamento. Per favore, te ne
prego!»

Alla sua richiesta premetti i due pulsanti gemelli e la lama vibrò,
scorrendo verso l'alto come acqua nella luce rossa dell'alba. Non era
quella che Olorin aveva usato nelle tenebre di Calagah ma prese a
risplendere nello stesso modo, con la lama che sembrava un piolo di
cristallo simile a luce lunare.

Il suo peso era una sorta di poesia, la lama lucente tremolava sotto i
raggi del sole fra lampeggianti riflessi argentei. Una volta estesa, il
suo bilanciamento era squisito al punto che l'arma sembrava più una
parte di me che qualcosa privo di peso. E cantava, vibrando mentre i
nuclei esotici dei suoi atomi si spostavano, muovendosi in modo che il
filo della lama corrispondesse sempre alla direzione in cui si muoveva.
Tremolò, vibrando, quando la sollevai fra me e il maestro di spada, e la
sua superficie si increspò come il mare.

«\emph{Inanna umorphi}» dissi, dove la parola \emph{umorphi} implicava
che la spada non era solo bella nell'aspetto ma anche nelle sue funzioni
e capacità, che realizzava bellezza, come un pittore o un poeta. O un
danzatore.

«Infatti!» convenne Olorin. «Sono lieto che ti piaccia!»

Premetti di nuovo i pulsanti e la lama svanì come fumo nella luce
carminia del sole. «Continuo a non poterla accettare.»

«Quando tornerò nella mia terra e mi inginocchierò davanti al mio
principe, sai cosa gli dirò?» Quella domanda pareva non avere nessuna
correlazione con quanto si era detto prima, quindi rimasi lì confuso con
la spada ancora in mano. Potevo vedere i miraggi di calore dove il campo
statico della rampa intrappolava l'aria fredda dell'interno e desiderai
trovarmi al di là di essa, dove Valka aspettava nella navetta con gli
altri che avevo chiesto a Raine di arruolare. Olorin indietreggiò di un
passo prima di rispondere alla sua stessa domanda. «Gli dirò che ho
incontrato un uomo, un nobile dello strano Impero, e gli parlerò della
tua qualità.» Il sorriso parve fare capriole sul suo volto. «E la tua
qualità sarà nota in Jadd e sui campi di battaglia della nostra guerra
prima che tu ti venga a trovare su di essi. Avrai degli amici.»

«La mia... qualità?» ripetei, con voce malferma. A disagio di fronte a
quelle lodi obiettai: «Tu mi hai aiutato... durante la riunione del
consiglio, intendo. Ti sei accertato che questo sarebbe successo.»
Accennai alla navetta. «Non credo che Smythe mi avrebbe dato ascolto se
tu non avessi sottolineato quanto fosse stata inutile la Cappellania.»

Il maeskolos indietreggiò di un altro mezzo passo, con le calzature
morbide che strisciavano sulla ceramica fusa della pista. «A Jadd
diciamo che un uomo deve essere uno spadaccino o un poeta. Non è vero,
naturalmente, perché abbiamo ogni sorta di uomini. E di donne. Però
queste persone...» Abbozzò un gesto vago in direzione della città e del
castello che incombeva su di essa come un dito segnato dagli elementi.
«Loro sono spadaccini, anche se si definiscono poeti e politici. Credo
che la guerra abbia avuto un numero sufficiente di uomini del genere.»
Si protese in avanti per battermi un colpetto sul braccio. «Quindi sto
mandando loro te! Ottieni la pace, amico mio. \emph{Iffero fosim}!»
`Porta la luce.' A quel punto si allontanò, un passo misurato dopo
l'altro, inclinandosi un poco in avanti quando il vento si impadronì
della manica e dell'ampio orlo della sua veste e l'allargò a ventaglio,
facendolo somigliare a un angelo con una sola ala. Alcuni mamelucchi che
non avevo notato emersero dall'ombra di altre navette simili a scarabei,
vestiti con uniformi il cui colore andava dal nero opaco a strisce blu e
arancioni. Per un momento il sole del mattino proiettò la loro ombra sul
campo di atterraggio... sagome che coprivano il sole, nessuna delle
quali era più grande della sua... e Olorin Mita si erse sulla persona,
alto ed eretto come un re.

Quando infine ebbe oltrepassato la porta malconcia nella recinzione di
maglia metallica del campo, infilai l'impugnatura della spada nella
tasca interna del pesante cappotto e, sudando, mi girai per salire la
rampa di accesso alla navetta, traendo un sospiro di sollievo
nell'oltrepassarne il campo di stasi. Un grido turbolento, gioioso e
pulito si levò dagli altri occupanti del velivolo e mi sorpresi a
sorridere nel trovarmi circondato -- immeritatamente -- da amici. Valka
mi rivolse un languido sorriso da un posto d'angolo, affiancato da uno
vuoto riservato a me. Alle sue spalle sedevano un misto di ufficiali
imperiali e jaddiani, tutti sconosciuti tranne Jinan Azhar, la tenente
jaddiana che aveva accompagnato Olorin a Calagah. Lei sola mi sorrise,
con gli occhi neri che brillavano nel volto olivastro e io ricambiai il
sorriso prima di spostare lo sguardo sugli altri, quell'equipaggio
eterogeneo che avevo richiesto di aggiungere alla nostra banda di finti
mercenari.

Dopotutto, avevo promesso loro una nave.

Ghen fu il primo a scattare in piedi e ad assestarmi una pacca sulla
schiena mentre ruggiva qualcosa come `Sua radiosità' e sogghignava come
uno squalo leuca. Lo seguì Siran, poi il vecchio Pallino, che disse
qualcosa sul fatto di essere un soldato fino alla fine. Alcuni degli
altri mirmidoni vennero avanti per stringermi la mano e assestarmi altre
pacche sulle spalle, ciascuno lieto di essere libero dal suo letale
contratto o da una condanna alla prigione. Switch fu l'ultimo ad
avvicinarsi con un piccolo sorriso timido.

«Sono felice che tu sia qui, Will!» esclamai, abbracciandolo.

Mi spinse indietro e mi assestò un pugno sul braccio. «Chiamami Switch,
signoria... o quello che sei.»

«Solo se tu mi chiamerai Had» ribattei, poi feci scorrere lo sguardo
sugli amici che mi avevano conosciuto come tale. «Per tutti voi sarò
sempre e soltanto... Had.» Mi girai, continuando a sorridere, poi sentii
quel sorriso che mi si induriva sul volto e si afflosciava. Bassander
Lin era fermo sotto lo stretto arco di accesso alla cabina di
pilotaggio, con i capelli color fumo ben pettinati e le tempie rasate di
fresco. Era il solo a portare l'uniforme più nera del nero della
Legione, con le mostrine sul colletto costituite dal triplo raggio di
sole e dai tre diamanti argentei che contrassegnavano il suo nuovo grado
di capitano. Ritrassi il braccio dalle spalle di Switch e lo salutai. «I
Cielcin sono pronti per essere trasportati, capitano?» domandai, con il
pugno premuto sul petto.

Bassander annuì con un gesto cortese ma rigidamente formale. «Quasi, ma
qui siamo pronti.» Mentre parlavo con gli altri la rampa era stata
ritirata e il portello sigillato. «Farete meglio ad allacciare tutti le
cinture. Dobbiamo decollare in cinque minuti se vogliamo effettuare il
rendez-vous con l'\emph{Incrollabile}.» Girò con precisione sui tacchi e
tornò dal suo ufficiale pilota, nella cabina, con la porta che gli si
chiudeva alle spalle con un freddo sibilo.

Decisi che l'ufficiale della Legione era un problema da affrontare in un
altro momento e presi posto accanto a Valka. Volevo ridere, piangere,
fare \emph{qualcosa}. «Stai bene?» mi chiese.

Mi girai a guardarla. Mi fissava con uno sguardo pieno di qualcosa di
raro e prezioso: preoccupazione. Deglutii a fatica e annuii perché non
mi fidavo di parlare. La sua mano mi poggiava calda su un ginocchio. Le
sorrisi, poi spostai lo sguardo sul finestrino e sul campo di
atterraggio quando cominciammo a muoverci. «Adesso sì.»

Erano successe così tante cose, così tanti cambiamenti, tutti
definitivi, tutti... conclusivi. Ero partito da Meidua durante una
tempesta, mentre qui salimmo in mezzo al calore e alla limpida luce del
sole, attraversando nubi e aria verso il silenzio al di là della notte.

\begin{figure}
	\centering
	\def\svgwidth{\columnwidth}
	\scalebox{0.2}{\input{divisore.pdf_tex}}
\end{figure}

Ci sono finali, lettore, e questo è uno di essi. Una parte di me giacerà
per sempre su Emesh, nei canali e nel colosseo, nel castello e nella
bastiglia di Borosevo, sul fondo di un corso d'acqua insieme a Cat e
nell'arena del Colosso. Giace con Gilliam e Uvanari, morti entrambi per
mia mano, e con Anaïs, che non ho mai più rivisto. Se quello che ho
fatto ti disturba, lettore, non ti biasimo, e se non vorrai continuare a
leggere lo capirò. Tu godi del lusso della preveggenza, sai dove va a
finire tutto questo.

Io andrò avanti da solo.
