\chapter{Il palazzo d'estate}

Nostra madre non ci venne incontro sul campo di atterraggio e neppure
alle porte del grande palazzo con le sue cupole di vetro. La cosa non fu
una sorpresa, e Crispin e io venimmo accompagnati nei nostri alloggi in
camere adiacenti che si affacciavano sui giardini con i giochi d'acqua.
Con l'estate che si avviava alla fine, perfino nel Sud i gigli acquatici
cominciavano a morire e le chiazze rosse e bianche sul verde dell'acqua
sbiadivano anche se la luce del sole scintillava iridescente sulle carpe
nei laghetti. Il mio blocco per gli schizzi era aperto sul tavolo e
stavo disegnando il Riposo del Diavolo com'era l'ultima volta che lo
avevo visto.

`La memoria tradisce la maggior parte delle persone.' Le parole di
Gibson mi echeggiavano nella mente così nitide che potevo vederlo in
piedi vicino alla finestra ad arco, con la schiena curva e il volto
avvizzito. `Sbiadisce, lasciandoci con impressioni opache e sfumate di
una vita più simile a un sogno che alla storia.' Diceva sempre che la
mia mente era così affilata che mi sarei tagliato, ma nel riguardare
quei ritratti fatti a memoria -- di Gibson o di chiunque altro -- ho
notato una cosa: non ce ne sono due uguali. Un naso curvo in un'immagine
è diritto in un'altra, fra sopracciglia che possono essere a loro volta
spesse o sottili. Si dice che gli scoliasti non dimentichino mai niente,
che tutto quello che sono e che sperimentano venga trattenuto, duro e
immutabile, dietro i loro occhi. Non ho mai imparato a dominare quel
trucco, quindi non posso dire per quanto tempo abbia indugiato davanti a
quella finestra, osservando gli uccelli che volteggiavano in alto mentre
le serve di mia madre facevano il bagno in una di quelle piscine
limpide.

La mia memoria è rispetto al mondo ciò che un disegno è rispetto alla
fotografia: imperfetta. Più perfetta. Ricordiamo quello che dobbiamo,
quello che scegliamo di ricordare perché è più bello e reale della
verità. Posso sentire Gibson che sbuffa con disprezzo, dicendo: `Il
melodramma è la più infima forma di arte.' Non ho mai trovato
un'obiezione a quelle parole.

Bussarono alla porta, congedando i miei ricordi raffazzonati.

Entrò Crispin. Capii che era lui senza neppure voltarmi o guardare il
suo riflesso nella finestra perché nessuno aveva un passo tanto pesante
o noncurante come il suo. Entrò come un intero plotone, facendo rumore e
clamore: «La mamma non è qui.»

Questo attirò la mia attenzione. «Cosa?» Volsi le spalle alla finestra e
tentai di assestare la camicia calzata male che avevo indosso. «Dov'è?»

Mio fratello scrollò le spalle e senza chiedere il permesso si lasciò
cadere su una poltrona verde, con una gamba drappeggiata su un
bracciolo. In una mano stringeva un coltello sfoderato e agitò quella
lama lattea per la camera. «Hai una stanza più grande della mia.»

«Lei dov'è, Crispin?»

«A Euclide, a quanto pare.» Scrollò di nuovo le spalle e sprofondò
maggiormente nella poltrona con uno scricchiolio di pelle contro pelle.
«Non ho idea del motivo. Me lo hanno detto le ragazze.» Sapevo che con
quel `le ragazze' si riferiva alle donne dell'harem della viceregina.
Dalle ultime notizie che avevo avuto, nel palazzo d'estate c'erano
trentasette concubini e concubine, fra uomini, donne e uno che era
entrambe le cose, se si doveva credere alle voci. La maggior parte dei
nobili mantiene persone del genere come simbolo della sua ricchezza.
«Sai che adesso ha un'omuncola? Con la pelle azzurra. Non crederesti mai
al genere di fianchi che ha.» Fece un gesto osceno e io distolsi lo
sguardo, pensando a Kyra, alla mia vergogna quasi dimenticata di fronte
al buco che la fustigazione di Gibson aveva lasciato dentro di me.

«Davvero non hai idea del perché lei sia a Euclide?» Era a miglia di
distanza verso sud. Era tipico di nostra madre non essere presente in un
momento come questo.

Crispin sbatté le palpebre. «La ragazza azzurra? Lei non è là, è...»

«Nostra madre, idiota» scattai, adocchiando il suo coltello. Avrei
voluto che sir Felix fosse presente: lui avrebbe strigliato Crispin per
il modo in cui brandiva un'arma in quel modo. Poi ricordai Felix mentre
frustava Gibson e cambiai immediatamente idea. «Non hai ricavato niente
dalle ragazze?»

Un po' confuso, lui si accigliò e abbassò lo sguardo sulla mano con il
coltello mentre replicava: «Questo è un modo rovesciato di...» Lo fissai
con occhi roventi e lo vidi farfugliare fino a tacere. Dal giardino
giunse un rumore di schizzi seguito da limpide risate.

Sorrisi. «Qui le cose sono diverse. Meno...»

«Noiose?»

«Fredde.» Mi diressi verso dove avevo lasciato il diario aperto e presi
la matita. Nelle tre ore di volo da Meidua l'avevo ridotta a un
mozzicone. «Posso prendere in prestito il tuo coltello?» Protesi la
mano. Per un momento Crispin si mostrò pensoso, nervoso, quindi feci
schioccare le dita. «Non intendo usarlo per trafiggerti, nel nome della
Terra!» Dopo un altro momento di esitazione mi consegnò la lama di
ceramica e io sedetti al tavolo, procedendo a fare la punta alla matita
e lasciando cadere i trucioli sul piano del tavolo. «A volte penso che
forse è solo casa nostra che è ebbra di politica. Quando sono qui...»
Accennai intorno. «È difficile immaginare che sia in corso una guerra.»

«Però c'è» replicò Crispin, assestandosi con una mano mentre scivolava
ulteriormente nella poltrona. Poi: «Non fabbricano un attrezzo di
qualche tipo per appuntire quegli arnesi?» Indicò la mia matita con un
cenno altezzoso della mano dalle dita tozze.

«Non funziona altrettanto bene. Voglio una punta migliore» replicai con
noncuranza, senza guardarlo. «Ho una serie di bisturi nella mia cassa,
ma...» Mi interruppi, soffiando via la polvere dall'estremità della
matita e pulendo la lama sui miei pantaloni. Posai il coltello sul
tavolo. «C'era qualche informazione su quando nostra madre sarebbe
tornata?»

Crispin si accigliò, lanciando un'occhiata al coltello che aveva
perduto. «Credo domani. Non lo hanno detto.»

Grugnii per indicare che avevo capito e riposi la matita nella costa
interna del diario aperto. «È per questo che sei venuto qui? Per farmi
sapere della mamma?»

«Ecco, sì» rispose, sorridendo. «Però pensavo... dato che stai per
partire...» Si schiarì la gola. «Perché non vieni giù nell'harem con me?
Devi proprio vedere quella ragazza azzurra.»

«No.»

Imperterrito, Crispin continuò: «Hanno anche ragazzi, se è questo che
preferisci. Lo è?»

Quindici anni, e non me lo aveva mai chiesto. Quanto è incoerente una
famiglia palatina. È una cabala di estranei legati da qualcosa di meno
denso dei legami delle famiglie umili. Non è il sangue, in realtà...
solo una costellazione genetica. Mio padre e mia madre non erano tanto
genitori quanto donatori genetici, mio fratello e io eravamo compagni di
classe, conoscenti che per caso condividevano lo stesso codice genetico
di base. Nella mia vita ho osservato molte famiglie palatine, ed è quasi
sempre così. Credo sia un altro modo in cui i nostri plebei sono più
umani di noi. Mio fratello non mi conosceva.

«No, grazie.» Adesso mi rendo conto che Crispin stava cercando di
essermi amico, di dare a entrambi un'esperienza condivisa prima che me
ne andassi per sempre. Avrei potuto proporre un'alternativa, una caccia
sulle colline al di sopra dei laghi, una gara fra velivoli. Avremmo
potuto perfino fare un dannato gioco di simulazione. Invece lo guardai
in tralice e gli restituii il coltello con l'impugnatura in avanti.

\begin{figure}
	\centering
	\def\svgwidth{\columnwidth}
	\scalebox{0.2}{\input{divisore.pdf_tex}}
\end{figure}

Nostra madre non tornò l'indomani, e neppure il giorno successivo. A
ogni ora che passava il cuore mi sprofondava sempre più nella terra. Un
semplice fatto mi allarmava più di tutto il resto: non aveva chiamato.
Avrebbe potuto farlo, anche se fosse stata fuori dal pianeta. Avrebbe
potuto usare il telegrafo quantico o perfino contattarmi tramite la rete
di comunicazione della sfera dati, e se le informazioni di Crispin erano
esatte lei non aveva neppure lasciato il pianeta. Suppongo che avrei
potuto essere io a chiamarla, ma ero io quello che stava per partire e
volevo avere per lei abbastanza importanza perché mi cercasse. Forse era
meschino da parte mia, ma avevo sperato che almeno a un genitore
importasse abbastanza da provarci.

A questo punto avevo difficoltà a sopportare qualsiasi cosa, quindi mi
ritrovai a dormire il più possibile. Invecchiando, scopro di aver meno
bisogno di dormire, ma a quei tempi per me il sonno era una benedizione,
un modo per sfuggire all'indecenza mondana del tempo. Quando dormivo
potevo dimenticare il nervosismo e il cieco terrore della mia
situazione, scordarmi di cosa ero, non un uomo, non il figlio palatino
di un arconte, ma una pedina. Quali che fossero i miei sentimenti nei
confronti di quel concetto letterario, esso era applicabile. Su Delos, e
in particolare a Meidua, ero soltanto un'estensione di mio padre, un
pezzo di sua proprietà che era libero di muovere. I plebei lo capiscono
di rado. Vedono la ricchezza e pensano che sia potere, ma loro sono
indegni dell'attenzione dei potenti e quindi sono spesso liberi di
prendere le loro decisioni, per quanto possano essere limitati. Come
figlio di mio padre, io non lo ero. Senza la lettera di Gibson ero
intrappolato e sarei stato mandato su Vesperad quali che fossero le mie
obiezioni al riguardo.

Nessuno sotto osservazione come lo ero io poteva essere libero. Come le
particelle di luce, chi non viene osservato è libero di diventare
qualsiasi cosa ci sia in lui e che il mondo gli consenta, ma sotto
l'occhio e gli auspici dello Stato può essere solo ciò che i suoi
superiori esigono da lui. Pedoni, cavalieri, alfieri. Perfino i re
possono muoversi di un solo passo alla volta.

Avevo mezza settimana prima della partenza dal ritiro collinare di
Haspida per lo spazioporto di Euclide. Mezza settimana prima di
abbandonare la forza di gravità del mio pianeta per quella aliena di un
altro mondo e l'oscuro sacramento della fede della Cappellania.
Immaginai le celle di prigione della bastiglia del college di Lorica, su
Vesperad, piene di prigionieri urlanti e tormentati, scuoiati e
infranti, persone che un tempo erano state grandi nobili dell'Impero o
re barbari. Immaginai un qualche supervisore del seminario dalla testa
rasata che mi metteva in mano un coltello e mi ordinava di tagliare la
narice di un uomo, di strappare la pelle dalla carne e la carne dalle
ossa in base agli ordini di un qualche priore o magistrato della fede.

Tutto questo e altro ancora si fece largo nel mio cervello come una
sonda medica o gli aghi sottili come fili dei tutori correttivi le cui
cicatrici mi spiccavano ancora sulla mano e sulle costole. Tutto questo
mi tormentò mentre facevo la mia abituale corsa del mattino sopra e
intorno alle fortificazioni ornamentali che formavano il perimetro del
palazzo vero e proprio. Haspida era piccolo, in rapporto alle dimensioni
dei palazzi della nobiltà, copriva appena sette ettari, includendo i
giardini interni e le cupole centrali delle serre dei giardini
d'inverno. Quelle dimensioni erano meno di un terzo dell'area del Riposo
del Diavolo, e non arrivavano a un decimo del palazzo ducale di Artemia.
Un luogo rustico, se un palazzo di duemilaquattrocento stanze poteva
essere definito tale.

I polmoni mi dolevano mentre scendevo a balzi una scala e sbucavo su una
distesa di terra piatta pavimentata con pietra bianca sbriciolata.
Sudavo abbondantemente, con gli indumenti da corsa sintetici che mi
aderivano alla schiena e i capelli incollati qua e là ai piani
scheletrici della mia faccia.

Di conseguenza mi sentii sporco e assolutamente mal vestito quando un
servo nella livrea bianca e azzurra della viceregina arrivò in tutta
fretta da un'apertura fra le siepi e mi fermò nell'ombra di un albero
ricurvo. «Padrone Hadrian, ti stavamo cercando dappertutto.»

I sassi schizzarono sotto i miei talloni quando mi fermai con una
scivolata. «Cosa c'è, messere?»

L'uomo si inchinò affrettatamente. «La tua signora madre, sire. È
tornata e richiede la tua presenza nel suo studio.»

Lanciai un'occhiata al cielo, poi abbassai lo sguardo sul cronometro che
avevo al polso. «Ho il tempo di lavarmi, messere?»

«Ha detto che è della massima urgenza, signore.» L'uomo chinò la testa
dalla mascella pronunciata e sfregò entrambe le mani contro una leggera
pancia che sporgeva sotto la sua uniforme spiegazzata.

L'appartamento di nostra madre era in una villa staccata dalla casa
principale, edificata molto tempo dopo l'avvio della costruzione del
palazzo d'estate. Aveva solo un secolo circa di vita, ed era modellata
sulle sue esigenze come compositrice di libretti e di olo-opere. Seguii
il corpulento servitore attraverso un profondo tunnel in una collinetta,
oltre una porzione dell'arboreto, con le foglie fra il blu e il nero e
la lussureggiante erba verde che splendeva nel pieno della luce diurna,

La villa in sé stessa evocava costruzioni più antiche perché era in
quello stile che i patiti definiscono prevagabondaggio, tutto linee
dritte e pulite e angoli retti. Appariva del tutto diversa dalle volute
rococò e dalle vacue decorazioni del palazzo d'inverno vero e proprio,
molto lontana nel colore e nella sostanza dal peso gotico del Riposo del
Diavolo. Un gioco d'acqua si riversava libero lungo una parete per
rovesciarsi in un'altra delle onnipresenti pozze di carpe. Un quartetto
di legionari che sfoggiava l'avorio imperiale e l'emblema del Casato
Kephalos sul braccio sinistro e il sole imperiale sul destro mi
salutarono quando entrai nel loro campo visivo, e di lì a poco ci
ritrovammo all'interno.


