\chapter{Misericordia è...}

La notte era scesa sul castello e le luci si accendevano sui canali di
Borosevo come tremolante gas di palude. C'erano anche roghi di cadaveri
che ardevano nelle piazze e sulle strade, a ricordare la pestilenza che
da tanto tempo era uscita dalle mie esperienze quotidiane e che
devastava ancora il popolo di salute cagionevole del mondo sottostante.
Così in alto non era possibile sentire l'odore del fumo e tantomeno il
puzzo della malattia unito all'odore del pesce e delle alghe che
marcivano, che costituivano i profumi della città.

Un paio di guardie mi oltrepassarono sulle scale che portavano alla
torre d'angolo e ai giardini a terrazza che coprivano la facciata
meridionale dello ziggurat su cui sorgeva il castello. In uno strano
modo, quel percorso mi ricordava la strada tortuosa che dal Riposo del
Diavolo portava al molo abbandonato dove mi rifugiavo da bambino. Non
notavo più l'umidità pesante e soffocante dell'aria o il peso opaco
della più intensa forza di gravità di Emesh. La brezza marina mi
sferzava selvaggia e pulita, raccogliendo i miei capelli fra le sue
dita. In alto potevo distinguere il punto fra il bianco e l'azzurro che
era l'\emph{Incrollabile}, ferma in orbita stazionaria al di sopra della
città e ospite di una costellazione di scintillanti velivoli da
riparazione più fiochi delle stelle. I detriti causati dallo scontro con
i Cielcin precipitavano ancora occasionalmente, cicatrici carminie sulla
cortina blu della notte. Osservai uno di essi cadere e trasformarsi in
cenere per il calore dell'ingresso nell'atmosfera.

Il vento agitava le palme terrestri piantate come sentinelle nel
terreno, e da qualche parte c'era un ornithon che sibilava. Quel
giardino in alto sul mare era un posto splendido, come lo erano stati il
vecchio molo, e Calagah, e la diga marina del Riposo del Diavolo. Credo
che in un'altra vita avrei potuto essere un marinaio -- forse lo eravamo
tutti -- perché in questi climi pelagici ho sempre trovato una pace
fuggevole ma immediata. Ignorando i cartelli di avvertimento mi
arrampicai sul parapetto e girai la faccia verso il vento che mi agitava
l'ampia camicia.

Ed ero solo.

C'erano sempre le videocamere, ma almeno ero libero dai servitori e dai
cortigiani, con i loro sussurri incessanti. Sentendomi instabile, mi
sedetti sul muro di pietra dondolando i piedi al di sopra della terrazza
successiva, cinquanta piedi più in basso. Una parte di me si sentiva
come un bambino e dovevo sembrare tale paragonato a quel possente
edificio di pietra. Il castello incombeva alle mie spalle e sopra di me,
sospeso come la spada di Damocle sopra la mia testa. Mille piedi più in
basso era accoccolata la bastiglia, una brutta costruzione di cemento
adiacente alla cupola di rame e alle snelle torri della Cappellania.

Non piansi, anche se avevo motivo di farlo.

Il solo suono era il frusciare del vento fra i cedri, infranto qua e là
dal verso di un uccello notturno o dal sibilo degli ornithon. Da qualche
parte una rana prese a gracchiare e una voce d'uomo giunse da ancora più
lontano sulle ali del vento.

Io non sentii niente di tutto questo, potevo udire solo le urla e gli
ansiti animaleschi emessi da Uvanari nella sua sofferenza. Quali che
fossero le diversità fra le nostre specie, il dolore non rientrava fra
di esse. Serrai la mascella fino a farla dolere, sentendo nella mente
quella supplica disperata: `Mi ucciderai, Hadrian?' Non ero certo di
poterlo fare.

I Cielcin combattevano per loro stessi, per il loro diritto di esistere,
e noi non eravamo diversi. Fintanto che la loro esistenza avesse
minacciato le nostre colonie, finché i nostri soldati avessero distrutto
le loro flotte di navi-mondo, non ci sarebbe stata pace. Finché si fosse
risposto con le atrocità alle atrocità, con l'omicidio all'omicidio, con
il fuoco al fuoco, non importava affatto chi brandisse la spada più
insanguinata. La Cappellania avrebbe torturato a morte Uvanari, poi
avrebbero cominciato con Tanaran o con uno degli altri e ottenuto...
niente. Tutti i tormenti dell'universo non {avrebbero} potuto fornire ai
preti quelle coordinate che gli xenobiti non conoscevano, e non sarebbe
cambiato nulla.

Dove avevo sbagliato? Le mie buone intenzioni si erano dissolte tutte,
lasciando solo questo labirinto. Ogni alternativa a mia disposizione
avrebbe generato sofferenza. Se avessi ucciso Uvanari -- sempre che
avessi potuto -- Tanaran sarebbe stato il prossimo a finire sulla croce.
O sarei stato io. Se non avessi fatto niente il capitano avrebbe
sofferto, e io con lui, anche se solo nell'anima. Eravamo in guerra e
tempi duri richiedevano scelte difficili. Immagini dello \emph{spiritus
	mundi} mi si accalcarono nella mente, estratte da quell'orribile incubo
che avevo visto a Calagah. I Cielcin che marciavano attraverso le
stelle, un grande esercito splendido e terribile, con i capelli bianchi,
che riluceva al sole. Li vidi consumati dal fuoco di quella stella
morente e sentii urla ancora più forti. Mi tremarono le mani mentre le
urla si trasformavano di nuovo nei vagiti di un neonato che non avevo
visto. Poi ci furono solo le tre parole pronunciate per me dalla voce
della Quiete: `Questo deve succedere.'

Nonostante la calura del giardino rabbrividii, e dopo quella che mi
parve mezza eternità richiamai uno schermo olografico sul terminale da
polso e digitai un messaggio.

Attesi.

«Sai, credo che Gilliam avesse ragione» commentai, sentendo passi che si
avvicinavano. «Sei davvero una strega, per avvicinarti così di
soppiatto.»

I passi si fermarono e la voce vivace di Valka risuonò nella penombra,
alquanto appiattita dal vento. «Come hai fatto a capire che ero io?»

«Non l'ho fatto,» ammisi, serio «ma sono lieto che sia tu. Sarei
sembrato ridicolo se si fosse trattato di chiunque altro.» Mi girai a
guardarla da sopra la spalla e battei un colpetto sulla ringhiera,
accanto a me. Sia che avesse più paura di me dell'altezza o che fosse
meno sciocca, rifiutò di sedersi. «Speravo che potessimo parlare. Sai,
tranquillamente.»

Si guardò intorno nel giardino a terrazza, e nel guardare le palme e i
fiori vivaci i cui colori erano attenuati dalle scarse luci incassate
nel muro di pietra alle nostre spalle si lisciò i capelli che aveva
accorciato. Alle sue spalle le lampade di sicurezza che {rischiaravano}
il camminamento sulle mura calarono di intensità, si spensero
silenziosamente, poi si riaccesero. Una di esse continuò a lampeggiare
come in preda a una paralisi. «Allora, ho messo in loop le tre
videocamere che ci sono qui fuori, ma non dovremmo parlare a lungo.»

Mi rosicchiai quello che restava dell'unghia del pollice. «Questa cosa
che riesci a fare è proprio quello di cui ti volevo parlare.»

«In che senso?»

«Mi serve il tuo aiuto. Uvanari, il capitano dei Cielcin...» Scossi il
capo. Non c'era un bel modo per dirlo, come non ce n'era uno per non
dirlo. «Mi ha chiesto di ucciderlo e credo che lo dovrò fare.» Non
guardavo verso di lei ma verso la sottostante mole della bastiglia. Lei
non parlò e avrei potuto pensare che se ne fosse andata se non fosse
stato per la rovente sensazione di essere osservato. «Lo stanno
torturando, Valka, nonostante tutte le loro promesse e quelle che ho
fatto io. Ho una responsabilità. Non posso salvare il capitano, ed è
colpa mia se è finito là dentro, quindi...» Le esposi il mio piano,
tutto quanto, senza nascondere niente, neppure il minimo dettaglio.
Parlai in fretta, consapevole del poco tempo che avevamo. «Potrei
riuscire a ricavarne qualcosa che impedisca alla Cappellania di fare del
male agli altri. Voglio restituirli al loro popolo, usarli per aprire un
dialogo con i loro capi, per porre fine ai combattimenti.» Deglutii a
fatica. «Per porre fine alla guerra.»

Solo allora mi girai e potei tollerare di vedere il giudizio in quegli
occhi dorati.

Non ce n'era traccia.

«Una volta hai detto che quello che era successo a Gilliam era colpa mia
e avevi ragione. Ma questa volta se non faccio niente sarà di nuovo
colpa mia, e non lo posso fare da solo.»

Il modo in cui compresse le labbra e in cui le sopracciglia si
corrugarono... non riuscii a decifrarlo. Si morse un labbro. «Benissimo,
lo farò» disse. Poi aggiunse due parole che non ho mai dimenticato o
meritato: «Per te.»

\begin{figure}
	\centering
	\def\svgwidth{\columnwidth}
	\scalebox{0.2}{\input{divisore.pdf_tex}}
\end{figure}

Gli altri Cielcin condividevano una cella nella bastiglia. Al contrario
della bolla d'acciaio della stanza degli interrogatori, la cella di
detenzione aveva pareti, pavimento e soffitto di cemento e la poca luce
presente proveniva da cavi pendenti, le cui lampadine gialle erano
regolate a una bassa intensità. Non mi fu permesso di entrare, ma le
tendenze melodrammatiche della Cappellania avevano fatto sì che la
sezione anteriore della cella comune fosse fatta di sbarre di metallo
dipinte di una vernice bianca che si scrostava a scaglie.

Uno dei Cielcin mi vide -- forse quello che avevo colpito con lo
storditore a Calagah -- e diede di gomito a un compagno. Come una serie
di vele che si gonfino di vento, tutti e dieci i Cielcin si alzarono e
si girarono verso di me. Se fossero stati umani avrei potuto dire che
aspettavano con silenziosa curiosità o che l'odio ardeva gelido nei loro
volti cadaverici, ma non erano umani e non avrei saputo dire cosa
provassero.

Erano stati spogliati dell'armatura e a parte Tanaran indossavano
soltanto una tuta aderente che scintillava delle spire dei tubi per il
riciclaggio dell'umidità e la regolazione della temperatura. I loro
piedi dotati di artigli erano nudi e sembravano mani più di quanto
paresse giusto. Uno di essi sibilò e snudò i denti, \emph{non} nel gesto
che la sua specie usava per sorridere. «Tu!»

Erano passate settimane dall'ultima volta che avevo parlato con uno
qualsiasi degli xenobiti, a parte Uvanari, e mai prima di allora mi ero
rivolto a loro in gruppo. Consapevole che l'inquisitrice Agari era in
piedi in fondo al corridoio e che le videocamere erano annidate come
ragni negli angoli, dissi: «Mi dispiace che siate tenuti in questo modo.
Mi era stato garantito che sarebbe stato fatto altrimenti.» `Parla
sempre a un membro di una folla' mi aveva spesso detto mio padre. `Una
folla ti può ignorare, ma un singolo uomo non può farlo.' Tanaran non
era un uomo, non in un qualsiasi senso di quella parola, ma mi rivolsi
comunque a lui. «Tanaran, so cosa sei.»

Il nobile Cielcin socchiuse gli occhi. «Non lo sai.»

Un debole sorriso mi affiorò sul volto. «L'\emph{ichakta} mi ha detto
che sei una radice. \emph{Baetan}. Non so cosa questo significhi, ma
vuol dire che sei importante.» Scoccai un'occhiata all'inquisitrice
Agari, poi mi girai per parlare con gli altri Cielcin, isolando Tanaran
nella massa.

Lui però parlò per primo, ergendosi sulla persona quanto più glielo
permetteva la bassa volta perché il suo bisogno di mettersi sull'attenti
stava avendo la meglio sul buonsenso. «Come sta Uvanari?»

Ringrazio gli dèi immaginari della Cappellania del fatto che quella
creatura non poteva interpretare le mie espressioni facciali, perché
aprii la bocca in preda all'angoscia. Poi guardai di nuovo verso Agari.
«Chiedono del loro capitano» le dissi in galstani.

«Rispondi che viene trattato bene ma che la ferita subita prima della
cattura non è ancora guarita.»

Deglutendo il nodo di vergogna che avevo in gola tornai a rivolgermi ai
Cielcin. Avrei potuto vomitare, se le passate settimane non mi avessero
fatto smettere di mangiare. Come potevo dire una cosa del genere? Però
lo feci e questo parve confortare Tanaran, che abbassò lo sguardo sui
nudi piedi dotati di artigli con i denti che scintillavano in un
sorriso. Incapace di trattenermi, aggiunsi: «Questo non è quello che
volevo. Se potessi scegliere, sareste tutti in viaggio...» Feci una
pausa per attirare l'attenzione di tutti e riflettere sulle mie parole
successive. «In viaggio per tornare dall'\emph{aeta} Aranata.»

Il suono che accolse quelle parole non aveva niente di umano, un lamento
ululante che era in parte di dolore e in parte di furia. Dovetti
reprimere l'impulso di coprirmi le orecchie. Tanaran si fece più vicino
alle sbarre, cosa che indusse i due legionari che avevo accanto a
irrigidirsi di riflesso. Il Cielcin era tanto alto che dovette
incurvarsi un poco nella cella per lui angusta. \emph{Come può una
	creatura del genere tollerare la forza di gravità di Emesh?} Non potei
non chiedermelo, ricordando quanto avessi sofferto nei primi mesi su
quel mondo enorme. E quelle creature dimoravano nello spazio, su navi
che avevano una frazione della gravità di Delos e ancor più di quella di
Emesh. Dovevano aver laminato le ossa, rinforzato i muscoli, essersi
alterati e adattati per sopravvivere.

Tanaran avvolse le dita troppo lunghe intorno alle sbarre e premette fra
di esse il volto piatto. «L'\emph{ichakta} non tradirebbe mai \emph{lui}
in quel modo. Cosa gli avete fatto?»

«A Uvanari? Niente.» Sbattei le palpebre e indietreggiai di un passo in
reazione a un sussulto nervoso di una delle mie guardie dall'armatura
del colore delle ossa. `Niente...' Quanto era stato difficile dire
quella parola. \emph{Niente.} E quanto era stato facile. Posso quasi
sentirne ancora il sapore sulla lingua. «Gli ho parlato proprio la
scorsa notte.» Sollevai lo sguardo verso le luci pendenti e mi
accigliai. Valka aveva detto che ci sarebbe stato un segnale, che
avrebbe spento le luci della bastiglia con la sua magia tavrosiana in
modo da farmi sapere che il fonorilevatore audio era stato troncato.
Cosa la tratteneva?

«Che altro sai?» Il giovane Cielcin emise una versione più sommessa e
lamentosa del gemito collettivo di poco prima e serrò gli occhi di un
nero liquido. «Sai dov'è la nostra \emph{scianda}? La nostra flotta? La
distruggerete?»

«\emph{Veih}!» Avanzai e mi arrestai sulla linea rossa dipinta sul
pavimento per indicare la distanza minima di sicurezza. «No.
L'\emph{ichakta} non ci ha voluto dare quell'informazione. Ha detto che
non poteva farlo.» Guardai ancora verso Agari. Dovevo aspettare.
Attendere Valka.

«È quello che dici tu!» gridò un altro Cielcin, che aveva la mascella
più squadrata e un fisico più massiccio del magro Tanaran. Senza
preavviso scattò verso le sbarre e protese le braccia in mezzo a esse
afferrandomi per il davanti della camicia. Troppo tardi realizzai che la
linea era stata dipinta tenendo conto della lunghezza delle braccia
umane, molto più corte di quelle dei Cielcin. Reagii senza riflettere,
supponendo che la struttura scheletrica della creatura fosse abbastanza
simile alla mia da rendere possibile liberami e spinsi entrambe le
braccia in mezzo ai pugni serrati dello xenobita, colpendogli i polsi
con i gomiti mentre finivo a faccia in avanti contro le sbarre. Mi
liberai barcollando e finii per cadere seduto.

Agari gridò un ordine e i due legionari vennero avanti, estraendo gli
storditori dalla fondina assicurata alla coscia.

«Non fate fuoco!» intimai, rimettendomi in piedi. I capelli mi erano
caduti sulla faccia e soffiai per allontanarli in modo quasi stizzoso.
Una dei legionari, una donna, mi aiutò a rialzarmi e la ringraziai prima
di fissare l'altro Cielcin con occhi roventi e i denti serrati.
«\emph{Rakur oyumn heiyui}.»

«È stata una cosa stupida» convenne Tanaran, con maggiore veemenza,
guardando con ira il compagno.

«Questo \emph{yukajji} è il motivo per cui siamo prigionieri...»

Tanaran interruppe la creatura più massiccia. «Lo so, Svatarom.
\emph{Svvv}.» Emise un suono fra il sibilante e il ronzante che
interpretai come un ordine di tacere, poi si passò le mani fra i capelli
tagliati alla buona, perso nelle riflessioni. Alla fine socchiuse gli
occhi. «Dici che la nostra gente è al sicuro?»

«Uvanari non vi ha traditi.» Adesso badavo a rimanere ben più indietro
della linea rossa, calcolando la piena portata delle braccia di
Svatarom. Volgendo lo sguardo sullo squallido corridoio sorpresi
l'inquisitrice Agari che mi osservava e le rivolsi quello che sperai
essere un sorriso rassicurante. «A parte il nome del vostro \emph{aeta}
non sappiamo nient'altro, solo che la vostra non era una forza di
invasione.»

Un terzo Cielcin intervenne con una voce più acuta e femminile di quella
degli altri, anche se una gradazione del genere non significava niente
fra gli xenobiti. «Siamo i soli superstiti? Le altre navi...
qualcuno...»

Guardai di nuovo verso Agari. Lei non avrebbe voluto che rivelassi
niente a quelle creature, né fatti né dati. Mi aveva raccomandato di non
dare loro nulla. Scossi il capo. «\emph{Veih}.» Era una risposta tanto
vaga che dovetti ricominciare. «Le altre navi hanno lasciato l'orbita.»
\emph{Lasciare}. Che eufemismo rivelatore. La creatura chinò la testa ed
esalò un respiro fischiante dalle narici, come attraverso denti
spezzati. Altri due Cielcin si affrettarono a sorreggerla quando si
accasciò in ginocchio. Stava singhiozzando?

Tanaran si indurì in volto e chiuse gli occhi. «Capisco.»

Le luci si abbassarono, rimpiazzate da quelle rosse di emergenza
installate in basso sul pavimento.

«Cosa succede?» esclamò Agari, poi ripeté la domanda al terminale da
polso. «Un altro sbalzo di corrente? Credevo che la manutenzione avesse
risolto il problema.»

Valka lo aveva fatto... o almeno speravo che lo avesse fatto. Non c'era
molto tempo, qualsiasi cosa avesse fatto con quel congegno neurale
impiantato nella sua testa, aveva detto che non sarebbe durato a lungo.
Tanaran si guardò intorno, confuso.

«Senti» gli dissi, parlando a bassa voce al di sotto della confusione di
Agari e degli ordini che venivano gridati. «Ascoltami. Ho un'amica che
ha oscurato la sorveglianza qui e nessuno degli altri conosce una sola
parola della vostra lingua. Per un momento possiamo parlare, tu e io.»

«\emph{Iugam}!» Svatarom picchiò le mani contro le sbarre. «È un
trucco.»

«Io non mento!» dichiarai, anche se avevo mentito troppo a lungo. «Cosa
succede alle luci?» chiesi ad Agari, fingendo ignoranza e di non sapere
niente della strega tavrosiana accampata sulle mura del castello con la
mente che si interfacciava con la sfera-dati e i sistemi di sicurezza
della bastiglia. Riflettei che lei era esattamente la ragione per cui la
Cappellania sorvegliava così attentamente la tecnologia all'interno dei
regni e della politica che controllava. L'inquisitrice mi diede la
risposta che mi aspettavo e io tornai a rivolgermi a Svatarom e agli
altri. «Tutto questo non è andato come speravo e stiamo sprecando tempo.
Uvanari mi ha chiesto di usargli misericordia.»

«\emph{Ndaktu}?» chiese Tanaran, con la voce pervasa di un dolore che
qualsiasi specie avrebbe potuto riconoscere. «Perché?»

Mi morsi un labbro, poi sibilai al di sotto delle grida di Agari che
provenivano dal fondo del corridoio: «Perché lo stanno torturando.»

Per quasi un minuto le mie parole non parvero penetrare e rimanemmo lì
fermi nella semioscurità a fissarci a vicenda. Alla fine Tanaran parlò.
«Stanno... facendo del male a Uvanari?» Annuii, poi mi resi conto
dell'inutilità di quel gesto ed emisi quel grugnito inarticolato che per
i Cielcin era un `sì'. Per un secondo pensai che lo xenobita potesse
scoppiare in lacrime e vidi tendersi un muscolo della mascella. «La tua
gente non vuole che lo sappiamo.» Non era una domanda.

«\emph{Veih}.» Scossi il capo, dimenticandomi di fare invece il gesto
alieno.

Tanaran abbassò lo sguardo sul nudo pavimento di cemento, sfregiato da
innumerevoli eoni in cui le porte di metallo erano state trascinate
sulla sua superficie. «Allora ti ringraziamo.»

Svatarom serrò la mascella con un lampeggiare dei denti di vetro. «Lo
\emph{yukajji} deve fare ammenda. È il responsabile.» Il Cielcin che
Uvanari aveva definito una `radice' rimase a lungo in silenzio. Troppo a
lungo. «Tanaran.»

Il giovane xenobita si serrò il labbro inferiore fra i denti.
Quell'espressione per me non aveva significato, ma esso disse: «Svatarom
ha ragione. Sei il responsabile.»

«Cosa significa?» domandai. «Mi ha chiesto di ucciderlo.»

«Sì.» Nella lingua dei Cielcin quella parola era a stento una {sillaba},
un respiro quasi silenzioso. «Fra il Popolo non è giusto che uno come il
capitano debba soffrire.»

Era quello che avevo temuto di sentirgli dire, che avevo temuto essere
vero, ed era il motivo per cui ero venuto lì, lo stesso per cui Valka
aveva oscurato le videocamere. \emph{Ormai non ho più molto tempo. Non
	molto.} «Quindi devo uccidere l'\emph{ichakta}?»

«Quello che causa il disonore deve fare tutto... \emph{tutto}... per
porvi fine. Hai detto che questa è colpa tua e hai ragione. Dici di
volerci mandare a casa ma è troppo tardi...» La voce gli si incrinò. «È
troppo tardi per l'\emph{ichakta.} Tornare a casa è...»

«Significherebbe la vergogna» ringhiò Svatarom, poi sputò. «Tu hai fatto
questo. Hai promesso che non ci avrebbero fatto del male. Hai dato la
tua parola.»

Le luci di sicurezza tremolarono, il che mi indusse a guardare verso
Agari mentre replicavo: «Lo so! Perché credi che sia qui? Capisco cosa
dobbiamo fare, ma ho bisogno del vostro aiuto per farlo.» La notte prima
avevo detto la stessa cosa a Valka, sussurrando sotto il soffio del
vento nel giardino a terrazza, sotto le palme terrestri: `Ho bisogno di
indurli a lasciarmi solo con Uvanari. Allora tu disattiverai le
videocamere e io... io...' La voce mi si era spezzata, ridotta a
qualcosa di soffocato e di molto, molto fievole.

Valka mi aveva posato una mano sul braccio e aveva mormorato che mi
capiva: `Però non sei obbligato a farlo.'

`Non posso continuare a fare questo. Non posso.' Avevo cercato di
spiegare quello che ritenevo il Cielcin stesse cercando di dirmi: che
voleva -- doveva -- morire.

«Cosa farai?» chiese Tanaran.

«Cosa gli stai dicendo?» domandò Agari.

Le segnalai di tacere. L'aria fredda della cella puzzava di marcio, come
se qualcosa di umido fosse morto e si fosse insediato nel cemento, ma
trassi lo stesso un profondo respiro senza distogliere lo sguardo da
quello di Tanaran. Le luci tremolarono di nuovo e sentii il ronzio
lontano dei generatori che si attivavano. Non c'era tempo. Non ce n'era.
«Ucciderò Uvanari. \emph{Ndaktu}. Misericordia.» Cercai di trovare
rifugio in un aforisma scoliastico, qualcosa che mi assicurasse che
stavo facendo la cosa giusta. \emph{La misericordia è... la misericordia
	è...} Non c'era niente, oppure niente che avessi mai appreso. «Ho
bisogno che facciate una cosa, la prossima volta che le luci si
spegneranno...»

E glielo dissi.

Le luci si riaccesero entro un minuto dalla fine del mio breve discorso
e le videocamere si riattivarono insieme a esse. «Un'altra cosa,
Tanaran» aggiunsi, fermandomi mentre fingevo di allontanarmi. «Uvanari
ti ha definito \emph{baetan}. Cosa significa?»

La pelle bianca come il gesso del giovane Cielcin si tinse del grigio
scuro del rossore quando il sangue nero gli invase i capillari delle
guance. Gli altri xenobiti vicini a lui sibilarono, sorprendendo le
guardie più vicine a me. Sollevai una mano per calmarli e ripetei la
domanda.

«Significa che appartengo a lui. All'\emph{aeta}.»

«Credevo che tutti i Cielcin appartenessero al loro \emph{aeta,} al suo
dominio.» Colsi Agari a osservarmi e \emph{} le rivolsi il sorriso più
rassicurane di cui ero capace, anche se sono certo che fu un'espressione
forzata. «Non siete tutti i suoi schiavi?»

Ci fu l'esalare di un respiro, le fessure delle narici si dilatarono.
Era un sì. Tanaran mosse un passo effemminato verso le sbarre. «Io sono
suo.»

\emph{Una concubina? Una moglie?} Scrutai fra le sbarre socchiudendo gli
occhi. Avevo cominciato a pensare a Tanaran come a un maschio -- a
pensarlo di tutti i Cielcin, a dire il vero --, ma rividi le mie
opinioni ricordando a me stesso che quello che avevo davanti non era
neppure una donna ma qualcosa di più, qualcosa di meno... qualcosa di
totalmente diverso. Qui ero al di là dell'umanità, al di là della
portata di una traduzione. Le modalità sessuali dei Cielcin non si
sovrapponevano alle nostre, né biologicamente né socialmente. A farlo
era il nostro desiderio di umanizzarli. «Cosa significa?»

«Sono suo. Lui è trasportato da me.» Si premette una mano sullo stomaco
in un gesto che non compresi.

«Trasportato?» ripetei. «Ha qualcosa a che fare con i bambini?» Mi resi
conto di non avere idea di come si riproducessero gli xenobiti.

E continuo a non averla, perché Tanaran indietreggiò, sorpreso. «Cosa?
No!» Ruotò la testa in un furioso gesto di diniego. «Io porto un suo
pezzo. La sua autorità.»

Immagini degli auctor \emph{} imperiali mi passarono davanti agli occhi,
quell'élite investita di tutta l'autorità imperiale, della capacità di
agire in sua vece e in sua assenza, titolati al punto da essere pari
all'imperatore in autorità, anche se non potevano senza di lui.
Condividevano la Presenza imperiale, parlavano con la sua voce. Si
trattava di questo? O era qualcosa d'altro? Forse era così, e allora
Tanaran era il capo di quella spedizione? Del pellegrinaggio? A corto di
tempo com'ero non insistetti oltre.

«Ancora una cosa» aggiunsi in fretta, consapevole di essere di nuovo
osservato dai diecimila occhi dello Stato. «Uvanari ha detto che siete
venuti qui per pregare. Pregare gli altri? I... i primi?» Avrei voluto
dire `la Quiete', ma sapevo che quell'etichetta non avrebbe significato
nulla.

«Gli dèi» confermò Tanaran. «Gli Osservatori.» Afferrò le sbarre. «Loro
ci hanno creati, \emph{yukajji}. Noi.» Snudò le zanne, facendosi in
qualche modo fiero nell'arco di un istante.

Percependo quel cambiamento nel suo modo di fare, l'inquisitrice si
avvicinò. «Ha parlato anche troppo, Marlowe.» Mi afferrò per un gomito.
«Che cosa ha detto?» Accennò con la testa rasata in direzione della
cella.

«Niente... mi hanno detto di andare all'inferno» risposi, scuotendo il
capo. «Ho creduto di riuscire a farmi ascoltare, ma... mi incolpano di
tutto. Posso farti una trascrizione. Hai una registrazione?»

«Parziale» rispose. «Un altro di quei cali di tensione, che è stato
localizzato nel castello, ma... non avrebbe dovuto essere possibile.»

Liberai il braccio con uno strattone, mi inchinai e mi diressi
all'ascensore che mi avrebbe riportato al livello superiore della
bastiglia e all'uscita. «Scriverò la trascrizione il più in fretta
possibile. Non c'è molto, ma ho avuto conferma di una cosa.» Non c'era
modo di evitarlo, dovevo dare qualcosa ad Agari, qualcosa che distraesse
lei e i suoi superiori da quello che i Cielcin volevano da me.

«Di cosa si tratta?»

Aprii la bocca. Qualsiasi cosa, pur di mascherare il vero scopo della
mia visita. Speravo solo di non condannare il giovane xenobita a finire
a sua volta appeso a una croce. Forse il suo stato di \emph{baetan} lo
avrebbe protetto. Quantomeno lo sperai mentre dicevo: «Hai presente
quello senza armatura? Quello piccolo?»

«Sì?»

«È un nobile, o quello che passa per un nobile fra i Pallidi.»

