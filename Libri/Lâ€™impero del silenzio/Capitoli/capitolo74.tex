\chapter{Il labirinto}

Le porte pneumatiche sibilarono alle mie spalle, lasciandomi come
imbalsamato nell'oscurità. Le settimane trascorse avevano richiesto il
loro prezzo, e come aveva sofferto il prigioniero così aveva fatto la
cella che adesso puzzava di carne marcia. Se mai siete passati accanto
alla baracca di un popolano, in campagna, e avete sentito l'odore della
carcassa di un daino cacciato di frodo, scuoiato e lasciato troppo a
lungo in attesa di finire nella pentola dello stufato, potete immaginare
di che genere di fetore si trattava. Luci rosse risplendevano fioche
sulle pareti, tenute basse per una perversa forma di cortesia verso la
creatura notturna che era il solo occupante della cella. Una parte di me
invidiò il povero, morto Gilliam. Il puzzo di carne marcia, di sangue e
di escrementi inumani riempiva l'aria e avrei voluto premermi sulla
faccia un fazzoletto profumato per mascherare quell'odore, come avrebbe
fatto il prete.

Osservai Uvanari nella penombra. L'\emph{ichakta} era legato di nuovo
alla croce ed era là da due giorni, con l'intera struttura inclinata
all'indietro per mantenere il paziente cosciente e mossa a intervalli
mediante un controllo automatico per ridurre l'incidenza di piaghe da
decubito.

Al rumore della porta la testa coronata si sollevò e gli occhi neri si
ridussero a due fessure quando mi vide. «Tu» sibilò, poi vide l'oggetto
che avevo in mano. «Quello cos'è? Sei venuto per uccidermi?»

«\emph{Veih}» risposi. `No.' Sollevai la siringa perché la esaminasse.
«È per il dolore. I nostri dottori hanno esaminato la chimica del tuo
sangue e pensano che questo funzionerà. Per favore.» Sollevai di nuovo
la siringa, offrendola. Uvanari girò la testa in un gesto debole ma...
pieno di sfida? Rassegnato? Era il gesto di un predatore ferito,
consapevole che il cacciatore è venuto ad abbatterlo e che gli offre il
collo. Invece gli infilai l'ago nel braccio per poi indietreggiare e
lasciare che il medicinale facesse effetto. Mentre aspettavamo in
silenzio riflettei su tutte le antiche leggende che parlavano di sieri
della verità, di droghe magiche capaci di indurre un uomo a rivelare i
suoi più intimi segreti. Tutte false. Non esistevano sostanze del
genere. Scopolamina. Tiopentale. Amvtal. Tutti alterano la mente, aprono
porte, ma la verità... la \emph{verità} è un'altra cosa, qualcosa di
separato dal sapere. Inoltre, quale che sia l'effetto che simili veleni
possono avere su uno spirito umano, Uvanari non era umano. Era già un
miracolo che potessimo dargli qualcosa per il dolore, figuriamoci
qualcosa per farlo parlare. Comunque, osai sperare.

Dopo un minuto il Cielcin cominciò visibilmente a rilassarsi. Il suo
aspetto era molto peggiorato da quando era prigioniero. Il braccio
destro era scuoiato, con la pelle rimossa senza versare sangue dal polso
al gomito, e quel tessuto bluastro era stato ricoperto di una sostanza
umida e glutinosa che gli impediva di marcire. Ero stato presente quando
la pelle era stata rimossa come la calza di una donna. Posso ancora
sentire le urla come le sentivo allora, nel silenzio della cella. Il
prefetto buono e quello cattivo, dissi a me stesso. «Mi dispiace,
\emph{ichakta}» cominciai. «Non sapevo che avrebbero fatto questo.» Però
lo avevo saputo, giusto? Avevo visto quello che la Cappellania aveva
fatto alla nostra stessa gente, i criminali macellati e gli eretici
marchiati che costellavano Meidua e Borosevo, e senza dubbio ogni altra
città di ogni pianeta che c'era nel mezzo. Ignorare una cosa non è non
conoscerla, e chi rimane in silenzio è un complice, ogni volta.

«\emph{Okun detu ne}?»

La domanda mi sorprese così tanto che la ripetei. «Perché io? Cosa vuoi
dire?»

«Per tutto questo tempo, tu. Tu stai lì con quelle... \emph{creature}.
Gli altri. Parli per loro. Perché? Chi sei?»

Il tempo passò mentre lottavo per trovare una risposta. Posai
l'anestetico sul carrello degli strumenti di tortura, un oggetto di
{misericordia} in mezzo a tanto dolore e sofferenza. «Te l'ho detto,
sono soltanto un uomo. Non sono nessuno, solo... conosco le parole. È
per questo che ero nella grotta, per questo sono qui.»

«Pensavo che lo avresti detto.» Mentre parlava mi portai alle sue spalle
e modificai l'angolazione della parte regolabile dell'impalcatura che
teneva al suo posto il braccio, orientandolo verso il basso in modo che
la circolazione del sangue e la sensibilità potessero tornare nell'arto
ferito. I cathar avevano fasciato le dita accorciate e private di
artigli, e il sangue secco brillava nero contro la garza bianca. Uvanari
gemette sommessamente quando il braccio scuoiato si mosse contro la base
imbottita dell'impalcatura e io ebbi un sussulto dettato dall'empatia.
«Allora sei uno schiavo?»

Scossi il capo, poi mi resi conto che non mi poteva vedere. «No.»

Istintivamente avevo parlato in galstani, ma il Cielcin parve
comprendere. «Non sei \emph{diyugatsayu}.»

«Libero?» ripetei. «Io sono libero.»

«Chi lavora per altri non è \emph{diyugatsayu}» dichiarò Uvanari. «Sei
uno schiavo.» Aggirai la croce per guardarlo in faccia. Era davvero alto
nove piedi? Le sofferenze lo avevano rattrappito come una sagoma
schiacciata da una luce intensa. «Ma questo è bene... siamo tutti
schiavi, Hadrian.» Pronunciato dalla sua lingua inumana, il mio nome
suonava come una sorta di accusa che mi strappò un sussulto. Sentivo
occhi che mi osservavano, due inumani e innumerevoli altri artificiali,
nelle pareti e nel soffitto, tramite i quali l'umanità presente e futura
mi avrebbe guardato. Sapevo che non avrei dovuto sussultare, che non
avevo il diritto di farlo perché in quel momento non ero me stesso ma un
avatar dell'umanità che parlava a nome di tutti noi con Uvanari, che a
sua volta parlava per la sua specie. Qualsiasi cosa Hadrian provasse,
l'umanità non doveva sussultare.

Per quanto desiderassi esplorare la filosofia con quella mente così
dissimile da quella degli uomini, avevo un copione da rispettare, una
missione a cui obbedire. Come aveva affermato Uvanari, ero una sorta di
schiavo. Il prefetto buono. Mi schiarii la gola. «Vogliono che tu dia
loro \emph{qualcosa}. Se lo farai, smetteranno tutto questo, ti
guariranno. Possono innestare la pelle del tuo braccio.» La parola
`innestare' era per me un mistero, e passai almeno dieci secondi a
cercare un modo di sostituirla. Alla fine ripiegai sull'usare di nuovo
il termine \emph{caenuri}, `risanare'. «Per favore. Chi è il vostro
capo? Il vostro \emph{aeta}?» Quella parola si traduceva generalmente
con `principe', ma aveva connessioni etimologiche con termini come
`creatore', `proprietario' e `padrone'.

«Ho finito di parlare con te, Hadrian.» L'alieno distolse il volto e
chiuse gli occhi.

Anche se al mio ingresso le sue braccia erano state allargate, i cathar
avevano ripiegato la vittima in una posizione seduta sulla croce
regolabile e avevano gettato una coperta sulla sua sporca nudità. Se ne
rimase lì seduto, a braccia allargate, con la faccia girata, senza
parlare. Da bambino avevo appreso delle vittime benedette di antiche
religioni, uomini e donne uccisi per la loro incrollabile fede in dèi
ora dimenticati. La Cappellania aveva adottato simili immagini e le
usava nella venerazione degli Eroi, uomini e donne che avevano dedicato
la vita al regno o dato la vita per esso. Figure di uomini legati ad
alberi o croci, con il volto sollevato verso il cielo con una composta
espressione di silenziosa devozione e non in una smorfia di sofferenza.
Con un volto alieno tanto demoniaco con la sua corona di corna d'osso
bianco, la parodia era quasi oscena. Quasi sacrilega, perfino per uno
come me che non sapeva pregare.

«Ci devi dare qualcosa.»

«Tutto quello che mi rimane da fare è morire» dichiarò Uvanari e mi
parve quasi di sentire Gibson che lo rimproverava per la sua
melodrammaticità proprio come aveva fatto con me. «Arrendermi è stato un
errore. Sapevo che si sarebbe arrivati a questo.»

Guardai le ustioni, il sangue secco sulle dita, la fascia di pelle
scuoiata. «Allora perché lo hai fatto?»

«Per lo stesso motivo che hai dato tu.» Poi la creatura fece una cosa
che non ho mai dimenticato, che rimane incisa nella mia memoria come con
un laser. Scosse il capo. Non era quel gesto del collo che indicava un
assenso o un diniego. Eseguì quel gesto umano che aveva appreso nel
parlare con me. «Questa non è la mia guerra come non è la tua. Noi
l'abbiamo ereditata, proprio come voi. Quando lo hai detto, ho osato
sperare.»

«Noi abbiamo un modo di dire.» Raddrizzai la schiena e protesi in fuori
il mento. «Il mio insegnante affermava sempre che la speranza è una
nuvola.»

Non so se fu l'introspezione o la sofferenza a rallentare la sua
risposta. «Esso sembra molto saggio.»

«Lui» lo corressi, e Uvatari sbuffò attraverso le fessure del naso.
«Voglio restituirti al tuo popolo, Itana Uvanari» continuai. «Al tuo
\emph{aeta}.» \emph{Al tuo padrone? Al tuo proprietario? Al tuo
	creatore?}

Era una luce quella che vedevo nei suoi occhi neri come l'inchiostro?
Era speranza?

`La speranza è una nuvola' mi disse all'orecchio la voce di Gibson, che
era parte di me.

Uvanari rivolse quegli occhi al soffitto e parlò come a un congresso di
osservatori invisibili, gli inquisitori e i logoteti, o forse agli ancor
più invisibili e sordi alle sue parole. «Al mio popolo? Dopo tutto
quello che la tua gente ha fatto? Credi che il mio padrone vi
ringrazierà per questo?» Cercò di muovere il braccio scuoiato contro i
legami elettromagnetici.

«Credevo avessi detto che eri uno schiavo. Quanto vale la vostra vita
per il tuo padrone?» Rigirai le mie domande come fossero coltelli.
«Quanto di più varranno quelle vite nel trattare con noi?»

«Noi siamo \emph{kasamnte}.» sussurrò. «`Niente.' Lo capisci? Neppure
Tanaran ha importanza per \emph{lui}.»

Con le mani paralizzate lungo i fianchi interruppi il mio lento girare
in cerchio intorno al Cielcin sulla sua croce. Per un momento ci fu
soltanto il gelo del nostro respiro che diventava vapore nell'aria.
C'era così tanto da sviscerare in quella frase, e se quella fosse stata
una lezione con Gibson gli avrei chiesto di ripeterla. \emph{Tanaran?}
Rinviai per un momento quel dettaglio su Tanaran e mi concentrai sul
pronome: `lui'. Uvanari aveva decisamente usato il pronome attivo,
\emph{o-kousun}, ma il problema era l'oggetto della frase, che seguiva
la struttura neutro-neutro che ci si aspettava di trovare intorno a un
verbo di collegamento, e che dava la sensazione di essere sbagliato
quanto un dente rotto.

«Lui?» chiesi. Uvatari non rispose, non volle neppure guardarmi. Questo
era il momento in cui l'inquisitrice Agari lo avrebbe colpito, o avrebbe
strappato via la coperta esponendo la sua nudità, o gli avrebbe fatto
rimuovere con le pinze un altro dito o artiglio. Io non ero
l'inquisitrice Agari. «Cosa intendi con quel `lui'? Il tuo padrone? Il
tuo \emph{aeta}?»

«Il mio...» Uvanari serrò la mascella e ruotò la testa in senso
antiorario. \emph{No.}

Cambiai tattica. «Cosa intendevi riguardo a Tanaran? `Neppure Tanaran ha
importanza per lui', hai detto. \emph{Hejato Tanaran higatseyu
	ti-kousun}. Cos'è Tanaran in questo?» Mentre sostavo di fronte a Uvanari
riflettei sul giovane Cielcin tormentandomi l'unghia di un pollice.
Tanaran non aveva indossato un'armatura da combattimento come quella che
portavano Uvanari e gli altri soldati, ma soltanto una stoffa leggera
verde e nera. «Tanaran è \emph{aeta}?»

«Tanaran?» Uvanari quasi rise, intontito a causa dell'antidolorifico che
gli avevo somministrato. «Tanaran è \emph{baetan}.»

Sbattei le palpebre, sconcertato, e mi rimossi dai denti una scheggia di
unghia del pollice. «Una radice?»

«Mi ucciderai, Hadrian?» Quelle parole giunsero come dal nulla,
illogiche, ma la loro presenza aveva aleggiato su tutto il colloquio
precedente, su ogni parola che ci eravamo scambiati quel giorno. Tutto
aveva portato a questo... questo...

Trapassai la creatura con uno sguardo penetrante attraverso il velo dei
mieli capelli flosci. Uvanari mi stava guardando, con il volto inumano
teso e pallido come il latte. Vidi che una delle corna della sua corona
era stata spezzata, e in modo tutt'altro che pulito. Quel che ne
rimaneva era uno spuntone irregolare. In qualche modo questo mi era
sfuggito nonostante tutte le mie osservazioni, tutte le sessioni in cui
ero rimasto come uno spettro in un angolo di quella stanza. Non uno
spettro... un demone, come le macchine che il popolo di Valka usava per
tradurre.

«\emph{Biqa o-okarin ne}?» chiesi. `Ucciderti?' «Non ti posso uccidere.
Cosa intendevi riguardo a Tanaran?»

«Sarà... un giorno potrebbe essere \emph{aeta}» rispose Uvanari, poi
cambiò argomento. «Mi devi uccidere. È l'usanza. Se davvero ti dispiace,
questa è l'usanza.»

Una pietra fredda mi gravò sulla gola, un blocco di carbone freddo, duro
e morto, senza il fuoco della redenzione delle antiche fiabe. Non
riuscii a parlare, non a Uvanari. Le parole che trovai erano in
galstani. «Volevo aiutare. Volevo... rendere le cose \emph{migliori}.»
Il pavimento scintillava del colore dei coltelli, spazzolato e lavato
fino a pulirlo. Quante persone avevano concluso la loro vita in quella
stanza, con la loro storia spenta, passato e futuro cancellati? «È colpa
mia.» Era questo che aveva inteso dire Gibson: `Le brutture del mondo.'
Tornai a esprimermi in cielcin. «Non posso farlo.»

«\emph{Devi}. È \emph{ndaktu}.»

`Misericordia.' No... \emph{una sorta di misericordia formale}. Cercai
di ricordare la definizione precisa della parola, che aveva implicazioni
quasi legali. Feci un altro tentativo. «Dov'è la vostra flotta, Uvanari?
La tua gente?»

Ruotò la testa in un gesto negativo antiorario e parve avere quasi una
sorta di convulsione. «No. No.»

«Voglio contattarli. Ci deve essere un modo, uno che non li metta a
rischio. Voglio vedere te, Tanaran e gli altri restituiti alla vostra
gente. Davvero.» Non ero un bugiardo, ma ero reso tale dal meccanismo
del contesto. Sapevo che qualsiasi cosa avessi fatto, qualsiasi cosa
avessi appreso, sarebbe stata distorta. Rammentai le ultime parole di un
antico generale il cui significato si era perso o non era mai stato
compreso: `Come farò mai a uscire da questo labirinto?' Però dovevo
continuare a camminare come aveva fatto Teseo, sempre avanti, sempre
verso il basso, mai a sinistra o a destra. Ero complice di quegli orrori
solo nella misura in cui una volpe è complice nell'essere cacciata o un
coniglio lo è in una corsa di cani. Cercavo una via di fuga. Sapevo che
qualsiasi cosa avessi detto o fatto sarebbe servita alla loro causa, e
tuttavia come potevo agire altrimenti?

L'\emph{ichakta} sputò sulla grata alla base della croce. «Non tradirò
la mia gente. Non con voi \emph{yukajjimn}.»

«Dovevi proprio dire `animali nocivi'?» borbottai in galstani mentre
chinavo la testa e mi sfregavo gli occhi con un frustrato senso di
sconfitta. Non ci chiamavamo più per nome, ma avevo appreso una cosa --
che dovevo parlare con Tanaran e non con Uvanari -- per cui mi girai per
andarmene.

«Aspetta» disse lo xenobita. Mi fermai, già a metà strada dalla porta.
«Il suo nome è Aranata.»

«Il nome di chi?» chiesi, anche se compresi cosa aveva inteso dire nel
momento in cui finii la frase.

«Del mio padrone» rispose il capitano. Aveva usato di nuovo il pronome
maschile... a chi altri avrebbe potuto riferirsi? «Aranata Otiolo. Non
lo troverete. Ma... adesso smetterete?»

«Allora collaborerai?» domandai, girandomi a fronteggiarlo. «Ci dirai
dov'è la tua gente?»

Ci fu un momento di silenzio, terribile come le stelle, poi: «No. Non
posso. Non lo so. Ci spostiamo.»

«Devi avere un modo per ritornare a casa» affermai, incredulo.

Ruotò la testa in un gesto di diniego. «No. \emph{Veih}. No.»

«Allora non posso farli smettere.» Non avrei potuto farlo comunque,
qualsiasi cosa ci avesse detto. Non avrei mai potuto farli smettere.

Ci fu un'altra pausa, questa volta più breve. «Se dico loro quello che
vogliono sapere mi uccideranno?» Quando non risposi, parlò ancora con
voce che era poco più dell'arido sussurro di foglie su un vetro rotto.
«\emph{Biqaun ne}?»

`Lo farai tu?'

