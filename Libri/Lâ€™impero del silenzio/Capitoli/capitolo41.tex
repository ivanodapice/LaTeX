\chapter{Amici}

Switch non mi rivolgeva la parola. Erano passati due giorni e non aveva
aperto bocca, neppure durante gli addestramenti. Non potevo biasimarlo.
Torna a suo merito il fatto che per quanto ne sapevo non aveva tradito
il mio segreto con Pallino o qualcuno degli altri, perché era leale
anche nelle profondità del suo dispiacere, il genere di amico che tutti
desiderano ma che nessun uomo merita. Una nube incombeva su di me. Avevo
perso il mio unico, vero alleato e con lui il mio impeto. Era come aver
perso di nuovo la lettera di Gibson, ma ancora peggio perché ero stato
io a farlo a me stesso e al mio amico.

Tre mirmidoni -- tutti nuovi venuti -- mi si paravano di fronte con le
loro spade corte dalla lama smussata. Indossavano un'armatura di acciaio
stampato e sembravano imitazioni di legionari imperiali medievali con la
tunica rossa lunga fino al ginocchio e la graffiata corazza dipinta di
un color avorio che in alcuni punti si scrostava. Il primo mi si lanciò
contro con gli occhi dilatati. Lo evitai con facilità e lo gettai a
terra con un calcio quando mi passò accanto. A quel punto la seconda,
una donna, si fece avanti tenendo alta la spada. Intercettai il fendente
con il vambrace e feci risuonare il suo elmo come una campana,
mandandola a rotolare via. Il terzo se la cavò un po' meglio. Parai
l'attacco con una precisa torsione del polso e risposi con un affondo.
Lui riuscì a prendere le distanze dalla mia risposta ma quando
contrattaccò gli afferrai il polso e lo strattonai in avanti,
poggiandogli contro la gola la lama smussata della mia spada.

«Arrenditi!» ingiunsi, con voce sorprendentemente alta.

Lo spinsi via con la faccia contratta per il disgusto. «Tutto qui?»
domandai, guardandomi intorno. «Tutto qui?» Mi piantai le mani sui
fianchi, lasciandomi ampiamente esposto. Ero in preda a un cupo
umorismo, una rabbia difficile da accantonare, e fissai con occhi
roventi quei tre, due dei quali erano distesi nella polvere del cortile
di addestramento. «Voi tre non resisterete cinque minuti al vostro primo
incontro.» Agitai le braccia. «Avete un netto vantaggio, sapete?»

La ragazza si rimise in piedi e questa volta avanzò con maggiore
cautela. Mantenni le braccia spalancate, inducendola a fare la sua
mossa. Lei diresse un fendente alla mia testa e io non lo parai, mi
inclinai. La lama mi sibilò vicino all'orecchio e io schivai un secondo
colpo con una torsione. Era così lenta, ma non era colpa sua, era
soltanto umana. A suo tempo cadde di nuovo nella polvere, imprecando. Io
ruotai su me stesso per affrontare gli altri due aspiranti mirmidoni. Mi
lanciai contro uno di essi, facendogli cadere la spada di mano e
girandomi proprio mentre l'altro cercava di sorprendermi alle spalle. Lo
afferrai per la corazza, appena sotto il braccio, e lo scaraventai
lontano da me.

«Nessuno di voi sembra capire» dichiarai con un sogghigno, muovendo
qualche passo pesante verso quello che si era arreso. Lui indietreggiò
barcollando e sollevò la spada in posizione di guardia. Mi girai
ridendo. «Siete in tre! Nel nome della Terra, ci si aspetta che operiate
come una squadra.» Ruotai sui talloni, lasciando che mi circondassero
nella luce delle lampade. Sopra di noi il cielo era buio e pesante, con
le due lune simili a occhi spaiati e semichiusi nell'ombra. «Noi siamo
\emph{sempre} numericamente superiori ai gladiatori. È il nostro unico
vantaggio.»

«Questo e la nostra affascinante personalità.» La voce di Pallino
fendette la notte come una frusta. «Lasciali in pace, Had.» Il vecchio
legionario mi fissò con un'espressione rovente nell'unico occhio e i
capelli bianchi ritti sotto il soffio della brezza densa. Aiutò la
ragazza a rialzarsi. «Voi tre andate a ripulirvi. Avete tempo prima del
vostro primo incontro e vi raddrizzeremo prima che arrivi il momento di
lavorare di coltello.» Li guardò allontanarsi, e intravidi Elara e Siran
che si tenevano nell'ombra delle colonne che circondavano il cortile di
addestramento come attori che {aspettassero} dietro le quinte. Quando i
cadetti se ne furono andati, Pallino mi si rivoltò contro. «Nel nome
dell'inferno nucleare, ragazzo, cosa ti è preso?» Rimasi momentaneamente
sconcertato con la spada ancora fra le dita allentate. «Quei combattenti
non hanno bisogno che li conci per le feste. Dannazione, li dovresti
addestrare!» Mi aspettai che mi colpisse, volevo che lo facesse, che ci
provasse.

«Non dureranno un solo incontro» dissi infine, nel tono più misurato e
controllato di cui ero capace.

«Mi pare di ricordare un certo ragazzo con i capelli rossi di cui Ghen
era solito dire la stessa cosa.»

Questo mi bloccò. Feci vagare lo sguardo per il cortile quasi vuoto, poi
mi fissai le mani. Negli anni trascorsi da quando avevo lasciato Delos
si erano indurite, inspessite, sembravano quasi quelle di Crispin.
Deglutii a fatica e lasciai defluire la rabbia fino a vederci di nuovo.

«Non mi parlare di Switch» ringhiai, proteggendo il mio orgoglio in
frantumi mentre riponevo nel fodero la spada da addestramento.

Mentre parlavo le due donne stavano venendo avanti, ed Elara inarcò le
sopracciglia. «Una lite fra innamorati?» La fissai con occhi roventi ma
non risposi. «Il ragazzo si è mostrato decisamente di cattivo umore
negli ultimi giorni. Mi sarei dovuta rendere conto che dietro c'eri tu.»
Per quanto silenziosa, Siran mi guardava con un sorrisetto che non mi
piaceva. «Che cosa hai fatto?»

«Cosa intendi con `che cosa ho fatto'?» Sollevai il mento in un
inconscio gesto di sfida. «Era lui ad avere un problema.»

«Ma sei tu quello che è pieno di rabbia» osservò Elara, battendo una
mano sulla spalla di Pallino. «A me sembra un atteggiamento colpevole.»

Sollevai un dito per rispondere e aprii la bocca, ma le parole non
vollero uscire, quindi la richiusi e il dito rimase lì sospeso come un
metronomo rotto. «È stato lui a spingervi a fare questo?»

«A fare cosa?» Pallino incrociò le braccia.

«Questo...» Qual era la parola? «Questo!» Agitai una mano in un ampio
gesto, poi mi sfilai l'elmo, con i capelli che mi ricadevano sulla
fronte. Nei loro occhi c'era qualcosa. Compassione? Sospetto? No.
«Abbiamo cercato di comprare un'astronave e ne abbiamo vista una nei
cantieri di navi di recupero del Distretto Inferiore.»

Pallino emise un sospiro, esalando il fiato dal naso con manifesta
derisione, mentre Siran inclinò la testa da un lato. «Perché andare a
vedere le navi?»

«Had e il suo grazioso amico hanno intenzione di lasciare il pianeta il
più in fretta possibile alla scadenza del contratto» intervenne Elara,
allontanando i capelli corti dal volto squadrato.

La prigioniera-mirmidone si mostrò indignata. «Perché non avete detto
niente?»

Sbattei le palpebre... non avevo voluto dirlo a Siran, o a Ghen, o a
nessun altro dei prigionieri, se era per questo, per il semplice fatto
che non sarebbe stato permesso loro di venire con noi. Non erano lì per
scelta e solo un condono scritto proveniente dall'ufficio del conte
avrebbe potuto liberarli. «Io...» Lasciai scorrere di nuovo lo sguardo
da un volto all'altro, cercando di decifrare la strana espressione tesa
di tutti e tre. Sospirai. «Mi dispiace. Non volevo che vi sentiste
esclusi.»

«Quindi ci hai lasciati fuori?» ribatté Siran, con un sorriso ironico, e
sentii di essere già sconfitto in quella discussione. Eppure lei non
sembrava ferita, e almeno questo era un sollievo.

D'un tratto la punta graffiata del mio stivale si fece molto
interessante e presi a studiarla con una concentrazione che avrebbe
suscitato le lodi di tor Gibson. «Mi dispiace. Switch e io... abbiamo
tenuto la cosa privata. Non volevamo che chiunque cercasse di unirsi a
noi.»

«La sola persona a cui riesco a pensare che tu possa non volere è Ghen,
e lui non può andare da nessuna parte, proprio come me.»

Adesso Siran stava sorridendo apertamente, e la sua narice tagliata era
quasi scomparsa nella luce delle lampade.

Ammisi la validità dell'argomentazione con quel poco di buona grazia che
mi era rimasta.

«Perché diavolo siete andati a vedere le navi?» intervenne Pallino.
«Credevo di averti detto che non avevamo i soldi e di non pensarci più.»

«Tecnicamente, mi hai detto che ne avremmo parlato allo scadere
dell'anno.»

Il vecchio imprecò, guardando da Elara a Siran come se non riuscisse a
credere a quello che avevo detto. «Quarant'anni al servizio
dell'imperatore ed è questo che ottengo? Insolenza?»

«Had è strano da quando quel prete lo ha stordito» intervenne Elara,
attirandosi una mia occhiata gelida.

Pallino sgranò gli occhi azzurri. «Come sarebbe?» Si girò a guardare la
sua amante mentre sollevava la mano per assestarsi la rozza benda di
pelle nera sull'occhio.

L'ultima cosa che volevo era raccontare quella particolare storia
proprio in quel momento, ma sospirai e riassunsi in modo succinto
l'incidente con l'intus Gilliam e i suoi compatrioti foederati. Per
quanto odiassi ammetterlo, ero lieto di quella tregua, di quella
momentanea distrazione da Switch e dal danno che avevo causato.
Nonostante quello che puoi pensare nel leggere questo resoconto, non mi
piace rivivere i miei errori, e quello mi bruciava ancora.

«Il gobbo?» Pallino si accigliò «È il bastardo della priora, giusto?»

«Sì» confermai in tono alquanto cupo.

«Credete che abbiano davvero portato di soppiatto uno dei Pallidi qui
sotto?» Pallino guardò verso Siran, che dopotutto era lei stessa una
prigioniera, libera solo in quel cortile di addestramento. Lei si limitò
a scrollare le spalle e io passai l'elmo da una mano all'altra, non
sapendo bene cosa dire. Avevo all'incirca una mezza dozzina di idee su
come penetrare nelle prigioni del colosseo per verificare se ci fosse
qualcosa di vero in quello che il mercenario Kogan aveva detto a me e a
Switch in una conversazione che sembrava risalire a mesi prima.
Guardando indietro, credo che sia stato quell'incontro ravvicinato con
il Buio Esterno a spingermi a tornare al deposito di Gila e a sbloccare
un pezzo della mia identità di palatino, sia pure solo per un momento.

«Forse» replicai.

«Adesso comunque non è importante, Pal» affermò Elara, posandogli una
mano sulla spalla. Con lo sguardo su di me, domandò: «Andrà tutto bene,
Had?»

«Non sono stato del tutto sincero con lui... con Switch, intendo»
ammisi. E di certo non ero del tutto sincero neppure con quei tre. Che
pensassero pure che avevo cercato di truffarlo nel nostro accordo. Non
mi importava che mi si considerasse un imbroglione, ero stato chiamato
in modi anche peggiori, e quello che avevo intenzione di fare alla
squadra di Gila e alla Contea di Mataro avrebbe dovuto essere una prova
sufficiente di questo.

«Tutto qui?» Pallino scrollò le spalle, appoggiandosi alla tozza colonna
più vicina. «Per la Terra nera, ragazzo, a giudicare da come si
preoccupavano queste due credevo che fosse qualcosa di molto più
grave...» Agitò una mano in direzione di Siran e di Elara, che si
mostrarono irritate. «Senti, qui dobbiamo addestrarci, come stavi
dicendo a quei poveracci, quindi rimettiti in sesto, Had. Non voglio che
ti lanci in queste assurdità da berserker in un vero combattimento,
perché non sei un fottuto maeskolos e non puoi affrontare tre avversari
contemporaneamente, che non saranno i novellini che stai maltrattando.»
Non seppi bene come rispondere a questo ma non dovetti farlo, perché
Pallino non aveva finito. «Possiamo anche essere tuoi amici, ragazzo, ma
se la prossima volta che saremo nell'arena ti butterai in un casino non
intendo seguirti.» Si passò un dito sulla gola a enfatizzare il suo
punto.

Pieno di vergogna abbassai la testa per indicare che avevo capito.

«Siamo solo preoccupati per te, ragazzo» aggiunse Elara, posandomi una
mano sulla spalla con fare conciliante.

Mi liberai con una scrollata e mi diressi verso la porta. Avevano
ragione, ma non ero obbligato a dirlo. «Questo non è un gioco, ragazzo!
Non per noi!» mi gridò dietro Pallino. «Ehi, stiamo parlando con te.»

Era troppo. Arriva un momento, al di là dei nostri errori, quando
dobbiamo decidere di smetterla di aggiungere altro carico al loro peso.
Arriva prima che siamo disposti a portare quel peso ma dopo che ce lo
siamo addossato. Serrai la mascella nel girarmi a guardarli con ira. Il
mio errore con Switch non aveva cancellato le necessità dettate dalla
mia condizione. Mi serviva quella nave e avrei fatto qualsiasi cosa per
averla.

«Non puoi semplicemente parlare con Switch?» intervenne Siran, cercando
di essere la voce della ragione. «Si è mostrato assolutamente
irragionevole...»

«Allora forse gli puoi parlare tu» ribattei, lieto di quella semplice
risposta.

\begin{figure}
	\centering
	\def\svgwidth{\columnwidth}
	\scalebox{0.2}{\input{divisore.pdf_tex}}
\end{figure}

Fu solo dopo che mi ebbero lasciato solo, come meritavo, che realizzai
cosa fosse stata quella strana espressione che avevo colto sulla loro
faccia alla luce incerta simile a quella delle candele. \emph{Non era}
avversione o sospetto, e neppure pietà. Era preoccupazione. {Avevano}
paura per me. Non per la mia vita, come aveva fatto Cat, e non era
neppure per timore di mio padre. Gli importava di me perché avevano
scelto di farlo con una burbera ma quieta indelicatezza che mi sosteneva
nella mia disperazione e mi sussurrava che era questo ciò che
significava avere una famiglia. Una famiglia irregolare e spaccona,
senza dubbio, ma non l'avrei barattata con quella naturale per tutte le
astronavi che ci sono nel cielo.

E tuttavia... tuttavia stavo per lasciarla, o almeno cercavo di farlo.
Ci avevo provato da prima di conoscerli, da quando avevo incontrato quel
marinaio, il Corvo, quel giorno in quel caffè. La storia di Kogan mi
risuonava ancora nelle orecchie, e le sue parole agivano dentro di me
come una scintilla sull'esca, proprio come avevano fatto quelle del
Corvo. Ricordavo il bambino che ero stato non molto tempo prima. Hadrian
Marlowe. Volevo conoscenza, come quella che aveva avuto Simeon il Rosso.
Era stato lì che avevo errato per la prima volta, giusto? Avevo
dimenticato che in latino \emph{errare} significava `vagare' e non
`commettere un errore'. Mi ero allontanato barcollando dalla visione che
mio padre aveva della mia vita e -- come il peccatore nell'antica
preghiera -- ero caduto dalla via stretta in uno sfortunato inferno.
Avevo vagato, ma non mi ero perso. Avevo la mia via di uscita e
soprattutto avevo amici a cui importava di me quanto bastava per
irritarmi e per sentirsi feriti a causa mia. Inoltre, sospettavo di
essere vicino a uno dei Cielcin, e questo presupponeva una conoscenza
molto speciale, qualcosa che neppure l'antico Simeon aveva visto o con
cui aveva parlato.

Era qualcosa di completamente diverso.

