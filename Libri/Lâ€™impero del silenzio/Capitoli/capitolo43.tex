\chapter{Il conte e il suo lord}

Mi svegliai aspettandomi di essere in catene, ma non era così. Sedevo
accasciato in una grossa poltrona in una camera dalla fioca
illuminazione e con l'aria condizionata. Non riuscivo a ricordare di
essere mai stato tanto comodo e tanto dolorante nello stesso tempo.
Cominciavo a ricordare che il bastone elettrico non era stato delicato
quanto i paralizzatori dei Cavalli Bianchi, e mi sentivo ammaccato quasi
come quando quella banda di strada mi aveva quasi ucciso, a Meidua,
anche se ero lieto che questa volta non fossero necessari tutori
correttivi. Una rapida serie di movimenti rivelò che non c'erano ossa
rotte, e così procedetti a studiare l'ambiente circostante. Dopo
parecchi anni nel mondo del neon e della plastica prodotti su larga
scala proprio dei plebei, quella stanza era una rivelazione sibaritica,
sontuosa al di là dei miei sogni. Le pareti erano rivestite di pannelli
che non erano una stampa a imitazione del legno ma tek originale... così
tanto che doveva essere stato importato da fuori il pianeta. Il
pavimento di piastrelle minuscole era coperto di tappeti tavrosiani di
diverse tonalità di verde, oro e mattone che descrivevano scene di
caccia in uno stile senza tempo, e tendaggi di seta si agitavano davanti
a una porta a due battenti aperta, agitate da venti rallentati dal
fievole scintillio di un campo statico. Tutto appariva fatto a mano,
perché nel nostro mondo di perfezione data dalle macchine, dove perfino
le pietre preziose possono essere generate su ordinazione, l'artigianato
è il più grande dei tesori.

«Sei sveglio.» Quella voce era cavernosa, da basso profondo della
lirica, cupamente raffinata quanto i pannelli di legno sulle {pareti}.
La conoscevo, ma dove l'avevo sentita? «Devi perdonare i miei uomini,
lord Marlowe. Avevano l'ordine di proteggere il piccolo tesoro del
cantore Vas.» Il conte Balian Mataro entrò lentamente nel mio campo
visivo con un bicchierino da brandy in una mano massiccia. Il suo cuoio
capelluto riluceva dove era stato incerato di recente, nero come un
pezzo degli scacchi, e lui risplendeva in un completo verde pallido e
bianco opaco la cui giacca arrivava quasi a terra, chiusa da una larga
fusciacca di seta in crema e oro. «Anche se mi pare di capire che li
avevi quasi ingannati. A proposito, è stata una performance eccellente,
stavo proprio guardando le registrazioni.» Batté un pugno contro una
credenza in un applauso mentre la sua littore -- una donna muscolosa
dalla pelle scura quasi quanto la sua -- si andava a posizionare davanti
alle tende. «Devo però confessare di essere un po' confuso sul perché il
figlio di un arconte deliano faccia il mirmidone nel mio Colosso. Com'è
che ti fai chiamare? Had di Teukros?»

Era ovvio che sapesse. Mentre ero privo di sensi e in suo potere doveva
aver fatto testare il mio sangue dai suoi scoliasti e controllato il
risultato con il Registro standard dell'Alto Collegio. Sapeva che ero un
palatino, a quale Casato appartenevo e in che anno ero stato estratto
dalle vasche. Conosceva la storia della mia famiglia, chi fossero i miei
parenti, che ero in parte un pari dell'Impero perché il mio sangue era
remotamente imparentato con quello dell'imperatore. Mille storie, tutte
menzogne, mi affiorarono nella mente come preghiere. Cosa potevo dire?
Lui aveva letto il mio sangue, a questo punto non c'erano menzogne che
tenessero, per quanto pensassi di essere astuto. A volte, se sei davvero
molto sfortunato, c'è una sola risposta.

«No, Vostra signoria. È come dici, sono un palatino e il mio nome è... è
Hadrian Marlowe, di Delos.» Deglutii a fatica, perché quelle parole
suonavano strane e quasi dolorose sulla mia lingua. Ancora sofferente
per gli effetti del bastone elettrico, mi resi conto di non aver
risposto alla sua domanda, e fu solo dopo che quel grosso nobile ebbe
inarcato le sopracciglia con fare sardonico, con un tintinnare di catene
d'oro intorno al collo taurino, che aggiunsi: «È una lunga storia,
Vostra signoria.» Le sopracciglia non si abbassarono, e dovetti
ricordare a me stesso che quello era un palatino e che la pazienza era
per lui un linguaggio innato.

Mancando di alternative, consapevole della donna littore alla mia destra
che aveva muscoli simili alle corde di una frusta, e immaginando guardie
nascoste dietro gli arazzi con scene di caccia appesi sulle pareti
interne, con bruciatori al plasma puntati contro il mio petto, gli
raccontai tutto. Di Demetri, della clinica e della vecchia, di Cat e
della pestilenza, di Teukros e della lettera che Gibson aveva scritto
per me. Ci volle meno tempo di quanto immaginassi, venti minuti per
raccontare quasi tre anni della mia vita. Omisi l'avventura con gli
Umandh e ogni riferimento a effettive attività criminali perché non
intendevo confessare al conte di aver quasi assassinato una negoziante
plebea o i miei furti seriali. Ben presto ebbi finito, e dopo una breve
pausa domandai: «Vostra signoria non ha contattato Delos, vero?»

Balian Mataro smise di camminare avanti e indietro -- cosa che aveva
fatto per tutto il tempo in cui avevo parlato -- e si appoggiò a una
credenza contenente scintillanti bottiglie di cristallo piene di
liquori. «Avrei dovuto?» Decisi che la sua pelle era \emph{troppo}
scura... troppo scura esattamente nello stesso modo artificiale in cui
la mia era troppo chiara. Entrambi eravamo senza difetti, scolpiti come
eravamo in due tipi opposti di pietra.

Durante tutta la mia narrazione non gli avevo fornito il motivo per cui
avevo lasciato Delos, e lui non lo aveva chiesto. «No, Vostra signoria.»

«Loro... le mie guardie, intendo... mi hanno detto che hai
\emph{parlato} al Cielcin.»

Con un certo sforzo mi misi a sedere più eretto sulla sedia. Portavo
ancora gli abiti usati sul campo di addestramento e il sudore sfregava
al loro interno, secco e granuloso. «Sì, Vostra signoria.» Mi passai
stancamente le mani fra i capelli. «Il mio campo di studio primario
erano le lingue. Non sono assolutamente un maestro, ma posso parlare con
quella creatura, in caso di necessità.» Risi, un suono fievole e debole.

«Cosa c'è di tanto divertente?» Il conte posò il bicchiere vuoto sulla
credenza.

Scivolai in avanti sulla poltrona, scuotendo la testa mentre cercavo di
alzarmi. Il littore si tese, al suo posto vicino alle tende, ma rimase
dove si trovava. «Il Cielcin ha detto che la mia pronuncia era
orribile.»

Balian Mataro sorrise, accarezzandosi la folta barba lanosa che gli
copriva la mascella squadrata. «È tanto brutta?» Nonostante il mio
disagio mi sorpresi a sorridere ancora. Il conte si girò e si versò un
altro drink da una caraffa di cristallo. «Che altro ha detto?»

«Mi ha chiesto se morirà, signore.» A quel punto mi alzai, anche se
badai a rimanere a distanza dalla sua persona reale. Orientandomi in
modo da continuare a essere rivolto verso di lui pur tenendo l'enorme
poltrona fra me e il littore, continuai: «Gli ho risposto di sì.»

Lord Balian inclinò il bicchiere e bevve un lungo sorso prima di
replicare. «Ecco, non hai mentito.» Si umettò le labbra mentre una
strana espressione pensosa gli modellava i muscoli della faccia.

Pensando alla parata che avrebbero tenuto, quella che secondo le guardie
sarebbe culminata con la morte dei Cielcin, domandai: «Un trionfo,
Vostra eccellenza?»

«Per l'Efebeia di mio figlio.» Il nobile agganciò il pollice nella
fusciacca con un motivo cashmere e accennò oltre la porta di legno ad
arco, in direzione di quello che doveva essere il resto del palazzo.
«Compirà ventuno anni standard in settembre.»

Cinque mesi. Pensai al Cielcin -- a Makisomn -- intrappolato in quella
cella-fogna per cinque mesi. Non ero certo che io avrei resistito per
cinque ore, il che naturalmente era in quel momento la mia maggiore
preoccupazione. «Le mie congratulazioni. Devi essere orgoglioso di lui.»

«Certamente!» esclamò il conte, con enfasi. «Lui è mio figlio.»
Pronunciò quelle parole con forza accattivante. Decisi che era quello il
modo in cui un padre avrebbe dovuto parlare dei suoi figli. «Però non
hai davvero risposto alla mia domanda, lord Marlowe.»

Dietro la poltrona, con la schiena addossata a un angolo del salotto e a
una bacheca di vetro contenente antiche armi a proiettili. Accennai un
accurato inchino. «Chiedo scusa a Vostra signoria. Qual era la domanda?»

«Perché sei qui?» Prima che potessi replicare sollevò una mano massiccia
e aggiunse in tono piano, con quella sua voce da basso: «Ho capito
\emph{come} sei finito qui, ma quello che mi interessa di più è il
perché. Questo, e cosa ci stavi facendo, esattamente, nella mia
prigione.» Eccola infine... la temuta domanda mi si era abbattuta sul
collo come una spada e l'ombra del conte avrebbe potuto benissimo essere
quella di un boia... il che costituiva il problema. «Non ho nessuna
familiarità con il tuo Casato, e se questo è un atto di \emph{poine},
una qualche segreta vendetta, non conosco nessun motivo per cui...»

«Non è \emph{poine}, Vostra signoria» dichiarai semplicemente,
allargando le mani. «Mio padre mi ha venduto, e io sono fuggito.»

Il conte accolse la cosa senza problemi, con il sollievo che gli si
dipingeva sul volto, sotto la barba. «Ti ha venduto? A una baronessa?»
Quando non risposi né annuii, inarcò un sopracciglio. «A un barone,
allora? Bene, anche questo ha il suo fascino.» Sfoggiò un ampio sorriso
che mi indusse a ricordare il lord suo marito, l'esile Mandari dai
lunghi capelli neri.

Mi accasciai e sarei potuto crollare letteralmente sullo schienale della
poltrona che avevo davanti se non fosse stato per l'acciaio
aristocratico che sorresse la mia colonna vertebrale. «Non è affatto
così, Vostra signoria.» Il conte Mataro attese che proseguissi, con la
sua pazienza nobiliare che si faceva valere di fronte ai miei momenti di
laconicità. Quando il silenzio si prolungò al punto di frantumarsi
dentro di me, precisai: «Dovevo entrare nella Cappellania.»

Dimentico del suo momentaneo sollievo, il conte si immobilizzò. Grigio
in volto, riuscì a parlare solo dopo qualche piccolo verso annaspante.
«Nessuno sa dove sei?»

È fatta, pensai, fiutando un'opportunità, se non un vantaggio. «No, a
meno che tu non abbia contattato mio padre il lord.» Anche come concetto
astratto, la Cappellania stava facendo quello che le riusciva meglio:
incutere negli uomini il timore di Dio e della Terra. Ogni cosa che
avevo visto per tre anni -- ogni nuvola e tramonto, ogni angolo di
strada e ogni serva, ogni frammento di terra su questo mondo in
sovrappeso -- apparteneva a quel colosso che avevo di fronte. Quando
fosse morto avrebbero scolpito una sua statua per qualche tempio, per un
qualche mausoleo come quello nella nostra necropoli al Riposo del
Diavolo. Lo avrebbero ritratto con i piedi calzati di stivali piantati
su Emesh, a mostrare il potere che aveva detenuto in vita, un vero
potere. Avrebbe potuto ordinare la mia morte nell'arco di secondi, ed
era lì ridotto al silenzio al pensiero di un fantasma lontano mezza
galassia.

Esitò per un momento prima di rispondere, giocherellando con uno degli
enormi anelli che aveva alle dita. «Non ho ordinato di inviare nessuna
comunicazione via \foreignlanguage{italian}{qet}» disse quindi,
riferendosi all'aggrovigliata rete telegrafica che teneva insieme
l'Impero e l'universo umano. «Te lo chiedo di nuovo: qualcuno sa che sei
qui?»

«Come ti ho detto, la nave su cui sono arrivato era stata abbandonata,
Vostra signoria, ed ero diretto su Teukros, non su Emesh. Non avrei mai
potuto prevedere di finire qui.» Il littore mi stava osservando con
espressione dura, con la mano chiusa intorno all'impugnatura della spada
ad altamateria disattivata. «Naturalmente, la cosa più saggia da fare
sarebbe uccidermi» continuai, decidendo di rischiare. «Di nascondere
qualsiasi prova che sia mai stato qui.» Detto questo rivolsi uno sguardo
astuto al palatino per mettere in chiaro che niente avrebbe potuto
essere meno vero, come se il semplice dire quelle parole ad alta voce
potesse cancellare quell'idea.

Funzionò. I suoi occhi neri si socchiusero e lui contrasse la mascella
sotto la folta barba. «Pensi che sia tanto stupido, Marlowe?» Aveva
omesso il titolo, `lord'.

\emph{Ti ho agganciato}. A questo punto una dozzina di battute mi
danzarono sulla lingua ma le ricacciai indietro, arrivando a mordermi un
labbro per trattenermi dal sorridere, dall'esalare un sospiro di
sollievo. \emph{Nessuna esecuzione capitale quel giorno, non per me}.
Per un momento lottai per trovare una risposta adeguata, ma prima che
potessi parlare la porta si aprì, lasciando entrare un Mandari esile
come uno stocco: era il marito del conte, lord Luthor Shin-Mataro. «Hai
cominciato senza di me, mio signore?» Inarcò un sottile sopracciglio con
un'espressione tesa e accigliata, incorniciando la piccola bocca con
solchi profondi. Lord Luthor aveva la carnagione color bronzo e gli
zigomi alti familiari per chiunque avesse mai visto un plutocrate
interstellare; aveva gli stessi capelli fra il blu e il nero, gli stessi
incredibili occhi a mandorla verde foresta e lo stesso gelido
autocontrollo. Quegli occhi... erano una mutazione acquistata su un
mercato secondario? Anche il Consorzio aveva i suoi segaossa, i suoi
chirurghi e i suoi magi, che erano meno interessati ai limiti imposti
dalla Cappellania alle modifiche umane. Il colore contava poco, ma negli
occhi di Luthor c'era qualcosa da cui si intuiva che vedesse più degli
altri uomini, forse nella fascia ultravioletta o nell'infrarosso. Non
avrei saputo dirlo, né ero certo che la Cappellania avrebbe trovato da
ridire sui suoi occhi al punto da estirparli. So solo che mi turbavano.

«Il ragazzo si è svegliato prima di quanto tor Vladimir si aspettasse,
Luthor» rispose il conte, indicandomi con un ampio gesto che mise in
mostra una manica a ventaglio di fine tessuto. «Abbiamo deciso solo ora
che la linea d'azione migliore è decapitarlo e farla finita.» Le tende
si agitarono un poco, mosse dal lento scambio di temperature attraverso
il campo statico dovuto alla porta aperta.

L'esile extraplanetario impallidì. «Balian, non puoi farlo!»

Sul volto ampio del conte apparve un sorriso che si dissolse in una
risata profonda. «No, certo che no,» replicò, ridacchiando ancora «ma...
per la Terra, che espressione avevi sulla faccia!» Indicò continuando a
sorridere, prima di tornare a girarsi verso di me. «Hadrian Marlowe,
posso presentarti il lord mio marito, Luthor Astin-Shin-Mataro, in
precedenza dell'ufficio di Marinus del Consorzio Wong-Hopper e mio
ministro delle Finanze?»

Ricordando l'etichetta mi girai e parlai in perfetto mandar.
«\emph{Rènshu ni hěn rónxong shun}, zhu Luthor.» Era un saluto formale e
cortese, e nel proferirlo mi inchinai profondamente, quasi ad angolo
retto con il pavimento, un inchino che si adeguava alla posizione
elevata del mio interlocutore. Rimpiansi immediatamente quel gesto
perché il sangue prese a pulsarmi negli occhi e i lividi che avevo sul
collo e sulla schiena reagirono in modo orribile, tanto che dovetti
aggrapparmi alla poltrona per raddrizzarmi.

Ignorando cortesemente le mie difficoltà e mostrandosi chiaramente
sorpreso, lord Luthor inarcò le sopracciglia scolpite alla perfezione
mentre mi rispondeva in galstani, senza dubbio a beneficio di suo
marito. «Parli il mandar molto bene. Dove lo hai imparato?»

«Da ragazzo ho avuto uno scoliasta. Come ho spiegato al lord tuo marito,
il mio addestramento primario è stato nelle lingue. Parlo anche lo
jaddiano e il lothriano, il durantino e l'inglese classico, oltre ad
avere un'infarinatura di alcune delle lingue tavrosiane, soprattutto il
nordei e il panthai.»

Il Mandari emise un fischio di approvazione. «Tutto questo oltre alla
lingua dei Cielcin? Davvero notevole.»

«Infatti.» Il conte si accigliò. «Il nostro intruso è pieno di
sorprese.» \emph{Intruso}. Immaginai qualcuno più alto e misterioso,
probabilmente mascherato e avvolto in un mantello nero. Intuendo che non
era il momento di interrompere, mantenni un diplomatico silenzio e
cercai di visualizzare i miei genitori -- una coppia più tradizionale,
secondo alcuni standard fuori moda -- mentre avevano una discussione
come questa, ma la visione non volle formarsi. Continuavo a immaginarli
mentre cercavano di surclassarsi a vicenda nell'impersonare un
ghiacciaio. «Eri nell'arena per... cosa? Per guadagnare denaro?»

Mi illuminai, lieto di avere una domanda concreta a cui rispondere. «Per
pagarmi un passaggio per lasciare il pianeta» replicai, e quasi troppo
tardi aggiunsi: «Vostra eccellenza, sire.»

«Per andare dove?»

«Su Tavros, credo» dissi, decidendo sul momento per quella risposta. Al
di là di quei poteri politici che si inginocchiavano davanti agli altari
della Cappellania, fra i tecnocrati e i demoniaci al limitare della
galassia. Là sarei potuto sopravvivere a mio padre e a qualsiasi
interesse imperiale per il mio futuro. «Dovunque la Cappellania non mi
cercherebbe.»

Il Mandari scambiò un'occhiata con suo marito. «La Cappellania?»

«Il ragazzo è un seminarista, Luthor» spiegò il conte. Posò di nuovo il
bicchiere sulla credenza e si rivolse alla littore. «Camilla, per
favore, apri quelle dannate tende. Qui dentro è già abbastanza buio.» La
donna dallo sguardo duro salutò, riuscendo in qualche modo a farlo senza
distogliere lo sguardo dal nobile che le era affidato. Notevole. Fuori,
la notte stava calando su Borosevo e sull'oceano circostante, e il cielo
livido era tatuato dalle nuvole. Volgendo le spalle al marito, il conte
proseguì: «Abbiamo per le mani un assente ingiustificato.»

«Un assente ingiustificato?» ripeté lord Astin-Shin-Mataro, sgranando
gli occhi verdi più per paura che per interesse. «Da Komadd?»

Scossi il capo. Non avevo mai sentito parlare di Komadd. Senza dubbio
era un qualche mondo provinciale controllato dalla Cappellania. «Non
Komadd, Vesperad.»

«Vesperad?» ripeterono all'unisono il conte e suo marito, sconvolti, e
per una buona ragione: Vesperad era dove si riuniva il Sinodo stesso,
non c'era sede più importante della santa autorità della Cappellania
nell'arco di quarantamila mondi. Posso dire questo di mio padre... quali
che fossero i favori che aveva chiesto per farmi approdare al mio oscuro
appuntamento, avevano dato i loro risultati. La storia potrà dire quello
che vorrà di Alistair Marlowe, ma lui era un maestro quando si trattava
di politica. «Hai rifiutato una nomina al college Lorica?» continuò
Mataro. «Sei pazzo?»

«Non l'ho rifiutata, ne sono fuggito» lo corressi con fluida sincerità.
Tutta quella verità cominciava a inacidirmi la lingua, trasformandosi
nell'arida pressione di qualcosa di molto simile alla paura. Cosa ne
sarebbe stato di me? Cosa intendevano fare quei nobili? Quasi mi
aspettavo di vedere un servo in livrea fare irruzione nella stanza per
dichiarare che il lord arconte Marlowe era al telegrafo nella stanza
vicina ed esigeva la restituzione del figlio apostata.

Quando non successe niente del genere, valutai brevemente la possibilità
di lasciarmi ricadere sulla poltrona su cui mi ero svegliato e mi
trattenni solo per il timore di addormentarmi davanti al grande nobile e
al suo consorte a causa del mio stato di debolezza. «Vostra eccellenza,
non posso andare in quel posto.» Pensai a Gibson, al cathar calvo e
bendato che gli aveva tagliato il naso. Pensai agli schiavi mutilati del
Colosso, vestiti in modo da sembrare dei Cielcin, e a Ghen e Siran, e al
suono di carta strappata della carne lacerata. Quella sarebbe stata la
mia vita, il mio retaggio, ma non sarei stato io.

Luthor dai troppi cognomi socchiuse gli occhi di smeraldo mentre
studiava il mio volto, ma quando parlò si rivolse al marito. «Non può
rimanere qui, Balian.»

Il conte sollevò una mano per farlo tacere, e io mi affrettai a
interloquire. «Allora fatemi partire. Non è necessario che sia una buona
nave, Vostra signoria, o anche solo una nave. Mi basta un passaggio su
qualcosa di affidabile.» Fu un errore, come compresi non appena ebbi
pronunciato quelle parole.

Il grosso volto di Balian Mataro si contrasse in un cipiglio. «Non ho
l'abitudine di sprecare persone preziose, lord Marlowe.» Lanciò
un'occhiata al marito. «Che ne è stato delle guardie che hanno portato
questo problema alla nostra attenzione?»

«Attualmente sono nostre ospiti» rispose il suo consorte, rimuovendo un
granello invisibile da una manica grigia. \emph{Ospiti, come no}. Ero
pronto a scommettere che erano in una cella di detenzione.

«Falle trasferire a un nuovo incarico il più lontano possibile da qui...
magari su una delle lune. Un posto dove le loro parole siano solo
chiacchiere. Camilla!» Girò la testa di scatto in direzione della
littore, segnalando che venisse a raggiungerci. La donna attraversò la
stanza con passo regolare e pesante, e mio malgrado io mi irrigidii, ben
sapendo che nelle mie condizioni attuali non avrei potuto fare niente
contro qualcuno armato e schermato. Però non successe nulla, e il conte
proseguì: «Sii tanto gentile da dire a uno dei tuoi compatrioti nel
corridoio di mandare a chiamare immediatamente Vladimir e lady Ogir.»

Dama Camilla -- quella donna era di certo un cavaliere -- mi guardò in
modo espressivo dall'alto del suo naso patrizio geneticamente
rimodellato. «Ma, mio signore...»

Il conte le batté un colpetto rassicurante sul braccio. «Non ho nulla da
temere da parte di lord Marlowe, certo non nei trenta secondi che ti ci
vorranno.» E la spinse con gentilezza verso la porta da cui era entrato
lord Luthor. «Vai.» Una volta che fu uscita, Balian Mataro mi chiese:
«Quante lingue parli, lord Marlowe? Quattro? Cinque?»

«Quasi otto, Vostra signoria» replicai, non vedendo la rilevanza di
quella domanda dato che stavamo parlando di astronavi. «Anche se sono
cinque quelle che parlo correntemente.» Era una lieve esagerazione
perché il mio durantino era abbastanza buono per ingannare qualche
povero inserviente delle cucine ma non ero certo nelle condizioni di
poter prendere il tè con uno dei consoli di quella lontana repubblica.

Il conte guardò suo marito con uno strano bagliore negli occhi scuri.
Approvazione? Trionfo? Qualsiasi cosa fosse dovette turbare lord Luthor,
perché chiese: «Cosa stai pensando, Balan?»

«E quanti anni hai, lord Marlowe?» continuò lord Balian, sorridendo
sotto la barba.

Esitai, sul punto di rispondere `ventuno anni standard'. Era probabile
che fosse il numero esatto, ma non potevo esserne certo perché il mio
Efebeia era passato senza nessuna celebrazione mentre vivevo sulle
strade di Borosevo. Inoltre conoscevo la data locale ma il calendario
standard contava solo per quanti viaggiavano al di là di quel mondo o
facevano affari con i marinai provenienti dal Buio. Era da anni che non
vedevo una data standard o avevo accesso a una sfera dati nobiliare. «Ne
avevo diciannove quando ho lasciato casa, Vostra signoria,» dissi infine
«ma non conosco l'attuale Data Stellare Imperiale.»

«Sedici uno settantuno zero quattro.»

Se fossi stato impegnato a bere avrei potuto sputare tutto fuori.
Trentacinque anni. Erano passati trentacinque anni standard umani da
quando ero entrato nella capsula criogenica a bordo
dell'\emph{Eurynasir}. Per la Terra e l'imperatore, Crispin doveva avere
quasi cinquant'anni, supponendo che non avesse mai lasciato casa. Non
ero più il fratello maggiore, proprio come avevo detto ad Haspida. Mi
ero aspettato un salto di tredici anni fra Delos e Teukros, ma...
trentacinque? È un dato di fatto dei viaggi spaziali l'essere lasciati
indietro, perché la freccia del tempo vola in una sola direzione.
Nonostante l'ubiquità di questo fatto nella vita dei palatini, chiusi
gli occhi e mi costrinsi a rimanere in silenzio.

Il conte mi posò una mano su una spalla. «Stai bene?»

Invece di rispondere dissi: «Ho ventitré anni, Vostra signoria.»

«Come sai, ho un figlio.»

«E una figlia, Anaïs, entrambi un po' più giovani di te» interloquì lord
Luthor.

Il conte Mataro riprese la parola. «Hanno ben pochi compagni della loro
età, perché ho il controllo di quattro soli Casati minori, due dei quali
di esuli provenienti da fuori. Intendo metterti sotto la tutela della
corte e vorrei che insegnassi le lingue ai miei... ai nostri figli.»
Sorrise nel lanciare un'occhiata a Luthor. «Sarebbe loro utile un po' di
esperienza pratica.»

Luthor si irritò. «Credo che sia un errore, Balian.»

«Quest'uomo è un lord palatino, Luthor. Un pari dell'Impero, della
costellazione Victoria. Sangue antico.» Balian Mataro inarcò un
sopracciglio per sottolineare quelle ultime due parole, poi tornò a
rivolgersi a me. «Naturalmente non userai il tuo vero nome, ma per come
la vedo io è meglio tenerti al sicuro qui che mandarti via.»

«Nel caso che mi vengano a cercare?» domandai. «Allora potrò dire: `No,
lord inquisitore, questi uomini gentili mi hanno salvato.' Potrò essere
il vostro scudo.» Lo vedevo con chiarezza... sarei stato un loro
prigioniero. Tenuto bene, ma comunque un prigioniero. Sentii le pareti
che mi si chiudevano intorno e compresi di essere con le spalle al muro.
`Sangue antico' aveva detto il conte. Quale che fosse il suo titolo
nobiliare, il conte Mataro era il signore di un mondo provinciale e
retrogrado, un lord con un piccolo nome, mentre io ero un pari
dell'Impero perché la mia famiglia poteva far risalire il proprio sangue
ad Avalon, ai primi giorni dell'Impero e allo stesso, vecchio, William
Windsor. Di certo era una cosa che non potevano ignorare.

«Hai da obiettare?»

«È una gabbia, Vostra eccellenza.»

«Molto migliore di quella che meriti» scattò Luthor, parole dirette più
a lord Mataro che a me. «Balian, devo protestare.»

La porta si aprì di nuovo, lasciando entrare uno scoliasta e una donna
che portava l'abito grigio di una logoteta imperiale. Entrambi si
inchinarono, mostrando di percepire l'improvvisa tensione presente nella
stanza. La donna -- chiaramente quella che fra i due deteneva un grado
superiore -- chiese: «Ci hai fatti chiamare, mio signore?» Era
attempata, e il volto patrizio rivelava i sottili potenziamenti
chirurgici da cui capii che era una popolana di nascita. I capelli che
stavano ingrigendo erano tagliati corti sopra un volto color rame e i
suoi occhi -- anch'essi grigi -- parevano guardare attraverso tutto ciò
che fissavano. Appresi in seguito che il suo nome era Liada Ogir, Alta
Cancelliera di Emesh e il potere dietro il trono di Mataro.

Il conte mi presentò brevemente e spiegò la mia situazione. «Avrà
bisogno di false credenziali nell'arco di un giorno. Qualcosa che non
faccia sollevare troppe sopracciglia. Vladimir ti consiglierà. Magari
qualcosa di patrizio che possa attirare di meno l'attenzione, non
credi?»

«Certamente, Vostra eccellenza.» Era entrato anche un drappello di
cinque guardie, opliti in armatura tinta di verde e con un lungo manto
bianco che strisciava sul terreno alle loro spalle. Non erano muniti di
lancia come i legionari imperiali o quelli di mio padre e portavano
invece al fianco una spada lunga di ceramica e avevano un disintegratore
a fase assicurato alla coscia... alternative migliori per operare
all'interno. A giudicare dalle dimensioni di quella camera, il castello
di Borosevo era un posto che offriva meno spazio di manovra rispetto al
Riposo del Diavolo. «Che nome gli dobbiamo dare, mio signore?» chiese la
cancelliera Ogir.

Lord Mataro mi squadrò da testa a piedi. «Nell'arena lo conoscono già
come Had, quindi Hadrian va bene. Hadrian...»

«Gibson» interloquii, senza neppure pensarci su. «Hadrian Gibson.»

Colsi lord Luthor a fissarmi con occhi roventi e chinai cortesemente il
capo. Avrei voluto spazzare via quell'espressione dalla sua faccia
troppo avvenente, ma mi controllai.

«Un'ultima domanda, M Gibson» disse il conte, sollevando una mano mentre
gli opliti si preparavano a scortarmi fuori dallo studio. «Non mi hai
detto perché sei andato a vedere quel Cielcin.»

Quella domanda mi lasciò perplesso per un momento. Non avevo una
risposta pronta, quindi mi tormentai un labbro mentre riflettevo.
«Volevo scoprire se era un mostro, mio signore.» Così tante sfumature in
quel diverso modo di interloquire, da parte mia e sua. Quanti dettagli
sottintendevano. «Volevo vederlo.»

L'altro palatino annuì con fare solenne. «E lo era? Un mostro, intendo.»

«Non lo credo, mio signore» replicai. «Aveva paura.»


