\chapter{Mera umanità}

Cid Arthur trovò più della povertà quando fuggì dal palazzo di suo
padre. Trovò anche la malattia, come feci anch'io. La necrosi grigia
imperversava su Emesh ormai da alcuni anni, importata da qualche
mercante poco scrupoloso venuto dall'esterno. I nativi non avevano
nessuna immunità e i microrganismi li divoravano come carta e spargevano
l'infezione per le strade. Io ero un palatino, ero immune, che la Madre
Terra abbia pietà di me.

Vi siete mai chiesti come sarebbe trovarvi nel bel mezzo di un'epidemia
senza esserne toccati? Mi sentivo come uno spettro. La biochimica quasi
aliena del mio corpo -- eredità di decine di generazioni e di
ricombinazioni genetiche per un valore di milioni di marchi imperiali --
mi preservava da ogni ulcera purulenta, da ogni attacco di necrosi,
dalla tosse emorragica. Sembra una benedizione, ma non lo è. Non è una
benedizione guardare gli altri morire, e lo è anche meno vedere chi ami
che si consuma. Quando ho cominciato questo resoconto, pensavo di
saltare questa parte, tanto dolorosa per la perdita di Cat, ma mi
sbagliavo. Lei conta. Deve contare.

Cat resistette più a lungo della maggior parte degli altri, più di
quanto si sarebbe potuto credere in qualcuno tanto minuto.

Puzzava nel caditoio dove l'avevo lasciata, su un alto costone al di
sopra della galleria principale. Eravamo al di sotto della borsa della
Strada Alta, e quindi al di sopra del livello dei canali. Era scesa la
notte, e la luce delle due lune di Emesh -- una bianca e una verde a
causa del nascente progetto di terraformazione che si andava
{estendendo} sulla sua superficie -- si insinuava su per il caditoio fin
dove la ragazza giaceva su un umido bancale di cartone. Al di sotto
dell'odore di muschio e di rifiuti che marcivano potevo sentire quello
dolciastro della malattia, la necrosi delle ulcere purulente. Si poteva
avvertire quell'odore in ogni strada, in ogni canale, su ogni tetto
della città. Prima di salire la scala di acciaio inossidabile fino a
dove l'avevo lasciata dovetti soffermarmi per un momento alla sua base,
abbastanza a lungo da chiamare a raccolta le forze, quietare lo stomaco
e calmare i nervi.

Eravamo insieme -- compagni nel crimine -- da poco meno di due anni
standard, e adesso sapevo che stava finendo tutto. Lo sapevo da
settimane.

Cat tremava sotto una sottile coperta che un tempo era stata una tenda
in uno stabile di appartamenti abbandonato. Avevamo passato una
settimana in quella casa, mentre si sgretolava fino a crollare nel mare,
giocando da quei barboni che eravamo a fingere di vivere come gente
normale. Se lo avesse voluto, Cat si sarebbe potuta procurare un lavoro,
ma io ero condannato, perché qualsiasi lavoro, anche al livello più
basso garantito dal ministero dei Servizi Sociali del conte avrebbe
richiesto di schedare il mio sangue. Avrebbero voluto esaminarlo per
individuare eventuali rischi per la salute, difetti congeniti,
dipendenza da droghe e deficienza mentale... qualsiasi cosa pur di
negarmi un lavoro onesto... e avrebbero scoperto chi ero per poi
spedirmi in una cella della torre ad attendere notizie e un inviato da
parte di mio padre. Cat e io eravamo stati felici durante quella
settimana; felici, nudi e puliti. Il disegno di giacinti viola, che
sulla finestra infranta era apparso bello e luminoso, adesso giaceva su
di lei come una ghirlanda funebre, ma non era morta, non ancora.

E non si accorse di me. Invece continuò a borbottare nel sonno, tremando
come una fiamma di candela. A Meidua, al Riposo del Diavolo, non avevo
mai conosciuto la malattia. Quando ero piccolo la mente di mia nonna
aveva ceduto, ma del resto lady Fuchsia Bellgrove-Marlowe aveva avuto
quasi settecento anni. Mio padre era suo figlio, proveniente dalle
stesse vasche da cui ero stato prelevato io. Cat aveva diciotto anni,
era più giovane di quanto io lo fossi stato quando avevo lasciato Delos,
quando la mia vita era cominciata davvero, e la sua esistenza era
finita. Le medicine {adatte} scarseggiavano, quindi avevo speso i soldi
che avevamo per acquistare compresse e nuove bende. Avevo visto i
notiziari, le immagini di splendide conduttrici che dagli schermi
disposti agli angoli delle strade dichiaravano che la malattia si stava
dimostrando resistente al trattamento con antibiotici. Intere porzioni
della città erano state isolate, i canali venivano dragati per
recuperare i corpi che cominciarono a essere bruciati nelle piazze
cittadine quando gli obitori giunsero al massimo della capienza.

«Ti ho portato una zuppa» dissi, posando la tazza di carta sulla pietra
vicino alla sua forma dormiente. Ormai il liquido era freddo. «Niente
carota, lo giuro.» Tirai indietro la tenda e arricciai il naso di fronte
alle macchie fra il verde e il marrone delle fasciature. Cat si mosse ma
non si svegliò. «Al notiziario dicono di ritenere che l'epidemia stia
completando il suo corso, che si stia esaurendo. Ho sentito un uomo dire
di pensare che la pestilenza fosse un'arma dei Cielcin...» La voce mi si
spense in un qualche corridoio della mia anima e sedetti a lungo in
silenzio. «Vorrei sapere come aiutarti meglio» dissi infine,
tormentandomi un'innocua crosta sull'avambraccio.

Cat continuò a non rispondere. Le posai una mano sulla fronte,
avvertendo la malattia come un fuoco sotto la pelle, tanto a indurre ad
aspettarsi che dentro di lei ci fosse del magma e non sangue. Sapevo che
non le rimaneva molto, un giorno o due, una settimana, non di più. Non
era giusto. Cominciai a rimuovere le bende su un braccio rivelando i
tricipiti magri e consumati, la pelle bruna che si era fatta grigia e si
era coperta di vesciche di liquido verde e giallo. Gettai da un lato la
benda rovinata e aprii un pacchetto per applicarne una nuova intrisa di
medicinale. Non trovando le parole che mi servivano canticchiai mentre
lavoravo, fasciandole le ferite al braccio, alla coscia e al petto.

Lei non si destò e la zuppa rimase intatta, con il poco calore che
poteva aver avuto che si disperdeva nell'aria tiepida e immobile. Sotto
di noi, l'acqua scorreva in un rivolo nel canale, qua e là i tubi sopra
la nostra testa facevano gocciolare la condensa sul terreno con le gocce
che scandivano i secondi dell'orologio senza tempo della natura. Come
faccio ancora spesso, ripensai al funerale di lady Fuchsia e a quello di
zio Lucian. Non ci sarebbe stata una processione funebre per Cat, niente
canopi, nessuno che {rimuovesse} i suoi organi vitali o bruciasse la sua
carne. Nessun vero rito funebre. Niente ceneri per il Mondo Natale o una
lanterna votiva liberata nel cielo.

«Had?» La parola suonò infinitesimale come un angstrom, la voce fievole
come il fruscio di una pagina girata.

Le strinsi una mano come avevo fatto migliaia di migliaia di volte.
«Sono qui.»

«Perché... qui?» gracchiò, dopo un secondo infinito.

Le sopracciglia mi si aggrottarono di loro iniziativa, e le parole mi
sfuggirono spontanee. «Cosa intendi, mi chiedi perché sono qui?» Lei
annuì debolmente, come a rispondermi. «Dove altro dovrei essere?»
Sorrisi e cercai di ridere. «Su questo pianeta non mi piace nessun
altro.»

La sua risata si trasformò in un attacco di tosse, e le sostenni la
testa mentre la saliva tinta di rosa schizzava le bende che coprivano il
suo petto rovinato. Mi morsi le labbra per ricacciare indietro le
lacrime, sperando -- quasi pregando -- che smettesse. «Mi dispiace»
sussurrò dopo un momento.

«Non ti scusare» replicai, scuotendola gentilmente e allontanandole i
capelli filacciosi dalla fronte madida di sudore. «Non ti scusare.
Presto starai bene. Vedrai. Io ti aiuterò.»

Lentamente -- così lentamente -- sollevò una mano per posarla contro la
mia faccia ed emise un verso come per farmi tacere. «Non devi rimanere»
mormorò, con le labbra che mostravano i vuoti dove i denti le erano
caduti. «Non manca molto...»

«Non lo dire.» Cercai di sorridere, ma mi resi conto che era
un'espressione sofferta. «Starai meglio.» Sapevamo entrambi che mentivo.
Era già quasi un cadavere, i suoi occhi un tempo pieni di fuoco erano
annebbiati. Pensai che uno di essi avesse perso la vista o fosse al di
là della capacità di mettere a fuoco. Con quanta rapidità era cambiata.
Settimane prima -- solo settimane prima -- era stata sana e integra. Chi
era questo spettro?

«No.» Scosse il capo. «Promettimi... promettimi un cosa.»

«Starai bene!» insistetti, aiutandola a riadagiare la testa sui pezzi di
stoffa appallottolati che facevano da cuscino.

Chiuse una mano intorno alla mia gamba. «Promettimi che non lascerai che
mi brucino.» Sapevo che si riferiva ai grandi roghi, ai corpi
ammucchiati nelle piazze.

Crediamo che la nostra vita sia una cosa coerente, che abbia un
significato, una direzione, una trama. Pensiamo di avere uno scopo come
ce l'ha l'attore di un dramma, e credo che questa sia l'anima della
religione, il motivo per cui così tante persone che ho conosciuto --
perfino mio fratello -- credono che il mondo debba essere controllato,
che l'universo vada pianificato e protetto. Quanto è confortante
immaginare che ci sia una ragione per tutte le cose. Milioni di teologi
e di magi, i preti del culto di un migliaio di dèi morti, hanno
impartito questa lezione. Cat mi ha insegnato un'altra cosa, morendo in
quel caditoio senza nessuna ragione, e adesso sono più saggio, ma so che
indipendentemente da quello che dissi non avrei potuto aiutarla. Non
potevo neppure morire con lei.

Potevo soltanto guardarla morire.

«Raccontami...» Per un momento perse le parole, e forse anche la
lucidità, e per quel momento il solo suono a parte il gocciolio e il
sommesso gorgoglio dell'acqua fu il suo respiro umido e irregolare.
Prima che potessi muovermi, prendere dell'acqua o lo straccio che usavo
per pulirle la faccia, continuò: «Raccontami una storia, vuoi? Per
l'ultima volta.»

Le mie dita trovarono quelle deboli di lei, si intrecciarono con esse.
«Non dovresti parlare in questo modo.» Non rispose e distolse il volto
sottile. Aveva finito di discutere con me. Rimanemmo a lungo seduti in
silenzio, mano nella mano, mentre guardavo la luce mista delle due lune
che colava nel caditoio e aveva il colore chiaro della giada. L'altra
mano mi si posò su un angolo delle tende con il disegno di giacinti. Le
sue coperte. Il suo sudario. Ricordai come le avessimo strappate dalla
parete nel fervore del momento, e come Cat le avesse rubate quando i
prefetti avevano sfondato le porte in risposta alla denuncia della
nostra presenza abusiva. Quella settimana, quella settimana perfetta...
risaliva a due mesi prima?

No, non erano neppure due.

«D'accordo.» Trassi un respiro rantolante e lo trattenni perché non mi
uscisse come un singhiozzo. «Ti racconterò una storia.» Mi parve che
passasse un anno, o un secolo, prima che scegliessi una storia per lei
come avevo fatto innumerevoli volte. Era una che aveva già sentito e che
conoscevo bene quasi quanto quella di Simeon. «Una volta, su un'isola
lontana dalla Terra, ai margini dello spazio inesplorato, c'era una
città di poeti. In quei tempi l'Impero era giovane e gli ultimi
Mericanii erano stati abbattuti.

«La città dei poeti era stata costruita come un rifugio, un posto dove
gli uomini si potevano nascondere dalla Guerra di Fondazione e comporre
la loro arte in pace. La città aveva una sola legge: nessuno poteva
usare la forza contro un altro. Di conseguenza essa fioriva ed era
abbellita da tutti gli artisti che dimoravano fra le sue mura e
prosperavano nella loro amicizia.

«Tutti tranne Kharn.

«Lui non aveva scelto quella città come sua casa, ci era nato, figlio di
un grande poeta. E come spesso i figli di grandi guerrieri non sono essi
stessi guerrieri, così lui non era un poeta. Sognava di essere un
soldato, un eroe come quelli nelle opere epiche che la sua gente
componeva, ma gli altri non ne volevano sapere. `Qui non abbiamo bisogno
di soldati e neppure del fardello delle armi,' affermavano i poeti
`perché siamo lontani dalla Terra e le mura della città sono forti.'

«`Chi non vive di spada perirà di spada' insisteva Kharn, perché era
questo che dicevano i poemi. I poeti però non credevano alle loro stesse
parole perché pensavano che le storie fossero sciocchezze che potevano
comandare. La verità però non è né un'opinione né la sua schiava, e
giunse il giorno in cui il cielo fu oscurato dalle vele. Erano arrivati
gli Extrasolari, uomini simili a mostri del Buio, i figli dei Mericanii
sulle loro navi dall'alberatura nera. Ed essi bruciarono i poeti nella
loro città.

«Tutti tranne Kharn.»

Feci una pausa per allontanare i capelli dalla faccia di Cat e
tamponarle la fronte, poi ripresi: «Kharn combatté contro di loro e i
Glorificati -- che sono i re degli Extrasolari -- riconoscono solo la
forza. Di conseguenza lo risparmiarono anche se estrassero il cuore alla
sua gente e usarono i corpi come equipaggi delle loro ignobili navi. Lo
risparmiarono, e lui visse con loro per molti anni, saccheggiando altre
città, altri mondi.»

Non so per quanto tempo parlai o quanto a lungo le tenni la mano.
Raccontai tutta la storia, narrando come Kharn Sagara avesse nutrito la
vendetta nel suo cuore per tutto quel tempo, come avesse messo i
Glorificati uno contro l'altro, uccidendo i loro capitani e prendendo
lui stesso il comando della nave per poi fare rotta verso il loro mondo,
il gelido Vorgossos, e la sua stella morta. Le dissi come si fosse
impadronito del pianeta, incoronandosi re di quel mondo oscuro e gelato.
Era la storia contenuta nel libro che Gibson mi aveva dato, \emph{Il re
	con diecimila occhi}. Non era una storia felice, e neppure breve.

In un qualche momento nel mezzo della narrazione, le dita di Cat si
rilassarono, facendosi sempre più fredde. Non piansi e non interruppi la
storia. Ne avevo avuto abbastanza delle lacrime e a lei non sarebbe
piaciuto se mi fossi interrotto. Invece, strinsi quella fragile mano, la
baciai e dissi: «Fine.» Solo che non lo era. La cosa strana delle fini è
che niente finisce finché i soli non si consumano e tutto si fa gelido.
Cambiano soltanto gli attori.

Anche se la storia di Cat era finita, il sole stava sorgendo e
prometteva un altro giorno dell'estate eterna di Emesh. Avvolsi il corpo
nella tenda punteggiata di fiori. La malattia l'aveva ridotta pelle e
ossa, ed era leggera fra le mie braccia. Non la bruciai. La trasportai
lungo vicoli secondari, gallerie di accesso, passerelle semisommerse che
correvano al livello dell'acqua lungo alcuni canali laterali. Non era
giusto che se ne fosse andata tanto presto, così giovane. Non lo era. La
deposi a riposare nell'acqua, come nella storia del marinaio fenicio, e
appesantii il fragile corpo con alcune pietre. Non ho mai ritrovato quel
posto, non sono mai tornato ad accendere una lanterna votiva e a mandare
una preghiera per la sua anima verso il cielo e la svanita Terra.

Veramente solo, mi allontanai e tornai al mondo dei malati e dei
viventi.

La mia storia non era ancora finita.

