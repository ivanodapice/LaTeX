\chapter{Il mondo più vasto}

Una vita rude all'aperto ha un significato alquanto diverso se un membro
del tuo gruppo è un cavaliere. L'alloggio di sir Elomas nel nostro
piccolo accampamento era una struttura prefabbricata a un solo piano con
uno spazio interno di circa millecinquecento piedi e l'esterno che mi
ricordava un poco lo studio del conte a Borosevo, tutto ceramica bianca
e vetro nero, anche se segnato e infangato da tutto il tempo in cui era
stato sferzato dalle tempeste estive. Tanto Binah quanto Armand erano
bloccate in orbite polari, quindi le maree a quella latitudine così
meridionale erano molto più forti delle onde gentili della quasi
tropicale Borosevo. Passarono le settimane, nelle quali seguii Valka e
sir Elomas attraverso parecchie miglia di gallerie, risultando essere
più di impiccio che di utilità.

Quando avevamo lasciato la tenuta della sua famiglia, a Fonteprofonda,
Elomas aveva portato con sé tre servitori. Il primo era il suo giovane
nipote Karthik Veisi, un ragazzo di circa quindici anni che fungeva da
suo scudiero. La seconda era una donna del posto, la sua cameriera, e
infine c'era un cuoco extraplanetario che proveniva dal lontano Asherah.
Quando rientrammo al campo Karthik stava sistemando sul basso tavolo un
piatto di pesci arrostiti interi con contorno di pomodori e di erbe
aromatiche. Per quanto piccola, la casa del cavaliere era riccamente
arredata con una cucina e un'unità doccia separate dalla sala da pranzo
principale che fungeva anche da studio. I pavimenti erano coperti da
tappeti tavrosiani spessi due pollici, con disegni di mandala blu e
rossi. In gioventù sir Elomas si era spinto lontano nei suoi viaggi, da
Jadd e dal sistema esterno di Perseo allo Sperone di Orione e al cuore
dell'Impero, fino alle colonie del Sagittario e di Centaurus,
percorrendo un vasto arco attraverso il quadrante insediato della
galassia prima di ritirarsi su Emesh presso la famiglia di sua nipote,
vicino al nucleo della galassia e ai confini dello spazio umano. Che
vita doveva aver avuto...

Il vecchio cavaliere tirò indietro una sedia per Valka e si mosse con
abilità per fare lo stesso per tor Ada prima che la scoliasta ne potesse
prendere una da sé. Soltanto dopo che si furono sedute e che io ebbi
occupato un posto libero si sedette a sua volta, dicendo: «Karthik, il
vino, per favore.»

Avevamo preso l'abitudine di cenare in quel modo; Elomas, Valka e Ada
come membri anziani della spedizione di scavo e io come loro onorato
ospite. Ogni sera ci sedevamo in quel modo e passavamo un paio d'ore a
conversare prima di dirigerci ai nostri rispettivi alloggi.

«Il cibo ha un'aria splendida» commentai, prelevando un panino marrone
dal cestino del pane. «Ancora grazie.»

Sir Elomas si versò una tazza di tè da una teiera di porcellana rossa
prima di servire anche tor Ada. «Ho sempre creduto che il cibo sia fatto
per essere condiviso» replicò, una frase più o meno simile a quella che
aveva detto ogni volta che ci eravamo seduti per mangiare insieme.
«Avanti, servitevi.» Ada si lanciò in una descrizione della giornata di
lavoro, che in quella fase iniziale comprendeva soprattutto il ripulire
le parti che erano state inondate.

«La verità è che ho paura di essere poco più di un turista, sir Elomas»
commentai. «Come ben sai, la mia esperienza è limitata alle lingue e
allo stare fra i piedi a tutti.» Questo strappò una breve risata al
cavaliere e alle due donne.

«Sciocchezze, caro ragazzo!» tuonò Elomas. «Sciocchezze! Maros mi diceva
che il tuo aiuto è stato essenziale nel mettere in funzione le pompe
della Galleria C. `Non avremmo potuto farlo senza Marlowe' ha affermato.
Ti do la mia parola di Redgrave.»

Sorrisi nel procedere con cura a rimuovere la pelle del pesce che
Karthik mi aveva messo nel piatto. «Sei gentile a dirlo, ma sono poco
più di un operaio sopravvalutato.» Posai il coltello e nascosi la mia
frustrazione bevendo un sorso d'acqua. «Comunque ci tengo a ribadire
quanto sia grato di essere stato invitato qui.»

Il vecchio cavaliere posò la tazza del tè e procedette a rimuovere di
netto la testa del pesce senza tante cerimonie. «Hai di certo ravvivato
le cose, e dopo quello che hai fatto a quel bastardo di un prete...»
Scosse tristemente il capo. Il rancore presente nel suo tono sorprese
visibilmente tor Ada, che impiegò un momento a ritrovare la sua abituale
impassibilità di scoliasta. «Ai miei tempi ero un duellante, lo sapevi?
C'è stata una volta in cui ero ospite di un ministro mandari che
proveniva da una qualche ditta di bioingegneria -- sono lieto di dire
che adesso è morto -- che era specializzato in esemplari unici di
omuncoli. Devi capire che le concubine erano il principale prodotto di
un'industria del genere.» Rabbrividì, mentre io lasciavo che divagasse.
«Vedi, temo di aver offeso durante la cena un membro anziano del suo
staff. Era solo una piccola battuta sulle... vogliamo dire sulla
preferenza che quell'uomo aveva per il suo lavoro?»

«Il suo lavoro, signore?» chiese il giovane Karthik, sedendosi in fondo
al piccolo tavolo con il suo pesce.

«Naturalmente intendevo sottintendere che stava clonando sé stesso!»
spiegò sir Elomas, rivolgendo una seria occhiata a tutti noi. «Un vizio
repellente, ma i Mandari commerciano in vizi repellenti. Come potete
immaginare, il tizio mi ha sfidato a duello e, ecco... io sono qui,
quindi...» Infine recise la testa del pesce attraversandone
rumorosamente la spina dorsale prima di procedere ad aprire a ventaglio
la carcassa, rivelando il ripieno di pomodori, spinaci e fine formaggio
bianco che occupava il posto dove c'erano stati gli organi. «Inutile
dire che il ministro non ha visto di buon occhio il fatto che avessi
ucciso il suo sottoposto, e quando la mia accusa di autosodomia è
risultata essere vera, ecco...»

«Ha cercato di ucciderti?» domandò Valka, che aveva portato a termine --
più delicatamente − il lavoro di aprire il suo pesce, e sorrise al di
sopra dell'orlo del suo bicchiere d'acqua. Come me, detestava il tè, e
questa era solo una di quelle minuzie biografiche che -- stupidamente e
a livello inconscio -- mi facevano sentire che eravamo più simili che
diversi.

Elomas annuì allegramente. «Veleno! Riuscite a immaginarlo? Così
pittoresco! È un bene che tu sia qui, Marlowe, sano e salvo. I preti
amano il veleno e la vecchia Ligeia ha la memoria lunga.»

Quando finii di mangiare Karthik si alzò e procedette a portare via i
miei piatti. Avevo visto di rado uno scudiero così coscienzioso. Una
volta rimossi i piatti mi rivolsi a Valka. «Da quanto tempo siete qui?»

Lei finì di inghiottire un boccone -- mangiava più lentamente del resto
di noi -- e replicò: «Da quattro anni locali, ma le inondazioni ci
interrompono.»

«Fin da quelle spaventose tempeste nel '68, vero?» chiese Elomas,
facendo una smorfia, mentre aggrediva il suo pesce. «Brutte, quelle
tempeste. Da allora la griglia di alimentazione di Borosevo non è più
stata la stessa.»

Valka finì di masticare prima di rispondere. «Sì, esattamente.»

«Ma questo pianeta è colonizzato da quasi un millennio. Di certo qui non
rimane più niente da scoprire.» Quello era un pensiero che mi disturbava
da quando eravamo arrivati e avevo passato più tempo ad aggirarmi per
Calagah. Nelle settimane appena trascorse non avevo visto quasi niente
degno di nota fra le rovine a parte le stesse sale nere. Era un posto di
fantasmi, senza una cultura visibile che lo abitasse.

Tor Ada si prese la libertà di rispondermi: «Sì, certo, è in mani
imperiali da un millennio, ma prima apparteneva alla Compagnia Normanna
Unita di Emesh. Solo la Terra e l'imperatore sanno cosa abbiano portato
via o venduto.»

«No» intervenne Valka, permettendo infine che i resti del suo pesce
venissero portati via. «Non ci sono documentazioni di manufatti
secondari in nessuno dei siti costruiti dalla Quiete.» Si schiarì la
gola e per un momento scivolò di nuovo nella sua lingua nativa
tavrosiana.

«Che si suppone siano stati costruiti da loro» la corresse Ada,
sollevando un dito. «Non possiamo confermare l'ipotesi della Quiete, a
causa della regolamentazione da parte della Cappellania di tutti i dati
relativi a xenobiti estinti all'interno dei suoi protettorati. Potremmo
non sapere mai se è vero, e anche tutto quello che scopriamo verrà
sequestrato non appena la Cappellania riuscirà a metterci sopra gli
artigli.» Un lieve cipiglio si dipinse sul suo volto dichiaratamente
patrizio. «Calagah è un sito minore e la Cappellania non sembra disposta
a impegnare la presenza di un guardiano qui fuori nel Velo. Costa
troppo. Questo però significa anche che considera questo posto un
rischio minore, dal punto di vista {teologico}.» A quel punto mi lanciò
un'occhiata in tralice, quasi timorosa che la denunciassi come eretica,
ma io le sorrisi con fare incoraggiante. «La loro gente ha passato al
setaccio Calagah nel primo secolo di occupazione. Ci tollerano perché
sanno che la nostra attuale spedizione è inutile, e non ti sarà permesso
di portare via nessuna annotazione o documentazione se non quelle nella
tua memoria.»

Indirizzò quelle ultime parole a Valka che si limitò a sorridere in quel
suo modo misterioso -- come se lei e Ada condividessero un qualche
scherzo segreto -- e tamburellò con le lunghe dita sulla superficie di
legno finto del tavolo. «Ogni sito della Quiete che ho visitato era
vuoto, quando è stato trovato. Probabilmente non c'era già niente che i
Normanni potessero saccheggiare, a parte vendere i biglietti per poter
visitare le gallerie.»

«Davvero!» convenne sir Elomas. «Di certo i Normanni sapevano come
mettere un prezzo su tutto. Dannati mercenari!» Posò la tazza. «A
proposito di stranieri, gli Jaddiani arriveranno presto, vero?» Lanciò
un'occhiata a Ada, che trangugiò in fretta la sua acqua, tossendo.

«Sì, signore, entro una quindicina di giorni, se non vado errata.»

«Molto bene! Sai, non vado nei principati da secoli.» Si girò verso di
me, puntando un dito nel brandire nella mia direzione la sua ciotola
come se fosse stata un bastone. «Marlowe, devi visitare Jadd, o almeno
uno degli altri mondi, magari Samara. Ci sono davvero persone notevoli.»

In quel momento Karthik tornò dalla cucina con aria stranamente
avvilita, con il volto squadrato e semplice che appariva indecifrabile.
Per un momento non mi soffermai a elaborare la cosa mentre ascoltavo
Valka descrivere i problemi con la politica della Cappellania. La morsa
in cui stringeva il sapere in tutto l'Impero e perfino su Jadd, dove le
icone e la Madre Terra non erano le uniche divinità, era motivo di
riflessione. Quello che era cominciato come una propaganda dell'Impero
contro le macchine e una minaccia da usare contro i lord palatini era
cresciuto fino a essere irriconoscibile e al di là di ogni controllo.
Perfino il nostro imperatore si inchinava davanti agli altari della
Cappellania e riceveva la corona e lo scettro dal sinarca in persona.

«Cosa c'è, ragazzo?» chiese sir Elomas, notando per la prima volta il
viso cinereo del suo scudiero. Karthik esitò con lo sguardo che saettava
dalle sue scarpe al volto del cavaliere, poi mosse un piccolo passo in
avanti. «Sputa il rospo, Karthik!»

Sussultando, il ragazzo si mise sull'attenti. «Si tratta dell'onda,
signore. Orso e Damara l'avevano accesa in cucina e...» Lanciò
un'occhiata in tralice a Valka prima di fissare lo sguardo su di me.

«Avanti, ragazzo, sono solo parole. Mettile insieme, adesso!»

«C'è stato un attacco, signore. Una battaglia.» Nel parlare continuò a
guardare verso di me, anche se il suo discorso era rivolto a sir Elomas.

Qualsiasi altra cosa fosse stato -- un duellante, un damerino -- Elomas
non era un soldato e sbiancò in volto. «I Cielcin?»

Karthik si limitò ad annuire, un gesto così piccolo intriso di così
tanta gravità. Un ribaltarsi di mondi.

«Dove?»

Karthik deglutì a fatica. «Ai confini del sistema.»

Elomas si alzò così in fretta che quasi ribaltò la sedia, i cui piedi
artigliati si impigliarono nel folto tappeto. «Non dirai sul serio.»

«Dovremmo avere qui l'audio da un momento all'altro, signore.»

Tutti e cinque mantenemmo un solenne silenzio. Anni di voci portate a
Meidua dai mercanti, di proclami della Cappellania, di rapporti della
Legione riferiti al consiglio di mio padre... tutto questo converse in
quel singolo momento, cadendo come le tessere di un gioco e rendendo
tutto reale. Abbassai lo sguardo sul tavolo, desiderando di poter
trasformare l'acqua in vino come il magio della leggenda.

Gli altoparlanti della casa prefabbricata si attivarono, trasmettendo la
voce leggermente metallica di un annunciatore che riferiva le notizie,
addolcite per essere comunicate al pubblico mediante trasmissione
planetaria. Era una voce maschile, il cui timbro profondo era accentuato
dalla tensione. «...che trentatré ore fa un'azione congiunta delle Forze
Difensive di Emesh e della 437° Legione Centaurina agli ordini della
tribuno-cavaliere Raine Smythe ha annientato un contingente d'incursione
cielcin nell'eliopausa, conquistando un'altra gloriosa vittoria...» Non
sentii il resto, per me ci fu solo silenzio, come nell'occhio di un
uragano. Valka arricciò il naso e una ruga le si formò fra le
sopracciglia arcuate. Confesso di aver sentito una porzione di quello
stesso disprezzo crescere dentro di me. `Un'altra gloriosa vittoria'?
Conoscevo il {genere} di uomini che scrivevano quei dispacci, i logoteti
del ministro dell'Illuminazione Pubblica, uomini dozzinali, piccoli
cinici insolenti definiti dalla loro avversione per gli altri uomini. Un
orecchio esperto poteva sentire i calcoli che si celavano dietro ogni
parola come uncini per prendere la mente all'amo. Noi tutti siamo stati
nei panni di quegli uomini, ma la maggior parte di noi ha avuto la
decenza di non farne una carriera.

Ascoltai il comunicato in silenzio, con le mani serrate intorno al
bicchiere mentre visualizzavo i relitti delle navi cielcin rimorchiati a
Borosevo, i corpi degli alieni e le loro armi accumulati ai piedi delle
icone nelle Cappellanie della città. Ci sarebbe stato un altro trionfo,
questa volta in città, lungo le strade e i canali. Dietro i miei occhi
rividi più e più volte la decapitazione di Makisomn, sentii la voce da
basso profondo del conte Mataro tuonare dagli altoparlanti: «Questo è un
giorno glorioso per Emesh, gente mia! Il nemico era alle nostre porte,
deciso a distruggere la nostra patria. Che questo sia un avvertimento
per tutte quelle bestie che dimorano nel Buio Esterno! Noi non...»

Seduto in quella piccola capanna ai confini del mondo, al di sopra delle
nere gallerie di Calagah, mi sembrava che non fosse successo niente. Se
il cuoco Orso non avesse ascoltato le trasmissioni planetarie, se
un'altra tempesta autunnale avesse spazzato via il nostro canale di
comunicazione, se lui si fosse semplicemente sintonizzato su un altro
canale, la serata sarebbe andata avanti immutata e il mondo con essa. Un
mondo è vasto, un sistema solare lo è ancora di più. Per quanto la
guerra fosse in effetti vicina, Emesh era intatto. È strano come il più
vasto mondo esterno proietti la sua ombra sul nostro, sui nostri
momenti, fugaci e piccoli se misurati confrontandoli con la spinta
ruggente del tempo.

«Basta così!» ruggì Elomas, a voce tanto alta che i suoi servitori
poterono sentirlo nella stanza posteriore della piccola casa. Gli
altoparlanti si spensero con uno scatto, facendoci annegare nel
silenzio.

