\frontmatter

{\pagestyle{empty} % Blocca le pagine dal venire considerate nella numerazione

{
	\centering
	
	~
	
	\vspace{24pt}
	{\scshape\Huge L'impero\\del silenzio \par}
}

\cleardoublepage

\newlength\drop
\makeatletter
\newcommand*\titleM{\begingroup% Misericords, T&H p 153
	\setlength\drop{0.08\textheight}
	\centering
	\vspace*{\drop}
	{\Huge\bfseries L'impero del silenzio}\\[\baselineskip]
	{\scshape Christopher Ruocchio}\\[\baselineskip]
	\vfill
	{\large\scshape \phantom{the author}}\par
	\vfill
	{\large\scshape DiracEdizioni \img{Immagini/DiracEdizioniLogo.png}}\par
	\endgroup}
\makeatother
	
\titleM

\space\pagebreak

\null\vfill % Portiamo il testo del copy in basso

\begin{flushleft}
	\begin{justify}
		{
		\footnotesize \textit{L'impero del silenzio}
		
		\bigskip
		
		COPYRIGHT © 2024 Christopher Ruocchio
		
		\bigskip
		
		Tutti i diritti riservati.\\
		Nessuna parte di questa pubblicazione può essere riprodotta,\\
		memorizzata o trasmessa in qualsiasi forma o con qualsiasi mezzo,\\
		elettronico, meccanico, di fotocopiatura, registrazione, scansione o altro\\
		senza il permesso scritto dell'editore. È illegale copiare questo libro, pubblicarlo\\ 
		su un sito web o distribuirlo con qualsiasi altro mezzo senza autorizzazione.
		
		\bigskip
		
		Questo romanzo è interamente un'opera di fantasia.\\
		I nomi, i personaggi e gli episodi in esso rappresentati\\
	    sono frutto dell'immaginazione dell'autore.\\
	    Qualsiasi somiglianza con persone reali, vive o morte, eventi o località\\
	    è del tutto casuale.
		
		\bigskip
		
		\textonesuperior Edizione, 2024
		
		\bigskip
		
		\begin{tabular}{rl}
			ISBN--10:& 0-4908-2330-0\\ 
			ISBN--13:& 978-2-0827-3855-2\\ 
		\end{tabular}	
		
		\bigskip
		
		Pubblicato da DiracEdizioni \img{Immagini/DiracEdizioniLogo.png}
		}
	\end{justify}
\end{flushleft}

\let\cleardoublepage\clearpage

\chapter*{\phantom{DEDICA}}


	\begin{dedica}
		Dedicato ai miei nonni:
		
		Albert ed Eleanor. Deslan e James.
		
		Ci è voluto troppo per finire questo libro.
		
		Mi dispiace che sia tardi.
	\end{dedica}
}

\setcounter{page}{1}
 
{\chapterstyle{bringhurst} \chapter*{RICONOSCIMENTI}
	
La parola autore evoca impressioni di solitudine, di individualisti dalla mente robusta. Ci si immagina il vecchio Milton, quasi cieco, accovacciato sul suo scrittoio a lume di candela. Ma se la solitudine è certamente un pilastro di questa professione, è un errore pensare che qualcuno sia veramente solo. È quindi opportuno che mi prenda questo piccolo spazio per ringraziare tutte le persone che mi hanno aiutato a portare a termine questo lavoro.

Sarebbe un errore se l'elenco non iniziasse con i miei genitori, Paul e Penny. Anche se non l'ho sempre apprezzato, mi hanno sempre sostenuto, indipendentemente dal mio cattivo comportamento o dalla mia ingratitudine. Sono davvero fortunato ad essere loro figlio, e anche umile. Anche i miei fratelli meritano una menzione speciale. Matthew, Andrew e io non siamo sempre stati amici, ma ora lo siamo e questo è stato indicibilmente importante per me negli ultimi anni. Se dovessi elencare tutti i membri della famiglia a cui devo una certa gratitudine, dovrei pubblicare una genealogia, quindi ecco un breve elenco: allo zio John, per il suo aiuto nel capire i contratti; a Brian, per aver letto il libro prima di chiunque altro in famiglia; allo zio Pete, per aver assecondato le mie richieste di opere d'arte quando ero piccolo e per avermi mostrato che era possibile essere un artista e avere successo nella vita; e alla madre di mia madre, Deslan, che mi ha comprato la mia copia de Il Signore degli Anelli, che insieme a Guerre Stellari mi ha fatto venire voglia di raccontare queste storie. E a tutti gli altri, per essere davvero la migliore famiglia - e una famiglia migliore di quella che merito.

Sarei negligente a parlare di famiglia senza menzionare i miei amici, l'ulteriore famiglia che ho scelto, o che ha scelto me (per ragioni che non capisco bene). Come nel caso della mia vera famiglia, sono stata più fortunata di quanto non mi senta di meritare. A Erin G., la mia più vecchia amica e il mio più grande critico; a Marek, D'Artagnan in persona; a Anthony, Michael, Jordan e Joe, tutti fratelli; a Victoria, capitano dei beta-lettori; a Jenna, per tutto il suo aiuto e il duro lavoro sul mio sito web (e per molto altro ancora); a Erin H. e Jackson; e a Madison e Kyle, per la loro lunga amicizia e il loro sostegno. E a Christopher-Marcus - da cui Tor Gibson ha preso il nome - forse più di tutti, per una vita di discussioni e illuminazioni. Arete, amico mio.

Ad alcuni dei miei insegnanti devo un ringraziamento speciale: ad Anne Sweeney, Diane Buckley, Chris Sutton e Nikki Wright, per aver incoraggiato la mia inclinazione per la letteratura. A Priscilla Chappell, per aver sopportato quattro anni di me al liceo, quando ero al massimo dell'insopportabilità; al dottor Joe Hoffman, per avermi fatto capire la storia in un contesto e con una chiarezza che non immaginavo possibili; e a Craig Goheen, per avermi mostrato che la fantascienza e il fantasy erano molto più di Tolkien e Herbert. Ai dottori Marvin Hunt, Cat Warren e Etta Barksdale, per aver fatto sì che l'università valesse la pena di essere frequentata. Un ringraziamento speciale va a Sam Wheeler, per avermi aiutato a capire esattamente come si può mangiare un sole e per avermi aiutato con altri problemi di fisica che vanno ben oltre le capacità di questo studente di inglese; e al dottor John Kessel, per il suo tutoraggio, il suo aiuto con la mia lettera di presentazione e per avermi detto di tagliare quella stupida cornice narrativa.

Infine, devo ringraziare tutti coloro che hanno partecipato alla realizzazione di questo libro. In primo luogo, a Betsy Wollheim e Sheila Gilbert di DAW e a tutti i membri del team, in particolare alla mia editor, Katie, per il suo intuito e la sua pazienza nel trattare con me. A Sarah, la mia prima editor, a Gillian Redfearn e a tutti i collaboratori di Gollancz. Ai miei agenti, Shawna McCarthy, Danny Baror e Heather Baror-Shapiro, per il loro incredibile sostegno. Non potrei chiedere agenti migliori, davvero. E infine a Toni Weisskopf e a tutti i miei collaboratori alla Baen, e non solo per avermi assunto.

Grazie a tutti.

}

\newpage

\setcounter{page}{0}

\begin{figure}
	\centering
	\def\svgwidth{\columnwidth}
	\scalebox{0.7}{\input{caratteri alieni.pdf_tex}}
\end{figure}

\begin{center}
	UN RESOCONTO STILATO DAL DIVORATORE DI SOLI\\
		HADRIAN MARLOWE\\
		RIGUARDO ALLA GUERRA FRA L’UMANITÀ\\
		E I CIELCIN.\\
		TRADOTTO IN INGLESE CLASSICO DA\\
		TOR PAULOS DI NOV BELGAER, SU COLCHIDE.
\end{center}

\newpage\blankpage
